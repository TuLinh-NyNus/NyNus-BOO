%%Cấu hình mức độ dùng chung.
[N] Nhận biết
[H] Thông Hiểu
[V] VD
[C] VD Cao
[T] VIP
[M] Note
%
%Cấu hình nội dung
%
%==============================================================BẮT ĐẦU LỚP 10
-[0] Lớp 10
----[P] 10-NGÂN HÀNG CHÍNH
-------[1] Mệnh đề và tập hợp
----------[1] Mệnh đề
-------------[1] Xác định mệnh đề, mệnh đề chứa biến
-------------[2] Tính đúng-sai của mệnh đề (cơ bản)
-------------[3] Phủ định của một mệnh đề (cơ bản)
-------------[4] Mệnh đề kéo theo, mệnh đề đảo, mệnh đề tương đương
-------------[5] Mệnh đề với mọi, tồn tại (và phủ định chúng)
-------------[6] Áp dụng mệnh đề vào suy luận có lí
-------------[0] Câu hỏi tổng hợp
-------------[A] Chưa phân dạng
----------[2] Tập hợp
-------------[1] Tập hợp và phần tử của tập hợp
-------------[2] Tập hợp con. Hai tập hợp bằng nhau
-------------[3] Ký hiệu khoảng, đoạn, nửa khoảng
-------------[0] Câu hỏi tổng hợp
-------------[A] Chưa phân dạng
----------[3] Các phép toán tập hợp
-------------[1] Giao và hợp của hai tập hợp (rời rạc)
-------------[2] Hiệu và phần bù của hai tập hợp (rời rạc)
-------------[3] Giao và hợp (dùng đoạn, khoảng)
-------------[4] Hiệu và phần bù (dùng đoạn, khoảng)
-------------[5] Toán thực tế ứng dụng tập hợp
-------------[6] Biểu đồ Ven
-------------[7] Các phép trên tập hợp với tham số m
-------------[0] Câu hỏi tổng hợp
-------------[A] Chưa phân dạng
----------[0] Chưa phân dạng
-------------[0] Chưa phân dạng
-------[2] BPT và hệ BPT bậc nhất hai ẩn
----------[1] Bất phương trình bậc nhất hai ẩn
-------------[1] Các khái niệm về BPT bậc I hai ẩn
-------------[2] Nghiệm BPT bậc I hai ẩn
-------------[3] Miền nghiệm của BPT bậc I hai ẩn
-------------[4] Toán thực tế về BPT bậc I hai ẩn
-------------[0] Câu hỏi tổng hợp
-------------[A] Chưa phân dạng
----------[2] Hệ BPT bậc nhất hai ẩn
-------------[1] Các khái niệm về Hệ BPT bậc I hai ẩn
-------------[2] Nghiệm BPT bậc I hai ẩn
-------------[3] Miền nghiệm của Hệ BPT bậc I hai ẩn
-------------[4] Tìm GTLN GTNN bằng miền nghiệm
-------------[5] Toán thực tế về Hệ BPT bậc I hai ẩn
-------------[0] Câu hỏi tổng hợp
-------------[A] Chưa phân dạng
----------[0] Chưa phân dạng
-------------[0] Chưa phân dạng
-------[3] HS bậc hai và ĐT
----------[1] Hàm số và đồ thị
-------------[1] Xác định một HS
-------------[2] Tập xác định của HS
-------------[3] Giá trị của HS
-------------[4] Đồ thị của HS
-------------[5] Tính đồng biến, nghịch biến
-------------[6] Tính chẵn, lẻ
-------------[7] Điểm thuộc ĐT HS
-------------[8] Sử dụng các tính chất của ĐT
-------------[9] Toán thực tế về HS
-------------[0] Câu hỏi tổng hợp
-------------[A] Chưa phân dạng
----------[2] HS bậc hai
-------------[1] Xác định HS bậc hai
-------------[2] Điểm thuộc hàm số
-------------[3] Đỉnh, Trục đối xứng
-------------[4] Đồ thị của HS bậc hai
-------------[5] Xét dấu a, b, c , delta của ĐT
-------------[6] Bảng biến thiên, tính đơn điệu
-------------[7] GTNN, GTLN HS bậc hai
-------------[8] Bài toán về sự tương giao
-------------[9] Toán thực tế ứng dụng HS bậc hai
-------------[0] Câu hỏi tổng hợp tính chất HS bậc hai
-------------[B] [Giảm] HS chứa dấu giá trị tuyệt đối
-------------[C] [Giảm] Các phép biến đổi đồ thị 
-------------[D] [Giảm] Bài toán có chứa tham số m
-------------[A] Chưa phân dạng
----------[0] Chưa phân dạng
-------------[0] Chưa phân dạng
-------[4] Hệ thức lượng trong tam giác
----------[1] Giá trị lượng giác của góc (0-180)
-------------[1] Câu hỏi lý thuyết, công thức lượng giác
-------------[2] Xét dấu của biểu thức lượng giác
-------------[3] Tính các giá trị lượng giác
-------------[4] Biến đổi, rút gọn biểu thức lượng giác
-------------[5] CT liên kết lượng giác và ứng dụng
-------------[6] Các bài toán thực tế, liên môn
-------------[0] Câu hỏi tổng hợp
-------------[A] Chưa phân dạng
----------[2] Định lý sin và định lý côsin
-------------[1] Lý thuyết công thức
-------------[2] Bài toán chỉ dùng định lý Côsin
-------------[3] Bài toán chỉ dùng định lý Sin 
-------------[4] Bài toán có dùng công thức diện tích
-------------[5] Biến đổi, rút gọn biểu thức
-------------[6] Nhận dạng tam giác
-------------[7] Bài toán sử dụng công thức khác
-------------[8] Các bài toán thực tế, liên môn
-------------[0] Câu hỏi tổng hợp
-------------[A] Chưa phân dạng
----------[3] Giải tam giác và ứng dụng thực tế
-------------[1] Giải tam giác
-------------[2] Các ứng dụng thực tế
-------------[0] Câu hỏi tổng hợp
-------------[A] Chưa phân dạng
----------[0] Chưa phân dạng
-------------[0] Chưa phân dạng
-------[5] Véctơ (chưa xét tọa độ)
----------[1] Khái niệm véctơ
-------------[1] Câu hỏi lý thuyết
-------------[2] Xác định một véctơ
-------------[3] Xét phương và hướng của các véctơ
-------------[4] Hai véctơ bằng nhau
-------------[5] Hai véctơ đối nhau
-------------[6] Độ dài của một véctơ
-------------[7] Toán thực tế, liên môn dùng véctơ
-------------[0] Câu hỏi tổng hợp kiến thức
-------------[A] Chưa phân dạng
----------[2] Tổng và hiệu của hai véctơ
-------------[1] Câu hỏi lý thuyết
-------------[2] Tính toán, thu gọn tổng các véctơ
-------------[3] Tính toán, thu gọn hiệu các véctơ
-------------[4] Tính đúng-sai, chứng minh đẳng thức véctơ
-------------[5] Tìm điểm nhờ đẳng thức véctơ
-------------[6] Tính độ dài của véctơ tổng, hiệu
-------------[7] Cực trị hình học
-------------[8] Toán thực tế, liên môn dùng véctơ
-------------[0] Câu hỏi tổng hợp kiến thức
-------------[A] Chưa phân dạng
----------[3] Tích của một số với véctơ
-------------[1] Câu hỏi lý thuyết
-------------[2] Xác định k.vec\{a\} và độ dài của nó
-------------[3] Biến đổi, thu gọn, chứng minh 1 đẳng thức véctơ
-------------[4] Tính đúng - sai của đẳng thức vectơ
-------------[5] Tìm điểm nhờ đẳng thức véctơ
-------------[6] Sự cùng phương của 2 véctơ và ứng dụng
-------------[7] Phân tích 1 véctơ theo 2 véctơ không cùng phương
-------------[8] Tính độ dài của các phép toán Vectơ
-------------[9] Chứng minh thẳng hàng, tập hợp điểm
-------------[B] Cực trị hình học
-------------[C] Toán thực tế, liên môn dùng tích 1 số với vecto
-------------[0] Câu hỏi tổng hợp kiến thức
-------------[A] Chưa phân dạng
----------[4] Tích vô hướng (chưa xét tọa độ)
-------------[1] Câu hỏi lý thuyết
-------------[2] Tích vô hướng giữa hai vectơ
-------------[3] Góc giữa 2 véctơ không sử dụng tích vô hướng
-------------[4] Tìm góc nhờ tích vô hướng
-------------[5] Đẳng thức về tích vô hướng hoặc độ dài
-------------[6] Điều kiện vuông góc
-------------[7] Các bài toán tìm điểm và tập hợp điểm
-------------[8] Cực trị và chứng minh bất đẳng thức
-------------[9] Toán thực tế, liên môn
-------------[0] Câu hỏi tổng hợp kiến thức
-------------[A] Chưa phân dạng
----------[0] Chưa phân dạng
-------------[0] Chưa phân dạng
-------[6] Thống kê
----------[1] Số gần đúng. Sai số
-------------[1] Câu hỏi lý thuyết
-------------[2] Tính và ước lượng sai số tuyệt đối, tương đối
-------------[3] Tính và xác định độ chính xác của kết quả
-------------[4] Quy tròn số gần đúng
-------------[5] Viết số gần đúng cho số đúng biết độ chính xác
-------------[0] Câu hỏi tổng hợp
-------------[A] Chưa phân dạng
----------[2] Mô tả và biểu diễn dữ liệu trên các bảng và biểu đồ
-------------[1] Câu hỏi lý thuyết
-------------[2] Đọc và phân tích thông tin trên bảng số liệu
-------------[3] Đọc và phân tích thông tin trên Biểu đồ
-------------[4] Số liệu bất thường trên bảng số liệu
-------------[5] Số liệu bất thường trên Biểu đồ
-------------[0] Câu hỏi tổng hợp
-------------[A] Chưa phân dạng
----------[3] Các số đặc trưng đo xu thế trung tâm của mẫu số liệu
-------------[1] Câu hỏi lý thuyết
-------------[2] Số trung bình cộng
-------------[3] Số trung vị
-------------[4] Tứ phân vị
-------------[5] Mốt
-------------[6] Ý nghĩa của số đặc trưng đo xu thế trung tâm
-------------[0] Câu hỏi tổng hợp
-------------[A] Chưa phân dạng
----------[4] Các số đặc trưng đo mức độ phân tán của mẫu số liệu
-------------[1] Câu hỏi lý thuyết
-------------[2] Khoảng biến thiên
-------------[3] Khoảng tứ phân vị
-------------[4] Giá trị bất thường của mẫu số liệu
-------------[5] Phương sai, độ lệch chuẩn của mẫu số liệu
-------------[6] Ý nghĩa của số đặc trưng đo mức độ phân tán
-------------[0] Câu hỏi tổng hợp
-------------[A] Chưa phân dạng
----------[0] Chưa phân dạng
-------------[0] Chưa phân dạng
-------[7] Bất phương trình bậc 2 một ẩn
----------[1] Dấu của tam thức bậc 2
-------------[1] Xác định tam thức bậc 2 và bảng xét dấu
-------------[2] Dấu của tam thức bậc 2
-------------[3] Bài toán xét dấu biết BXD, ĐT
-------------[4] Xét dấu biểu thức dạng tích, thương
-------------[5] Toán thực tế, liên môn
-------------[0] Câu hỏi tổng hợp
-------------[A] Chưa phân dạng
-------------[B] [Giảm] Xét dấu biểu thức dạng tích, thương có tham số m
----------[2] Giải BPT bậc 2 một ẩn
-------------[1] Giải BPT bậc 2
-------------[2] Giải BPT bậc hai biết BXD, ĐT
-------------[3] Bài toán có chứa tham số m
-------------[4] Toán thực tế, liên môn
-------------[5] Bất phương trình dạng tích, thương
-------------[0] Câu hỏi tổng hợp
-------------[A] Chưa phân dạng
-------------[B] [Giảm] Hệ BPT BPT bậc 2
-------------[C] [Giảm] Bất phương trình chứa căn, |.|
-------------[D] [Giảm] Giải BPT bằng cách đặt ẩn phụ
----------[3] PT quy về phương trình bậc hai
-------------[1] PT căn fx = căn gx và mở rộng
-------------[2] PT căn fx = gx và mở rộng
-------------[3] PT căn thức có tham số
-------------[4] Toán hình, toán thực tế vận dụng PT bậc 2
-------------[0] Câu hỏi tổng hợp
-------------[A] Chưa phân dạng
-------------[B] [Giảm] PT căn thức (dạng khác)
-------------[C] [Giảm] PT khác quy về PT bậc 2
----------[0] Chưa phân dạng
-------------[0] Chưa phân dạng
-------[8] Đại số tổ hợp
----------[1] QT cộng-QT nhân
-------------[1] Bài toán chỉ sử dụng QT cộng
-------------[2] Bài toán chỉ sử dụng QT nhân
-------------[3] Bài toán kết hợp QT cộng và QT nhân
-------------[4] Bài toán dùng QT bù trừ
-------------[5] Bài toán đếm số tự nhiên
-------------[6] Sơ đồ hình cây
-------------[0] Câu hỏi tổng hợp
-------------[A] Chưa phân dạng
----------[2] Hoán vị-chỉnh hợp-tổ hợp ĐƠN GIẢN
-------------[1] LT và CT về P, C, A
-------------[2] Bài toán có biểu thức P
-------------[3] Bài toán có biểu thức C
-------------[4] Bài toán có biểu thức A
-------------[5] Bài toán kết hợp P, C, A
-------------[6] Tính toán biểu thức có chứa P,C,A
-------------[0] Câu hỏi tổng hợp
-------------[A] Chưa phân dạng
----------[3] Các bài toán đếm 
-------------[1] Bài toán đếm số tự nhiên
-------------[2] Bài toán chọn người
-------------[3] Bài toán chọn đối tượng khác
-------------[4] Bài toán có yếu tố hình học
-------------[5] Bài toán xếp chỗ (không tròn, không lặp)
-------------[6] Hoán vị bàn tròn
-------------[7] Hoán vị lặp
-------------[8] Các phương pháp đếm khác
-------------[9] Giải phương trình hoán vị - chỉnh hợp -  tổ hợp
-------------[0] Câu hỏi tổng hợp
-------------[A] Chưa phân dạng
----------[4] Nhị thức Newton
-------------[1] Lý thuyết tổng hợp 
-------------[2] Khai triển một nhị thức Newton
-------------[3] Tìm hệ số, số hạng trong khai triển bằng tam giác Pascal
-------------[4] Tìm hệ số, số hạng trong khai triển
-------------[5] Tính tổng nhờ khai triển nhị thức Newton
-------------[6] Toán tổ hợp có dùng nhị thức Newton
-------------[7] Nhị thức có mũ lớn hơn 5
-------------[0] Câu hỏi tổng hợp
-------------[A] Chưa phân dạng
----------[0] Chưa phân dạng
-------------[0] Chưa phân dạng
-------[9] Véctơ (trong hệ tọa độ)
----------[1] Toạ độ của véctơ
-------------[1] Tọa độ điểm, độ dài đại số của véctơ trên 1 trục
-------------[2] Phép toán véctơ (tổng, hiệu, tích với số) trong Oxy
-------------[3] Tọa độ điểm 
-------------[4] Tọa độ véc-tơ
-------------[5] Sự cùng phương của 2 véctơ và ứng dụng
-------------[6] Phân tích một véctơ theo 2 véctơ không cùng phương
-------------[7] Toán thực tế dùng hệ toạ độ
-------------[8] Độ dài vecto và ứng dụng
-------------[9] Cực trị
-------------[0] Câu hỏi tổng hợp
-------------[A] Chưa phân dạng
----------[2] Tích vô hướng (theo tọa độ)
-------------[1] Câu hỏi lý thuyết
-------------[2] Tính tích vô hướng
-------------[3] Tìm góc giữa hai véc-tơ
-------------[4] Đẳng thức về tích vô hướng hoặc độ dài
-------------[5] Điều kiện vuông góc
-------------[6] Các bài toán tìm điểm và tập hợp điểm
-------------[7] Cực trị và chứng minh bất đẳng thức
-------------[8] Toán thực tế, liên môn
-------------[0] Câu hỏi tổng hợp
-------------[A] Chưa phân dạng
----------[3] ĐT trong MP toạ độ
-------------[1] Điểm, véctơ, hệ số góc của ĐT
-------------[2] PT đường thẳng có VTCP
-------------[3] PT đường thẳng có VTPT
-------------[4] Vị trí tương đối giữa hai ĐT
-------------[5] Bài toán về góc giữa hai ĐT
-------------[6] Bài toán về khoảng cách
-------------[7] Bài toán tìm điểm
-------------[8] Bài toán dùng cho tam giác, tứ giác
-------------[9] Bài toán thực tế, PP tọa độ hóa
-------------[B] Bài toán liên quan đến tìm giao điểm
-------------[C] Câu hỏi cực trị
-------------[0] Câu hỏi tổng hợp
-------------[A] Chưa phân dạng
-------------[D] [Giảm] Bài toán có dùng PT chính tắc
----------[4] Đường tròn trong MP toạ độ
-------------[1] Tìm tâm, bán kính và điều kiện là đường tròn
-------------[2] PT đường tròn
-------------[3] PT tiếp tuyến của đường tròn
-------------[4] Vị trí tương đối liên quan đường tròn
-------------[5] Toán tổng hợp ĐT và đường tròn
-------------[6] Bài toán dùng cho tam giác, tứ giác
-------------[7] Bài toán thực tế, PP tọa độ hóa
-------------[8] Bài toán tìm điểm
-------------[0] Câu hỏi tổng hợp
-------------[A] Chưa phân dạng
----------[5] Elip và các vấn đề liên quan
-------------[1] Xác định các yếu tố của elip
-------------[2] PT chính tắc của elip
-------------[3] Bài toán điểm trên elip
-------------[4] Liên quan trục lớn, trục bé, tâm sai
-------------[5] Bài toán thực tế, PP tọa độ hóa
-------------[6] Bài toán tổng hợp
-------------[A] Chưa phân dạng
----------[6] Hypebol và các vấn đề liên quan
-------------[1] Xác định các yếu tố của hypebol
-------------[2] PT chính tắc của hypebol
-------------[3] Bài toán điểm trên hypebol
-------------[4] Liên quan trục thực, trục ảo, tâm sai
-------------[5] Bài toán thực tế, PP tọa độ hóa
-------------[0] Bài toán tổng hợp
-------------[A] Chưa phân dạng
----------[7] Parabol và các vấn đề liên quan
-------------[1] Xác định các yếu tố của parabol
-------------[2] PT chính tắc của parabol
-------------[3] Bài toán điểm trên parabol
-------------[4] Một số vấn đề liên quan
-------------[5] Bài toán thực tế, PP tọa độ hóa
-------------[0] Bài toán tổng hợp
-------------[A] Chưa phân dạng
----------[0] Chưa phân dạng
-------------[0] Chưa phân dạng
-------[0] Xác suất
----------[1] Không gian mẫu và biến cố
-------------[1] LT tổng hợp
-------------[2] Mô tả không gian mẫu, biến cố
-------------[3] Đếm phần tử không gian mẫu, biến cố
-------------[4] Mô tả biến cố đối
-------------[0] Câu hỏi tổng hợp
-------------[A] Chưa phân dạng
----------[2] Xác suất của biến cố
-------------[1] LT tổng hợp
-------------[2] Liên quan xúc xắc, đồng tiền (PP liệt kê)
-------------[3] Liên quan việc sắp xếp chỗ
-------------[4] Liên quan việc chọn người
-------------[5] Liên quan việc chọn đối tượng khác
-------------[6] Liên quan hình học
-------------[7] Liên quan việc đếm số
-------------[8] Liên quan bàn tròn hoặc hoán vị lặp
-------------[9] Liên quan vấn đề khác
-------------[B] Liên quan đến sơ đồ hình cây
-------------[C] Liên quan đến viên bi
-------------[D] Giải phương trình xác suất
-------------[0] Câu hỏi tổng hợp
-------------[A] Chưa phân dạng
----------[0] Chưa phân dạng
-------------[0] Chưa phân dạng
%==============================================================ID 11 DPS
----[D] 10 - Đại số và giải tích
-------[1] Mệnh đề và tập hợp
----------[1] Mệnh đề
-------------[1] Xác định mệnh đề, mệnh đề chứa biến
-------------[2] Tính đúng-sai của mệnh đề (cơ bản)
-------------[3] Phủ định của một mệnh đề (cơ bản)
-------------[4] Mệnh đề kéo theo, mệnh đề đảo, mệnh đề tương đương
-------------[5] Mệnh đề với mọi, tồn tại (và phủ định chúng)
-------------[6] Áp dụng mệnh đề vào suy luận có lí
----------[2] Tập hợp
-------------[1] Tập hợp và phần tử của tập hợp
-------------[2] Tập hợp con. Hai tập hợp bằng nhau
-------------[3] Ký hiệu khoảng, đoạn, nửa khoảng
----------[3] Các phép toán tập hợp
-------------[1] Giao và hợp của hai tập hợp (rời rạc)
-------------[2] Hiệu và phần bù của hai tập hợp (rời rạc)
-------------[3] Giao và hợp (dùng đoạn, khoảng)
-------------[4] Hiệu và phần bù (dùng đoạn, khoảng)
-------------[5] Toán thực tế ứng dụng tập hợp
-------------[6] Biểu đồ Ven
----------[0] Chưa phân dạng
-------------[0] Chưa phân dạng
-------[2] BPT và hệ BPT bậc nhất hai ẩn
----------[1] Bất phương trình bậc nhất hai ẩn
-------------[1] Các khái niệm về BPT bậc I hai ẩn
-------------[2] Miền nghiệm của BPT bậc I hai ẩn
-------------[3] Toán thực tế về BPT bậc I hai ẩn
-------------[4] Nghiệm BPT bậc I hai ẩn
----------[2] Hệ BPT bậc nhất hai ẩn
-------------[1] Các khái niệm về Hệ BPT bậc I hai ẩn
-------------[2] Miền nghiệm của Hệ BPT bậc I hai ẩn
-------------[3] Toán thực tế về Hệ BPT bậc I hai ẩn
-------------[4] Nghiệm BPT bậc I hai ẩn
-------------[5] Tìm GTLN GTNN bằng miền nghiệm
----------[0] Chưa phân dạng
-------------[0] Chưa phân dạng
-------[3] HS bậc hai và ĐT
----------[1] Hàm số và đồ thị
-------------[1] Xác định một HS
-------------[2] Tập xác định của HS
-------------[3] Giá trị của HS
-------------[4] Đồ thị của HS
-------------[5] Tính đồng biến, nghịch biến
-------------[6] Tính chẵn, lẻ
-------------[7] Toán thực tế về HS
-------------[8] Điểm thuộc ĐT HS
-------------[9] Sử dụng các tính chất của ĐT
----------[2] HS bậc hai
-------------[1] Xác định HS bậc hai
-------------[2] Bảng biến thiên, tính đơn điệu
-------------[3] Đồ thị của HS bậc hai
-------------[4] Bài toán về sự tương giao
-------------[5] HS chứa dấu giá trị tuyệt đối
-------------[6] Toán thực tế ứng dụng HS bậc hai
-------------[7] GTNN, GTLN HS bậc hai
-------------[8] Đỉnh, Trục đối xứng
-------------[9] Xét dấu a, b, c , delta của ĐT
-------------[A] Điểm thuộc ĐT HS
-------------[B] Câu hỏi TH tính chất HS bậc hai
-------------[C] Các phép biến đổi đồ thị 
----------[0] Chưa phân dạng
-------------[0] Chưa phân dạng
-------[6] Thống kê
----------[1] Số gần đúng. Sai số
-------------[1] Tính và ước lượng sai số tuyệt đối, tương đối
-------------[2] Tính và xác định độ chính xác của kết quả
-------------[3] Quy tròn số gần đúng
-------------[4] Viết số gần đúng cho số đúng biết độ chính xác
----------[2] Mô tả và biểu diễn dữ liệu trên các bảng và biểu đồ
-------------[1] Đọc và phân tích thông tin trên bảng số liệu
-------------[2] Đọc và phân tích thông tin trên Biểu đồ
-------------[3] Số liệu bất thường trên bảng số liệu
-------------[4] Số liệu bất thường trên Biểu đồ
----------[3] Các số đặc trưng đo xu thế trung tâm của mẫu số liệu
-------------[1] Câu hỏi lý thuyết
-------------[2] Số trung bình cộng
-------------[3] Số trung vị
-------------[4] Tứ phân vị
-------------[5] Mốt
-------------[5] Ý nghĩa của số đặc trưng đo xu thế trung tâm
----------[4] Các số đặc trưng đo mức độ phân tán của mẫu số liệu
-------------[1] Câu hỏi lý thuyết
-------------[2] Khoảng biến thiên, khoảng tứ phân vị
-------------[3] Giá trị bất thường của mẫu số liệu
-------------[4] Phương sai, độ lệch chuẩn của mẫu số liệu
-------------[5] Ý nghĩa của số đặc trưng đo mức độ phân tán
----------[0] Chưa phân dạng
-------------[0] Chưa phân dạng
-------[7] Bất phương trình bậc 2 một ẩn
----------[1] Dấu của tam thức bậc 2
-------------[1] Xác định tam thức bậc 2 và bảng xét dấu
-------------[2] Dấu của tam thức bậc 2
-------------[3] Bài toán xét dấu biết BXD, ĐT
-------------[4] Xét dấu biểu thức dạng tích, thương
-------------[5] Toán thực tế, liên môn
-------------[6] [Giảm] Xét dấu biểu thức dạng tích, thương có tham số m
----------[2] Giải BPT bậc 2 một ẩn
-------------[1] Giải BPT bậc 2
-------------[2] Giải BPT bậc hai biết BXD, ĐT
-------------[3] Toán thực tế, liên môn
-------------[4] Bất phương trình dạng tích, thương
-------------[7] Bài toán có chứa tham số m
-------------[8] Giải BPT bằng cách đặt ẩn phụ
-------------[5] [Giảm] Hệ BPT BPT bậc 2
-------------[6] [Giảm] Bất phương trình chứa căn, |.|
----------[3] PT quy về phương trình bậc hai
-------------[1] PT căn fx = căn gx và mở rộng
-------------[2] PT căn fx = gx và mở rộng
-------------[3] PT căn thức có tham số
-------------[6] Toán hình, toán thực tế vận dụng PT bậc 2
-------------[4] [Giảm] PT căn thức (dạng khác)
-------------[5] [Giảm] PT khác quy về PT bậc 2
----------[0] Chưa phân dạng
-------------[0] Chưa phân dạng
-------[8] Đại số tổ hợp
----------[1] QT cộng-QT nhân
-------------[1] Bài toán chỉ sử dụng QT cộng
-------------[2] Bài toán chỉ sử dụng QT nhân
-------------[3] Bài toán kết hợp QT cộng và QT nhân
-------------[4] Bài toán dùng QT bù trừ
-------------[5] Bài toán đếm số tự nhiên
-------------[6] Sơ đồ hình cây
----------[2] Hoán vị-chỉnh hợp-tổ hợp
-------------[1] LT và CT về P, C, A
-------------[2] Bài toán có biểu thức P
-------------[3] Bài toán đếm số tự nhiên
-------------[4] Bài toán chọn người
-------------[5] Bài toán chọn đối tượng khác
-------------[6] Bài toán có yếu tố hình học
-------------[7] Bài toán xếp chỗ (không tròn, không lặp)
-------------[8] Hoán vị bàn tròn
-------------[9] Hoán vị lặp
-------------[A] Bài toán có biểu thức C
-------------[B] Bài toán có biểu thức A
-------------[C] Bài toán kết hợp P, C, A
----------[3] Nhị thức Newton
-------------[1] Lý thuyết tổng hợp 
-------------[2] Khai triển một nhị thức Newton
-------------[3] Tìm hệ số, số hạng trong khai triển bằng tam giác Pascal
-------------[4] Tìm hệ số, số hạng trong khai triển
-------------[5] Tính tổng nhờ khai triển nhị thức Newton
-------------[6] Toán tổ hợp có dùng nhị thức Newton
----------[0] Chưa phân dạng
-------------[0] Chưa phân dạng
-------[0] Xác suất
----------[1] Không gian mẫu và biến cố
-------------[1] LT tổng hợp
-------------[2] Mô tả không gian mẫu, biến cố
-------------[3] Đếm phần tử không gian mẫu, biến cố
----------[2] Xác suất của biến cố
-------------[1] LT tổng hợp
-------------[2] Liên quan xúc xắc, đồng tiền (PP liệt kê)
-------------[3] Liên quan việc sắp xếp chỗ
-------------[4] Liên quan việc chọn người
-------------[5] Liên quan việc chọn đối tượng khác
-------------[6] Liên quan hình học
-------------[7] Liên quan việc đếm số
-------------[8] Liên quan bàn tròn hoặc hoán vị lặp
-------------[9] Liên quan vấn đề khác
----------[0] Chưa phân dạng
-------------[0] Chưa phân dạng
-------[A] Chưa phân dạng
----------[0] Chưa phân dạng
-------------[0] Chưa phân dạng
----[H] 10 - Hình học và đo lường
-------[4] Hệ thức lượng trong tam giác
----------[1] Giá trị lượng giác của góc (0-180)
-------------[1] Xét dấu của biểu thức lượng giác
-------------[2] Tính các giá trị lượng giác
-------------[3] Biến đổi, rút gọn biểu thức lượng giác
-------------[4] Câu hỏi lý thuyết, công thức lượng giác
-------------[5] CT liên kết lượng giác và ứng dụng
----------[2] Định lý sin và định lý côsin
-------------[1] Bài toán chỉ dùng định lý Sin 
-------------[2] Bài toán có dùng công thức diện tích
-------------[3] Biến đổi, rút gọn biểu thức
-------------[4] Nhận dạng tam giác
-------------[5] Bài toán chỉ dùng định lý Côsin
-------------[6] Lý thuyết công thức
----------[3] Giải tam giác và ứng dụng thực tế
-------------[1] Giải tam giác
-------------[2] Các ứng dụng thực tế
----------[0] Chưa phân dạng
-------------[0] Chưa phân dạng
-------[5] Véctơ (chưa xét tọa độ)
----------[1] Khái niệm véctơ
-------------[1] Xác định một véctơ
-------------[2] Xét phương và hướng của các véctơ
-------------[3] Hai véctơ bằng nhau
-------------[4] Hai véctơ đối nhau
-------------[5] Độ dài của một véctơ
-------------[6] Toán thực tế, liên môn dùng véctơ
----------[2] Tổng và hiệu của hai véctơ
-------------[1] Tính toán, thu gọn tổng các véctơ
-------------[2] Tính đúng-sai của 1 đẳng thức véctơ
-------------[3] Tìm điểm nhờ đẳng thức véctơ
-------------[4] Tính độ dài của véctơ tổng, hiệu
-------------[5] Cực trị hình học
-------------[6] Toán thực tế, liên môn dùng véctơ
-------------[7] Tính toán, thu gọn hiệu các véctơ
----------[3] Tích của một số với véctơ
-------------[1] Xác định k.vec\{a\} và độ dài của nó
-------------[2] Biến đổi, thu gọn 1 đẳng thức véctơ
-------------[3] Tìm điểm nhờ đẳng thức véctơ
-------------[4] Sự cùng phương của 2 véctơ và ứng dụng
-------------[5] Phân tích 1 véctơ theo 2 véctơ không cùng phương
-------------[6] Tính độ dài của véctơ tổng, hiệu
-------------[7] Chứng minh thẳng hàng, tập hợp điểm
-------------[8] Cực trị hình học
-------------[9] Toán thực tế, liên môn dùng tích 1 số với vecto
-------------[A] Tính đúng sai hệ thức vecto có k.vecto
----------[4] Tích vô hướng (chưa xét tọa độ)
-------------[1] Tích vô hướng, góc giữa 2 véctơ
-------------[2] Tìm góc nhờ tích vô hướng
-------------[3] Đẳng thức về tích vô hướng hoặc độ dài
-------------[4] Điều kiện vuông góc
-------------[5] Các bài toán tìm điểm và tập hợp điểm
-------------[6] Cực trị và chứng minh bất đẳng thức
-------------[7] Toán thực tế, liên môn
----------[0] Chưa phân dạng
-------------[0] Chưa phân dạng
-------[9] Véctơ (trong hệ tọa độ)
----------[1] Toạ độ của véctơ
-------------[1] Tọa độ điểm, độ dài đại số của véctơ trên 1 trục
-------------[2] Phép toán véctơ (tổng, hiệu, tích với số) trong Oxy
-------------[3] Tọa độ điểm và véctơ trên hệ trục Oxy
-------------[4] Sự cùng phương của 2 véctơ và ứng dụng
-------------[5] Phân tích một véctơ theo 2 véctơ không cùng phương
-------------[6] Toán thực tế dùng hệ toạ độ
-------------[7] Độ dài vecto và ứng dụng
----------[2] Tích vô hướng (theo tọa độ)
-------------[1] Tích vô hướng, góc giữa 2 véctơ
-------------[2] Tìm góc nhờ tích vô hướng
-------------[3] Đẳng thức về tích vô hướng hoặc độ dài
-------------[4] Điều kiện vuông góc
-------------[5] Các bài toán tìm điểm và tập hợp điểm
-------------[6] Cực trị và chứng minh bất đẳng thức
-------------[7] Toán thực tế, liên môn
----------[3] ĐT trong MP toạ độ
-------------[1] Điểm, véctơ, hệ số góc của ĐT
-------------[2] PT đường thẳng có VTCP
-------------[3] Vị trí tương đối giữa hai ĐT
-------------[4] Bài toán về góc giữa hai ĐT
-------------[5] Bài toán về khoảng cách
-------------[6] Bài toán tìm điểm
-------------[7] Bài toán dùng cho tam giác, tứ giác
-------------[8] Bài toán thực tế, PP tọa độ hóa
-------------[A] Bài toán liên quan đến tìm giao điểm
-------------[B] PT đường thẳng có VTPT
-------------[9] [Giảm] Bài toán có dùng PT chính tắc
----------[4] Đường tròn trong MP toạ độ
-------------[1] Tìm tâm, bán kính và điều kiện là đường tròn
-------------[2] PT đường tròn
-------------[3] PT tiếp tuyến của đường tròn
-------------[4] Vị trí tương đối liên quan đường tròn
-------------[5] Toán tổng hợp ĐT và đường tròn
-------------[6] Bài toán dùng cho tam giác, tứ giác
-------------[7] Bài toán thực tế, PP tọa độ hóa
----------[5] Ba đường conic trong MP toạ độ
-------------[1] Xác định các yếu tố của elip
-------------[2] PT chính tắc của elip
-------------[3] Bài toán điểm trên elip
-------------[4] Xác định các yếu tố của hypebol
-------------[5] PT chính tắc của hypebol
-------------[6] Bài toán điểm trên hypebol
-------------[7] Xác định các yếu tố của parabol
-------------[8] PT chính tắc của parabol
-------------[9] Bài toán điểm trên parabol
-------------[0] Bài toán tổng hợp 3 đường conic
-------------[A] Bài toán thực tế, PP tọa độ hóa
----------[0] Chưa phân dạng
-------------[0] Chưa phân dạng
-------[0] Chưa phân dạng
----------[0] Chưa phân dạng
-------------[0] Chưa phân dạng
----[C] 10 - CĐ
-------[0] Chưa phân dạng
----------[0] Chưa phân dạng
-------------[0] Chưa phân dạng
-------[1] HPT bậc nhất 3 ẩn
----------[1] HPT bậc nhất 3 ẩn
-------------[1] Các khái niệm về HPT bậc nhất 3 ẩn
-------------[2] Giải HPT bậc nhất 3 ẩn
----------[2] Toán thực tế ứng dụng HPT bậc nhất 3 ẩn
-------------[1] Toán thực tế ứng dụng HPT bậc nhất 3 ẩn
-------[2] PP quy nạp toán học
----------[3] PP quy nạp toán học
-------------[1] Quy nạp chứng minh đẳng thức
-------------[2] Quy nạp chứng minh BĐT
----------[4] Nhị thức Newton(hệ số lớn hơn 5)
-------------[1] Khai triển nhị thức Newton
-------------[2] Tìm hệ số trong khai triển
-------[3] Ba đường Conic và ứng dụng
----------[5] Elip
-------------[1] Bài toán về Elip có trục lớn, trục bé, tâm sai
----------[6] Hypebol
-------------[1] Bài toán về Hypebol trục thực, trục ảo
----------[7] Parabol
-------------[1] Bài toán về Parabol
----------[8] Sự thống nhất giữa ba đường Conic
-------------[1] Bài toán về sự thống nhất giữa ba đường Conic
-------[4] BPT và hệ BPT (cơ bản)
----------[1] BPT và hệ BPT
-------------[1] Bất phương trình
-------------[2] Hệ bất phương trình
-------------[0] Chưa phân dạng
----[G] 10 - HSG
-------[0] Chưa phân dạng
----------[0] Chưa phân dạng
-------------[0] Chưa phân dạng
-------[1] HSG - Đại số và Số học 
----------[1] PT, Hệ phương trình, BPT
-------------[1] PT 
-------------[2] Hệ phương trình 
-------------[0] Chưa phân dạng
----------[2] Bất đẳng thức
-------------[0] Chưa phân dạng
-------------[1] AM-GM
-------------[2] Cauchy-Schwarz
-------------[3] Minkowski và Schur
-------------[4] Khác
----------[3] PT hàm tính đại số
-------------[0] Chưa phân dạng
----------[4] Tổ hợp 
-------------[0] Chưa phân dạng
-------------[1] Nguyên lí Diricle và ứng dụng
-------------[2] Đại lượng bất biến, nữa bất biến và ứng dụng
----------[5] Số học 
-------------[0] Chưa phân dạng
-------------[1] Số nguyên và tính chất
-------------[2] Đồng dư
-------------[3] Định lí Ferma, Euleur,...
-------------[4] PT đồng dư
-------------[5] Số chính phương module n
-------------[6] Các HS học
-------------[7] Định lí Euleur mở rộng
----------[6] Một số yếu tố của lí thuyết Graf và ứng dụng
-------------[0] Chưa phân dạng
----------[7] Lượng giác
-------------[0] Chưa phân dạng
----------[8] Đa thức
-------------[0] Chưa phân dạng
----------[9] Tập hợp, ánh xạ (không giới hạn)
-------------[0] Chưa phân dạng
----------[0] Chưa phân dạng
-------------[0] Chưa phân dạng
-------[2] HSG - Hình học
----------[1] Hình học phẳng sử dụng phương pháp vectơ
-------------[0] Chưa phân dạng
----------[2] Hình học phẳng sử dụng phương pháp tọa độ
-------------[0] Chưa phân dạng
----------[3] Các bài toán hình học cổ điển
-------------[0] Chưa phân dạng
-------------[1] tỉ số đơn, tỉ số kép, hàng điều hòa, chùm điều hòa, tứ giác điều hòa
-------------[2] Góc định hướng
-------------[3] Phương tích, trục đẳng phương, tâm đẳng phương
-------------[4] Cực, đối cực
----------[0] Chưa phân dạng
-------------[0] Chưa phân dạng
----[0] 10 - Chưa phân dạng
-------[0] Chưa phân dạng
----------[0] Chưa phân dạng
-------------[0] Chưa phân dạng
%==============================================================KẾT THÚC MAP ID 10

%==============================================================BẮT ĐẦU MAP ID 11
-[1] Lớp 11
----[P] 11-NGÂN HÀNG CHÍNH
-------[1] HS lượng giác và phương trình lượng giác
----------[1] Góc lượng giác
-------------[1] Câu hỏi lý thuyết
-------------[2] Chuyển đổi đơn vị độ và radian
-------------[3] Số đo của một góc lượng giác
-------------[4] Độ dài của một cung tròn
-------------[5] Đường tròn lượng giác và ứng dụng
-------------[6] Các bài toán thực tế, liên môn
-------------[0] Câu hỏi tổng hợp
-------------[A] Chưa phân dạng
----------[2] Giá trị lượng giác của một góc lượng giác
-------------[1] Câu hỏi lý thuyết
-------------[2] Xét dấu các giá trị lượng giác
-------------[4] CT liên kết LG
-------------[6] Tính giá trị lượng giác của một góc
-------------[7] Tính giá trị biểu thức lượng giác
-------------[3] Biến đổi, thu gọn 1 biểu thức lượng giác
-------------[5] Các bài toán có yếu tố thực tế, liên môn
-------------[0] Câu hỏi tổng hợp
-------------[A] Chưa phân dạng
----------[3] Các công thức lượng giác
-------------[1] Câu hỏi lý thuyết
-------------[2] Tính áp dụng công thức cộng
-------------[3] Tính áp dụng công thức nhân đôi - hạ bậc
-------------[4] Tính áp dụng công thức biến đổi tích <-> tổng
-------------[5] Tính kết hợp nhiều công thức lượng giác
-------------[6] Nhận dạng tam giác
-------------[8] Tính giá trị biểu thức
-------------[9] Chứng minh, biến đổi đẳng thức 
-------------[7] Các bài toán có yếu tố thực tế, liên môn
-------------[0] Câu hỏi tổng hợp
-------------[A] Chưa phân dạng
----------[4] HS lượng giác và ĐT
-------------[1] Câu hỏi lý thuyết
-------------[2] Tìm tập xác định
-------------[3] Xét tính đơn điệu
-------------[4] Xét tính chẵn, lẻ
-------------[5] Xét tính tuần hoàn, tìm chu kỳ
-------------[6] Tìm tập giá trị và min, max
-------------[7] Bảng biến thiên và ĐT
-------------[8] Các bài toán có yếu tố thực tế, liên môn
-------------[9] Câu hỏi tổng hợp
-------------[A] Chưa phân dạng
----------[5] PT lượng giác cơ bản
-------------[1] Câu hỏi lý thuyết. Khái niệm phương trình tương đương
-------------[2] Điều kiện có nghiệm
-------------[3] PT lượng giác sinx
-------------[4] PT lượng giác cosx
-------------[5] PT lượng giác tanx, cotx
-------------[6] PT đưa về dạng cơ bản
-------------[8] PT lượng giác trên đoạn
-------------[9] Biểu diễn nghiệm trên đường tròn LG
-------------[A] Tìm nghiệm PTLG bằng ĐT
-------------[7] Toán thực tế, liên môn
-------------[B] Câu hỏi tổng hợp
-------------[C] Chưa phân dạng
----------[6] [Giảm] PT lượng giác thường gặp
-------------[1] PT bậc n theo một HS lượng giác
-------------[2] PT đẳng cấp bậc n đối với sinx và cosx
-------------[3] PT bậc nhất đối với sinx và cosx
-------------[4] PT đối xứng, phản đối xứng
-------------[5] PT lượng giác không mẫu mực
-------------[6] PT lượng giác có chứa ẩn ở mẫu số
-------------[7] PT lượng giác có chứa tham số
-------------[8] Bài toán thực tế
-------------[A] Chưa phân dạng
----------[0] Chưa phân dạng
-------------[0] Chưa phân dạng
-------[2] Dãy số. Cấp số cộng. Cấp số nhân
----------[1] Dãy số
-------------[1] Câu hỏi lý thuyết
-------------[9] Dãy số hữu hạn, dãy số vô hạn
-------------[2] Số hạng tổng quát, biểu diễn dãy số
-------------[3] Tìm số hạng bằng công thức truy hồi
-------------[4] Tìm số hạng bằng công thức tổng quát của dãy số
-------------[5] Dãy số tăng, dãy số giảm
-------------[6] Dãy số bị chặn
-------------[7] Toán thực tế về dãy số
-------------[B] Tính tổng dãy số
-------------[8] Câu hỏi tổng hợp
-------------[A] Chưa phân dạng
----------[2] Cấp số cộng
-------------[1] Câu hỏi lý thuyết
-------------[2] Nhận diện cấp số cộng, công sai d
-------------[3] Số hạng tổng quát của cấp số cộng
-------------[4] Giải hệ phương trình CSC
-------------[5] Tìm số hạng cụ thể trong cấp số cộng
-------------[6] ĐK là CSC, tính chất CSC
-------------[7] Tính tổng của cấp số cộng
-------------[8] Các bài toán thực tế
-------------[B] Bài toán liên hệ cấp số cộng
-------------[A] Chưa phân dạng
-------------[9] Câu hỏi tổng hợp
----------[3] Cấp số nhân
-------------[1] Câu hỏi lý thuyết
-------------[2] Nhận diện cấp số nhân, công bội q
-------------[3] Số hạng tổng quát của cấp số nhân
-------------[4] Giải hệ phương trình CSC
-------------[5] Tìm số hạng cụ thể trong cấp số nhân
-------------[6] Điều kiện để dãy số là cấp số nhân
-------------[7] Tính tổng của cấp số nhân
-------------[8] Kết hợp cấp số nhân và cấp số cộng
-------------[9] Các bài toán thực tế
-------------[B] Bài toán liên hệ cấp số nhân
-------------[0] Câu hỏi tổng hợp
-------------[A] Chưa phân dạng
----------[0] Chưa phân dạng
-------------[0] Chưa phân dạng
-------------[1] Xác suất liên quan đến tính chất cấp số cộng
-------------[2] Xác suất liên quan đến tính chất cấp số nhân
-------[3] Giới hạn. HS liên tục
----------[1] Giới hạn của dãy số
-------------[1] Câu hỏi lý thuyết
-------------[C] GH dãy số đơn giản
-------------[B] Phép toán giới hạn dãy số
-------------[2] PP đặt thừa số chung, kết quả hữu hạn
-------------[3] PP lượng liên hợp, kết quả hữu hạn
-------------[4] PP đặt thừa số chung, kết quả vô hạn
-------------[5] PP lượng liên hợp, kết quả vô hạn
-------------[D] Giới hạn dãy số truy hồi
-------------[E] Giới hạn dãy số có chứa tham số
-------------[6] CSN lùi vô hạn 
-------------[7] GH hàm ẩn
-------------[8] Toán thực tế, liên môn
-------------[0] Câu hỏi tổng hợp
-------------[A] Chưa phân dạng
-------------[9] [Giảm] Nguyên lí kẹp
----------[2] Giới hạn của HS HH hoặc VH
-------------[1] Câu hỏi lý thuyết
-------------[B] GH hàm số đơn giản
-------------[2] GH hữu hạn PP đặt thừa số chung, kết quả HH
-------------[3] GH hữu hạn PP đặt thừa số chung, kết quả vô cực
-------------[4] GH hữu hạn PP lượng liên hợp, kết quả HH
-------------[5] GH hữu hạn PP lượng liên hợp, kết quả vô cực
-------------[6] GH vô hạn PP đặt thừa số chung, kết quả HH
-------------[7] GH vô hạn đặt thừa số chung, kết quả vô cực
-------------[8] GH vô hạn lượng liên hợp, kết quả HH
-------------[9] GH vô hạn lượng liên hợp, kết quả vô cực
-------------[C] Bài toán giới hạn có chứa tham số
-------------[D] Giới hạn vô định
-------------[0] Câu hỏi tổng hợp
-------------[A] Chưa phân dạng
----------[3] Giới hạn của HS khác
-------------[1] GH thay số trực tiếp
-------------[B] Phép toán giới hạn hàm số
-------------[2] Giới hạn một bên
-------------[7] Bài toán giới hạn một bên có chứa tham số
-------------[3] Giới hạn HS dựa vào ĐT
-------------[4] GH hàm ẩn
-------------[5] GH hàm đặc biệt
-------------[6] Toán thực tế, liên môn
-------------[0] Câu hỏi tổng hợp
-------------[A] Chưa phân dạng
----------[4] HS liên tục
-------------[1] Câu hỏi lý thuyết
-------------[2] Tính liên tục thể hiện qua ĐT
-------------[3] HS liên tục tại một điểm
-------------[4] HS liên tục trên khoảng, đoạn
-------------[6] Bài toán phương trình có nghiệm
-------------[7] Toán thực tế, liên môn
-------------[5] Bài toán có chứa tham số
-------------[0] Câu hỏi tổng hợp
-------------[B] [G] Chứng minh phương trình có nghiệm (ĐL giá trị trung gian)
-------------[A] Chưa phân dạng
----------[0] Chưa phân dạng
-------------[0] Chưa phân dạng
-------[4] ĐT, MP. Quan hệ song song trong không gian
----------[1] Điểm, ĐT và MP
-------------[1] Câu hỏi lý thuyết
-------------[2] Biểu diễn của một hình không gian, Lý thuyết hình chóp
-------------[A] Bài tập liên hệ điểm, đường thẳng, mặt phẳng
-------------[3] Tìm GIAO TUYẾN của hai MP
-------------[4] Tìm GIAO ĐIỂM của ĐT và MP
-------------[5] Xác định thiết diện
-------------[6] Ba điểm thẳng hàng, ba ĐT đồng quy
-------------[8] Bài toán liên quan đến tỉ số
-------------[9] Xét tính đồng phẳng
-------------[7] Bài toán thực tế
-------------[0] Câu hỏi tổng hợp
-------------[A] Chưa phân dạng
----------[2] Hai ĐT song song
-------------[1] Câu hỏi lý thuyết
-------------[2] Hai ĐT song song
-------------[3] Tìm GIAO TUYẾN bằng kẻ song song
-------------[4] Tìm GIAO ĐIỂM của ĐT và MP bằng kẻ song song
-------------[A] Định lí 3 giao tuyến
-------------[5] Xác định thiết diện và ứng dụng
-------------[6] Ba điểm thẳng hàng
-------------[7] Bài toán quỹ tích và điểm cố định, tỉ số (Thales)
-------------[9] Vị trí tương đối 2 đường thẳng
-------------[8] Bài toán thực tế
-------------[0] Câu hỏi tổng hợp
-------------[A] Chưa phân dạng
----------[3] ĐT và MP song song
-------------[1] Câu hỏi lý thuyết
-------------[2] Vị trí tương đối ĐT và MP
-------------[3] ĐT song song với MP
-------------[4] Tìm giao tuyến bằng cách kẻ song song
-------------[5] Tìm giao điểm của ĐT và MP
-------------[6] Xác định thiết diện và ứng dụng
-------------[7] Ba điểm thẳng hàng
-------------[8] Bài toán quỹ tích và điểm cố định, tỉ số
-------------[9] Bài toán thực tế
-------------[0] Câu hỏi tổng hợp
-------------[A] Chưa phân dạng
----------[4] Hai MP song song
-------------[1] Câu hỏi lý thuyết
-------------[2] Vị trí tương đối 2 MP
-------------[3] Hai MP song song
-------------[4] Chứng minh ĐT song song MP
-------------[5] Xác định MP đi qua một điểm và song song với mp
-------------[6] Xác định MP song song với hai đường thẳng
-------------[9] Bài toán quỹ tích, điểm cố định, tỉ số 
-------------[8] Bài toán thực tế
-------------[7] Bài toán tổng hợp
-------------[A] Chưa phân dạng
----------[5] Hình lăng trụ và hình hộp
-------------[1] Câu hỏi lý thuyết
-------------[2] Bài toán về hình lăng trụ xiên
-------------[3] Bài toán về hình hộp xiên
-------------[4] Toán thực tế áp dụng hình lăng trụ và hình hộp
-------------[0] Câu hỏi tổng hợp
-------------[A] Chưa phân dạng
----------[6] Phép chiếu song song
-------------[1] Câu hỏi lý thuyết
-------------[2] Hình biểu diễn của một hình không gian
-------------[3] Xác định yếu tố song song
-------------[4] Xác định phương chiếu
-------------[5] Tính tỉ số đoạn thẳng, diện tích qua phép chiếu
-------------[0] Câu hỏi tổng hợp
-------------[A] Chưa phân dạng
----------[0] Chưa phân dạng
-------------[0] Chưa phân dạng
-------[5] Các số đặc trưng đo xu thế trung tâm cho mẫu số liệu ghép nhóm
----------[1] Số trung bình và mốt của mẫu số liệu ghép nhóm
-------------[1] Câu hỏi lý thuyết
-------------[2] Mẫu số liệu ghép nhóm
-------------[3] Số trung bình
-------------[4] Mốt
-------------[5] Nhận xết các số đặc trưng
-------------[0] Câu hỏi tổng hợp
-------------[A] Chưa phân dạng
----------[2] Trung vị và tứ phân vị của mẫu số liệu ghép nhóm
-------------[1] Câu hỏi lý thuyết
-------------[2] Trung vị
-------------[3] Tứ phân vị
-------------[4] Nhận xét khoảng tứ phân vị
-------------[0] Câu hỏi tổng hợp
-------------[A] Chưa phân dạng
-------[6] HS mũ và HS lôgarít
----------[1] Phép tính luỹ thừa
-------------[1] Công thức, lý thuyết
-------------[2] Tính giá trị của biểu thức chứa lũy thừa
-------------[3] Biến đổi, rút gọn biểu thức chứa lũy thừa
-------------[4] Điều kiện cho luỹ thừa, căn thức
-------------[5] Điều kiện hàm lũy thừa (cũ)
-------------[6] 
-------------[0] Câu hỏi tổng hợp
-------------[A] Chưa phân dạng
----------[2] Phép tính lôgarít
-------------[1] Công thức, lý thuyết
-------------[2] Tính giá trị biểu thức chứa lôgarít
-------------[3] Biến đổi, biểu diễn biểu thức chứa lôgarít
-------------[4] Rút gọn, chứng minh biểu thức chứa lôgarít
-------------[5] [G] Liên hệ với kiến thức khác
-------------[6] [G]
-------------[0] Câu hỏi tổng hợp
-------------[A] Chưa phân dạng
----------[3] HS mũ
-------------[1] Lý thuyết tổng hợp HS mũ
-------------[2] Tập xác định của HS mũ
-------------[3] Sự biến thiên và ĐT của HS mũ
-------------[4] So sánh các luỹ thừa
-------------[5] Bài toán thực tế, liên môn
-------------[7] Số e và bài toán LÃI KÉP
-------------[6] Tổng hợp các kiến thức và liên môn
-------------[0] Câu hỏi tổng hợp
-------------[A] Chưa phân dạng
----------[4] HS lôgarít
-------------[1] Lý thuyết tổng hợp HS lôgarít
-------------[2] Tập xác định của HS lôgarít
-------------[3] Sự biến thiên và ĐT của HS lôgarít
-------------[4] So sánh các lôgarít
-------------[5] Bài toán thực tế, liên môn
-------------[6] Tổng hợp các kiến thức và liên môn
-------------[0] Câu hỏi tổng hợp
-------------[A] Chưa phân dạng
----------[5] Phương trình, Bất phương trình  mũ 
-------------[1] Điều kiện có nghiệm PT mũ
-------------[2] Nghiệm của PT BPT mũ
-------------[3] Phương trình mũ cơ bản
-------------[5] Phương trình mũ đưa về cùng cơ số
-------------[4] Bất phương trình mũ cơ bản
-------------[6] Bất phương trình mũ đưa về cùng cơ số
-------------[B] Phương trình, bất phương trình đưa về dạng cơ bản
-------------[9] Câu hỏi tổng hợp, liên hệ kiến thức khác
-------------[7] Bài toán thực tế, liên môn
-------------[C] Bài toán lãi kép
-------------[8] [G] PT BPT có tham số
-------------[0] Câu hỏi tổng hợp
-------------[A] Chưa phân dạng
----------[6] Phương trình, Bất phương trình lôgarít
-------------[1] Điều kiện có nghiệm PT lôgarít
-------------[2] Nghiệm của PT BPT lôgarit
-------------[3] Phương trình lôgarít cơ bản
-------------[5] Phương trình lôgarít đưa về cùng cơ số
-------------[4] Bất phương trình lôgarít cơ bản
-------------[6] Bất phương trình lôgarít đưa về cùng cơ số
-------------[7] Bài toán thực tế, liên môn
-------------[8] PT BPT có tham số
-------------[B] Phương trình, bất phương trình đưa về dạng cơ bản
-------------[9] Câu hỏi tổng hợp, liên hệ kiến thức khác
-------------[0] Câu hỏi tổng hợp 
-------------[A] Chưa phân dạng
----------[7] [Giảm] Các phương pháp giải được giảm tải
-------------[1] Phương pháp đặt ẩn phụ cho PT mũ, lôgarít
-------------[2] Phương pháp lôgarít hóa, mũ cho PT mũ, lôgarít
-------------[3] Phương pháp HS, đánh giá cho PT mũ, lôgarít (KT 12)
-------------[4] Hệ PT mũ, lôgarít
-------------[5] Phương pháp đặt ẩn phụ với BPT mũ, lôgarít
-------------[6] Phương pháp lôgarít hóa, mũ cho BPT mũ, lôgarít
-------------[7] Phương pháp HS, đánh giá cho BPT mũ, lôgarít (KT 12)
-------------[8] Hệ BPT mũ, lôgarít
-------------[0] Câu hỏi tổng hợp
-------------[A] Chưa phân dạng
----------[0] Chưa phân dạng
-------------[0] Chưa phân dạng
-------------[2] Câu hỏi liên quan hàm lũy thừa
-------[7] Đạo hàm
----------[1] Đạo hàm
-------------[1] Tính đạo hàm bằng định nghĩa
-------------[2] Câu hỏi lý thuyết
-------------[3] Số e và định nghĩa
-------------[4] Số gia HS, số gia biến số
-------------[5] Ý nghĩa hình học của đạo hàm
-------------[6] Ý nghĩa Vật lý của đạo hàm
-------------[7] Bài toán thực tế, liên môn khác
-------------[8] Mối quan hệ đạo hàm và giới hạn, liên tục và các bài toán liên quan
-------------[0] Câu hỏi tổng hợp
-------------[A] Chưa phân dạng
----------[2] Các QT đạo hàm
-------------[1] Câu hỏi lý thuyết
-------------[E] Tính đạo hàm tại điểm
-------------[2] Tính đạo hàm và đa thức, căn thức, lũy thừa 
-------------[3] Tính đạo hàm lượng giác
-------------[4] Tính đạo hàm mũ, log
-------------[5] Tính đạo hàm của tổng, hiệu, tích, thương
-------------[8] Tính đạo hàm hợp (hàm ẩn) nâng cao
-------------[9] Tìm tham số m thỏa điều kiện
-------------[B] Dùng đạo hàm cho nhị thức Newton hoặc dãy số
-------------[C] Giới hạn hàm lượng giác, hàm mũ
-------------[D] Mối quan hệ đạo hàm và liên tục với các bài toán nâng cao
-------------[6] Bài toán có liên quan đến đẳng thức, bất đẳng thức có y và y' 
-------------[7] Bài toán thực tế, liên môn
-------------[0] Câu hỏi tổng hợp
-------------[E] Chưa phân dạng
----------[3] Đạo hàm cấp hai
-------------[1] Tính đạo hàm cấp hai
-------------[2] Đẳng thức có y và (y', y'')
-------------[3] Ý nghĩa Vật lý của đạo hàm cấp hai
-------------[4] Vi phân (cũ)
-------------[5] Đạo hàm cấp n (n>2) (cũ)
-------------[0] Câu hỏi tổng hợp
-------------[A] Chưa phân dạng
----------[4] Phương trình tiếp tuyến
-------------[1] Câu hỏi lý thuyết
-------------[2] Tiếp tuyến tại một điểm
-------------[3] Tiếp tuyến biết hệ số góc
-------------[4] Tiếp tuyến qua một điểm 
-------------[5] Tiếp tuyến biết song song với ĐT
-------------[6] Tiếp tuyến biết vuông góc với ĐT
-------------[7] Tiếp tuyến với các kiến thức liên quan
-------------[0] Câu hỏi tổng hợp
-------------[A] Chưa phân dạng
----------[0] Chưa phân dạng
-------------[0] Chưa phân dạng
-------[8] Quan hệ vuông góc trong không gian
----------[1] Hai ĐT vuông góc
-------------[1] Câu hỏi lí thuyết
-------------[2] Xác định hai ĐT vuông góc
-------------[3] Các bài toán thực tế
-------------[4] Các bài toán tổng hợp 
-------------[0] Câu hỏi tổng hợp
-------------[A] Chưa phân dạng
----------[2] ĐT vuông góc với MP
-------------[1] Câu hỏi lí thuyết
-------------[2] Xác định ĐT và MP vuông góc
-------------[3] Xác định hai ĐT vuông góc
-------------[4] Dựng MP, tìm thiết diện và các vấn đề liên quan
-------------[5] Các bài toán thực tế
-------------[6] Bài toán tổng hợp (góc,...)
-------------[0] Câu hỏi tổng hợp
-------------[A] Chưa phân dạng
----------[3] Phép chiếu vuông góc
-------------[1] Lý thuyết về phép chiếu vuông góc
-------------[2] Hình chiếu vuông góc của điểm lên MP
-------------[3] Hình chiếu vuông góc của đa giác trên MP
-------------[4] Các bài toán thực tế
-------------[5] Bài toán tổng hợp (góc,...)
-------------[0] Câu hỏi tổng hợp
-------------[A] Chưa phân dạng
----------[4] Hai MP vuông góc
-------------[1] Câu hỏi lí thuyết
-------------[2] Xác định đường vuông góc với mặt phẳng và các vấn đề liên quan
-------------[3] XĐ/CM 2 MP vuông góc
-------------[4] Dựng MP, thiết diện và các vấn đề liên quan
-------------[5] Các bài toán thực tế
-------------[6] Bài toán tổng hợp (góc,...)
-------------[0] Câu hỏi tổng hợp
-------------[A] Chưa phân dạng
----------[5] Khoảng cách
-------------[1] Câu hỏi lí thuyết
-------------[2] Khoảng cách từ 1 điểm đến 1 ĐT
-------------[3] Khoảng cách từ 1 điểm đến 1 MP
-------------[4] Khoảng cách giữa hai ĐT chéo nhau
-------------[5] Đường vuông góc chung của hai ĐT chéo nhau
-------------[6] Khoảng cách giữa đường thằng và mặt phẳng song song
-------------[7] Khoảng cách thông qua công thức tỉ số
-------------[8] Các bài toán thực tế
-------------[9] Bài toán tổng hợp (góc,...)
-------------[0] Câu hỏi tổng hợp
-------------[A] Chưa phân dạng
----------[6] Góc giữa ĐT và MP. Góc nhị diện
-------------[9] Câu hỏi lí thuyết
-------------[1] Góc giữa hai đường thẳng
-------------[2] Góc giữa đường thẳng và mặt phẳng
-------------[3] Góc nhị diện, góc phẳng nhị diện
-------------[4] Xác định góc giữa hai mặt phẳng
-------------[5] Góc giữa hai MP thông qua tỉ số diện tích
-------------[6] Các bài toán thực tế
-------------[7] Bài toán tổng hợp (góc,...)
-------------[0] Câu hỏi tổng hợp
-------------[A] Chưa phân dạng
----------[7] Thể tích các hình khối
-------------[1] Câu hỏi lý thuyết hình, công thức
-------------[2] Nhận diện và tính toán các yếu tố của các hình khối thông dụng
-------------[3] Thể tích khối chóp tam giác
-------------[4] Thể tích khối chóp tứ giác
-------------[5] Thể tích các khối lăng trụ tam giác
-------------[6] Thể tích các khối lăng trụ tứ giác
-------------[7] Thể tích khối chóp cụt
-------------[8] Các bài toán thực tế
-------------[0] Câu hỏi tổng hợp
-------------[A] Chưa phân dạng
----------[8] ỨNG DỤNG
-------------[1] Ứng dụng thể tích tính góc và khoảng cách
-------------[2] Bài toán cực trị
-------------[3] Bài toán tổng hợp
-------------[4] Các bài toán thực tế
-------------[0] Câu hỏi tổng hợp
-------------[A] Chưa phân dạng
----------[9] CÂU HỎI NGOÀI CHƯƠNG TRÌNH
-------------[1] Tỉ số thể tích
-------------[3] Thể tích ứng dụng tỉ số khoảng cách
-------------[2] Câu hỏi ngoài chương trình
-------------[A] Chưa phân dạng
----------[0] Chưa phân dạng
-------------[0] Chưa phân dạng
-------[9] Xác suất
----------[1] Các khái niệm về biến cố
-------------[1] Câu hỏi lí thuyết
-------------[2] Xác định biến cố giao
-------------[3] Xác định biến cố xung khắc
-------------[4] Xác định biến cố độc lập
-------------[5] Xác định biến cố hợp
-------------[7] Xác định biến cố đối
-------------[6] Mô tả không gian mẫu, biến cố
-------------[0] Câu hỏi tổng hợp
-------------[A] Chưa phân dạng
----------[2] Công thức xác suất
-------------[1] Câu hỏi lí thuyết
-------------[2] QT nhân cho hai biến cố độc lập
-------------[3] QT cộng cho hai biến cố xung khắc
-------------[4] QT cộng cho hai biến cố bất kì
-------------[5] Tính xác suất kết hợp QT cộng và QT nhân
-------------[6] Tính xác suất bằng sơ đồ hình cây
-------------[7] Tính xác suất cổ điển KT lớp 11
-------------[0] Câu hỏi tổng hợp
-------------[A] Chưa phân dạng
----------[0] Chưa phân dạng
-------------[0] Chưa phân dạng
%==============================================================ID 11 DPS
----[D] 11 - Đại số và giải tích
-------[1] HS lượng giác và phương trình lượng giác
----------[1] Góc lượng giác
-------------[1] Câu hỏi lý thuyết
-------------[2] Chuyển đổi đơn vị độ và radian
-------------[3] Số đo của một góc lượng giác
-------------[4] Độ dài của một cung tròn
-------------[5] Đường tròn lượng giác và ứng dụng
-------------[6] Các bài toán thực tế, liên môn
----------[2] Giá trị lượng giác của một góc lượng giác
-------------[1] Câu hỏi lý thuyết
-------------[2] Xét dấu các giá trị lượng giác
-------------[3] Biến đổi, thu gọn 1 biểu thức lượng giác
-------------[4] CT liên kết LG
-------------[5] Các bài toán có yếu tố thực tế, liên môn
-------------[6] Tính giá trị lượng giác của một góc
----------[3] Các công thức lượng giác
-------------[1] Câu hỏi lý thuyết
-------------[2] Áp dụng công thức cộng
-------------[3] Áp dụng công thức nhân đôi - hạ bậc
-------------[4] Áp dụng công thức biến đổi tích <-> tổng
-------------[5] Kết hợp nhiều công thức lượng giác
-------------[6] Nhận dạng tam giác
-------------[7] Các bài toán có yếu tố thực tế, liên môn
----------[4] HS lượng giác và ĐT
-------------[1] Câu hỏi lý thuyết
-------------[2] Tìm tập xác định
-------------[3] Xét tính đơn điệu
-------------[4] Xét tính chẵn, lẻ
-------------[5] Xét tính tuần hoàn, tìm chu kỳ
-------------[6] Tìm tập giá trị và min, max
-------------[7] Bảng biến thiên và ĐT
-------------[8] Toán thực tế áp dụng HS lượng giác
----------[5] PT lượng giác cơ bản
-------------[1] Câu hỏi lý thuyết. Khái niệm phương trình tương đương
-------------[2] Điều kiện có nghiệm
-------------[3] PT cơ bản dùng Radian
-------------[4] PT cơ bản dùng Độ
-------------[5] PT đưa về dạng cơ bản
-------------[6] Toán thực tế, liên môn
----------[6] [Giảm] PT lượng giác thường gặp
-------------[1] PT bậc n theo một HS lượng giác
-------------[2] PT đẳng cấp bậc n đối với sinx và cosx
-------------[3] PT bậc nhất đối với sinx và cosx
-------------[4] PT đối xứng, phản đối xứng
-------------[5] PT lượng giác không mẫu mực
-------------[6] PT lượng giác có chứa ẩn ở mẫu số
-------------[7] PT lượng giác có chứa tham số
-------------[8] Bài toán thực tế
----------[0] Chưa phân dạng
-------------[0] Chưa phân dạng
-------[2] Dãy số. Cấp số cộng. Cấp số nhân
----------[1] Dãy số
-------------[1] Câu hỏi lý thuyết
-------------[2] Số hạng tổng quát, biểu diễn dãy số
-------------[3] Tìm số hạng cụ thể của dãy số
-------------[4] Dãy số tăng, dãy số giảm
-------------[5] Dãy số bị chặn
-------------[6] Toán thực tế về dãy số
----------[2] Cấp số cộng
-------------[1] Câu hỏi lý thuyết
-------------[2] Nhận diện cấp số cộng, công sai d
-------------[3] Số hạng tổng quát của cấp số cộng
-------------[4] Tìm số hạng cụ thể trong cấp số cộng
-------------[5] ĐK là CSC, tính chất CSC
-------------[6] Tính tổng của cấp số cộng
-------------[7] Các bài toán thực tế
----------[3] Cấp số nhân
-------------[1] Câu hỏi lý thuyết
-------------[2] Nhận diện cấp số nhân, công bội q
-------------[3] Số hạng tổng quát của cấp số nhân
-------------[4] Tìm số hạng cụ thể trong cấp số nhân
-------------[5] Điều kiện để dãy số là cấp số nhân
-------------[6] Tính tổng của cấp số nhân
-------------[7] Kết hợp cấp số nhân và cấp số cộng
-------------[8] Các bài toán thực tế
----------[0] Chưa phân dạng
-------------[0] Chưa phân dạng
-------[3] Giới hạn. HS liên tục
----------[1] Giới hạn của dãy số
-------------[1] Câu hỏi lý thuyết
-------------[2] PP đặt thừa số chung, kết quả hữu hạn
-------------[3] PP lượng liên hợp, kết quả hữu hạn
-------------[4] PP đặt thừa số chung, kết quả vô hạn
-------------[5] CSN lùi vô hạn 
-------------[6] Toán thực tế, liên môn
-------------[8] PP lượng liên hợp, kết quả vô hạn
-------------[9] GH hàm ẩn
-------------[7] [Giảm] Nguyên lí kẹp
----------[2] Giới hạn của HS
-------------[1] Câu hỏi lý thuyết
-------------[2] Thay số trực tiếp
-------------[3] GH hữu hạn PP đặt thừa số chung, kết quả HH
-------------[4] GH hữu hạn PP đặt thừa số chung, kết quả vô cực
-------------[5] GH hữu hạn PP lượng liên hợp, kết quả HH
-------------[6] GH hữu hạn PP lượng liên hợp, kết quả vô cực
-------------[7] Giới hạn một bên
-------------[8] Toán thực tế, liên môn
-------------[9] Giới hạn HS dựa vào ĐT
-------------[A] GH vô hạn PP đặt thừa số chung, kết quả HH
-------------[B] GH vô hạn đặt thừa số chung, kết quả vô cực
-------------[C] GH vô hạn lượng liên hợp, kết quả HH
-------------[D] GH vô hạn lượng liên hợp, kết quả vô cực
-------------[E] GH hàm ẩn
----------[3] HS liên tục
-------------[1] Câu hỏi lý thuyết
-------------[2] Tính liên tục thể hiện qua ĐT
-------------[3] HS liên tục tại một điểm
-------------[4] HS liên tục trên khoảng, đoạn
-------------[5] Bài toán phương trình có nghiệm
-------------[6] Toán thực tế, liên môn
-------------[7] Bài toán có chứa tham số
----------[0] Chưa phân dạng
-------------[0] Chưa phân dạng
-------[5] Các số đặc trưng đo xu thế trung tâm cho mẫu số liệu ghép nhóm
----------[1] Số trung bình và mốt của mẫu số liệu ghép nhóm
-------------[1] Câu hỏi lý thuyết
-------------[2] Mẫu số liệu ghép nhóm
-------------[3] Số trung bình
-------------[4] Mốt
----------[2] Trung vị và tứ phân vị của mẫu số liệu ghép nhóm
-------------[1] Câu hỏi lý thuyết
-------------[2] Trung vị
-------------[3] Tứ phân vị
-------------[4] Nhận xét tứ phân vị
----------[0] Chưa phân dạng
-------------[0] Chưa phân dạng
-------[6] HS mũ và HS lôgarít
----------[1] Phép tính luỹ thừa
-------------[1] Tính giá trị của biểu thức chứa lũy thừa
-------------[2] Biến đổi, rút gọn biểu thức chứa lũy thừa
-------------[3] Điều kiện cho luỹ thừa, căn thức
-------------[4] So sánh các lũy thừa
----------[2] Phép tính lôgarít
-------------[1] Tính giá trị biểu thức chứa lôgarít
-------------[2] Biến đổi, biểu diễn biểu thức chứa lôgarít
-------------[3] Rút gọn, chứng minh biểu thức chứa lôgarít
-------------[4] Số e và bài toán lãi kép
-------------[5] Toán thực tế áp dụng phép tính lôgarit
----------[3] HS mũ. HS lôgarít
-------------[1] Lý thuyết tổng hợp HS lũy thừa, mũ, lôgarít
-------------[2] Tập xác định của HS mũ
-------------[3] Sự biến thiên và ĐT của HS mũ
-------------[4] So sánh các luỹ thừa
-------------[5] Bài toán thực tế, liên môn
-------------[6] Tập xác định của HS lôgarít
-------------[7] Sự biến thiên và ĐT của HS lôgarít
-------------[8] So sánh các lôgarít
----------[4] PT, BPT mũ và lôgarít
-------------[1] Điều kiện có nghiệm PT mũ
-------------[2] PT mũ cơ bản
-------------[3] Bất phương trình mũ cơ bản
-------------[4] PT mũ đưa về cùng cơ số
-------------[5] Bất phương trình mũ đưa về cùng cơ số
-------------[6] Bài toán thực tế, liên môn
-------------[7] Điều kiện có nghiệm PT lôgarít
-------------[8] PT lôgarít cơ bản
-------------[9] Bất phương trình lôgarít cơ bản
-------------[A] PT lôgarít đưa về cùng cơ số
-------------[B] Bất phương trình lôgarít đưa về cùng cơ số
-------------[C] PT BPT có tham số
----------[5] [Giảm] Các phương pháp giải được giảm tải
-------------[1] Phương pháp đặt ẩn phụ cho PT mũ, lôgarít
-------------[2] Phương pháp lôgarít hóa, mũ cho PT mũ, lôgarít
-------------[3] Phương pháp HS, đánh giá cho PT mũ, lôgarít (KT 12)
-------------[4] Hệ PT mũ, lôgarít
-------------[5] Phương pháp đặt ẩn phụ với BPT mũ, lôgarít
-------------[6] Phương pháp lôgarít hóa, mũ cho BPT mũ, lôgarít
-------------[7] Phương pháp HS, đánh giá cho BPT mũ, lôgarít (KT 12)
-------------[8] Hệ BPT mũ, lôgarít
----------[0] Chưa phân dạng
-------------[0] Chưa phân dạng
-------[7] Đạo hàm
----------[1] Đạo hàm
-------------[1] Tính đạo hàm bằng định nghĩa
-------------[2] Số gia HS, số gia biến số
-------------[3] Ý nghĩa hình học của đạo hàm
-------------[4] Ý nghĩa Vật lý của đạo hàm
-------------[5] Bài toán thực tế, liên môn khác
----------[2] Các QT đạo hàm
-------------[1] Tính đạo hàm
-------------[2] Đẳng thức có y và y'
-------------[3] Tiếp tuyến tại một điểm
-------------[4] Tiếp tuyến biết trước hệ số góc
-------------[5] Tiếp tuyến chưa biết tiếp điểm và hệ số góc
-------------[6] Giới hạn HS lượng giác, HS mũ, lôgarít
-------------[7] Dùng đạo hàm cho nhị thức Newton hoặc dãy số
-------------[8] Bài toán thực tế, liên môn
----------[3] Đạo hàm cấp hai
-------------[1] Tính đạo hàm cấp hai
-------------[2] Đẳng thức có y và (y', y'')
-------------[3] Ý nghĩa Vật lý của đạo hàm cấp hai
----------[0] Chưa phân dạng
-------------[0] Chưa phân dạng
-------[9] Xác suất
----------[1] Các khái niệm về biến cố
-------------[1] Câu hỏi lí thuyết
-------------[2] Mô tả không gian mẫu, biến cố
-------------[3] Xác định biến cố giao
-------------[4] Xác định biến cố xung khắc
-------------[5] Xác định biến cố độc lập
----------[2] Công thức xác suất
-------------[1] Câu hỏi lí thuyết
-------------[2] Mô tả biến cố hợp
-------------[3] QT nhân cho hai biến cố độc lập
-------------[4] QT cộng cho hai biến cố xung khắc
-------------[5] QT cộng cho hai biến cố bất kì
-------------[6] Tính xác suất bằng sơ đồ hình cây
-------------[7] Tính xác suất kết hợp QT cộng và QT nhân
----------[0] Chưa phân dạng
-------------[0] Chưa phân dạng
-------[0] Chưa phân dạng
----------[0] Chưa phân dạng
-------------[0] Chưa phân dạng
----[H] 11 - Hình học và đo lường
-------[4] ĐT, MP. Quan hệ song song trong không gian
----------[1] Điểm, ĐT và MP
-------------[1] Câu hỏi lý thuyết
-------------[2] Hình biểu diễn của một hình không gian
-------------[3] Tìm GIAO TUYẾN của hai MP
-------------[4] Tìm GIAO ĐIỂM của ĐT và MP
-------------[5] Xác định thiết diện
-------------[6] Ba điểm thẳng hàng, ba ĐT đồng quy
-------------[7] Bài toán thực tế
-------------[8] Tỉ số nâng cao thông qua Menelaus và Ceva
----------[2] Hai ĐT song song
-------------[1] Câu hỏi lý thuyết
-------------[2] Hai ĐT song song
-------------[3] Tìm GIAO TUYẾN bằng kẻ song song
-------------[4] Tìm GIAO ĐIỂM của đường thẳng và MP bằng kẻ song song
-------------[5] Xác định thiết diện bằng kẻ song song
-------------[6] Ba điểm thẳng hàng
-------------[7] Bài toán quỹ tích và điểm cố định, tỉ số (Thales)
-------------[8] Bài toán thực tế
-------------[9] Vị trí tương đối 2 đt
----------[3] ĐT và MP song song
-------------[1] Câu hỏi lý thuyết
-------------[2] ĐƯỜNG THẲNG song song với MẶT PHẲNG
-------------[3] Tìm giao tuyến bằng cách kẻ song song
-------------[4] Tìm giao điểm của ĐT và MP
-------------[5] Xác định thiết diện và các bài toán liên quan
-------------[6] Ba điểm thẳng hàng
-------------[7] Bài toán quỹ tích và điểm cố định, tỉ số
-------------[8] Bài toán thực tế
-------------[9] Vị trí tương đối ĐT và MP
----------[4] Hai MP song song
-------------[1] Câu hỏi lý thuyết
-------------[2] Hai MP song song
-------------[3] Chứng minh ĐT song song MP
-------------[4] Xác định mp đi qua một điểm và song song với mp
-------------[5] Xác định mp chứa ĐT hoặc đi qua 2 điểm và song song vs mp
-------------[6] Bài toán tổng hợp
-------------[7] Bài toán thực tế
-------------[8] Vị trí tương đối 2 MP
----------[5] Hình lăng trụ và hình hộp
-------------[1] Câu hỏi lý thuyết
-------------[2] Bài toán về hình lăng trụ xiên
-------------[3] Bài toán về hình hộp xiên
-------------[4] Toán thực tế áp dụng hình lăng trụ và hình hộp
----------[6] Phép chiếu song song
-------------[1] Câu hỏi lý thuyết
-------------[2] Hình biểu diễn của một hình không gian
-------------[3] Xác định yế tố song song
-------------[4] Xác định phương chiếu
-------------[5] Tính tỉ số đoạn thẳng, diện tích qua phép chiếu
----------[0] Chưa phân dạng
-------------[0] Chưa phân dạng
-------[8] Quan hệ vuông góc trong không gian
----------[1] Hai ĐT vuông góc
-------------[1] Câu hỏi lí thuyết
-------------[2] Xác định hai ĐT vuông góc
-------------[3] Tìm góc giữa hai ĐT
-------------[4] Các bài toán thực tế
-------------[5] Các bài toán tổng hợp  (góc,...)
----------[2] ĐT vuông góc với MP
-------------[1] Câu hỏi lí thuyết
-------------[2] Xác định ĐT và MP vuông góc
-------------[3] Xác định hai ĐT vuông góc
-------------[4] Dựng MP, tìm thiết diện và các vấn đề liên quan
-------------[5] Các bài toán thực tế
-------------[6] Bài toán tổng hợp (góc,...)
----------[3] Phép chiếu vuông góc
-------------[1] Lý thuyết về phép chiếu vuông góc
-------------[2] Hình chiếu vuông góc của đa giác trên MP
-------------[3] Các bài toán thực tế
-------------[4] Bài toán tổng hợp (góc,...)
-------------[5] Hình chiếu vuông góc của điểm lên MP
----------[4] Hai MP vuông góc
-------------[1] Câu hỏi lí thuyết
-------------[2] XĐ/CM 2 MP vuông góc
-------------[3] Xác định góc giữa hai MP
-------------[4] Dựng MP, thiết diện
-------------[5] Nhận dạng và tính toán liên quan các hình thông dụng
-------------[6] Bài toán cho trước góc giữa d và (P)
-------------[7] Các bài toán thực tế
-------------[9] Bài toán tổng hợp (góc,...)
----------[5] Khoảng cách
-------------[1] Câu hỏi lí thuyết
-------------[2] Khoảng cách từ 1 điểm đến 1 ĐT
-------------[3] Khoảng cách từ 1 điểm đến 1 MP
-------------[4] Khoảng cách giữa hai ĐT chéo nhau
-------------[5] Đường vuông góc chung của hai ĐT chéo nhau
-------------[6] Các bài toán thực tế
-------------[7] Bài toán tổng hợp (góc,...)
-------------[8] Khoảng cách giữa đường thằng và mặt phẳng song son
----------[6] Góc giữa ĐT và MP. Góc nhị diện
-------------[1] Góc giữa ĐT và MP
-------------[2] Góc nhị diện, góc phẳng nhị diện
-------------[3] Góc giữa 2 MP, biết trước góc (d,(P))
-------------[4] Khoảng cách giữa điểm, đường, biết trước góc (d,(P))
-------------[5] Khoảng cách giữa điểm - MP, biết trước góc (d,(P))
-------------[6] Khoảng cách giữa 2 đường chéo nhau, biết trước góc (d,(P))
-------------[7] Các bài toán thực tế
-------------[8] Bài toán tổng hợp (góc,...)
----------[7] Hình lăng trụ đứng. Hình chóp đều. Thể tích
-------------[1] Câu hỏi lý thuyết, công thức
-------------[2] Thể tích khối chóp tam giác
-------------[3] Thể tích khối chóp tứ giác
-------------[4] Thể tích các khối lăng trụ tam giác
-------------[5] Thể tích các khối lăng trụ tứ giác
-------------[6] Thể tích khối chóp cụt
-------------[7] Tỉ số thể tích
-------------[8] Ứng dụng thể tích tính góc và khoảng cách
-------------[9] Các bài toán thực tế
-------------[A] Bài toán cực trị
-------------[B] Bài toán tổng hợp (góc,...)
----------[0] Chưa phân dạng
-------------[0] Chưa phân dạng
-------[0] Chưa phân dạng
----------[0] Chưa phân dạng
-------------[0] Chưa phân dạng
----[C] 11 - Chuyên đề
-------[0] Chưa phân dạng
----------[0] Chưa phân dạng
-------------[0] Chưa phân dạng
-------[1] [Chuyên đề] Phép dời hình và phép đồng dạng trong MP
----------[1] Phép biến hình
-------------[1] Câu hỏi lý thuyết
-------------[2] Bài toán xác định một phép đặt tương ứng có là phép dời hình hay không
-------------[A] Chưa phân dạng
----------[2] Phép tịnh tiến
-------------[1] Câu hỏi lý thuyết
-------------[2] Tìm ảnh hoặc tạo ảnh khi thực hiện phép tịnh tiến
-------------[3] Ứng dụng phép tịnh tiến
-------------[A] Chưa phân dạng
----------[3] Phép đối xứng trục
-------------[1] Câu hỏi lý thuyết
-------------[2] Tìm ảnh hoặc tạo ảnh khi thực hiện phép đối xứng trục
-------------[3] Xác định trục đối xứng và số trục đối xứng của một hình
-------------[4] Ứng dụng phép đối xứng trục
-------------[A] Chưa phân dạng
----------[4] Phép đối xứng tâm
-------------[1] Câu hỏi lý thuyết
-------------[2] Tìm ảnh, tạo ảnh khi thực hiện phép đối xứng tâm
-------------[3] Xác định hình có tâm đối xứng
-------------[4] Ứng dụng phép đối xứng tâm
-------------[A] Chưa phân dạng
----------[5] Phép quay
-------------[1] Câu hỏi lý thuyết
-------------[2] Xác định vị trí ảnh của điểm, hình khi thực hiện phép quay cho trước
-------------[3] Tìm tọa độ ảnh của điểm, phương trình của một ĐT
-------------[A] Chưa phân dạng
----------[6] Khái niệm về phép dời hình và hai hình bằng nhau
-------------[1] Câu hỏi lý thuyết
-------------[2] Xác định ảnh khi thực hiện phép dời hình
-------------[A] Chưa phân dạng
----------[7] Phép vị tự
-------------[1] Câu hỏi lý thuyết
-------------[2] Xác định ảnh, tạo ảnh khi thực hiện phép vị tự
-------------[3] Tìm tâm vị tự của hai đường tròn
-------------[4] Ứng dụng phép vị tự
-------------[A] Chưa phân dạng
----------[8] Phép đồng dạng
-------------[1] Câu hỏi lý thuyết
-------------[2] Xác định ảnh, tạo ảnh khi thực hiện phép đồng dạng
-------------[A] Chưa phân dạng
-------[2] [Chuyên đề] Lý thuyết ĐT
----------[1] Đồ thị
-------------[1] Câu hỏi đỉnh, cạnh của ĐT
-------------[2] Bậc của ĐT
-------------[3] Câu hỏi tổng hợp
-------------[A] Chưa phân dạng
----------[2] Đường đi Euler và Harmilton
-------------[1] Đường đi Euler
-------------[2] Đường đi Harmilton
-------------[3] Câu hỏi tổng hợp
-------------[A] Chưa phân dạng
----------[3] Bài toán đường đi ngắn nhất
-------------[1] Bài toán tìm đường đi ngắn nhất
-------------[2] Tổng hợp
-------------[A] Chưa phân dạng
-------[3] [Chuyên đề] Một số yếu tố vẽ kĩ thuật
----------[1] Hình biểu diễn của một hình, khối
-------------[1] LT về phép chiếu và hình biểu diễn song song
-------------[2] LT về phép chiếu vuông góc
-------------[3] LT về phép chiếu trục đo
-------------[3] Tổng hợp
-------------[A] Chưa phân dạng
----------[2] Bản vẽ kĩ thuật
-------------[1] LT bản vẽ kỹ thuật
-------------[2] PP biểu diễn bản vẽ kĩ thuật
-------------[3] Tổng hợp
-------------[A] Chưa phân dạng
-------[4] [Cũ] Véc-tơ trong không gian. Quan hệ vuông góc trong không gian
----------[1] Véc-tơ trong không gian
-------------[1] Câu hỏi lý thuyết
-------------[2] Đẳng thức véc-tơ
-------------[3] Phân tích véc-tơ theo các véc-tơ cho trước
-------------[4] Điều kiện đồng phẳng của ba véc-tơ
-------------[5] Ba điểm thẳng hàng, hai ĐT song song
-------------[6] Ứng dụng tích vô hướng của hai véc-tơ
-------------[A] Chưa phân dạng
-------[5] HS mũ và HS lôgarít
----------[1] Phép tính luỹ thừa
-------------[1] Tính giá trị của biểu thức chứa lũy thừa
-------------[2] Biến đổi, rút gọn biểu thức chứa lũy thừa
-------------[3] Điều kiện cho luỹ thừa, căn thức
-------------[4] So sánh các lũy thừa
-------------[A] Chưa phân dạng
----------[2] Phép tính lôgarít
-------------[1] Tính giá trị biểu thức chứa lôgarít
-------------[2] Biến đổi, biểu diễn biểu thức chứa lôgarít
-------------[3] Rút gọn, chứng minh biểu thức chứa lôgarít
-------------[4] Số e và bài toán lãi kép
-------------[5] Toán thực tế áp dụng phép tính lôgarit
-------------[A] Chưa phân dạng
----------[3] HS mũ. HS lôgarít
-------------[1] Lý thuyết tổng hợp HS lũy thừa, mũ, lôgarít
-------------[2] Tập xác định của HS
-------------[3] Sự biến thiên và ĐT của HS mũ, lôgarít
-------------[4] So sánh các luỹ thừa và lôgarít
-------------[5] Bài toán thực tế, liên môn
-------------[A] Chưa phân dạng
----------[4] PT, BPT mũ và lôgarít
-------------[1] Điều kiện có nghiệm
-------------[2] PT mũ, lôgarít cơ bản
-------------[3] BPT mũ, lôgarít cơ bản
-------------[4] PT mũ, lôgarít đưa về cùng cơ số
-------------[5] BPT mũ, lôgarít đưa về cùng cơ số
-------------[6] Bài toán thực tế, liên môn
-------------[A] Chưa phân dạng
----------[5] [Giảm] Các phương pháp giải được giảm tải
-------------[1] Phương pháp đặt ẩn phụ cho PT mũ, lôgarít
-------------[2] Phương pháp lôgarít hóa, mũ cho PT mũ, lôgarít
-------------[3] Phương pháp HS, đánh giá cho PT mũ, lôgarít
-------------[4] Hệ PT mũ, lôgarít
-------------[5] Phương pháp đặt ẩn phụ với BPT mũ, lôgarít
-------------[6] Phương pháp lôgarít hóa, mũ cho BPT mũ, lôgarít
-------------[7] Phương pháp HS, đánh giá cho BPT mũ, lôgarít
-------------[8] Hệ BPT mũ, lôgarít
-------------[A] Chưa phân dạng
-------[6] [Cũ] Thể tích
----------[1] Khái niệm về khối đa diện
-------------[1] Nhận diện hình đa diện, khối đa diện
-------------[2] Xác định số đỉnh, cạnh, mặt bên của một khối đa diện
-------------[3] Phân chia, lắp ghép các khối đa diện
-------------[4] Phép biến hình trong không gian
-------------[A] Chưa phân dạng
----------[2] Khối đa diện lồi và khối đa diện đều
-------------[1] Nhận diện đa diện lồi
-------------[2] Nhận diện loại đa diện đều
-------------[3] Tính chất đối xứng
-------------[A] Chưa phân dạng
----------[3] Khái niệm về thể tích của khối đa diện
-------------[1] Câu hỏi lý thuyết, công thức
-------------[2] Thể tích khối chóp tam giác
-------------[3] Thể tích khối chóp tứ giác
-------------[4] Thể tích các khối lăng trụ tam giác
-------------[5] Thể tích các khối lăng trụ tứ giác
-------------[6] Thể tích khối chóp cụt
-------------[7] Tỉ số thể tích
-------------[8] Ứng dụng thể tích tính góc và khoảng cách
-------------[9] Các bài toán thực tế
-------------[A] Bài toán cực trị
-------------[A] Chưa phân dạng
-------[7] [Cũ] Đạo hàm
----------[1] Vi phân
-------------[1] Vi phân
----------[2] Đạo hàm cấp n
-------------[1] Đạo hàm cấp n
----------[3] HS mũ
-------------[1] Lý thuyết tổng hợp HS mũ
-------------[2] Tập xác định của HS mũ
-------------[4] Sự biến thiên và ĐT của HS mũ
-------------[5] So sánh các luỹ thừa
-------------[6] Bài toán thực tế, liên môn
-------------[7] Tổng hợp các kiến thức và liên môn
----------[4] HS lôgarít
-------------[3] Lý thuyết tổng hợp HS lôgarít
-------------[7] Tập xác định của HS lôgarít
-------------[8] Sự biến thiên và ĐT của HS lôgarít
-------------[9] So sánh các lôgarít
-------------[6] Bài toán thực tế, liên môn
-------------[7] Tổng hợp các kiến thức và liên môn
----[G] 11 - HSG
-------[0] Chưa phân dạng
----------[0] Chưa phân dạng
-------------[0] Chưa phân dạng
-------[1] Một số chủ đề Đại số, giải tích bồi dưỡng HSG lớp 11
----------[0] Chưa phân dạng
-------------[0] Chưa phân dạng
----------[1] Dãy số nguyên; dãy số và giới hạn của dãy số
-------------[0] Chưa phân dạng
-------------[1] Giới hạn
-------------[2] Dãy số
----------[2] Phương trình hàm (những phương trình hàm mà đề bài và lời giải có sử dụng các kết quả liên quan đến giải tích như giới hạn, liên tục)
-------------[0] Chưa phân dạng
----------[3] Đa thức (những bài toán đa thức mà đề bài và lời giải có sử dụng các kết quả liên quan đến giải tích)
-------------[0] Chưa phân dạng
----------[4] Bất đẳng thức lượng giác
-------------[0] Chưa phân dạng
----------[5] Bất đẳng thức trong dãy số
-------------[0] Chưa phân dạng
----------[6] Tổ hợp
-------------[0] Chưa phân dạng
-------[2] Một số chủ đề Hình học bồi dưỡng HSG lớp 11
----------[0] Chưa phân dạng
-------------[0] Chưa phân dạng
----------[1] Phép biến hình và phép đồng dạng
-------------[0] Chưa phân dạng
----------[2] Hình học không gian
-------------[0] Chưa phân dạng
----[0] 10 - Chưa phân dạng
-------[0] Chưa phân dạng
----------[0] Chưa phân dạng
-------------[0] Chưa phân dạng
%==============================================================KẾT THÚC MAP ID 11

%==============================================================BẮT ĐẦU  MAP ID 12
%==============================================================ID 12 CŨ
-[2] Lớp 12
----[P] 12-NGÂN HÀNG CHÍNH
-------[1] Ứng dụng đạo hàm để khảo sát HS
----------[1] Sự đồng biến và nghịch biến của HS
-------------[1] Lý thuyết
-------------[2] Xét tính đơn điệu của HS cho bởi CT
-------------[8] Xét tính đơn điệu của HS cho bởi CT y'
-------------[B] Tìm hàm đơn điệu 
-------------[4] Xét tính đơn điệu dựa vào bảng biến thiên, ĐT
-------------[C] Xét tính đơn điệu dựa vào bảng biến thiên, ĐT hàm đạo hàm
-------------[7] Bài toán thực tế, liên môn
-------------[0] Câu hỏi tổng hợp
-------------[D] [G] Xác định hệ số
-------------[3] [G] Xét tính đơn điệu của hàm hợp, hàm trị tuyệt đối
-------------[6] [G] AD tính đơn điệu để chứng minh BĐT, giải PT, BPT, HPT
-------------[5] [G] Tìm tham số m để HS đơn điệu
-------------[A] Chưa phân dạng
----------[2] Cực trị của HS
-------------[1] Lý thuyết
-------------[2] Tìm cực trị của HS cho bởi CT
-------------[A] Tìm cực trị của HS cho bởi CT y'
-------------[4] Tìm cực trị dựa vào BBT, ĐT
-------------[9] Bài toán thực tế, liên môn
-------------[0] Câu hỏi tổng hợp
-------------[3] [G] Tìm cực trị của hàm hợp, hàm trị tuyệt đối
-------------[5] [G] Tìm m để HS đạt cực trị tại 1 điểm cho trước
-------------[6] [G] Tìm m để HS, ĐT HS bậc ba có cực trị thỏa mãn điều kiện
-------------[7] [G] Tìm m để HS, ĐT HS phân thức thỏa mãn điều kiện
-------------[8] [G] Tìm m để HS, ĐT HS các HS khác có cực trị thỏa mãn điều kiện
-------------[B] Chưa phân dạng
----------[3] Giá trị lớn nhất và giá trị nhỏ nhất của HS
-------------[1] Lý thuyết
-------------[2] GTLN, GTNN trên đoạn, khoảng
-------------[3] GTLN, GTNN trên đoạn, khoảng (căn thức)
-------------[C] GTLN, GTNN trên đoạn, khoảng (hàm phân thức)
-------------[D] GTLN, GTNN trên đoạn, khoảng (hàm lượng giác)
-------------[E] GTLN, GTNN trên đoạn, khoảng (hàm log, mũ)
-------------[F] GTLN, GTNN hàm đạo hàm
-------------[5] GTLN, GTNN thông qua BBT và ĐT
-------------[G] GTLN, GTNN thông qua BBT và ĐT hàm đạo hàm
-------------[8] Bài toán ứng dụng, tối ưu, thực tế
-------------[9] Bài toán ứng dụng hình học không gian
-------------[0] Câu hỏi tổng hợp
-------------[4] [G] GTLN, GTNN hàm hợp, hàm trị tuyệt đối
-------------[6] [G] Ứng dụng GTNN, GTLN trong bài toán PT, BPT 
-------------[7] [G] GTLN, GTNN hàm nhiều biến
-------------[A] [G] Giải bài toán có chứa tham số m
-------------[B] Chưa phân dạng
----------[4] Đường tiệm cận
-------------[1] Lý thuyết
-------------[6] Xác định các ĐTC của HS biết công thức
-------------[2] Xác định các ĐTC của HS biết BBT, ĐT
-------------[8] Tâm đối xứng
-------------[9] Trục đối xứng
-------------[4] Bài toán liên quan đến ĐT HS và các đường tiệm cận
-------------[5] Bài toán thực tế, liên môn liên quan đến đường tiệm cận
-------------[0] Câu hỏi tổng hợp
-------------[7] [G] Xác định các ĐTC của HS KHÁC khi biết công thức
-------------[3] [G] Bài toán xác định các đường tiệm cận của HS có chứa tham số
-------------[A] Chưa phân dạng
----------[5] Khảo sát sự biến thiên và vẽ ĐT HS
-------------[1] Đồ thị hàm bậc ba và các câu hỏi mở rộng
-------------[2] Đồ thị hàm nhất biến và các câu hỏi mở rộng
-------------[3] Đồ thị hàm bậc hai trên bậc nhất và các câu hỏi mở rộng
-------------[4] Nhận dạng ĐT, bảng biến thiên KHÁC và các câu hỏi mở rộng
-------------[6] Đồ thị của hàm đạo hàm
-------------[5] Xét dấu hệ số và CH tổng hợp hàm bậc ba
-------------[8] Xét dấu hệ số và CH tổng hợp hàm nhất biến
-------------[9] Xét dấu hệ số và CH tổng hợp hàm bậc hai trên bậc nhất
-------------[7] Điểm đặc biệt của ĐT HS
-------------[0] Câu hỏi tổng hợp hàm số khác
-------------[A] Chưa phân dạng
----------[6] Các câu hỏi liên quan khác đến khảo sát hàm số
-------------[4] PT tiếp tuyến của ĐT HS
-------------[0] Câu hỏi tổng hợp
-------------[1] [G] Các phép biến đổi ĐT và vẽ ĐT
-------------[2] [G] Biện luận số giao điểm dựa vào ĐT, bảng biến thiên
-------------[3] [G] Sự tương giao của hai ĐT (liên quan đến tọa độ giao điểm)
-------------[5] [G] Câu hỏi liên quan đến đồ thị hàm hợp, hàm trị tuyệt đối
-------------[7] [G] Ứng dụng đồ thị để giải PT, BPT và HPT
-------------[6] [G] Giải bài toán có chứa tham số m
-------------[A] Chưa phân dạng
----------[7] Ứng dụng của đạo hàm để giải các bài toán tối ưu khác
-------------[1] Về cực trị hình nón hình trụ
-------------[2] Về cực trị hình cầu, đường tròn ngoại tiếp 
-------------[3] Về cực trị hình chóp
-------------[6] Về cực trị hình lăng trụ
-------------[7] Về cực trị hình hộp
-------------[8] Bài toán tìm đường đi ngắn nhất quanh hình khối
-------------[5] Bài toán cực trị hình học khác
-------------[C] Bài toán qua sông và mở rộng
-------------[4] Bài toán thực tế, liên môn
-------------[B] Bài toán liên hệ trong vật lí
-------------[9] Bài toán thực tế, liên môn liên quan đến kinh tế
-------------[0] Câu hỏi tổng hợp
-------------[A] Chưa phân dạng
----------[0] Chưa phân dạng
-------------[0] Chưa phân dạng
-------[2] Tọa độ của các véc-tơ trong KG
----------[1] Véc-tơ và các phép toán trong KG
-------------[1] Công thức lý thuyết
-------------[8] Phương, hướng, độ dài vectơ
-------------[2] Tổng, hiệu, tích một số với vectơ
-------------[7] Phân tích một vectơ theo các vectơ
-------------[3] Tích vô hướng và ứng dụng
-------------[4] Ứng dụng tính góc và khoảng cách
-------------[5] Toán thực tế áp dụng các phép toán véc-tơ
-------------[6] Cực trị 
-------------[0] Câu hỏi tổng hợp
-------------[A] Chưa phân dạng
----------[2] Tọa độ của véc-tơ và các công thức
-------------[1] Công thức lý thuyết
-------------[2] Tìm tọa độ điểm
-------------[3] Tìm tọa độ véc-tơ
-------------[9] Vectơ cùng phương và ứng dụng
-------------[4] Công thức tọa độ của tích vô hướng và UD
-------------[5] Công thức tọa độ của tích có hướng và UD
-------------[6] UD tìm góc và khoảng cách
-------------[7] Toán thực tế áp dụng các phép toán tọa độ hóa véc-tơ
-------------[8] Cực trị
-------------[0] Câu hỏi tổng hợp
-------------[A] Chưa phân dạng
----------[0] Chưa phân dạng
-------------[0] Chưa phân dạng
-------[3] Các số đặc trưng đo mức độ phân tán cho mẫu số liệu ghép nhóm
----------[1] Khoảng biến thiên, khoảng tứ phân vị của mẫu số liệu ghép nhóm
-------------[1] Công thức lý thuyết
-------------[2] Tìm khoảng biến thiên 
-------------[3] Tìm khoảng tứ phân vị
-------------[5] Nhận xét
-------------[4] Câu hỏi tổng hợp
-------------[A] Chưa phân dạng
----------[2] Phương sai, độ lệch chuẩn của mẫu số liệu ghép nhóm
-------------[1] Công thức lý thuyết
-------------[2] Tìm phương sai, độ lệch chuẩn
-------------[4] Nhận xét
-------------[3] Câu hỏi tổng hợp
-------------[A] Chưa phân dạng
----------[0] Chưa phân dạng
-------------[0] Chưa phân dạng
-------[4] Nguyên hàm, tích phân và ứng dụng
----------[1] Nguyên hàm
-------------[1] Công thức lý thuyết
-------------[7] Các phép toán nguyên hàm
-------------[2] Nguyên hàm cơ bản hàm đa thức, phân thức
-------------[B] Nguyên hàm cơ bản hàm căn thức
-------------[3] Nguyên hàm cơ bản hàm lượng giác
-------------[4] Nguyên hàm cơ bản hàm mũ
-------------[8] Nguyên hàm cơ bản hàm logarit
-------------[9] Nguyên hàm tổng hợp
-------------[5] Nguyên hàm trên các khoảng
-------------[6] Toán thực tế áp dụng Nguyên hàm
-------------[C] Toán thực tế áp dụng nguyên hàm công thức hàm hợp
-------------[0] Câu hỏi tổng hợp
-------------[A] Chưa phân dạng
----------[2] Tích phân
-------------[1] Công thức lý thuyết
-------------[9] Các phép toán tích phân
-------------[2] Tích phân cơ bản hàm đa thức, phân thức
-------------[3] Tích phân cơ bản hàm lượng giác
-------------[4] Tích phân cơ bản hàm mũ
-------------[5] Tích phân cơ bản hàm logarit
-------------[8] Tích phân trên các khoảng
-------------[B] Áp dụng tích phân tính giá trị biểu thức 
-------------[7] Áp dụng tích phân để tính một số tổng khác
-------------[6] Toán thực tế áp dụng Tích phân hàm hợp
-------------[0] Câu hỏi tổng hợp
-------------[A] Chưa phân dạng
----------[3] Ứng dụng thực tế và hình học của tích phân
-------------[7] Câu hỏi lý thuyết
-------------[8] Công thức diện tích, thể tích cho bởi hình ảnh đồ thị hàm số
-------------[1] Diện tích hình phẳng được giới hạn bởi đồ thị
-------------[3] Thể tích giới hạn bởi các ĐT (tròn xoay)
-------------[4] Thể tích tính theo mặt cắt S(x)
-------------[2] Bài toán thực tế sử dụng DIỆN TÍCH hình phẳng
-------------[C] [G] Bài toán thực tế sử dụng DIỆN TÍCH hình phẳng HÀM HỢP
-------------[5] Bài toán thực tế và ứng dụng THỂ TÍCH
-------------[D] [G] Bài toán thực tế và ứng dụng THỂ TÍCH hình phẳng HÀM HỢP
-------------[9] Bài toán thực tế vật lí
-------------[6] Ứng dụng TP vào bài toán liên môn (hóa, sinh, kinh tế)
-------------[B] Bài toán có chứa tham số
-------------[0] Câu hỏi tổng hợp
-------------[A] Chưa phân dạng
----------[4] Nguyên hàm - Tích phân hàm hợp
-------------[1] Công thức lý thuyết
-------------[2] Nguyên hàm f(ax+b) hàm đa thức, phân thức
-------------[3] Nguyên hàm f(ax+b) hàm căn thức
-------------[4] Nguyên hàm f(ax+b) hàm lượng giác
-------------[5] Nguyên hàm f(ax+b) hàm mũ, logarit
-------------[6] Tích phân f(ax+b) hàm đa thức, phân thức
-------------[7] Tích phân f(ax+b) hàm căn thức
-------------[8] Tích phân f(ax+b) hàm lượng giác
-------------[9] Tích phân f(ax+b) hàm mũ, logarit
-------------[D] Tính liên tục của nguyên hàm và tích phân
-------------[B] Nguyên hàm hàm hợp
-------------[C] Tích phân hàm hợp
-------------[0] Câu hỏi tổng hợp
-------------[A] Chưa phân dạng
----------[0] Chưa phân dạng
-------------[0] Chưa phân dạng
-------[5] Một số yếu tố xác suất
----------[1] Xác suất có điều kiện
-------------[1] Công thức lý thuyết
-------------[6] Xác định biến cố đơn giản
-------------[2] Tính XS có điều kiện bằng công thức
-------------[3] Tính XS có điều kiện bằng sơ đồ hình cây
-------------[4] Bài toán tổng hợp
-------------[5] Xác suất cổ điển sử dụng KT12
-------------[0] Câu hỏi tổng hợp
-------------[A] Chưa phân dạng
----------[2] Công thức xác suất toàn phần. Công thức Bayes
-------------[1] Công thức lý thuyết
-------------[2] Tính XS bằng công thức XS toàn phần
-------------[3] Tính XS bằng công thức XS Bayes
-------------[4] Bài toán tổng hợp
-------------[A] Chưa phân dạng
----------[0] Chưa phân dạng
-------------[0] Chưa phân dạng
-------[6] PT MP, ĐT, MC trong KG 
----------[1] PT mặt phẳng trong KG
-------------[1] Câu hỏi lý thuyết
-------------[2] Xác định véc-tơ pháp tuyến, cặp véc-tơ chỉ phương
-------------[3] Viết PT tổng quát MP
-------------[4] Vị trí tương đối giữa hai MP
-------------[5] Khoảng cách từ điểm đến MP
-------------[6] Góc giữa hai MP
-------------[8] Xác định điểm trên mặt thẳng
-------------[9] Cực trị
-------------[7] Toán thực tế áp dụng PT mặt phẳng
-------------[0] Câu hỏi tổng hợp
-------------[A] Chưa phân dạng
----------[2] PT đường thẳng trong KG
-------------[1] Câu hỏi lý thuyết
-------------[2] Xác định véc-tơ chỉ phương, cặp véc-tơ pháp tuyến
-------------[3] Viết PT tổng quát, chính tắc, tham số ĐT
-------------[4] Vị trí tương đối giữa hai ĐT
-------------[5] VTTĐ ĐT và MP
-------------[6] Khoảng cách từ điểm đến ĐT
-------------[7] Góc giữa hai ĐT, ĐT và MP
-------------[9] Xác định điểm trên đường thẳng
-------------[B] Cực trị
-------------[C] Giao điểm ĐT và MP
-------------[8] Câu hỏi thực tế
-------------[0] Câu hỏi tổng hợp
-------------[A] Chưa phân dạng
----------[3] PT mặt cầu trong KG
-------------[1] Câu hỏi lý thuyết
-------------[2] Xác định tâm, bán kính, đường kính MC
-------------[3] Viết PT tổng quát MC
-------------[4] Toán thực tế áp dụng phương trình MC
-------------[5] Vị trí tương đối
-------------[7] Xác định điểm trên mặt cầu
-------------[7] Cực trị
-------------[6] Câu hỏi thực tế
-------------[0] Câu hỏi tổng hợp
-------------[A] Chưa phân dạng
----------[0] Chưa phân dạng
-------------[0] Chưa phân dạng
%==============================================================ID 12 CŨ
----[D] Giải tích
-------[1] Ứng dụng đạo hàm để khảo sát HS
----------[1] Sự đồng biến và nghịch biến của HS
-------------[1] Xét tính đơn điệu của hàm số cho bởi công thức
-------------[2] Xét tính đơn điệu dựa vào bảng biến thiên, đồ thị
-------------[3] Tìm tham số $m$ để hàm số đơn điệu
-------------[4] Ứng dụng tính đơn điệu để chứng minh bất đẳng thức, giải phương trình, bất phương trình, hệ phương trình
-------------[5] Toán thực tế ứng dụng sự đồng biến nghịch biến
-------------[0] Câu hỏi tổng hợp
----------[2] Cực trị của HS
-------------[1] Tìm cực trị của hàm số cho bởi công thức
-------------[2] Tìm cực trị dựa vào BBT, đồ thị
-------------[3] Tìm $m$ để hàm số đạt cực trị tại một điểm $x_0$ cho trước
-------------[4] Tìm $m$ để hàm số, đồ thị hàm số bậc ba có cực trị thỏa mãn điều kiện
-------------[5] Tìm $m$ để hàm số, đồ thị hàm số trùng phương có cực trị thỏa mãn điều kiện
-------------[6] Tìm $m$ để hàm số, đồ thị hàm số các hàm số khác có cực trị thỏa mãn điều kiện
-------------[7] Toán thực tế ứng dụng cực trị của hàm số
-------------[0] Câu hỏi tổng hợp
----------[3] Giá trị lớn nhất và giá trị nhỏ nhất của HS
-------------[1] GTLN, GTNN trên đoạn $[a;b]$
-------------[2] GTLN, GTNN trên khoảng
-------------[3] Sử dụng các đánh giá, bất đẳng thức cổ điển
-------------[4] Ứng dụng GTNN, GTLN trong bài toán phương trình, bất phương trình, hệ phương trình
-------------[5] GTLN, GTNN hàm nhiều biến
-------------[6] Toán thực tế ứng dụng GTLN, GTNN của hàm số
-------------[0] Câu hỏi tổng hợp
----------[4] Đường tiệm cận
-------------[1] Bài toán xác định các đường tiệm cận của hàm số (không chứa tham số) hoặc biết BBT, đồ thị
-------------[2] Bài toán xác định các đường tiệm cận của hàm số có chứa tham số
-------------[3] Bài toán liên quan đến đồ thị hàm số và các đường tiệm cận
-------------[4] Toán thực tế ứng dụng tiệm cận
-------------[0] Câu hỏi tổng hợp
----------[5] Khảo sát sự biến thiên và vẽ ĐT HS
-------------[1] Nhận dạng đồ thị
-------------[2] Các phép biến đổi đồ thị
-------------[3] Biện luận số giao điểm dựa vào đồ thị, bảng biến thiên
-------------[4] Sự tương giao của hai đồ thị (liên quan đến tọa độ giao điểm)
-------------[5] Đồ thị của hàm đạo hàm
-------------[6] Phương trình tiếp tuyến của đồ thị hàm số
-------------[7] Điểm đặc biệt của đồ thị hàm số
-------------[8] Toán thực tế ứng dụng khảo sát hàm số
-------------[0] Câu hỏi tổng hợp
----------[0] Chưa phân dạng
-------------[0] Chưa phân dạng
-------[3] Các số đặc trưng đo mức độ phân tán cho mẫu số liệu ghép nhóm
----------[1] Khoảng biến thiên, khoảng tứ phân vị của mẫu số liệu ghép nhóm
-------------[1] Công thức lý thuyết
-------------[2] Tìm khoảng biến thiên 
-------------[3] Tìm khoảng tứ phân vị
-------------[4] Câu hỏi tổng hợp
----------[2] Phương sai, độ lệch chuẩn của mẫu số liệu ghép nhóm
-------------[1] Công thức lý thuyết
-------------[2] Tìm phương sai, độ lệch chuẩn
-------------[3] Câu hỏi tổng hợp
----------[0] Chưa phân dạng
-------------[0] Chưa phân dạng
-------[4] Nguyên hàm, tích phân và ứng dụng
----------[1] Nguyên hàm
-------------[1] Công thức lý thuyết
-------------[2] NH cơ bản hàm đa thức, phân thức
-------------[3] NH cơ bản hàm lượng giác
-------------[4] NH cơ bản hàm mũ, lượng giác
-------------[5] Phương pháp đổi biến số cơ bản
-------------[6] Toán thực tế áp dụng NH
----------[2] Tích phân
-------------[1] Công thức lý thuyết
-------------[2] TP cơ bản hàm đa thức, phân thức
-------------[3] TP cơ bản hàm lượng giác
-------------[4] TP cơ bản hàm mũ, lũy thừa
-------------[5] Phương pháp đổi biến số cơ bản
-------------[6] Toán thực tế áp dụng TP
----------[3] Ứng dụng thực tế và hình học của tích phân
-------------[1] Diện tích hình phẳng được giới hạn bởi các ĐT
-------------[2] Bài toán thực tế sử dụng diện tích hình phẳng
-------------[3] Thể tích giới hạn bởi các ĐT (tròn xoay)
-------------[4] Thể tích tính theo mặt cắt S(x)
-------------[5] Bài toán thực tế và ứng dụng thể tích
-------------[6] Ứng dụng TP vào bài toán liên môn (lý, hóa, sinh, kinh tế)
----------[0] Chưa phân dạng
-------------[0] Chưa phân dạng
-------[5] Một số yếu tố xác suất
----------[1] Xác suất có điều kiện
-------------[1] Công thức lý thuyết
-------------[2] Tính XS có điều kiện bằng công thức
-------------[3] Tính XS có điều kiện bằng sơ đồ hình cây
-------------[4] Bài toán tổng hợp
----------[2] Công thức xác suất toàn phần. Công thức Bayes
-------------[1] Công thức lý thuyết
-------------[2] Tính XS bằng công thức XS toàn phần
-------------[3] Tính XS bằng công thức XS Bayes
-------------[4] Bài toán tổng hợp
-------[6] Một số yếu tố xác suất
----------[1] Xác suất có điều kiện
-------------[1] Công thức lý thuyết
-------------[2] Tính XS có điều kiện bằng công thức
-------------[3] Tính XS có điều kiện bằng sơ đồ hình cây
-------------[4] Bài toán tổng hợp
----------[2] Công thức xác suất toàn phần. Công thức Bayes
-------------[1] Công thức lý thuyết
-------------[2] Tính XS bằng công thức XS toàn phần
-------------[3] Tính XS bằng công thức XS Bayes
-------------[4] Bài toán tổng hợp
-------[0] Chưa phân dạng
----------[0] Chưa phân dạng
-------------[0] Chưa phân dạng
----[H] Hình học
-------[2] Tọa độ của các véc-tơ trong KG
----------[1] Véc-tơ và các phép toán trong KG
-------------[1] Công thức lý thuyết
-------------[2] Tổng, hiệu, tích một số với véc-tơ
-------------[3] Tích vô hướng và ứng dụng
-------------[4] Toán thực tế áo dụng các phép toán véc-tơ
-------------[5] Góc giữ hai vecto
----------[2] Tọa độ của véc-tơ và các công thức
-------------[1] Công thức lý thuyết
-------------[2] Tìm tọa độ điểm
-------------[3] Tìm tọa độ véc-tơ
-------------[4] Công thức tọa độ của tích vô hướng và UD
-------------[5] Công thức tọa độ của tích có hướng và UD
-------------[6] Toán thực tế áp dụng các phép toán tọa độ hóa véc-tơ
----------[0] Chưa phân dạng
-------------[0] Chưa phân dạng
-------[5] PT MP, ĐT, MC trong KG 
----------[1] PT mặt phẳng trong KG
-------------[1] Câu hỏi lý thuyết
-------------[2] Xác định véc-tơ pháp tuyến, cặp véc-tơ chỉ phương
-------------[3] Viết PT tổng quát MP
-------------[4] Vị trí tương đối giữa hai MP
-------------[5] Khoảng cách từ điểm đến MP
-------------[6] Góc giữa hai MP
-------------[7] Toán thực tế áp dụng PT mặt phẳng
----------[2] PT đường thẳng trong KG
-------------[1] Câu hỏi lý thuyết
-------------[2] Xác định véc-tơ chỉ phương, cặp véc-tơ pháp tuyến
-------------[3] Viết PT tổng quát, chính tắc, tham số ĐT
-------------[4] Vị trí tương đối giữa hai ĐT
-------------[5] Vị trí tương đối giữa ĐT và MP
-------------[6] Khoảng cách từ điểm đến ĐT
-------------[7] Góc giữa hai ĐT, ĐT và MP
-------------[8] Toán thực tế áp dụng PT đường thẳng
----------[3] PT mặt cầu trong KG
-------------[1] Câu hỏi lý thuyết
-------------[2] Xác định tâm, bán kính, đường kính MC
-------------[3] Viết PT tổng quát MC
-------------[4] Toán thực tế áp dụng phương trình MC
----------[0] Chưa phân dạng
-------------[0] Chưa phân dạng
-------[0] Chưa phân dạng
----------[0] Chưa phân dạng
-------------[0] Chưa phân dạng
----[C] Chuyên đề
-------[1] Chuyên đề 1: Ứng dụng toán học giải các bài toán tối ưu
----------[0] Chưa phân dạng
-------------[0] Chưa phân dạng
-------[2] Chuyên đề 2: Ứng dụng toán học trong một số vấn đề liên quan đến tài chính
----------[0] Chưa phân dạng
-------------[0] Chưa phân dạng
-------[3] Chuyên đề 3: Biến ngẫu nhiên rời rạc. Các số đặc trưng của biến ngẫu nhiên rời rạc
----------[0] Chưa phân dạng
-------------[0] Chưa phân dạng
-------[4] [Cũ] Nguyên hàm, tích phân và ứng dụng
----------[1] Nguyên hàm
-------------[1] Định nghĩa, tính chất và nguyên hàm cơ bản
-------------[2] Phương pháp đổi biến số
-------------[3] Phương pháp nguyên hàm từng phần
----------[2] Tích phân
-------------[1] Định nghĩa, tính chất và tích phân cơ bản
-------------[2] Phương pháp đổi biến số
-------------[3] Phương pháp tích phân từng phần
-------------[4] Tích phân của hàm ẩn. Tích phân đặc biệt
-------------[5] Kỹ thuật bình phương
----------[3] Ứng dụng của tích phân
-------------[1] Diện tích hình phẳng được giới hạn bởi các ĐT
-------------[2] Bài toán thực tế sử dụng diện tích hình phẳng
-------------[3] Thể tích giới hạn bởi các ĐT (tròn xoay)
-------------[4] Thể tích tính theo mặt cắt S(x)
-------------[5] Bài toán thực tế và ứng dụng thể tích
-------------[6] Ứng dụng vào tính tổng khai triển nhị thức
-------------[7] Ứng dụng tích phân vào bài toán liên môn (lý, hóa, sinh, kinh tế)
----------[0] Chưa phân dạng
-------------[0] Chưa phân dạng
-------[5] [Cũ] Số phức
----------[1] Khái niệm số phức
-------------[1] Xác định các yếu tố cơ bản của số phức
-------------[2] Biểu diễn hình học cơ bản của số phức
----------[2] Phép cộng, trừ và nhân số phức
-------------[1] Thực hiện phép tính
-------------[2] Xác định các yếu tố cơ bản của số phức qua các phép toán
-------------[3] Bài toán quy về giải phương trình, hệ phương trình nghiệm thực
-------------[4] Bài toán tập hợp điểm
----------[3] Phép chia số phức
-------------[1] Thực hiện phép tính
-------------[2] Xác định các yếu tố cơ bản của số phức qua các phép toán
-------------[3] Bài toán quy về giải phương trình, hệ phương trình nghiệm thực
-------------[4] Bài toán tập hợp điểm
----------[4] PT bậc hai hệ số thực
-------------[1] Giải phương trình. Tính toán biểu thức nghiệm
-------------[2] Định lí Viet và ứng dụng
-------------[3] PT quy về bậc hai
----------[5] Cực trị
-------------[1] Phương pháp hình học
-------------[2] Phương pháp đại số
----------[0] Chưa phân dạng
-------------[0] Chưa phân dạng
-------[6] [Cũ] Mặt nón, mặt trụ, mặt cầu
----------[1] Khái niệm về mặt tròn xoay
-------------[1] Thể tích khối nón, khối trụ
-------------[2] Sxq, Stp, độ dài đường sinh, chiều cao, bán kính đáy, thiết diện
-------------[3] Khối tròn xoay nội tiếp, ngoại tiếp khối đa diện
-------------[4] Bài toán thực tế về khối nón, khối trụ
-------------[5] Bài toán cực trị về khối nón, khối trụ
-------------[6] Câu hỏi lý thuyết
----------[2] Mặt cầu
-------------[1] Bài toán sử dụng định nghĩa, tính chất, vị trí tương đối
-------------[2] Khối cầu ngoại tiếp khối đa diện
-------------[3] Khối cầu nội tiếp khối đa diện
-------------[4] Bài toán thực tế về khối cầu
-------------[5] Bài toán cực trị về khối cầu
-------------[6] Bài toán tổng hợp về khối nón, khối trụ, khối cầu
----------[0] Chưa phân dạng
-------------[1] Chưa phân dạng hình trụ
-------------[2] Chưa phân dạng hình nón
-------------[3] Chưa phân dạng hình cầu
-------[7] [Cũ] Phương pháp tọa độ trong không gian
----------[1] Hệ tọa độ trong không gian
-------------[1] Công thức lý thuyết
-------------[1] Tìm tọa độ điểm, véc-tơ liên quan đến hệ trục Oxyz
-------------[2] Tích vô hướng và ứng dụng
-------------[3] Xác định tâm, bán kính, viết PT mặt cầu đơn giản,...
-------------[4] Các bài toán cực trị
----------[2] PT MP
-------------[1] Tích có hướng và ứng dụng
-------------[2] Xác định VTPT
-------------[3] Viết phương trình MP
-------------[4] Tìm tọa độ điểm liên quan đến MP
-------------[5] Góc
-------------[6] Khoảng cách
-------------[7] Vị trí tương đối giữa hai MP, giữa mặt cầu và MP
-------------[8] Các bài toán cực trị
----------[3] PT ĐT trong không gian
-------------[1] Xác định VTCP
-------------[2] Viết phương trình ĐT
-------------[3] Tìm tọa độ điểm liên quan đến ĐT
-------------[4] Góc
-------------[5] Khoảng cách
-------------[6] Vị trí tương đối giữa hai ĐT, giữa ĐT và MP
-------------[7] Bài toán liên quan giữa ĐT - MP - mặt cầu
-------------[8] Các bài toán cực trị
----------[4] Ứng dụng của phương pháp tọa độ
-------------[1] Bài toán HHKG
-------------[2] Bài toán đại số
-------[8] [Cũ] Ứng dụng đạo hàm để khảo sát HS
----------[1] Sự đồng biến và nghịch biến của HS
-------------[1] Xét tính đơn điệu của HS cho bởi công thức
-------------[2] Xét tính đơn điệu dựa vào bảng biến thiên, ĐT
-------------[3] Tìm tham số m để HS đơn điệu
-------------[4] Ứng dụng tính đơn điệu để chứng minh BĐT, giải PT, BPT, HPT
----------[2] Cực trị của HS
-------------[1] Tìm cực trị của HS cho bởi công thức
-------------[2] Tìm cực trị dựa vào BBT, ĐT
-------------[3] Tìm m để HS đạt cực trị tại 1 điểm cho trước
-------------[4] Tìm m để HS, ĐT HS bậc ba có cực trị thỏa mãn điều kiện
-------------[5] Tìm m để HS, ĐT HS trùng phương có cực trị thỏa mãn điều kiện
-------------[6] Tìm m để HS, ĐT HS các HS khác có cực trị thỏa mãn điều kiện
----------[3] Giá trị lớn nhất và giá trị nhỏ nhất của HS
-------------[1] GTLN, GTNN trên đoạn
-------------[2] GTLN, GTNN trên khoảng
-------------[3] Sử dụng các đánh giá, bất đẳng thức cổ điển
-------------[4] Ứng dụng GTNN, GTLN trong bài toán PT, BPT
-------------[5] GTLN, GTNN hàm nhiều biến
-------------[6] Bài toán ứng dụng, tối ưu, thực tế
----------[4] Đường tiệm cận
-------------[1] Xác định các ĐTC của HS (không chứa tham số) hoặc biết BBT, ĐT
-------------[2] Bài toán xác định các đường tiệm cận của HS có chứa tham số
-------------[3] Bài toán liên quan đến ĐT HS và các đường tiệm cận
----------[5] Khảo sát sự biến thiên và vẽ ĐT HS
-------------[1] Nhận dạng ĐT, bảng biến thiên
-------------[2] Các phép biến đổi ĐT
-------------[3] Biện luận số giao điểm dựa vào ĐT, bảng biến thiên
-------------[4] Sự tương giao của hai ĐT (liên quan đến tọa độ giao điểm)
-------------[5] Đồ thị của hàm đạo hàm
-------------[6] PT tiếp tuyến của ĐT HS
-------------[7] Điểm đặc biệt của ĐT HS
-------------[8] Câu hỏi tổng hợp
----------[0] Chưa phân dạng
-------------[0] Chưa phân dạng
----[G] HSG
-------[1] Một số chủ đề Đại số, giải tích bồi dưỡng HSG lớp 12
----------[1] Bất đẳng thức, cực trị (những bài toán về bất đẳng thức có lời giải sử dụng các kết quả của giải tích chẳng hạn như khảo sát HS, đạo hàm, cực trị, \ldots Các bất đẳng thức liên quan đến hàm mũ, lôga)
-------------[0] Chưa phân dạng
----------[2] PT, BPT, hệ phương trình (những bài toán về phương trình có lời giải sử dụng các kết quả của giải tích chẳng hạn như định lí Lagrange, khảo sát HS)
-------------[0] Chưa phân dạng
----------[3] PT hàm đa thức
-------------[0] Chưa phân dạng
----------[4] Bất đẳng thức tích phân
-------------[0] Chưa phân dạng
----------[0] Chưa phân dạng
-------------[0] Chưa phân dạng
-------[2] Một số chủ đề Hình học bồi dưỡng HSG lớp 12
----------[1] Một số bài toán hình học không gian sử dụng phương pháp thể tích, các hệ thức liên quan đến thể tích
-------------[0] Chưa phân dạng
----------[2] Ứng dụng của tọa độ, vectơ giải các bài toán hình học không gian
-------------[0] Chưa phân dạng
----------[0] Chưa phân dạng
-------------[0] Chưa phân dạng
-------[0] Chưa phân dạng
----------[0] Chưa phân dạng
-------------[0] Chưa phân dạng
----[T] CÂU HỎI TƯ DUY
-------[1] Câu hỏi hình học
----------[0] Chưa phân dạng
-------------[0] Chưa phân dạng
-------[0] Chưa phân dạng
----------[0] Chưa phân dạng
-------------[0] Chưa phân dạng
----[0] Chưa phân dạng
-------[0] Chưa phân dạng
----------[0] Chưa phân dạng
-------------[0] Chưa phân dạng
%==============================================================KẾT THÚC ID 12 
%==============================================================BẮT ĐẦU ID HSG THCS
%==============================================================BẮT ĐẦU ID LỚP 6
-[6] Lớp 6
----[P] 6 - NGÂN HÀNG TẠM
-------[1] SỐ TỰ NHIÊN
----------[1] TẬP HỢP. PHẦN TỬ CỦA TẬP HỢP
-------------[1] Làm quen với tập hợp
-------------[2] Các kí hiệu
-------------[3] Cách cho tập hợp
-------------[4] Bài Toán Thực Tế
-------------[0] Chưa phân dạng
----------[2] TẬP HỢP SỐ TỰ NHIÊN. GHI SỐ TỰ NHIÊN
-------------[1] Tập hợp N và N*
-------------[2] Thứ tự trong tập hợp số tự nhiên
-------------[3] Ghi số tự nhiên
-------------[4] Bài Toán Thực Tế
-------------[0] Chưa phân dạng
----------[3] CÁC PHÉP TÍNH TRONG TẬP HỢP SỐ TỰ NHIÊN
-------------[1] Phép cộng và phép nhân
-------------[2] Tính chất của phép cộng và phép nhân số tự nhiên
-------------[3] Phép trừ và phép chia hết
-------------[4] Bài Toán Thực Tế
-------------[0] Chưa phân dạng
----------[4] LUỸ THỪA VỚI SỐ MŨ TỰ NHIÊN
-------------[1] Luỹ thừa
-------------[2] Nhân hai luỹ thừa cùng cơ số
-------------[3] Chia hai luỹ thừa cùng cơ số
-------------[4] Bài Toán Thực Tế
-------------[0] Chưa phân dạng
----------[5] THỨ TỰ THỰC HIỆN CÁC PHÉP TÍNH
-------------[1] Thứ tự thực hiện các phép tính
-------------[2] Sử dụng máy tính cầm tay
-------------[3] Bài Toán Thực Tế
-------------[0] Chưa phân dạng
----------[6] CHIA HẾT VÀ CHIA CÓ DƯ. TÍNH CHẤT CHIA HẾT CỦA MỘT TỔNG
-------------[1] Chia hết và chia có dư
-------------[2] Tính chất chia hết của một tổng
-------------[3] Bài toán thực tế
-------------[0] Chưa phân dạng
----------[7] DẤU HIỆU CHIA HẾT CHO 2, CHO 5
-------------[1] Dấu hiệu chia hết cho 2
-------------[2] Dấu hiệu chia hết cho 5
-------------[3] Bài Toán Thực Tế
-------------[0] Chưa phân dạng
----------[8] DẤU HIỆU CHIA HẾT CHO 3, CHO 9
-------------[1] Dấu hiệu chia hết cho 9
-------------[2] Dấu hiệu chia hết cho 3
-------------[3] Bài Toán Thực Tế
-------------[0] Chưa phân dạng
----------[9] ƯỚC VÀ BỘI
-------------[1] Ước và bội
-------------[2] Cách tìm ước
-------------[3] Cách tìm bội
-------------[4] Bài Toán Thực Tế
-------------[0] Chưa phân dạng
----------[A] SỐ NGUYÊN TỐ. HỢP SỐ. PHÂN TÍCH MỘT SỐ RA THỪA SỐ NGUYÊN TỐ
-------------[1] Số nguyên tố. Hợp số
-------------[2] Phân tích một số ra thừa số nguyên tố
-------------[3] Bài Toán thực tế
-------------[0] Chưa phân dạng
----------[B] ƯỚC CHUNG. ƯỚC CHUNG LỚN NHẤT
-------------[1] Ước chung
-------------[2] Ước chung lớn nhất
-------------[3] Tìm ước chung lớn nhất bằng cách phân tích các số ra thừa số nguyên tố
-------------[4] Ứng dụng trong rút gọn phân số
-------------[5] Bài Toán Thực Tế
-------------[0] Chưa phân dạng
----------[C] BỘI CHUNG. BỘI CHUNG NHỎ NHẤT
-------------[1] Bội chung
-------------[2] Bội chung lớn nhất
-------------[3] Tìm bội chung lớn nhất bằng cách phân tích các số ra thừa số nguyên tố
-------------[4] Ứng dụng trong quy đồng mẫu các phân số
-------------[5] Bài Toán Thực Tế
-------------[0] Chưa phân dạng
----------[0] Chưa phân dạng
-------------[0] Chưa phân dạng
-------[2] SỐ NGUYÊN
----------[1] Số nguyên âm và tập hợp các số nguyên
-------------[1] Làm quen với số nguyên âm
-------------[2] Tập hợp số nguyên
-------------[3] Biểu diễn số nguyên tố trên trục số
-------------[4] Số đối của một số nguyên
-------------[5] Bài Toán Thực Tế
-------------[0] Chưa phân dạng
----------[2] Thứ tự trong tập hợp số nguyên
-------------[1] So sánh hai số nguyên
-------------[2] Thứ tự trong tập hợp số nguyên
-------------[3] Bài Toán Thực Tế
-------------[0] Chưa phân dạng
----------[3] Phép cộng và phép trừ hai số nguyên
-------------[1] Cộng hai số nguyên cùng dấu
-------------[2] Cộng hai số nguyên khác dấu
-------------[3] Tính chất phép cộng các số nguyên
-------------[4] Phép trừ hai số nguyên
-------------[5] Quy tắc dấu ngoặc
-------------[6] Bài Toán Thực Tế
-------------[0] Chưa phân dạng
----------[4] Phép nhân và phép chia hai số nguyên
-------------[1] Nhân hai số nguyên khác dấu
-------------[2] Nhân hai số nguyên cùng dấu
-------------[3] Tính chất phép nhân các số nguyên
-------------[4] Quan hệ chia hết và phép chia hết trong tập hợp số nguyên (Bài toán tìm x)
-------------[5] Bội và ước của một số nguyên
-------------[6] Bài Toán Thực Tế
-------------[0] Chưa phân dạng
----------[0] Chưa phân dạng
-------------[0] Chưa phân dạng
-------[3] MỘT SỐ YẾU TỐ THỐNG KÊ
----------[1] Thu thập và phân loại dữ liệu
-------------[1] Thu thập dữ liệu
-------------[2] Phân loại dữ liệu
-------------[3] Tính hợp lí của dữ liệu
-------------[4] Bài Toán Thực Tế
-------------[0] Chưa phân dạng
----------[2] Biểu diễn dữ liệu trên bảng
-------------[1] Bảng dữ liệu ban đầu
-------------[2] Bảng thống kê
-------------[3] Bài Toán Thực Tế
-------------[0] Chưa phân dạng
----------[3] Biểu đồ tranh
-------------[1] Ôn tập và bổ sung kiến thức
-------------[2] Đọc biểu đồ tranh
-------------[3] Vẽ biểu đồ tranh
-------------[4] Bài Toán Thực Tế
-------------[0] Chưa phân dạng
----------[4] Biểu đồ cột - biểu đồ cột kép
-------------[1] Ôn tập biểu đồ cột
-------------[2] Đọc biểu đồ cột
-------------[3] Vẽ biểu đồ cột
-------------[4] Giới thiệu biểu đồ kép
-------------[5] Đọc biểu đồ kép
-------------[6] Vẽ biểu đồ kép
-------------[7] Bài toán thực tế
-------------[0] Chưa phân dạng
----------[0] Chưa phân dạng
-------------[0] Chưa phân dạng
-------[4] PHÂN SỐ
----------[1] Phân số với tử số và mẫu số là số nguyên
-------------[1] Mở rộng khái niệm phân số
-------------[2] Phân số bằng nhau
-------------[3] Biểu diễn số nguyên ở dạng phân số
-------------[4] Bài Toán Thực Tế
-------------[0] Chưa phân dạng
----------[2] Tính chất cơ bản của phân số
-------------[1] Tính chất 1 (nhân tử mẫu cùng một số nguyên)
-------------[2] Tính chất 2 (chia tử mẫu cùng một số nguyên)
-------------[3] Bài Toán Thực Tế
-------------[0] Chưa phân dạng
----------[3] So sánh phân số
-------------[1] So sánh hai phân số cùng mẫu
-------------[2] So sánh hai phân số khác mẫu
-------------[3] Áp dụng quy tắc so sánh phân số
-------------[4] Bài Toán Thực Tế
-------------[0] Chưa phân dạng
----------[4] Phép cộng và phép trừ phân số
-------------[1] Phép cộng hai phân số
-------------[2] Một số tính chất của phép cộng phân số
-------------[3] Số đối
-------------[4] Phép trừ hai phân số
-------------[5] Bài Toán Thực Tế
-------------[0] Chưa phân dạng
----------[5] Phép nhân và phép chia phân số
-------------[1] Nhân hai phân số
-------------[2] Một số tính chất của phép nhân phân số
-------------[3] Chia phân số
-------------[4] Bài Toán Thực Tế
-------------[0] Chưa phân dạng
----------[6] Giá trị phân số của một số
-------------[1] Tính giá trị phân số của một số
-------------[2] Tìm một số khi biết giá trị phân số của số đó
-------------[3] Bài Toán tìm x
-------------[4] Bài Toán Thực Tế
-------------[0] Chưa phân dạng
----------[7] Hỗn số
-------------[1] Hỗn số
-------------[2] Đổi hỗn số ra phân số
-------------[3] Bài Toán Thực Tế
-------------[0] Chưa phân dạng
----------[0] Chưa phân dạng
-------------[0] Chưa phân dạng
-------[5] SỐ THẬP PHÂN
----------[1] Số thập phân
-------------[1] Số thập phân âm
-------------[2] Số đối của một số thập phân
-------------[3] So sánh hai số thập phân
-------------[4] Bài Toán Thực Tế
-------------[0] Chưa phân dạng
----------[2] Các phép tính với số thập phân
-------------[1] Cộng, trừ hai số thập phân
-------------[2] Nhân, chia hai số thập phân dương
-------------[3] Nhân, chia hai số thập phân có dấu bất kì
-------------[4] Tính chất của các phép tính với số thập phân
-------------[5] Bài Toán Thực Tế
-------------[0] Chưa phân dạng
----------[3] Làm tròn số thập phân và ước lượng kết quả
-------------[1] Làm tròn số thập phân
-------------[2] Ước lượng kết quả
-------------[3] Bài Toán Thực Tế
-------------[0] Chưa phân dạng
----------[4] Tỉ số và tỉ số phần trăm
-------------[1] Tỉ số của hai đại lượng
-------------[2] Tỉ số phần trăm của hai đại lượng
-------------[3] Bài Toán Thực Tế
-------------[0] Chưa phân dạng
----------[5] Bài toán về tỉ số phần trăm
-------------[1] Tìm giá trị phần trăm của một số
-------------[2] Tìm một số khi biết giá trị phần trăm của số đó
-------------[3] Sử dụng tỉ số phần trăm trong thực tế
-------------[0] Chưa phân dạng
----------[0] Chưa phân dạng
-------------[0] Chưa phân dạng
-------[6] MỘT SỐ YẾU TỐ XÁC SUẤT
----------[1] Phép thử nghiệm. Sự kiện
-------------[1] Phép thử nghiệm
-------------[2] Sự kiện
-------------[3] Bài Toán Thực Tế
-------------[0] Chưa phân dạng
----------[2] Xác suất thực nghiệm
-------------[1] Khả năng xảy ra của một sự kiện
-------------[2] Xác suất thực nghiệm
-------------[3] Bài Toán Thực Tế
-------------[0] Chưa phân dạng
----------[0] Chưa phân dạng
-------------[0] Chưa phân dạng
-------[7] CÁC HÌNH PHẲNG TRONG THỰC TIỄN
----------[1] Hình Vuông - Tam Giác Đều - Lục Giác Đều
-------------[1] Hình vuông
-------------[2] Tam giác đều
-------------[3] Lục giác đều
-------------[4] Bài Toán Thực Tế
-------------[0] Chưa phân dạng
----------[2] Hình Chữ Nhật - Hình Thoi - Hình Bình Hành - Hình Thang Cân
-------------[1] Hình chữ nhật
-------------[2] Hình thoi
-------------[3] Hình bình hành
-------------[4] Hình thang cân
-------------[5] Bài Toán Thực Tế
-------------[0] Chưa phân dạng
----------[3] Chu Vi và Diện Tích của một số hình trong thực tiễn
-------------[1] Chu vi và diện tích hình chữ nhật, hình vuông, hình tam giác, hình thang
-------------[2] Tính chu vi và diện tích hình bình hành, hình thoi
-------------[3] Tính chu vi và diện tích một số hình trong thực tiễn
-------------[0] Chưa phân dạng
----------[0] Chưa phân dạng
-------------[0] Chưa phân dạng
-------[8] TÍNH ĐỐI XỨNG CỦA HÌNH PHẲNG TRONG THẾ GIỚI TỰ NHIÊN
----------[1] Hình có trục đối xứng
-------------[1] Hình có trục đối xứng. Trục đối xứng
-------------[2] Nhận biết hình phẳng trong tự nhiên có trục đối xứng
-------------[3] Bài Toán Thực Tế
-------------[0] Chưa phân dạng
----------[2] Hình có tâm đối xứng
-------------[1] Hình có tâm đối xứng. Tâm đối xứng
-------------[2] Nhận biết hình phẳng trong tự nhiên có tâm đối xứng
-------------[3] Bài Toán Thực Tế
-------------[0] Chưa phân dạng
----------[3] Vai trò của tính đối xứng trong thế giới tự nhiên
-------------[1] Vẻ đẹp của thế giới tự nhiên biểu hiện qua tính đối xứng
-------------[2] Tính đối xứng trong khoa học kĩ thuật và đời sống
-------------[0] Chưa phân dạng
----------[0] Chưa phân dạng
-------------[0] Chưa phân dạng
-------[9] CÁC HÌNH HÌNH HỌC CƠ BẢN
----------[1] Điểm và đường thẳng
-------------[1] Điểm
-------------[2] Đường thẳng
-------------[3] Vẽ đường thẳng
-------------[4] Điểm thuộc đường thẳng. Điểm không thuộc đường thẳng
-------------[5] Bài Toán Thực Tế
-------------[0] Chưa phân dạng
----------[2] Ba điểm thẳng hàng. Ba điểm không thẳng hàng
-------------[1] Ba điểm thẳng hàng
-------------[2] Quan hệ giữa ba điểm thẳng hàng
-------------[3] Bài Toán Thực Tế
-------------[0] Chưa phân dạng
----------[3] Hai đường thẳng cắt nhau, song song. Tia
-------------[1] Hai đường thẳng cắt nhau, song song
-------------[2] Tia
-------------[3] Bài Toán Thực Tế
-------------[0] Chưa phân dạng
----------[4] Đoạn thẳng. Độ dài đoạn thẳng
-------------[1] Đoạn thẳng
-------------[2] Độ dài đoạn thẳng
-------------[3] So sánh hai đoạn thẳng
-------------[4] Một số dụng cụ đo độ dài
-------------[5] Bài Toán Thực Tế
-------------[0] Chưa phân dạng
----------[5] Trung điểm của đoạn thẳng
-------------[1] Trung điểm của đoạn thẳng
-------------[2] Cách vẽ trung điểm của đoạn thẳng
-------------[3] Bài Toán Thực Tế
-------------[0] Chưa phân dạng
----------[6] Góc
-------------[1] Góc
-------------[2] Cách vẽ góc
-------------[3] Góc bẹt
-------------[4] Điểm trong góc
-------------[5] Bài Toán Thực Tế
-------------[0] Chưa phân dạng
----------[7] Số đo góc. Các góc đặc biệt
-------------[1] Thước đo góc
-------------[2] Cách đo góc. Số đo góc
-------------[3] So sánh hai góc
-------------[4] Các góc đặc biệt
-------------[5] Bài Toán Thực Tế
-------------[0] Chưa phân dạng
----------[0] Chưa phân dạng
-------------[0] Chưa phân dạng
-------[0] Chưa phân dạng
----------[0] Chưa phân dạng
-------------[0] Chưa phân dạng
%==============================================================
----[D] Đại số
-------[1] SỐ TỰ NHIÊN
----------[1] TẬP HỢP. PHẦN TỬ CỦA TẬP HỢP
-------------[1] Làm quen với tập hợp
-------------[2] Các kí hiệu
-------------[3] Cách cho tập hợp
-------------[4] Bài Toán Thực Tế
-------------[0] Chưa phân dạng
----------[2] TẬP HỢP SỐ TỰ NHIÊN. GHI SỐ TỰ NHIÊN
-------------[1] Tập hợp N và N*
-------------[2] Thứ tự trong tập hợp số tự nhiên
-------------[3] Ghi số tự nhiên
-------------[4] Bài Toán Thực Tế
-------------[0] Chưa phân dạng
----------[3] CÁC PHÉP TÍNH TRONG TẬP HỢP SỐ TỰ NHIÊN
-------------[1] Phép cộng và phép nhân
-------------[2] Tính chất của phép cộng và phép nhân số tự nhiên
-------------[3] Phép trừ và phép chia hết
-------------[4] Bài Toán Thực Tế
-------------[0] Chưa phân dạng
----------[4] LUỸ THỪA VỚI SỐ MŨ TỰ NHIÊN
-------------[1] Luỹ thừa
-------------[2] Nhân hai luỹ thừa cùng cơ số
-------------[3] Chia hai luỹ thừa cùng cơ số
-------------[4] Bài Toán Thực Tế
-------------[0] Chưa phân dạng
----------[5] THỨ TỰ THỰC HIỆN CÁC PHÉP TÍNH
-------------[1] Thứ tự thực hiện các phép tính
-------------[2] Sử dụng máy tính cầm tay
-------------[3] Bài Toán Thực Tế
-------------[0] Chưa phân dạng
----------[6] CHIA HẾT VÀ CHIA CÓ DƯ. TÍNH CHẤT CHIA HẾT CỦA MỘT TỔNG
-------------[1] Chia hết và chia có dư
-------------[2] Tính chất chia hết của một tổng
-------------[3] Bài toán thực tế
-------------[0] Chưa phân dạng
----------[7] DẤU HIỆU CHIA HẾT CHO 2, CHO 5
-------------[1] Dấu hiệu chia hết cho 2
-------------[2] Dấu hiệu chia hết cho 5
-------------[3] Bài Toán Thực Tế
-------------[0] Chưa phân dạng
----------[8] DẤU HIỆU CHIA HẾT CHO 3, CHO 9
-------------[1] Dấu hiệu chia hết cho 9
-------------[2] Dấu hiệu chia hết cho 3
-------------[3] Bài Toán Thực Tế
-------------[0] Chưa phân dạng
----------[9] ƯỚC VÀ BỘI
-------------[1] Ước và bội
-------------[2] Cách tìm ước
-------------[3] Cách tìm bội
-------------[4] Bài Toán Thực Tế
-------------[0] Chưa phân dạng
----------[A] SỐ NGUYÊN TỐ. HỢP SỐ. PHÂN TÍCH MỘT SỐ RA THỪA SỐ NGUYÊN TỐ
-------------[1] Số nguyên tố. Hợp số
-------------[2] Phân tích một số ra thừa số nguyên tố
-------------[3] Bài Toán thực tế
-------------[0] Chưa phân dạng
----------[B] ƯỚC CHUNG. ƯỚC CHUNG LỚN NHẤT
-------------[1] Ước chung
-------------[2] Ước chung lớn nhất
-------------[3] Tìm ước chung lớn nhất bằng cách phân tích các số ra thừa số nguyên tố
-------------[4] Ứng dụng trong rút gọn phân số
-------------[5] Bài Toán Thực Tế
-------------[0] Chưa phân dạng
----------[C] BỘI CHUNG. BỘI CHUNG NHỎ NHẤT
-------------[1] Bội chung
-------------[2] Bội chung lớn nhất
-------------[3] Tìm bội chung lớn nhất bằng cách phân tích các số ra thừa số nguyên tố
-------------[4] Ứng dụng trong quy đồng mẫu các phân số
-------------[5] Bài Toán Thực Tế
-------------[0] Chưa phân dạng
----------[0] Chưa phân dạng
-------------[0] Chưa phân dạng
-------[2] SỐ NGUYÊN
----------[1] Số nguyên âm và tập hợp các số nguyên
-------------[1] Làm quen với số nguyên âm
-------------[2] Tập hợp số nguyên
-------------[3] Biểu diễn số nguyên tố trên trục số
-------------[4] Số đối của một số nguyên
-------------[5] Bài Toán Thực Tế
-------------[0] Chưa phân dạng
----------[2] Thứ tự trong tập hợp số nguyên
-------------[1] So sánh hai số nguyên
-------------[2] Thứ tự trong tập hợp số nguyên
-------------[3] Bài Toán Thực Tế
-------------[0] Chưa phân dạng
----------[3] Phép cộng và phép trừ hai số nguyên
-------------[1] Cộng hai số nguyên cùng dấu
-------------[2] Cộng hai số nguyên khác dấu
-------------[3] Tính chất phép cộng các số nguyên
-------------[4] Phép trừ hai số nguyên
-------------[5] Quy tắc dấu ngoặc
-------------[6] Bài Toán Thực Tế
-------------[0] Chưa phân dạng
----------[4] Phép nhân và phép chia hai số nguyên
-------------[1] Nhân hai số nguyên khác dấu
-------------[2] Nhân hai số nguyên cùng dấu
-------------[3] Tính chất phép nhân các số nguyên
-------------[4] Quan hệ chia hết và phép chia hết trong tập hợp số nguyên (Bài toán tìm x)
-------------[5] Bội và ước của một số nguyên
-------------[6] Bài Toán Thực Tế
-------------[0] Chưa phân dạng
----------[0] Chưa phân dạng
-------------[0] Chưa phân dạng
-------[3] MỘT SỐ YẾU TỐ THỐNG KÊ
----------[1] Thu thập và phân loại dữ liệu
-------------[1] Thu thập dữ liệu
-------------[2] Phân loại dữ liệu
-------------[3] Tính hợp lí của dữ liệu
-------------[4] Bài Toán Thực Tế
-------------[0] Chưa phân dạng
----------[2] Biểu diễn dữ liệu trên bảng
-------------[1] Bảng dữ liệu ban đầu
-------------[2] Bảng thống kê
-------------[3] Bài Toán Thực Tế
-------------[0] Chưa phân dạng
----------[3] Biểu đồ tranh
-------------[1] Ôn tập và bổ sung kiến thức
-------------[2] Đọc biểu đồ tranh
-------------[3] Vẽ biểu đồ tranh
-------------[4] Bài Toán Thực Tế
-------------[0] Chưa phân dạng
----------[4] Biểu đồ cột - biểu đồ cột kép
-------------[1] Ôn tập biểu đồ cột
-------------[2] Đọc biểu đồ cột
-------------[3] Vẽ biểu đồ cột
-------------[4] Giới thiệu biểu đồ kép
-------------[5] Đọc biểu đồ kép
-------------[6] Vẽ biểu đồ kép
-------------[7] Bài toán thực tế
-------------[0] Chưa phân dạng
----------[0] Chưa phân dạng
-------------[0] Chưa phân dạng
-------[4] PHÂN SỐ
----------[1] Phân số với tử số và mẫu số là số nguyên
-------------[1] Mở rộng khái niệm phân số
-------------[2] Phân số bằng nhau
-------------[3] Biểu diễn số nguyên ở dạng phân số
-------------[4] Bài Toán Thực Tế
-------------[0] Chưa phân dạng
----------[2] Tính chất cơ bản của phân số
-------------[1] Tính chất 1 (nhân tử mẫu cùng một số nguyên)
-------------[2] Tính chất 2 (chia tử mẫu cùng một số nguyên)
-------------[3] Bài Toán Thực Tế
-------------[0] Chưa phân dạng
----------[3] So sánh phân số
-------------[1] So sánh hai phân số cùng mẫu
-------------[2] So sánh hai phân số khác mẫu
-------------[3] Áp dụng quy tắc so sánh phân số
-------------[4] Bài Toán Thực Tế
-------------[0] Chưa phân dạng
----------[4] Phép cộng và phép trừ phân số
-------------[1] Phép cộng hai phân số
-------------[2] Một số tính chất của phép cộng phân số
-------------[3] Số đối
-------------[4] Phép trừ hai phân số
-------------[5] Bài Toán Thực Tế
-------------[0] Chưa phân dạng
----------[5] Phép nhân và phép chia phân số
-------------[1] Nhân hai phân số
-------------[2] Một số tính chất của phép nhân phân số
-------------[3] Chia phân số
-------------[4] Bài Toán Thực Tế
-------------[0] Chưa phân dạng
----------[6] Giá trị phân số của một số
-------------[1] Tính giá trị phân số của một số
-------------[2] Tìm một số khi biết giá trị phân số của số đó
-------------[3] Bài Toán tìm x
-------------[4] Bài Toán Thực Tế
-------------[0] Chưa phân dạng
----------[7] Hỗn số
-------------[1] Hỗn số
-------------[2] Đổi hỗn số ra phân số
-------------[3] Bài Toán Thực Tế
-------------[0] Chưa phân dạng
----------[0] Chưa phân dạng
-------------[0] Chưa phân dạng
-------[5] SỐ THẬP PHÂN
----------[1] Số thập phân
-------------[1] Số thập phân âm
-------------[2] Số đối của một số thập phân
-------------[3] So sánh hai số thập phân
-------------[4] Bài Toán Thực Tế
-------------[0] Chưa phân dạng
----------[2] Các phép tính với số thập phân
-------------[1] Cộng, trừ hai số thập phân
-------------[2] Nhân, chia hai số thập phân dương
-------------[3] Nhân, chia hai số thập phân có dấu bất kì
-------------[4] Tính chất của các phép tính với số thập phân
-------------[5] Bài Toán Thực Tế
-------------[0] Chưa phân dạng
----------[3] Làm tròn số thập phân và ước lượng kết quả
-------------[1] Làm tròn số thập phân
-------------[2] Ước lượng kết quả
-------------[3] Bài Toán Thực Tế
-------------[0] Chưa phân dạng
----------[4] Tỉ số và tỉ số phần trăm
-------------[1] Tỉ số của hai đại lượng
-------------[2] Tỉ số phần trăm của hai đại lượng
-------------[3] Bài Toán Thực Tế
-------------[0] Chưa phân dạng
----------[5] Bài toán về tỉ số phần trăm
-------------[1] Tìm giá trị phần trăm của một số
-------------[2] Tìm một số khi biết giá trị phần trăm của số đó
-------------[3] Sử dụng tỉ số phần trăm trong thực tế
-------------[0] Chưa phân dạng
----------[0] Chưa phân dạng
-------------[0] Chưa phân dạng
-------[6] MỘT SỐ YẾU TỐ XÁC SUẤT
----------[1] Phép thử nghiệm. Sự kiện
-------------[1] Phép thử nghiệm
-------------[2] Sự kiện
-------------[3] Bài Toán Thực Tế
-------------[0] Chưa phân dạng
----------[2] Xác suất thực nghiệm
-------------[1] Khả năng xảy ra của một sự kiện
-------------[2] Xác suất thực nghiệm
-------------[3] Bài Toán Thực Tế
-------------[0] Chưa phân dạng
----------[0] Chưa phân dạng
-------------[0] Chưa phân dạng
-------[0] Chưa phân dạng
----------[0] Chưa phân dạng
-------------[0] Chưa phân dạng
-------[0] Một số chủ đề Đại số, Số học bồi dưỡng HSG lớp 6
----------[1] Số tự nhiên
----------[2] Dãy số tự nhiên, phân số có quy luật
----------[3] Dấu hiệu chia hết. Chữ số tận cùng
----------[4] Ước, bội, ƯCLN, BCNN
----------[5] Số nguyên tố, hợp số
----------[6] Số chính phương
----------[7] Đồng dư
----------[8] Bất đẳng thức, giá trị lớn nhất, giá trị nhỏ nhất
----------[9] Suy luận toán học; các bài toán về trò chơi; đại số tổ hơp
----[H] Hình học
-------[1] CÁC HÌNH PHẲNG TRONG THỰC TIỄN
----------[1] Hình Vuông - Tam Giác Đều - Lục Giác Đều
-------------[1] Hình vuông
-------------[2] Tam giác đều
-------------[3] Lục giác đều
-------------[4] Bài Toán Thực Tế
-------------[0] Chưa phân dạng
----------[2] Hình Chữ Nhật - Hình Thoi - Hình Bình Hành - Hình Thang Cân
-------------[1] Hình chữ nhật
-------------[2] Hình thoi
-------------[3] Hình bình hành
-------------[4] Hình thang cân
-------------[5] Bài Toán Thực Tế
-------------[0] Chưa phân dạng
----------[3] Chu Vi và Diện Tích của một số hình trong thực tiễn
-------------[1] Chu vi và diện tích hình chữ nhật, hình vuông, hình tam giác, hình thang
-------------[2] Tính chu vi và diện tích hình bình hành, hình thoi
-------------[3] Tính chu vi và diện tích một số hình trong thực tiễn
-------------[0] Chưa phân dạng
----------[0] Chưa phân dạng
-------------[0] Chưa phân dạng
-------[2] TÍNH ĐỐI XỨNG CỦA HÌNH PHẲNG TRONG THẾ GIỚI TỰ NHIÊN
----------[1] Hình có trục đối xứng
-------------[1] Hình có trục đối xứng. Trục đối xứng
-------------[2] Nhận biết hình phẳng trong tự nhiên có trục đối xứng
-------------[3] Bài Toán Thực Tế
-------------[0] Chưa phân dạng
----------[2] Hình có tâm đối xứng
-------------[1] Hình có tâm đối xứng. Tâm đối xứng
-------------[2] Nhận biết hình phẳng trong tự nhiên có tâm đối xứng
-------------[3] Bài Toán Thực Tế
-------------[0] Chưa phân dạng
----------[3] Vai trò của tính đối xứng trong thế giới tự nhiên
-------------[1] Vẻ đẹp của thế giới tự nhiên biểu hiện qua tính đối xứng
-------------[2] Tính đối xứng trong khoa học kĩ thuật và đời sống
-------------[0] Chưa phân dạng
----------[0] Chưa phân dạng
-------------[0] Chưa phân dạng
-------[3] CÁC HÌNH HÌNH HỌC CƠ BẢN
----------[1] Điểm và đường thẳng
-------------[1] Điểm
-------------[2] Đường thẳng
-------------[3] Vẽ đường thẳng
-------------[4] Điểm thuộc đường thẳng. Điểm không thuộc đường thẳng
-------------[5] Bài Toán Thực Tế
-------------[0] Chưa phân dạng
----------[2] Ba điểm thẳng hàng. Ba điểm không thẳng hàng
-------------[1] Ba điểm thẳng hàng
-------------[2] Quan hệ giữa ba điểm thẳng hàng
-------------[3] Bài Toán Thực Tế
-------------[0] Chưa phân dạng
----------[3] Hai đường thẳng cắt nhau, song song. Tia
-------------[1] Hai đường thẳng cắt nhau, song song
-------------[2] Tia
-------------[3] Bài Toán Thực Tế
-------------[0] Chưa phân dạng
----------[4] Đoạn thẳng. Độ dài đoạn thẳng
-------------[1] Đoạn thẳng
-------------[2] Độ dài đoạn thẳng
-------------[3] So sánh hai đoạn thẳng
-------------[4] Một số dụng cụ đo độ dài
-------------[5] Bài Toán Thực Tế
-------------[0] Chưa phân dạng
----------[5] Trung điểm của đoạn thẳng
-------------[1] Trung điểm của đoạn thẳng
-------------[2] Cách vẽ trung điểm của đoạn thẳng
-------------[3] Bài Toán Thực Tế
-------------[0] Chưa phân dạng
----------[6] Góc
-------------[1] Góc
-------------[2] Cách vẽ góc
-------------[3] Góc bẹt
-------------[4] Điểm trong góc
-------------[5] Bài Toán Thực Tế
-------------[0] Chưa phân dạng
----------[7] Số đo góc. Các góc đặc biệt
-------------[1] Thước đo góc
-------------[2] Cách đo góc. Số đo góc
-------------[3] So sánh hai góc
-------------[4] Các góc đặc biệt
-------------[5] Bài Toán Thực Tế
-------------[0] Chưa phân dạng
----------[0] Chưa phân dạng
-------------[0] Chưa phân dạng
-------[0] Một số chủ đề Hình học bồi dưỡng HSG lớp 6
----------[1] Đoạn thẳng
----------[2] Góc
----------[3] Tia
----------[4] Tam giác
----------[5] Đường tròn
----------[6] Hình học tổ hợp
-------[0] Chưa phân dạng
----------[0] Chưa phân dạng
-------------[0] Chưa phân dạng
%==============================================================
----[G] 6-HSG
-------[1] Một số chủ đề Đại số, Số học bồi dưỡng HSG lớp 6
----------[1] Số tự nhiên
-------------[0] Chưa phân dạng
----------[2] Dãy số tự nhiên, phân số có quy luật
-------------[0] Chưa phân dạng
----------[3] Dấu hiệu chia hết. Chữ số tận cùng
-------------[0] Chưa phân dạng
----------[4] Ước, bội, ƯCLN, BCNN
-------------[0] Chưa phân dạng
----------[5] Số nguyên tố, hợp số
-------------[0] Chưa phân dạng
----------[6] Số chính phương
-------------[0] Chưa phân dạng
----------[7] Đồng dư
-------------[0] Chưa phân dạng
----------[8] Bất đẳng thức, giá trị lớn nhất, giá trị nhỏ nhất
-------------[0] Chưa phân dạng
----------[9] Suy luận toán học; các bài toán về trò chơi; đại số tổ hơp
-------------[0] Chưa phân dạng
-------[2] Một số chủ đề Hình học bồi dưỡng HSG lớp 6
----------[1] Đoạn thẳng
-------------[0] Chưa phân dạng
----------[2] Góc
-------------[0] Chưa phân dạng
----------[3] Tia
-------------[0] Chưa phân dạng
----------[4] Tam giác
-------------[0] Chưa phân dạng
----------[5] Đường tròn
-------------[0] Chưa phân dạng
----------[6] Hình học tổ hợp
-------------[0] Chưa phân dạng
----[0] Chưa phân dạng
-------[0] Chưa phân dạng
----------[0] Chưa phân dạng
-------------[0] Chưa phân dạng
%==============================================================BẮT ĐẦU ID LỚP 7
-[7] Lớp 7
----[P] 7 - NGÂN HÀNG TẠM
-------[1] SỐ HỮU TỈ
----------[1] TẬP HỢP CÁC SỐ HỮU TỈ 
-------------[1] Số Hữu Tỉ
-------------[2] Thứ tự trong tập hợp các Số Hữu T
-------------[3] Biểu diễn Số Hữu Tỉ trên trục
-------------[4] Số đối của Số Hữu Tỉ
-------------[5] Bài Toán Thực Tế
-------------[0] Chưa phân dạng
----------[2] CÁC PHÉP TÍNH VỚI SỐ HỮU TỈ
-------------[1] Cộng và trừ 2 Số Hữu Tỉ
-------------[2] Tính chất của phép cộng 2 Số Hữu Tỉ
-------------[3] Phân 2 Số Hữu Tỉ
-------------[4] Tính chất của phép nhân 2 Số Hữu Tỉ
-------------[5] Chia 2 Số Hữu Tỉ
-------------[6] Bài Toán Thực Tế
-------------[0] Chưa phân dạng
----------[3] LUỸ THỪA MỘT SỐ HỮU TỈ
-------------[1] Luỹ thừa với số mũ tự nhiên
-------------[2] Tích và thương 2 luỹ thừa cùng cơ số
-------------[3] Luỹ thừa của luỹ thừa
-------------[4] Bài toán tìm x (phép nhân và phép chia)
-------------[5] Bài Toán Thực Tế
-------------[0] Chưa phân dạng
----------[4] QUY TẮC DẤU NGOẶC VÀ CHUYỂN VẾ
-------------[1] Quy tắc dấu ngoặc
-------------[2] Quy tắc chuyển vế
-------------[3] Thứ tự phép tính
-------------[4] Bài toán tìm x (phép cộng và trừ, kết hợp nhân và chia (nếu có))
-------------[5] Bài Toán Thực Tế
-------------[0] Chưa phân dạng
----------[0] Chưa phân dạng
-------------[0] Chưa phân dạng
-------[2] SỐ THỰC
----------[1] SỐ VÔ TỈ. CĂN BẬC HAI SỐ HỌC
-------------[1] Biểu diễn thập phân của số hữu tỉ
-------------[2] Số vô tỷ
-------------[3] Căn bậc hai số học
-------------[4] Tính căn bậc hai số học bằng máy tính
-------------[5] Bài Toán Thực Tế
-------------[0] Chưa phân dạng
----------[2] SỐ THỰC. GIÁ TRỊ TUYỆT ĐỐI CỦA MỘT SỐ THỰC
-------------[1] Số thực và tập hợp các số thực
-------------[2] Thứ tự trong tập hợp các số thực
-------------[3] Trục số thực
-------------[4] Số đối của một số thực
-------------[5] Giá trị tuyệt đối của một số thực
-------------[6] Bài toán thực tế
-------------[0] Chưa phân dạng
----------[3] LÀM TRÒN SỐ, ƯỚC LƯỢNG KẾT QUẢ
-------------[1] Làm tròn số
-------------[2] Làm tròn số căn cứ vào độ chính xác cho trước
-------------[3] Ước lượng các phép tính
-------------[4] Bài Toán Thực Tế
-------------[0] Chưa phân dạng
----------[0] Chưa phân dạng
-------------[0] Chưa phân dạng
-------[3] MỘT SỐ YẾU TỐ THỐNG KÊ
----------[1] THU THẬP VÀ PHÂN LOẠI DỮ LIỆU
-------------[1] Thu thập dữ liệu
-------------[2] Phân loại dữ liệu theo các tiêu chí
-------------[3] Tính hợp lí của dữ liệu
-------------[0] Chưa phân dạng
----------[2] BIỂU ĐỒ HÌNH QUẠT
-------------[1] Ôn tập về biểu đồ hình quạt tròn
-------------[2] Biểu diễn dữ liệu vào biểu đồ hình quạt tròn
-------------[3] Phân tích dữ liệu trên biểu đồ hình quạt tròn
-------------[0] Chưa phân dạng
----------[3] BIỂU ĐỒ ĐOẠN THẲNG
-------------[1] Giới thiệu biểu đồ đoạn thẳng
-------------[2] Vẽ biểu đồ đoạn thẳng
-------------[3] Đọc và phân tích dữ liệu từ biểu đồ đoạn thẳng
-------------[0] Chưa phân dạng
----------[0] Chưa phân dạng
-------------[0] Chưa phân dạng
-------[4] CÁC ĐẠI LƯỢNG TỈ LỆ
----------[1] Tỉ Lệ Thức - Dãy tỉ số bằng nhau
-------------[1] Tỉ lệ thức
-------------[2] Dãy tỉ số bằng nhau
-------------[3] Tìm x, y, z bằng dãy tỉ số bằng nhau
-------------[4] Bài Toán Thực Tế
-------------[0] Chưa phân dạng
----------[2] Đại Lượng Tỉ Lệ Thuận
-------------[1] Đại Lượng Tỉ Lệ Thuận
-------------[2] Tính chất của Đại Lượng Tỉ Lệ Thuận
-------------[3] Các bài toán về Đại Lượng Tỉ Lệ Thuận
-------------[4] Bài Toán Thực Tế
-------------[0] Chưa phân dạng
----------[3] Đại lượng tỉ lệ nghịch
-------------[1] Đại lượng tỉ lệ nghịch
-------------[2] Tính chất của các đại lượng tỉ lệ nghịch
-------------[3] Các bài toán về đại lượng tỉ lệ nghịch
-------------[4] Bài Toán Thực Tế
-------------[0] Chưa phân dạng
----------[0] Chưa phân dạng
-------------[0] Chưa phân dạng
-------[5] BIỂU THỨC ĐẠI SỐ
----------[1] Biểu thức số, Biểu thức đại số
-------------[1] Biểu thức số
-------------[2] Biểu thức đại số
-------------[3] Giá trị của biểu thức số
-------------[4] Bài Toán Thực Tế
-------------[0] Chưa phân dạng
----------[2] Đa thức một biến
-------------[1] Đa thức một biến
-------------[2] Cách biểu diễn đa thức một biến
-------------[3] Giá trị của đa thức một biến
-------------[4] Nghiệm đa thức một biến
-------------[0] Chưa phân dạng
----------[3] Phép cộng và phép trừ đa thức một biến
-------------[1] Phép cộng hai đa thức một biến
-------------[2] Phép trừ hai đa thức một biến
-------------[3] Tính chất phép cộng đa thức một biến
-------------[4] Bài Toán Thực Tế
-------------[0] Chưa phân dạng
----------[4] Phép nhân và phép chia đa thức một biến
-------------[1] Phép nhân đa thức một biến
-------------[2] Phép chia đa thức một biến
-------------[3] Tính chất phép nhân đa thức một biến
-------------[4] Bài Toán Thực Tế
-------------[0] Chưa phân dạng
----------[0] Chưa phân dạng
-------------[0] Chưa phân dạng
-------[6] CÁC HÌNH KHỐI TRONG THỰC TIỄN
----------[1] Hình Hộp Chữ Nhật, Hình Lập Phương
-------------[1] Hình hộp chữ nhật
-------------[2] Hình lập phương
-------------[3] Bài Toán Thực Tế
-------------[0] Chưa phân dạng
----------[2] DTXQ VÀ THỂ TÍCH HHCN VÀ HÌNH LẬP PHƯƠNG
-------------[1] Nhắc lại công thức tính diện tích xung quanh và thể tích
-------------[2] Một số bài toán thực tế
-------------[0] Chưa phân dạng
----------[3] HÌNH LĂNG TRỤ ĐỨNG TAM GIÁC, HÌNH LĂNG TRỤ ĐỨNG TỨ GIÁC
-------------[1] Hình Lăng Trụ Đứng Tam giác, Hình Lăng Trụ Đứng Tứ Giác
-------------[2] Tạo lập Hình Lăng Trụ Đứng Tam giác, Hình Lăng Trụ Đứng Tứ Giác
-------------[3] Bài Toán Thực Tế
-------------[0] Chưa phân dạng
----------[0] Chưa phân dạng
-------------[0] Chưa phân dạng
-------[7] HÌNH HỌC PHẲNG GÓC VÀ ĐƯỜNG THẰNG SONG SONG
----------[1] TÌM CÁC GÓC Ở VỊ TRÍ ĐẶC BIỆT
-------------[1] Hai góc kề bù
-------------[2] Hai góc đối đỉnh
-------------[3] Tính chất hai góc đối đỉnh
-------------[4] Bài Toán Thực Tế
-------------[0] Chưa phân dạng
----------[2] TIA PHÂN GIÁC
-------------[1] Tia Phân Giác Của Một Góc
-------------[2] Cách vẽ tia phân giác
-------------[3] Bài Toán Thực Tế
-------------[0] Chưa phân dạng
----------[3] Hai Đường Thẳng Song Song
-------------[1] Dấu hiệu nhận biết hai đường thẳng song song
-------------[2] Tiên đề Euclid về đường thẳng song song
-------------[3] Tính chất hai đường thẳng song song 
-------------[4] Bài Toán Thực Tế
-------------[0] Chưa phân dạng
----------[4] Định Lí Và Chứng Minh Một Định Lí
-------------[1] Định Lí là gì?
-------------[2] Chứng Minh Định Lí
-------------[3] DTXQ và Thể Tích của một số hình khối trong thực tiễn
-------------[0] Chưa phân dạng
----------[0] Chưa phân dạng
-------------[0] Chưa phân dạng
-------[8] HÌNH HỌC PHẲNG TAM GIÁC
----------[1] Góc Và Cạnh Của Một Tam Giác
-------------[1] Tổng số đo 3 góc của một tam giác
-------------[2] Quan hệ giữa 3 cạnh của một tam giác
-------------[3] Bài Toán Thực Tế
-------------[0] Chưa phân dạng
----------[2] Tam Giác Bằng Nhau
-------------[1] Hai tam giác bằng nhau
-------------[2] Trường hợp bằng nhau thứ nhất cạnh cạnh cạnh (c.c.c)
-------------[3] Trường hợp bằng nhau thứ hai cạnh góc cạnh (c.g.c)
-------------[4] Trường hợp bằng nhau thứ ba góc cạnh góc (g.c.g)
-------------[5] Trường hợp bằng nhau hai cạnh góc vuông (c.g.c)
-------------[6] Trường hợp bằng nhau cạnh góc vuông và một góc nhọn (g.c.g)
-------------[7] Trường hợp bằng nhau cạnh huyền và một góc nhọn (g.c.g)
-------------[8] Trường hợp bằng nhau cạnh huyền và một cạnh góc vuông (g.c.g)
-------------[9] Bài Toán Thực Tế
-------------[0] Chưa phân dạng
----------[3] Tam Giác Cân
-------------[1] Tam Giác Cân
-------------[2] Tính chất của tam giác cân
-------------[3] Bài Toán Thực Tế
-------------[0] Chưa phân dạng
----------[4] Đường Vuông Góc và Đường Xiên
-------------[1] Quan hệ giữa cạnh và góc trong một tam giác
-------------[2] Đường vuông góc và đường xiên
-------------[3] Mối quan hệ giữa đường vuông góc và đường xiên
-------------[4] Bài Toán Thực Tế
-------------[0] Chưa phân dạng
----------[5] Đường Trung Trực Của Một Đoạn Thẳng
-------------[1] Đường trung trực của một đoạn thẳng
-------------[2] Tính chất của đường trung trực
-------------[3] Bài Toán Thực Tế
-------------[0] Chưa phân dạng
----------[6] Tính Chất 3 Đường Trung Trực Của Tam Giác
-------------[1] Đường trung trực của tam giác
-------------[2] Tính chất 3 đường trung trực của tam giác
-------------[3] Bài Toán Thực Tế
-------------[0] Chưa phân dạng
----------[7] Tính Chất 3 Đường Trung Tuyến Của Tam Giác
-------------[1] Đường trung tuyến của tam giác
-------------[2] Tính chất 3 đường trung tuyến của tam giác
-------------[3] Bài Toán Thực Tế
-------------[0] Chưa phân dạng
----------[8] Tính Chất 3 Đường Cao Của Tam Giác
-------------[1] Đường cao của tam giác
-------------[2] Tính chất 3 đường cao của tam giác
-------------[3] Bài Toán Thực Tế
-------------[0] Chưa phân dạng
----------[9] Tính Chất 3 Đường Phân Giác Của Tam Giác
-------------[1] Đường phân giác của tam giác
-------------[2] Tính chất 3 đường phân giác của tam giác
-------------[3] Bài Toán Thực Tế
-------------[0] Chưa phân dạng
----------[0] Chưa phân dạng
-------------[0] Chưa phân dạng
-------[0] Chưa phân dạng
----------[0] Chưa phân dạng
-------------[0] Chưa phân dạng
%==============================================================
----[D] Đại số
-------[1] SỐ HỮU TỈ
----------[1] TẬP HỢP CÁC SỐ HỮU TỈ 
-------------[1] Số Hữu Tỉ
-------------[2] Thứ tự trong tập hợp các Số Hữu T
-------------[3] Biểu diễn Số Hữu Tỉ trên trục
-------------[4] Số đối của Số Hữu Tỉ
-------------[5] Bài Toán Thực Tế
-------------[0] Chưa phân dạng
----------[2] CÁC PHÉP TÍNH VỚI SỐ HỮU TỈ
-------------[1] Cộng và trừ 2 Số Hữu Tỉ
-------------[2] Tính chất của phép cộng 2 Số Hữu Tỉ
-------------[3] Phân 2 Số Hữu Tỉ
-------------[4] Tính chất của phép nhân 2 Số Hữu Tỉ
-------------[5] Chia 2 Số Hữu Tỉ
-------------[6] Bài Toán Thực Tế
-------------[0] Chưa phân dạng
----------[3] LUỸ THỪA MỘT SỐ HỮU TỈ
-------------[1] Luỹ thừa với số mũ tự nhiên
-------------[2] Tích và thương 2 luỹ thừa cùng cơ số
-------------[3] Luỹ thừa của luỹ thừa
-------------[4] Bài toán tìm x (phép nhân và phép chia)
-------------[5] Bài Toán Thực Tế
-------------[0] Chưa phân dạng
----------[4] QUY TẮC DẤU NGOẶC VÀ CHUYỂN VẾ
-------------[1] Quy tắc dấu ngoặc
-------------[2] Quy tắc chuyển vế
-------------[3] Thứ tự phép tính
-------------[4] Bài toán tìm x (phép cộng và trừ, kết hợp nhân và chia (nếu có))
-------------[5] Bài Toán Thực Tế
-------------[0] Chưa phân dạng
----------[0] Chưa phân dạng
-------------[0] Chưa phân dạng
-------[2] SỐ THỰC
----------[1] SỐ VÔ TỈ. CĂN BẬC HAI SỐ HỌC
-------------[1] Biểu diễn thập phân của số hữu tỉ
-------------[2] Số vô tỷ
-------------[3] Căn bậc hai số học
-------------[4] Tính căn bậc hai số học bằng máy tính
-------------[5] Bài Toán Thực Tế
-------------[0] Chưa phân dạng
----------[2] SỐ THỰC. GIÁ TRỊ TUYỆT ĐỐI CỦA MỘT SỐ THỰC
-------------[1] Số thực và tập hợp các số thực
-------------[2] Thứ tự trong tập hợp các số thực
-------------[3] Trục số thực
-------------[4] Số đối của một số thực
-------------[5] Giá trị tuyệt đối của một số thực
-------------[6] Bài toán thực tế
-------------[0] Chưa phân dạng
----------[3] LÀM TRÒN SỐ, ƯỚC LƯỢNG KẾT QUẢ
-------------[1] Làm tròn số
-------------[2] Làm tròn số căn cứ vào độ chính xác cho trước
-------------[3] Ước lượng các phép tính
-------------[4] Bài Toán Thực Tế
-------------[0] Chưa phân dạng
----------[0] Chưa phân dạng
-------------[0] Chưa phân dạng
-------[3] MỘT SỐ YẾU TỐ THỐNG KÊ
----------[1] THU THẬP VÀ PHÂN LOẠI DỮ LIỆU
-------------[1] Thu thập dữ liệu
-------------[2] Phân loại dữ liệu theo các tiêu chí
-------------[3] Tính hợp lí của dữ liệu
-------------[0] Chưa phân dạng
----------[2] BIỂU ĐỒ HÌNH QUẠT
-------------[1] Ôn tập về biểu đồ hình quạt tròn
-------------[2] Biểu diễn dữ liệu vào biểu đồ hình quạt tròn
-------------[3] Phân tích dữ liệu trên biểu đồ hình quạt tròn
-------------[0] Chưa phân dạng
----------[3] BIỂU ĐỒ ĐOẠN THẲNG
-------------[1] Giới thiệu biểu đồ đoạn thẳng
-------------[2] Vẽ biểu đồ đoạn thẳng
-------------[3] Đọc và phân tích dữ liệu từ biểu đồ đoạn thẳng
-------------[0] Chưa phân dạng
----------[0] Chưa phân dạng
-------------[0] Chưa phân dạng
-------[4] CÁC ĐẠI LƯỢNG TỈ LỆ
----------[1] Tỉ Lệ Thức - Dãy tỉ số bằng nhau
-------------[1] Tỉ lệ thức
-------------[2] Dãy tỉ số bằng nhau
-------------[3] Tìm x, y, z bằng dãy tỉ số bằng nhau
-------------[4] Bài Toán Thực Tế
-------------[0] Chưa phân dạng
----------[2] Đại Lượng Tỉ Lệ Thuận
-------------[1] Đại Lượng Tỉ Lệ Thuận
-------------[2] Tính chất của Đại Lượng Tỉ Lệ Thuận
-------------[3] Các bài toán về Đại Lượng Tỉ Lệ Thuận
-------------[4] Bài Toán Thực Tế
-------------[0] Chưa phân dạng
----------[3] Đại lượng tỉ lệ nghịch
-------------[1] Đại lượng tỉ lệ nghịch
-------------[2] Tính chất của các đại lượng tỉ lệ nghịch
-------------[3] Các bài toán về đại lượng tỉ lệ nghịch
-------------[4] Bài Toán Thực Tế
-------------[0] Chưa phân dạng
----------[0] Chưa phân dạng
-------------[0] Chưa phân dạng
-------[5] BIỂU THỨC ĐẠI SỐ
----------[1] Biểu thức số, Biểu thức đại số
-------------[1] Biểu thức số
-------------[2] Biểu thức đại số
-------------[3] Giá trị của biểu thức số
-------------[4] Bài Toán Thực Tế
-------------[0] Chưa phân dạng
----------[2] Đa thức một biến
-------------[1] Đa thức một biến
-------------[2] Cách biểu diễn đa thức một biến
-------------[3] Giá trị của đa thức một biến
-------------[4] Nghiệm đa thức một biến
-------------[0] Chưa phân dạng
----------[3] Phép cộng và phép trừ đa thức một biến
-------------[1] Phép cộng hai đa thức một biến
-------------[2] Phép trừ hai đa thức một biến
-------------[3] Tính chất phép cộng đa thức một biến
-------------[4] Bài Toán Thực Tế
-------------[0] Chưa phân dạng
----------[4] Phép nhân và phép chia đa thức một biến
-------------[1] Phép nhân đa thức một biến
-------------[2] Phép chia đa thức một biến
-------------[3] Tính chất phép nhân đa thức một biến
-------------[4] Bài Toán Thực Tế
-------------[0] Chưa phân dạng
----------[0] Chưa phân dạng
-------------[0] Chưa phân dạng
-------[0] Chưa phân dạng
----------[0] Chưa phân dạng
-------------[0] Chưa phân dạng
----[H] Hình học
-------[1] CÁC HÌNH KHỐI TRONG THỰC TIỄN
----------[1] Hình Hộp Chữ Nhật, Hình Lập Phương
-------------[1] Hình hộp chữ nhật
-------------[2] Hình lập phương
-------------[3] Bài Toán Thực Tế
-------------[0] Chưa phân dạng
----------[2] DTXQ VÀ THỂ TÍCH HHCN VÀ HÌNH LẬP PHƯƠNG
-------------[1] Nhắc lại công thức tính diện tích xung quanh và thể tích
-------------[2] Một số bài toán thực tế
-------------[0] Chưa phân dạng
----------[3] HÌNH LĂNG TRỤ ĐỨNG TAM GIÁC, HÌNH LĂNG TRỤ ĐỨNG TỨ GIÁC
-------------[1] Hình Lăng Trụ Đứng Tam giác, Hình Lăng Trụ Đứng Tứ Giác
-------------[2] Tạo lập Hình Lăng Trụ Đứng Tam giác, Hình Lăng Trụ Đứng Tứ Giác
-------------[3] Bài Toán Thực Tế
-------------[0] Chưa phân dạng
----------[0] Chưa phân dạng
-------------[0] Chưa phân dạng
-------[2] HÌNH HỌC PHẲNG GÓC VÀ ĐƯỜNG THẰNG SONG SONG
----------[1] TÌM CÁC GÓC Ở VỊ TRÍ ĐẶC BIỆT
-------------[1] Hai góc kề bù
-------------[2] Hai góc đối đỉnh
-------------[3] Tính chất hai góc đối đỉnh
-------------[4] Bài Toán Thực Tế
-------------[0] Chưa phân dạng
----------[2] TIA PHÂN GIÁC
-------------[1] Tia Phân Giác Của Một Góc
-------------[2] Cách vẽ tia phân giác
-------------[3] Bài Toán Thực Tế
-------------[0] Chưa phân dạng
----------[3] Hai Đường Thẳng Song Song
-------------[1] Dấu hiệu nhận biết hai đường thẳng song song
-------------[2] Tiên đề Euclid về đường thẳng song song
-------------[3] Tính chất hai đường thẳng song song 
-------------[4] Bài Toán Thực Tế
-------------[0] Chưa phân dạng
----------[4] Định Lí Và Chứng Minh Một Định Lí
-------------[1] Định Lí là gì?
-------------[2] Chứng Minh Định Lí
-------------[3] DTXQ và Thể Tích của một số hình khối trong thực tiễn
-------------[0] Chưa phân dạng
----------[0] Chưa phân dạng
-------------[0] Chưa phân dạng
-------[3] HÌNH HỌC PHẲNG TAM GIÁC
----------[1] Góc Và Cạnh Của Một Tam Giác
-------------[1] Tổng số đo 3 góc của một tam giác
-------------[2] Quan hệ giữa 3 cạnh của một tam giác
-------------[3] Bài Toán Thực Tế
-------------[0] Chưa phân dạng
----------[2] Tam Giác Bằng Nhau
-------------[1] Hai tam giác bằng nhau
-------------[2] Trường hợp bằng nhau thứ nhất cạnh cạnh cạnh (c.c.c)
-------------[3] Trường hợp bằng nhau thứ hai cạnh góc cạnh (c.g.c)
-------------[4] Trường hợp bằng nhau thứ ba góc cạnh góc (g.c.g)
-------------[5] Trường hợp bằng nhau hai cạnh góc vuông (c.g.c)
-------------[6] Trường hợp bằng nhau cạnh góc vuông và một góc nhọn (g.c.g)
-------------[7] Trường hợp bằng nhau cạnh huyền và một góc nhọn (g.c.g)
-------------[8] Trường hợp bằng nhau cạnh huyền và một cạnh góc vuông (g.c.g)
-------------[9] Bài Toán Thực Tế
-------------[0] Chưa phân dạng
----------[3] Tam Giác Cân
-------------[1] Tam Giác Cân
-------------[2] Tính chất của tam giác cân
-------------[3] Bài Toán Thực Tế
-------------[0] Chưa phân dạng
----------[4] Đường Vuông Góc và Đường Xiên
-------------[1] Quan hệ giữa cạnh và góc trong một tam giác
-------------[2] Đường vuông góc và đường xiên
-------------[3] Mối quan hệ giữa đường vuông góc và đường xiên
-------------[4] Bài Toán Thực Tế
-------------[0] Chưa phân dạng
----------[5] Đường Trung Trực Của Một Đoạn Thẳng
-------------[1] Đường trung trực của một đoạn thẳng
-------------[2] Tính chất của đường trung trực
-------------[3] Bài Toán Thực Tế
-------------[0] Chưa phân dạng
----------[6] Tính Chất 3 Đường Trung Trực Của Tam Giác
-------------[1] Đường trung trực của tam giác
-------------[2] Tính chất 3 đường trung trực của tam giác
-------------[3] Bài Toán Thực Tế
-------------[0] Chưa phân dạng
----------[7] Tính Chất 3 Đường Trung Tuyến Của Tam Giác
-------------[1] Đường trung tuyến của tam giác
-------------[2] Tính chất 3 đường trung tuyến của tam giác
-------------[3] Bài Toán Thực Tế
-------------[0] Chưa phân dạng
----------[8] Tính Chất 3 Đường Cao Của Tam Giác
-------------[1] Đường cao của tam giác
-------------[2] Tính chất 3 đường cao của tam giác
-------------[3] Bài Toán Thực Tế
-------------[0] Chưa phân dạng
----------[9] Tính Chất 3 Đường Phân Giác Của Tam Giác
-------------[1] Đường phân giác của tam giác
-------------[2] Tính chất 3 đường phân giác của tam giác
-------------[3] Bài Toán Thực Tế
-------------[0] Chưa phân dạng
----------[0] Chưa phân dạng
-------------[0] Chưa phân dạng
-------[0] Chưa phân dạng
----------[0] Chưa phân dạng
-------------[0] Chưa phân dạng
-------[0] Một số chủ đề Hình học bồi dưỡng HSG lớp 7
----------[1] Chứng minh đẳng thức, tam giác bằng nhau
-------------[0] Chưa phân dạng
----------[2] Chứng minh quan hệ song song, vuông góc
-------------[0] Chưa phân dạng
----------[3] Các bài toán định lượng
-------------[0] Chưa phân dạng
----------[4] Chứng minh ba điểm thẳng hàng, ba ĐT đồng quy
-------------[0] Chưa phân dạng
----------[5] Yếu tố cố định
-------------[0] Chưa phân dạng
----------[6] Bất đẳng thức, GTLN, GTNN
-------------[0] Chưa phân dạng
----------[7] Hình học tổ hợp
-------------[0] Chưa phân dạng
----------[0] Chưa phân dạng
-------------[0] Chưa phân dạng
%==============================================================
----[G] 7-HSG
-------[1] Một số chủ đề Đại số, Số học bồi dưỡng HSG lớp 7
----------[1] Biểu thức đại số
-------------[0] Chưa phân dạng
----------[2] PT, BPT. Hệ phương trình, hệ BPT
-------------[0] Chưa phân dạng
----------[3] Đa thức
-------------[0] Chưa phân dạng
----------[4] Dãy số nguyên
-------------[0] Chưa phân dạng
----------[5] Số học (tính chia hết, ƯCLN, BCNN, số nguyên tố, hợp số, số chính phương, phần nguyên, \ldots)
-------------[0] Chưa phân dạng
----------[6] PT nghiệm nguyên
-------------[0] Chưa phân dạng
----------[7] Đồng dư
-------------[0] Chưa phân dạng
----------[8] Bất đẳng thức, giá trị lớn nhất, giá trị nhỏ nhất
-------------[0] Chưa phân dạng
----------[9] Suy luận toán học; bài toán trò chơi; đại số tổ hơp
-------------[0] Chưa phân dạng
----------[0] Chưa phân dạng
-------------[0] Chưa phân dạng
-------[2] Một số chủ đề Hình học bồi dưỡng HSG lớp 7
----------[1] Chứng minh đẳng thức, tam giác bằng nhau
-------------[0] Chưa phân dạng
----------[2] Chứng minh quan hệ song song, vuông góc
-------------[0] Chưa phân dạng
----------[3] Các bài toán định lượng
-------------[0] Chưa phân dạng
----------[4] Chứng minh ba điểm thẳng hàng, ba ĐT đồng quy
-------------[0] Chưa phân dạng
----------[5] Yếu tố cố định
-------------[0] Chưa phân dạng
----------[6] Bất đẳng thức, GTLN, GTNN
-------------[0] Chưa phân dạng
----------[7] Hình học tổ hợp
-------------[0] Chưa phân dạng
----------[0] Chưa phân dạng
-------------[0] Chưa phân dạng
----[0] Chưa phân dạng
-------[0] Chưa phân dạng
----------[0] Chưa phân dạng
-------------[0] Chưa phân dạng
%==============================================================BẮT ĐẦU ID LỚP 8
-[8] Lớp 8
----[P] 8 - NGÂN HÀNG TẠM
-------[1] BIỂU THỨC ĐẠI SỐ
----------[1] ĐƠN THỨC VÀ ĐA THỨC NHIỀU BIẾN
-------------[1] Đơn thức và đa thức
-------------[2] Đơn thức thu gọn
-------------[3] Cộng trừ, đơn thức đồng dạng
-------------[4] Đa thức thu gọn
-------------[5] Bài Toán Thực Tế
-------------[0] Chưa phân dạng
----------[2] CÁC PHÉP TOÁN VỚI ĐA THỨC NHIỀU BIẾN
-------------[1] Cộng, trừ hai đa thức
-------------[2] Nhân hai đa thức
-------------[3] Chia đa thức cho đơn thức
-------------[4] Bài Toán Thực Tế
-------------[0] Chưa phân dạng
----------[3] HẰNG ĐẲNG THỨC ĐÁNG NHỚ
-------------[1] Bình phương của một tổng, một hiệu
-------------[2] Hiệu của hai bình phương
-------------[3] Lập phương của một tổng, một hiệu
-------------[4] Tổng và hiệu của hai lập phương
-------------[5] Bài Toán Thực Tế
-------------[0] Chưa phân dạng
----------[4] PHÂN TÍCH ĐA THỨC THÀNH NHÂN TỬ
-------------[1] Phương pháp đặt nhân tử chung
-------------[2] Phương pháp sử dụng hằng đẳng thức
-------------[3] Phương pháp nhóm hạng tử
-------------[4] Bài Toán Thực Tế
-------------[0] Chưa phân dạng
----------[5] PHÂN THỨC ĐẠI SỐ
-------------[1] Phân thức đại số
-------------[2] Hai phân thức bằng nhau
-------------[3] Tính chất cơ bản của phân thức
-------------[4] Bài Toán Thực Tế
-------------[0] Chưa phân dạng
----------[6] CỘNG, TRỪ PHÂN THỨC
-------------[1] Cộng, trừ hai phân thức cùng mẫu
-------------[2] Cộng, trừ hai phân thức khác mẫu
-------------[3] Bài Toán Thực Tế
-------------[0] Chưa phân dạng
----------[7] NHÂN, CHIA PHÂN THỨC
-------------[1] Nhân hai phân thức
-------------[2] Chia hai phân thức
-------------[3] Bài Toán Thực Tế
-------------[0] Chưa phân dạng
----------[0] Chưa phân dạng
-------------[0] Chưa phân dạng
-------[2] MỘT SỐ YẾU TỐ THỐNG KÊ
----------[1] THU THẬP VÀ PHÂN LOẠI DỮ LIỆU
-------------[1] Thu thập dữ liệu
-------------[2] Phân loại dữ liệu theo các tiêu chí
-------------[3] Tính hợp lí của dữ liệu
-------------[0] Chưa phân dạng
----------[2] LỰA CHỌN DẠNG BIỂU ĐỒ ĐỂ BIỂU DIỄN DỮ LIỆU
-------------[1] Lựa chọn dạng biểu đồ để biểu diễn dữ liệu
-------------[2] Các dạng biểu diễn khác nhau cho một tập dữ liệu
-------------[0] Chưa phân dạng
----------[3] PHÂN TÍCH DỮ LIỆU
-------------[1] Phát hiện vấn đề qua phân tích dữ liệu thống kê
-------------[2] Giải quyết các vấn đề qua phân tích biểu đồ thống kê
-------------[0] Chưa phân dạng
----------[0] Chưa phân dạng
-------------[0] Chưa phân dạng
-------[3] HÀM SỐ VÀ ĐỒ THỊ
----------[1] KHÁI NIỆM HÀM SỐ
-------------[1] Khái niệm hàm số
-------------[2] Giá trị của hàm số
-------------[3] Bài toán thực tế
-------------[0] Chưa phân dạng
----------[2] TOẠ ĐỘ CỦA MỘT ĐIỂM VÀ ĐỒ THỊ CỦA HÀM SỐ
-------------[1] Toạ độ của một điểm
-------------[2] Xác định một điểm trên mặt phẳng toạ độ khi biết toạ độ của nó
-------------[3] Đồ thị của hàm số
-------------[4] Bài toán thực tế
-------------[0] Chưa phân dạng
----------[3] HÀM SỐ BẬC NHẤT
-------------[1] Hàm số bậc nhất
-------------[2] Bảng giá trị của hàm số bậc nhất
-------------[3] Đồ thị của hàm số bậc nhất
-------------[4] Bài toán thực tế
-------------[0] Chưa phân dạng
----------[4] HỆ SỐ GÓC CỦA ĐƯỜNG THẲNG
-------------[1] Hệ số góc của đường thẳng y = ax + b
-------------[2] Hai đường thẳng song song, hai đường thẳng cắt nhau
-------------[3] Bài toán thực tế
-------------[0] Chưa phân dạng
----------[0] Chưa phân dạng
-------------[0] Chưa phân dạng
-------[4] PHƯƠNG TRÌNH
----------[1] PHƯƠNG TRÌNH BẬC NHẤT MỘT ẨN
-------------[1] Phương trình một ẩn
-------------[2] Phương trình bậc nhất một ẩn và cách giả
-------------[3] Bài toán thực tế
-------------[0] Chưa phân dạng
----------[2] GIẢI TOÁN BẰNG CÁCH LẬP PHƯƠNG TRÌNH BẬC NHẤT
-------------[1] Biểu diễn một đại lượng bởi biểu thức chứa ẩn
-------------[2] Giải bài toán bằng cách lập phương trình bậc nhất
-------------[0] Chưa phân dạng
----------[0] Chưa phân dạng
-------------[0] Chưa phân dạng
-------[5] MỘT SỐ XÁC SUẤT
----------[1] MÔ TẢ XÁC SUẤT BẰNG TỈ SỐ
-------------[1] Kết quả thuận lợi
-------------[2] Mô tả xác suất bằng tỉ số
-------------[0] Chưa phân dạng
----------[2] XÁC SUẤT LÝ THUYẾT VÀ XÁC SUẤT THỰC NGHIỆM
-------------[1] Xác suất lý thuyết
-------------[2] Xác suất thực nghiệm
-------------[0] Chưa phân dạng
----------[0] Chưa phân dạng
-------------[0] Chưa phân dạng
-------[6] CÁC HÌNH KHỐI TRONG THỰC TIỄN
----------[1] HÌNH CHÓP TAM GIÁC ĐỀU - HÌNH CHÓP TỨ GIÁC ĐỀU
-------------[1] Hình chóp tam giác đều - Hình chóp tứ giác đều
-------------[2] Tạo lập hình chóp tam giác đều, hình chóp tứ giác đều
-------------[3] Bài toán thực tế
-------------[0] Chưa phân dạng
----------[2] DTXQ VÀ THỂ TÍCH CỦA HÌNH CHÓP TAM GIÁC ĐỀU, TỨ GIÁC ĐỀU
-------------[1] DTQX của hình chóp tam giác đều, tứ giác đều
-------------[2] Thể tích của hình chóp tam giác đều, tứ giác đều
-------------[3] Bài toán thực tế
-------------[0] Chưa phân dạng
----------[0] Chưa phân dạng
-------------[0] Chưa phân dạng
-------[7] ĐỊNH LÝ PYTHAGORE. CÁC LOẠI TỨ GIÁC THƯỜNG GẶP
----------[1] ĐỊNH LÝ PYTHAGORE
-------------[1] Định lý Pythagore
-------------[2] Định lý Pythagore đảo
-------------[3] Vận dụng định lý Pythagore
-------------[4] Bài toán thực tế
-------------[0] Chưa phân dạng
----------[2] TỨ GIÁC
-------------[1] Tứ giác
-------------[2] Tổng các góc của một tứ giác
-------------[3] Bài toán thực tế
-------------[0] Chưa phân dạng
----------[3] HÌNH THANG, HÌNH THANG CÂN
-------------[1] Hình thang, hình thang cân
-------------[2] Tính chất của hình thang cân
-------------[3] Dấu hiệu nhận biết hình thang cân
-------------[4] Bài toán thực tế
-------------[0] Chưa phân dạng
----------[4] HÌNH BÌNH HÀNH - HÌNH THOI
-------------[1] Hình bình hành
-------------[2] Hình thoi
-------------[3] Bài toán thực tế
-------------[0] Chưa phân dạng
----------[5] HÌNH CHỮ NHẬT - HÌNH VUÔNG
-------------[1] Hình chữ nhật
-------------[2] Hình vuông
-------------[3] Bài toán thực tế
-------------[0] Chưa phân dạng
----------[0] Chưa phân dạng
-------------[0] Chưa phân dạng
-------[8] ĐỊNH LÝ THALÈS
----------[1] ĐỊNH LÝ THALÈS TRONG TAM GIÁC
-------------[1] Đoạn thẳng tỉ lệ
-------------[2] Định lý Thalès trong tam giác
-------------[3] Bài toán thực tế
-------------[0] Chưa phân dạng
----------[2] ĐƯỜNG TRUNG BÌNH CỦA TAM GIÁC
-------------[1] Đường trung bình của tam giác
-------------[2] Tính chất của đường trung bình
-------------[3] Bài toán thực tế
-------------[0] Chưa phân dạng
----------[3] Tính chất đường phân giác của tam giác
-------------[1] Tính chất đường phân giác
-------------[2] Áp dụng tính chia tỉ lệ của đường phân giác của tam giác
-------------[3] Bài toán thực tế
-------------[0] Chưa phân dạng
----------[0] Chưa phân dạng
-------------[0] Chưa phân dạng
-------[9] HÌNH ĐỒNG DẠNG
----------[1] HAI TAM GIÁC ĐỒNG DẠNG
-------------[1] Tam giác đồng dạng
-------------[2] Tính chất
-------------[3] Định lí
-------------[4] Bài toán thực tế
-------------[0] Chưa phân dạng
----------[2] CÁC TRƯỜNG HỢP ĐỒNG DẠNG CỦA HAI TAM GIÁC
-------------[1] Trường hợp đồng dạng (c.c.c) (thứ nhất)
-------------[2] Trường hợp đồng dạng (c.g.c) (thứ hai)
-------------[3] Trường hợp đồng dạng (g.g) (thứ ba)
-------------[4] Bài toán thực tế
-------------[0] Chưa phân dạng
----------[3] CÁC TRƯỜNG HỢP ĐỒNG DẠNG CỦA HAI TAM GIÁC VUÔNG
-------------[1] Trường hợp đồng dạng góc nhọn của tam giác vuông
-------------[2] Trường hợp đồng dạng 2 cạnh góc vuông
-------------[3] Trường hợp đồng dạng cạnh huyền cạnh góc vuông
-------------[4] Bài toán thực tế
-------------[0] Chưa phân dạng
----------[4] HAI HÌNH ĐỒNG DẠNG
-------------[1] Hình đồng dạng phối cảnh
-------------[2] Hai hình đồng dạng
-------------[3] Hình đồng dạng trong tự nhiên và đời sống
-------------[0] Chưa phân dạng
----------[0] Chưa phân dạng
-------------[0] Chưa phân dạng
-------[0] Chưa phân dạng
----------[0] Chưa phân dạng
-------------[0] Chưa phân dạng
%==============================================================
----[D] Đại số
-------[1] BIỂU THỨC ĐẠI SỐ
----------[1] ĐƠN THỨC VÀ ĐA THỨC NHIỀU BIẾN
-------------[1] Đơn thức và đa thức
-------------[2] Đơn thức thu gọn
-------------[3] Cộng trừ, đơn thức đồng dạng
-------------[4] Đa thức thu gọn
-------------[5] Bài Toán Thực Tế
-------------[0] Chưa phân dạng
----------[2] CÁC PHÉP TOÁN VỚI ĐA THỨC NHIỀU BIẾN
-------------[1] Cộng, trừ hai đa thức
-------------[2] Nhân hai đa thức
-------------[3] Chia đa thức cho đơn thức
-------------[4] Bài Toán Thực Tế
-------------[0] Chưa phân dạng
----------[3] HẰNG ĐẲNG THỨC ĐÁNG NHỚ
-------------[1] Bình phương của một tổng, một hiệu
-------------[2] Hiệu của hai bình phương
-------------[3] Lập phương của một tổng, một hiệu
-------------[4] Tổng và hiệu của hai lập phương
-------------[5] Bài Toán Thực Tế
-------------[0] Chưa phân dạng
----------[4] PHÂN TÍCH ĐA THỨC THÀNH NHÂN TỬ
-------------[1] Phương pháp đặt nhân tử chung
-------------[2] Phương pháp sử dụng hằng đẳng thức
-------------[3] Phương pháp nhóm hạng tử
-------------[4] Bài Toán Thực Tế
-------------[0] Chưa phân dạng
----------[5] PHÂN THỨC ĐẠI SỐ
-------------[1] Phân thức đại số
-------------[2] Hai phân thức bằng nhau
-------------[3] Tính chất cơ bản của phân thức
-------------[4] Bài Toán Thực Tế
-------------[0] Chưa phân dạng
----------[6] CỘNG, TRỪ PHÂN THỨC
-------------[1] Cộng, trừ hai phân thức cùng mẫu
-------------[2] Cộng, trừ hai phân thức khác mẫu
-------------[3] Bài Toán Thực Tế
-------------[0] Chưa phân dạng
----------[7] NHÂN, CHIA PHÂN THỨC
-------------[1] Nhân hai phân thức
-------------[2] Chia hai phân thức
-------------[3] Bài Toán Thực Tế
-------------[0] Chưa phân dạng
----------[0] Chưa phân dạng
-------------[0] Chưa phân dạng
-------[2] MỘT SỐ YẾU TỐ THỐNG KÊ
----------[1] THU THẬP VÀ PHÂN LOẠI DỮ LIỆU
-------------[1] Thu thập dữ liệu
-------------[2] Phân loại dữ liệu theo các tiêu chí
-------------[3] Tính hợp lí của dữ liệu
-------------[0] Chưa phân dạng
----------[2] LỰA CHỌN DẠNG BIỂU ĐỒ ĐỂ BIỂU DIỄN DỮ LIỆU
-------------[1] Lựa chọn dạng biểu đồ để biểu diễn dữ liệu
-------------[2] Các dạng biểu diễn khác nhau cho một tập dữ liệu
-------------[0] Chưa phân dạng
----------[3] PHÂN TÍCH DỮ LIỆU
-------------[1] Phát hiện vấn đề qua phân tích dữ liệu thống kê
-------------[2] Giải quyết các vấn đề qua phân tích biểu đồ thống kê
-------------[0] Chưa phân dạng
----------[0] Chưa phân dạng
-------------[0] Chưa phân dạng
-------[3] HÀM SỐ VÀ ĐỒ THỊ
----------[1] KHÁI NIỆM HÀM SỐ
-------------[1] Khái niệm hàm số
-------------[2] Giá trị của hàm số
-------------[3] Bài toán thực tế
-------------[0] Chưa phân dạng
----------[2] TOẠ ĐỘ CỦA MỘT ĐIỂM VÀ ĐỒ THỊ CỦA HÀM SỐ
-------------[1] Toạ độ của một điểm
-------------[2] Xác định một điểm trên mặt phẳng toạ độ khi biết toạ độ của nó
-------------[3] Đồ thị của hàm số
-------------[4] Bài toán thực tế
-------------[0] Chưa phân dạng
----------[3] HÀM SỐ BẬC NHẤT
-------------[1] Hàm số bậc nhất
-------------[2] Bảng giá trị của hàm số bậc nhất
-------------[3] Đồ thị của hàm số bậc nhất
-------------[4] Bài toán thực tế
-------------[0] Chưa phân dạng
----------[4] HỆ SỐ GÓC CỦA ĐƯỜNG THẲNG
-------------[1] Hệ số góc của đường thẳng y = ax + b
-------------[2] Hai đường thẳng song song, hai đường thẳng cắt nhau
-------------[3] Bài toán thực tế
-------------[0] Chưa phân dạng
----------[0] Chưa phân dạng
-------------[0] Chưa phân dạng
-------[4] PHƯƠNG TRÌNH
----------[1] PHƯƠNG TRÌNH BẬC NHẤT MỘT ẨN
-------------[1] Phương trình một ẩn
-------------[2] Phương trình bậc nhất một ẩn và cách giả
-------------[3] Bài toán thực tế
-------------[0] Chưa phân dạng
----------[2] GIẢI TOÁN BẰNG CÁCH LẬP PHƯƠNG TRÌNH BẬC NHẤT
-------------[1] Biểu diễn một đại lượng bởi biểu thức chứa ẩn
-------------[2] Giải bài toán bằng cách lập phương trình bậc nhất
-------------[0] Chưa phân dạng
----------[0] Chưa phân dạng
-------------[0] Chưa phân dạng
-------[5] MỘT SỐ XÁC SUẤT
----------[1] MÔ TẢ XÁC SUẤT BẰNG TỈ SỐ
-------------[1] Kết quả thuận lợi
-------------[2] Mô tả xác suất bằng tỉ số
-------------[0] Chưa phân dạng
----------[2] XÁC SUẤT LÝ THUYẾT VÀ XÁC SUẤT THỰC NGHIỆM
-------------[1] Xác suất lý thuyết
-------------[2] Xác suất thực nghiệm
-------------[0] Chưa phân dạng
----------[0] Chưa phân dạng
-------------[0] Chưa phân dạng
-------[0] Chưa phân dạng
----------[0] Chưa phân dạng
-------------[0] Chưa phân dạng
-------[0] Một số chủ đề Đại số, Số học bồi dưỡng HSG lớp 8
----------[1] Biểu thức đại số
----------[2] PT, BPT. Hệ phương trình, hệ BPT
----------[3] Đa thức
----------[4] Dãy số nguyên
----------[5] Số học (tính chia hết, ƯCLN, BCNN, số nguyên tố, hợp số, số chính phương, phần nguyên, \ldots)
----------[6] PT nghiệm nguyên
----------[7] Đồng dư
----------[8] Bất đẳng thức, giá trị lớn nhất, giá trị nhỏ nhất
----------[9] Suy luận toán học; các bài toán về trò chơi; đại số tổ hơp
----[H] Hình học
-------[1] CÁC HÌNH KHỐI TRONG THỰC TIỄN
----------[1] HÌNH CHÓP TAM GIÁC ĐỀU - HÌNH CHÓP TỨ GIÁC ĐỀU
-------------[1] Hình chóp tam giác đều - Hình chóp tứ giác đều
-------------[2] Tạo lập hình chóp tam giác đều, hình chóp tứ giác đều
-------------[3] Bài toán thực tế
-------------[0] Chưa phân dạng
----------[2] DTXQ VÀ THỂ TÍCH CỦA HÌNH CHÓP TAM GIÁC ĐỀU, TỨ GIÁC ĐỀU
-------------[1] DTQX của hình chóp tam giác đều, tứ giác đều
-------------[2] Thể tích của hình chóp tam giác đều, tứ giác đều
-------------[3] Bài toán thực tế
-------------[0] Chưa phân dạng
----------[0] Chưa phân dạng
-------------[0] Chưa phân dạng
-------[2] ĐỊNH LÝ PYTHAGORE. CÁC LOẠI TỨ GIÁC THƯỜNG GẶP
----------[1] ĐỊNH LÝ PYTHAGORE
-------------[1] Định lý Pythagore
-------------[2] Định lý Pythagore đảo
-------------[3] Vận dụng định lý Pythagore
-------------[4] Bài toán thực tế
-------------[0] Chưa phân dạng
----------[2] TỨ GIÁC
-------------[1] Tứ giác
-------------[2] Tổng các góc của một tứ giác
-------------[3] Bài toán thực tế
-------------[0] Chưa phân dạng
----------[3] HÌNH THANG, HÌNH THANG CÂN
-------------[1] Hình thang, hình thang cân
-------------[2] Tính chất của hình thang cân
-------------[3] Dấu hiệu nhận biết hình thang cân
-------------[4] Bài toán thực tế
-------------[0] Chưa phân dạng
----------[4] HÌNH BÌNH HÀNH - HÌNH THOI
-------------[1] Hình bình hành
-------------[2] Hình thoi
-------------[3] Bài toán thực tế
-------------[0] Chưa phân dạng
----------[5] HÌNH CHỮ NHẬT - HÌNH VUÔNG
-------------[1] Hình chữ nhật
-------------[2] Hình vuông
-------------[3] Bài toán thực tế
-------------[0] Chưa phân dạng
----------[0] Chưa phân dạng
-------------[0] Chưa phân dạng
-------[3] ĐỊNH LÝ THALÈS
----------[1] ĐỊNH LÝ THALÈS TRONG TAM GIÁC
-------------[1] Đoạn thẳng tỉ lệ
-------------[2] Định lý Thalès trong tam giác
-------------[3] Bài toán thực tế
-------------[0] Chưa phân dạng
----------[2] ĐƯỜNG TRUNG BÌNH CỦA TAM GIÁC
-------------[1] Đường trung bình của tam giác
-------------[2] Tính chất của đường trung bình
-------------[3] Bài toán thực tế
-------------[0] Chưa phân dạng
----------[3] Tính chất đường phân giác của tam giác
-------------[1] Tính chất đường phân giác
-------------[2] Áp dụng tính chia tỉ lệ của đường phân giác của tam giác
-------------[3] Bài toán thực tế
-------------[0] Chưa phân dạng
----------[0] Chưa phân dạng
-------------[0] Chưa phân dạng
-------[4] HÌNH ĐỒNG DẠNG
----------[1] HAI TAM GIÁC ĐỒNG DẠNG
-------------[1] Tam giác đồng dạng
-------------[2] Tính chất
-------------[3] Định lí
-------------[4] Bài toán thực tế
-------------[0] Chưa phân dạng
----------[2] CÁC TRƯỜNG HỢP ĐỒNG DẠNG CỦA HAI TAM GIÁC
-------------[1] Trường hợp đồng dạng (c.c.c) (thứ nhất)
-------------[2] Trường hợp đồng dạng (c.g.c) (thứ hai)
-------------[3] Trường hợp đồng dạng (g.g) (thứ ba)
-------------[4] Bài toán thực tế
-------------[0] Chưa phân dạng
----------[3] CÁC TRƯỜNG HỢP ĐỒNG DẠNG CỦA HAI TAM GIÁC VUÔNG
-------------[1] Trường hợp đồng dạng góc nhọn của tam giác vuông
-------------[2] Trường hợp đồng dạng 2 cạnh góc vuông
-------------[3] Trường hợp đồng dạng cạnh huyền cạnh góc vuông
-------------[4] Bài toán thực tế
-------------[0] Chưa phân dạng
----------[4] HAI HÌNH ĐỒNG DẠNG
-------------[1] Hình đồng dạng phối cảnh
-------------[2] Hai hình đồng dạng
-------------[3] Hình đồng dạng trong tự nhiên và đời sống
-------------[0] Chưa phân dạng
----------[0] Chưa phân dạng
-------------[0] Chưa phân dạng
-------[0] Chưa phân dạng
----------[0] Chưa phân dạng
-------------[0] Chưa phân dạng
-------[0] Một số chủ đề Hình học bồi dưỡng HSG lớp 8
----------[1] Tứ giác và đường trung bình
----------[2] Diện tích đa giác
----------[3] Định lí Thales, tính chất phân giác
----------[4] Tam giác đồng dạng
----------[5] Các bài toán sử dụng các định lí hình học cổ điển (Ceva, Menelaus, ...)
----------[6] Chứng minh ba điểm thẳng hàng, ba ĐT đồng quy
----------[7] Yếu tố cố định
----------[8] Bất đẳng thức, GTLN, GTNN
----------[9] Hình học tổ hợp
%==============================================================
----[G] 8-HSG
-------[1] Một số chủ đề Đại số, Số học bồi dưỡng HSG lớp 8
----------[1] Biểu thức đại số
-------------[0] Chưa phân dạng
----------[2] PT, BPT. Hệ phương trình, hệ BPT
-------------[0] Chưa phân dạng
----------[3] Đa thức
-------------[0] Chưa phân dạng
----------[4] Dãy số nguyên
-------------[0] Chưa phân dạng
----------[5] Số học (tính chia hết, ƯCLN, BCNN, số nguyên tố, hợp số, số chính phương, phần nguyên, \ldots)
-------------[0] Chưa phân dạng
----------[6] PT nghiệm nguyên
-------------[0] Chưa phân dạng
----------[7] Đồng dư
-------------[0] Chưa phân dạng
----------[8] Bất đẳng thức, giá trị lớn nhất, giá trị nhỏ nhất
-------------[0] Chưa phân dạng
----------[9] Suy luận toán học; các bài toán về trò chơi; đại số tổ hơp
-------------[0] Chưa phân dạng
----------[0] Chưa phân dạng
-------------[0] Chưa phân dạng
-------[2] Một số chủ đề Hình học bồi dưỡng HSG lớp 8
----------[1] Tứ giác và đường trung bình
-------------[0] Chưa phân dạng
----------[2] Diện tích đa giác
-------------[0] Chưa phân dạng
----------[3] Định lí Thales, tính chất phân giác
-------------[0] Chưa phân dạng
----------[4] Tam giác đồng dạng
-------------[0] Chưa phân dạng
----------[5] Các bài toán sử dụng các định lí hình học cổ điển (Ceva, Menelaus, ...)
-------------[0] Chưa phân dạng
----------[6] Chứng minh ba điểm thẳng hàng, ba ĐT đồng quy
-------------[0] Chưa phân dạng
----------[7] Yếu tố cố định
-------------[0] Chưa phân dạng
----------[8] Cực trị hình học
-------------[0] Chưa phân dạng
----------[9] Hình học tổ hợp
-------------[0] Chưa phân dạng
----------[0] Chưa phân dạng
-------------[0] Chưa phân dạng
-------[0] Chưa phân dạng
----------[0] Chưa phân dạng
-------------[0] Chưa phân dạng
----[0] Chưa phân dạng
-------[0] Chưa phân dạng
----------[0] Chưa phân dạng
-------------[0] Chưa phân dạng
%==============================================================BẮT ĐẦU ID LỚP 8
-[9] Lớp 9
----[T] Tuyển sinh 10
-------[C] Chuyên Toán
----------[1] Căn thức bậc hai và các vấn đề liên quan
----------[2] HS bậc nhất, bậc hai
----------[3] PT bậc hai. Hệ thức Vi-ét và ứng dụng
----------[4] Giải bài toán bằng cách lập phương trình, hệ phương trình
-------------[1] Toán chuyển động đường bộ
-------------[2] Toán chuyển động đường thủy
-------------[3] Toán có nội dung hình học
-------------[4] Toán năng suất làm việc
-------------[5] Toán số học. Toán phần trăm
-------------[6] Toán làm chung, làm riêng
----------[5] Hệ phương trình
-------------[1] Giải hệ phương trình bằng cách đặt ẩn phụ
-------------[2] Hệ phương trình có chứa tham số
-------------[3] Hệ phương trình đưa về dạng tích
-------------[4] Hệ phương trình đưa về dạng tổng tích
-------------[5] Hệ phương trình đưa về tổng các bình phương (căn thức, trị tuyệt đối)
-------------[6] Hệ phương trình đối xứng
----------[6] PT vô tỉ
-------------[1] Phương pháp đặt ẩn phụ
-------------[2] Phương pháp đưa về phương trình tích
-------------[3] Phương pháp liên hợp
-------------[4] Phương pháp đánh giá
----------[7] Đường tròn và các vấn đề liên quan
-------------[1] Chứng minh tứ giác nội tiếp
-------------[2] Chứng minh các điểm thuộc đường tròn
-------------[3] Chứng minh đẳng thức hình học
-------------[4] Tính toán độ dài, số đo góc, diện tích, khoảng cách
-------------[5] Tứ giác nội tiếp với trung điểm của đoạn thẳng
-------------[6] Tứ giác nội tiếp với tia phân giác của một góc
-------------[7] Bài toán quỹ tích, tìm yếu tố cố định
-------------[8] Cực trị hình học
----------[8] Số học. Các bài toán suy luận logic. Nguyên lý Dirichlet
-------------[1] Bài toán liên quan đến số chính phương
-------------[2] Bài toán liên quan đến số nguyên tố
-------------[3] Bài toán suy luận logic
-------------[4] Nguyên Lý Dirichlet
-------------[5] Đồng dư
-------------[6] Phần nguyên
-------------[7] PT nghiệm nguyên
----------[9] Bất đẳng thức. Giá trị lớn nhất - Nhỏ nhất
-------------[1] Tìm giá trị nhỏ nhất của biểu thức
-------------[2] Tìm giá trị lớn nhất của biểu thức
-------------[3] Chứng minh đẳng thức
-------------[4] Chứng minh bất đẳng thức
-------------[5] Tính giá trị biểu thức
-------[K] Môn Toán chung
----------[1] Căn bậc hai số học
----------[2] Căn thức bậc hai
----------[3] Hệ phương trình bậc nhất hai ẩn
----------[4] HS bậc nhất, bậc hai
----------[5] PT bậc hai một ẩn
----------[6] Hệ thức Vi-ét và ứng dụng
----------[7] Hệ thức lượng trong tam giác vuông
----------[8] Đường tròn và các vấn đề liên quan
----------[9] Giá trị nhỏ nhất, lớn nhất
----------[A] Toán thực tế
-------------[1] Toán về số học và suy luận
-------------[2] Toán liên quan đến HS
-------------[3] Toán quy về giải phương trình, hệ phương trình
-------------[4] Toán liên quan đến tam giác đồng dạng, hệ thức lượng
-------------[5] Toán liên quan đến đường tròn
-------------[6] Lãi suất ngân hàng
%==============================================================
----[P] 9 - NGÂN HÀNG TẠM
-------[1] PHƯƠNG TRÌNH VÀ HỆ PHƯƠNG TRÌNH
----------[1] PHƯƠNG TRÌNH QUY VỀ PHƯƠNG TRÌNH BẬC NHẤT
-------------[1] Phương trình tích
-------------[2] Phương trình chứa ẩn ở mẫu quy về phương trình bậc nhất
-------------[3] Giải bài toán bằng cách lập phương trình
-------------[0] Chưa phân dạng
----------[2] PHƯƠNG TRÌNH BẬC NHẤT 2 ẨN VÀ HỆ PHƯƠNG TRÌNH BẬC NHẤT 2 ẨN
-------------[1] Phương trình bậc nhất hai ẩn
-------------[2] Hệ phương trình bậc nhất hai ẩn
-------------[3] Bài Toán Thực Tế
-------------[0] Chưa phân dạng
----------[3] GIẢI HỆ PHƯƠNG TRÌNH BẬC NHẤT 2 ẨN
-------------[1] Giải hệ phương trình bằng phương pháp thế
-------------[2] Giải hệ phương trình bằng phương pháp cộng đại số
-------------[3] Tìm nghiệm của hệ bằng máy tính cầm tay
-------------[4] Giải bài toán bằng cách lập hệ phương trình
-------------[0] Chưa phân dạng
----------[0] Chưa phân dạng
-------------[0] Chưa phân dạng
-------[2] BẤT ĐẲNG THỨC. BẤT PHƯƠNG TRÌNH BẬC NHẤT MỘT ẨN
----------[1] BẤT ĐẲNG THỨC
-------------[1] Khái niệm bất đẳng thức
-------------[2] Tính chất bất đẳng thức
-------------[3] Bài Toán Thực Tế
-------------[0] Chưa phân dạng
----------[2] BẤT PHƯƠNG TRÌNH BẬC NHẤT MỘT ẨN
-------------[1] Bất phương trình bậc nhất một ẩn và nghiệm của nó
-------------[2] Cách giải bất phương trình bậc nhất một ẩn
-------------[3] Bài Toán Thực Tế
-------------[0] Chưa phân dạng
----------[0] Chưa phân dạng
-------------[0] Chưa phân dạng
-------[3] CĂN THỨC
----------[1] CĂN BẬC HAI
-------------[1] Căn bậc hai
-------------[2] Tính căn bậc hai bằng máy tính cầm tay
-------------[3] Căn thức bậc hai
-------------[4] Bài toán thực tế
-------------[0] Chưa phân dạng
----------[2] CĂN BẬC BA
-------------[1] Căn bậc ba của một số
-------------[2] Tính căn bậc ba bằng máy tính cầm tay
-------------[3] Căn thức bậc 3
-------------[4] Bài toán thực tế
-------------[0] Chưa phân dạng
----------[3] TÍNH CHẤT CỦA PHÉP KHAI PHƯƠNG
-------------[1] Căn thức bậc hai của một bình phương
-------------[2] Căn thức bậc hai của một tích
-------------[3] Căn thức bậc hai của một thương
-------------[4] Bài toán thực tế
-------------[0] Chưa phân dạng
----------[4] BIẾN ĐỔI ĐƠN GIẢN BIỂU THỨC CHỨA CĂN BẬC HAI
-------------[1] Trục căn thức ở mẫu
-------------[2] Rút gọn biểu thức chứa căn bậc hai
-------------[3] Bài toán thực tế
-------------[0] Chưa phân dạng
----------[0] Chưa phân dạng
-------------[0] Chưa phân dạng
-------[4] HÀM SỐ y=Ax^2 VÀ PHƯƠNG TRÌNH BẬC HAI MỘT ẨN
----------[1] HÀM SỐ y=Ax^2
-------------[1] Hàm số y=Ax^2
-------------[2] Bảng giá trị của hàm số y=Ax^2
-------------[3] Đồ thị của hàm số y=Ax^2
-------------[4] Bài toán thực tế
-------------[0] Chưa phân dạng
----------[2] PHƯƠNG TRÌNH BẬC 2 MỘT ẨN
-------------[1] Phương trình bậc hai một ẩn
-------------[2] Giải phương trình bậc hai bằng cách đưa về phương trình tích
-------------[3] Công thức nghiệm của phương trình bậc 2
-------------[4] Tìm nghiệm của phương trình bậc 2 bằng máy tính cầm tay
-------------[5] Giải bài toán bằng cách lập phương trình
-------------[0] Chưa phân dạng
----------[3] ĐỊNH LÝ VIÈTET
-------------[1] Định lí Vietet
-------------[2] Tìm hai số khi biết tổng và tích của chúng
-------------[3] Bài toán thực tế
-------------[0] Chưa phân dạng
----------[0] Chưa phân dạng
-------------[0] Chưa phân dạng
-------[5] MỘT SỐ YẾU THỐNG KÊ
----------[1] BẢNG TẦN SỐ VÀ BIỂU ĐỒ TẦN SỐ
-------------[1] Tần số và bảng tần số
-------------[2] Biểu đồ tần số
-------------[0] Chưa phân dạng
----------[2] BẢNG TẦN SỐ TƯƠNG ĐỐI VÀ BIỂU ĐỒ TẦN SỐ TƯƠNG ĐỐI
-------------[1] Bảng tần số tương đối
-------------[2] Biểu đồ tần số tương đối
-------------[0] Chưa phân dạng
----------[3] BIỂU DIỄN SỐ LIỆU GHÉP NHÓM
-------------[1] Bảng tần số ghép nhóm
-------------[2] Bảng tần số tương đối ghép nhóm
-------------[3] Biểu đồ tần số tương đối ghép nhóm
-------------[0] Chưa phân dạng
----------[0] Chưa phân dạng
-------------[0] Chưa phân dạng
-------[6] MỘT SỐ YẾU TỐ XÁC SUẤT
----------[1] KHÔNG GIAN MẪU VÀ BIẾN CỐ
-------------[1] Không gian mẫu
-------------[2] Biến cố
-------------[0] Chưa phân dạng
----------[2] XÁC SUẤT CỦA BIẾN CỐ
-------------[1] Kết quả đồng khả năng
-------------[2] Xác suất của biến cố
-------------[0] Chưa phân dạng
----------[3] BIỂU DIỄN SỐ LIỆU GHÉP NHÓM
-------------[1] Bảng tần số ghép nhóm
-------------[2] Bảng tần số tương đối ghép nhóm
-------------[3] Biểu đồ tần số tương đối ghép nhóm
-------------[0] Chưa phân dạng
----------[0] Chưa phân dạng
-------------[0] Chưa phân dạng
-------[7] MỘT SỐ BÀI TOÁN THỰC TẾ TRONG THI TUYỂN SINH
----------[1] Bài toán thực tế
-------------[1] Bài Toán Can Chi
-------------[2] Bài Toán tính ngày này năm sau, năm trước là thứ mấy
-------------[3] Bài Toán tính múi giờ
-------------[4] Bài Toán tính chỉ số bằng phép cộng, trừ số nguyên
-------------[5] Bài Toán thiết lập đồ thị hàm số bậc nhất
-------------[6] Bài Toán thiết lập đồ thị hàm số bậc hai
-------------[7] Bài Toán ngân hàng, lãi suất
-------------[8] Bài Toán tối ưu (max min, ít nhất, nhiều nhất)
-------[8] HỆ THỨC LƯỢNG TRONG TAM GIÁC VUÔNG
----------[1] TỈ SỐ LƯỢNG GIÁC CỦA GÓC NHỌN
-------------[1] Định nghĩa tỉ số lượng giác của góc nhọn
-------------[2] Tỉ số lượng giác của hai góc phụ nhau
-------------[3] Tính tỉ số lượng giác của góc nhọn bằng máy tính cầm tay
-------------[4] Bài toán thực tế
-------------[0] Chưa phân dạng
----------[2] HỆ THỨC GIỮA CẠNH VÀ GÓC CỦA TAM GIÁC VUÔNG
-------------[1] Hệ thức giữa cạnh và góc của tam giác vuông
-------------[2] Giải tam giác vuông
-------------[3] Bài toán thực tế
-------------[0] Chưa phân dạng
----------[0] Chưa phân dạng
-------------[0] Chưa phân dạng
-------[9] ĐƯỜNG TRÒN
----------[1] ĐƯỜNG TRÒN
-------------[1] Khái niệm đường tròn
-------------[2] Tính chất đối xứng của đường tròn
-------------[3] Đường kính và dây cung
-------------[4] Vị trí tương đối của hai đường tròn
-------------[5] Bài toán thực tế
-------------[0] Chưa phân dạng
----------[2] TIẾP TUYẾN CỦA ĐƯỜNG TRÒN
-------------[1] Vị trí tương đối của đường thẳng và đường tròn
-------------[2] Dấu hiệu nhận biết tiếp tuyến của đường tròn
-------------[3] Tính chất hai tiếp tuyến cắt nhau
-------------[4] Bài toán thực tế
-------------[0] Chưa phân dạng
----------[3] GÓC Ở TÂM, GÓC NỘI TIẾP
-------------[1] Góc ở tâm
-------------[2] Cung, số đo cung
-------------[3] Góc nội tiếp
-------------[4] Bài toán thực tế
-------------[0] Chưa phân dạng
----------[4] HÌNH QUẠT TRÒN VÀ HÌNH VÀNH KHUYÊN
-------------[1] Độ dài cung tròn
-------------[2] Hình quạt tròn
-------------[3] Hình vành khuyên
-------------[4] Bài toán thực tế
-------------[0] Chưa phân dạng
----------[0] Chưa phân dạng
-------------[0] Chưa phân dạng
-------[3] TỨ GIÁC NỘI TIẾP. ĐA GIÁC ĐỀU
----------[1] ĐƯỜNG TRÒN NGOẠI TIẾP TAM GIÁC. ĐƯỜNG TRÒN NỘI TIẾP TAM GIÁC
-------------[1] Đường tròn ngoại tiếp tam giác
-------------[2] Đường tròn nội tiếp tam giác
-------------[3] Bài toán thực tế
-------------[0] Chưa phân dạng
----------[2] TỨ GIÁC NỘI TIẾP
-------------[1] Định nghĩa tứ giác nội tiếp
-------------[2] Tính chất tứ giác nội tiếp
-------------[3] Đường tròn ngoại tiếp hình chữ nhật, hình vuông
-------------[4] Bài toán thực tế
-------------[0] Chưa phân dạng
----------[3] ĐA GIÁC ĐỀU VÀ PHÉP QUAY
-------------[1] Khái niệm đa giác đều
-------------[2] Phép quay
-------------[3] Hình phẳng đều trong thực tế
-------------[0] Chưa phân dạng
----------[4] BÀI HÌNH NHIỀU CÂU
-------------[1] Bài toán từ một điểm bên ngoài, kẻ hai tiếp tuyến đến (O)
-------------[2] Bài toán nửa đường tròn
-------------[3] Bài toán tam giác ABC có một cạnh là đường kính
-------------[4] Bài toán tam giác ABC nhọn nội tiếp đường tròn (O)
-------------[5] Bài toán dựng thêm đường tròn thứ hai khác (O)
-------------[6] Bài toán kẻ 2 dây cung vuông góc
-------------[7] Bài toán max-min
-------------[8] Bài toán khác
-------------[0] Chưa phân dạng
----------[5] BÀI HÌNH LIÊN QUAN CÁC HÌNH KHÁC
-------------[1] Bài toán kẻ 2 đường kính (hình chữ nhật)
-------------[2] Bài toán hình vuông
-------------[3] Bài toán hình tam giác đều
-------------[4] Bài toán hình tam giác cân
-------------[5] Bài toán hình tứ giác nội tiếp (cho 4 điểm thuộc đường tròn)
-------------[6] Bài toán hình thang
-------------[7] Bài toán hình bình hành
-------------[8] Bài toán max-min
-------------[0] Chưa phân dạng
----------[6] BÀI HÌNH LIÊN QUAN TIẾP TUYẾN CHUNG
-------------[1] Hai đường tròn tiếp xúc ngoài
-------------[2] Hai đường tròn tiếp xúc trong
-------------[3] Hai đường tròn ngoài nhau
-------------[4] Hai đường tròn đựng nhau
-------------[5] Hai đường tròn cắt nhau
-------------[0] Chưa phân dạng
----------[0] Chưa phân dạng
-------------[0] Chưa phân dạng
-------[A] CÁC HÌNH KHỐI TRONG THỰC TIỄN
----------[1] HÌNH TRỤ
-------------[1] Hình trụ
-------------[2] Diện tích xung quanh hình trụ
-------------[3] Thể tích của hình trụ
-------------[4] Bài toán thực tế
-------------[0] Chưa phân dạng
----------[2] HÌNH NÓN
-------------[1] Hình nón
-------------[2] Diện tích xung quanh của hình nón
-------------[3] Thể tích của hình nón
-------------[4] Bài toán thực tế
-------------[0] Chưa phân dạng
----------[3] HÌNH CẦU
-------------[1] Hình cầu
-------------[2] Diện tích của mặt cầu
-------------[3] Thể tích của hình cầu
-------------[4] Bài toán thực tế
-------------[0] Chưa phân dạng
----------[1] CÁC HÌNH KHÔNG GIAN KẾT HỢP
-------------[1] Hình trụ và hình nón
-------------[2] Hình trụ và hình cầu
-------------[3] Hình cầu và hình nón
-------------[4] Các hình khác (lăng trụ, hình hộp)
-------------[0] Chưa phân dạng
----------[0] Chưa phân dạng
-------------[0] Chưa phân dạng
-------[0] Chưa phân dạng
----------[0] Chưa phân dạng
-------------[0] Chưa phân dạng
%==============================================================
----[D] Đại số
-------[1] PHƯƠNG TRÌNH VÀ HỆ PHƯƠNG TRÌNH
----------[1] PHƯƠNG TRÌNH QUY VỀ PHƯƠNG TRÌNH BẬC NHẤT
-------------[1] Phương trình tích
-------------[2] Phương trình chứa ẩn ở mẫu quy về phương trình bậc nhất
-------------[3] Giải bài toán bằng cách lập phương trình
-------------[0] Chưa phân dạng
----------[2] PHƯƠNG TRÌNH BẬC NHẤT 2 ẨN VÀ HỆ PHƯƠNG TRÌNH BẬC NHẤT 2 ẨN
-------------[1] Phương trình bậc nhất hai ẩn
-------------[2] Hệ phương trình bậc nhất hai ẩn
-------------[3] Bài Toán Thực Tế
-------------[0] Chưa phân dạng
----------[3] GIẢI HỆ PHƯƠNG TRÌNH BẬC NHẤT 2 ẨN
-------------[1] Giải hệ phương trình bằng phương pháp thế
-------------[2] Giải hệ phương trình bằng phương pháp cộng đại số
-------------[3] Tìm nghiệm của hệ bằng máy tính cầm tay
-------------[4] Giải bài toán bằng cách lập hệ phương trình
-------------[0] Chưa phân dạng
----------[0] Chưa phân dạng
-------------[0] Chưa phân dạng
-------[2] BẤT ĐẲNG THỨC. BẤT PHƯƠNG TRÌNH BẬC NHẤT MỘT ẨN
----------[1] BẤT ĐẲNG THỨC
-------------[1] Khái niệm bất đẳng thức
-------------[2] Tính chất bất đẳng thức
-------------[3] Bài Toán Thực Tế
-------------[0] Chưa phân dạng
----------[2] BẤT PHƯƠNG TRÌNH BẬC NHẤT MỘT ẨN
-------------[1] Bất phương trình bậc nhất một ẩn và nghiệm của nó
-------------[2] Cách giải bất phương trình bậc nhất một ẩn
-------------[3] Bài Toán Thực Tế
-------------[0] Chưa phân dạng
----------[0] Chưa phân dạng
-------------[0] Chưa phân dạng
-------[3] CĂN THỨC
----------[1] CĂN BẬC HAI
-------------[1] Căn bậc hai
-------------[2] Tính căn bậc hai bằng máy tính cầm tay
-------------[3] Căn thức bậc hai
-------------[4] Bài toán thực tế
-------------[0] Chưa phân dạng
----------[2] CĂN BẬC BA
-------------[1] Căn bậc ba của một số
-------------[2] Tính căn bậc ba bằng máy tính cầm tay
-------------[3] Căn thức bậc 3
-------------[4] Bài toán thực tế
-------------[0] Chưa phân dạng
----------[3] TÍNH CHẤT CỦA PHÉP KHAI PHƯƠNG
-------------[1] Căn thức bậc hai của một bình phương
-------------[2] Căn thức bậc hai của một tích
-------------[3] Căn thức bậc hai của một thương
-------------[4] Bài toán thực tế
-------------[0] Chưa phân dạng
----------[4] BIẾN ĐỔI ĐƠN GIẢN BIỂU THỨC CHỨA CĂN BẬC HAI
-------------[1] Trục căn thức ở mẫu
-------------[2] Rút gọn biểu thức chứa căn bậc hai
-------------[3] Bài toán thực tế
-------------[0] Chưa phân dạng
----------[0] Chưa phân dạng
-------------[0] Chưa phân dạng
-------[4] HÀM SỐ y=Ax^2 VÀ PHƯƠNG TRÌNH BẬC HAI MỘT ẨN
----------[1] HÀM SỐ y=Ax^2
-------------[1] Hàm số y=Ax^2
-------------[2] Bảng giá trị của hàm số y=Ax^2
-------------[3] Đồ thị của hàm số y=Ax^2
-------------[4] Bài toán thực tế
-------------[0] Chưa phân dạng
----------[2] PHƯƠNG TRÌNH BẬC 2 MỘT ẨN
-------------[1] Phương trình bậc hai một ẩn
-------------[2] Giải phương trình bậc hai bằng cách đưa về phương trình tích
-------------[3] Công thức nghiệm của phương trình bậc 2
-------------[4] Tìm nghiệm của phương trình bậc 2 bằng máy tính cầm tay
-------------[5] Giải bài toán bằng cách lập phương trình
-------------[0] Chưa phân dạng
----------[3] ĐỊNH LÝ VIÈTET
-------------[1] Định lí Vietet
-------------[2] Tìm hai số khi biết tổng và tích của chúng
-------------[3] Bài toán thực tế
-------------[0] Chưa phân dạng
----------[0] Chưa phân dạng
-------------[0] Chưa phân dạng
-------[5] MỘT SỐ YẾU THỐNG KÊ
----------[1] BẢNG TẦN SỐ VÀ BIỂU ĐỒ TẦN SỐ
-------------[1] Tần số và bảng tần số
-------------[2] Biểu đồ tần số
-------------[0] Chưa phân dạng
----------[2] BẢNG TẦN SỐ TƯƠNG ĐỐI VÀ BIỂU ĐỒ TẦN SỐ TƯƠNG ĐỐI
-------------[1] Bảng tần số tương đối
-------------[2] Biểu đồ tần số tương đối
-------------[0] Chưa phân dạng
----------[3] BIỂU DIỄN SỐ LIỆU GHÉP NHÓM
-------------[1] Bảng tần số ghép nhóm
-------------[2] Bảng tần số tương đối ghép nhóm
-------------[3] Biểu đồ tần số tương đối ghép nhóm
-------------[0] Chưa phân dạng
----------[0] Chưa phân dạng
-------------[0] Chưa phân dạng
-------[6] MỘT SỐ YẾU TỐ XÁC SUẤT
----------[1] KHÔNG GIAN MẪU VÀ BIẾN CỐ
-------------[1] Không gian mẫu
-------------[2] Biến cố
-------------[0] Chưa phân dạng
----------[2] XÁC SUẤT CỦA BIẾN CỐ
-------------[1] Kết quả đồng khả năng
-------------[2] Xác suất của biến cố
-------------[0] Chưa phân dạng
----------[3] BIỂU DIỄN SỐ LIỆU GHÉP NHÓM
-------------[1] Bảng tần số ghép nhóm
-------------[2] Bảng tần số tương đối ghép nhóm
-------------[3] Biểu đồ tần số tương đối ghép nhóm
-------------[0] Chưa phân dạng
----------[0] Chưa phân dạng
-------------[0] Chưa phân dạng
-------[7] MỘT SỐ BÀI TOÁN THỰC TẾ TRONG THI TUYỂN SINH
----------[1] Bài toán thực tế
-------------[1] Bài Toán Can Chi
-------------[2] Bài Toán tính ngày này năm sau, năm trước là thứ mấy
-------------[3] Bài Toán tính múi giờ
-------------[4] Bài Toán tính chỉ số bằng phép cộng, trừ số nguyên
-------------[5] Bài Toán thiết lập đồ thị hàm số bậc nhất
-------------[6] Bài Toán thiết lập đồ thị hàm số bậc hai
-------------[7] Bài Toán ngân hàng, lãi suất
-------------[8] Bài Toán tối ưu (max min, ít nhất, nhiều nhất)
-------------[0] Chưa phân dạng
-------[0] Một số chủ đề Đại số, Số học bồi dưỡng HSG lớp 9
----------[1] Biểu thức đại số
-------------[0] Chưa phân dạng
----------[2] PT, BPT. Hệ phương trình, hệ BPT
-------------[0] Chưa phân dạng
----------[3] Đa thức
-------------[0] Chưa phân dạng
----------[4] Số học (tính chia hết, ƯCLN, BCNN, số nguyên tố, hợp số, số chính phương, phần nguyên, \ldots)
-------------[0] Chưa phân dạng
----------[5] PT nghiệm nguyên
-------------[0] Chưa phân dạng
----------[6] Đồng dư
-------------[0] Chưa phân dạng
----------[7] Bất đẳng thức, giá trị lớn nhất, giá trị nhỏ nhất
-------------[0] Chưa phân dạng
----------[8] Suy luận toán học; các bài toán về trò chơi; đại số tổ hợp
-------------[0] Chưa phân dạng
----------[0] Chưa phân dạng
-------------[0] Chưa phân dạng
-------[0] Chưa phân dạng
----------[0] Chưa phân dạng
-------------[0] Chưa phân dạng
----[H] Hình học
-------[1] HỆ THỨC LƯỢNG TRONG TAM GIÁC VUÔNG
----------[1] TỈ SỐ LƯỢNG GIÁC CỦA GÓC NHỌN
-------------[1] Định nghĩa tỉ số lượng giác của góc nhọn
-------------[2] Tỉ số lượng giác của hai góc phụ nhau
-------------[3] Tính tỉ số lượng giác của góc nhọn bằng máy tính cầm tay
-------------[4] Bài toán thực tế
-------------[0] Chưa phân dạng
----------[2] HỆ THỨC GIỮA CẠNH VÀ GÓC CỦA TAM GIÁC VUÔNG
-------------[1] Hệ thức giữa cạnh và góc của tam giác vuông
-------------[2] Giải tam giác vuông
-------------[3] Bài toán thực tế
-------------[0] Chưa phân dạng
----------[0] Chưa phân dạng
-------------[0] Chưa phân dạng
-------[2] ĐƯỜNG TRÒN
----------[1] ĐƯỜNG TRÒN
-------------[1] Khái niệm đường tròn
-------------[2] Tính chất đối xứng của đường tròn
-------------[3] Đường kính và dây cung
-------------[4] Vị trí tương đối của hai đường tròn
-------------[5] Bài toán thực tế
-------------[0] Chưa phân dạng
----------[2] TIẾP TUYẾN CỦA ĐƯỜNG TRÒN
-------------[1] Vị trí tương đối của đường thẳng và đường tròn
-------------[2] Dấu hiệu nhận biết tiếp tuyến của đường tròn
-------------[3] Tính chất hai tiếp tuyến cắt nhau
-------------[4] Bài toán thực tế
-------------[0] Chưa phân dạng
----------[3] GÓC Ở TÂM, GÓC NỘI TIẾP
-------------[1] Góc ở tâm
-------------[2] Cung, số đo cung
-------------[3] Góc nội tiếp
-------------[4] Bài toán thực tế
-------------[0] Chưa phân dạng
----------[4] HÌNH QUẠT TRÒN VÀ HÌNH VÀNH KHUYÊN
-------------[1] Độ dài cung tròn
-------------[2] Hình quạt tròn
-------------[3] Hình vành khuyên
-------------[4] Bài toán thực tế
-------------[0] Chưa phân dạng
----------[0] Chưa phân dạng
-------------[0] Chưa phân dạng
-------[3] TỨ GIÁC NỘI TIẾP. ĐA GIÁC ĐỀU
----------[1] ĐƯỜNG TRÒN NGOẠI TIẾP TAM GIÁC. ĐƯỜNG TRÒN NỘI TIẾP TAM GIÁC
-------------[1] Đường tròn ngoại tiếp tam giác
-------------[2] Đường tròn nội tiếp tam giác
-------------[3] Bài toán thực tế
-------------[0] Chưa phân dạng
----------[2] TỨ GIÁC NỘI TIẾP
-------------[1] Định nghĩa tứ giác nội tiếp
-------------[2] Tính chất tứ giác nội tiếp
-------------[3] Đường tròn ngoại tiếp hình chữ nhật, hình vuông
-------------[4] Bài toán thực tế
-------------[0] Chưa phân dạng
----------[3] ĐA GIÁC ĐỀU VÀ PHÉP QUAY
-------------[1] Khái niệm đa giác đều
-------------[2] Phép quay
-------------[3] Hình phẳng đều trong thực tế
-------------[0] Chưa phân dạng
----------[4] BÀI HÌNH NHIỀU CÂU
-------------[1] Bài toán từ một điểm bên ngoài, kẻ hai tiếp tuyến đến (O)
-------------[2] Bài toán nửa đường tròn
-------------[3] Bài toán tam giác ABC có một cạnh là đường kính
-------------[4] Bài toán tam giác ABC nhọn nội tiếp đường tròn (O)
-------------[5] Bài toán dựng thêm đường tròn thứ hai khác (O)
-------------[6] Bài toán kẻ 2 dây cung vuông góc
-------------[7] Bài toán max-min
-------------[8] Bài toán khác
-------------[0] Chưa phân dạng
----------[5] BÀI HÌNH LIÊN QUAN CÁC HÌNH KHÁC
-------------[1] Bài toán kẻ 2 đường kính (hình chữ nhật)
-------------[2] Bài toán hình vuông
-------------[3] Bài toán hình tam giác đều
-------------[4] Bài toán hình tam giác cân
-------------[5] Bài toán hình tứ giác nội tiếp (cho 4 điểm thuộc đường tròn)
-------------[6] Bài toán hình thang
-------------[7] Bài toán hình bình hành
-------------[8] Bài toán max-min
-------------[0] Chưa phân dạng
----------[6] BÀI HÌNH LIÊN QUAN TIẾP TUYẾN CHUNG
-------------[1] Hai đường tròn tiếp xúc ngoài
-------------[2] Hai đường tròn tiếp xúc trong
-------------[3] Hai đường tròn ngoài nhau
-------------[4] Hai đường tròn đựng nhau
-------------[5] Hai đường tròn cắt nhau
-------------[0] Chưa phân dạng
----------[0] Chưa phân dạng
-------------[0] Chưa phân dạng
-------[4] CÁC HÌNH KHỐI TRONG THỰC TIỄN
----------[1] HÌNH TRỤ
-------------[1] Hình trụ
-------------[2] Diện tích xung quanh hình trụ
-------------[3] Thể tích của hình trụ
-------------[4] Bài toán thực tế
-------------[0] Chưa phân dạng
----------[2] HÌNH NÓN
-------------[1] Hình nón
-------------[2] Diện tích xung quanh của hình nón
-------------[3] Thể tích của hình nón
-------------[4] Bài toán thực tế
-------------[0] Chưa phân dạng
----------[3] HÌNH CẦU
-------------[1] Hình cầu
-------------[2] Diện tích của mặt cầu
-------------[3] Thể tích của hình cầu
-------------[4] Bài toán thực tế
-------------[0] Chưa phân dạng
----------[4] CÁC HÌNH KHÔNG GIAN KẾT HỢP
-------------[1] Hình trụ và hình nón
-------------[2] Hình trụ và hình cầu
-------------[3] Hình cầu và hình nón
-------------[4] Các hình khác (lăng trụ, hình hộp)
-------------[0] Chưa phân dạng
----------[0] Chưa phân dạng
-------------[0] Chưa phân dạng
-------[0] Một số chủ đề Hình học bồi dưỡng HSG lớp 9
----------[0] Chưa phân dạng
-------------[0] Chưa phân dạng
----------[1] Hình học cổ điển
-------------[0] Chưa phân dạng
-------------[1] Nội tiếp
-------------[2] Ngoại tiếp
-------------[3] Trực tâm
-------------[4] Cát tuyến
-------------[5] Tiếp tuyến
----------[2] Chứng minh đẳng thức
-------------[0] Chưa phân dạng
----------[3] Chứng minh quan hệ song song, vuông góc
-------------[0] Chưa phân dạng
----------[4] Các bài toán định lượng
-------------[0] Chưa phân dạng
----------[5] Các bài toán sử dụng các định lí hình học cổ điển (Ceva, Menelaus, Ptolemy, ...)
-------------[0] Chưa phân dạng
----------[6] Chứng minh ba điểm thẳng hàng, ba ĐT đồng quy
-------------[0] Chưa phân dạng
----------[7] Yếu tố cố định
-------------[0] Chưa phân dạng
----------[8] Bất đẳng thức, GTLN, GTNN
-------------[0] Chưa phân dạng
----------[9] Hình học tổ hợp
-------------[0] Chưa phân dạng
-------[0] Chưa phân dạng
----------[0] Chưa phân dạng
-------------[0] Chưa phân dạng
%==============================================================
----[G] 9-HSG
-------[1] Một số chủ đề Đại số, Số học bồi dưỡng HSG lớp 9
----------[1] Biểu thức đại số
-------------[0] Chưa phân dạng
----------[2] PT, BPT. Hệ phương trình, hệ BPT
-------------[0] Chưa phân dạng
----------[3] Đa thức
-------------[0] Chưa phân dạng
----------[4] Số học (tính chia hết, ƯCLN, BCNN, số nguyên tố, hợp số, số chính phương, phần nguyên, \ldots)
-------------[0] Chưa phân dạng
----------[5] PT nghiệm nguyên
-------------[0] Chưa phân dạng
----------[6] Đồng dư
-------------[0] Chưa phân dạng
----------[7] Bất đẳng thức, giá trị lớn nhất, giá trị nhỏ nhất
-------------[0] Chưa phân dạng
----------[8] Suy luận toán học; các bài toán về trò chơi; đại số tổ hợp
-------------[0] Chưa phân dạng
----------[0] Chưa phân dạng
-------------[0] Chưa phân dạng
-------[2] Một số chủ đề Hình học bồi dưỡng HSG lớp 9
----------[1] Hình học cổ điển
-------------[0] Chưa phân dạng
-------------[1] Nội tiếp
-------------[2] Ngoại tiếp
-------------[3] Trực tâm
-------------[4] Cát tuyến
-------------[5] Tiếp tuyến
----------[2] Chứng minh đẳng thức
-------------[0] Chưa phân dạng
----------[3] Chứng minh quan hệ song song, vuông góc
-------------[0] Chưa phân dạng
----------[4] Các bài toán định lượng
-------------[0] Chưa phân dạng
----------[5] Các bài toán sử dụng các định lí hình học cổ điển (Ceva, Menelaus, Ptolemy, ...)
-------------[0] Chưa phân dạng
----------[6] Chứng minh ba điểm thẳng hàng, ba ĐT đồng quy
-------------[0] Chưa phân dạng
----------[7] Yếu tố cố định
-------------[0] Chưa phân dạng
----------[8] Bất đẳng thức, GTLN, GTNN
-------------[0] Chưa phân dạng
----------[9] Hình học tổ hợp
-------------[0] Chưa phân dạng
----------[0] Chưa phân dạng
-------------[0] Chưa phân dạng
-------[0] Chưa phân dạng
----------[0] Chưa phân dạng
-------------[0] Chưa phân dạng
----[0] Chưa phân dạng
-------[0] Chưa phân dạng
----------[0] Chưa phân dạng
-------------[0] Chưa phân dạng
%Kết thúc nội dung ID