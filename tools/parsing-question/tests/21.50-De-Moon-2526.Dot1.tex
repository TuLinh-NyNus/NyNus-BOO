
\begin{ex}%[Nguồn: Bộ đề minh họa Moon 2024-2025]%[2P6V3-5]%
	Trong không gian với hệ tọa độ $Oxyz$, cho điểm $E(1;1;1)$, mặt cầu $(S)\colon x^2 + y^2 + z^2 = 4$ và mặt phẳng $(P)\colon x - 3y + 5z - 3 = 0$. Gọi $\Delta$ là đường thẳng đi qua $E$, nằm trong $(P)$ và cắt mặt cầu $(S)$ tại hai điểm $A$, $B$ sao cho tam giác $OAB$ là tam giác đều. Đường thẳng $\Delta$ có một vectơ chỉ phương là $(a; b; 10)$. Tính $a^2 + b^2$.
	\shortans[oly]{$6500$}
	\loigiai{
	\begin{itemize}
		\item Mặt phẳng $(P)$ có VTPT là $\overrightarrow{n}=(1;-3;5)$. \\
		Đường thẳng $\Delta$ có VTCP là $\overrightarrow{a}=(a;b;10)$.\\
		Đường thẳng $\Delta$ nằm trong $(P)$ nên $\overrightarrow{n}\cdot\overrightarrow{a}=0\Leftrightarrow a-3b+50=0\Leftrightarrow a=3b-50$.\\
		Hay $\overrightarrow{a}=(3b-50;b;10)$.
		\item Mặt cầu $(S)$ có tâm $O(0;0;0)$ và bán kính $R=2$.\\
		Gọi $M$ là trung điểm của $AB$, do $\triangle OAB$ là tam giác đều nên $OM=\dfrac{OA\sqrt{3}}{2}=\sqrt{3}$.\\
		Gọi $d$ là khoảng cách từ tâm $O$ đến $\Delta$ thì $d=OM=\sqrt{3}$.
		\item Mặt khác, $d=\dfrac{\big|[\overrightarrow{a},\overrightarrow{OE}]\big|}{\big|\overrightarrow{a}\big|}=\sqrt{\dfrac{14b^2-780b+6200}{10b^2-300b+2600}}$.
		\item Do đó
		\allowdisplaybreaks
		\begin{eqnarray*}
			&&\sqrt{\dfrac{14b^2-580b+6200}{10b^2-300b+2600}}=\sqrt{3}\Leftrightarrow\dfrac{14b^2-780b+6200}{10b^2-300b+2600}=3\\&\Leftrightarrow&14b^2-580b+6200=3(10b^2-300b+2600)\\
			&\Leftrightarrow&16b^2-320b+1600=0\Leftrightarrow b=-10.
		\end{eqnarray*}
		Khi đó $a=-80$.\\
		Vậy $a^2+b^2=(-80)^2+(-10)^2=6500$.
		
	\end{itemize}
	}
\end{ex}

\begin{ex}%[Nguồn: Bộ đề minh họa Moon 2024-2025]%[1C2H3-1]
	Có $4$ người và $1$ đèn pin muốn qua sông phải đi qua một cây cầu. Biết cây cầu chỉ đi $1$ lần tối đa được $2$ người và phải có đèn pin mới có thể di chuyển trên cầu. $A$ đi qua cầu hết $1$ phút, $B$ hết $2$ phút, $C$ hết $5$ phút, $D$ hết $10$ phút. Hai người đi cùng nhau thì phải đi với tốc độ của người đi chậm hơn. Hỏi mất ít nhất bao nhiêu phút để tất cả đều qua được sông?
	\shortans[]{$17$}
	\loigiai{
	Để tối ưu hóa thời gian ta qua cầu sẽ có xu hướng xếp hai người đi chậm nhất là $C$ và $D$ cùng qua cầu một lượt. Ta có cách sắp xếp sau
	\begin{itemize}
		\item Lượt 1. $A$ và $B$ cùng qua cầu hết $2$ phút.
		\item Lượt 2. $A$ quay lại hết $1$ phút.
		\item Lượt 3. $C$ và $D$ cùng qua cầu hết $10$ phút.
		\item Lượt 4. $B$ quay lại hết $2$ phút.
		\item Lượt 5. $A$ và $B$ cùng qua cầu hết $2$ phút.
	\end{itemize}
	Vậy tổng thời gian ngắn nhất để cả bốn người qua cầu là $17$ phút.
	}
\end{ex}

\begin{ex}%[Nguồn: Bộ đề minh họa Moon 2024-2025]%[2P1V7-9]
	[TL.309840]
	Theo thống kê tại một nhà máy Z, nếu áp dụng tuần làm việc $40$ giờ thì mỗi tuần có $100$ công nhân đi làm và mỗi công nhân làm được $120$ sản phẩm trong một giờ. Nếu tăng thời gian làm việc thêm $2$ giờ mỗi tuần thì sẽ có $1$ công nhân nghỉ việc và năng suất lao động giảm đi $5$ sản phẩm/$1$ công nhân/$1$ giờ (và như vậy, nếu giảm thời gian làm việc $2$ giờ mỗi tuần thì sẽ có thêm $1$ công nhân đi làm đồng thời năng suất lao động tăng $5$ sản phẩm/$1$ công nhân /$1$ giờ). Ngoài ra, số phế phẩm mỗi tuần ước tính là $P(x) = \dfrac{95x^2+120x}{4}$, với $x$ là thời gian làm việc trong $1$ tuần. Nhà máy cần áp dụng thời gian làm việc mỗi tuần mấy giờ để số lượng sản phẩm thu được mỗi tuần đạt cực đại?
	\shortans[oly]{$36$}
	\loigiai{
	Với $x$ là thời gian làm việc trong một tuần, theo bài toán ta có $x\in[0;42]$ và số lượng sản phẩm thu được mỗi tuần của nhà máy $Z$ được tính theo công thức
	\begin{align*}
		S(x) & =x\cdot\left(100-\dfrac{x-40}{2}\right)\cdot\left(120-5\cdot\dfrac{x-40}{2}\right)-\dfrac{95x^2+120x}{4} \\
		& =\dfrac{5x^3-1735x^2+105480x}{4}.
	\end{align*}
	Ta có $S'(x)=\dfrac{15x^2-3470x+105480}{4}\Rightarrow S'(x)=0\Leftrightarrow \hoac{x&=36\\x&=\dfrac{586}{19}.\ (\text{không thỏa mãn})}$\\
	Bảng biến thiên của hàm số
	\begin{center}
		\begin{tikzpicture}[scale=1, font=\footnotesize, line join=round, line cap=round, >=stealth]
			\tkzTabInit[nocadre=false,lgt=1.2,espcl=2.5,deltacl=0.6]
			{$x$ /0.6,$S’(x)$ /0.6,$S(x)$ /2}
			{$0$,$36$,$42$}
			\tkzTabLine{,+,0,-,}
			\tkzTabVar{-/$0$,+/$445500$,-/$435015$}
		\end{tikzpicture}
	\end{center}
	\noindent Dựa vào bảng biến thiên, vậy thời gian làm việc trong một tuần để đạt số lượng sản phẩm đạt cực đại là $36$ giờ.
	}
\end{ex}

\begin{ex}%[Nguồn: Bộ đề minh họa Moon 2024-2025]%[0D0H2-5]
	Mỗi hộp đựng $12$ bóng đèn, các bóng đèn trong cùng hộp thì cùng màu. Số hộp đựng bóng đèn màu xanh nhiều gấp $9$ lần số hộp đựng bóng đèn màu vàng. Trong mỗi hộp đựng bóng đèn màu xanh có $3$ bóng bị hỏng, mỗi hộp đựng bóng đèn màu vàng có $2$ bóng bị hỏng. Lấy ngẫu nhiên ra hai bóng đèn từ một hộp bất kì, tính xác xuất để lấy ra hai bóng đèn màu xanh, biết cả hai bóng đều bị hỏng (kết quả làm tròn đến hàng phần trăm).
	
	\shortans{$0{,}96$}
	\loigiai{
	Gọi $A$ à biến cố lấy được một hộp đựng bóng đèn màu vàng.\\
	Suy ra $\mathrm{P}(A)=\dfrac{1}{1+9}=\dfrac{1}{10}$.\\
	Gọi $B$ à biến cố lấy được một hộp đựng bóng đèn màu xanh.\\
	Suy ra $\mathrm{P}(B)=\dfrac{9}{1+9}=\dfrac{9}{10}$.\\
	Gọi $C$ là biến cố lấy được hai bóng đèn hỏng ở cùng $1$ hộp.\\
	Ta có xác suất lấy được $2$ bóng đèn hỏng từ một hộp đựng bóng đèn vàng là $\mathrm{P}(C\mid A)=\dfrac{\mathrm{C}_2^2}{\mathrm{C}_{12}^2} $.\\
	(vì trong mỗi hộp đựng bóng đèn vàng có $2$ bóng bị hỏng).\\
	Tương tự, vì trong mỗi hộp đựng bóng đèn màu xanh có $3$ bóng bị hỏng nên xác suất lấy được $2$ bóng đèn hỏng từ một hộp đựng bóng đèn xanh là
	$\mathrm{P}(C\mid B)=\dfrac{\mathrm{C}_3^2}{\mathrm{C}_{12}^2} $.\\
	Ta có sơ đồ cây sau
	\begin{center}
		\tikzstyle{rect} = [rectangle, draw, rounded corners, text centered, minimum width=2.5cm, minimum height=1cm]
		\tikzstyle{arrow} = [thick, ->, >=stealth]
		\begin{tikzpicture}[node distance=2cm and .5cm] % Increase vertical spacing
			
			% Root node
			\node[rect,fill=red!50] (root) {Gốc};
			
			% Level 1 nodes (adjusted heights)
			\node[rect,fill=yellow!30, right=of root, yshift=3cm] (pass) {Hộp đèn màu vàng};
			\node[rect,fill=blue!30, right=of root, yshift=-3cm] (fail) {Hộp đèn màu xanh};
			
			% Level 2 nodes (adjusted positions for symmetry)
			\node[rect,fill=blue!30, above right=1cm and 1.5cm of pass] (pass-recog) {Lấy được 2 bóng đèn hỏng ở cùng 1 hộp};
			\node[rect, fill=pink!30, below right=1cm and 1.5cm of pass] (pass-notrecog) {Lấy được 2 bóng đèn hỏng ở 2 hộp khác nhau};
			
			\node[rect,fill=blue!30, above right=1cm and 1.5cm of fail] (fail-recog) {Lấy được 2 bóng đèn hỏng ở cùng 1 hộp};
			\node[rect, fill=pink!30, below right=1cm and 1.5cm of fail] (fail-notrecog) {Lấy được 2 bóng đèn hỏng ở 2 hộp khác nhau};
			
			% Draw arrows with labels
			\draw[arrow] (root) -- node[pos=0.3, above] {A} node[pos=0.5, below,yshift=-5pt] {$1/10$} (pass);
			\draw[arrow] (root) -- node[pos=0.3, above] {$B$} node[pos=0.4, below,yshift=-5pt] {$9/10$} (fail);
			
			% Arrows for "Đạt chuẩn" subtree
			\draw[arrow] (pass) -- node[pos=0.5, left,yshift=5pt] {$C\mid A $} node[pos=0.7, below,yshift=-5pt] {$\dfrac{\mathrm{C}_2^2}{\mathrm{C}_{12}^2}$} (pass-recog);
			\draw[arrow] (pass) -- node[pos=0.5, above] {$\overline{C}\mid A $} node[pos=0.3, below] {} (pass-notrecog);
			
			% Arrows for "Không đạt chuẩn" subtree
			\draw[arrow] (fail) -- node[pos=0.5, left,yshift=5pt] {$C\mid B $} node[pos=0.7, below,yshift=-5pt] {$\dfrac{\mathrm{C}_3^2}{\mathrm{C}_{12}^2}$} (fail-recog);
			\draw[arrow] (fail) -- node[pos=0.5,above] {$\overline{C}\mid B$} node[pos=0.7, left,yshift=-5pt] {} (fail-notrecog);
			
		\end{tikzpicture}
	\end{center}
	Ta có
	\allowdisplaybreaks
	\begin{eqnarray*}
		\mathrm{P}(C)&=&\mathrm{P}(A)\cdot \mathrm{P}(C\mid A)+\mathrm{P}(B)\cdot \mathrm{P}(C\mid B)
		\\
		&=&\dfrac{1}{10}\cdot \dfrac{\mathrm{C}_2^2}{\mathrm{C}_{12}^2} +\dfrac{9}{10}\cdot \dfrac{\mathrm{C}_3^2}{\mathrm{C}_{12}^2}
		\\
		&=&\dfrac{7}{165}.
	\end{eqnarray*}
	Suy ra
	\allowdisplaybreaks
	\begin{eqnarray*}
		\mathrm{P}(B\mid C)&=&\dfrac{\mathrm{P}(B)\cdot \mathrm{P}(C\mid B)}{\mathrm{P}(C)}
		\\
		&=&\dfrac{\dfrac{9}{10}\cdot \dfrac{\mathrm{C}_3^2}{\mathrm{C}_{12}^2} }{\dfrac{7}{165}}
		\\
		&=&\dfrac{27}{28} \approx 0{,}96.
	\end{eqnarray*}
	}
\end{ex}

%

\begin{ex}%[Nguồn: Bộ đề minh họa Moon 2024-2025]%[0D2V2-3]
	Một công ty sản xuất hai loại thiết bị dạy học A và B dành cho môn Toán lớp 12. Mỗi sản phẩm loại A cần $9$ giờ lao động để gia công và $1$ giờ lao động để hoàn thiện. Mỗi sản phầm loại B cần $12$ giờ lao động để gia công và $3$ giờ lao động để hoàn thiện. Số giờ lao động tối đa có sẵn mỗi tuần cho gia công và hoàn thiện lần lượt là $180$\,giờ và $30$\,giờ. Công ty thu được lợi nhuận là $80\,000$\,đồng trên mỗi sản phẩm loại A và $120\,000$\,đồng trên mỗi sản phẩm loại B. Cần sản xuất $x$ sản phẩm loại A và $y$ sản phẩm loại B mỗi tuần để có được lợi nhuận tối đa. Lợi nhuận tối đa mỗi tuần là bao nhiêu nghìn đồng?
	\shortans{1680}
	\loigiai{
	Từ giả thiết, ta có hệ điều kiện $\heva{&x\ge 0\\&y\ge 0\\&9x+12y\le 180\\&x+3y\le 30.}$\\
	Hàm mục tiêu (nghìn đồng) theo tuần là $F(x,y)=80x+120y$.\\
	\begin{center}
		\begin{tikzpicture}[line cap=butt,line join=miter,>=stealth,xscale=0.5,yscale=0.5]
			\tikzset{declare function={xmin=-1;xmax=22;
			ymin=-1;ymax=12;}}
			
			\fill[pattern=north east lines](xmin,ymin) rectangle (xmax,ymax);
			\fill[white] (0,0)--(20,0)--(12,6)--(0,10)--cycle;
			\draw[->] (xmin-0.25,0)--(xmax+0.5,0)
			node[shift={(-100:7pt)},font=\normalsize]{$ x $};
			\draw[->] (0,ymin-0.25)--(0,ymax+0.5)
			node[shift={(-5:7pt)},font=\normalsize]{$ y $};
			\fill (0,0) node[shift={(-45:9pt)},font=\normalsize]{$ O $};
			\begin{scope}
				\clip (xmin,ymin) rectangle (xmax,ymax);
				\draw plot[domain=xmin:xmax] (\x, {-0.75*\x+15});
				\draw plot[domain=xmin:xmax] (\x, {-1/3*(\x)+10});
			\end{scope}
			\foreach \x in {12, 20}{
			\draw (\x,0) circle (1pt) node[shift={(-90:7pt)},font=\footnotesize,circle,fill=white, inner sep=.1]{$\x$};}
			\foreach \y in {6, 10}{
			\draw (0,\y) circle (1pt) node[shift={(180:7pt)},font=\footnotesize,circle,fill=white, inner sep=.1]{$\y$};}
			\draw[dashed] (12,0)|-(0,6);
			\draw (20,0)node[shift={(70:7pt)},font=\footnotesize,circle,fill=white, inner sep=.1]{$A$}
			(12,6)node[shift={(70:7pt)},font=\footnotesize,circle,fill=white, inner sep=.1]{$B$}
			(0,10)node[shift={(30:8pt)},font=\footnotesize,circle,fill=white, inner sep=.1]{$C$};
		\end{tikzpicture}
	\end{center}
	Miền nghiệm của hệ điều kiện là miền tứ giác $OABC$, trong đó $O(0;0)$, $A(20;0)$, $B(12;6)$ và $C(0;10)$.\\
	Khi đó $F(O)=0$, $F(A)=1600$, $F(B)=1680$, $F(C)=1200$.\\
	Vậy $F_{max}=1680$ đạt tại đỉnh $B$.
	}
\end{ex}

%Câu 4

\begin{ex}%[Nguồn: Bộ đề minh họa Moon 2024-2025]%[0D8H2-5]
	Trong lễ tổng kết năm học 2021-2022, lớp 10A1 nhận được $20$ cuốn sách gồm $5$ cuốn sách Toán, $7$ cuốn sách Vật lí, $8$ cuốn sách Hóa học, các sách cùng môn là giống nhau. Số sách này được chia đều cho $10$ học sinh trong lớp, mỗi học sinh chỉ nhận được hai cuốn sách khác môn học. Bình và Bảo là hai trong số $10$ học sinh đó. Hỏi có bao nhiêu cách chia quà sao cho hai cuốn sách mà Bình nhận được giống hai cuốn sách của Bảo?
	\shortans[]{$784$}
	\loigiai
	{
	
	Gọi $x$, $y$, $z$ lần lượt là số phần quà gồm sách Toán và Vật lí, Toán và Hóa học, Vật lí và Hóa học.\\
	Ta có hệ phương trình $\heva{&x+y=5\\&x+z=7\\&y+z=8}\Leftrightarrow \heva{&x=2\\&y=3\\&z=5.}$\\
	Bình và Bảo nhận được phần quà giống nhau là
	\begin{itemize}
		\item Toán và Vật lý có $\mathrm{C}_8^3\cdot \mathrm{C}_5^5$ cách.
		\item Toán và Hóa có $\mathrm{C}_8^1\cdot \mathrm{C}_7^5\cdot \mathrm{C}_3^3$ cách.
		\item Hóa học và Vật lý có $\mathrm{C}_8^3\cdot \mathrm{C}_5^2\cdot \mathrm{C}_2^2$ cách.
	\end{itemize}
	Vậy có $784$ cách thỏa yêu cầu bài toán.
	}
\end{ex}

\begin{ex}%[Nguồn: Bộ đề minh họa Moon 2024-2025]%[0D8V1-1]
	Trên đường Mạnh đi từ nhà $(M)$ đến công ty $(C)$ có điểm $A$ người ta đang thi công sửa chữa đường nên không thể đi qua $A$.
	\begin{center}
		\begin{tikzpicture}[declare function={a=6; b=4; }]
			\path
			(0,0) coordinate (D)
			(a,0) coordinate (C)
			(0,-b) coordinate (M)
			($(M)+(C)$) coordinate (N)
			($(M)!1/4!(D)$) coordinate (a)
			($(M)!2/4!(D)$) coordinate (b);
			\path
			($(N)!1/4!(C)$) coordinate (a')
			($(M)!1/6!(N)$) coordinate (x)
			($(M)!2/6!(N)$) coordinate (y)
			($(M)!5/6!(N)$) coordinate (z);
			\foreach \i in {1, 2, 4, 5} {
			\path ($(a)!{\i/6}!(a')$) coordinate (a_\i);
			}
			\path
			($(a)!3/6!(a')$) coordinate (A)
			($(D)!.5!(C)$)  coordinate (td)
			($(A)!1/3!(td)$) coordinate (b')
			($(b)!1/3!(b')$) coordinate (b_1)
			($(b)!2/3!(b')$) coordinate (b_2)
			($(td)!1/3!(C)$) coordinate (a_4')
			($(A)!2/3!(td)$) coordinate (b'')
			($(a_4)!2/3!(a_4')$) coordinate (b''')
			($(M)+(0,-0.2)$) coordinate (M')
			($(x)+(0,-0.2)$) coordinate (x')
			($(M)+(-0.2,0)$) coordinate (M'')
			($(a)+(-0.2,0)$) coordinate (a'')
			;
			\draw[->] (M')--(x');
			\draw[->] (M'')--(a'');
			\draw (D)--(C)--(N)--(M)--cycle
			(a)--(a')
			(A)--(td)
			(b)--(b')
			(x)--(b_1) (y)--(b_2) (z)--(a_5)
			(a_4)--(a_4')
			(b'')--(b''')
			;
			\foreach \t/\g in {M/210,A/-90,C/45}{
			\draw[fill=black] (\t) circle (1pt) node[shift={(\g:9pt)},font=\scriptsize]{$ \t $};
			}
		\end{tikzpicture}
	\end{center}
	Biết rằng toàn bộ cung đường theo bản đồ từ dưới lên trên và từ trái qua phải là đường một chiều vì vậy Mạnh chỉ được phép đi lên hoặc đi sang phải. Vậy Mạnh có bao nhiêu cách đến công ty?
	
	\shortans{$15$}
	\loigiai{
	\begin{center}
		\begin{tikzpicture}[declare function={a=6; b=4; }]
			\path
			(0,0) coordinate (D)
			(a,0) coordinate (C)
			(0,-b) coordinate (M)
			($(M)+(C)$) coordinate (N)
			($(M)!1/4!(D)$) coordinate (a)
			($(M)!2/4!(D)$) coordinate (b);
			\path
			($(N)!1/4!(C)$) coordinate (a')
			($(M)!1/6!(N)$) coordinate (x)
			($(M)!2/6!(N)$) coordinate (y)
			($(M)!5/6!(N)$) coordinate (z);
			\foreach \i in {1, 2, 4, 5} {
			\path ($(a)!{\i/6}!(a')$) coordinate (a_\i);
			}
			\path
			($(a)!3/6!(a')$) coordinate (A)
			($(D)!.5!(C)$)  coordinate (H)
			($(A)!1/3!(H)$) coordinate (b')
			($(b)!1/3!(b')$) coordinate (b_1)
			($(b)!2/3!(b')$) coordinate (b_2)
			($(H)!1/3!(C)$) coordinate (a_4')
			($(A)!2/3!(H)$) coordinate (b'')
			($(a_4)!2/3!(a_4')$) coordinate (b''')
			($(M)+(0,-0.2)$) coordinate (M')
			($(x)+(0,-0.2)$) coordinate (x')
			($(M)+(-0.2,0)$) coordinate (M'')
			($(a)+(-0.2,0)$) coordinate (a'')
			($(x)!.5!(b_1)$) coordinate (I)
			($(y)!.5!(b_2)$) coordinate (K)
			;
			\draw (D)--(C)--(N)--(M)--cycle
			(a)--(a')
			(A)--(H)
			(b)--(b')
			(x)--(b_1) (y)--(b_2) (z)--(a_5)
			(a_4)--(a_4')
			(b'')--(b''')
			;
			\foreach \t/\g in {M/210,A/-90,C/45,H/90,b_1/90,b_2/90,a_4'/90,b''/180,b'''/0}{
			\draw[fill=black] (\t) circle (1pt) node[shift={(\g:9pt)},font=\scriptsize]{$ \t $};
			}
			\foreach \t/\g in {D/120,C/45,N/-30,M/210,a/180,b/180,x/-90,y/-90,z/-90,a'/0,b'/0,A/-90,I/120,K/120,a_5/90}{
			\draw[fill=black] (\t) circle (1pt) node[shift={(\g:9pt)},font=\scriptsize]{$ \t $};
			}
		\end{tikzpicture}
	\end{center}
	Các cách đi từ $M$	đến $C$ mà không qua $A$ thỏa yêu cầu bài toán như sau
	\begin{itemize}
		\item $M-D-C$.
		\item $M-b-b'-H-C$.
		\item $M-b-b'-b''-b'''-a_4'-C$.
		\item $M-a-I-b_1-b'-H-C$.
		\item $M-a-I-b_1-b'-b''-b'''-a_4'-C$.
		\item $M-a-I-K-b_2-b'-H-C$.
		\item $M-a-I-K-b_2-b'-b''-b'''-a_4'-C$.
		\item $M-x-I-b_1-b_2-b'-H-C$.
		\item $M-x-I-b_1-b_2-b'-b''-b'''-a_4'-C$.
		\item $M-x-I-K-b_2-b'-H-C$.
		\item $M-x-I-K-b_2-b'-b''-b'''-a_4'-C$.
		\item $M-x-y-K-b_2-b'-H-C$.
		\item $M-x-y-K-b_2-b'-b''-b'''-a_4'-C$.
		\item $M-x-y-z-a_5-a'-C$.
		\item $M-x-y-z-N-C$.
	\end{itemize}
	Vậy có $15$ cách đi từ $M$ đến $C$ mà không qua $A$ thỏa yêu cầu bài toán.
	}
\end{ex}

\begin{ex}%[Nguồn: Bộ đề minh họa Moon 2024-2025]%[0D8V2-4]
	Một lớp học hè có $15$ học sinh. Biết rằng mỗi ngày $3$ học sinh trong lớp có nhiệm vụ trực nhật sau giờ học. Sau khi kết thúc khóa học hè, người ta thấy rằng hai học sinh bất kỳ trực nhật cùng nhau đúng một ngày. Hỏi lớp học hè kéo dài trong bao nhiêu ngày?
	\shortans{35}
	\loigiai{
	Tổng số cặp học sinh có thể tạo thành từ $15$ học sinh là một tổ hợp chập $2$ của $15$ phần tử
	$$\mathrm{C}_{15}^2=105~\text{cặp}.$$
	Mỗi nhóm trực có $3$ học sinh nên số cặp trong mỗi nhóm là $3$ cặp.\\
	Để mỗi học sinh trực nhật cùng nhau đúng một ngày thì tổng số ngày tối đa là $105:3=35$ ngày.
	
	}
\end{ex}

%

\begin{ex}%[Nguồn: Bộ đề minh họa Moon 2024-2025]%[0D8V2-5]
	Tổng số điểm vòng bảng của $4$ đội trong một giải đấu bóng đá bằng $16$. Biết rằng, mỗi đội thi đấu vòng tròn $1$ lượt, vậy có bao nhiêu trận hòa trong vòng bảng? (Hai đội bất kỳ sẽ phải gặp nhau $1$ lần trong vòng bảng và cách tính điểm như sau: đội thắng được $3$ điểm, đội thua được $0$ điểm, hoà thì mỗi đội được $1$ điểm).
	\loigiai{
	Số trận đấu là $\mathrm{C}^3_4=6$ trận.\\
	Gọi $x$ số trận hòa. Khi đó tổng số điểm vòng bảng là $2x+3(6-x)$ \[2x+3(6-x)=16\Leftrightarrow x=2.\]
	Vậy có $2$ trận hòa.
	}
\end{ex}

\begin{ex}%[Nguồn: Bộ đề minh họa Moon 2024-2025]%[0H5V4-1]
	Cho hình chóp $S.ABC$ có đáy $ABC$ là tam giác đều, $AB=1$, cạnh bên $SB$ vuông góc với mặt phẳng đáy và $SB=1$. Tích vô hướng của hai vectơ $\overrightarrow{SA}$ và $\overrightarrow{SB}$ bằng
	\begin{center}
		\begin{tikzpicture}[scale=1, font=\footnotesize,>=stealth, line width=1pt]%<DTools>
			%Gán số liệu.
			\def\canhBC{4};\def\canhAB{2};\def\gocABC{-50};\def\h{3};\def\xdinhS{0};
			%Gán tọa độ.
			\coordinate (B) at (0,0);
			\coordinate (A) at ($(B)+(\gocABC:\canhAB)$);
			\coordinate (C) at ($(B)+(0:\canhBC)$);
			\coordinate (S) at ($(B)+(\xdinhS,\h)$);
			%Vẽ khối chóp S.BAC.
			\draw (S)--(A) (S)--(B)--(A) (S)--(C)--(A);
			\draw[dashed] (B)--(C);
			%Gán nhãn.
			\foreach \x/\y in {S/90,B/180,A/-90,C/0}{\fill (\x) circle (1pt) ($(\x)+(\y:0.3cm)$) node{$\x$};}
		\end{tikzpicture}
	\end{center}
	\choice
	{\True $1$}
	{$\sqrt{2}$}
	{$2$}
	{$\dfrac{\sqrt{2}}{2}$}
	\loigiai
	{
	$SB\perp (ABC) \Rightarrow SB \perp AB$ mà $SB=AB=1$ nên tam giác $ABC$ là tam giác vuông cân.\\
	Suy ra $\widehat{ASB}=45^\circ$.\\
	Ta có $SA=\sqrt{SB^2+AB^2}=\sqrt{1^2+1^2}=\sqrt{2}$.\\
	Vậy $\overrightarrow{SA}\cdot \overrightarrow{SB}=\left|\overrightarrow{SA}\right|\cdot \left|\overrightarrow{SB}\right|\cdot \cos \widehat{ASB}=\sqrt{2} \cdot 1 \cdot \dfrac{\sqrt{2}}{2}=1$.
	}
\end{ex}

\begin{ex}%[Nguồn: Bộ đề minh họa Moon 2024-2025]%[1C2H3-1]
	Từ kho $D$, xe bưu chính đi đến các hộp thư tại $E$, $F$, $G$, và $H$ rồi quay lại kho. Sơ đồ bên dưới hiển thị thời gian xe bưu chính di chuyển giữa các hộp thư (đơn vị: phút).
	\begin{center}
		\begin{tikzpicture}
			\def\a{3}
			\path (0:0) coordinate (H)
			++(0:\a) coordinate (G)
			++(70:3) coordinate (F)
			++(160:2.3) coordinate (E)
			++(210:3) coordinate (D);
			\draw[thick] (H) -- (E) node[midway,sloped, above] {$9$};
			\draw[thick] (D) -- (G) node[midway, above] {7};
			\draw[thick] (E) -- (G) node[midway,sloped, above] {$10$};
			\draw[thick] (D) -- (F) node[midway,sloped, above] {$9$};
			\draw[thick] (D) -- (E) node[midway,sloped, above] {$11$};
			\draw[thick] (E) -- (F) node[midway,sloped, above] {$7$};
			\draw[thick] (D) -- (H) node[midway,sloped,below left] {$3$};
			\draw[thick] (G) -- (F) node[midway,sloped, below] {$13$};
			\draw [thick] (H)--(G)node[midway,sloped, below] {$6$};
			\foreach \x/\g in {H/-90,G/-30,F/0,E/90,D/180}
			\fill[black] (\x) circle (3pt)
			($(\g:4mm)+(\x)$) node {$\x$};
			%Hình chóp S.ABC có SA vuông góc đáy
		\end{tikzpicture}
	\end{center}
	Thời gian ngắn nhất để xe bưu chính thực hiện điều đó là bao nhiêu phút?
	\shortans[0]{$35$}
	\loigiai{
	Đường đi ngắn nhất chính là $D\rightarrow H\rightarrow G\rightarrow E\rightarrow F\rightarrow D$ với tổng thời gian là $35$ phút.
	}
\end{ex}

%Điền đáp án 2

\begin{bt}%[Nguồn: Bộ đề minh họa Moon 2024-2025]%[1C2H3-1]
	Biểu đồ thể hiện các con đường nối giữa các thị trấn (đơn vị: km). Cán bộ thanh tra xuất phát từ thị trấn $L$ đi kiểm tra tất cả các tuyến đường nối giữa các thị trấn $M$, $N$, $O$ và quay lại $L$. Chiều dài quãng đường tối thiểu thanh tra cần phải đi là bao nhiêu km?
	\begin{center}
		\begin{tikzpicture}[font=\footnotesize, line join=round, line cap=round, >=stealth, scale=1]
			\def\bankinh{2}
			\path (0,0) coordinate (O) (-\bankinh,0) coordinate (L) (\bankinh,0) coordinate (N) (60:\bankinh) coordinate (M);
			\draw
			(L) arc(180:60:\bankinh) node[pos=0.5, above]{$16$}
			(M) arc(60:0:\bankinh) node[pos=0.5, above]{$8$}
			(L) to[bend right=60] node[pos=0.5, below]{$15$} (N)
			(L) to[bend right=40] node[pos=0.5, below]{$9$} (O)
			(L) to[bend left=40] node[pos=0.5, above]{$7$} (O)
			(O) to[bend right=40] node[pos=0.5, below]{$8$} (N)
			(O) to[bend left=40] node[pos=0.5, above]{$6$} (N)
			;
			\foreach \x/\g in {O/90, L/180, N/0, M/60}{
			\fill (\x) circle (2pt)+(\g:0.3)node{$\x$};
			}
		\end{tikzpicture}
	\end{center}
	\par\shortans{$37$}
	\loigiai{
	Để chiều dài quãng đường là tối thiểu, cán bộ thanh tra cần xuất phát từ $L$, đi qua mỗi thị trấn $M$, $N$, $O$ đúng một lần trước khi qua lại $L$ và ưu tiên chọn con đường ngắn hơn trong hai con đường nối hai thị trấn.\\
	Ta có quãng đường tối ưu là $L\rightarrow M \rightarrow N \rightarrow O \rightarrow L$ với độ dài là $16+8+6+7=37$ km.
	}
\end{bt}

%

\begin{ex}%[Nguồn: Bộ đề minh họa Moon 2024-2025]%[1C2V1-3]
	Một con bọ di chuyển từ điểm $A$ đến điểm $B$ dọc theo các đoạn thẳng trong mạng lưới lục giác như hình bên dưới
	\begin{center}
		\begin{tikzpicture}[>=stealth,line join=round,line cap=round,font=\footnotesize,scale=1]
			\def\l{1.2}
			\newcommand{\drawhexagon}[3]{
			\begin{scope}[shift={(#1,#2)}]
				% Vẽ lục giác
				\draw (0:\l) -- (60:\l) -- (120:\l) -- (180:\l) -- (240:\l) -- (300:\l) -- cycle;
			\end{scope}
			}
			% Hàng 1
			\drawhexagon{0}{0}{}
			\draw [->] (0,1.04)--(0.1,1.04) node [above] {$1$} ;
			\draw [->] (0,-1.04)--(0.1,-1.04)node [below] {$2$};
			\draw (0,1.3) node[above] {$(C_1)$};
			% Hàng 2
			\drawhexagon{1.8}{1.04}{}
			\drawhexagon{1.8}{-1.04}{}
			\draw [->] (1.8,2.07)--(1.9,2.07) node [above] {$3$};
			\draw [<-] (1.8,0)--(1.9,0) ;
			\draw [->] (1.8,-2.08)--(1.9,-2.08)node [below] {$4$};
			\draw (1.8,2.4) node[above] {$(C_2)$};
			% Hàng 3
			\drawhexagon{3.6}{2.08}{}
			\drawhexagon{3.6}{0}{}
			\drawhexagon{3.6}{-2.08}{}
			\draw [->] (3.5,3.12)--(3.6,3.12) node [above] {$5$};
			\draw [->] (3.5,1.04)--(3.6,1.04) node [above] {$6$} ;
			\draw [->] (3.5,-1.04)--(3.6,-1.04) node [above] {$7$} ;
			\draw [->] (3.5,-3.12)--(3.6,-3.12) node [above] {$8$} ;
			\draw (3.5,3.4) node[above] {$(C_3)$};
			% Hàng 4
			\drawhexagon{5.4}{1.04}{}
			\drawhexagon{5.4}{-1.04}{}
			\draw [->] (5.2,2.07)--(5.3,2.07) node [above] {$9$};
			\draw [<-] (5.2,0)--(5.3,0) ;
			\draw [->] (5.2,-2.08)--(5.3,-2.08)node [below] {$10$};
			\draw (5.2,2.4) node[above] {$(C_4)$};
			% Hàng 5
			\drawhexagon{7.2}{0}{}
			\draw [->] (7,1.04)--(7.1,1.04) node [above] {$11$} ;
			\draw [->] (7,-1.04)--(7.1,-1.04)node [below] {$12$};
			\draw (7,1.3) node[above] {$(C_5)$};
			% Điểm A và B
			\node[fill=black,circle,inner sep=1pt,label=left:A] at (-1.2,0) {};
			\node[fill=black,circle,inner sep=1pt,label=right:B] at (8.4,0) {};
			
		\end{tikzpicture}
		
	\end{center}
	Các đoạn thẳng có dấu mũi tên chỉ được di chuyển theo hướng của mũi tên, các đọạn thẳng không có dấu mũi tên được di chuyển theo hướng tùy ý và con bọ không bao giờ di chuyển trên cùng một đoạn thẳng quá một lần. Vậy con bọ có bao nhiêu con đường khác nhau từ $A$ đến $B$?
	\shortans[1]{$100$}
	\loigiai{
	\begin{itemize}
		\item Từ $A$ có $2$ cách đến $C1$
		\begin{itemize}
		\item Từ $A$ có $1$ cách đến mũi tên số $1$;
		\item Từ $A$ có $1$ cách đến mũi tên số $2$.
	\end{itemize}
	\item Từ $C1$ có $5$ cách đến $C3$\\
	Không mất tính tổng quát giả sử đi từ $C1$ mũi tên số $1$ đến $C3$ mũi tên số $6$ hoặc mũi tên số $7$.
	\begin{itemize}
		\item Từ mũi tên số $1$ có $2$ cách để đi đến mũi tên số $6$.
		\item Từ mũi tên số $1$ có $3$ cách để đi đến mũi tên số $7$.
	\end{itemize}
	Suy ra từ $C1$ có $5$ cách đến $C3$.
	\item Từ $C3$ có $5$ cách đến $C5$\\
	Không mất tính tổng quát giả sử đi từ $C3$ mũi tên số 5 đến $C5$ mũi tên số $11$ hoặc mũi tên số $12$.
	\begin{itemize}
		\item Từ mũi tên số $5$ có $2$ cách để đi đến mũi tên số $11$.
		\item Từ mũi tên số $5$ có $3$ cách để đi đến mũi tên số $11$.
	\end{itemize}
	Suy ra từ $C3$ có $5$ cách đến $C5$.
	\item Từ $C5$ có $2$ cách đến $B$
	\begin{itemize}
		\item Từ mũi tên số $11$ có $1$ cách đến $B$.
		\item Từ mũi tên số $12$ có $1$ cách đến $B$.
	\end{itemize}
	\end{itemize}
	Vậy có $2\cdot 5\cdot 5\cdot 2 = 100$ cách đi từ $A$ đến $B$.
	}
\end{ex}

\begin{ex}%[Nguồn: Bộ đề minh họa Moon 2024-2025]%[1C2V1-3]
	Để làm một quả bóng như hình bên, người ta dùng một số mảnh da hình lục giác đều và hình ngũ giác đều có các cạnh bằng nhau, rồi khâu các cạnh chung với năm hình lục giác đều như hình vẽ. Hỏi cần dùng bao nhiêu mảnh da để làm một quả bóng?
	\begin{center}
		\begin{tikzpicture}[line join=round, line cap=round,scale=1,transform shape]
			%\clip (-3,-2) rectangle (3,2);
			\tikzset{bong/.pic={
			\draw (0,0) circle (44pt);
			\def\C{
			(-.2,1.5)%x
			..controls +(-150:.12) and +(45:.1) ..(-.4,1.2)%1
			--(-.82,.3)%3
			..controls +(-150:.15) and +(65:.25) ..(-1.1,-.3)%4c
			--(-.86,-1.15)%2c
			..controls +(-75:.38) and +(-160:.3) ..(.2,-1.48)%y
			..controls +(55:.2) and +(-110:.1) ..(.35,-1.16)
			--(.8,-.22)%5b
			..controls +(45:.3) and +(-110:.1) ..(1.1,.34)%3a
			--(.87,1.15)%1a
			..controls +(110:.35) and +(15:.33) ..(-.2,1.5)%x
			%--------------------------
			(-.9,.87)%2
			..controls +(-160:.12) and +(20:.1) ..  (-1.25,.8)
			..controls +(-160:.22) and +(150:.3) ..  (-1.42,-.3)%1c
			--(-.75,-.86)%3c
			..controls +(-10:.1) and +(-175:.1) ..  (-.1,-.85)%2b
			--(.9,-.8)%4b
			..controls +(-15:.1) and +(-175:.1) ..  (1.28,-.78)%z
			..controls +(50:.5) and +(-50:.3) ..  (1.4,.35)%4b
			%--------------------------
			(.08,.9)%5
			..controls +(20:.12) and +(175:.1) ..  (.72,.88)%2a
			(-.22,.3)%4
			..controls +(-40:.12) and +(145:.1) .. (.2,-.24)%1b
			;}
			\draw \C;
			
			%------------------------------------------
			\def\B{
			(-.4,1.2)%1
			..controls +(-157:.12) and +(45:.1) ..  (-.9,.87)%2
			..controls +(-100:.12) and +(160:.1) ..  (-.82,.3)%3
			..controls +(-5:.12) and +(175:.1) ..  (-.22,.3)%4
			..controls +(45:.1) and +(-140:.1) ..  (.08,.9)%5
			..controls +(145:.4) and +(-35:.2) ..  cycle
			
			(.87,1.15)%1a
			..controls +(-150:.12) and +(45:.1) ..  (.72,.88)%2a
			..controls +(-40:.12) and +(110:.4) ..  (1.1,.34)%3a
			..controls +(-5:.12) and +(175:.1) ..  (1.4,.35)%4a
			..controls +(60:.25) and +(-15:.38) ..  cycle
			
			(.2,-.24)%1b
			..controls +(-130:.1) and +(45:.1) ..  (-.1,-.85)%2b
			..controls +(-30:.1) and +(150:.1) ..  (.35,-1.16)%3b
			..controls +(-25:.12) and +(-140:.1) ..  (.9,-.8)%4b
			..controls +(80:.2) and +(-55:.1) ..  (.8,-.22)%5b
			..controls +(-170:.3) and +(-5:.2) ..  cycle
			
			(-1.42,-.3)%1c
			..controls +(-100:.41) and +(160:.36) ..  (-.86,-1.15)--(-.75,-.86)%2c 3c
			..controls +(130:.1) and +(-70:.1) ..(-1.1,-.3)--cycle %4c
			%%%%%%%%%mẩu nhỏ
			(0,-1.52)
			..controls +(0:0) and +(-120:.01) ..(.2,-1.48)%y
			--(.3,-1.51)
			..controls +(-150:0.1) and +(-30:0.01) ..cycle
			%------------
			(-1.25,.8)--(-1.25,.9)
			..controls +(-125:0.1) and +(65:0.1) ..(-1.36,.69)
			%------------
			(1.28,-.78)--(1.25,-.9)
			..controls +(45:0.1) and +(-120:0.01) ..(1.37,-.66)--cycle
			(-.2,1.5)%x
			--(-.3,1.51)
			..controls +(20:.01) and +(-175:.1) ..(-.1,1.54)--cycle
			;}
			\draw \B;
			\fill \B;
			
			}}
			\path (0,0)pic[scale=1]{bong};
		\end{tikzpicture}
	\end{center}
	\shortans{$32$}
	\loigiai{
	Gọi $m$ là số mảnh da hình ngũ giác, $n$ là số mảnh da hình lục giác.\\
	Số mảnh da của quả bóng là $M = m + n$.\\
	Mỗi mảnh da hình ngũ giác tiếp xúc với $5$ mảnh da hình lục giác nên số đường khâu giữa các mảnh da hình ngũ giác và các mảnh da hình lục giác là $5m$.\\
	Mỗi mảnh da hình lục giác tiếp xúc với $3$ mảnh da hình ngũ giác nên số đường khâu ghép các mảnh da hình lục giác và mảnh da hình ngũ giác là $3n$.\\
	Suy ra $5m = 3n \Leftrightarrow m = \dfrac{3n}{5}$.\\
	Khi đó số mảnh da của quả bóng là $M = m + n = \dfrac{3n}{5} + n = \dfrac{8n}{5}$.\\
	Mặt khác, số đường khâu giữa các mảnh da hình lục giác là $\dfrac{3n}{2}$. Vì cứ mỗi mảnh da hình lục giác này lại tiếp xúc với $3$ mảnh da hình lục giác khác và mỗi đường khâu ghép ta đã đếm $2$ lần.\\
	Tổng só đường khâu ghép trên quả bóng là $3n + \dfrac{3n}{2} = \dfrac{9n}{2}$.\\
	Số đỉnh của tất cả các mảnh da là $5m$ hay $3n$ (bằng tổng tất cả các đỉnh của mảnh da đen).\\
	Theo công thức Euler ta có: Số đỉnh + Số mặt = Số cạnh + 2 nên ta có
	\begin{align*}
			3n + \dfrac{8n}{5} = \dfrac{9n}{2} + 2 \Leftrightarrow n = 20 \Rightarrow m = 12.
	\end{align*}
	Vậy cần dùng $32$ mảnh da để làm một quả bóng.
	}
\end{ex}

%

\begin{bt}%[Nguồn: Bộ đề minh họa Moon 2024-2025]	%[1C2V2-2]
	Một trò chơi điện từ quy định như sau: Có $4$ trụ $A$, $B$, $C$, $D$ với số lượng các thử thách trên đường đi giữa các cặp trụ được mô tả trong hình bên.
	\begin{center}
		\begin{tikzpicture}[line join=round, line cap=round,>=stealth,thick,scale=0.8]
			\path
			(3,5) coordinate (A)
			(0,0) coordinate (B)
			(7,0) coordinate (C)
			(2.8,1.3) coordinate (D)
			;
			\draw (B)--(A)node[left=2pt,midway]{10}
			--(C)node[right=2pt,midway]{11}
			--(D)node[above,midway]{14}
			--cycle
			(B)--(D)node[above,midway]{11}
			--(A)node[left=2pt,midway]{9}
			--(C)--(B)node[below,midway]{12}
			;
			\foreach \x/\g in {A/90,B/180,C/0,D/40} \draw[fill=white] (\x) circle (.03)+(\g:0.4)node{$\x$};
		\end{tikzpicture}
	\end{center}
	Người chơi xuất phát từ một trụ nào đó, đi qua tất cả các trụ còn lại, mỗi khi đi qua một trụ thì trụ đó sẽ bị phá hủy và không thể quay trở lại trụ đó được nữa, nhưng người chơi vẫn phải trở về trụ ban đầu. Tổng số thử thách của đường đi thỏa mãn điều kiện trên nhận giá trị nhỏ nhất là bao nhiêu?
	\shortans{43}
	\loigiai{
	Câu này tương đương với bài toán tìm chu trình Hamilton tối ưu trong một đồ thị đầy đủ, nơi mỗi cạnh giữa hai trụ là một cạnh của đồ thị với trọng số là số lượng thử thách giữa các trụ. Chu trình này cần phải xuất phát từ một điểm bất kỳ, đi qua tất cả các đỉnh đúng một lần và trở về điểm xuất phát, sao cho tổng trọng số là nhỏ nhất. Các cạnh nối giữa các trụ
	$$
	\begin{aligned}
		& A \leftrightarrow B: 10 \text{ thử thách } \\
		& A \leftrightarrow C: 11 \text{ thử thách } \\
		& A \leftrightarrow D: 9 \text{ thử thách } \\
		& B \leftrightarrow C: 12 \text{ thử thách } \\
		& B \leftrightarrow D: 11 \text{ thử thách } \\
		& C \leftrightarrow D: 14 \text{ thử thách }
	\end{aligned}
	$$
	
	Trong các đường đi thì đi từ $A$ đến $D$ là ngắn nhất nên ta xét hai trường hợp sau
	
	Đường đi 1: $A\to D\to B\to C\to A$
	$$
	A \to D: 9, \quad D \to B: 11, \quad B \to C: 12, \quad C \to A: 11
	$$
	
	Tổng: $9+11+12+11=43$ thử thách.
	
	Đường đi 2: $A\to D\to C\to B\to A$
	$$
	A \to D: 9, \quad D \to C: 14, \quad C \to B: 12, \quad B \to A: 10
	$$
	
	Tổng: $9+14+12+10=45$ thử thách.
	
	Kết luận: Đường đi ngắn nhất có tổng thử thách là $43$.
	
	Đường này là $A\to D\to B\to C\to A$.
	
	Vậy tổng số thử thách nhỏ nhất mà người chơi phải vượt qua là $43$.
	}
\end{bt}

\begin{ex}%[Nguồn: Bộ đề minh họa Moon 2024-2025]%[1D2H2-2]
	Cho cấp số cộng $\left(u_n \right)$ thoả mãn $u_4 - u_1 = 6$. Công sai của $\left(u_n \right)$ bằng
	\choice
	{$-2$}
	{$-3$}
	{\True $2$}
	{$3$}
	\loigiai
	{
	Ta có $u_4 - u_1 = 6 \Rightarrow u_1 + 3d - u_1 = 6 \Rightarrow d = 2$.
	}
\end{ex}

\begin{ex}%[Nguồn: Bộ đề minh họa Moon 2024-2025]%[1D2H2-2]
	Cho $\left(u_n\right)$ là cấp số cộng có $u_1=-3$, $u_6=27$. Tìm công sai $d$ của $\left(u_n\right)$.
	\choice
	{$d=8$}
	{$d=7$}
	{\True $d=6$}
	{$d=5$}
	\loigiai{
	Ta có $u_6=u_1+5d$ nên $27=-3+5d\Leftrightarrow d=6$.
	}
\end{ex}

%

\begin{ex}%[Nguồn: Bộ đề minh họa Moon 2024-2025]%[1D2H2-4]
	Cấp số cộng $(u_n)$ có $u_1=1$ và $u_2=3$. Số hạng $u_5$ của cấp số cộng là
	\choice
	{$5$}
	{$7$}
	{\True $9$}
	{$11$}
	\loigiai{
	Ta có $d=u_2 -u_1 =3-1=2$.\\
	Suy ra $u_{5}=u_1 +4 d=1+4 \cdot 2=9$.
	}
\end{ex}

\begin{ex}%[Nguồn: Bộ đề minh họa Moon 2024-2025]%[1D2H2-4]
	Cho cấp số cộng $6$, $17$, $28$, $\ldots$, số hạng thứ $10$ của cấp số cộng đã cho bằng
	\choice
	{$108$}
	{$106$}
	{$107$}
	{\True $105$}
	\loigiai
	{
	Cấp số cộng có $u_1=6$ và công sai $d=11$.\\
	Số hạng thứ $10$ là $u_{10}=u_1+(10-1)d=6+9\cdot 11=105$.
	}
\end{ex}

%

\noindent\textbf{PHẦN I. TRẮC NGHIỆM 4 PHƯƠNG ÁN.} Thí sinh trả lời từ câu 1 đến câu 12. Mỗi câu hỏi thí sinh chỉ chọn một phương án.
\setcounter{ex}{0}

\begin{ex}%[Nguồn: Bộ đề minh họa Moon 2024-2025]%[1D2H3-2]
	Cho cấp số nhân $\left(u_n\right)$ với $u_1=2$ và $u_4=16$. Công bội của cấp số nhân đã cho bằng
	\choice
	{$4$}
	{\True $2$}
	{$-2$}
	{$-4$}
	\loigiai{
	Ta có $u_4=u_1\cdot q^3\Rightarrow q^3=\dfrac{u_4}{u_2}=\dfrac{16}{2}=8\Rightarrow q=2$.\\
	Vậy công bội của cấp số nhân là $q=2$.
	}
\end{ex}

\begin{ex}%[Nguồn: Bộ đề minh họa Moon 2024-2025]%[1D2H3-2]
	Cho cấp số nhân $(u_n)$ với $u_1 = 2$, $u_8 = 256$. Công bội của cấp số nhân đã cho bằng
	\choice
	{$6$}
	{$4$}
	{\True $2$}
	{$\dfrac{1}{4}$}
	\loigiai{
	Ta có $u_n = u_1 \cdot q^{n-1}$. \\
	Do đó \allowdisplaybreaks
	\begin{eqnarray*}
		u_8 = u_1 \cdot q^7
		&\Rightarrow &256 = 2 \cdot q^7\\
		&\Leftrightarrow &q^7 = 128\\
		&\Leftrightarrow &q = 2.
	\end{eqnarray*}
	}
\end{ex}

\begin{ex}%[Nguồn: Bộ đề minh họa Moon 2024-2025]%[1D2H3-4]
	Cho cấp số nhân $(u_n)$ với $u_1=3; q=\dfrac{1}{2}$. Số $\dfrac{3}{512}$ là số hạng thứ mấy?
	\choice
	{$11$}
	{$9$}
	{$10$}
	{$12$}
	\loigiai{
	Áp dụng công thức $u_n=u_1\cdot q^{n-1}$.\\
	Ta có $\dfrac{3}{512}=3\cdot\left(\dfrac{1}{2}\right)^{n-1}$$\Leftrightarrow \left(\dfrac{1}{2}\right)^{n-1}=\dfrac{1}{512}=\left(\dfrac{1}{2}\right)^9\Leftrightarrow n-1=9\Leftrightarrow n=10$.\\
	Vậy số $\dfrac{3}{512}$ là số hạng thứ $10$.
	}
\end{ex}

%

\begin{ex}%[Nguồn: Bộ đề minh họa Moon 2024-2025]%[1D2H3-4]
	Một cấp số nhân có hai số hạng liên tiếp là $6$ và $18$. Số hạng tiếp theo là
	\choice
	{$30$}
	{$12$}
	{$24$}
	{\True $54$}
	\loigiai{Ta có $6$, $18$, $x$ liên tiếp lập thành cấp số nhân nên \[6\cdot x=18^2\Leftrightarrow x=54.\]
	}
\end{ex}

\begin{ex}%[Nguồn: Bộ đề minh họa Moon 2024-2025]%[1D2H3-4]
	Một siêu thị chạy chương trình khuyến mãi cho nước tăng lực có giá niêm yết là $9\,000$ (đồng/lon) như sau:
	\begin{itemize}
		\item Nếu mua $1$ lon thì không giảm giá.
		\item Nếu mua $2$ lon thì lon thứ hai được giàm $500$ đồng.
		\item Nếu mua $3$ lon thì lon thứ hai được giàm $500$ đồng và lon thứ ba được giảm giá $10 \%$.
		\item Nếu mua trên $3$ lon thì lon thứ hai được giảm $500$ đồng, lon thứ ba được giảm $10 \%$ và những lon thứ tư trở đi đều được giảm thêm $2 \%$ trên giá đã giảm của lon thứ ba.
	\end{itemize}
	Hòa phải trả $422\,500$ đồng để thanh toán khi mua những lon nước tăng lực trên. Hòa đã mua bao nhiêu lon nước?
	\shortans{$53$}
	\loigiai{
	Để giải bài toán này, chúng ta cần xác định số lượng lon nước tăng lực mà Hòa đã mua dựa trên tổng số tiền phải trả là $422\,500$ đồng. Chúng ta sẽ phân tích từng trường hợp mua hàng và tính toán chi tiết.\\
	Tính toán cho từng trường hợp
	\begin{itemize}
		\item Trường hợp mua $1$ lon\\
		Tổng tiền $9\,000$ đồng.\\
		Không phù hợp với tổng tiền $422\,500$ đồng.
		\item Trường hợp mua $2$ lon\\
		Lon thứ nhất $9\,000$ đồng.\\
		Lon thứ hai $9\,000 - 500 = 8\,500$ đồng.\\
		Tổng tiền $9\,000 + 8\,500 = 17\,500$ đồng. Không phù hợp.
		\item Trường hợp mua $3$ lon\\
		Lon thứ nhất $9\,000$ đồng.\\
		Lon thứ hai $9\,000 - 500 = 8\,500$ đồng.\\
		Lon thứ ba $9\,000 - (10\%\cdot 9\,000) = 8\,100$ đồng.
		
		Tổng tiền $9\,000 + 8\,500 + 8\,100 = 25\,600$ đồng.	Không phù hợp.
		\item Trường hợp mua trên $3$ lon\\
		Giả sử Hòa mua $n$ lon $(n>3)$.\\
		Lon thứ nhất $9\,000$ đồng.\\
		Lon thứ hai $9\,000 - 500 = 8\,500$ đồng.\\
		Lon thứ ba $9\,000 - (10\% \cdot 9\,000) = 8\,100$ đồng.\\
		Các lon từ thứ tư trở đi: Mỗi lon giảm thêm $2\%$ trên giá đã giảm của lon thứ ba, tức là $8\,100 - (2\% \cdot 8\,100) = 7\,938$ đồng.\\
		Tổng tiền cho $n$ lon:
		$$9\,000+8\,500+8\,100+
		7\,938 \cdot (n-3)=	422\,500.$$
		Giải phương trình trên ta được $n \approx 53.$
	\end{itemize}
	Vậy, Hòa đã mua $53$ lon nước tăng lực.
	}
\end{ex}

\begin{ex}%[Nguồn: Bộ đề minh họa Moon 2024-2025]%[1D2N1-3]
	Cho dãy số $(u_n)$ với $u_n=\dfrac{2n}{3n+2}, n \in \mathbb{N}^*$. Khẳng định nào sau đây đúng?
	\choice
	{$u_2=1$}
	{\True $u_2=\dfrac{1}{2}$}
	{$u_2=-\dfrac{1}{2}$}
	{$u_2=\dfrac{1}{3}$}
	\loigiai{
	Ta có $u_2=\dfrac{2\cdot 2}{3\cdot 2+2}=\dfrac{1}{2}$.
	}
\end{ex}

\begin{ex}%[Nguồn: Bộ đề minh họa Moon 2024-2025]%[1D2N1-3]
	Cho dãy số $(u_n)$ có số hạng tổng quát $u_n = 1 - \dfrac{n}{n^2 + 1}$ (với $n \in \mathbb{N}^*$). Số hạng đầu tiên của dãy là
	\choice
	{$2$}
	{$\dfrac{3}{5}$}
	{$0$}
	{\True $\dfrac{1}{2}$}
	\loigiai{
	Số hạng đầu tiên của dãy là $u_1 = 1 - \dfrac{1}{1^2 + 1}=\dfrac{1}{2}$.
	}
\end{ex}

\begin{ex}%[Nguồn: Bộ đề minh họa Moon 2024-2025]%[1D2N2-5]
	Bốn số $-2; x; 4; y$ theo thứ tự lập thành cấp số cộng, khi đó $x-y$ bằng
	\choice
	{$6$}
	{$3$}
	{$\mathbf{-3}$}
	{$-6$}
	\loigiai{ Theo giả thiết bốn số $-2; x; 4; y$ theo thứ tự lập thành cấp số cộng nên ta có hệ
	$$
	\begin{aligned}
		& \heva{
		&2x = -2+ 4 \\
		&x + y = 2\cdot4
		}  \Leftrightarrow  \heva{&x=1\\
		&1+y=2\cdot4}  \\
		& \Leftrightarrow \heva{&x=1\\
		&y=7}  \\
		& \Rightarrow x-y=1-7=-6 .
	\end{aligned}$$
	}
\end{ex}

\begin{ex}%[Nguồn: Bộ đề minh họa Moon 2024-2025]%[1D2N3-2]
	Cho cấp số nhân $(u_{n})$ có số hạng đầu $u_{1}=\dfrac{1}{2}$, công bội $q=2$. Giá trị của $u_{25}$ bằng
	\choice
	{$2^{26}$}
	{\True $2^{23}$}
	{$2^{24}$}
	{$2^{25}$}
	\loigiai{
	Ta có công thức số hạng tổng quát của cấp số nhân $u_n = u_1 \cdot q^{n-1}$.\\
	Vậy $u_{25} = \dfrac{1}{2} \cdot 2^{25-1} = \dfrac{1}{2} \cdot 2^{24} = 2^{-1} \cdot 2^{24} = 2^{23}$.
	}
\end{ex}

%

\begin{ex}%[Nguồn: Bộ đề minh họa Moon 2024-2025]%[1D2N3-2]
	Cho cấp số nhân $(u_n)$, biết $u_1=1$; $u_4=64$. Công bội $q$ của cấp số nhân bằng
	\choice
	{$q=2$}
	{$q=8$}
	{\True $q=4$}
	{$q=2\sqrt{2}$}
	\loigiai{Ta có $u_4=u_1q^3\Leftrightarrow q^3=64\Leftrightarrow q=4$.}
\end{ex}

\begin{ex}%[Nguồn: Bộ đề minh họa Moon 2024-2025]%[1D2N3-4]
	Cho cấp số nhân $(u_n)$ có $u_1=2$ và công bội $q=3$. Giá trị của $u_2$ bằng
	\choice
	{$5$}
	{$9$}
	{$8$}
	{\True $6$}
	\loigiai{
	Ta có $u_2=q\cdot u_1=3\cdot 2=6$.
	}
\end{ex}

%

\begin{ex}%[Nguồn: Bộ đề minh họa Moon 2024-2025]%[1D2N3-4]
	Cho cấp số nhân $\left(u_n\right)$ với $u_1=3$ và công bội $q=-2$. Giá trị của $u_4$ bằng
	\choice
	{$12$}
	{$24$}
	{\True $-24$}
	{$-12$}
	\loigiai{
	$u_4=u_1\cdot q^3=3\cdot (-2)^3=-24$}
\end{ex}

\begin{ex}%[Nguồn: Bộ đề minh họa Moon 2024-2025]%[1D2N3-4]
	Cho cấp số nhân có $u_1=4$, $q=3$. Hãy tính giá trị của $u_3$.
	\choice
	{$u_3=-2$}
	{$u_3=7$}
	{$u_3=10$}
	{\True $u_3=36$}
	\loigiai{
	Ta có $u_3=u_1\cdot q^2=4\cdot 3^2=36$.
	}
\end{ex}

\begin{ex}%[Nguồn: Bộ đề minh họa Moon 2024-2025]%[1D2V1-6]
	\immini{Một chai soda có giá $1$ đô. Sau khi uống, hai chai rỗng sẽ được đổi lấy một chai soda. Bạn có thể uống nhiều nhất bao nhiêu chai soda nếu bạn có $100$ đô?}{\begin{tikzpicture}[very thick,>=stealth',scale=0.7]
		\filldraw[black] (-0.25,3.8)--(-0.25,3.5) arc (180:360:0.25 and 0.1)--(0.25,3.8) arc (0:-180:0.25 and 0.1);
		\filldraw[blue!70!green!30] (-0.6,0.5)--(-0.6,-2) arc (180:0:0.6 and 0.3)--(0.6,0.5);
		\filldraw[blue!50!green!50,,] (0,0.5) ellipse (0.6 and 0.3);
		\filldraw[blue!50!green!50] (0,-2) ellipse (0.6 and 0.3);
		\draw (-0.6,1) -- (-0.6,-2) (0.6,1) -- (0.6,-2);
		\draw (0,-2) ellipse (0.6 and 0.3);
		\draw (-0.6,1)--(-0.6,1.5)--(-0.25,2.8)--(-0.25,3.8);
		\draw (0.6,1)--(0.6,1.5)--(0.25,2.8)--(0.25,3.8);
		\draw (0,3.8) ellipse (0.25 and 0.1);
	\end{tikzpicture}}
	\shortans[oly]{199}
	\loigiai{
	\textbf{Mua số lượng chai soda ban đầu:}\\
	Với $100$ đô ta sẽ mua được $100$ chai soda.\\
	Đổi chai rỗng lấy soda: Sau khi uống $100$ chai, ta sẽ có $100$ chai rỗng, $100$ chai rỗng ta sẽ đổi được 50
	chai soda mới.\\
	\textbf{	Tiếp tục quy trình:}\\
	Sau khi uống $50$ chai soda mới, ta được $50$ chai rỗng.
	Đổi $50$ chai rỗng lấy $25$ chai soda mới.\\
	\textbf{Lặp lại quy trình:}\\
	Uống $25$ chai so da mới, ta được $25$ chai rỗng.\\
	Đổi $25$ chai rỗng ta được $12$ chai soda mới và dư $1$ chai rỗng.\\
	\textbf{	Tiếp tục đổi:}\\
	Uống $12$ chai soda mới ta sẽ có $12$ chai rỗng.\\
	Đổi $12$ chai rỗng ta được $6$ chai so đa mới.\\
	\textbf{Lặp lại quy trình:}\\
	Uống $6$ chai so da mới, ta được $6$ chai rỗng.\\
	Đổi $6$ chai rỗng ta được $3$ chai soda mới.\\
	\textbf{Tiếp tục đổi:}\\
	Uống chai $3$ soda mới vừa đổi ta được $3$ chai rỗng.\\
	Cộng với $1$ chai rỗng còn dư ở phía trên ta được $4$ chai rỗng và đổi thêm được $2$ chai soda mới.\\
	\textbf{Kết thúc:}
	Uống $2$ chai so da mới, ta được $2$ chai rỗng.\\
	Đổi $2$ chai rỗng ta được 1 chai soda mới.\\
	Uống chai $1$ soda mới vừa đổi ta được 1 chai rỗng.\\
	Vậy ta có thể uống được nhiều nhất là $100 + 50 + 25 + 12 + 6 + 3 + 2 + 1 = 199$ chai.
	}
\end{ex}

%

\begin{ex}%[Nguồn: Bộ đề minh họa Moon 2024-2025]%[1D2V1-6]
	Nhân dịp khai trương, cửa hàng tổ chức chương trình tri ân dành cho 50 người đầu tiên đứng xếp hàng (theo thứ tự từ 1 đến 50). Chủ cửa hàng sẽ tặng cho người cuối cùng một món quà đặc biệt với thể thức cuộc chơi như sau. Chủ cửa hàng nói \lq\lq Các khách hàng số lẻ sẽ bị loại\rq\rq. Các khách hàng còn lại sẽ sắp xếp lại (theo thứ tự từ 1 đến hết), cuộc chơi sẽ dừng khi tìm được người cuối cùng. Vậy người may mắn nhận được quà thì bạn đã ở thứ tự số mấy?
	
	\shortans{$32$}
	
	\loigiai{
	\begin{itemize}
		\item Sắp xếp lần thứ nhất ta có\[a_1;a_2;\ldots;a_{50}.\]
		Loại tất cả các vị trí lẻ ta còn lại $25$ khách hàng ở các vị trí chẵn.
		\item Sắp xếp lần thứ hai cho $25$ vị khách còn lại ta được \[b_1;b_2;\ldots;b_{25}~\text{với}~ b_{k}=a_{2k}.\]
		Loại tất cả các vị trí lẻ ta còn lại $12$ khách hàng ở các vị trí chẵn.
		\item Sắp xếp lần thứ ba cho $12$ vị khách còn lại ta được \[c_1;c_2;\ldots;c_{25}~\text{với}~ c_{k}=b_{2k}.\]
		Loại tất cả các vị trí lẻ ta còn lại $6$ khách hàng ở các vị trí chẵn.
		\item Sắp xếp lần thứ tư cho $6$ vị khách còn lại ta được \[d_1;d_2;\ldots;d_{6}~\text{với}~ d_{k}=c_{2k}.\]
		Loại tất cả các vị trí lẻ ta còn lại $3$ khách hàng ở các vị trí chẵn.
		\item Sắp xếp lần thứ năm cho $3$ vị khách còn lại ta được \[e_1;e_2;e_{3}~\text{với}~ e_{k}=d_{2k}.\]
		Loại tất cả các vị trí lẻ ta còn lại khách hàng ở vị trí $e_2$.
	\end{itemize}
	Từ đó suy ra được $e_2=d_4=c_8=b_{16}=a_{32}$.\\
	Vậy khách hàng nhận được giải thưởng đặc biệt là người ở vị trí thứ $32$.
	}
\end{ex}

%

\begin{ex}%[Nguồn: Bộ đề minh họa Moon 2024-2025]%[1D3H1-4]
	$\lim \left(-3n^3+2n^2-5\right)$ bằng
	\choice
	{$-3$}
	{$-6$}
	{\True $-\infty$}
	{$+\infty$}
	\loigiai{
	Ta có $\lim \left(-3n^3+2n^2-5\right)= \lim n^3 \left(-3+\dfrac{2}{n}-\dfrac{5}{n^3}\right)$.\\
	Vì $
	\heva{&\lim n^3=+\infty\\&\lim\left(-3+\dfrac{2}{n}- \dfrac{5}{n^3}\right)=-3<0}$ nên  $\lim n^3 \left(-3+\dfrac{2}{n}-\dfrac{5}{n^3}\right)=-\infty$.\\ Suy ra $\lim \left(-3n^3+2n^2-5\right)=-\infty$ .
	}
\end{ex}

%Câu trắc nghiệm 1

\begin{ex}%[Nguồn: Bộ đề minh họa Moon 2024-2025]%[1D3N2-3]
	$\lim\limits_{n\to +\infty} \dfrac{1}{5n+3}$ bằng
	\choice
	{\True $0$}
	{$\dfrac{1}{3}$}
	{$+\infty$}
	{$\dfrac{1}{5}$}
	\loigiai{
	Ta có $\lim\dfrac{1}{5n+3}=0$.
	}
\end{ex}

\begin{ex}%[Nguồn: Bộ đề minh họa Moon 2024-2025]%[1D5H2-3]
	Một bệnh viện thống kê chiều cao của $50$ trẻ sơ sinh $12$ ngày tuổi một cách ngẫu nhiên. Kết quả thu được như sau
	\begin{center}
		\begin{tabular}{|c|c|c|c|c|c|c|c|}
			\hline
			Độ cao (cm) & $[40;42)$ & $[42;44)$ & $[44;46)$ & $[46;48)$ & $[48;50)$ &$[50;52)$&$[52;54)$\\
			\hline
			Tần số & $4$ & $4$ & $5$ & $10$ & $14$&$8$&$5$\\
			\hline
		\end{tabular}
	\end{center}
	Bộ nào sau đây là tứ phân vị của mẫu số liệu ghép nhóm trên (làm tròn kết quả đến hàng phần mười).
	\choice
	{$Q_1=45{,}1$, $Q_2=47{,}2$, $Q_3=51{,}3$}
	{$Q_1=45{,}5$, $Q_2=47{,}5$, $Q_3=51{,}5$}
	{$Q_1=45{,}3$, $Q_2=48{,}3$, $Q_3=53{,}3$}
	{\True $Q_1=45{,}8$, $Q_2=48{,}3$, $Q_3=50{,}1$}
	\loigiai{
	Áp dụng công thức chung tính tứ phân vị $Q_i=u_m+\dfrac{i\cdot \dfrac{n}{4}-C}{n_m}\left(u_{m+1}-u_m\right)$ với $i=1,2,3$, trong đó $\left[u_m;u_{m+1}\right)$ là nhóm chứa tứ phân vị $Q_i$, $n_m$ là tần số nhóm chứa tứ phân vị $Q_i$, $C=n_1+\cdots+ n_{m-1}$.\\
	Cỡ mẫu $n=50$.
	\begin{itemize}
		\item[•] Tính $Q_1$:\\
		Ta có $\dfrac{n}{4}=12{,}5$, suy ra nhóm chứa $Q_1$ là $[44;46)$, có tần số $n_3=5$, tần số tích lũy $C=4+4=8$. Vậy
		\[Q_1=44+\dfrac{12{,}5-8}{5}\left(46-44\right)=45{,}8.\]
		\item[•] Tính $Q_2$:\\
		Ta có $2\cdot \dfrac{n}{4}=25$, suy ra nhóm chứa $Q_2$ là $[48;50)$, có tần số $n_5=14$, tần số tích lũy $C=4+4+5+10=23$. Vậy
		\[Q_2=48+\dfrac{2\cdot 12{,}5-23}{14}\left(50-48\right)\approx 48{,}3.\]
		\item[•] Tính $Q_3$:\\
		Ta có $3\cdot \dfrac{n}{4}=37{,}5$, suy ra nhóm chứa $Q_3$ là $[50;52)$, có tần số $n_6=8$, tần số tích lũy $C=4+4+5+10+14=37$. Vậy
		\[Q_3=50+\dfrac{3\cdot 12{,}5-37}{8}\left(52-50\right)\approx 50{,}1.\]
	\end{itemize}
	}
\end{ex}

\begin{ex}%[Nguồn: Bộ đề minh họa Moon 2024-2025]%[1D5H2-3]
	Một trang báo điện tử thống kê thời gian người sử dụng đọc thông tin trên trang trong mỗi lần truy cập ở bảng sau
	\begin{center}
		\begin{tabular}{|c|c|c|c|c|c|}
			\hline
			Thời gian đọc (phút)& $[0;2)$ & $[2;4)$ & $[4;6)$ & $[6;8)$ & $[8;10)$  \\
			\hline
			Số lượt truy cập& $45$ & $34$ & $23$ & $18$ & $5$ \\
			\hline
		\end{tabular}
	\end{center}
	Khoảng tứ phân vị của mẫu số liệu ghép nhóm trên (\textit{làm tròn kết quả đến hàng phần trăm}) là
	\choice
	{$5{,}28$}
	{$4{,}51$}
	{$2{,}25$}
	{\True $3{,}89$}
	\loigiai{
	Ta có cỡ mẫu $n=125$.\\
	Xét $\dfrac{n}{4}=31{,25}$. Suy ra tứ phân vị thứ nhất $Q_1$ thuộc nhóm $[0;2)$.
	\begin{eqnarray*}
		Q_1&=a_p+\dfrac{\dfrac{n}{4}-\left(m_1+\cdots +n_{p-1}\right)}{m_p}\cdot \left(a_{p+1}-a_p\right)\\
		&=0+\dfrac{31{,25}-0}{45}\cdot 2\\
		&=\dfrac{25}{18}.
	\end{eqnarray*}
	Xét $\dfrac{3n}{4}=93{,75}$. Suy ra tứ phân vị thứ ba $Q_3$ thuộc nhóm $[4;6)$.
	\begin{eqnarray*}
		Q_3&=a_p+\dfrac{\dfrac{3n}{4}-\left(m_1+\cdots +n_{p-1}\right)}{m_p}\cdot (a_{p+1}-a_p)\\
		&=4+\dfrac{93{,75}-79}{23}\cdot 2\\
		&=\dfrac{243}{46}.
	\end{eqnarray*}
	Do đó ta có khoảng tứ phân vị $\Delta Q =Q_3-Q_1=\dfrac{243}{46}-\dfrac{25}{18}\approx 3{,}89$.
	}
\end{ex}

%Câu trắc nghiệm 2

\begin{ex}%[Nguồn: Bộ đề minh họa Moon 2024-2025]%[1D5N1-1]
	Điểm kiểm tra giữa học kì I của lớp 11T được thống kê theo bảng sau
	\begin{center}
		\begin{tabular}{|c|c|c|c|c|c|}
			\hline
			Số điểm & $[0;2)$ & $[2;4)$ & $[4;6)$ & $[6;8)$ & $[8;10)$ \\
			\hline
			Số học sinh & $1$ & $3$ & $8$ & $18$ & $10$ \\
			\hline
		\end{tabular}
	\end{center}
	Độ dài các nhóm của mẫu số liệu ghép nhóm trên là
	\choice
	{\True $2$}
	{$1$}
	{$6$}
	{$4$}
	\loigiai{Độ dài các nhóm của mẫu số liệu ghép nhóm trên là $2-0=2$.}
\end{ex}

%

\begin{ex}%[Nguồn: Bộ đề minh họa Moon 2024-2025]%[1D5N1-2]
	Thống kê số phút học bài ở nhà buổi tối của $100$ học sinh ta có bảng phân bố tần số ghép nhóm như sau:
	\begin{center}
		\begin{tabular}{|c|c|c|c|c|}
			\hline Số phút & {$[30 ; 60)$} & {$[60 ; 90)$} & {$[90 ; 120)$} & {$[120 ; 150)$} \\
			\hline Số học sinh & $18$ & $15$ & $42$ & $25$ \\
			\hline
		\end{tabular}
	\end{center}
	Số học sinh học bài ở nhà buổi tối ít hơn $120$ phút là
	\choice
	{$42$ học sinh}
	{$33$ học sinh}
	{\True $75$ học sinh}
	{$57$ học sinh}
	\loigiai{
	Số học sinh học bài ở nhà buổi tối ít hơn $120$ phút là $18+15+42=75$ học sinh.
	}
\end{ex}

\begin{ex}%[Nguồn: Bộ đề minh họa Moon 2024-2025]%[1D5N1-3]
	Bảng dưới đây thống kê chiều cao của học sinh nữ lớp $12$.
	\begin{center}
		\begin{tabular}{|l|c|c|c|c|c|}
			\hline Chiều cao (cm) &{$[160 ; 164)$}&{$[164 ; 168)$}&{$[168 ; 172)$}&{$[172 ; 176)$}&{$[176 ; 180)$}\\
			\hline Số học sinh & $5$ & $6$ & $8$ & $2$ & $1$ \\
			\hline
		\end{tabular}
	\end{center}
	Số trung bình của mẫu số liệu ghép nhóm (kết quả làm tròn đến hàng phần chục) là
	\choice
	{\True $167{,}8$}
	{$167{,}1$}
	{$170{,}0$}
	{$168{,}2$}
	\loigiai{
	Số trung bình của mẫu số liệu là
	\[\overline{x}=\dfrac{162 \cdot 5+166 \cdot 6+170 \cdot 8+174 \cdot 2+178 \cdot 1}{22}=\dfrac{3692}{22} \approx 167,8.\]
	}
\end{ex}

\begin{ex}%[Nguồn: Bộ đề minh họa Moon 2024-2025]%[1D6H2-2]
	Với $a$ là số thực dương tùy ý, $\log \dfrac{a^2}{100}$ bằng
	\choice
	{$2\log a-10$}
	{$\dfrac{1}{2}(\log a-2)$}
	{\True$2\log a-2$}
	{$\log a-5$}
	\loigiai{
	Ta có $\log \dfrac{a^2}{100}=\log a^2-\log 100=2\log a-2$.
	}
\end{ex}

\begin{ex}%[Nguồn: Bộ đề minh họa Moon 2024-2025]%[1D6H2-2]
	Cho $a$, $b$ là hai số thực dương thỏa mãn $a^{3}\cdot b^{5}=\mathrm{e}^{9}$. Giá trị của biểu thức $3\ln a+5 \ln b$ bằng
	\choice
	{$\mathrm{e}^{9}$}
	{\True $9$}
	{$\ln 9$}
	{$9\mathrm{e}$}
	\loigiai{Ta có $a^3\cdot b^5=\mathrm{e}^9\Leftrightarrow\ln\left(a^3\cdot b^5 \right)=\ln\left(\mathrm{e}^9 \right) \Leftrightarrow\ln a^3+\ln b^9=9\Leftrightarrow3\ln a+5\ln b=9$.\\
	Vậy $3\ln a+5\ln b=9$.}
\end{ex}

%

\begin{ex}%[Nguồn: Bộ đề minh họa Moon 2024-2025]%[1D6H2-2]
	Cho $a$, $b$ là các số thực dương thỏa mãn $\log_5 \dfrac{a}{b}=\log_{25} a^4$. Khẳng định nào sau đây đúng?
	\choice
	{$a=b$}
	{$a^2 b=1$}
	{$a b^2=1$}
	{\True $a b=1$}
	\loigiai{
	\begin{eqnarray*}
		\log_5 \dfrac{a}{b}=\log_{25} a^4&\Leftrightarrow&\log_5 a -\log_5 b=2\log_5 a\\
		&\Leftrightarrow& \log_5 a+\log_5 b=0\\
		&\Leftrightarrow&\log_5 ab=0 \\
		&\Leftrightarrow& ab=1.
	\end{eqnarray*}}
\end{ex}

\begin{ex}%[Nguồn: Bộ đề minh họa Moon 2024-2025]%[1D6H2-3]
	Với $a$ là số thực dương tùy ý, $\log_{\sqrt{3}}\left(9a^3\right)$ bằng
	\choice
	{\True $4 + 6\log_3a$}
	{$1 + \dfrac{3}{2}\log_3a$}
	{$4 - 6\log_3a$}
	{$1 - \dfrac{3}{2}\log_3a$}
	\loigiai{
	Ta có $\log_{\sqrt{3}}\left(9a^3\right)=2\log_3\left(9a^3\right)=2\left(\log_39+\log_3a^3\right)=2\left(2+3\log_3a\right)=4+6\log_3a$.
	}
\end{ex}

\begin{ex}%[Nguồn: Bộ đề minh họa Moon 2024-2025]%[1D6H2-3]
	Với $a$ là số thực dương tùy ý, khẳng định nào sau đây đúng?
	\choice
	{\True $\log\left(100a\right)=2+\log a$}
	{$\log\left(100a\right)=\dfrac{1}{2}+\log a$}
	{$\log\left(100a\right)=2\log a$}
	{$\log\left(100a\right)=\left(\log a\right)^2$}
	\loigiai{
	Ta có $\log\left(100a\right)=\log 100+\log a=2+\log a$.
	}
\end{ex}

\begin{ex}%[Nguồn: Bộ đề minh họa Moon 2024-2025]%[1D6H4-2]
	Tập nghiệm của bất phương trình $3^{-x} \geq \dfrac{1}{27}$ là
	\choice
	{$(-\infty; -3]$}
	{$[3; +\infty)$}
	{$[-3; +\infty)$}
	{\True $(-\infty; 3]$}
	\loigiai{Ta có $3^{-x} \geq \dfrac{1}{27}\Leftrightarrow 3^{-x}\geq 3^{-3}\Leftrightarrow -x\geq -3\Leftrightarrow x\leq 3.$\\
	Vậy bất phương trình có tập nghiệm $(-\infty; 3]$.
	}
\end{ex}

%

\begin{ex}%[Nguồn: Bộ đề minh họa Moon 2024-2025]%[1D6H4-2]
	Cho các số thực dương $a$, $b$ thỏa mãn $3\log a+2\log b=1$. Mệnh đề nào sau đây đúng?
	\choice
	{$a^3+b^2=1$}
	{$3a+2b=10$}
	{\True $a^3b^2=10$}
	{$a^3+b^2=10$}
	\loigiai{
	Ta có
	\[
	3\log a+2\log b=1\Leftrightarrow
	\log a^3+\log b^2 =1 \Leftrightarrow \log \left(a^3b^2\right) =1 \Leftrightarrow a^3b^2=10.
	\]
	}
\end{ex}

\begin{ex}%[Nguồn: Bộ đề minh họa Moon 2024-2025]%[1D6H4-2]
	Nếu $\log_8p=m$ thì $\log_2p$ bằng
	\choice
	{$\dfrac{m}{3}$}
	{$\dfrac{3}{m}$}
	{$m^3$}
	{\True $3m$}
	\loigiai
	{
	$\log_8p=m \Rightarrow p=8^m \Rightarrow \log_2p=\log_28^m=m\log_28=3m$.
	}
\end{ex}

%

\begin{ex}%[Nguồn: Bộ đề minh họa Moon 2024-2025]%[1D6H4-3]
	Tập nghiệm của bất phương trình $\log_2(x-1) < 3$ là
	\choice
	{\True $(1; 9)$}
	{$(-\infty; 9)$}
	{$(9;+\infty)$}
	{$(1; 7)$}
	\loigiai{
	Điều kiện: $x-1>0 \Leftrightarrow x>1$.\\
	Ta có $\log _2 (x-1)<3 \Leftrightarrow x-1<2^{3} \Leftrightarrow x<9$.\\
	Kết hợp điều kiện, ta có tập nghiệm của bất phương trình là $(1 ; 9)$.
	}
\end{ex}

\begin{ex}%[Nguồn: Bộ đề minh họa Moon 2024-2025]%[1D6H4-3]
	Tìm tập nghiệm của bất phương trình $\left(\dfrac{1}{3}\right)^x \leq\left(\dfrac{1}{3}\right)^{-x+2}$.
	\choice
	{$(-\infty; 1)$}
	{\True$[1;+\infty)$}
	{$(-\infty; 1]$}
	{$(1;+\infty)$}
	\loigiai{
	$\left(\dfrac{1}{3}\right)^x \leq\left(\dfrac{1}{3}\right)^{-x+2}\Leftrightarrow x\geq -x+2 \Leftrightarrow 2x\geq 2 \Leftrightarrow x\geq1$.\\
	Vậy tập nghiệm của bất phương trình đã cho là $[1;+\infty)$.
	}
\end{ex}

\begin{ex}%[Nguồn: Bộ đề minh họa Moon 2024-2025]%[1D6H4-3]
	Tập nghiệm $S$ của bất phương trình $\log_{\tfrac{1}{2}}\left(x+1\right)<\log_{\tfrac{1}{2}}\left(2x-1\right)$ là
	\choice
	{\True $\left(\dfrac{1}{2}; 2\right)$}
	{$(-\infty; 2)$}
	{$(2; +\infty)$}
	{$(-1; 2)$}
	\loigiai{
	Điều kiện $\heva{&x+1>0	\\&2x-1>0}\Leftrightarrow x>\dfrac{1}{2}$.\\
	Ta có $\log_{\tfrac{1}{2}}(x+1)<\log_{\tfrac{1}{2}}(2x-1)\Leftrightarrow x+1>2x-1\Leftrightarrow x<2$.\\
	Kết hợp điều kiện ta được $S=\left(\dfrac{1}{2}; 2\right)$.
	}
\end{ex}

\begin{ex}%[Nguồn: Bộ đề minh họa Moon 2024-2025]%[1D6H4-3]
	Tập nghiệm của bất phương trình $\log_{\dfrac{1}{3}}{\left(x+1 \right)} \geq 0$ là
	\choice
	{$\left[0; +\infty \right)$}
	{$\left(-\infty; 0 \right]$}
	{$\left(-1; -\dfrac{1}{2} \right]$}
	{\True $\left(-1; 0 \right]$}
	\loigiai
	{Điều kiện $x + 1 > 0 \Leftrightarrow x >-1$.\\
	Ta có $\log_{\dfrac{1}{3}}{\left(x+1 \right)} \geq 0  \Leftrightarrow \log_{\dfrac{1}{3}}{\left(x+1 \right)} \geq \log_{\dfrac{1}{3}}{1}  \Leftrightarrow x + 1 \leq 1 \Leftrightarrow x \leq 0$.\\
	Tập nghiệm của bất phương trình là $\left(-1; 0 \right]$.
	}
\end{ex}

\begin{ex}%[Nguồn: Bộ đề minh họa Moon 2024-2025]%[1D6H4-3]
	Tập nghiệm của bất phương trình $\log_{\tfrac{1}{2}}x\geq-3$ là
	\choice
	{$S=(0;8]$}
	{\True $S=[0;8]$}
	{$S=(-\infty;8]$}
	{$S=[8;+\infty)$}
	\loigiai{
	Điều kiện $x>0$.
	\begin{eqnarray*}
		&&\log_{\tfrac{1}{2}}x\geq-3\\
		&\Leftrightarrow& x\leq \left(\dfrac{1}{2}\right)^{-3}\\
		&\Leftrightarrow& x\leq 8.
	\end{eqnarray*}
	Kết hợp điều kiện suy ra tập nghiệm của bất phương trình là
	$S=[0;8]$.
	}
\end{ex}

%

\begin{ex}%[Nguồn: Bộ đề minh họa Moon 2024-2025]%[1D6H4-4]
	Gọi $S$ là tập nghiệm của phương trình $9^x-10\cdot 3^x+9=0$. Tổng các phần tử của $S$ bằng
	\choice
	{$1$}
	{\True $2$}
	{$10$}
	{$\dfrac{10}{3}$}
	\loigiai{
	Ta có
	\allowdisplaybreaks
	\begin{eqnarray*}
		&&9^x-10\cdot 3^x+9=0\\
		&\Leftrightarrow& (3^x)^2-10\cdot 3^x+9=0\\
		&\Leftrightarrow&\hoac{&3^x=9\\&3^x=1}\Rightarrow\hoac{&x=2\\&x=0.}
	\end{eqnarray*}
	Vậy tổng các phần tử của $S$ là $2+0=2$.
	}
\end{ex}

%

\begin{ex}%[Nguồn: Bộ đề minh họa Moon 2024-2025]%[1D6H4-5]
	Tìm tập nghiệm $S$ của bất phương trình
	$\log_{\tfrac{1}{5}}\left(x^2-1\right) < \log_{\tfrac{1}{5}}(3x-3)$.
	\choice
	{\True $S=(2;+\infty)$}
	{$S=(-\infty; 1) \cup(2;+\infty)$}
	{$S=(-\infty;-1) \cup(2;+\infty)$}
	{$S=(1; 2)$}
	\loigiai{
	Phương trình đã cho tương đương
	\[
	\heva{&x^2-1>3x-3\\&3x-3>0}\Leftrightarrow \heva{&x^2-3x+2>0\\&x>1} \Leftrightarrow \heva{&\hoac{&x<1\\&x>2}\\&x>1} \Leftrightarrow x>2.
	\]
	Vậy tập nghiệm của bất phương trình là $S=(2;+\infty)$.
	}
\end{ex}

\begin{ex}%[Nguồn: Bộ đề minh họa Moon 2024-2025]%[1D6H4-5]
	Tập nghiệm của bất phương trình $5^{x^2}\le 25^x$ chứa bao nhiêu số nguyên?
	\choice
	{\True $3$}
	{$1$}
	{$2$}
	{$4$}
	\loigiai
	{
	$5^{x^2}\le 25^x \Leftrightarrow x^2 \leq 2x \Leftrightarrow x^2-2x\leq 0 \Leftrightarrow 0\leq x \leq 2$.\\
	Suy ra tập nghiệm của bất phương trình chứa các số nguyên là $0$; $1$; $2$.\\
	Vậy tập nghiệm của bất phương trình chứa ba số nguyên.
	}
\end{ex}

\begin{ex}%[Nguồn: Bộ đề minh họa Moon 2024-2025]%[1D6H4-5]
	Bất phương trình $\left(\dfrac{1}{2}\right)^{x^2+4x} > \dfrac{1}{32}$ có tập nghiệm là $S=(a; b)$. Khi đó giá trị của $b-a$ là
	\choice
	{$4$}
	{$2$}
	{\True $6$}
	{$8$}
	\loigiai{
	\begin{eqnarray*}
		\left(\dfrac{1}{2}\right)^{x^2+4x} > \dfrac{1}{32}& \Leftrightarrow&\left(\dfrac{1}{2}\right)^{x^2+4 x}>\left(\dfrac{1}{2}\right)^5 \\
		& \Leftrightarrow& x^2+4 x<5 \\
		& \Leftrightarrow& x^2+4 x-5<0.
	\end{eqnarray*}
	Vậy bất phương trình có tập nghiệm là $(-5 ; 1)$. Khi đó $b=1, a=-5 \Rightarrow b-a=1+5=6 .$}
\end{ex}

\begin{ex}%[Nguồn: Bộ đề minh họa Moon 2024-2025]%[1D6N1-2]
	Cho $x$, $y$ là hai số thực dương và $m$, $n$ là hai số thực tùy ý. Đẳng thức nào sau đây là \textbf{sai}?
	\choice
	{$x^m\cdot  x^n = x^{m+n}$}
	{\True $x^m\cdot  y^n = (xy)^{m+n}$}
	{$(xy)^n = x^n y^n$}
	{$(x^n)^m = x^{nm}$}
	\loigiai{
	Đẳng thức sai là $x^m y^n = (xy)^{m+n}$.
	}
\end{ex}

\begin{ex}%[Nguồn: Bộ đề minh họa Moon 2024-2025]%[1D6N2-1]
	Với $a$, $b$ là các số thực dương khác $1$ thoả mãn $\log_a b=\dfrac{3}{2}$. Giá trị của biểu thức $\log_a b^4$ bằng
	\choice
	{$\dfrac{8}{3}$}
	{$3$}
	{$\dfrac{4}{3}$}
	{\True $6$}
	\loigiai{
	Ta có $\log_a b^4=4\log_a b=4\cdot \dfrac{3}{2}=6$.
	}
\end{ex}

%

\begin{ex}%[Nguồn: Bộ đề minh họa Moon 2024-2025]%[1D6N2-1]
	Với mọi số thực $a$ dương, $\log_3^2(a^2)$ bằng
	\choice
	{$2\log_3^2a$}
	{$\dfrac{1}{4}\log_3^2a$}
	{\True $4\log_3^2a$}
	{$\dfrac{1}{2}\log_3^2a$}
	\loigiai{Ta có $\log_3^2 (a^2)=\left(\log_3(a^2)\right)^2=\left(2\cdot\log_3 a\right)^2=4\log_3^2 a$.}
\end{ex}

\begin{ex}%[Nguồn: Bộ đề minh họa Moon 2024-2025]%[1D6N2-2]
	Với $a$, $b$ là các số thực dương tuỳ ý và $a \neq 1$, $\log_{\dfrac{1}{a}}\left(\dfrac{1}{b^3}\right)$ bằng
	\choice
	{$3\log_{a}b$}
	{$\log_{a}b$}
	{$-3\log_{a}b$}
	{$-\dfrac{1}{3}\log_{a}b$}
	\loigiai{
	Ta có
	\[\log_{\dfrac{1}{a}}\left(\dfrac{1}{b^3}\right)=\log_{a^{-1}}b^{-3}=\dfrac{-3}{-1}\log _ab=3\log_ab.\]
	}
\end{ex}

\begin{ex}%[Nguồn: Bộ đề minh họa Moon 2024-2025]%[1D6N2-2]
	Với số thực dương $a$ tuỳ ý, biểu thức $\log_2(a^3)$ bằng
	\choice
	{$\dfrac{1}{3}+\log_2a$}
	{$\dfrac{1}{3}\log_2a$}
	{$3+\log_2a$}
	{\True $3\log_2a$}
	\loigiai{
	Ta có $\log_2(a^3)=3\log_2a$.
	}
\end{ex}

\begin{ex}%[Nguồn: Bộ đề minh họa Moon 2024-2025]%[1D6N2-2]
	Hàm số nào dưới đây là hàm số mũ?
	\choice
	{$y=x^{2025}$}
	{$y=x^{-5}$}
	{$y=\log _3 x$}
	{\True $y=2025^x$}
	\loigiai{
	Hàm số mũ là hàm số cho bởi $y=a^x$ với $a\in\mathbb{R}$,  $0<a\ne 1$.
	}
\end{ex}

%Câu trắc nghiệm 8

\begin{ex}%[Nguồn: Bộ đề minh họa Moon 2024-2025]%[1D6N3-2]
	%%%% Tập xác định của hàm số $y=5^x+\log_2(3-x)$ là %%% Đây là đề gốc, nhưng không có đáp án.
	Tập xác định của hàm số $y=5^x+\log_2(x-3)$ là
	\choice
	{\True $(3;+\infty)$}
	{$(0;3)$}
	{$[3;+\infty)$}
	{$[0;3]$}
	\loigiai{
	Điều kiện xác định $x-3>0 \Leftrightarrow x>3$.\\
	Tập xác định của hàm số là $\mathscr{D}=(3;+\infty)$.
	}
\end{ex}

%

\begin{ex}%[Nguồn: Bộ đề minh họa Moon 2024-2025]%[1D6N3-2]
	Tập xác định của hàm số $y = \log_{\frac{1}{5}}(x-2)$ là
	\choice
	{$(-\infty;+\infty)$}
	{\True $(2;+\infty)$}
	{$\left[2;+\infty\right)$}
	{$\left(\dfrac{1}{5};+\infty\right)$}
	\loigiai{Điều kiện xác định của hàm số là $x-2>0\Leftrightarrow x>2$.\\
	Vậy tập xác định của hàm số là $\mathscr{D}=(2;+\infty)$.
	}
\end{ex}

\begin{ex}%[Nguồn: Bộ đề minh họa Moon 2024-2025]%[1D6N3-2]
	Tập xác định của hàm số $y=2^x$ là
	\choice
	{\True $\mathbb{R}$}
	{$\left(0;+\infty\right)$}
	{$\left[0;+\infty\right)$}
	{$\mathbb{R}\setminus\left\{0\right\}$}
	\loigiai{
	Tập xác định của hàm số $y=2^x$ là $\mathbb{R}$.
	}
\end{ex}

%

\begin{ex}%[Nguồn: Bộ đề minh họa Moon 2024-2025]%[1D6N3-2]
	Hàm số nào dưới đây có tập xác định là $\mathbb{R}$?
	\choice
	{$y=2^{\tfrac{1}{x}}$}
	{\True $y=\dfrac{1}{e^x}$}
	{$y=\ln |x|$}
	{$y=2^{\sqrt{x}}$}
	\loigiai{
	}
\end{ex}

%

\begin{ex}%[Nguồn: Bộ đề minh họa Moon 2024-2025]%[1D6N4-2]
	Nghiệm của phương trình $2^x=6$ là
	\choice
	{$x=\log_6 2$}
	{$x=3$}
	{$x=4$}
	{\True $x=\log_2 6$}
	\loigiai{
	Ta có $2^{x}=6 \Leftrightarrow x=\log _2  6$.
	}
\end{ex}

\begin{ex}%[Nguồn: Bộ đề minh họa Moon 2024-2025]%[1D6N4-2]
	Nghiệm của phương trình $\log_2\left( 3^x-1\right)=3$.
	\choice
	{$x=3$}
	{$x=1$}
	{$x=0$}
	{\True $x=2$}
	\loigiai{Điều kiện: $3^x-1>0$.\\
	$\log_2\left( 3^x-1\right)=3\Leftrightarrow3^x-1=8\Leftrightarrow3^x=9\Leftrightarrow x=2$ (nhận).
	}
\end{ex}

\begin{ex}%[Nguồn: Bộ đề minh họa Moon 2024-2025]%[1D6N4-2]
	Số nghiệm của phương trình $\log_{2}x = \log_{2}(x^{2}-x)$ là
	\choice
	{$0$}
	{\True $1$}
	{$2$}
	{$3$}
	\loigiai{
	Điều kiện xác định: $ x > 1$.\\
	Phương trình tương đương với $x = x^2 - x \Leftrightarrow x^2 - 2x = 0 \Leftrightarrow x(x-2) = 0\Leftrightarrow \hoac{&x=0\quad(\text{loại})\\&x=2\quad (\text{thỏa mãn}).}$\\
	Vậy phương trình có một nghiệm duy nhất $x = 2$.
	}
\end{ex}

%

\begin{ex}%[Nguồn: Bộ đề minh họa Moon 2024-2025]%[1D6N4-2]
	Phương trình $2^x=7$ có nghiệm là
	\choice
	{\True $x=\log_2 7$}
	{$x=\log_7 2$}
	{$x=3$}
	{$x=2$}
	\loigiai{
	Ta có $2^x=7\Leftrightarrow x=\log_2 7$.}
\end{ex}

%Câu trắc nghiệm 5

\begin{ex}%[Nguồn: Bộ đề minh họa Moon 2024-2025]%[1D6N4-3]
	Tập nghiệm của bất phương trình $\log x \leq-2$ là
	\choice
	{$\left(-\infty;\dfrac{1}{100}\right]$}
	{$\left(0;100\right]$}
	{\True $\left(0;\dfrac{1}{100}\right]$}
	{$\left[\dfrac{1}{100};+\infty\right)$}
	\loigiai{
	Điều kiện xác định $x > 0$.\\
	Ta có $\log x \leq-2 \Leftrightarrow x \leq 10^{-2} \Leftrightarrow x \leq \dfrac{1}{100}$.\\
	Vậy tập nghiệm của bất phương trình là $\left(0;\dfrac{1}{100}\right]$.
	}
\end{ex}

\begin{ex}%[Nguồn: Bộ đề minh họa Moon 2024-2025]%[1D6N4-3]
	Tập nghiệm của bất phương trình $\log_{\frac{1}{2}}\left(x+2\right)>-1$ là
	\choice
	{$\left(0;+\infty \right)$}
	{$\left(-\infty;0\right)$}
	{\True $\left(-2;0\right)$}
	{$\left(-2;1\right)$}
	\loigiai{ Ta có
	\[\log_{\frac{1}{2}}\left(x+2\right)>-1\Leftrightarrow \heva{&x+2>0\\&x+2<2}\Leftrightarrow \heva{&x>-2\\&x<0}\Leftrightarrow -2<x<0.\]
	Vậy, tập nghiệm của bất phương trình là $\left(-2;0\right)$.
	}
\end{ex}

\begin{ex}%[Nguồn: Bộ đề minh họa Moon 2024-2025]%[1D6N4-5]
	Tập nghiệm của bất phương trình $3^{2x-3} \leq \dfrac{1}{3}$ là
	\choice
	{$(-\infty; 1]$}
	{$[1;+\infty)$}
	{\True $(-\infty; 2]$}
	{$[2;+\infty)$}
	\loigiai{
	Ta có $3^{2x-3} \leq \dfrac{1}{3}\Leftrightarrow 2x-3\leq -1\Leftrightarrow x\leq 2$.\\
	Tập nghiệm của bất phương trình $3^{2x-3} \leq \dfrac{1}{3}$ là $(-\infty; 2]$.
	}
\end{ex}

%

\begin{ex}%[Nguồn: Bộ đề minh họa Moon 2024-2025]%[1D7H2-1]
	Cho hàm số $f(x)=4\sin x \cos x+2x$.
	\choiceTF
	{\True $f(0)=0$, $f(\pi)=2\pi$}
	{Đạo hàm của hàm số đã cho là $f'(x)=4\sin 2x+2$}
	{\True Hàm số $y=f(x)$ có hai điểm cực trị thuộc đoạn $[0; \pi]$}
	{Giá trị lớn nhất của $f(x)$ trên đoạn $[0; \pi]$ là $\dfrac{2\pi}{3}+\sqrt{3}$}
	\loigiai{
	Ta có\\
	$f(x)=4\sin x \cos x+2x=2\sin 2x+2x$.\\
	Đạo hàm $f'(x)=(2\sin 2x+2x)'=4\cos 2x+2$.
	\allowdisplaybreaks
	\begin{eqnarray*}
		&&f'(x)=0\\
		&\Leftrightarrow& 4\cos 2x+2=0\\
		&\Leftrightarrow& \cos 2x=-\dfrac{1}{2}\\
		&\Leftrightarrow& 2x=\pm \dfrac{2\pi}{3}+k 2\pi\\
		&\Leftrightarrow& x=\pm \dfrac{\pi}{3}+k \pi,\, k\in \mathbb{Z}.
	\end{eqnarray*}
	Xét $x\in [0; \pi]\Rightarrow x \in \left\{\dfrac{\pi}{3};\dfrac{2\pi}{3}\right\}$.\\
	Bảng biến thiên\\
	\centerline{
	\begin{tikzpicture}[font=\normalsize]
		%dòng khai báo
		\tkzTabInit[nocadre=true,lgt=1.2,espcl=2.5,deltacl=0.65]
		{$x$ /0.75, $f'(x)$/0.75, $f(x)$/2}
		{$0$,$\tfrac{\pi}{3}$,$\tfrac{2\pi}{3}$,$\pi$}
		%dòng xét dấu
		\tkzTabLine{,+,0,-,0,+}
		%dòng biến thiên
		\tkzTabVar{-/$0$,+/$f(\tfrac{\pi}{3})$,-/$f(\tfrac{2\pi}{3})$,+/$2\pi$}
	\end{tikzpicture}
	}
	\begin{itemchoice}
		\itemch \textbf{Đúng.} \\
		Vì $f(0)=2\sin(2\cdot 0)+2\cdot 0=0$ và $f(\pi)=2\sin (2\pi)+2\pi=2\pi$.
		\itemch \textbf{Sai.} \\
		\itemch \textbf{Đúng.} \\
		Dựa vào bảng biến thiên, ta thấy $f(x)$ có hai điểm cực trị trên đoạn $[0; \pi]$.
		\itemch \textbf{Sai.} \\
		Vì $f\left(\dfrac{\pi}{3}\right)=\sqrt{3}+\dfrac{2\pi}{3}<2\pi$.
	\end{itemchoice}
	}
\end{ex}

\begin{ex}%[Nguồn: Bộ đề minh họa Moon 2024-2025]%[1D7N2-1]
	Đạo hàm của hàm số $y=\ln (x+1)$ trên khoảng $(-1;+\infty)$ là
	\choice
	{$y'=\dfrac{1}{(x+1)^2}$}
	{$y'=\dfrac{-1}{x+1}$}
	{$y'=\dfrac{-1}{(x+1)^2}$}
	{\True $y'=\dfrac{1}{x+1}$}
	\loigiai{ Vì $[\ln u]'=\dfrac{u^{\prime}}{u}$ nên ta có $y'=\dfrac{1}{x+1}$.
	
	}
\end{ex}

%

\begin{ex}%[Nguồn: Bộ đề minh họa Moon 2024-2025]%[1D7N2-1]
	Vào lúc $12$ giờ trưa, tàu $B$ đang nằm ở vị trí $O$, tàu $A$ cách tàu $B$ $12$ km. Tàu $A$ đang di chuyển về phía $O$ với vận tốc $12$ km/h và tiếp tục di chuyển như vậy cả ngày. Tàu $B$ có vận tốc $8$ km/h đang di chuyển theo hướng vuông góc với hướng đi của tàu $A$ và tiếp tục di chuyển như vậy cả ngày. Quãng đường tàu $A$ và tàu $B$ di chuyển được sau $t$ (giờ) (tính từ lúc $12$ giờ trưa lần lượt là $S_A$ và $S_B$.%[1D7V1-4]
	\choiceTF
	{\True $S_A=12t$ (km) và $S_B=8t$ (km)}
	{Khoảng cách giữa $2$ tàu được xác định bởi công thức $S=\sqrt{S_A^2+S_B^2}$ (km)}
	{Lúc $13$ giờ, khoảng cách giữa $2$ tàu bằng $8\sqrt{10}$ (km)}
	{Lúc $13$ giờ, tốc độ thay đổi khoảng cách giữa $2$ tàu bằng $\dfrac{22\sqrt{10}}{5}$ km/h}
	\loigiai{
	\begin{itemchoice}
		\itemch {\bf Đúng}.\\
		Quãng đường tàu $A$ đi được sau $t$ giờ là $S_A=12t$ (km).\\
		Quãng đường tàu $B$ đi được sau $t$ giờ là $S_B=8t$ (km).
		\itemch {\bf Sai}.\\
		Gọi $M$ là vị trí ban đầu của tàu $A$, sau $t$ giờ tàu $A$ đến được vị trí $A$ mới và đi được quãng đường $MA=S_A$, và tàu $B$ đến vị trí $B$ như hình vẽ.
		\begin{center}
			\begin{tikzpicture}[scale=1, font=\footnotesize,line join=round, line cap=round, >=stealth]
				\path
				(0,0) coordinate (O)
				+(-90:3) coordinate (x)
				+(-90:2) coordinate (B)
				+(180:5) coordinate (M)
				+(180:4) coordinate (A)
				;
				\draw[-stealth] (M)--(A);
				\draw[-stealth] (O)--(B);
				\draw (A)--(O)--(x)
				(B)--(A)
				;
				\draw[stealth-stealth] ($(M)+(-90:.3)$)--($(A)+(-90:.3)$) node[midway,below] {$S_A$};
				\draw[stealth-stealth] ($(O)+(0:.5)$)--($(B)+(0:.5)$) node[midway,right] {$S_B$};
				\foreach \x/\g in {M/90,A/90,O/45,B/0}\fill (\x) circle (1pt)+(\g:3mm) node{$\x$};
			\end{tikzpicture}
		\end{center}
		Khoảng cách giữa hai tàu tại thời điểm $t$ là $$S(t)=\sqrt{(12-S_A)^2+S_B^2}=\sqrt{(12-12t)^2+(8t)^2}.$$
		\itemch {\bf Sai}.\\
		Lúc $13$ giờ tàu $B$ cách $O$ một khoảng là $OB=8\cdot 1=8$ (km).\\
		Tàu $A$ cách $O$ một đoạn $OA=12-12=0$ (km).\\
		Vậy khoảng cách giữa hai tàu lúc này chính bằng đoạn $OB=8$ (km).
		\itemch {\bf Sai}.\\
		Ta có
		$S(t)=\sqrt{(12-12t)^2+(8t)^2}=\sqrt{208t^2-288t+144}$.\\
		Vậy $S'(t)=\dfrac{416t-288}{2\sqrt{208t^2-288t+144}}$.\\
		Lúc $13$ giờ ứng với $t=1$ nên ta có
		$S'(1)=\dfrac{32}{2\sqrt{16}}=8$ km/h.
	\end{itemchoice}
	
	}
\end{ex}

%

\begin{ex}%[Nguồn: Bộ đề minh họa Moon 2024-2025]%[1D7V2-8]
	Một khinh khi cầu bay với độ cao (so với mặt đất) tại thời điểm $t$ là $h(t)$, trong đó $t$ tính bằng phút, $h(t)$ tính bằng mét. Vận tốc bay của khinh khí cầu được cho bởi hàm số $v(t)=-0{,}12t^2+1{,}2t$, với $t$ tính bằng phút, $v(t)$ tính bằng mét/phút. Từ thời điểm xuất phát $(t=0)$ thì $5$ phút sau khi xuất phát, khinh khí cầu đã ở độ cao $530$\,m.
	\begin{flushright}
		\textit{(Nguồn: Bigalke et al., Mathematik, Grundkurs ma-1, Cornelsen 2016.)}
	\end{flushright}
	\choiceTF
	{\True $\displaystyle\int v(t)\mathrm{\,d}t=-0{,}04t^3+0{,}6t^2+C$ với $C$ là hằng số}
	{\True Độ cao tối đa của khinh khí cầu khi bay là $540$\,mét}
	{\True Sau $15$ phút, khinh khí cầu sẽ trở lại độ cao khi xuất phát}
	{Khi bắt đầu tiếp đất tốc độ của khí cầu là $70{,}3$\,mét/phút (làm tròn kết quả đến hàng phần chục)}
	\loigiai{
	Ta có $h(t)=\displaystyle\int v(t)\mathrm{\,d}t
	=-0{,}04t^3+0{,}6t^2+C$.\\
	Mà $h(5)=530\Rightarrow C=520$.\\
	Do đó $h(t)=-0{,}04t^3+0{,}6t^2+520$\,(m).
	\begin{itemchoice}
		\itemch \textbf{Đúng.} \\
		
		\itemch \textbf{Đúng.} \\
		Vì $h'(t)=v(t)=-0{,}12t^2+1{,}2t \Rightarrow h'(t)=0\Rightarrow \hoac{&t=0\\&t=10.}$\\
		Khi đó $h(0)=520$; $h(10)=540$.\\
		Do đó độ cao tối đa của khinh khí cầu khi bay là $h(10)=540$\,mét.
		\itemch \textbf{Đúng.} \\
		Vì $h(15)=520=h(0)$.
		\itemch \textbf{Sai.} \\
		Khi tiếp đất thì độ cao của kinh khí cầu bằng $0$ hay
		\allowdisplaybreaks
		\begin{eqnarray*}
			&&h(t)=0\\
			&\Leftrightarrow&-0{,}04t^3+0{,}6t^2+520=0\\
			&\Leftrightarrow& t^3-15t^2-13\,000=0\\
			&\Leftrightarrow& t\approx 29{,}72.
		\end{eqnarray*}
		Do đó vận tốc của kinh khí cầu khi bắt đầu tiếp đất là $v(29{,}72)\approx -70{,}3$\,mét/phút.
	\end{itemchoice}
	}
\end{ex}

\begin{ex}%[Nguồn: Bộ đề minh họa Moon 2024-2025]%[1D9H1-3]
	Ba khẩu súng độc lập bắn vào một mục tiêu. Xác suất để khẩu thứ nhất bắn trúng bằng $0{,}7$ để khẩu thứ hai bắn trúng bằng $0{,}8$ để khẩu thứ ba bắn trúng bằng $0{,}5$. Mỗi khẩu bắn $1$ viên. Tính xác suất để khẩu thứ nhất bắn trúng biết rằng chỉ có $2$ viên trúng mục tiêu (\textit{kết quả làm tròn đến hàng phần trăm}).
	\shortans[]{$0{,}35$}
	\loigiai{
	Gọi $A$ là biến cố \lq\lq khẩu súng thứ nhất bắn trúng mục tiêu\rq\rq.\\
	Gọi $B$ là biến cố \lq\lq khẩu súng thứ hai bắn trúng mục tiêu\rq\rq.\\
	Gọi $C$ là biến cố \lq\lq khẩu súng thứ ba bắn trúng mục tiêu\rq\rq.\\
	Ta có xác suất để khẩu thứ nhất bắn trúng biết rằng chỉ có $2$ viên trúng mục tiêu là
	\begin{eqnarray*}
		\mathrm{P}=\mathrm{P}(AB\overline{C})+\mathrm{P}(A\overline{B}C)=0{,}7 \cdot 0{,}8 \cdot 0{,}5	+0{,}7 \cdot 0{,}2 \cdot 0{,}5=0{,}35.
	\end{eqnarray*}
	}
\end{ex}

%

\begin{ex}%[Nguồn: Bộ đề minh họa Moon 2024-2025]%[1D9H2-3]
	Một người đi săn, xác suất người thợ săn này bắn trúng thú trong mỗi lần bắn tỉ lệ nghịch với khoảng cách bắn. Trong một lần đi săn, anh ta bắn lần đầu ở khoảng cách $20 \mathrm{~m}$ với xác suất trúng thú là $50 \%$. Nếu bị trượt, anh ta sẽ bắn viên thứ hai ở khoảng cách $30 \mathrm{~m}$. Nếu lại trượt, anh ta sẽ bắn viên thứ ba ở khoảng cách $50 \mathrm{~m}$. Người thợ săn sẽ dừng bắn nếu bắn trúng thú và chỉ bắn tối đa ba lần. Biết rằng anh ta đã bắn trúng thú, xác suất người thợ săn bắn trúng ở lần bắn thứ ba là bao nhiêu phần trăm (làm tròn kết quả đến hàng phần trăm).
	\loigiai{
	Gọi $A$, $B$, $C$ lần lượt là biến cố bắn thú trúng trong lần thứ 1, 2, 3. Khi đó
	\[P(A)=0{,}5;\quad P(B)=\dfrac{1}{3};\quad P(C)=0{,}2.\]
	Xác suất người thợ săn bắn trúng ở lần bắn thứ ba là \[P(\overline A)\cdot P(\overline B)\cdot P(C)=0{,}5\cdot\dfrac{2}{3}\cdot0{,}2=\dfrac{1}{15}\approx6{,}67\%.\]
	}
\end{ex}

\begin{ex}%[Nguồn: Bộ đề minh họa Moon 2024-2025]%[1D9V1-4]
	Bắc đi từ nhà (điểm $A$) đến trường (điểm $B$). Biết rằng toàn bộ cung đường theo bản đồ từ trên xuống dưới, từ trái qua phải là đường một chiều, vì vậy Bắc chỉ được phép đi xuống hoặc đi sang phải.
	\begin{center}
		\begin{tikzpicture}[line join=round, line cap=round,>=stealth,thick]
			\foreach \diem/\hoanh/\tung in {A/0/0,A1/2/0,A2/4/0,A3/6/0,A4/8/0} \coordinate (\diem) at (\hoanh,\tung);\foreach \x in {A,A1,A2,A3,A4} \fill[black] (\x) circle (1.5pt);
			\foreach \diem/\hoanh/\tung in {B1/0/-2,B2/2/-2,B3/4/-2,B4/6/-2,B5/8/-2,B6/10/-2} \coordinate (\diem) at (\hoanh,\tung);\foreach \x in {B1,B2,B3,B4,B5,B6} \fill[black] (\x) circle (1.5pt);
			\foreach \diem/\hoanh/\tung in {C1/0/-4,C2/2/-4,C3/4/-4,C4/6/-4,C5/8/-4,C6/10/-4} \coordinate (\diem) at (\hoanh,\tung);\foreach \x in {C1,C2,C3,C4,C5,C6} \fill[black] (\x) circle (1.5pt);
			\foreach \diem/\hoanh/\tung in {D2/2/-6,D4/6/-6,D6/10/-6} \coordinate (\diem) at (\hoanh,\tung);\foreach \x in {D2,D4,D6} \fill[black] (\x) circle (1.5pt);
			\draw(A)--(A1)--(A4)--(B5)--(B6)--(D6)
			(A)--(C1)--(C4)--(D4)--(D6) (A1)--(D2)--(D4) (B1)--(B2) (A2)--(B3)--(B5)--(C5)--(C6) (B3)--(C3) (A3)--(B4)
			;
			\path (A) node[above]{$A$};
			\path (D6) node[below]{$B$};
		\end{tikzpicture}
	\end{center}
	Hỏi Bắc có bao nhiêu cách đến trường? \shortans[oly]{$13$}
	\loigiai{
	Đặt tên các nút giao thông như hình vẽ
	\begin{center}
		\begin{tikzpicture}[line join=round, line cap=round,>=stealth,thick]
			\foreach \diem/\hoanh/\tung in {A/0/0,A1/2/0,A2/4/0,A3/6/0,A4/8/0} \coordinate (\diem) at (\hoanh,\tung);\foreach \x in {A,A1,A2,A3,A4} \fill[black] (\x) circle (1.5pt);
			\foreach \diem/\hoanh/\tung in {B1/0/-2,B2/2/-2,B3/4/-2,B4/6/-2,B5/8/-2,B6/10/-2} \coordinate (\diem) at (\hoanh,\tung);\foreach \x in {B1,B2,B3,B4,B5,B6} \fill[black] (\x) circle (1.5pt);
			\foreach \diem/\hoanh/\tung in {C1/0/-4,C2/2/-4,C3/4/-4,C4/6/-4,C5/8/-4,C6/10/-4} \coordinate (\diem) at (\hoanh,\tung);\foreach \x in {C1,C2,C3,C4,C5,C6} \fill[black] (\x) circle (1.5pt);
			\foreach \diem/\hoanh/\tung in {D2/2/-6,D4/6/-6,D6/10/-6} \coordinate (\diem) at (\hoanh,\tung);\foreach \x in {D2,D4,D6} \fill[black] (\x) circle (1.5pt);
			\draw(A)--(A1)--(A4)--(B5)--(B6)--(D6)
			(A)--(C1)--(C4)--(D4)--(D6) (A1)--(D2)--(D4) (B1)--(B2) (A2)--(B3)--(B5)--(C5)--(C6) (B3)--(C3) (A3)--(B4)
			;
			\foreach \diem/\nhan in {A1/A1,A2/A2,A3/A3,A4/A4} \path (\diem) node[above]{$\nhan$};
			\foreach \diem/\nhan in {B1/B1,B2/B2,B3/B3,B4/B4,B5/B5,B6/B6} \path (\diem) node[above left]{$\nhan$};
			\foreach \diem/\nhan in {C1/C1,C2/C2,C3/C3,C4/C4,C5/C5,C6/C6} \path (\diem) node[above left]{$\nhan$};
			\foreach \diem/\nhan in {D2/D2,D4/D4} \path (\diem) node[below]{$\nhan$};
			\path (A) node[above]{$A$};
			\path (D6) node[below]{$B$};
		\end{tikzpicture}
	\end{center}
	Ta có sơ đồ cây
	\begin{center}
		\begin{tikzpicture}[scale=1, font=\footnotesize, line join=round, line cap=round, >=stealth]
			\path (0,0) node[shift=(90:0.3)]{A};
			\draw[->](0,0)--(-5,-1) node[shift=(-135:0.2)]{$A1$};
			\draw[->](0,0)--(4,-1) node[shift=(-45:0.2)]{$B1$};
			\draw[->](-5,-1)++(0,-.4)--(-4,-2) node[shift=(-90:0.2)]{$A2$};
			\draw[->](-5,-1)++(0,-.4)--(-8,-2) node[shift=(-90:0.2)]{$B2$};
			\draw[->](-4,-2)++(0,-.4)--(-5,-3) node[shift=(-90:0.2)]{$B3$};
			\draw[->](-8,-2) ++(0,-.4)--(-8,-3)node[shift=(-90:0.2)]{$C2$};
			\draw[->](-8,-3) ++(0,-.4)--(-8.5,-4)node[shift=(-90:0.2)]{$C3$};
			\draw[->](-8,-3) ++(0,-.4)--(-7.5,-4)node[shift=(-90:0.2)]{$D2$};
			\draw[->](-8.5,-4) ++(0,-.4)--(-8.5,-5)node[shift=(-90:0.2)]{$C4$};
			\draw[->](-8.5,-5) ++(0,-.4)--(-8.5,-6)node[shift=(-90:0.2)]{$D4$};
			\draw[->](-8.5,-6) ++(0,-.4)--(-8.5,-8)node[shift=(-90:0.2)]{$B$};
			\draw[->](-7.5,-4) ++(0,-.4)--(-7.5,-6)node[shift=(-90:0.2)]{$D4$};
			\draw[->](-7.5,-6) ++(0,-.4)--(-7.5,-8)node[shift=(-90:0.2)]{$B$};
			\draw[->](-5,-3)++(0,-.4)--(-6,-4) node[shift=(-90:0.2)]{$C3$};
			\draw[->](-6,-4)++(0,-.4)--(-6,-5) node[shift=(-90:0.2)]{$C4$};
			\draw[->](-6,-5)++(0,-.4)--(-6,-6) node[shift=(-90:0.2)]{$D4$};
			\draw[->](-6,-6)++(0,-.4)--(-6,-8) node[shift=(-90:0.2)]{$B$};
			\draw[->](-5,-3)++(0,-.4)--(-4,-4) node[shift=(-90:0.2)]{$B4$};
			\draw[->](-4,-4)++(0,-.4)--(-4,-5) node[shift=(-90:0.2)]{$B5$};
			\draw[->](-4,-5)++(0,-.4)--(-4,-6) node[shift=(-90:0.2)]{$B6$};
			\draw[->](-4,-6)++(0,-.4)--(-4,-7) node[shift=(-90:0.2)]{$C6$};
			\draw[->](-4,-7)++(0,-.4)--(-4,-8) node[shift=(-90:0.2)]{$B$};
			\draw[->](-4,-5)++(0,-.4)--(-5,-6) node[shift=(-90:0.2)]{$C5$};
			\draw[->](-5,-6)++(0,-.4)--(-5,-7) node[shift=(-90:0.2)]{$C6$};
			\draw[->](-5,-7)++(0,-.4)--(-5,-8) node[shift=(-90:0.2)]{$B$};
			\draw[->](-4,-2)++(0,-.4)--(-1,-3) node[shift=(-90:0.2)]{$A3$};
			\draw[->](-1,-3)++(0,-.4)--(-2,-4) node[shift=(-90:0.2)]{$A4$};
			\draw[->](-2,-4)++(0,-.4)--(-2,-5) node[shift=(-90:0.2)]{$B5$};
			\draw[->](-2,-5)++(0,-.4)--(-3,-6) node[shift=(-90:0.2)]{$C5$};
			\draw[->](-3,-6)++(0,-.4)--(-3,-7) node[shift=(-90:0.2)]{$C6$};
			\draw[->](-3,-7)++(0,-.4)--(-3,-8) node[shift=(-90:0.2)]{$B$};
			\draw[->](-2,-5)++(0,-.4)--(-2,-6) node[shift=(-90:0.2)]{$B6$};
			\draw[->](-2,-6)++(0,-.4)--(-2,-7) node[shift=(-90:0.2)]{$C6$};
			\draw[->](-2,-7)++(0,-.4)--(-2,-8) node[shift=(-90:0.2)]{$B$};
			\draw[->](-1,-3)++(0,-.4)--(-1,-4) node[shift=(-90:0.2)]{$B4$};
			\draw[->](-1,-4)++(0,-.4)--(-1,-5) node[shift=(-90:0.2)]{$B5$};
			\draw[->](-1,-5)++(0,-.4)--(0,-6) node[shift=(-90:0.2)]{$B6$};
			\draw[->](0,-6)++(0,-.4)--(0,-7) node[shift=(-90:0.2)]{$C6$};
			\draw[->](0,-7)++(0,-.4)--(0,-8) node[shift=(-90:0.2)]{$B$};
			\draw[->](-1,-5)++(0,-.4)--(-1,-6) node[shift=(-90:0.2)]{$C5$};
			\draw[->](-1,-6)++(0,-.4)--(-1,-7) node[shift=(-90:0.2)]{$C6$};
			\draw[->](-1,-7)++(0,-.4)--(-1,-8) node[shift=(-90:0.2)]{$B$};
			\draw[->](4,-1)++(0,-.4)--(2,-2) node[shift=(-90:0.2)]{$B2$};
			\draw[->](2,-2)++(0,-.4)--(2,-3) node[shift=(-90:0.2)]{$C2$};
			\draw[->](2,-3)++(0,-.4)--(1,-4) node[shift=(-90:0.2)]{$C3$};
			\draw[->](1,-4)++(0,-.4)--(1,-5) node[shift=(-90:0.2)]{$C4$};
			\draw[->](1,-5)++(0,-.4)--(1,-6) node[shift=(-90:0.2)]{$D4$};
			\draw[->](1,-6)++(0,-.4)--(1,-8) node[shift=(-90:0.2)]{$B$};
			\draw[->](2,-3)++(0,-.4)--(3,-4) node[shift=(-90:0.2)]{$D2$};
			\draw[->](3,-4)++(0,-.4)--(3,-6) node[shift=(-90:0.2)]{$D4$};
			\draw[->](3,-6)++(0,-.4)--(3,-8) node[shift=(-90:0.2)]{$B$};
			\draw[->](4,-1)++(0,-.4)--(5,-2) node[shift=(-90:0.2)]{$C1$};
			\draw[->](5,-2)++(0,-.4)--(5,-3) node[shift=(-90:0.2)]{$C2$};
			\draw[->](5,-3)++(0,-.4)--(6,-4) node[shift=(-90:0.2)]{$C3$};
			\draw[->](6,-4)++(0,-.4)--(6,-5) node[shift=(-90:0.2)]{$C3$};
			\draw[->](6,-5)++(0,-.4)--(6,-6) node[shift=(-90:0.2)]{$D4$};
			\draw[->](6,-6)++(0,-.4)--(6,-8) node[shift=(-90:0.2)]{$B$};
			\draw[->](5,-3)++(0,-.4)--(4,-4) node[shift=(-90:0.2)]{$D2$};
			\draw[->](4,-4)++(0,-.4)--(4,-6) node[shift=(-90:0.2)]{$D4$};
			\draw[->](4,-6)++(0,-.4)--(4,-8) node[shift=(-90:0.2)]{$B$};
		\end{tikzpicture}
	\end{center}
	Vậy có $13$ cách đến trường.
	}
\end{ex}

\begin{ex}%[Nguồn: Bộ đề minh họa Moon 2024-2025]%[1D9V2-2]
	Một đoàn tình nguyện đến một trường tiểu học miền núi để trao tặng cho $100$ em học sinh nghèo học giỏi. Đoàn tình nguyện có $70$ chiếc áo mùa đông, $90$ thùng sữa tươi và $40$ chiếc cặp sách được chia thành $100$ suất quà (mỗi suất quà gồm $2$ món quà: một chiếc áo và một thùng sữa tươi hoặc một chiếc áo và một cặp sách, hoặc một thùng sữa tươi và một cặp sách). Tất cả các suất quà đều có giá trị tương đương nhau. Trong số các em được nhận quà có hai em Việt và Nam. Gọi $P$ là xác suất để hai em Việt và Nam nhận được suất quà giống nhau. Tính $11000P$.
	
	\shortans[oly]{5000}
	\loigiai{
	Gọi $a$, $b$, $c$ lần lượt là số suất quà thuộc loại một chiếc áo và một thùng sữa tươi, một chiếc áo và một cặp sách, một thùng sữa tươi và một cặp sách.\\
	Khi đó ta có hệ:
	$$
	\heva{
	& a + b + c = 100, \\
	& a + b = 70, \\
	& a + c = 90, \\
	& b + c = 40.
	} \Leftrightarrow
	\heva{
	& a = 60, \\
	& b = 10, \\
	& c = 30.
	}
	$$
	Không gian mẫu $\Omega$ là \lq\lq Chọn $2$ suất quà trong $100$ suất quà\rq\rq.\\
	Khi đó $n(\Omega) = \mathrm{C}^2_{100} = 4\,950$.\\
	Gọi $A$ là biến cố: \lq\lq Bạn Việt và Nam nhận được phần quà giống nhau\rq\rq.\\
	Ta có các trường hợp sau
	\begin{itemize}
		\item Cả hai nhận suất quà loại: một chiếc áo và một thùng sữa tươi $\mathrm{C}_2^{60} = \dfrac{60 \cdot 59}{2} = 1\,770$.
		\item Cả hai nhận suất quà loại: một chiếc áo và một cặp sách $\mathrm{C}^2_{10}=45$.
		\item Cả hai nhận suất quà loại: một thùng sữa tươi và một cặp sách $\mathrm{C}^1_{30} = \dfrac{30 \cdot 29}{2} = 435$.
	\end{itemize}
	Tổng số cách để hai em Việt và Nam nhận được suất quà giống nhau là $n(A) = 1\,770 + 45 + 435 = 2\,250$.\\
	Xác suất để hai em Việt và Nam nhận được suất quà giống nhau là $\mathrm{P} = \dfrac{n(A)}{n(\Omega)} = \dfrac{2\,250}{4\,950} = \dfrac{5}{11}$.\\
	Khi đó $11\,000P = 11\,000 \cdot \dfrac{5}{11} = 5\,000$.
	}
\end{ex}

\begin{ex}%[Nguồn: Bộ đề minh họa Moon 2024-2025]%[1H8H1-3]
	Cho hình lập phương $ABCD.A_1B_1C_1D_1$ có cạnh $a$.
	\begin{center}
		\begin{tikzpicture}
			\def\a{3.5}
			\path (0:0) coordinate (A)
			++(50:\a/2) coordinate (D)
			++(0:\a) coordinate (C)
			($(A)+(C)-(D)$) coordinate (B)
			($(A)+(-90:\a)$) coordinate (A_1)
			($(B)+(-90:\a)$) coordinate (B_1)
			($(C)+(-90:\a)$) coordinate (C_1)
			($(D)+(-90:\a)$) coordinate (D_1)
			($(B)!0.5!(D)$) coordinate (I);
			\draw[dashed,thick] (D_1)--(D)--(A_1)--(D_1)--(C_1) (B_1)--(I);
			\draw[thick] (D)--(B) (A)--(B)--(C)--(D)--(A)--(A_1)--(B_1)--(C_1)--(C) (B)--(B_1);
			\foreach \x/\g in {A/160,B/-10,C/60,D/90,A_1/-90,B_1/-90,C_1/-90,D_1/-80,I/60}
			\fill[black] (\x) circle (1pt)
			($(\g:4mm)+(\x)$) node {$\x$};
		\end{tikzpicture}
	\end{center}
	Gọi $I$ là trung điểm của cạnh $BD$. Góc giữa hai đường thẳng $A_1D$ và $B_1I$ bằng bao nhiêu độ?
	\shortans[0]{$30$}
	\loigiai{
	\begin{center}
		\begin{tikzpicture}
			\def\a{3.5}
			\path (0:0) coordinate (A)
			++(50:\a/2) coordinate (D)
			++(0:\a) coordinate (C)
			($(A)+(C)-(D)$) coordinate (B)
			($(A)+(-90:\a)$) coordinate (A_1)
			($(B)+(-90:\a)$) coordinate (B_1)
			($(C)+(-90:\a)$) coordinate (C_1)
			($(D)+(-90:\a)$) coordinate (D_1)
			($(B)!0.5!(D)$) coordinate (I)
			($(B_1)!0.5!(D_1)$) coordinate (J);
			\draw[dashed,thick] (D_1)--(D)--(A_1)--(D_1)--(C_1) (B_1)--(I) (D_1)--(B_1) (D)--(J)--(A_1);
			\draw[thick] (D)--(B) (A)--(B)--(C)--(D)--(A)--(A_1)--(B_1)--(C_1)--(C) (B)--(B_1);
			\foreach \x/\g in {A/160,B/-10,C/60,D/90,A_1/-90,B_1/-90,C_1/-90,D_1/-80,I/60,J/40}
			\fill[black] (\x) circle (1pt)
			($(\g:4mm)+(\x)$) node {$\x$};
		\end{tikzpicture}
	\end{center}
	Gọi $J$ là trung điểm của $B_1D_1$, ta có $DJ\parallel IB_1$ và $JA_1\perp (DD_1B_1)$ nên $\triangle DJA_1$ vuông tại $J$.\\
	Suy ra góc giữa $A_1D$ và $B_1I$ bằng góc giữa $A_1D$ và $DJ$ và bằng góc $\widehat{A_1DJ}$.\\
	Ta có $\cos\widehat{A_1DJ}=\dfrac{DJ}{DA_1}=\dfrac{\sqrt{3}}{2}$. Suy ra $\widehat{A_1DJ}\approx 30^\circ$.
	}
\end{ex}

%\indapan{3}{ans/ans-DS-DE 22}

\caukq
\Opensolutionfile{ans}[ans/ans-KQ-DE 22]

\begin{ex}%[Nguồn: Bộ đề minh họa Moon 2024-2025]%[1H8H1-3]
	Cho hình chóp $S.ABCD$ có đáy $ABCD$ là hình vuông cạnh bằng $2$, $SA$ vuông góc với mặt phẳng đáy và $SA=1$. Gọi $M$, $N$ lần lượt là trung điểm của $SA$, $CD$. Côsin của góc giữa hai đường thẳng $MN$ và $SC$ bằng bao nhiêu? (làm tròn kết quả đến hàng phần trăm).
	\shortans[]{$0{,}95$}
	\loigiai
	{\begin{center}
		\begin{tikzpicture}[line join=round,line cap=round,>=stealth,scale=0.8,font=\footnotesize]
			\foreach \x/\y/\n in {0/0/A,-2/-2/D,6/0/B} \coordinate (\n) at (\x,\y);
			\coordinate (C) at ($(B)+(D)-(A)$);
			\coordinate (O) at ($(B)!0.5!(D)$);
			\coordinate (S) at ($(A)+(0,4)$);
			\coordinate (M) at ($(S)!0.5!(A)$);
			\coordinate (N) at ($(C)!0.5!(D)$);
			\draw (S)--(B)--(C)--(D)--(S)--(C);
			\draw[dashed] (S)--(A)--(C) (A)--(B)--(D)--(A)--(N) (O)--(M)--(N)--(O);
			\foreach \t/\g in {S/90,A/170,D/-120,C/-30,B/30,O/70,M/40,N/-70}{
			\draw[fill=black] (\t) circle (1pt) node[shift={(\g:8pt)}]{$ \t $};
			}
			\draw pic[thin,draw,angle radius=2mm] {right angle= C--O--B};
			\draw pic[thin,draw,angle radius=2mm] {right angle= B--A--S};
		\end{tikzpicture}
	\end{center}
	Gọi $O$ là tâm của hình vuông.\\
	Vì $MO$ là đường trung bình của $\triangle SAC$ nên $MO\parallel SC$.\\
	Suy ra $(MN, SC)=(MN,MO)$.\\
	Tam giác $SAC$ vuông tại $A$ nên $SC=\sqrt{SA^2+AC^2}=\sqrt{1+\left(2\sqrt{2}\right)^2}=3$.\\
	Tam giác $ADN$ vuông tại $D$ nên $AN=\sqrt{AD^2+DN^2}=\sqrt{2^2+1^2}=\sqrt{5}$.\\
	Vì $\heva{&SA\perp (ABCD)\\&AN\subset (ABCD)}$ nên $SA\perp AN$.\\
	Tam giác $MAN$ vuông tại $A$ nên $MN=\sqrt{AM^2+AN^2}=\sqrt{\left(\dfrac{1}{2} \right)^2+\left(\sqrt{5} \right)^2}=\dfrac{\sqrt{21}}{2}$.\\
	Xét $\triangle MNO$ có $\heva{&MO=\dfrac{1}{2}SC=\dfrac{3}{2}\\&NO=\dfrac{1}{2}AD=1\\&MN=\dfrac{\sqrt{21}}{2}.}$\\
	Áp dụng định lí côsin ta có $$\cos\widehat{NMO}=\dfrac{MN^2+MO^2-NO^2}{2\cdot MN \cdot MO}=\dfrac{\left(\dfrac{\sqrt{21}}{2}\right)^2+\left(\dfrac{3}{2}\right)^2-1^2}{2\cdot \dfrac{\sqrt{21}}{2} \cdot \dfrac{3}{2}}=\dfrac{25}{3\sqrt{21}}\approx 0{,}95.$$
	Vậy $\cos(MN,SC)=\cos\widehat{NMO}\approx 0{,}95$.
	}
\end{ex}

%

\begin{ex}%[Nguồn: Bộ đề minh họa Moon 2024-2025]%[1H8H2-2]
	\immini{	Cho hình chóp tứ giác đều $S.ABCD$ (xem hình minh họa bên). Đường thẳng $AC$ vuông góc với mặt phẳng nào sau đây?
	\choice
	{\True $(SBD)$}
	{$(SAB)$}
	{$(SAD)$}
	{$(SBC)$}}{
	\begin{tikzpicture}
		\def\a{4}
		\def\h{4.5}
		\path	(0:0) coordinate (A)
		++(0:\a) coordinate (D)
		++(-130:\a/2) coordinate (C)
		($(A)+(C)-(D)$) coordinate (B)
		(intersection of A--C and B--D) coordinate (O)
		($(O)+(90:\h)$) coordinate (S);
		\draw[dashed]	(B)--(A)--(D)	(A)--(S)
		(A)--(C)		;
		\draw			(B)--(C)--(D)
		(B)--(S)	(C)--(S)	(D)--(S);
		\foreach \x/\g in {A/135,B/-135,C/-45,D/45,S/90}
		\fill[black]	(\x) circle (1.5pt)
		($(\g:3mm)+(\x)$) node {$\x$};
	\end{tikzpicture}}
	\loigiai{\begin{center}
		\begin{tikzpicture}
			\def\a{4}
			\def\h{4.5}
			\path	(0:0) coordinate (A)
			++(0:\a) coordinate (D)
			++(-130:\a/2) coordinate (C)
			($(A)+(C)-(D)$) coordinate (B)
			(intersection of A--C and B--D) coordinate (O)
			($(O)+(90:\h)$) coordinate (S);
			\draw[dashed]	(B)--(A)--(D)	(A)--(S)
			(A)--(C)	(B)--(D)	(S)--(O)	;
			\draw			(B)--(C)--(D)
			(B)--(S)	(C)--(S)	(D)--(S);
			\foreach \x/\g in {A/135,B/-135,C/-45,D/45,S/90,O/-90}
			\fill[black]	(\x) circle (1.5pt)
			($(\g:3mm)+(\x)$) node {$\x$};
			\draw pic[draw,angle radius=3mm]{right angle=D--O--S};%Theo chiều dương
		\end{tikzpicture}
	\end{center}Ta có $\heva{&AC\perp BD\\&AC\perp SO}\Rightarrow AC\perp (SBD)$.}
\end{ex}

\begin{ex}%[Nguồn: Bộ đề minh họa Moon 2024-2025]%[1H8H2-2]
	Cho hình chóp $S.ABCD$ có đáy là hình bình hành tâm $O$ có $AB = 2AC$. Hai mặt phẳng $\left(SAC \right)$, $\left(SBD \right)$ cùng vuông góc với mặt phẳng đáy. Khẳng định nào sau đây đúng?
	\choice
	{$SA \perp \left(ABCD \right)$}
	{\True $SO \perp \left(ABCD \right)$}
	{$AC \perp \left(SBD \right)$}
	{$BD \perp \left(SAC \right)$}
	\loigiai
	{
	\begin{center}
		\begin{tikzpicture}[>=stealth,line join=round,line cap=round,font=\footnotesize,scale=1]
			\path
			(0,0) coordinate (A)
			(4,0) coordinate (B)
			(5.5,-1.5) coordinate (C)
			(1.5,-1.5) coordinate (D)
			($(A)!0.5!(C)$) coordinate (O)
			($(O) + (0,4)$) coordinate (S)
			;
			\draw
			(S) -- (A) -- (D) -- (C) -- (S) -- (D)
			
			;
			\draw[dashed]
			(A) -- (B) -- (C) -- (A)
			(S) -- (B) -- (D)
			(S) -- (O)
			;
			\fill
			(S) circle(1pt) node[above]{$S$}
			(A) circle(1pt) node[left]{$A$}
			(B) circle(1pt) node[right]{$B$}
			(C) circle(1pt) node[below]{$C$}
			(D) circle(1pt) node[below]{$D$}
			(O) circle(1pt) node[below]{$O$}
			;
			\draw pic[draw,angle radius=2mm]{right angle = C--O--S};
			\draw pic[draw,angle radius=2mm]{right angle = S--O--D};
		\end{tikzpicture}
	\end{center}
	Ta có $\left(SAC \right) \cap \left(SBD \right) = SO$.\\
	Ta có $\heva{&\left(SAC \right) \perp \left(ABCD \right)\\&\left(SBD \right) \perp \left(ABCD \right)} \Rightarrow \heva{&SO \perp AC \\ &SO \perp BD.}$\\
	Từ đó ta có, $SO \perp \left(ABCD \right)$.
	}
\end{ex}

%

\begin{ex}%[Nguồn: Bộ đề minh họa Moon 2024-2025]%[1H8H2-3]
	Cho hình chóp $S. ABCD$ có đáy $ABCD$ là hình thoi tâm $O$. Biết rằng $SA=SC$ và $SB=SD$. Khẳng định nào sau đây là \textbf{sai}?
	\choice
	{$SO\perp(ABCD)$}
	{$AC\perp BD$}
	{$(SBD) \perp(SAC)$}
	{\True $BD\perp SD$}
	\loigiai{
	\immini{
	Ta có $\heva{& SO\perp AC\\& SO\perp BD}\Rightarrow SO\perp (ABCD)$ đúng.\\
	Vì $ABCD$ là hình thoi nên $AC\perp BD$ đúng.\\
	Ta có $\heva{& BD\perp AC\\& BD\perp SO} \Rightarrow BD\perp (SAC)$
	Suy ra $\heva{& BD\subset (SBD)\\& BD\perp (SAC)}\Rightarrow (SBD)\perp (SAC)
	$ đúng.\\
	Ta có $BD\subset (SBD)$ mà tam giác $SBD$ cân tại $S$ nên $\widehat{SDB}=\widehat{SBD}<90^\circ$ do đó $BD\perp SD$ sai.
	}{
	\begin{tikzpicture}[declare function={a=2;b=4;h=4;},line join=round]
		\path (0,0) coordinate (A)
		(-145:a) coordinate (B)
		(b,0) coordinate (D)
		($ (B)!0.5!(D) $) coordinate (O)
		($(O)+(0,h) $) coordinate (S);
		\path ($(D)-(A)+(B)$) coordinate (C);
		\draw[dashed] (S)--(A)--(B) (A)--(D)--(B) (A)--(C) (S)--(O);
		\draw (B)--(C)--(D) (B)--(S)  (D)--(S)--(C);
		\foreach \x/\y/\z in {S/O/B,S/O/C,A/O/B}{
		\path pic[draw,angle radius=5pt]{right angle= \x--\y--\z};
		}
		\foreach \t/\g in {A/150,B/-90,C/-90,D/0,S/90,O/-90}{
		\draw[fill=black] (\t) circle (1pt) node[shift={(\g:7pt)},font=\scriptsize]{$ \t $};
		}
	\end{tikzpicture}
	}
	}
\end{ex}

\begin{ex}%[Nguồn: Bộ đề minh họa Moon 2024-2025]%[1H8H2-3]
	Cho hình chóp $S.ABCD$ có đáy $ABCD$ là hình thoi tâm $O$. Cạnh bên $SA \perp (ABCD)$. Khẳng định nào dưới đây \textbf{sai}?
	\choice
	{$SA \perp BD$}
	{$SC \perp BD$}
	{$SO \perp BD$}
	{\True $AD \perp SC$}
	\loigiai{
	\immini{
	Ta có $SA\perp (ABCD)$ nên $SA\perp BD\subset (ABCD)$.\\
	Lại có tứ giác $ABCD$ là hình thoi, do đó $BD\perp AC$.\\
	Từ đó suy ra $BD\perp (SAC)$.\\
	Mặt khác $SO\subset (SAC)$ và $SC\subset (SAC)$, do đó $BD\perp SO$ và $BD\perp SC$.\\
	Ta có $AD$ và $SC$ là hai đường thẳng chéo nhau và $\left(AD,SC\right)=\left(BC,SC\right)=\widehat{SCB}$.\\
	Do đó việc kết luận $AD\perp SC$ là chưa đủ cơ sở.
	}{
	\begin{tikzpicture}[>=stealth,line join=round,line cap=round,font=\footnotesize,scale=.6]
		\tikzset{
		pics/hinhChopTuGiac/.style  n args={5}{
		code={
		\tikzset{
		declare function={a=4;b=2;h=3;goc=-120;}
		}
		\path
		(0,0)coordinate (#1)+(0:a)coordinate (#2)+(goc:b)coordinate (#4)+(90:h)coordinate (#5)
		($(#2)+(#4)-(#1)$)coordinate (#3)
		;
		}
		}}
		\path
		(0,0)pic {hinhChopTuGiac={A}{B}{C}{D}{S}}
		(intersection of A--C and B--D)coordinate (O)
		;
		\foreach \pointo/\pointt in {S/A,A/B,A/C,B/D,A/D,S/O}{
		\draw[fill=black,dashed](\pointo)--(\pointt);
		}
		\foreach \pointo/\pointt in {S/B,S/C,S/D,C/D,B/C}{
		\draw[fill=black](\pointo)--(\pointt);
		}
		\foreach \point/\goc in {S/90,A/150,C/-45,B/10,D/190,O/-90}{
		\draw[fill=black](\point)circle(.8pt)+(\goc:2mm)node[scale=.8]{$\point$};
		}
	\end{tikzpicture}
	}
	}
\end{ex}

%

\begin{ex}%[Nguồn: Bộ đề minh họa Moon 2024-2025]%[1H8H4-2]
	Cho hình chóp $S.ABCD$ có đáy $ABCD$ là hình chữ nhật và $SA\perp(ABCD)$. Mặt phẳng nào sau đây vuông góc với mặt phẳng $(ABCD)$?
	\choice
	{\True $(SAB)$}
	{$(SBC)$}
	{$(SCD)$}
	{$(SBD)$}
	\loigiai{
	Vì $\heva{&SA\perp(ABCD) \\	&SA\subset (SAB)}$ nên $(SAB) \perp(ABCD)$.
	\begin{center}
		\begin{tikzpicture}[line join=round, line cap=round,>=stealth,thick,scale=0.8]
			\path
			(0,0) coordinate (A)
			(-2,-2) coordinate (B)
			(4,0) coordinate (D)
			(0,3.5) coordinate (S)
			($(B)+(D)-(A)$) coordinate (C)
			;
			\draw (S)--(B)--(C)--(D) (D)--(S)--(C)
			;
			\draw[dashed] (S)--(A) (B)--(A)--(D) ;
			\foreach \x/\g in {A/160,B/180,C/-30,D/0,S/90}
			\draw[fill=white] (\x) circle (.03)+(\g:0.3)node{$\x$};
		\end{tikzpicture}
	\end{center}
	}
\end{ex}

%Câu trắc nghiệm 9

\begin{ex}%[Nguồn: Bộ đề minh họa Moon 2024-2025]%[1H8H4-2]
	\immini[thm]
	{Cho hình chóp $S.ABC$ có đáy $ABC$ là tam giác vuông tại $B$ và cạnh bên $SA$ vuông góc với mặt phẳng đáy. Phát biểu nào sau đây là \textbf{sai}?
	\choice
	{$(SAB) \perp (ABC)$}
	{$(SAC) \perp (ABC)$}
	{\True $(SAC) \perp (SBC)$}
	{$(SAB) \perp (SBC)$}
	}
	{\begin{tikzpicture}[font=\footnotesize, line join=round, line cap=round, >=stealth, scale=1]
		\path (0,0) coordinate (A) (0,2) coordinate (S) (-40:1.5) coordinate (B) (3.5,0) coordinate (C);
		\draw (S)--(A)--(B)--(C)--cycle (S)--(B);
		\draw[dashed] (A)--(C);
		\foreach \x/\g in {S/90, A/180, B/-90, C/0}{
		\fill (\x) circle (1pt)+(\g:0.3)node{$\x$};
		}
	\end{tikzpicture}}
	\loigiai{
	Do $SA\perp (ABC)$ nên $(SAB) \perp (ABC)$ và $(SAC) \perp (ABC)$.\\
	Ta có $BC\perp AB$ và $BC\perp SA$ nên $BC\perp (SAB)$. Do đó $(SAB) \perp (SBC)$.\\
	Vậy khẳng định sai là $(SAC) \perp (SBC)$.
	}
\end{ex}

\begin{ex}%[Nguồn: Bộ đề minh họa Moon 2024-2025]%[1H8H4-3]
	Cho hình chóp tam giác đều $S.ABC$ có $AB=2, SA=3$. Gọi $\alpha$ là số đo của góc nhị diện $[S, BC, A]$. Giá trị $\tan \alpha$ bằng bao nhiêu? (làm tròn kết quả đến hàng phần mười).
	\shortans{$4{,}8$}
	\loigiai{
	\begin{center}
		\begin{tikzpicture}[declare function={goc=-65; a=5; b=0.45*a; h=5;}]
			\path
			(0,0) coordinate (A)
			(a,0) coordinate (C)
			(goc:b) coordinate (B)
			($(C)!.5!(B)$) coordinate (M)
			($(A)!.66!(M)$) coordinate (O)
			+(0,h) coordinate (S);
			\draw
			pic[angle radius=3mm,draw] {right angle = S--O--M}
			pic[angle radius=3mm,draw] {right angle = A--M--B};
			\draw[dashed] (M)--(A)--(C) (S)--(O) ;
			\draw (S)--(B) (S)--(A)--(B)--(C)--cycle (S)--(M);
			\foreach \x/\goc in {S/90,A/240,C/-60,B/-90,M/-40, O/-90}
			\draw[fill=black] (\x) node[shift={(\goc:7pt)},font=\scriptsize]{$\x$} circle (1pt);
		\end{tikzpicture}
	\end{center}
	Gọi $O$ là tâm của tam giác đều $ABC$ suy ra $SO\perp (ABC)$.\\
	Gọi $M$ là trung điểm $BC$ ta có $AM=\dfrac{AB\sqrt{3}}{2}=\sqrt{3}$, $AO=\dfrac{2}{3}\cdot AM=\dfrac{2\sqrt{3}}{3}$, $OM=\dfrac{1}{3}\cdot AM=\dfrac{\sqrt{3}}{3}$.\\
	Ta có $SO=\sqrt{SA^2-AO^2}=\sqrt{3^2-\left(\dfrac{2\sqrt{3}}{3}\right)^2}= \dfrac{\sqrt{69}}{3}$.\\
	Ta có $\heva{&(SBC)\cap (ABC)=BC\\&SM\perp BC\\&AM\perp BC}$ suy ra số đo của góc nhị diện $[S, BC, A]$ bằng $\widehat{SMA}=\alpha$.\\
	Ta có $\tan \alpha=\dfrac{SO}{OM}=\dfrac{\dfrac{\sqrt{69}}{3}}{\dfrac{\sqrt{3}}{3}}=\sqrt{23}\approx 4{,}8$.
	}
\end{ex}

\begin{ex}%[Nguồn: Bộ đề minh họa Moon 2024-2025]%[1H8H5-3]
	\immini[thm]{Giá đỡ ba chân ở hình vẽ đang được mở sao cho ba góc chân cách đều nhau một khoảng cách bằng $110$\,cm. Tính chiều cao của giá đỡ, biết các chân của giá đỡ dài $129$\,cm (\textit{kết quả làm tròn đến hàng đơn vị}).}{
	\begin{tikzpicture}[line join = round, line cap=round,>=stealth,font=\footnotesize,scale=0.7]
		\draw[dashed,orange] (0,0) ellipse (2cm and 1cm);
		\path
		(0,0) coordinate (O)
		($(0,0)+(0,4)$) coordinate (S)
		(-110:2cm and 1cm) coordinate (A2)
		(160:2cm and 1cm) coordinate (A3)
		(30:2cm and 1cm) coordinate (A1)
		;
		\fill
		(A1) circle(3pt)
		(A3) circle(3pt)
		(A2) circle(3pt)
		(S) circle(3pt)
		(O) circle(3pt)
		;
		\draw[line width=1pt] (S)--(A1) (S)--(A2) (S)--(A3);
		\draw ($(S)+(-0.2,0.5)$) node[]{
		\begin{tikzpicture}[line join = round, line cap=round,>=stealth,font=\footnotesize,scale=0.6]
		\draw[fill=black] (-0.75,-0.5)rectangle (0.5,0.5);
		\draw[fill=cyan] (-0.65,-0.35)rectangle (0.35,-0.25);
		\draw[fill=white] (-0.65,0.35)rectangle (0.35,0.25);
		\draw[fill=black] (-0.65,0.5)rectangle (-0.5,0.75);
		\draw[fill=black] (-0.65,0.65)rectangle (0.5,0.85);
		\draw[fill=black]
		(-0.4,0)--(-1,0.5)--(-1,-0.5)--cycle
		;
		\draw ;
	\end{tikzpicture}
	};
	\end{tikzpicture}
	}
	\shortans[]{$112$}
	\loigiai{
	\begin{center}
		\begin{tikzpicture}[>=stealth,line join=round,line cap=round,font=\footnotesize,scale=1]
			\def\r{4cm}
			\path
			(0,0) coordinate (A)
			(0:\r) coordinate (C)
			(-55:\r*0.6) coordinate (B)
			($(B)!0.5!(C)$) coordinate (I)
			($(A)!2/3!(I)$) coordinate (O)
			++(90:\r) coordinate (S)
			;
			\draw (S)--(A)--(B)--(C)--(S)--(B);
			\draw[dashed] (S)--(O) (C)--(A)--(I);
			\foreach \i/\g in {S/90,A/180,B/-90,C/0,O/-90,I/-90}{\draw[fill=black](\i) circle (1pt) ($(\i)+(\g:3mm)$) node[scale=1]{$\i$};}
		\end{tikzpicture}
	\end{center}
	Giá đỡ ba chân có dạng là hình chóp tam giác đều $S.ABC$ như hình vẽ.\\
	Gọi $I$ là trung điểm của $BC$, $O$ là tâm tam giác $ABC$. Khi đó $SO\perp (ABC)$ hay $SO$ là chiều cao của giá đỡ.\\
	Theo đề bài ta có $\heva{&AB=AC=BC=110\\&SA=129.}$\\
	Khi đó ta có $AI=\dfrac{110\sqrt{3}}{2}=55\sqrt{3}$.\\
	Suy ra $AO=\dfrac{2}{3}AI=\dfrac{2}{3}\cdot 55\sqrt{3}=\dfrac{110}{\sqrt{3}}$.\\
	Vậy $SO=\sqrt{SA^2-AO^2}=\sqrt{129^2-\left(\dfrac{110}{\sqrt{3}} \right)^2}\approx 112$ (cm).
	}
\end{ex}

%\subsection{Câu trắc nghiệm trả lời ngắn}
\caukq

\begin{ex}%[Nguồn: Bộ đề minh họa Moon 2024-2025]%[1H8H6-1]
	Cho hình chóp $S.ABCD$ có đáy là hình vuông cạnh $a$, $SA \perp (ABCD)$, số đo của góc nhị diện $[S, BC, A]$ bằng $60^\circ$. Khoảng cách giữa hai đường thẳng $SC$ và $BD$ bằng $\dfrac{a\sqrt{30}}{n}$. Tính giá trị của $n$.
	\shortans[oly]{$10$}
	\loigiai{
	\immini{
	Ta có $SA\perp (ABCD)$ nên $BC\perp SA$.\\
	Mà $BC\perp AB$ (do $ABCD$ là hình vuông).\\
	Suy ra, $BC\perp SB$. Do đó, số đo góc nhị diện $[S,BC,A]$ bằng số đo góc $\widehat{SBA}$, hay $\widehat{SBA}=60^\circ$.\\
	Tam giác $SAB$ vuông tại $A$ có\\ $SA=AB\cdot\tan\widehat{SBA}=a\sqrt{3}$.}{\begin{tikzpicture}[line join=round, line cap=round,thick,scale=0.5]
		\coordinate (A) at (0,0);
		\coordinate (B) at (-2,-3);
		\coordinate (D) at (7,0);
		\coordinate (C) at ($(B)+(D)-(A)$);
		\coordinate (S) at ($(A)+(0,5)$);
		\coordinate (O) at (intersection of A--C and B--D);
		\coordinate (H) at ($(S)!(3:4)!(C)$);
		\draw(S)--(B) (S)--(C) (S)--(D) (B)--(C)--(D);
		\draw[dashed,thin](S)--(A)--(C) (A)--(B) (A)--(D)--(B) (O)--(H);
		\pic[draw,thin,angle radius=2mm] {right angle = S--A--D} pic[draw,thin,angle radius=2mm] {right angle = S--A--B} pic[draw,thin,angle radius=2mm] {right angle = O--H--C};
		\foreach \i/\g in {S/90,A/-90,B/-90,C/-90,D/0,O/-90,H/0}{\draw[fill=white](\i) circle (1.5pt) ($(\i)+(\g:5mm)$) node[scale=1]{$\i$};}
	\end{tikzpicture}}
	Gọi $O=AC\cap BD$. Kẻ $OH\perp SC$, $H\in SC$.\\
	Ta có $BD\perp SC, BD\perp SA$ nên $BD\perp (SAC)$. Do đó, $BD\perp OH$.\\
	Suy ra $OH$ là đoạn vuông góc chung của $BD$ và $SC$ hay $d[BD,SC]=OH$.\\
	Ta có $\triangle ACS\sim \triangle HCO$ nên $\dfrac{OH}{SA}=\dfrac{OC}{SC}$.	Suy ra $$OH=\dfrac{OC\cdot SA}{SC}=\dfrac{SA\cdot OC}{\sqrt{SA^2+AC^2}}=\dfrac{a\sqrt{3}\cdot \tfrac{a\sqrt{2}}{2}}{\sqrt{3a^2+2a^2}}=\dfrac{a\sqrt{30}}{10}.$$\\
	Vậy giá trị của $n$ là $10$.
	}
\end{ex}

\begin{ex}%[Nguồn: Bộ đề minh họa Moon 2024-2025]%[1H8H6-1]
	\immini[thm]{Cho hình chóp $S.ABCD$ đáy là hình vuông cạnh $a$, $SA=a\sqrt{3}$. Đường thẳng $SA\perp (ABCD)$ (hình vẽ). Góc giữa đường thẳng $SB$ và mặt phẳng $(ABCD)$ là}{\begin{tikzpicture}[scale=.8, font=\footnotesize, line join=round, line cap=round, >=stealth]
		\def\bc{3} \def\ba{1.5} \def\h{3} \def\gocB{30}
		\coordinate[label=below left:$B$] (B) at (0,0);
		\coordinate[label=above left:$A$] (A) at (\gocB:\ba);
		\coordinate[label=below:$C$] (C) at (\bc,0);
		\coordinate[label=right:$D$] (D) at ($(C)-(B)+(A)$);
		\coordinate[label=above:$S$] (S) at ($(A)+(90:\h)$);
		\draw (B)--(C)--(D)--(S)--cycle (S)--(C);
		\draw[dashed] (A)--(D) (S)--(A)--(B);
		\foreach \diem in {A,B,C,D,S}
		\fill (\diem)circle(1pt);
		\draw (A)--++(90:0.2)--++(0:0.2)--++(-90:0.2);
	\end{tikzpicture}}
	\choice
	{$90^{\circ}$}
	{$30^{\circ}$}
	{$45^{\circ}$}
	{\True$60^{\circ}$}
	\loigiai{
	Do $SA\perp (ABCD)$ nên $AB$ là hình chiếu của $SB$ lên mặt phẳng $ABCD$. Do đó góc giữa đường thẳng $SB$ và mặt phẳng $(ABCD)$ là $\widehat{SBA}$.\\
	Ta có $\tan\widehat{SBA}=\dfrac{SA}{AB}=\dfrac{a\sqrt{3}}{a}=\sqrt{3}\Rightarrow\widehat{SBA}=60^\circ$.
	}
\end{ex}

%

\begin{ex}%[Nguồn: Bộ đề minh họa Moon 2024-2025]%[1H8H6-2]
	Cho hình chóp $S.ABCD$ có đáy là hình vuông, $BD=2 a$, cạnh bên $SA$ vuông góc với mặt phẳng đáy và $SA=a$. Số đo của góc nhị diện $[S, BD, C]$ bằng bao nhiêu độ?
	\loigiai{
	\begin{center}
		\begin{tikzpicture}[line join = round, line cap = round, thick, font = \small, scale = 1]
			\path
			(0:0) coordinate (A)
			+(0:5) coordinate (B)
			+(-150:2.5) coordinate (D)
			+(90:3) coordinate (S)
			($(B)+(D)-(A)$) coordinate (C);
			\path (intersection of A--C and D--B) coordinate (O);
			\draw[dashed]
			(A)--(B) (C)--(A)--(D)--(B) (A)--(S)--(O);
			\draw
			(D)--(C)--(B)
			(S)--(B) (S)--(C) (S)--(D);
			\foreach \x/\g in {A/135,B/0,C/-45,D/-135,S/90,O/-90}
			\fill [blue] (\x) circle (1.5pt)
			+(\g:3.5mm) node {$\x$};
		\end{tikzpicture}
	\end{center}
	Ta có $AB=\dfrac{BD}{\sqrt{2}}=a\sqrt{2}$.\\ $SD=SB=\sqrt{SA^2+AB^2}=a\sqrt{3}$.\\
	Xét $\triangle SOA$ vuông tại A, ta có $\tan \widehat{SOA}=\dfrac{SA}{AO}=1\Rightarrow \widehat{SOA}=45^\circ$.\\
	Ta lại có $\heva{&SO \perp BD\ (\triangle SBD\text{ cân tại }S)\\&CO \perp BD \ (ABCD \text{ là hình vuông})}\Rightarrow [S, BD, C]=\widehat{SOC}=180^\circ -\widehat{SOA}=135^\circ$.
	}
\end{ex}

\begin{ex}%[Nguồn: Bộ đề minh họa Moon 2024-2025]%[1H8H7-5]
	Cho khối lăng trụ đứng có cạnh bên bằng $5$, đáy là hình vuông có cạnh bằng $4$. Hỏi thể tích khối lăng trụ bằng bao nhiêu?
	\choice
	{$100$}
	{$20$}
	{$64$}
	{\True $80$}
	\loigiai{
	Diện tích đáy của lăng trụ bằng $4^2=16$ nên thể tích của lăng trụ bằng $5\cdot 16=80$.
	}
\end{ex}

\begin{ex}%[Nguồn: Bộ đề minh họa Moon 2024-2025]%[1H8N1-2]
	Cho hình hộp chữ nhật $ABCD.A'B'C'D'$. Hai đường thẳng nào sau đây vuông góc với nhau?
	\choice
	{$BD$ và $C'D'$}
	{\True $AA'$ và $BD$}
	{$A'B$ và $CD$}
	{$BB'$ và $DD'$}
	\loigiai
	{
	\begin{center}
		\begin{tikzpicture}[scale=1, font=\footnotesize,>=stealth, line width=1pt]%<DTools>
			%Gán số liệu.
			\def\canhAD{3};\def\canhBA{2};\def\gocBAD{-130};\def\h{4};\def\xdinhA'{0};
			%Gán tọa độ.
			\coordinate (A) at (0,0);
			\coordinate (B) at ($(A)+(\gocBAD:\canhBA)$);
			\coordinate (C) at ($(B)+(0:\canhAD)$);
			\coordinate (D) at ($(A)+(0:\canhAD)$);
			\coordinate (A') at ($(A)+(\xdinhA',\h)$);
			\coordinate (B') at ($(B)+(\xdinhA',\h)$);
			\coordinate (C') at ($(C)+(\xdinhA',\h)$);
			\coordinate (D') at ($(D)+(\xdinhA',\h)$);
			%Vẽ khối lẳng trụ ABCD.A'B'C'D'.
			\draw (A')--(B')--(B)--(C)--(C')--(D')--cycle (B')--(C') (D')--(D)--(C);
			\draw[dashed] (A)--(D) (A')--(A)--(B);
			%Gán nhãn.
			\foreach \x/\y in {A/180, B/180, C/0, D/0, A'/180, B'/180, C'/0, D'/0}{\fill (\x) circle(1pt) ($(\x)+(\y:0.3cm)$) node{$\x$};}
		\end{tikzpicture}
	\end{center}
	Ta có $AA'\perp (ABCD) \Rightarrow AA'\perp BD$.
	}
\end{ex}

\begin{ex}%[Nguồn: Bộ đề minh họa Moon 2024-2025]%[1H8N1-3]
	Cho hình lập phương $ABCD.EFGH$. Hãy xác định góc giữa hai vectơ $\overrightarrow{AB}$ và $\overrightarrow{EG}$?
	\choice
	{\True $45^\circ$}
	{$90^\circ$}
	{$60^\circ$}
	{$120^\circ$}
	\loigiai{
	\immini{Ta có $\left(\overrightarrow{AB},\overrightarrow{EG}\right)=\left(\overrightarrow{EF},\overrightarrow{EG}\right)=45^\circ$.}
	{\begin{tikzpicture}[scale=0.6, font=\footnotesize,line join=round, line cap=round, >=stealth]
		\path
		(0,0) coordinate (A)
		++(-130:3) coordinate (B)
		++(0:4) coordinate (C)
		($(A)+(C)-(B)$) coordinate (D)
		;
		\coordinate (E) at ($(A)+(0,4)$);
		\coordinate (F) at ($(B)+(0,4)$);
		\coordinate (G) at ($(C)+(0,4)$);
		\coordinate (H) at ($(D)+(0,4)$);
		
		\draw (E)--(F)--(G)--(H)--cycle;
		\draw (B)--(F) (C)--(G) (D)--(H)  (B)--(C)--(D) (E)--(G);
		\draw[dashed](B)--(A)--(E) (A)--(D);
		\foreach \i/\g in {E/90,F/90,G/90,H/90,A/-90,B/-90,C/-90,D/-90}
		\fill[black] (\i) circle(1pt)+(\g:5mm)node[scale=1]{$\i$};
	\end{tikzpicture}}
	}
\end{ex}

\begin{ex}%[Nguồn: Bộ đề minh họa Moon 2024-2025]%[1H8N7-2]
	Cho hình chóp $S . A B C D$ có đáy $A B C D$ là hình chữ nhật, $S A$ vuông góc với mặt phẳng đáy. Góc giữa đường thẳng $S B$ và mặt phẳng $(A B C D)$ là
	\choice
	{$\widehat{B S A}$}
	{$\widehat{S A B}$}
	{\True $\widehat{S B A}$}
	{$\widehat{S B C}$}
	\loigiai{
	\immini{Ta có $SB\cap(ABCD)=B$ và $SA\perp (ABCD)$ nên $(SB;(ABCD))=\widehat{SBA}$.}{\begin{tikzpicture}[line join=round, line cap=round,thick,scale=0.4]
		\coordinate (A) at (0,0);
		\coordinate (B) at (-2,-3);
		\coordinate (D) at (7,0);
		\coordinate (C) at ($(B)+(D)-(A)$);
		\coordinate (S) at ($(A)+(0,5)$);
		\draw(S)--(B) (S)--(C) (S)--(D) (B)--(C)--(D);
		\draw[dashed,thin](S)--(A) (A)--(B) (A)--(D);
		\pic[draw,thin,angle radius=3mm] {right angle = S--A--D} pic[draw,thin,angle radius=3mm] {right angle = S--A--B};
		\foreach \i/\g in {S/90,A/-90,B/-90,C/-90,D/0}{\draw[fill=black](\i) circle (1.5pt) ($(\i)+(\g:3mm)$) node[scale=1]{$\i$};}
	\end{tikzpicture}}
	}
\end{ex}

%

\begin{ex}%[Nguồn: Bộ đề minh họa Moon 2024-2025]%[1H8N7-2]
	Cho khối hộp có diện tích đáy là $B$ và chiều cao là $3h$. Thể tích $V$ của khối hộp đã cho là
	\choice
	{$V=\dfrac{1}{3} Bh$}
	{$V=Bh$}
	{\True $V=3Bh$}
	{$V=\dfrac{1}{6} Bh$}
	\loigiai{
	$V=B\cdot 3h=3Bh$.}
\end{ex}

\begin{ex}%[Nguồn: Bộ đề minh họa Moon 2024-2025]%[1H8N7-3]
	Cho khối chóp $S. ABCD$ có chiều cao bằng $5$, đáy $ABCD$ là hình bình hành có diện tích bằng $6$. Thể tích khối chóp $S. ABC$ bằng
	\choice
	{$10$}
	{\True $15$}
	{$5$}
	{$30$}
	\loigiai{
	\immini{
	Ta có $V_{S.ABC}=\dfrac{1}{2}\cdot V_{S.ABCD}=\dfrac{1}{2}\cdot 6\cdot 5=15$.
	}
	{
	\begin{tikzpicture}[declare function={gocx=80; goc=-150; a=5; b=a/2; h=3;}]
		\path (0,0) coordinate (A)--+(gocx:h) coordinate (S)
		(a,0) coordinate (B)
		(goc:b) coordinate (D)
		+(a,0) coordinate (C);
		\draw[dashed] (S)--(A) (D)--(A)--(B) (A)--(C);
		\draw (S)--(D)--(C)--(S)--(B)--(C)--cycle;
		\foreach \x/\goc in {S/90,A/150,B/0,C/-60,D/-180}
		\draw[fill=black] (\x) node[shift={(\goc:7pt)},font=\scriptsize]{$\x$} circle (1pt);
	\end{tikzpicture}
	}
	}
\end{ex}

\begin{ex}%[Nguồn: Bộ đề minh họa Moon 2024-2025]%[1H8N7-3]
	Cho hình chóp $S.ABCD$ có đáy $ABCD$ là hình vuông cạnh $a$, $SA\perp \left(ABD\right)$, $SA=3a$. Thể tích $V$ của hình chóp $S.ABCD$ là
	\choice
	{$V=2a^3$}
	{\True $V=a^3$}
	{$V=\dfrac{1}{3}a^3$}
	{$V=3a^3$}
	\loigiai{
	\immini{Ta có
	\[V=\dfrac{1}{3}\cdot SA\cdot S_{ABCD}=\dfrac{1}{3}\cdot 3a\cdot a^2=a^3.\]
	}
	{\begin{tikzpicture}[>=stealth,line join=round,line cap=round,font=\footnotesize,scale=0.4]
		\clip (-4,-4) rectangle (9,7);
		\path
		(0,0) coordinate (A)
		(-3,-3) coordinate (B)
		(5,-3) coordinate (C)
		(8,0) coordinate (D)
		(0,6) coordinate (S)
		;
		\fill[black] (A) circle (1.5pt) node[above left]{$A$}(C) circle (1.5pt) node[below right]{$C$}(B) circle (1.5pt) node[below left]{$B$}(S) circle (1.5pt) node[above]{$S$}(D) circle (1.5pt) node[above right]{$D$};
		\draw  (S)--(B)--(C)--(D)--(S)--(C);
		\draw[dashed]  (S)--(A)--(D)(B)--(A);
		\draw pic[draw, angle radius = 5pt] {right angle =S--A--B};
		\draw pic[draw, angle radius = 5pt] {right angle =S--A--D};
	\end{tikzpicture}
	}
	}
\end{ex}

\begin{ex}%[Nguồn: Bộ đề minh họa Moon 2024-2025]%[1H8N7-3]
	Biết rằng thể tích của một khối lập phương bằng $8$. Tính tổng diện tích các mặt của hình lập phương đó.
	\choice
	{$16$}
	{\True $24$}
	{$36$}
	{$27$}
	\loigiai{
	Gọi độ dài cạnh của khối lập phương đã cho là $a$.\\
	Theo bài ta có $a^3=8 \Leftrightarrow a=2$.\\
	Vậy tổng diện tích các mặt của khối lập phương là
	$$S=6\cdot a^2=6\cdot 2^2=24.$$
	}
\end{ex}

\begin{ex}%[Nguồn: Bộ đề minh họa Moon 2024-2025]%[1H8N7-4]
	Cho khối lăng trụ tam giác đều có tất cả các cạnh bằng $a$. Thể tích của khối lăng trụ là
	\choice
	{\True $\dfrac{a^{3}\sqrt{3}}{4}$}
	{$\dfrac{a^{3}\sqrt{3}}{12}$}
	{$\dfrac{a^{3}\sqrt{3}}{2}$}
	{$\dfrac{a^{3}\sqrt{3}}{6}$}
	\loigiai{\immini{Theo giả thiết mặt đáy của lăng trụ là tam giác đều cạnh $a$ nên đáy có diện tích $B=\dfrac{a^2\sqrt{3}}{4}$.\\
	Lăng trụ đứng chiều cao $h=a$, do vậy thể tích của khối lăng trụ đã cho là
	$$V=B\cdot h=a\cdot\dfrac{a^2\sqrt{3}}{4}=\dfrac{a^3\sqrt{3}}{4}.$$}{\begin{tikzpicture}[scale=0.8, font=\footnotesize,line join=round, line cap=round, >=stealth]
		\path
		(0,0) coordinate (A)
		++(-60:2) coordinate (B)
		(3,0) coordinate (C)
		;
		\foreach \i in{A,B,C}{
		\coordinate (\i') at ($(\i)+(0,3)$);
		\draw (\i)--(\i');};
		\draw (A)--(B)--(C) (A')--(B')--(C')--cycle;
		\draw[dashed] (A)--(C);
		\foreach \i/\g in {A/-120,B/-90,C/-90,A'/90,B'/80,C'/90}
		\fill[black] (\i) circle(1pt)+(\g:4mm)node[scale=1]{$\i$};
	\end{tikzpicture}}}
\end{ex}

\begin{ex}%[Nguồn: Bộ đề minh họa Moon 2024-2025]%[1H8N7-4]
	Cho hình lăng trụ có thể tích bằng $18$, đáy là hình vuông cạnh bằng $3$. Chiều cao của khối lăng trụ đã cho bằng
	\choice
	{$\dfrac{8\sqrt{3}}{3}$}
	{$6$}
	{\True $2$}
	{$3$}
	\loigiai{Ta có $V=S_d \cdot h \Leftrightarrow 18=3^2\cdot h \Rightarrow h=2.$
	}
\end{ex}

\begin{ex}%[Nguồn: Bộ đề minh họa Moon 2024-2025]%[1H8N7-5]
	Khối lập phương có cạnh bằng $3$ có thể tích bằng
	\choice
	{\True $27$}
	{$8$}
	{$9$}
	{$6$}
	\loigiai{
	Khối lập phương có cạnh  bằng $3$ thì có thể tích $V=3^3=27$ (đ.v.t.t).
	}
\end{ex}

%

\begin{bt}%[Nguồn: Bộ đề minh họa Moon 2024-2025]	%[1H8V5-4]
	Cho hình lăng trụ đứng $ABC.A' B' C'$ có $AB=5$, $BC=6$, $CA=7$. Khoảng cách giữa hai đường thẳng $AA'$ và $BC$ bằng bao nhiêu? (làm tròn kết quả đến hàng phần mười).
	\shortans{4,9}
	\loigiai{
	\begin{center}
		\begin{tikzpicture}[line join=round, line cap=round,>=stealth,thick,scale=1]
			\path
			(0,0) coordinate (A)
			(2.8,-2) coordinate (B)
			(4,0) coordinate (C)
			(0,3.2) coordinate (A')
			($(B)+(A')-(A)$) coordinate (B')
			($(B')+(C)-(B)$) coordinate (C')
			;
			\draw (A')--(A)--(B)--(C)--(C')--(B')--cycle (A')--(C') (B)--(B')
			;
			\draw[dashed] (A)--(C);
			\foreach \x/\g in {A/180,B/180,C/0,A'/90,B'/-140,C'/90} \draw[fill=white] (\x) circle (.03)+(\g:0.3)node{$\x$};
		\end{tikzpicture}
	\end{center}
	
	Kẻ $AH\perp BC$ với $H\in BC$. Khi đó, vì $ABC\cdot A' B' C'$ là lăng trụ đứng nên
	$$
	AA' \perp(A BC) \Rightarrow\left\{\begin{array}{l}
		AA' \perp AH \\
		AA' \perp BC.
	\end{array}\right.
	$$
	
	Mà $BC\perp AH$ nên $AH$ là đoạn vuông góc chung của $AA'$ và $BC$.
	
	Do đó, khoảng cách giữa $AA'$ và $BC$ là $AH$.
	
	Xét $\triangle ABC$, đặt $p=\dfrac{AB+BC+CA}{2}=\dfrac{5+6+7}{2}=9$.
	
	Ta có $S_{\triangle ABC}=\sqrt{p(p-AB)(p-BC)(p-AC)}=$ $\sqrt{9\cdot 4\cdot 3\cdot 2}=6\sqrt{6}$.
	
	Suy ra $AH=\dfrac{2S_{\triangle ABC}}{BC}=\dfrac{12\sqrt{6}}{6}=2\sqrt{6} \approx 4{,}9$.
	}
\end{bt}

\begin{ex}%[Nguồn: Bộ đề minh họa Moon 2024-2025]%[1H8V5-4]
	Cho tứ diện $O.ABC$ có $OA$, $OB$, $OC$ đôi một vuông góc với nhau và $OA=OC=\sqrt{3}$, $OB=1$. Gọi $M$ là trung điểm của cạnh $BC$. Khoảng cách giữa hai đường thẳng $AB$ và $OM$ bằng bao nhiêu? (làm tròn kết quả đến hàng phần mười). \shortans[oly]{$0{,}8$}
	\loigiai{
	\begin{center}
		\begin{tikzpicture}[line join=round, line cap=round,thick, scale=.6]
			\coordinate (O) at (-3,0);
			\coordinate (B) at (0,-3);
			\coordinate (C) at (3,0);
			\coordinate (A) at ($(O)+(0,6)$);
			\coordinate (M) at ($(B)!0.5!(C)$);
			\coordinate (N) at ($(O)!-1!(C)$);
			\coordinate (K) at ($(B)!2/3!(N)$);
			\coordinate (H) at ($(A)!2/3!(K)$);
			\draw (A)--(C) (B)--(C) (A)--(B)--(N)--(A)--(K);
			\draw[dashed,thin](M)--(O)--(C) (K)--(O)--(N)  (O)--(H) (O)--(A) (O)--(B);
			\foreach \i/\g in {A/90,O/-120,B/-90,C/0,M/-30,N/180,K/-90,H/150}{\draw[fill=white](\i) circle (1.5pt) ($(\i)+(\g:3.5mm)$) node[scale=.7]{$\i$};}
		\end{tikzpicture}
	\end{center}
	Gọi $N$ là điểm đối xứng với $C$ qua $O$; $K$ là hình chiếu vuông góc của $O$ trên $NB$; $H$ là hình chiếu vuông góc của $O$ trên $AK$.\\
	Ta có $OM\parallel BN\Rightarrow OM\parallel (ABN)$.\\
	Suy ra $\mathrm{d}(OM, AB)=\mathrm{d}\left(  OM; (ABN)\right)=\mathrm{d}\left(  O; (ABN)\right)$.\\
	Lại có $\heva{&OA\perp BN	\\&OK\perp BN}\Rightarrow BN\perp (OAK)\Rightarrow BN\perp OH$.\\
	Khi đó $\heva{&OH\perp AK	\\&OH\perp BN}\Rightarrow OH\perp (ABN)\Rightarrow \mathrm{d}\left(  O; (ABN)\right)=OH$.\\
	Ta có
	\begin{eqnarray*}
		&& \dfrac{1}{OK^2}=\dfrac{1}{OB^2}+\dfrac{1}{ON^2}=\dfrac{1}{OB^2}+\dfrac{1}{OC^2}\\
		&\Rightarrow& \dfrac{1}{OH^2}=\dfrac{1}{OA^2}+\dfrac{1}{OK^2}=\dfrac{1}{OA^2}+\dfrac{1}{OB^2}+\dfrac{1}{OC^2}=\dfrac{5}{3}\\
		&\Rightarrow& OH=\dfrac{\sqrt{15}}{5}\\
		&\Rightarrow&\mathrm{d}(OM, AB)=\dfrac{\sqrt{15}}{5}\approx 0{,}8.
	\end{eqnarray*}
	}
\end{ex}

%

\begin{ex}%[Nguồn: Bộ đề minh họa Moon 2024-2025]%[1H8V5-4]
	Cho hình lăng trụ $ABC.A'B'C'$ có $(A'B'B) \perp (ABC)$, $AA' = 2$, $AB = 3$, $\widehat{AA'B} = 60^\circ$. Khoảng cách giữa hai đường thẳng $AB$ và $A'C'$ bằng bao nhiêu? (làm tròn kết quả đến hàng phần mười).
	
	\shortans{$1,1$}
	
	\loigiai{\begin{center}
		\begin{tikzpicture}
			\def\a{4}
			\def\h{4.5}
			\path	(0:0) coordinate (A)
			++(0:\a) coordinate (C)
			++(-135:\a/2) coordinate (B)
			($(A)!2/3!(B)$) coordinate (H)
			($(H)+(90:\h)$) coordinate (A')
			($(A')+(C)-(A)$) coordinate (C')
			($(C')+(B)-(C)$) coordinate (B');
			\draw[dashed] (A)--(C);
			\draw (C)--(C')	(B)--(B')	(A)--(A')--(H) (A')--(B)
			(A)--(B)--(C) (A')--(B')--(C')--cycle;
			\foreach \x/\g in {A/180,B/-45,C/0,A'/180,B'/-45,C'/0,H/-135}
			\fill[black]	(\x) circle (1pt)
			($(\g:4mm)+(\x)$) node {$\x$};
			\draw pic[draw,angle radius=2mm]{right angle=A'--H--A};%Theo chiều dương
		\end{tikzpicture}
	\end{center}
	Ta có $\heva{&A'C'\subset (A'B'C')\\&AB\subset(ABC)\\&(ABC)\parallel (A'B'C')}\Rightarrow\mathrm{d}(A'C',AB)=\mathrm{d}\left((ABC),(A'B'C')\right)$.\\
	Trong $(ABB'A')$, kẻ $A'H\perp AB$.\\
	Khi đó, ta có $\heva{&A'H\perp AB=(ABC)\cap (ABB'A')\\&(ABB'A')\perp (ABC)}\Rightarrow A'H\perp (ABC)$.\\
	Mà $A\in (A'B'C')$ nên $\mathrm{d}\left((ABC),(A'B'C')\right)=A'H$.\\
	Suy ra $\mathrm{d}(AB,A'C')=A'H$.\\
	Xét tam giác $ABA'$ có
	\begin{eqnarray*}
		&&AB^2=AA'^2+A'B^2-2AA'\cdot A'B\cdot \cos 60^\circ\\
		&\Leftrightarrow&9= 4+A'B^2-2A'B\\
		&\Leftrightarrow&A'B^2-2A'B-5=0\\
		&\Leftrightarrow&\hoac{&A'B=1-\sqrt{6}~(\text{loại})\\&A'B=1+\sqrt{6}~(\text{thỏa}).}
	\end{eqnarray*}
	Mặt khác \begin{eqnarray*}
		&&S_{\triangle ABA'}=\dfrac{1}{2}\cdot AA'\cdot A'B\cdot \cos 60^\circ=\dfrac{1}{2}\cdot A'H\cdot AB\\
		&\Rightarrow& A'H=\dfrac{AA'\cdot A'B\cdot \cos 60^\circ}{AB}=\dfrac{2\cdot(1+\sqrt{6})\cdot \dfrac{1}{2}}{3}=\dfrac{1+\sqrt{6}}{3}\approx 1,1
	\end{eqnarray*}
	}
\end{ex}

\begin{ex}%[Nguồn: Bộ đề minh họa Moon 2024-2025]%[1H8V5-4]
	Cho hình chóp $S.ABC$ có đáy $ABC$ là tam giác vuông tại $C$ với $AC = 6$, $BC = 12$. Cạnh bên $SC = 3$ và $SC \perp \left(ABC \right)$. Gọi $M$ là trung điểm cạnh $AC$. Khoảng cách giữa hai đường thẳng $SM$ và $AB$ bằng bao nhiêu.
	\shortans{$2$}
	\loigiai{
	\begin{center}
		\begin{tikzpicture}[>=stealth,line join=round,line cap=round,font=\footnotesize,scale=1]
			\path
			(0,3) coordinate (S)
			(0,0) coordinate (C)
			(4,0) coordinate (B)
			(-1.5,-1.5) coordinate (A)
			($(A)!0.5!(C)$) coordinate (M)
			($(B)!0.5!(C)$) coordinate (N)
			($(M)!0.5!(N)$) coordinate (K)
			($(S)!0.7!(K)$) coordinate (H)
			;
			\draw
			(S) -- (A) -- (B) -- (S)
			;
			\draw[dashed]
			(A) -- (C) -- (B)
			(C) -- (S) -- (M) -- (N) -- (S) -- (K) -- (C) -- (H)
			;
			\fill
			(S) circle(1pt) node[above]{$S$}
			(A) circle(1pt) node[left]{$A$}
			(B) circle(1pt) node[right]{$B$}
			(C) circle(1pt) node[above left]{$C$}
			(M) circle(1pt) node[left]{$M$}
			(N) circle(1pt) node[above]{$N$}
			(K) circle(1pt) node[below]{$K$}
			(H) circle(1pt) node[right]{$H$}
			;
		\end{tikzpicture}
	\end{center}
	Gọi $N$ là trung điểm của $BC$ ta có $MN \parallel AB$ suy ra $AB \parallel \left(SMN \right)$.\\
	Từ đó ta có $\mathrm{d} \left(MN,AB \right) = \mathrm{d} \left(A, \left(SMN \right) \right)$.\\
	Mặt khác ta có $\dfrac{\mathrm{d} \left(A, \left(SMN \right) \right)}{\mathrm{d} \left(C, \left(SMN \right) \right)} = \dfrac{AM}{MC} = 1$ hay $\mathrm{d} \left(A, \left(SMN \right) \right) = \mathrm{d} \left(C, \left(SMN \right) \right)$.\\
	Dựng $CH \perp MN = \left\{K \right\}$ ta được $MN \perp \left(SCK \right)$.\\
	Dựng $CH \perp SK$ ta được $CH \perp \left(SMN \right)$ suy ra $\mathrm{d} \left(C, \left(SMN \right) \right) = CH$.\\
	Xét $\triangle CMN$ ta có $\dfrac{1}{CK^2} = \dfrac{1}{CM^2} + \dfrac{1}{CN^2} \Rightarrow CK = \dfrac{6\sqrt{5}}{5}$.\\
	Xét $\triangle SCK$ ta có $\dfrac{1}{CH^2} = \dfrac{1}{SC^2} + \dfrac{1}{CK^2} \Rightarrow CH = 2$.\\
	Vậy $\mathrm{d} \left(A, \left(SMN \right) \right) = 2$.
	}
\end{ex}

\begin{ex}%[Nguồn: Bộ đề minh họa Moon 2024-2025]%[1H8V5-4]
	Cho hình lăng trụ đứng $ABC.A'B'C'$ có đáy $ABC$ là tam giác đều cạnh là $\sqrt{2}$, $A'B=\sqrt{6}$. Khoảng cách giữa hai đường thẳng $A'B$ và $B'C$ bằng bao nhiêu? (kết quả làm tròn đến hàng phần trăm).
	\loigiai{
	\begin{center}
		
		\begin{tikzpicture}[scale=0.8, font=\footnotesize,line join=round, line cap=round, >=stealth]
			\coordinate (A) at (-2,0);
			\coordinate (B) at (-0.7,-2);
			\coordinate (C) at (2,0);
			\foreach \i in {A,B,C}{\coordinate (\i') at ($(\i)+(0,4)$);\draw (\i)--(\i');}
			\path
			($(B)+(B')-(A')$) coordinate (K)
			($(C)!(B)!(K)$) coordinate (E)
			($(B')!(B)!(E)$) coordinate (F)
			;
			\draw (A')--(B')--(C')--cycle (A')--(B);
			\draw (A)--(B)--(K)--(C) (B')--(K) (C)--(B')--(E);
			\draw[dashed,thin](A)--(C) (B)--(C) (B)--(E) (B)--(F) ;
			\foreach \i/\g in {A/180,B/-90,C/0,A'/180,B'/90,C'/0,K/0,E/0,F/0}{\draw[fill=black](\i) circle (1.5pt) ($(\i)+(\g:4mm)$) node[scale=1]{$\i$};}
			\draw pic[draw,angle radius=2mm]{right angle=B--E--K};
		\end{tikzpicture}
	\end{center}
	Kẻ $B'K\parallel A'B$ ($K\in AB$). Suy ra $A'B \parallel (B'CK)$.\\
	Do đó $\mathrm{d}(A'B,B'C)=\mathrm{d}(A'B,(B'CK))=\mathrm{d}(B,(B'CK))$.\\
	Kẻ $BE\perp CK$ ($E\in CK$); $BF\perp B'E$ ($F\in B'E$).\\
	Ta có $\heva{&CK\perp BE\\&CK\perp BB'}\Rightarrow CK\perp (BB'E)\Rightarrow CK\perp BF\quad(1)$.\\
	Mặt khác $BF\perp B'E$ (theo cách dựng) $\quad(2)$.\\
	Từ $(1)$ và $(2)$, suy ra $BF\perp(B'CK)$. Do đó $\mathrm{d}(B,(B'CK))=BF$.\\
	Ta có $BB'=\sqrt{A'B^2-A'B'^2}=\sqrt{(\sqrt{6})^2-(\sqrt{2})^2}=2$.\\
	$\widehat{ABC}=60^\circ\Rightarrow \widehat{CBK}=180^\circ-60^\circ=120^\circ$.\\
	Vì $A'B'KB$ là hình bình hành nên $A'B'=BK=BC=\sqrt{2}$. Suy ra tam giác $CBK$ cân tại $B$.\\
	$\Rightarrow \widehat{KBE}=\dfrac{1}{2}\widehat{CBK}=60^\circ$.\\
	Xét tam giác vuông $KBE$, ta có $\cos\widehat{KBE}=\dfrac{BE}{BK}\Rightarrow BE=BK\cdot\cos\widehat{KBE}=\sqrt{2}\cdot\cos 60^\circ=\dfrac{\sqrt{2}}{2}$.\\
	Áp dụng hệ thức lượng cho tam giác vuông $B'BE$ ta có\\
	$$\dfrac{1}{BF^2}=\dfrac{1}{BB'^2}+\dfrac{1}{BE^2}\Rightarrow BF=\dfrac{BB'\cdot BE}{\sqrt{BB'^2+BE^2}}=\dfrac{2}{3}\approx 0{,}67.$$
	Vậy $\mathrm{d}(A'B,B'C)=BF\approx 0{,}67$.
	}
\end{ex}

%

\begin{ex}%[Nguồn: Bộ đề minh họa Moon 2024-2025]%[1H8V5-6]
	\immini{Ta coi Trái Đất là hình cầu hoàn hảo với bán kính $R=6\,370$ (km) và diện tích toàn phần là $S=4\pi R^2$. Các phi hành gia từ tàu vũ trụ chỉ có thể nhìn thấy một phần bề mặt Trái Đất. Ở độ cao $h$, phần diện tích Trái Đất các phi hành gia có thể nhìn thấy sẽ được tính theo công thức $S_T=2\pi R^2\left(1-\dfrac{R}{R+h}\right)$, trong đó $R$ là bán kính Trái Đất. Gọi $K$ là tỉ số diện tích bề mặt Trái Đất nhìn thấy được ở độ cao $h$ với diện tích toàn phần của Trái Đất.}{	\begin{tikzpicture}
		\tikzset{earth/.pic={
		\shade[ball color=blue!80!black] (1.12,1.34) circle (2);
		\shade[ball color=green!90!black] (1.6,3.16) .. controls +(-18:0.68) and +(98:0.72) ..
		(2.84,1.66) -- (2.78,1.68) .. controls +(-120:0.1) and +(90:0.1) ..
		(2.82,1.38) .. controls +(-90:0.02) and +(90:0.02) ..
		(2.74,1.28) -- (2.66,1.4) -- (2.66,1.24) -- (2.62,1.24) -- (2.6,1.52) --
		(2.54,1.52) -- (2.48,1.72) -- (2.26,1.48) -- (2.26,1.32) -- (2.2,1.28) --
		(2.12,1.7) -- (2.08,1.66) -- (1.96,1.8) -- (1.82,1.8) -- (1.76,1.88) --
		(1.7,1.84) -- (1.58,1.98) -- (1.56,1.96) -- (1.64,1.78) -- (1.72,1.78) --
		(1.72,1.82) -- (1.76,1.82) .. controls +(-50:0.4) and +(14:0.14) ..
		(1.46,1.4) .. controls +(-90:0.2) and +(180:0.08) ..
		(1.64,1.36) .. controls +(-90:0.18) and +(54:0.16) ..
		(1.34,0.82) -- (1.38,0.56) .. controls +(-79:0.06) and +(63:0.08) ..
		(1.22,0.36) -- (1.22,0.26) -- (1.16,0.22) .. controls +(-96:0.1) and +(0:0.1) ..
		(0.94,0) -- (0.8,0) .. controls +(90:0.16) and +(-120:0.3) ..
		(0.74,0.68) .. controls +(60:0.1) and +(-90:0.02) ..
		(0.64,0.96) -- (0.64,1.12) -- (0.56,1.12) .. controls +(104:0.04) and +(0:0.04) ..
		(0.5,1.18) -- (0.46,1.18) .. controls +(190:0.4) and +(-90:0.1) ..
		(0,1.42) -- (0,1.68) .. controls +(90:0.1) and +(-146:0.08) ..
		(0.16,1.96) -- (0.16,2.06) -- (0.26,2.16) -- (0.32,2.16) .. controls +(0:0.12) and +(180:0.08) ..
		(0.5,2.24) -- (0.68,2.24) -- (0.68,2.08) -- (0.88,2) -- (0.92,2.08) .. controls +(-16:0.08) and +(180:0.1) ..
		(1.14,2.02) -- (1.18,2.02) .. controls +(0:0.08) and +(-90:0.08) ..
		(1.28,2.2) -- (1.1,2.18) -- (1.04,2.36) -- (0.94,2.32) .. controls +(-68:0.06) and +(0:0.04) ..
		(0.94,2.2) -- (0.86,2.38) -- (0.7,2.5) -- (0.68,2.48) -- (0.82,2.34) ..
		controls +(-117:0.06) and +(-53:0.12) ..
		(0.72,2.24) -- (0.8,2.28) -- (0.62,2.46) -- (0.52,2.42) -- (0.4,2.32) ..
		controls +(-98:0.08) and +(-10:0.2) .. (0.2,2.24) -- (0.2,2.44) --
		(0.38,2.44) -- (0.32,2.58) -- (0.3,2.56) -- (0.3,2.6) -- (0.6,2.74) ..
		controls +(90:0.04) and +(-90:0.04) ..
		(0.58,2.82) .. controls +(0:0) and +(0:0) ..
		(0.62,2.86) -- (0.7,2.8) -- (0.68,2.9) -- (0.56,2.88) --
		(0.52,2.98) .. controls +(27:0.1) and +(180:0.08) ..
		(0.92,3.22) -- (0.98,3.22) .. controls +(0:0.08) and +(124:0.04) ..
		(1.24,3.12) -- (1.08,3.14) -- (1.2,3.04) -- (1.22,3.08) --
		(1.3,3.1) -- (1.3,3.16) -- (1.36,3.16) -- (1.36,3.1) -- cycle;
		}}
		\tikzset{vetinh/.pic={
		\draw (5.76,24.88) .. controls +(-160:0.3) and +(70:0.3) ..
		(1.48,20.56) .. controls +(-110:1) and +(-165:1) ..
		(7.92,14.08) .. controls +(22:0.4) and +(-135:0.8) ..
		(8.84,14.84) -- (9.56,15.56) -- (9.96,15.12) -- (10.4,14.72) -- (8.96,13.28) .. controls +(-10:2) and +(110:2) ..
		(13.2,9.04) -- (14.08,9.96) -- (15,10.88) -- (15.8,10.04) -- (14.92,9.16) .. controls +(-133:1.2) and +(63:0) ..
		(13.96,8.04) .. controls +(-110:1) and +(-165:1) ..
		(20.44,1.6) .. controls +(11:0.4) and +(-136:2.4) ..
		(22.72,3.68) .. controls +(46:2.4) and +(-117:0) ..
		(24.84,5.96) .. controls +(75:0.4) and +(-45:4.4) ..
		(21.88,9.48) .. controls +(135:4.4) and +(15:0.4) ..
		(18.36,12.44) .. controls +(-166:0) and +(45:0.8) ..
		(17.44,11.68) -- (16.72,10.96) -- (15.92,11.8)
		.. controls +(45:1.5) and +(-60:0.5) ..
		(17.08,14.1) .. controls +(43:2) and +(-45:0.65) ..
		(17.16,17.36) .. controls +(135:1) and +(45:1.5) ..
		(14,17.2) .. controls +(150:0.6) and +(45:1.5) ..
		(11.96,16.28) -- (11.32,15.64) -- (10.48,16.48) .. controls +(47:1.2) and +(-104:0) ..
		(12.32,18.48) .. controls +(70:0.4) and +(-45:4.4) ..
		(9.44,21.96) .. controls +(136:2.8) and +(-18:0) ..
		(6.4,24.88) .. controls +(153:0.4) and +(27:0.4) ..
		(5.76,24.88) --cycle;
		
		%Ô Vuông trên cánh
		\foreach \luc in {0,2.4,4.8,17.6,20,22.4}{
		\draw[rounded corners,shift={(-45:\luc)}] (7.2,21.96) -- (6.12,20.88) -- (4.96,22) -- (6.08,23.08) --cycle
		[shift={(-135:2.5)}] (7.2,21.96) -- (6.12,20.88) -- (4.96,22) -- (6.08,23.08) --cycle;
		}
		
		%Thân giữa
		\draw (3.12,12.32) .. controls +(-120:0.4) and +(106:0.4) ..
		(3.16,10.96) .. controls +(-71:0.8) and +(126:1.2) ..
		(4.72,8.2) -- (5,7.8) --(5.6,8.84) .. controls +(90:0) and +(-90:0.4) ..
		(4.32,10.92) .. controls +(90:0.4) and +(147:1.2) ..
		(6.2,10.44) .. controls +(-33:1.6) and +(124:1.6) ..
		(10.28,6.4) .. controls +(-56:1.2) and +(0:0.4) ..
		(10.84,4.44) .. controls +(180:0.4) and +(0:0) ..
		(8.72,5.68) --
		(7.8,5.16) .. controls +(-45:2) and +(-150:1) ..
		(12.24,3.24) .. controls +(25:1) and +(-50:3) ..
		(11.72,10.04) .. controls +(125:0.8) and +(-38:0.8) ..
		(10.04,11.76) .. controls +(140:2.5) and +(65:1.6) .. cycle;
		
		%Ống nhiệt thân giữa
		\draw (6.92,8.32) --
		(5.56,6.04) .. controls +(0:0) and +(-45:0) ..
		(5.48,6.08) .. controls +(135:0) and +(-18:0) ..
		(5.16,6.2) .. controls +(167:0.8) and +(90:0.8) ..
		(3.8,5.08) .. controls +(-90:0.8) and +(180:0.8) ..
		(4.96,3.92) .. controls +(0:1) and +(-55:0.7) ..
		(6,5.64) --
		(8.28,7.12) .. controls +(39:0.4) and +(-45:0.4) ..
		(7.96,8.08) .. controls +(135:0.8) and +(54:0.4) ..
		(6.92,8.32) --cycle;
		
		\draw (1.8,5.4) .. controls +(-143:0.4) and +(102:0.4) ..
		(1.68,3.68) .. controls +(-77:1.2) and +(174:1.2) ..
		(3.88,1.76) .. controls +(-6:0.4) and +(-162:0.4) ..
		(5.16,1.84) .. controls +(27:0) and +(-53:0) ..
		(5.28,2.28) .. controls +(170:0.5) and +(-40:0.5) ..
		(3.28,2.88) .. controls +(160:0.5) and +(-85:1) ..
		(2.2,5.4) .. controls +(162:0) and +(34:0) ..
		(1.8,5.4) --cycle;
		
		\draw (0.08,5.28) .. controls +(-119:0.4) and +(105:0.8) ..
		(0,2.92) .. controls +(-76:1.2) and +(153:1.2) ..
		(1.88,0.44) .. controls +(-35:0.8) and +(195:1) ..
		(5.2,0.2) -- (5.23,0.75) .. controls +(-174:0.4) and +(-17:0.4) ..
		(3.2,0.88) .. controls +(163:0.8) and +(-63:0.8) ..
		(1.04,2.68) .. controls +(120:1) and +(-90:0) ..
		(0.6,5.28) .. controls +(124:0) and +(63:0) ..
		(0.08,5.28) --cycle;
		}}
		\path (0,0) coordinate (O);
		\path (0,0) coordinate (A);
		\draw[red,dashed] (0.5,-1.5) arc (-90:90:2cm);
		\path (0,0) pic[scale=0.4]{earth};
		\path (2,0) pic[scale=0.04,rotate=-80]{vetinh};
	\end{tikzpicture}}
	\choiceTF
	{Công thức tính $K$ là $K=\dfrac{1}{2}\left(1-\dfrac{h}{R+h}\right)$}
	{Trong một chuyến bay của tàu con thoi, các phi hành gia đã thực hiện một hoạt động ngoài tàu ở độ cao $280$ km. Có $2{,}5\%$ (làm tròn kết quả đến hàng phần mười) diện tích bè mặt Trái Đất có thể nhìn thấy ở độ cao đó}
	{Muốn nhìn thấy $\dfrac{1}{4}$ diện tích bề mặt Trái Đất, các phi hành gia cần đưa tàu con thoi đạt đến độ cao $6\,470$ km}
	{\True Khi độ cao $h$ càng tăng lên thì $K$ càng tăng nhưng không vượt quá $50\%$}
	\loigiai{
	\begin{itemchoice}
		\itemch	Ta có tỷ số $K$ là diện tích nhìn thấy được chia cho diện tích toàn phần
		\[
		K =\dfrac{S_T}{S}=\dfrac{2\pi R^2\left(1-\dfrac{R}{R+h}\right)}{4\pi R^2}
		= \dfrac{1}{2}\left(1-\dfrac{R}{R+h}\right)=\dfrac{1}{2}\cdot \dfrac{h}{R+h}.
		\]
		\itemch Với bán kính $R=6\,370$ km và chiều cao $h=280$ km thì tỷ số $K$ tại độ cao $h$ là
		\[
		K=\dfrac{1}{2}\cdot \dfrac{h}{R+h}=\dfrac{1}{2}\cdot \dfrac{280}{6\,370+280}=\dfrac{2}{95}\approx0{,}021\approx 2{,}1\%.
		\]
		\itemch Với $K=\dfrac{1}{4}$ thì
		\[
		K=\dfrac{1}{2}\cdot \dfrac{h}{R+h}\Leftrightarrow \dfrac{1}{4}=\dfrac{1}{2}\cdot \dfrac{h}{6\,370+h}\Leftrightarrow h=6\,370\,\,\rm{(km)}.\]
		\itemch	Khi $h$ càng lớn ta có
		\[
		\lim\limits_{h\rightarrow +\infty} K =\lim\limits_{h\rightarrow +\infty} \dfrac{1}{2}\dfrac{h}{R+h}=\dfrac{1}{2}\lim\limits_{h\rightarrow +\infty} \dfrac{1}{\dfrac{R}{h}+1}=\dfrac{1}{2}=0{,}5=50\%.
		\]
	\end{itemchoice}
	}
\end{ex}

\begin{ex}%[Nguồn: Bộ đề minh họa Moon 2024-2025]%[1H8V5-6]
	Một quả bóng hình cầu có bán kính $r$ đang được treo trong một góc của tường nhà (hai bờ tường vuông góc), một điểm $B$ cố định nằm trên mép của hai bờ tường và cách mặt đất $80$ cm, sợi dây treo bóng có độ dài $AB = 30$ cm và dây cũng là độ dài ngắn nhất nối điểm $B$ với mặt xung quanh của quả bóng. Biết rằng quả bóng tiếp xúc với hai bên bờ tường và điểm thấp nhất của quả bóng cách mặt đất $20$ cm. Hỏi đường kính của quả bóng là bao nhiêu centimet (làm tròn kết quả đến hàng đơn vị).
	\begin{center}
		\begin{tikzpicture}[scale=1,>=stealth, font=\footnotesize, line join=round, line cap=round]
			\def\a{4}
			\path	(0:0) coordinate (O)
			($(O)+(90:\a)$) coordinate (z)
			($(O)+(90:3.6)$) coordinate (T)
			($(O)+(90:3)$) coordinate (B)
			($(B)+(90:1)$) coordinate (B1)
			
			($(B)+(-100:2)$) coordinate (I)
			($(B)+(-100:1)$) coordinate (A)
			($(I)+(-90:2)$) coordinate (H)
			($(H)+(180:0.5)$) coordinate (H1)
			($(H)+(0:0.5)$) coordinate (H2)
			($(H2)+(-90:0.15)$) coordinate (H3)
			($(H1)+(-90:0.2)$) coordinate (H4)
			($(O)+(-30:\a)$) coordinate (y)
			($(O)+(-150:\a)$) coordinate (x)
			($(B1)+(-45:3)$) coordinate (y1)
			($(B1)+(-150:3)$) coordinate (x1)
			($(O)+(-45:3)$) coordinate (y2)
			($(O)+(-150:3)$) coordinate (x2);
			%	\draw[dashed,thick]	(B)--(D);
			\draw [fill=gray!40] (B1)--(y1)--(y2)--(O)--(B1);
			\draw [fill=gray!40] (B1)--(x1)--(x2)--(O)--(B1);
			\draw[->,thick] (O)--(-150:4)node[left]{$x$};
			\draw[->,thick] (O)--(-45:4)node[right]{$y$};
			\draw[->,thick] (O)--(90:4.7)node[left]{$z$};
			\draw [dashed] (A)--(I)--(H);
			\draw (H1)--(H2);
			\fill[pattern=north east lines, pattern color=black] (H1) rectangle (H3);
			\draw [thick] (B)--(A);
			\shade[ball color=blue!60,opacity=0.5] (I) circle (1cm);
			\foreach \x/\g in {O/-90,I/170,A/110,B/0,H/45}
			\fill[black]	(\x) circle (1pt)
			($(\g:3mm)+(\x)$) node {$\x$};
			%Hình chóp S.ABC có SA vuông góc đáy
		\end{tikzpicture}
	\end{center}
	\shortans[0]{$38$}
	\loigiai{
	Gọi $r$ là bán kính của quả cầu, ta thấy tâm $I$ của quả cầu cách trục $Oz$ một khoảng bằng $r\sqrt{2}$.\\
	\immini{Gọi $K$ là hình chiếu vuông góc của $I$ lên trục $Oz$.\\
	Ta tính được $IK=\sqrt{2}r$, $IH=r+20$, $BK=60-r$, $IB=r+30$.\\
	Xét tam giác vuông $IBK$, vuông tại $K$, theo định lý Pytago ta có
	\[IK^2+BK^2=IB^2\Leftrightarrow -2x^2+180x-2700=0.\]
	Giải phương trình được hai nghiệm $r=45+3\sqrt{3}$ (loại vì $r+20>80$) và $r=45-3\sqrt{3}\approx 19$ (thỏa mãn).\\
	Vậy đường kính của quả cầu bằng $38$ cm.
	}{\begin{tikzpicture}[scale=0.6,>=stealth, font=\footnotesize, line join=round, line cap=round]
		\def\a{4}
		\path	(0:0) coordinate (O)
		($(O)+(90:8)$) coordinate (B)
		($(O)+(180:3)$) coordinate (H)
		($(O)+(90:4)$) coordinate (K)
		($(H)+(90:4)$) coordinate (I)
		($(B)!0.6!(I)$) coordinate (A);
		\draw (I) rectangle (O);
		\draw [] (I)--(B)--(K);
		\draw (I) circle (2cm);
		\foreach \x/\g in {O/-90,I/170,K/0,B/0,H/-90,A/90}
		\fill[black]	(\x) circle (1pt)
		($(\g:3mm)+(\x)$) node {$\x$};
		%Hình chóp S.ABC có SA vuông góc đáy
	\end{tikzpicture}}
	}
\end{ex}

%
\caukq

\begin{ex}%[Nguồn: Bộ đề minh họa Moon 2024-2025]%[1H8V6-2]
	Cho hình chóp $S.ABC$ có $ABC$, $SAB$ là các tam giác đều và mặt bên $(SAB)$ vuông góc với mặt đáy. Gọi $\alpha$ là góc phẳng nhị diện $[S,BC,A]$. Tính $\cos^2\alpha$.
	\par\shortans{$0{,}2$}
	\loigiai{
	\begin{center}
		\begin{tikzpicture}[scale=1, font=\footnotesize,>=stealth, line width=1pt]%<DTools>
			%Gán số liệu.
			\def\canhAC{4};\def\canhBA{2};\def\gocBAC{-50};\def\h{3};\def\xdinhS{0};
			%Gán tọa độ.
			\coordinate (A) at (0,0);
			\coordinate (B) at ($(A)+(\gocBAC:\canhBA)$);
			\coordinate (C) at ($(A)+(0:\canhAC)$);
			\path
			($(A)!.5!(B)$) coordinate (M)
			+(90:\h) coordinate (S)
			($(B)!.5!(C)$) coordinate (N)
			($(B)!.5!(N)$) coordinate (P)
			;
			
			%Vẽ khối chóp S.ABC.
			\draw (S)--(B) (S)--(A)--(B) (S)--(C)--(B)
			(S)--(M)
			(S)--(P)
			\foreach \x/\y/\z in {B/M/S,P/M/S,N/P/M,C/N/A}{
			pic[draw, thin, angle radius = 6pt]{right angle = \x--\y--\z}
			}
			pic[draw, thin, angle radius = 12pt, "$\alpha$", angle eccentricity = 1.5]{ angle = S--P--M}
			
			;
			\draw[dashed] (A)--(C)
			(M)--(P)
			(A)--(N)
			
			;
			%Gán nhãn.
			\foreach \x/\y in {S/90,A/180,B/-90,C/0,M/180,P/-30,N/-30}{\fill (\x) circle (1pt) ($(\x)+(\y:0.3cm)$) node{$\x$};}
		\end{tikzpicture}
	\end{center}
	Gọi $M$ là trung điểm của $AB$, $N$ là trung điểm của $BC$, $P$ là trung điểm của $BN$.\\
	Ta có $MP\parallel AN$ mà $AN\perp BC$ (do tam giác $ABC$ đều) nên $MP\perp BC$.\\
	Mặt bên $(SAB)$ vuông góc với đáy mà $SM\perp AB$ ($SAB$ là tam giác đều), suy ra $SM \perp (ABC) \Rightarrow SM \perp BC$.\\
	Vậy góc $\alpha =\widehat{SPM} \Rightarrow \cos \alpha = \dfrac{MP}{SP}$.\\
	Gọi $a$ là độ dài cạnh của tam giác đều $ABC$ và $SAB$.\\
	Ta có $MP=\dfrac{AN}{2}=\dfrac{a\sqrt{3}}{4}$; $AM=\dfrac{a\sqrt{3}}{2}$.\\
	Suy ra $SP=\sqrt{SM^2+MP^2}=\sqrt{\dfrac{3a^2}{4}+\dfrac{3a^2}{16}}=\dfrac{a\sqrt{15}}{4}$.\\
	Vậy $\cos \alpha = \dfrac{MP}{SP}=\dfrac{\dfrac{a\sqrt{3}}{4}}{\dfrac{a\sqrt{15}}{4}}=\dfrac{1}{\sqrt{5}}$.
	Suy ra $\cos^2 \alpha=\dfrac{1}{5}=0{,}2$.
	}
\end{ex}

\begin{ex}%[Nguồn: Bộ đề minh họa Moon 2024-2025]%[1H8V6-2]
	Cho hình lập phương $ABCD.A'B'C'D'$. Số đo của góc nhị diện $\left[B',A'C,D' \right]$ bằng bao nhiêu độ?
	\shortans[oly]{120}
	\loigiai{
	\begin{center}
		\begin{tikzpicture}[scale=0.6, font=\footnotesize,line join=round, line cap=round, >=stealth]
			\path
			(0,0) coordinate (A)
			++(-130:3) coordinate (B)
			++(0:4) coordinate (C)
			($(A)+(C)-(B)$) coordinate (D)
			($(A')!0.3!(C)$) coordinate (H)
			;
			\foreach \i in {A,B,C,D}{
			\coordinate (\i') at ($(\i)+(0,4)$);
			}
			\draw (A')--(B')--(C')--(D')--cycle;
			\draw (B)--(B') (C)--(C') (D)--(D')  (B)--(C)--(D) (A')--(C') (B')--(D');
			\draw[dashed,thin](B)--(A)--(A')--(C) (A)--(D) (B')--(C)--(D') (D')--(H)--(B');
			\draw (A')--(C') node[midway,above right]{$O$};
			\foreach \i/\g in {A'/90,B'/90,C'/90,D'/90,A/-90,B/-90,C/-90,D/-90,H/-20}
			\fill[black] (\i) circle(1pt)+(\g:5mm)node[scale=1]{$\i$};
		\end{tikzpicture}
	\end{center}
	Ta có $\heva{&A'C'\perp B'O\\&AA'\perp B'O\\&AA',A'C'\in \left( AA'CC'\right) }$ suy ra $B'O\perp\left(AA'CC'\right)\Rightarrow B'O\perp A'C$.\\
	Lại có $OH\perp A'C\Rightarrow A'C\perp \left(B'OH \right)\Rightarrow A'C\perp B'H$.\\
	Tương tự ta chứng minh $A'C\perp D'H$.\\
	Suy ra góc nhị diện $\left[B',A'C,D' \right]$ là góc $\widehat{B'HD'}$.\\
	Cho các cạnh hình lập phương là $a$.\\
	Ta có $B'D'=a\sqrt{2}$.\\
	Xét tam giác $A'B'C$ vuông tại $B'$ ta có
	$$\dfrac{1}{B'H^2}=\dfrac{1}{A'B'^2}+\dfrac{1}{B'C^2}=\dfrac{1}{a^2}+\dfrac{1}{2a^2}=\dfrac{3}{2a^2}.$$
	Suy ra $B'H=a\sqrt{\dfrac{2}{3}}$.\\
	Tương tự $D'H=a\sqrt{\dfrac{2}{3}}$.\\
	Ta có $\cos \widehat{B'HD'}=\dfrac{B'H^2+HD'^2-B'D'^2}{2\cdot B'H\cdot HD'}\Rightarrow \widehat{B'HD'}=120^\circ$.\\
	Vậy số đo của góc nhị diện $\left[B',A'C,D' \right]$ bằng $120^\circ$.
	}
\end{ex}

\begin{ex}%[Nguồn: Bộ đề minh họa Moon 2024-2025]%[1H8V6-2]
	Cho hình chóp $S.ABCD$ có đáy là hình thoi, $AC = BC = 2$, $SAB$ là tam giác đều, số đo của góc nhị diện $\left[S, CD, B\right]$ bằng $60^\circ$. Thể tích khối chóp $S.ABCD$ bằng bao nhiêu? (Làm tròn kết quả đến hàng phần mười).
	
	\shortans[oly]{1{,}7}
	\loigiai{
	\begin{center}
		\begin{tikzpicture}[>=stealth,line join=round,line cap=round,font=\footnotesize,scale=1]
			\tikzset{
			pics/hinhChopTuGiacDeu/.style n args={6}{
			code={
			\tikzset{
			declare function={a=4;b=2.5 ;h=3;goc=-150;}
			}
			\path
			(0,0)coordinate (#1)+(0:a)coordinate (#2)+(goc:b)coordinate (#4)
			($(#2)+(#4)-(#1)$)coordinate (#3)
			($(#1)!.5!(#4)$)coordinate (m) ($(m)!.5!(#3)$)coordinate (#5) ($(#5)+(90:h)$)coordinate (#6)
			;
			}
			}}
			\path
			(0,0)pic {hinhChopTuGiacDeu={A}{D}{C}{B}{O}{S}}
			($(A)!.5!(B)$)coordinate (M)
			($(C)!.5!(D)$)coordinate (N)
			($(M)!.5!(C)$)coordinate (H)
			pic[draw,angle radius=2mm,angle eccentricity=2.4]{right angle=S--H--C}
			;
			\foreach \pointo/\pointt in {S/B,S/C,S/D,B/C,C/D}{
			\draw[fill=black](\pointo)--(\pointt);
			}
			\foreach \pointo/\pointt in {S/A,A/B,A/D,A/C,S/M,C/M,S/H}{
			\draw[fill=black,dashed](\pointo)--(\pointt);
			}
			\foreach \point/\goc in {S/90,A/135,B/190,D/10,C/-60,M/-90,H/-90}{
			\draw[fill=black](\point)circle(.8pt)+(\goc:2mm)node[scale=.8]{$\point$};
			}
		\end{tikzpicture}
	\end{center}
	Ta có $ABCD$ là hình thoi và $AC = BC = 2$, do đó $\triangle ABC$ là tam giác đều.\\
	Gọi $M$ là trung điểm của $AB$.\\
	Ta có $\triangle SAB$ và $\triangle ABC$ là các tam giác đều nên $SM\perp AB$ và $CM\perp AB$.\\
	Suy ra $AB\perp (SCM)$.\\
	Lại có $CD\parallel AB$ nên $CD\perp (SCM)$, do đó $CD\perp CM$ và $CD\perp SC$.\\
	Từ đó suy ra góc $SCM$ là góc nhị diện $\left[S, CD, B\right]$. Do đó $\widehat{SCM}=60^{\circ}$.\\
	Gọi $H$ là hình chiếu của $S$ trên cạnh $MC$, khi đó $SH\perp MC$ và $SH\perp AB$ ($AB\perp (SMC)\supset SH$).\\
	Suy ra $SH\perp (ABCD)$.\\
	Lại có $\triangle SAB=\triangle ABC$ (c.c.c) nên $SM=CM=\sqrt{3}$.\\
	Do đó $\triangle SMC$ là tam giác cân.\\
	Mà $\widehat{SCM}=60^{\circ}$ nên $\triangle SCM$ là tam giác đều.\\
	Suy ra $SH=\dfrac{3}{2}$.\\
	Ta có $S_{ABCD}=2S_{AB3C}=2\cdot \dfrac{2^2\sqrt{3}}{4}=2\sqrt{3}$.\\
	Do đó $V_{S.ABCD}=\dfrac{1}{3}\cdot S_{ABCD}\cdot SH=\dfrac{1}{3}\cdot 2\sqrt{3}\cdot \dfrac{3}{2}\approx1{,}7$ (đvtt).
	}
\end{ex}

%================================PHẦN III
\subsubsection*{Phần III. Câu trắc nghiệm trả lời ngắn. Thí sinh trả lời từ câu 1 và câu 6.}
\setcounter{ex}{0}
\Opensolutionfile{ans}[ans/ans-De4-phanIII]

\begin{ex}%[Nguồn: Bộ đề minh họa Moon 2024-2025]%[1H8V7-3]
	Cho hình chóp đều $S.ABCD$ có cạnh đáy bằng $4$, khoảng cách giữa hai đường thẳng $SA$ và $CD$ bằng $2$. Thể tích của khối chóp $S.ABCD$ bằng bao nhiêu? (làm tròn kết quả đến hàng phần mười)
	\shortans[1]{6{,}2}
	\loigiai{
	\immini{
	Diện tích mặt đáy $S_{ABCD}=4^2=16$.\\
	Chọn $(SAB)$ chứa $SA$, ta có $\heva{
	&CD\parallel AB\\
	&AB\subset (SAB).
	}$\\
	Suy ra $CD\parallel (SAB)$ nên
	\begin{align*}
		\mathrm{d}(CD,SA)&=\mathrm{d}(CD,(SAB))=\mathrm{d}(D,(SAB))\\
		&=2\mathrm{d}(O,(SAB))=2\\
		\Rightarrow &\mathrm{d}(O,(SAB))=1.
	\end{align*}
	Gọi $H$ là hình chiếu vuông góc của $O$ trên $AB$ và
	}{
	\begin{tikzpicture}[declare function={a=2;b=4;h=4;},line join=round]
		\path (0,0) coordinate (A)
		(-145:a) coordinate (B)
		(b,0) coordinate (D)
		($ (B)!0.5!(D) $) coordinate (O)
		($(O)+(0,h) $) coordinate (S);
		\path ($(D)-(A)+(B)$) coordinate (C);
		\path ($(A)!0.5!(B)$) coordinate (H);
		\path ($(S)!0.7! (H)$) coordinate (K);
		\draw[dashed] (S)--(A)--(B) (A)--(D)--(B) (A)--(C) (S)--(O)--(H)--(S) (K)--(O)--(H);
		\draw (B)--(C)--(D) (B)--(S)  (D)--(S)--(C);
		\foreach \x/\y/\z in {S/O/B,S/O/C,A/O/B,O/H/A,O/K/S}{
		\path pic[draw,angle radius=5pt]{right angle= \x--\y--\z};
		}
		\foreach \t/\g in {A/160,B/-90,C/-90,D/0,S/90,O/-90,H/160,K/-140}{
		\draw[fill=black] (\t) circle (1pt) node[shift={(\g:7pt)},font=\scriptsize]{$ \t $};
		}
	\end{tikzpicture}
	}
	$K$ là hình chiếu vuông góc của $O$ trên $SH$, ta có
	\[
	\heva{
	&AB\perp OH\\
	&AB\perp SO
	}\Rightarrow AB\perp (SOH)\Rightarrow AB\perp OK.\quad (1)
	\]
	Mà $OK\perp SH$. \quad (2)\\
	Suy ra $OK\perp (SAB)\Rightarrow \mathrm{d}(O,(SAB))=OK=1$.\\
	Với $OH=\dfrac{1}{2}AD=2$ ta có
	\[
	\dfrac{1}{OK^2}=\dfrac{1}{OH^2}+\dfrac{1}{SO^2}\Leftrightarrow
	\dfrac{1}{SO^2}=\dfrac{1}{1^2}-\dfrac{1}{2^2}=\dfrac{3}{4}\Rightarrow
	SO^2=\dfrac{4}{3}\Rightarrow SO=\dfrac{2\sqrt{3}}{3}.
	\]
	Thể tích khối chóp $S.ABCD$ là
	\[
	V=\dfrac{1}{3}\cdot S_{ABCD}\cdot SO=\dfrac{32\sqrt{3}}{9}\approx 6{,}2.
	\]
	}
\end{ex}

\begin{ex}%[Nguồn: Bộ đề minh họa Moon 2024-2025]%[1H8V7-3]
	Cho khối chóp đều $S.ABCD$ có $A C=4$, hai mặt phẳng $(SAB)$ và $(SCD)$ vuông góc với nhau. Thể tích của khối chóp đã cho bằng bao nhiêu? (làm tròn kết quả đến hàng phần mười).
	\par\shortans[]{$3{,}8$} \par
	\loigiai{
	Gọi $O$ là tâm của  hình vuông $ABCD$. Do $S.ABCD$ là hình chóp đều nên $SO \perp (ABCD) \Rightarrow SO \perp AB$.\\ Ta có  $S$ là một điểm chung của hai mặt phẳng $(SAB)$ và  $(SCD)$. Lại có $A B \subset(S A B) C D \subset(S C D)
	$, $A B \parallel C D$ suy ra hai mặt phẳng $(SAB)$ và mặt phẳng $(SCD)$  cắt nhau theo giao tuyến là đường thẳng $\Delta$ đi qua $S$ và song song với $AB$ và $CD$. Gọi $H$ và $K$ lần lượt là trung điểm của $AB$ và $CD$ thì $HK$ đi qua $O$ và $HK \perp AB$.\\
	Ta có $\heva{&SO \perp AB\\&HK \perp AB} \Rightarrow AB \perp (SHK)\Rightarrow \Delta \perp (SHK)$ (Do $\Delta \parallel AB$)\\
	$\Rightarrow(\widehat{(S A) ;(S C D))})=(\widehat{SH ; SK} )=90^{\circ} \Rightarrow SH\perp SK \Rightarrow $ Tam giác $SHK$  vuông tại $S$.\\
	$\Rightarrow AB=\dfrac{AC}{\sqrt{2}}=2\sqrt{2}$; $ S O=\dfrac{1}{2} H K=\dfrac{1}{2} A B=\sqrt{2}$. $S_{A B C D}=A B^2=8$ .\\
	Vậy thể tích khối chóp $S.ABCD$ là
	$V_{S.ABCD}=\dfrac{1}{3} S O \cdot S_{A B C D} =\dfrac{1}{3} \sqrt{2} \cdot 8=\dfrac{8 \sqrt{2}}{3}$.
	
	}
\end{ex}

\begin{ex}%[Nguồn: Bộ đề minh họa Moon 2024-2025]%[2D1C1-5]
	Một hồ nước ở Bắc Ontario đã phục hồi sau một vụ tràn axit khiến tất cả cá hồi ở đó chết. Một chương trình tái thả cá đã thả 800 con cá hồi vào hồ. Ba năm sau, số lượng được ước tính là 6000 con. Sức chứa của hồ được cho là 8000 con. Để đánh giá khả năng tăng trưởng, người ta mô phỏng số lượng cá trong hồ qua từng năm thông qua hàm số $P(t)=\dfrac{c}{1+a \cdot b^{-t}}(a, b, c \in \mathbb{R})$ có đồ thị nhự hình vẽ bên dưới (trong đó $t$ tính theo năm kể từ lúc bắt đầu thả cá vào hồ).
	\begin{center}
		\begin{tikzpicture}[yscale=0.6,>=stealth, font=\footnotesize, line join=round, line cap=round]
			\def\xmin{-1} \def\xmax{9}
			\def\ymin{-1} \def\ymax{9}
			\draw[color=gray!50,dashed] (\xmin,\ymin) grid (\xmax,\ymax);
			\draw[->] (\xmin,0)--(\xmax,0) node [below]{$t$};
			\draw[->] (0,\ymin)--(0,\ymax) node [left]{$P(t)$};
			\node at (0,0) [below left]{$O$};
			\clip (\xmin+0.1,\ymin+0.1) rectangle (\xmax-0.1,\ymax-0.1);
			\draw[smooth,samples=300,domain=0:10] plot(\x,{(8)/(1+9*3^(-\x))});
			\draw (5,7.1) node {$y=P(t)$};
		\end{tikzpicture}
	\end{center}
	Sử dụng mô hình trên hãy tính tốc độ tăng trưởng tối đa (đơn vị con/năm) của đàn cá. Kết quả làm tròn đến hàng đơn vị.
	\loigiai{
	Dựa vào dữ kiện bài toán ta có
	\begin{itemize}
		\item $P(0)=800\Leftrightarrow \dfrac{c}{1+ab^{-0}}=800\Leftrightarrow \dfrac{c}{1+a}=800$.
		\item $P(3)=6000\Leftrightarrow \dfrac{c}{1+ab^{-3}}=6000\Leftrightarrow \dfrac{c}{1+ab^{-3}}=6000$.
		\item Sức chứa của hồ là $8000$ con $\Leftrightarrow \lim\limits_{t\to+\infty}P(t)=8000$.\\
		Khi $t\to+\infty\Rightarrow \heva{&b<1\Rightarrow b^{-t}\to+\infty(L)\\&b>1\Rightarrow b^{-t}\to 0(TM)}\Rightarrow P(t)\to c$.\\
		Khi đó $\lim\limits_{t\to+\infty}P(t)=c=8000$.
	\end{itemize}
	Mà $\dfrac{c}{1+a}=800\Leftrightarrow \dfrac{8000}{1+a}=800\Rightarrow a=9$.\\
	Thay $c=8000$ và $a=9$ vào phương trình $\dfrac{c}{1+ab^{-3}}=6000$ ta được\\
	$\dfrac{8000}{1+9b^-3}=6000\Leftrightarrow 1+9b^{-3}=\dfrac{4}{3}\Leftrightarrow b^{-3}=\dfrac{1}{27}\Leftrightarrow b=3$.\\
	Suy ra $P(t)=\dfrac{8000}{1+9\cdot3^{-t}}$.\\
	Hàm số $P'(t)=-8000\cdot\dfrac{-9\cdot3^{-t}\cdot\ln3}{\left(1+9\cdot3^{-t}\right)^2}=72000\ln3\cdot\dfrac{3^{-t}}{\left(1+9\cdot3^{-t}\right)^2}$.\\
	Để tìm max của $P(t)$ ta có hai cách làm như sau\\
	\textbf{Cách 1:} Tính $P''(t)=\dfrac{-72000\ln^23\cdot3^{-t}\cdot(1-81\cdot3^{-2t})}{(1+9\cdot3^{-t})^2}$
	Ta có $P''(t)=0\Leftrightarrow1-81\cdot3^{-2t}=0\Leftrightarrow t=2$.\\
	Ta có bảng biến thiên
	\begin{center}
		\begin{tikzpicture}
			\tkzTabInit[nocadre=false,lgt=1.2,espcl=2.5,deltacl=0.6]
			{$t$ /0.6,$P''(t)$ /0.6,$P'(t)$ /2}
			{$-\infty$,$2$,$+\infty$}
			\tkzTabLine{,+,$0$,-,}
			\tkzTabVar{-/$ $, +/$2000\ln3$,-/$ $}
		\end{tikzpicture}
	\end{center}
	Vậy tốc độ tăng trưởng tối đa là $2197$ con.\\
	\textbf{Cách 2:} Đặt $3^{-t}=a$ ($a>0$).\\
	$\Rightarrow g(t)=\dfrac{a}{(1+9a)^2}=\dfrac{a}{1+18a+81a^2}=\dfrac{1}{\dfrac{1}{a}+18+81a}$.\\
	Nhận xét $g(t)$ max khi $\left(\dfrac{1}{a}+18+81a\right)$ min.\\
	Áp dụng bất đẳng thức Cauchy cho hai số dương $\dfrac{1}{a}$ và $81a$, ta có\\
	$\dfrac{1}{a}+81a\ge 2\sqrt{\dfrac{1}{a}\cdot81a}\Leftrightarrow \dfrac{1}{a}+81a\ge2\sqrt{81}=18\Rightarrow\dfrac{1}{a}+18+81a\ge36$.\\
	Do đó $\dfrac{1}{a}+18+81a\le\dfrac{1}{36}$.\\
	Dấu \lq\lq=\rq\rq\,xảy ra khi và chỉ khi $\dfrac{1}{a}=81a\Leftrightarrow a=\dfrac{1}{9}$.\\
	Khi đó $P'(t)_{\text{max}}=72000\ln3\cdot\dfrac{1}{36}\approx2197$.
	}
\end{ex}

%Câu đúng sai 4

\begin{ex}%[Nguồn: Bộ đề minh họa Moon 2024-2025]%[2D1C2-7]
	Trên trục $Os$, cho hai chất điểm chuyển động có toạ độ theo thời gian $t$ (giây) lần lượt là $s_1=\sin t$ và $s_2=\sin\left(t+\dfrac{\pi}{3}\right)$ (đơn vị: mét).
	\begin{center}
		\begin{tikzpicture}[line join=round, line cap=round, >=stealth, scale=1]
			\draw[->] (-3.2,0)--(4.5,0)node[below]{$s$};
			\draw (0,0)node[above]{$O$};
			\fill (-0.8,0) circle (2pt) node[above]{$s_1$} (1.5,0) circle (2pt) node[above]{$s_2$};
			\foreach \x in {-3,...,4}{
			\draw (\x,0.1)--(\x,-0.1)node[below]{$\x$};
			}
		\end{tikzpicture}
	\end{center}
	\choiceTF
	{Tại thời điểm ban đầu hai chất điểm cách nhau một khoảng bằng $50$ cm}
	{Khoảng cách giữa hai chất điểm được xác định bởi hàm số $d=s_1-s_2$ (mét)}
	{\True Trong $6$ giây đầu tiên, có hai thời điểm mà vận tốc của hai chất điểm bằng nhau}
	{\True Trong $6$ giây đầu tiên, khoảng cách xa nhất của hai chất điểm là $100$ cm}
	\loigiai{
	\begin{itemchoice}
		\itemch Tại thời điểm bắt đầu thì $t=0$. Khi đó $s_1=\sin 0 = 0$; $s_2=\sin\left(0+\dfrac{\pi}{3}\right)=\dfrac{\sqrt3}{2}$.\\
		Khoảng cách giữa hai chất điểm là $s_2-s_1=\dfrac{\sqrt3}{2}$ m.
		\itemch Khoảng cách giữa hai chất điểm được xác định bởi hàm số $d=|s_1-s_2|$ (mét).
		\itemch Vận tốc của chất điểm thứ nhất và thứ hai lần lượt là $v_1=s_1'=\cos t$ và $v_2=s_2'=\cos\left(t+\dfrac{\pi}{3}\right)$.\\
		Khi hai chất điểm có vận tốc bằng nhau
		\begin{eqnarray*}
			&& v_1=v_2 \\
			&\Leftrightarrow& \cos t = \cos\left(t+\dfrac{\pi}{3}\right) \\
			&\Leftrightarrow& \hoac{&t=t+\dfrac{\pi}{3}+k2\pi \\& t=-t-\dfrac{\pi}{3}+k2\pi} \\
			&\Leftrightarrow& t=-\dfrac{\pi}{6}+k\pi,\, (k\in\mathbb{Z}).
		\end{eqnarray*}
		Trong $6$ giây đầu tiên, tức $0\leq t \leq 6 \Leftrightarrow 0\leq -\dfrac{\pi}{6}+k\pi \leq 6 \Leftrightarrow \dfrac{1}{6} \leq k < 2{,}08$.\\
		Do $k\in\mathbb{Z}$ nên $k\in\{1;2\}$.\\
		Vậy trong $6$ giây đầu tiên, có hai thời điểm mà vận tốc của hai chất điểm bằng nhau.
		\itemch Xét $y=s_1-s_2=\sin t - \sin\left(t+\dfrac{\pi}{3}\right)$.\\
		Tập xác định $\mathscr{D}=\mathbb{R}$.\\
		Ta có $y'=\cos t - \cos\left(t+\dfrac{\pi}{3}\right)$.\\
		Cho $y'=0 \Leftrightarrow \cos t- \cos\left(t+\dfrac{\pi}{3}\right)=0 \Leftrightarrow t=-\dfrac{\pi}{6}+k\pi$, $k\in\mathbb{Z}$.\\
		Bảng biến thiên của $y$ trên đoạn $[0;6]$
		\begin{center}
			\begin{tikzpicture}[font=\footnotesize, line join=round, line cap=round, >=stealth, scale=1]
				\tkzTabInit[lgt=1.2,espcl=2.5,deltacl=0.6]
				{$x$/1, $y'$/0.7, $y$/2}
				{$0$, $\dfrac{5\pi}{6}$, $\dfrac{11\pi}{6}$, $6$}
				\tkzTabLine
				{, + , $0$ , - , $0$ , + , }
				\tkzTabVar
				{-/$-\dfrac{\sqrt3}{2}$ , +/$1$, -/$-1$, +/$-0{,}97$}
			\end{tikzpicture}
		\end{center}
		Suy ra, khoảng cách giữa hai chất điểm là $d=|y|$ có giá trị lớn nhất là $1$ m, hay $100$ cm khi $t=\dfrac{5\pi}{6}$ và $t=\dfrac{11\pi}{6}$.\\
	\end{itemchoice}
	}
\end{ex}

%

\begin{bt}%[Nguồn: Bộ đề minh họa Moon 2024-2025]%[2D1C3-6]
	Một doanh nghiệp dự định sản xuất không quá $500$ sản phẩm. Nếu doanh nghiệp sản xuất $x$ sản phẩm $(1\leq x \leq 500)$ thì doanh thu nhận được khi bán hết số sản phẩm đó là $F(x)=x^3-1999x^2+1001000x+250000$ (đồng), trong khi chi phi sản xuất bình quân cho một sản phẩm là $G(x)=x+1000+\dfrac{250000}{x}$ (đồng). Doanh nghiệp cần sản xuất bao nhiêu sản phẩm để lợi nhuận thu được là lớn nhất?
	\shortans{333}
	\loigiai{
	
	Ta có hàm lợi nhuận
	$$
	\begin{aligned}
		P(x)	&=F(x)-x G(x) \\
		&=x^3-1999 x^2+1001000 x+250000-x^2-1000 x-250000 \\
		&=x^3-2000 x^2+1000000 x.
	\end{aligned}
	$$
	
	Suy ra $P'(x)=3x^2-4000x+1000000$; khi đó $ P'(x)=0\Leftrightarrow\hoac{&x=1000\text{(loại) } \\&x=\dfrac{1000}{3}.}$
	
	Bảng biến thiên
	
	\begin{center}
		\begin{tikzpicture}[font=\normalsize]
			%dòng khai báo
			\tkzTabInit[nocadre=true,lgt=1.2,espcl=3,deltacl=0.6]
			{$x$/0.75, $P(x)$/2}
			{$ 1 $, $ \frac{1000}{3} $, $ 500 $}
			%dòng biến thiên
			\tkzTabVar{-/$P(1)$,+/$ P\left( \frac{1000}{3}\right)  $,-/$P(500)$}
		\end{tikzpicture}
	\end{center}
	
	Ta có $\dfrac{1000}{3}=333{,}33$ và $P(333) > P(334)$ nên dựa vào bảng biến thiên suy ra doanh nghiệp cần sản xuất $333$ sản phẩm để lợi nhuận thu được là lớn nhất.
	}
\end{bt}

%Điền đáp án 5

\begin{bt}%[Nguồn: Bộ đề minh họa Moon 2024-2025]%[2D1C3-6]
	Hệ thống mạch máu chứa các mạch máu gồm động mạch chính, động mạch con, mao mạch và tĩnh mạch để giúp đưa máu từ tim đến các cơ quan và ngược lại. Hệ thống hoạt động để tối ưu hoá (tối thiểu) năng lượng mà tim sử dụng trong quá trình bơm máu. Đặc biệt năng lượng này giảm khi sức cản của máu giảm. Hình vẽ dưới đây minh hoạ một mạch máu chính có bán kính $r_1$ phân nhánh với một góc $\alpha^\circ$ tạo thành một mạch máu nhỏ hơn với bán kính $r_2$.
	\begin{center}
		\begin{tikzpicture}[font=\footnotesize, line join=round, line cap=round, >=stealth, scale=1]
			\draw
			(1,1) ellipse ({0.3} and {0.5})
			(8,1) ellipse ({0.3} and {0.5})
			(8,4) ellipse ({0.15} and {0.4})
			;
			\draw (1,0.5)--(8,0.5) (1,1.5)--(3,1.5) (4.5,1.5)--(8,1.5);
			\draw (7.95,4.38)--(3,1.5) (8,3.6)--(4.5,1.5);
			\draw[dashed] (1,1)--(8,1) (8,4)--(3,1)node[below]{$B$};
			\fill
			(1,1) circle (1pt)node[left=-1mm]{$A$}
			(3,1) circle (1pt)node[below]{$B$}
			(8,4) circle (1pt)node[right=1mm]{$C$}
			;
			\draw[|<->|] (8.75,1)--(8.75,4)node[pos=0.5, right]{$b$};
			\draw[|<->|] (1,0.25)--(8,0.25)node[pos=0.5, below]{$a$};
			\draw[<->] (2,1)--(2,1.5)node[pos=0.5, right]{$r_1$};
			\draw[<->] (6.75,3.25)--(6.5,3.53)node[pos=0.5, right]{$r_2$};
			\path (8,1) coordinate (D) (3,1) coordinate (B) (8,4) coordinate (C)
			pic[angle eccentricity=1.2,"$\alpha$"]{angle=D--B--C};
			\draw (2,2)node[align=left]{Phân nhánh\\mạch máu};
		\end{tikzpicture}
	\end{center}
	Sử dụng mô tả Định luật Poiseuille, người ta đã chứng minh được sức cản của máu theo con đường $ABC$ là
	$$R(\alpha)=C\left(\dfrac{a-b\cot\alpha}{r_1^4}+\dfrac{b}{r_2^4\sin\alpha}\right)$$
	với $C, a, b$ là các hằng số. Khi bán kính mạch máu nhỏ bằng $\dfrac{2}{3}$ bán kính mạch máu chính. Xác định $\alpha$ để sức cản này là nhỏ nhất. (Làm tròn kết quả đến hàng đơn vị).
	\par\shortans{$79$}
	\loigiai{
	Các hằng số $C$, $a$, $b$ là các hằng số dương.\\
	Ta có $r_2=\dfrac{2}{3}r_1 \Leftrightarrow r_2^4=\dfrac{16}{81}r_1^4$.\\
	Khi đó
	$$ R(\alpha)=C\left(\dfrac{a-b\cot\alpha}{r_1^4}+\dfrac{81b}{16r_1^4\sin\alpha}\right) =\dfrac{C}{r_1^4}\left(a-b\cot\alpha + \dfrac{81b}{16\sin\alpha}\right).$$
	Điều kiện xác định $0^\circ<\alpha<180^\circ$.\\
	Ta có $R'(\alpha)=\dfrac{Cb}{r_1^4}\cdot\dfrac{16-81\cos\alpha}{16\sin^2\alpha}$.\\
	Ta có
	\begin{eqnarray*}
		&& R'(\alpha)=0 \\
		&\Leftrightarrow& 16-81\cos\alpha=0 \\
		&\Leftrightarrow& \cos\alpha=\dfrac{16}{81} \\
		&\Leftrightarrow& \hoac{&\alpha\approx 79^\circ + k360^\circ \\& \alpha\approx -79^\circ + k360^\circ.}
	\end{eqnarray*}
	Do $0^\circ < \alpha < 180^\circ$ suy ra $\alpha\approx 79^\circ$.\\
	Bảng biến thiên
	\begin{center}
		\begin{tikzpicture}[font=\footnotesize, line join=round, line cap=round, >=stealth, scale=1]
			\tkzTabInit[lgt=1.2,espcl=2.5,deltacl=0.6]
			{$\alpha$/1, $R'(\alpha)$/0.7, $R(\alpha)$/2}
			{$0^\circ$, $79^\circ$ , $180^\circ$}
			\tkzTabLine
			{, - , $0$ , + ,}
			\tkzTabVar
			{+/, -/, +/}
		\end{tikzpicture}
	\end{center}
	Từ bảng biến thiên, sức cản của mạch máu là nhỏ nhất đạt được khi $\alpha\approx 79^\circ$.
	}
\end{bt}

%

\begin{ex}%[Nguồn: Bộ đề minh họa Moon 2024-2025]%[2D1C3-6]
	Hai lưỡi kéo của một chiếc kéo được gắn vào điểm $A$ như trong hình vẽ. Gọi $a = 20$ cm là khoảng cách từ $A$ đến đầu lưỡi kéo (điểm $B$). Gọi $\beta = 5^\circ$ là góc ở đầu lưỡi kéo được tạo bởi đường thẳng $AB$ và cạnh dưới của lưỡi kéo (đường thẳng $BC$) và gọi $\alpha$ là góc giữa $AB$ với phương ngang. Giả sử một mảnh giấy được cắt theo cách mà tâm của kéo tại $A$ và giấy đều được cố định. Khi các lưỡi kéo đóng lại (tức là góc $\alpha \in (0; 90^\circ)$ trong hình giảm), khoảng cách $f(\alpha)$ giữa $A$ và $C$ tăng lên khi cắt giấy. Giả sử $\alpha$ đang giảm với tốc độ không đổi là $50^\circ$/giây. Tại thời điểm $\alpha = 30^\circ$, tốc độ mà giấy đang bị cắt bằng bao nhiêu cm/s? (làm tròn kết quả đến hàng đơn vị). Biết rằng tốc độ mà giấy bị cắt bằng với tốc độ thay đổi của $f(\alpha)$.
	\begin{center}
		\begin{tikzpicture}
			% Định nghĩa các điểm cơ bản
			\coordinate (A) at (0,0); % Tâm kéo (điểm A)
			\coordinate (B) at (5,3); % Đầu lưỡi kéo dưới
			\coordinate (B') at (5,-3); % Đầu lưỡi kéo trên
			\coordinate (P1) at (-1,-1); % Tay cầm dưới
			\coordinate (P2) at (-1,1); % Tay cầm trên
			\coordinate (D) at (0,1); %điểm trên tâm A
			\coordinate (D') at (0,-1); %điểm trên tâm A
			\coordinate (C) at (1.5,0);
			\coordinate (E) at (-1,0);
			%Tô màu tờ giấy
			\fill[red!30]
			(A)--(5,0)--(B)--(0,3)--cycle;
			\draw[dashed]
			(D) .. controls (2.5,2.5) .. (B); % Đường viền kéo (trên)
			\draw[thick]
			(B) -- (C) (A)--(B)--(C)--cycle; % Đường viền lưỡi kéo (trên)
			\draw[dashed]
			(D') .. controls (2.5,-2.5) .. (B'); % Đường viền kéo (dưới)
			\draw[dashed]
			(A)--(B')--(C); % Đường viền lưỡi kéo (dưới)
			% Tay cầm dưới
			\draw[dashed] (D') .. controls (-1,-2.5) and (-4,-1)..(E);
			\draw[dashed] (-0.5,-0.5) .. controls (-1,-1.8) and (-3,-0.7)..(-0.5,-0.5)--cycle;
			% Tay cầm trên
			\draw[dashed] (D) .. controls (-1,2.5) and (-4,1)..(E);
			\draw[dashed] (-0.5,0.5) .. controls (-1,1.8) and (-3,0.7)..(-0.5,0.5)--cycle;
			% Ghi chú các điểm
			\node[left] at (A) {$A$};
			\node[above right] at (B) {$B$};
			\node[right] at (C) {$C$};
			
		\end{tikzpicture}
	\end{center}
	\shortans{$217$}
	
	\loigiai{
	Trong tam giác $ABC$ có $AB=20$ cm, $f(\alpha)=AC$, $\widehat{ABC}=5^\circ$, $\widehat{BAC}=\alpha$.\\
	Áp dụng định lí sin trong tam giác $ABC$ ta có \[\dfrac{f(\alpha)}{\sin 5^\circ}=\dfrac{20}{\sin\left(175^\circ-\alpha\right)}\Rightarrow f(\alpha)=\dfrac{20\cdot\sin 5^\circ}{\sin\left(175^\circ-\alpha\right)}.\]
	Gọi $\alpha (t)$ là độ lớn góc $\alpha$ tại thời điểm $t$ nên $h(t)=f(\alpha(t))=\dfrac{20\cdot\sin 5^\circ}{\sin\left(175^\circ-\alpha(t)\right)}$.\\
	Tốc độ mà giấy bị cắt là $h'(t)=\dfrac{20\cdot\sin5^\circ\cdot\cos(175^\circ-\alpha)}{\sin^2(175^\circ-\alpha)}\cdot \alpha'(t)$.\\
	Theo đề ta có tại thời điểm $\alpha=30^\circ$ thì $\alpha'(t)=-50$ $\mathrm{cm}/\mathrm{s}$.\\
	Thay vào $h'(t)$ ta được tốc độ giấy bị cắt khi $\alpha=30^\circ$ là
	\[h'(t)=\dfrac{20\cdot\sin5^\circ\cdot\cos(175^\circ-30^\circ)}{\sin^2(175^\circ-30^\circ)}\cdot (-50)\approx 217~\mathrm{cm}/\mathrm{s}.\]
	}
\end{ex}

%

\begin{ex}%[Nguồn: Bộ đề minh họa Moon 2024-2025]%[2D1C4-4]
	Người ta muốn làm một sàn nổi hình vuông nối liền một sân khấu nổi trên mặt hồ có bờ là một nhánh đồ thị của hàm số $y=\dfrac{x+1}{x-1}\, (C)$, với đất liền là nửa mặt phẳng giới hạn bởi đường thẳng $y=-x+1$.
	\begin{center}
		\begin{tikzpicture}[line cap=butt,line join=miter,>=stealth,
			declare function={
			a=1; b=1; c=1; d=-1;k=4.5;
			f(\x)=(a*(\x) + b)/(c*(\x) + d);
			tcd=-d/c; tcn=a/c;
			xmin=tcd-3; xmax=tcd+k; ymin=tcn-3; ymax=tcn+k;
			} ]
			\path ({5/3},4)coordinate (A) (4,{f(4)})coordinate (B)
			(-1,2)coordinate (M) (2,-1)coordinate (N)
			($(M)!(B)!(N)$)coordinate (C)($(M)!(A)!(N)$)coordinate (D);
			\fill[green!50] (1.44,5.5)--plot[domain=1.44:xmax] (\x, {f(\x)})--(xmax,ymax)--cycle;
			\fill[green!50] (-2,3)--(-2,-2)--(3,-2)--cycle;
			\fill[red!30] (A)--(B)--(C)--(D)--cycle;
			% Trục toạ độ
			\draw[->] (xmin,0)--(xmax,0) node[shift={(-100:7pt)},font=\normalsize]{$x$};
			\draw[->] (0,ymin)--(0,ymax) node[shift={(-30:7pt)},font=\normalsize]{$y$};
			\fill (0,0) circle (1pt) node[shift={(-45:9pt)},font=\normalsize]{$O$};
			
			% Vẽ đồ thị hàm số
			\clip (xmin,ymin) rectangle (xmax,ymax);
			\draw[dashed, smooth, samples=100, line width=.5] plot[domain=xmin:tcd-0.15] (\x, {f(\x)});
			\draw[dashed, smooth, samples=100, line width=.5] plot[domain=tcd+0.15:xmax] (\x, {f(\x)});
			\draw[line width=.5] plot[domain=xmin:xmax] (\x, {-(\x)+1});
			% Vẽ đường tiệm cận
			%\draw[thin] (tcd,ymin)--(tcd,ymax) (xmin,tcn)--(xmax,tcn);
			\draw (D)--(A)--(B)--(C);
			\foreach \x/\goc in {A/150,B/-60,C/0,D/90}{
			\draw[fill] (\x) circle (1pt) node[shift={(\goc:7pt)},font=\small]{$\x$};}
		\end{tikzpicture}
	\end{center}
	Tính diện tích $S$ của mặt sàn nổi (làm tròn kết quả đến hàng phần chục của mét vuông), biết hình vuông có hai đỉnh nằm trên $(C)$, hai đỉnh còn lại nằm trên $d$.
	\shortans{10,9}
	\loigiai{
	\begin{center}
		\begin{tikzpicture}[line cap=butt,line join=miter,>=stealth,>=stealth,
			declare function={
			a=1; b=1; c=1; d=-1;k=4.5;
			f(\x)=(a*(\x) + b)/(c*(\x) + d);
			tcd=-d/c; tcn=a/c;
			xmin=tcd-3; xmax=tcd+k; ymin=tcn-3; ymax=tcn+k;
			} ]
			\path ({5/3},4)coordinate (A) (4,{f(4)})coordinate (B)
			(-1,2)coordinate (M) (2,-1)coordinate (N)
			($(M)!(B)!(N)$)coordinate (C)($(M)!(A)!(N)$)coordinate (D) (1,1)coordinate (I);
			\fill[green!50] (1.44,5.5)--plot[domain=1.44:xmax] (\x, {f(\x)})--(xmax,ymax)--cycle;
			\fill[green!50] (-2,3)--(-2,-2)--(3,-2)--cycle;
			\fill[red!30] (A)--(B)--(C)--(D)--cycle;
			% Trục toạ độ
			\draw[->] (xmin,0)--(xmax,0) node[shift={(-100:7pt)},font=\normalsize]{$x$};
			\draw[->] (0,ymin)--(0,ymax) node[shift={(-30:7pt)},font=\normalsize]{$y$};
			\fill (0,0) circle (1pt) node[shift={(-45:9pt)},font=\normalsize]{$O$};
			
			% Vẽ đồ thị hàm số
			\clip (xmin,ymin) rectangle (xmax,ymax);
			\draw[dashed, smooth, samples=100, line width=.5] plot[domain=xmin:tcd-0.15] (\x, {f(\x)});
			\draw[dashed, smooth, samples=100, line width=.5] plot[domain=tcd+0.15:xmax] (\x, {f(\x)});
			\draw[line width=.5] plot[domain=xmin:xmax] (\x, {-(\x)+1});
			\draw[line width=.5,red] plot[domain=xmin:xmax] (\x, {\x}) (5,5)node[above]{$\Delta$};
			% Vẽ đường tiệm cận
			\draw[thin] (tcd,ymin)--(tcd,ymax) (xmin,tcn)--(xmax,tcn);
			\draw (D)--(A)--(B)--(C);
			\foreach \x/\goc in {A/150,B/-60,C/0,D/90,I/-45}{
			\draw[fill] (\x) circle (1pt) node[shift={(\goc:7pt)},font=\small]{$\x$};}
		\end{tikzpicture}
	\end{center}
	Đồ thị $(C)$ có đường tiệm cận đứng là $x=1$, đường tiệm cận ngang là $y=1$, tâm đối xứng $I(1;1)$.\\
	Gọi $\Delta$ là đường thẳng qua $I$ và vuông góc với $d$ nên có phương trình $y=x$.\\
	Dễ thấy $\Delta$ là trục đối xứng của hình vuông $ABCD$. Gọi $A\left(x_0;\dfrac{x_0+1}{x_0-1}\right)$ với $1<x_0<x_B$. Khi đó
	\allowdisplaybreaks
	\begin{eqnarray*}
		&&\mathrm{d}(A,d)=2\mathrm{d}(A,\Delta)\\
		&\Rightarrow&\dfrac{\left|x_0+\dfrac{x_0+1}{x_0-1}-1\right|}{\sqrt{1+1}}=2\cdot \dfrac{\left|x_0-\dfrac{x_0+1}{x_0-1}\right|}{\sqrt{1+1}}\\
		&\Rightarrow&\big|x_0^2-x_0+2\big|=2\big|x_0^2-2x_0-1\big|\\
		&\Rightarrow&\hoac{&x_0^2-x_0+2=2(x_0^2-2x_0-1)\\&x_0^2-x_0+2=-2(x_0^2-2x_0-1)}\\
		&\Rightarrow&\hoac{&x_0^2-3x_0-4=0\\&3x_0^2-5x_0=0}\\
		&\Rightarrow&\hoac{&x_0=4=x_B&\text{ (nhận)}\\&x_0=-1&\text{ (loại)}\\&x_0=0&\text{ (loại)}\\&x_0=\dfrac{5}{3}=x_A&\text{ (nhận)}.}
	\end{eqnarray*}
	Suy ra $A\left(\dfrac{5}{3}; 4\right)\Rightarrow \mathrm{d}(A,d)= a=\dfrac{7\sqrt{2}}{3}$.\\
	Vậy diện tích hình vuông $ABCD$ là $a^2=\dfrac{98}{9}\approx 10{,}9$.
	}
\end{ex}

%

\begin{ex}%[Nguồn: Bộ đề minh họa Moon 2024-2025]%[2D1C5-8]
	Cho hình chữ nhật $MNPQ$ có hai cạnh là $x$ và $y$, nội tiếp tam giác $ABC$ vuông tại $C$. Biết tam giác $ABC$ có độ dài các cạnh góc vuông là $CB=6$ và $CA=8$ như hình vẽ dưới đây:
	\begin{center}
		\begin{tikzpicture}[join = round, cap = round, thick, font = \small, scale = 1]
			\path
			(0,0) coordinate (C)
			+(0:3) coordinate (B)
			+(90:4) coordinate (A);
			\path (barycentric cs:A=2,C=3) coordinate (P);
			\path (barycentric cs:C=3,B=2) coordinate (Q);
			\path ($(B)!(P)!(A)$) coordinate (N);
			\path ($(A)!(Q)!(B)$) coordinate (M);
			\draw[fill=gray] (P)--(N)--(M)--(Q)--cycle;
			\draw (P)--(N) node[above, midway]{$x$};
			\draw (P)--(Q) node[below, midway]{$y$};
			\foreach \x/\y/\z in {A/C/B,P/Q/M,A/N/P,Q/M/B}{
			\draw[thin] pic[red,draw, angle radius = 6pt]{right angle = \x--\y--\z};
			}
			\draw (N)--(A)--(C)--(B)--(M);
			\foreach \x/\g in {C/180,B/0,A/90,P/180,N/45,M/45,Q/-90}
			\fill [blue] (\x) circle (1.5pt)
			+(\g:3mm) node{$\x$};
		\end{tikzpicture}
	\end{center}
	\choiceTF
	{\True Độ dài cạnh huyền của tam giác $ABC$ là $AB=10$}
	{$\sin A=\dfrac{x}{AN}$ và $\sin B=\dfrac{x}{BQ}$}
	{\True Diện tích hình chữ nhật $MNPQ$ được xác định bởi hàm số $S(x)=10x-\dfrac{25}{12} x^2$ ($0<x<4{,}8$)}
	{\True Khi diện tích hình chữ nhật $MNPQ$ đạt giá trị lớn nhất thì diện tích tam giác $ANP$ là $\dfrac{54}{25}$}
	\loigiai{
	\begin{enumerate}[a)]
		\item $AB=\sqrt{AC^2+BC^2}=\sqrt{8^2+6^2}=10$.
		\item $\sin A=\dfrac{x}{AP}$ và $\sin B=\dfrac{x}{BQ}$ ($AP>AN$).
		\item Độ dài đường cao $CH$ là
		\[\dfrac{1}{CH^2}=\dfrac{1}{AC^2}+\dfrac{1}{BC^2}=\dfrac{1}{8^2}+\dfrac{1}{6^2}=\dfrac{25}{576}\Rightarrow CH=\dfrac{24}{5}=4{,}8\]
		Diện tích hình chữ nhật $MNPQ$ là
		\begin{eqnarray*}
			S(x)&=&xy=x (10-AN-MB)=x(10-x\tan A-x\tan B)\\
			&=&x(10-\dfrac{3}{4}x-\dfrac{4}{3}x)=10x-\dfrac{25}{12} x^2\ (0<x<4{,}8).
		\end{eqnarray*}
		\item $S'(x)=10-\dfrac{25}{6}x=0\Leftrightarrow x=\dfrac{12}{5}$.\\
		\begin{center}
			\begin{tikzpicture}
				\tkzTabInit[lgt=1.2,espcl=4]
				{$x$/1,$f’(x)$/1,$f(x)$/2}
				{$0$,$\dfrac{12}{5}$,$\dfrac{24}{5}$}
				\tkzTabLine{ ,+,0,-, }
				\tkzTabVar{-/$ $,+/$ $,-/$ $}
			\end{tikzpicture}
		\end{center}
		$S(x)$ đạt giá trị lớn nhất khi $x=\dfrac{12}{5}$. Khi đó
		\[S_{ANP}=\dfrac{1}{2} AN\cdot NP=\dfrac{1}{2}\cdot x\tan A\cdot x=\dfrac{1}{2}\cdot \dfrac{3}{4}\cdot \left(\dfrac{12}{5}\right)^2=\dfrac{54}{25}\quad\text{(đvdt)}.\]
	\end{enumerate}}
\end{ex}

\begin{ex}%[Nguồn: Bộ đề minh họa Moon 2024-2025]%[2D1C5-8]
	Hình vẽ sau mô tả một con thuyền đang kéo một người đàn ông trượt ván bằng một đoạn dây dài 9 mét. Xét trên hệ trục $Oxy$ (đơn vị trên các hệ trục bằng mét), ban đầu con thuyền đang ở gốc tọa độ và di chuyền trên tia $Oy$, người đàn ông xuất phát từ điểm có tọa độ $(9; 0)$ bị kéo theo và quãng đường di chuyền tạo thành một đường cong $y=f(x)$ (tham khảo hình vẽ bên), bờ biển là đường thẳng $x+2y+1=0$. Khi người đàn ông đến gần bờ biển nhất thì khoảng cách giữa người đàn ông và trục $Oy$ bằng bao nhiêu mét (làm tròn đến hàng phần trăm)? Biết rằng trong quá trình di chuyền, người đàn ông luôn hướng về phía thuyền, đoạn dây luôn căng và nằm trên tiếp tuyến của đường cong $y=f(x)$.
	\begin{center}
		\includegraphics[scale=0.5]{cau-6-TLN-De-5}
	\end{center}
	
	\shortans{$8{,}05$}
	\loigiai{
	\begin{center}
		\includegraphics[scale=0.5]{cau-6-TLN-De-5-loigiai}
	\end{center}
	Giả sử người đó ở vị trí $M$, $M_0$ là điểm mà tiếp tuyến tại $M_0$ song song với đường thẳng\break $d\colon x+2y+1=0$.\\
	Dựng $M_0N\perp d$, $MK\perp d$.\\
	Dựa vào hình vẽ , ta có $MK\geq M_0N $ nên khoảng cách từ người đến bờ biển ngắn nhất khi tiếp tuyến tại $M$ song song với đường thẳng $x+2y+1=0$.\\
	Dựa vào hình, ta có
	\begin{itemize}
		\item $\widehat{PM_0Q}=\widehat{PFQ}$ (hai góc đồng vị).
		\item $\widehat{PFO}=\widehat{FED}$ (hai góc so le trong).
	\end{itemize}
	Suy ra $\widehat{PM_0Q}=\widehat{FED}$.\\
	Thay lần lượt $x=0$, $y=0$ vào đường thẳng $x+2y+1=0$, ta có \[E(-1;0), D\left(0;-\dfrac{1}{2}\right)\Rightarrow OE=1, OD=\dfrac{1}{2}.\]
	Xét tam giác vuông $OED$, ta có $\tan \widehat{OED}=\dfrac{OD}{OE}=\dfrac{1}{2}\Rightarrow \widehat{OED}\approx 26{,}565^\circ$ suy ra $\widehat{PM_0Q}\approx 26{,}565^\circ$.\\
	Xét tam giác vuông $PM_0Q$, ta có $\cos  \widehat{PM_0Q}=\dfrac{QM_0}{PM_0}\\
	\Rightarrow QM_0=PM_0\cdot \cos  \widehat{PM_0Q}=9\cdot \cos 26{,}565^\circ\approx 8{,}05$ m.
	}
\end{ex}

\begin{ex}%[Nguồn: Bộ đề minh họa Moon 2024-2025]%[2D1C5-8]
	\immini{Cấu trúc tổ ong là một cấu trúc đặc biệt, mỗi lỗ ong là một lăng kính hình lục giác, một đầu hở còn một đầu tạo thành một góc tam diện. Ong đã xây các lỗ này với một cách làm tối ưu về diện tích bề mặt (đã sử dụng lượng sáp ong ít nhất để xây tổ). Người ta đã quan sát, nghiên cứu thì thấy rằng góc $\theta(\mathrm{rad})$ ở đỉnh nhất quán một cách đáng kinh ngạc, dựa trên cấu trúc hình học của lỗ ong người ta đã chứng minh được diện tích bề mặt $S$ của lỗ ong là $S=6 s \cdot h-\dfrac{3}{2} s^2 \cdot \cot \theta+\dfrac{3 \sqrt{3}}{2} s^2 \cdot \dfrac{1}{\sin \theta}$ ( $s$ là chiều dài các cạnh của lỗ ong, $h$ là chiều cao, $s$ và $h$ đều là hằng số). Vậy để tối thiểu hoá diện tích bề mặt con ong đã xây một góc $\theta$ bằng bao nhiêu? (làm tròn đến hàng phần trăm).}{
	\begin{tikzpicture}[scale=1, line cap=round, line join=round, >=stealth,font=\footnotesize]
		\def\a{2} % Độ dài cạnh lục giác
		\def\theta{-10} % Góc nghiêng lục giác
		\def\scaleY{0.4} % Tỷ lệ nén theo trục y
		\def\h{5} % chiều cao
		% Định nghĩa các đỉnh của hình lục giác đã nghiêng và nén dọc theo trục y
		\foreach \i in {0,60,120,180,240,300} {
		\path ({\a*cos(\i + \theta)}, {\scaleY*\a*sin(\i + \theta)}) coordinate (P\i);
		}
		% Vẽ lục giác
		\draw  (P180)--(P240) -- (P300)--(P0) ;
		% Vẽ các đường chéo để nhấn mạnh hình dạng
		\draw[dashed] (P0) -- (P60) -- (P120) -- (P180) (P0) -- (P180) (P60) -- (P240)(P120) -- (P300);
		\coordinate (P0') at ($(P0) + (0,\h-0.2)$) ;
		\coordinate (P60') at ($(P60) + (0,\h+0.15)$);
		\coordinate (P120') at ($(P120) + (0,\h-1)$);
		\coordinate (P180') at ($(P180) + (0,\h)$);
		\coordinate (P240') at ($(P240) + (0,\h-0.2)$);
		\coordinate (P300') at ($(P300) + (0,\h+0.5)$);
		
		\coordinate (O) at ($(P0)!0.5!(P180)$);
		\coordinate (O') at ($(O) + (0,\h+1)$);
		\draw[dashed] (P60) -- (P60')  (P120) -- (P120') (O) -- (O') ;
		\draw  (P0)--(P0')  (P180)--(P180') (P240)--(P240') (P300)--(P300') (P0')--(P300')--(P240')--(P180')--(O') (O')--(P180') (O')--(P300') (P0')--(P60')--(O') ;
		\draw[dashed]  (P60')--(P120')--(P180') ;
		
		\draw ($(O)!0.4!(O')$) node[right]{$h$};
		\draw ($(P300)!0.5!(P240)$) node[below]{$s$};
	\end{tikzpicture}
	}
	\shortans[]{$0{,}96$}
	\loigiai{
	Ta có
	\begin{eqnarray*}
		S^{\prime}(\theta)&=&\dfrac{3}{2} s^2 \cdot \dfrac{1}{\sin ^2 \theta}-3 s^2 \dfrac{\sqrt{3}}{2} \cdot \dfrac{\cos \theta}{\sin ^2 \theta} \\
		& =&\dfrac{3}{2} s^2 \cdot \dfrac{1}{\sin ^2 \theta}(1-\sqrt{3} \cos \theta)=0 \\
		&\Leftrightarrow& \cos \theta=\dfrac{1}{\sqrt{3}} \Rightarrow \theta \approx 0{,}96 .
	\end{eqnarray*}
	Bảng biến thiên
	\begin{center}
		\begin{tikzpicture}[scale=1, font=\footnotesize, line join=round, line cap=round, >=stealth]
			\tkzTabInit[lgt=1]	{$x$/1,$y'$/1,$y$/2}
			{$0$,$0{,}96$,$2\pi$}
			\tkzTabLine{,-,0,+,}
			\tkzTabVar{+/ $+\infty$,-/ $6 s\left(h+\dfrac{1}{2 \sqrt{2}} s\right)$, +/ $+\infty$}
		\end{tikzpicture}
	\end{center}
	Vậy $S_{\min }=6 s\left(h+\dfrac{1}{2 \sqrt{2}} s\right)$ khi
	$\theta=0{,}96$
	}
\end{ex}

\begin{ex}%[Nguồn: Bộ đề minh họa Moon 2024-2025]%[2D1C5-8]
	Một thành phố nằm bên một con sông chảy qua hẻm núi. Hẻm có chiều ngang $80$ m, một bên cao $40$ m và một bên cao $30$ m. Một cây cầu sẽ được xây dựng bắc qua sông và hẻm núi. Sơ đồ thiết kế của cây cầu được gắn hệ trục toạ độ như hình vẽ dưới đây.
	\begin{center}
		\begin{tikzpicture}[>=stealth,line join=round,line cap=round,font=\footnotesize,scale=1]
			%bờ bên trái
			\fill[gray!40,rotate=90,shift={(3,-2.75)}] plot[samples=100,smooth,domain=-3:0] (\x,{(1/6)*sin((\x*180)/2*pi)-0.2}) coordinate(Y) -- (0,{(1/6)*sin(90/pi)-0.2}) -- (0,-1.5) -- (-3,-1.5) -- cycle;
			%bờ bên phải
			\fill[gray!40,rotate=-90,shift={(-1,-2.75)}] plot[samples=100,smooth,domain=1:-3.5] (\x,{(1/6)*sin((\x*180)/2*pi)-0.2}) coordinate (X) -- (-3.5,{(1/6)*sin(-360/pi)-0.2}) -- (-3.5,-1.5) -- (1,-1.5) -- cycle;
			\draw[->]
			(-5,0) -- (5,0)node[below]{$x$};
			\draw[->]
			(0,-1) -- (0,7)node[right]{$h(x)$};
			
			\draw[samples=100,smooth,domain=-3:3,name path=dt] plot(\x,{-(2/3)*(\x)^2 + 6});
			
			\draw[name path=duongcong] (X) .. controls +(0:2.5) and +(180:2.5) .. (Y)
			;
			\path
			(0.5,{35/6}) coordinate (N)
			(-0.5,{35/6}) coordinate (Q)
			(0.5,0) coordinate (Mt)
			(-0.5,0) coordinate (Pt)
			(-3,0) coordinate (A) node[below]{$-40$}
			(3,0) coordinate (B) node[below]{$40$}
			;
			\path[name path=duongm] (N) -- (Mt);
			\path[name path=duongp] (Q) -- (Pt);
			\path[name intersections={of=dt and duongcong}]
			(intersection-1) coordinate (E)
			(intersection-2) coordinate (F)
			;
			\path[name intersections={of=duongm and duongcong}]
			(intersection-1) coordinate (M)
			;
			\path[name intersections={of=duongp and duongcong}]
			(intersection-1) coordinate (P)
			;
			%vẽ sóng
			\draw[samples=100,smooth,domain=-5:5] plot(\x,{(1/6)*sin((\x*180)/2*pi)-0.2});
			\draw[samples=100,smooth,domain=-5:5] plot(\x,{(1/6)*cos((\x*180)/2*pi)-0.5});
			\draw (Q) -- (P) (N) -- (M);
			\fill
			(A) circle(1.5pt) node[above left]{$A$}
			(B) circle(1.5pt) node[above right]{$B$}
			(X) circle(1.5pt) node[above left]{$X\left(-40,40 \right)$}
			(Y) circle(1.5pt) node[above right]{$Y\left(40,30 \right)$}
			(E) circle(1.5pt) node[above left]{$E$}
			(F) circle(1.5pt) node[above right]{$F$}
			(N) circle(1.5pt) node[right]{$N$}
			(M) circle(1.5pt) node[below]{$M$}
			(Q) circle(1.5pt) node[left]{$Q$}
			(P) circle(1.5pt) node[below]{$P$}
			(0,0) circle(1.5pt) node[above right]{$O$}
			(0,6) circle(1.5pt) node[above right]{$60$}
			;
		\end{tikzpicture}
	\end{center}
	Con đường $XY$ xuyên qua hẻm núi được mô hình hoá bằng phương trình $y = \dfrac{x^3}{25600} - \dfrac{3x}{16} + 35$. Hai cột đỡ dọc $MN$ và $PQ$ (song song với trục $Oy$) là đoạn nối khung của Parabol và đường $XY$. Tính tổng độ dài đoạn $MN$ và $PQ$ biết rằng $N$, $Q$ là hai điểm đối xứng qua $Oy$ và $MN$ là đoạn có độ dài lớn nhất (làm tròn kết quả đến hàng phần chục).
	\shortans{$49{,}5$}
	\loigiai{
	Từ hình vẽ ta thấy Parabol là đồ thị của hàm số $y = ax^2 + 60 \, \left(a < 0 \right)$. \\
	Mà Parabol đi qua hai điểm $A\left(-40,0 \right)$ và $B\left(40,0 \right)$ nên ta tìm được $a = \dfrac{-3}{80}$.\\
	Suy ra Parabol trong hình là đồ thị của hàm số $y = \dfrac{-3}{80}x^2 + 60$. \\
	Xét phương trình hoành độ giao điểm của đường cong $XY$ và Parabol ta có
	\begin{align*}
		\dfrac{x^3}{25600} - \dfrac{3x}{16} + 35 = \dfrac{-3}{80}x^2 + 60 \Leftrightarrow \hoac{&x \approx -23{,}7\\&x \approx 28{,}0.}
	\end{align*}
	Gọi $N \left(t; \dfrac{-3}{80}t^2 + 60 \right)$ là một điểm thuộc Parabol trên đoạn $\left[-23{,}7; 28 \right]$.\\
	Giả sử $M$ là một điểm thuộc đường cong $XY$ có hoành độ $x = t$ khi đó $M \left(t; \dfrac{t^3}{25600} - \dfrac{3t}{16} + 35\right)$.\\
	Khi đó ta có
	\begin{align*}
		MN = -\dfrac{t^3}{25600} - \dfrac{3}{80}t^2 + \dfrac{3}{16}t + 25.
	\end{align*}
	Xét hàm số $S(t) = -\dfrac{t^3}{25600} - \dfrac{3}{80}t^2 + \dfrac{3}{16}t + 25$.\\
	Khảo sát hàm số trên ta có $ \underset{\left[-23{,}7; 28 \right]}{\max S(t)} \approx 25{,}2$ khi $x \approx 2{,}5$.\\
	Từ đó ta có $PQ \approx 24{,}3$ m.\\
	Suy ra $MN + PQ \approx 49{,}5$ m.
	}
\end{ex}

%

\begin{ex}%[Nguồn: Bộ đề minh họa Moon 2024-2025]%[2D1C5-8]
	Một hòn đảo nằm trong một hồ nước. Biết rằng đường cong tạo nên hòn đảo được mô hình hóa vào hệ trục tọa độ $Oxy$ là một phần của đồ thị hàm số bậc ba $f(x)$ (tham khảo hình vẽ).
	\begin{center}
		\begin{tikzpicture}[>=stealth,x=1cm,y=1cm,scale=1]
			\draw[->] (-1,0) -- (5,0) node[below] {$x$};
			\draw[->] (0,-1) -- (0,5) node[left] {$y$};
			\filldraw (0,0) circle (1pt)node[below left]{$O$};
			\def\f[#1]{-(#1)^3+3*(#1)^2};
			\def\g[#1]{16-4*(#1)};
			\draw[domain=0:3.1,samples=300] plot (\x,{\f[\x]});
			\draw[domain=2.8:4.2,samples=300] plot (\x,{\g[\x]});
			\path (4.2,4.5) node {$y=16-4x$};
			\path (1,3.5) node {$f(x)$};
			\path (4.8,2) node {Mặt đường};
			\path (1.8,1.5) node {Hòn đảo};
		\end{tikzpicture}
	\end{center}
	Vị trí điểm cực đại là $(2 ; 4)$ với đơn vị của hệ trục là $100 \mathrm{~m}$ và vị trí điểm cực tiểu là gốc tọa độ $O$. Mặt đường chạy trên một đường thẳng có phương trình $y=16-4 x$. Người ta muốn làm một cây cầu có dạng một đoạn thẳng nối từ hòn đảo ra mặt đường. Độ dài ngắn nhất của cây cầu bằng bao nhiêu mét? (làm tròn kết quả đến hàng phần mười).
	\loigiai{
	Đường cong $(C)$ có dạng $f(x)=ax^3+bx^2+cx+d$ ($a\ne 0$) $\Rightarrow f'(x)=3ax^2+2bx+c$. Khi đó
	\[\heva{&f(0)=0\\&f(2)=4\\&f'(0)=0\\&f'(2)=0}\Leftrightarrow \heva{&d=0\\&8a+4b+2c+d=4\\&c=0\\&12a+4b+c=0}\Leftrightarrow \heva{&a=-1\\&b=3\\&c=d=0.}\]
	Vậy $f(x)=-x^3+3x^2$. Gọi $A\in (C)$ ta có $A(m;-m^3+3m^2)$.\\
	Khoảng cách từ $A$ đến $d\colon 4x+y-16=0$ là \[\mathrm{d}(A,d)=\dfrac{|4m-m^3+3m^2-16|}{\sqrt{4^2+1}}=\dfrac{|-m^3+3m^2+4m-16|}{\sqrt{17}}.\]
	$f(x)=0\Leftrightarrow -x^3+3x^2=0\Leftrightarrow \hoac{&x=0\\&x=3.}$\\
	Đặt $g(x)=-x^3+3x^2+4x-16$ $(x\in [0;3])$.\\
	Suy ra $g'(x)=-3x^2+6x+4=0\Leftrightarrow \hoac{&x=\dfrac{3-\sqrt{21}}{3}\text{ (loại)}\\&x=\dfrac{3+\sqrt{21}}{3}.}$
	\begin{center}
		\begin{tikzpicture}
			\tkzTabInit[lgt=1.2,espcl=4]
			{$x$/1,$g’(x)$/1,$g(x)$/2}
			{$0$,$\dfrac{3+\sqrt{21}}{3}$,$3$}
			\tkzTabLine{ ,+,0,-, }
			\tkzTabVar{-/$-16$,+/$\dfrac{-90+14\sqrt{21}}{9}$,-/$-4$}
		\end{tikzpicture}
	\end{center}
	Suy ra $\min\limits_{x \in [0;3]} |g(x)|=\left|g\left(\dfrac{3+\sqrt{21}}{3}\right)\right|=\dfrac{90-14\sqrt{21}}{9}\Rightarrow\mathrm{d}(A,d)=\dfrac{\dfrac{90-14\sqrt{21}}{9}}{\sqrt{17}}=0{,}696$.\\
	Vậy độ dài ngắn nhất của cây cầu là $69{,}6$ mét.
	}
\end{ex}

\begin{ex}%[Nguồn: Bộ đề minh họa Moon 2024-2025]%[2D1H1-1]
	Cho hàm số $f(x) = \left(x^2-3x-3\right)\mathrm{e}^x$.
	\choiceTF
	{\True Hàm số đã cho xác định với mọi $x \in \mathbb{R}$}
	{Đạo hàm của $f(x)$ là $f'(x) = \left(x^2 + x - 6\right)\mathrm{e}^x$}
	{\True Phương trình $f'(x) = 0$ có hai nghiệm thực phân biệt}
	{\True Hàm số $f(x)$ nghịch biến trên khoảng $(-2; 3)$}
	\loigiai{
	\begin{itemchoice}
		\itemch \textbf{Đúng}.\\
		Hàm số $f(x) = \left(x^2-3x-3\right)e^x$ có tập xác định $\mathbb{R}$.
		\itemch \textbf{Sai}.\\
		Đạo hàm của $f(x)$ là $f'(x) = \left(x^2-x-6\right)\mathrm{e}^x$.
		\itemch \textbf{Đúng}.\\
		Phương trình $f'(x) = 0\Leftrightarrow \hoac{&x=-2\\&x=3.}$
		\itemch \textbf{Đúng}.\\
		Ta có bảng biến thiên của hàm số $f(x)$ như sau
		\begin{center}
			\begin{tikzpicture}
				\tkzTabInit[espcl=2.5,lgt=1.5,nocadre]
				{$x$/0.7,$y'$/0.7,$y$/2.1}
				{$-\infty$,$-2$,$3$,$+\infty$}
				\tkzTabLine{,+,0,-,0,+,}
				\tkzTabVar{-/$-\infty$,+/$f(-2)$,-/$f(3)$,+/$+\infty$}
			\end{tikzpicture}
		\end{center}
	\end{itemchoice}
	}
\end{ex}

\begin{ex}%[Nguồn: Bộ đề minh họa Moon 2024-2025]%[2D1H1-1]
	Hàm số $y=x^4-2x^2+5$ đồng biến trên khoảng nào sau đây?
	\choice
	{$(-\infty;-1)$}
	{$(0;1)$}
	{\True $(-1;0)$}
	{$(0;+\infty)$}
	\loigiai
	{
	Hàm số trên có bảng biến thiên
	\begin{center}
		\begin{tikzpicture}[scale=1, font=\footnotesize, line width=1pt]%<DTools>
			\tkzTabInit[nocadre=true, lgt=1.2, espcl=2, deltacl=0.6]
			{$x$/0.8,$f'(x)$/0.6,$f(x)$/2}
			{$-\infty$,$-1$,$0$,$1$,$+\infty$};
			\tkzTabLine{,-,$0$,+,$0$,-,$0$,+,};
			\tkzTabVar{+/$+\infty$,-/$4$,+/$5$,-/$4$,+/$+\infty$};
		\end{tikzpicture}
	\end{center}
	Từ bảng biến thiên ta thấy hàm số đồng biến trên $(-1;0)$ và $(1;+\infty)$.
	}
\end{ex}

%

\begin{ex}%[Nguồn: Bộ đề minh họa Moon 2024-2025]%[2D1H1-2]
	Cho hàm số có đồ thị như hình vẽ bên dưới. Hàm số đã cho đồng biến trên khoảng nào sau đây?
	\begin{center}
		\begin{tikzpicture}[line cap=butt,line join=miter,>=stealth,xscale=0.83,yscale=0.83]
			\tikzset{declare function={xmin=-3;xmax=3;
			ymin=-3;ymax=3;
			a=-1; b=0; c=3; d=0;
			xm=-1; ym=-2;
			xh=1; yh=2;
			f(\x)=a*(\x)^3+b*(\x)^2+c*(\x)+d;
			},
			smooth,samples=100
			}
			\draw[->] (xmin-0.25,0)--(xmax+0.5,0)
			node[shift={(-100:7pt)},font=\normalsize]{$ x $};
			\draw[->] (0,ymin-0.25)--(0,ymax+0.5)
			node[shift={(-5:7pt)},font=\normalsize]{$ y $};
			\fill (0,0) node[shift={(-45:9pt)},font=\normalsize]{$ O $};
			\draw[dashed,cyan!75!blue] (xm,0) |- (0,ym) (xh,0) |- (0,yh);
			\foreach \x in {-3, -2, -1, 1, 2, 3}{
			\draw (\x,2pt)--(\x,-2pt) +(0,-9pt) node[font=\scriptsize,fill=white,inner sep=1pt]{$\x$};
			}
			\foreach \y in {-3, -2, -1, 1, 2, 3}{
			\draw (2pt,\y)--(-2pt,\y) +(-3pt,0) node[font=\scriptsize,anchor=east,fill=white,inner sep=1pt]{$\y$};
			}
			\begin{scope}
				\clip (xmin,ymin) rectangle (xmax,ymax);
				\draw[blue,thick] plot[domain=xmin:xmax] (\x, {f(\x)});
			\end{scope}
		\end{tikzpicture}
	\end{center}
	\choice
	{$(-\infty;-1)$}
	{$(-\infty; 1)$}
	{\True $(-1; 1)$}
	{$(1;+\infty)$}
	\loigiai{
	Theo đồ thị, ta có hàm số đồng biến trên $(-1 ; 1)$.
	}
\end{ex}

\begin{ex}%[Nguồn: Bộ đề minh họa Moon 2024-2025]%[2D1H1-2]
	Cho hàm số $f(x)$ xác định trên $\mathbb{R}$, có bảng xét dấu đạo hàm như sau
	\begin{center}
		\begin{tikzpicture}
			\tkzTabInit[deltacl=0.5,espcl=2.5,lgt=2,nocadre]
			{$x$/1,$f'(x)$/1}
			{$-\infty$,$-1$,$6$,$+\infty$}
			\tkzTabLine{,+,0,-,0,+,}
		\end{tikzpicture}
	\end{center}
	Khẳng định nào sau đây đúng?
	\choice
	{\True $f(-1) > f(3)$}
	{$f(6) > f(8)$}
	{$f(-3) > f(-1)$}
	{$f(5) < f(6)$}
	\loigiai{
	Bảng xét dấu đạo hàm cho thấy\\
	Hàm số nghịch biến trên khoảng $(-1;6)$ và $-1<3$ do đó $f(-1)>f(3)$.
	}
\end{ex}

\begin{ex}%[Nguồn: Bộ đề minh họa Moon 2024-2025]%[2D1H2-1]
	Cho hàm số $f(x)=\dfrac{2x-3}{x^2+4}$.
	\choiceTF
	{\True Hàm số đã cho xác định với mọi $x\in\mathbb{R}$}
	{\True Nghiệm của phương trình $f'(x)=0$ là $x=-1$; $x=4$}
	{\True Giá trị cực đại của hàm số $f(x)$ là $\dfrac{1}{4}$}
	{Tập giá trị của hàm số $f(x)$ là đoạn $[a;b]$ thì $3a+2b=-2$}
	\loigiai{
	\begin{itemchoice}
		\itemch Ta có $x^2+4>0$ với mọi giá trị $x$.\\
		Hàm số đã cho xác định với mọi $x\in\mathbb{R}$.
		\itemch Ta có $f'(x)=\dfrac{-2x^2+6x+8}{(x^2+4)^2}$.\\
		Xét $f'(x)=0\Leftrightarrow -2x^2+6x+8=0\Leftrightarrow\hoac{&x=-1\\&x=4.}$
		\itemch Bảng biến thiên của $f(x)$
		\begin{center}
			\begin{tikzpicture}
				\tkzTabInit[nocadre=false,lgt=1.5,espcl=2.5,deltacl=0.6]
				{$x$/0.7,$f'(x)$/0.7,$f(x)$/1.5}{$-\infty$,$-1$,$4$,$+\infty$}
				\tkzTabLine{,-,0,+,0,-,}
				\tkzTabVar{+/$0$,-/$-1$,+/$\dfrac{1}{4}$,-/$0$}
			\end{tikzpicture}
		\end{center}
		\itemch Dựa vào bảng biến thiên ta có tập giá trị của hàm số là $\left[-1;\dfrac{1}{4} \right]$.\\
		Vậy $3a+2b=3\cdot(-1)+2\cdot\dfrac{1}{4}=\dfrac{5}{2}$.
		
	\end{itemchoice}
	}
\end{ex}

\begin{ex}%[Nguồn: Bộ đề minh họa Moon 2024-2025]%[2D1H2-1]
	Hàm số $f(x)=x^4(x-1)$ có bao nhiêu điểm cực trị?
	\choice
	{$3$}
	{$0$}
	{$5$}
	{\True $2$}
	\loigiai{    Ta có
	$$\begin{aligned}
		f^{\prime}(x)&=4 x^3(x-1)+x^4  =x^3[4(x-1)+x]=0 \\
		& \Leftrightarrow\heva{
		&x^3=0 \\
		&4(x-1)+x=0.
		} \Leftrightarrow\heva{
		&x=0 \\
		&x=\dfrac{4}{5}.
		}
	\end{aligned}
	$$
	Phương trình $f'(x)=0$ có hai nghiệm bội lẻ phân biệt.\\ Vậy hàm số $f(x)=x^4(x-1)$ có hai điểm cực trị}
\end{ex}

%

\begin{ex}%[Nguồn: Bộ đề minh họa Moon 2024-2025]%[2D1H2-1]
	Hàm số nào sau đây \textbf{không} có cực trị?
	\choice
	{$y=\dfrac{x^2-3}{x-2}$}
	{\True $y=\dfrac{2x-1}{x+2}$}
	{$y=x^3-2x-1$}
	{$y=-x^4-x^2+1$}
	\loigiai{
	Hàm số $y=\dfrac{2x-1}{x+2}\Rightarrow y'=\dfrac{5}{(x+2)^2}>0,\ \forall x\ne -2$ nên không có cực trị.}
\end{ex}

%\subsection{Câu trắc nghiệm nhiều phương án lựa chọn}
\cauds

\begin{ex}%[Nguồn: Bộ đề minh họa Moon 2024-2025]%[2D1H2-1]
	Cho hàm số $f(x)=\sqrt{x+2}+2\sqrt{4-x}$.
	\choiceTF
	{\True Tập xác định của hàm số đã cho là $D=[-2;4]$}
	{\True Đạo hàm của hàm số đã cho là $f'(x)=\dfrac{1}{2\sqrt{x+2}}-\dfrac{1}{\sqrt{4-x}}$}
	{\True Nghiệm của phương trình $f'(x)=0$ trên đoạn $[-2;4]$ là $-\dfrac{4}{5}$}
	{Tập giá trị của hàm số $f(x)$ chứa đúng $4$ số nguyên}
	\loigiai{
	\begin{itemchoice}
		\itemch Điều kiện xác định của hàm số là
		$x+2 \ge 0$ và $4-x \ge 0$ $\Leftrightarrow -2\le x\le 4$.\\
		Vậy tập xác định là $D=[-2;4]$.
		\itemch  Đạo hàm của hàm số là
		\begin{align*}
			f^{\prime}(x) &= \frac{1}{2\sqrt{x+2}} + 2\cdot\frac{-1}{2\sqrt{4-x}} \\
			&= \frac{1}{2\sqrt{x+2}} - \frac{1}{\sqrt{4-x}}
		\end{align*}
		\itemch  Giải phương trình $f^{\prime}(x)=0$.\\
		Với $x\in[-2;4]$, xét
		\allowdisplaybreaks
		\begin{eqnarray*}
			f'(x)=0&\Leftrightarrow&	\dfrac{1}{2\sqrt{x+2}} - \dfrac{1}{\sqrt{4-x}} = 0 \\&\Leftrightarrow&
			\dfrac{1}{2\sqrt{x+2}} = \frac{1}{\sqrt{4-x}} \Leftrightarrow
			\sqrt{4-x} = 2\sqrt{x+2} \\
			&\Leftrightarrow& 4-x = 4(x+2) \Leftrightarrow
			x= -\dfrac{4}{5}.
		\end{eqnarray*}
		Ta thấy	$-\dfrac{4}{5} \in [-2;4]$, suy ra khẳng định đúng.
		
		\itemch Tính giá trị của $f(x)$ tại các điểm đặc biệt
		\begin{align*}
			f(-2) &= \sqrt{0} + 2\sqrt{6} = 2\sqrt{6} \approx 4{,}9. \\
			f(4) &= \sqrt{6} + 2\sqrt{0} = \sqrt{6} \approx 2{,}4. \\
			f\left(-\dfrac{4}{5}\right) &= \sqrt{\dfrac{6}{5}} + 2\sqrt{\dfrac{24}{5}} = \sqrt{\dfrac{6}{5}} + 4\sqrt{\dfrac{6}{5}} = 5\sqrt{\dfrac{6}{5}} = \sqrt{30} \approx 5{,}5.
		\end{align*}
		Do đó, tập giá trị của hàm số là $[\sqrt{6}; \sqrt{30}]$; suy ra, các số nguyên thuộc tập này là $3$, $4$, $5$. \\
		Vậy có $3$ số nguyên, hay phát biểu là \textbf{Sai}.
	\end{itemchoice}
	}
\end{ex}

\begin{ex}%[Nguồn: Bộ đề minh họa Moon 2024-2025]%[2D1H3-1]
	Giá trị nhỏ nhất của hàm số $y = 2^{1-x}$ trên đoạn $\left[0; 1\right]$ bằng
	\choice
	{$2$}
	{$0$}
	{\True $1$}
	{$\dfrac{1}{2}$}
	\loigiai
	{
	Ta có $y' = -2^{1-x}\ln{2} < 0, \forall x \in \left[0; 1\right]$.\\
	Do đó giá trị nhỏ nhất của hàm số trên đoạn $\left[0; 1 \right]$ là $y\left(1 \right) = 1$.
	}
\end{ex}

\begin{ex}%[Nguồn: Bộ đề minh họa Moon 2024-2025]%[2D1H4-1]
	Cho hàm số $y=f(x)$ có bảng biến thiên như hình vẽ. Tiệm cận đứng của đồ thị hàm số đã cho là đường thẳng có phương trình
	\begin{center}
		\begin{tikzpicture}
			\tikzset{double style/.append style={double distance=1.5pt}}
			\tkzTabInit[nocadre=false,lgt=1.2,espcl=2.5,deltacl=0.6]
			{$x$ /0.6,$y'$ /0.6,$y$ /2}
			{$-\infty$,$1$,$+\infty$}
			\tkzTabLine{,-,d,-,}
			\tkzTabVar{+/$2$,-D+/$-\infty$/$+\infty$,-/$2$}
		\end{tikzpicture}
	\end{center}
	\choice
	{$x=2$}
	{$y=2$}
	{\True$x=1$}
	{$y=1$}
	\loigiai{
	Dựa vào bảng biến thiên ta có
	$\lim\limits_{x\to1^-}f(x)=-\infty$; $\lim\limits_{x\to1^+}f(x)=+\infty$.\\
	Vậy tiệm cận đứng dủa đồ thị hàm số đã cho là $x=1$.
	}
\end{ex}

\begin{ex}%[Nguồn: Bộ đề minh họa Moon 2024-2025]%[2D1H4-1]
	Đồ thị của hàm số nào sau đây có đường tiệm cận đứng là đường thẳng $x=1$ và đường tiệm cận ngang là đường thẳng $y=-2$?
	\choice
	{$y=\dfrac{x+2}{x-1}$}
	{$y=\dfrac{2x-1}{x+1}$}
	{\True $y=\dfrac{2x}{1-x}$}
	{$y=\dfrac{1-2x}{1-x}$}
	\loigiai{
	Xét hàm số $y=\dfrac{2x}{1-x}$.\\
	Ta có $\lim \limits_{x\to 1^+}\dfrac{2x}{1-x}=-\infty$, $\lim \limits_{x\to 1^-}\dfrac{2x}{1-x}=+\infty$.\\
	Do đó đồ thị hàm số có đường tiệm cận đứng $x=1$.\\
	Lại có $\lim \limits_{x\to +\infty}\dfrac{2x}{1-x}=\lim \limits_{x\to -\infty}\dfrac{2x}{1-x}=-2$ nên đồ thị hàm số có đường tiệm cận ngang $y=-2$.
	}
\end{ex}

\begin{ex}%[Nguồn: Bộ đề minh họa Moon 2024-2025]%[2D1H4-1]
	Tiệm cận xiên của đồ thị hàm số $y=\dfrac{x^2-3x+6}{x+2}$ là đường thẳng
	\choice
	{\True $y=x-5$}
	{$y=x+5$}
	{$y=x+2$}
	{$y=x-3$}
	\loigiai{
	Gọi phương trình đường tiệm cận xiên của hàm số đã cho là $y=ax+b$.\\
	Ta có
	\begin{itemize}
		\item Hệ số $a$ $$a=\lim\limits_{x\to+\infty}\dfrac{f(x)}{x}=\lim\limits_{x\to+\infty}\dfrac{x^2-3x+6}{x(x+2)}=1.$$
		\item Hệ số $b$
		\begin{eqnarray*}
			b&=&\lim\limits_{x\to+\infty}\left(f(x)-ax\right)\\
			&=&\lim\limits_{x\to+\infty}\left(\dfrac{x^2-3x+6}{x+2}-x\right)\\
			&=&\lim\limits_{x\to+\infty}\dfrac{x^2-3x+6-x^2-2x}{x+2}\\
			&=&-5.
		\end{eqnarray*}
	\end{itemize}
	Vậy phương trình đường tiệm cận xiên của hàm số đã cho là $y=x-5$.
	}
\end{ex}

\begin{ex}%[Nguồn: Bộ đề minh họa Moon 2024-2025]%[2D1H4-3]
	Tiệm cận đứng của đồ thị hàm số là đường thẳng có phương trình
	\choice
	{$y=-x-1$}
	{$y=1$}
	{$y=-1$}
	{\True $x=-1$}
	\loigiai{Từ đồ thị, ta có ta có tiệm cận đứng của đồ thị hàm số là đường thẳng có phương trình $x=-1$.}
\end{ex}

\begin{ex}%[Nguồn: Bộ đề minh họa Moon 2024-2025]%[2D1H5-1]
	Cho hàm số đa thức bậc ba $y=f(x)$ có bảng biến thiên như hình vẽ dưới đây
	\begin{center}
		\begin{tikzpicture}
			\tkzTabInit[espcl=2.5,lgt=1.5,nocadre]
			{$x$/0.7,$y'$/0.7,$y$/2.1}
			{$-\infty$,$-1$,$1$,$+\infty$}
			\tkzTabLine{,+,0,-,0,+,}
			\tkzTabVar{-/$-\infty$,+/$2$,-/$-2$,+/$+\infty$}
		\end{tikzpicture}
	\end{center}
	Hàm số nào dưới đây có bảng biến thiên như hình vẽ trên?
	\choice
	{$y = \dfrac{x+1}{x-1}$}
	{$y = -x^3 + 3x$}
	{$y = x^3 + 3x$}
	{\True $y = x^3 - 3x$}
	\loigiai{
	Hàm số $y=x^3-3x$ có bảng biến thiên như sau
	\begin{center}
		\begin{tikzpicture}
			\tkzTabInit[espcl=2.5,lgt=1.5,nocadre]
			{$x$/0.7,$y'$/0.7,$y$/2.1}
			{$-\infty$,$-1$,$1$,$+\infty$}
			\tkzTabLine{,+,0,-,0,+,}
			\tkzTabVar{-/$-\infty$,+/$2$,-/$-2$,+/$+\infty$}
		\end{tikzpicture}
	\end{center}
	}
\end{ex}

\begin{ex}%[Nguồn: Bộ đề minh họa Moon 2024-2025] %[2D1H5-1]
	\immini{Đồ thị trong hình vẽ là đồ thị của hàm số nào trong các hàm số dưới đây?
	\choice
	{$y=\dfrac{2x+1}{x-1}$}
	{$y=\dfrac{2x-1}{x-1}$}
	{$y=\dfrac{-x^2+3x+1}{x-1}$}
	{\True$y=\dfrac{x^2-3x+1}{x-1}$}
	}{\begin{tikzpicture}[scale=0.5,line join=round, line cap=round,>=stealth,thick]
		\tikzset{every node/.style={scale=0.9}}
		\draw[->] (-4.1,0)--(5.1,0) node[below left] {$x$};
		\draw[->] (0,-5.1)--(0,5.1) node[below left] {$y$};
		\draw (0,0) node [below left] {$O$};
		\draw[dashed,thin] (1.01,-5)--(1.01,5);
		\begin{scope}
			\clip (-5,-5) rectangle (5,5);
			\draw[samples=200,domain=-5:0.99,smooth,variable=\x] plot (\x,{(1*((\x)^2)+-3*(\x)+1)/(1*(\x)+-1)});
			\draw[samples=200,domain=1.01:5,smooth,variable=\x] plot (\x,{(1*((\x)^2)+-3*(\x)+1)/(1*(\x)+-1)});
			\draw[dashed,thin] (-5.1,-7.1)--(5.1,3.1);
		\end{scope}
	\end{tikzpicture}}
	\loigiai{
	Đồ thị hàm số có tiệm cận xiên nên ta loại hai hàm số $y=\dfrac{2x+1}{x-1}$ và $y=\dfrac{2x-1}{x-1}$.\\
	Dựa vào đồ thị ta thấy hàm số luôn đồng biến nên ta loại hàm số $y=\dfrac{-x^2+3x+1}{x-1}$.\\
	Vậy đồ thị hàm số trong hình vẽ là hàm số $y=\dfrac{x^2-3x+1}{x-1}$.
	}
\end{ex}

\begin{ex}%[Nguồn: Bộ đề minh họa Moon 2024-2025]%[2D1H5-1]
	Hình vẽ trên là đồ thị của hàm số nào trong các hàm số dưới đây?
	\choice
	{\True $y=\dfrac{x^{2}+2x+2}{-x-1}$}
	{$y=\dfrac{x^{2}+2x+2}{x+1}$}
	{$y=\dfrac{x^{2}-2x+2}{x-1}$}
	{$y=\dfrac{x^{2}-2x+2}{x+1}$}
	\loigiai{Ta có tiệm cận đứng của đồ thị hàm số là đường thẳng $x=-1$ và đồ thị đi qua điểm $(0;-2)$ do đó thỏa với $y=\dfrac{x^{2}+2x+2}{-x-1}$.}
\end{ex}

\begin{ex}%[Nguồn: Bộ đề minh họa Moon 2024-2025]%[2D1H5-1]
	Cho hàm số bậc ba $y = ax^3 - 2x + d$ $\left(a, d \in \mathbb{R} \right)$ có đồ thị như hình vẽ.
	\begin{center}
		\begin{tikzpicture}[>=stealth,line join=round,line cap=round,font=\footnotesize,scale=0.7]
			\draw[->]
			(-3.5,0) -- (4,0) node[below]{$x$}
			(0,-1.5) -- (0,3.5) node[right]{$y$}
			;
			
			\draw[samples=100,smooth,domain=-2.45:2.45] plot(\x,{(1/2)*(\x)^3 - 2*(\x)+1});
			\fill
			(0,0) circle(1pt) node[below left]{$O$}
			;
		\end{tikzpicture}
	\end{center}
	Khẳng định nào dưới đây đúng?
	\choice
	{$a > 0$, $d < 0$}
	{\True $a > 0$, $d > 0$}
	{$a < 0$, $d > 0$}
	{$a < 0$, $d < 0$}
	\loigiai
	{Từ đồ thị hàm số ta thấy $a > 0$ và $d > 0$.}
\end{ex}

\begin{ex}%[Nguồn: Bộ đề minh họa Moon 2024-2025]%[2D1H5-8]
	\immini{Một chất điểm chuyển động trong 3 giây với vận tốc $v(t)=a \cos (\pi t)+b$ (mét/giây) (trong đó, $t(s)$ là biến thời gian; $a, b$ là các hằng số) có đồ thị là một đường hình sin như hình sau:}{
	\begin{tikzpicture}[scale=0.3, line cap=round, line join=round, >=stealth, xscale=5, yscale=1]
		% Vẽ trục tọa độ
		\draw[->] (-0.5, 0) -- (3.5, 0) node[right] {$t (s)$};
		\draw[->] (0, -0.5) -- (0, 11.5) node[above] {$v (m/s)$};
		% Đánh dấu các điểm trên trục
		\foreach \x in {1, 2, 3}
		\fill (\x,0) circle (0.5pt) node[below] {\x};
		\draw (0,10) node[left]{10};
		\draw[thick, smooth, domain=0:3, samples=100] plot (\x, {5*cos(pi*\x r) + 5});
		
		% Vẽ các đường nét đứt
		\draw[dashed] (2, 0) -- (2, 10) -- (0,10);
		
		% Đánh dấu các điểm đặc biệt
		\fill (0, 0) circle (0.5pt) node[below left] {$O$};
	\end{tikzpicture}
	}
	\choiceTF
	{\True Vận tốc của vật tại thời điểm $t=2$ giây là $10\mathrm{~m} / s$}
	{\True Giá trị của $a$ là 5}
	{Giá trị của $b$ là 10}
	{Tổng quãng đường vật đi được sau 3 giây là $27,93\mathrm{~m}$}
	\loigiai{
	\begin{itemchoice}
		\itemch
		Dựa vào hình vẽ ta thấy vận tốc của vật tại thời điểm $t=2$ giây là $10m/s$
		\itemch
		Quan sát hình vẽ ta có
		\begin{eqnarray*}
			&&\heva{
			&v(1)=0 \\
			&v(2)=10
			}. \\
			&\Leftrightarrow&\heva{
			&a \cos (\pi)+b=0 \\
			&a \cos (2 \pi)+b=10
			}. \\
			& \Leftrightarrow&\heva{
			&-a+b=0 \\
			&a+b=10
			}. \Leftrightarrow a=b=5.
		\end{eqnarray*}
		Suy ra $v(t)=5 \cos (\pi t)+5$
		\itemch
		$b=5$.
		\itemch
		Ta có quãng đường vật đi được sau $3$s là
		$s(t)=\displaystyle\int_0^3 v(t) \mathrm{d} t
		=\displaystyle\int_0^3(5 \cos (\pi t)+5) \mathrm{d} t=15 .$
	\end{itemchoice}
	}
\end{ex}

%Câu trắc nghiệm 6

\begin{ex}%[Nguồn: Bộ đề minh họa Moon 2024-2025]%[2D1N1-1]
	Cho hàm số $y=f(x)$ có đạo hàm $f'(x)=-x^3$, $\forall x \in \mathbb{R}$. Hàm số đã cho đồng biến trên khoảng nào sau đây?
	\choice
	{$(-\infty;1)$}
	{$(-\infty;+\infty)$}
	{\True $(-\infty;0)$}
	{$(0;+\infty)$}
	\loigiai{
	Cho $f'(x)=0 \Leftrightarrow -x^3=0 \Leftrightarrow x=0$.\\
	Bảng biến thiên
	\begin{center}
		\begin{tikzpicture}[font=\footnotesize, line join=round, line cap=round, >=stealth, scale=1]
			\tkzTabInit[lgt=1.2,espcl=2.5,deltacl=0.6]
			{$x$/0.7, $f'(x)$/0.7, $f(x)$/1.5}
			{$-\infty$, $0$ , $+\infty$}
			\tkzTabLine
			{, + , $0$ , -}
			\tkzTabVar
			{-/ , +/ , -/}
		\end{tikzpicture}
	\end{center}
	Từ bảng biến thiên, hàm số $y=f(x)$ đồng biến trên khoảng $(-\infty;0)$.
	}
\end{ex}

\begin{ex}%[Nguồn: Bộ đề minh họa Moon 2024-2025]%[2D1N1-1]
	Hàm số nào sau đây nghịch biến trên $\mathbb{R}$?
	\choice
	{$y=-x^2+2$}
	{\True $y=-2011x+1$}
	{$y=x^2-3x+4$}
	{$y=\dfrac{1}{x-1}$}
	\loigiai{
	Xét hàm số $y=-2011x+1$.\\
	Tập xác định $\mathscr{D}=\mathbb{R}$.\\
	Ta có $y'=-2011<0,\, \forall x\in \mathbb{R}$.\\
	Vậy hàm số $y=-2011x+1$ nghịch biến trên $\mathbb{R}$.
	}
\end{ex}

%

\begin{ex}%[Nguồn: Bộ đề minh họa Moon 2024-2025]%[2D1N1-2]
	Cho hàm số $y=f(x)$ có bảng biến thiên như sau
	\begin{center}
		\begin{tikzpicture}
			\tkzTabInit[nocadre,lgt=1.2,espcl=2,deltacl=0.5]
			{$x$/0.6,$f'(x)$/0.6,$f(x)$/2}
			{$-\infty$,$-1$,$0$,$1$,$+\infty$}
			\tkzTabLine{,-,0,+,0,-,0,+,}
			\tkzTabVar{+/$+\infty$,-/$1$,+/$2$,-/$1$,+/$+\infty$}
		\end{tikzpicture}
	\end{center}
	Hàm số đã cho nghịch biến trên khoàng nào dưới đây?
	\choice
	{$(0;+\infty)$}
	{$(-\infty; 1)$}
	{\True $(0; 1)$}
	{$(-1; 0)$}
	\loigiai{
	Quan sát bảng biến thiên ta thấy $y'<0$ trên các khoảng $(-\infty;-1)$ và $(0;1)$ nên hàm số đã cho nghịch biến trên các khoảng $(-\infty;-1)$ và $(0;1)$.
	}
\end{ex}

%

\begin{ex}%[Nguồn: Bộ đề minh họa Moon 2024-2025]%[2D1N1-2]
	Cho hàm số $y = f(x)$ có bảng biến thiên như hình sau:
	\begin{center}
		\begin{tikzpicture}[scale=1]
			\tkzTabInit
			{$x$/1,$f'(x) $/1,$f(x)$/2}
			{$-\infty$,$-2$,$0$,$+\infty$}
			\tkzTabLine{,-,0,+,0,-,} %
			\tkzTabVar{+/$+\infty$,-/$0$, +/$4$,-/$-\infty$} %dấu mũi tên, + trên, -dưới
		\end{tikzpicture}
	\end{center}
	\choice
	{\True $(1;2)$}
	{$(-2;+\infty)$}
	{$(-1;2)$}
	{$(-\infty;0)$}
	\loigiai{Dựa vào bảng biến thiên ta có hàm số nghịch biến trên từng khoảng $(-\infty;-2)$ và $(0;+\infty)$.}
\end{ex}

%

\begin{ex}%[Nguồn: Bộ đề minh họa Moon 2024-2025]%[2D1N1-2]
	\immini{
	Cho hàm số $y=f(x)$ có đồ thị như hình vẽ bên. Hàm số đã cho đồng biến trên khoảng nào dưới đây?
	\choice
	{\True $(0;+\infty)$}
	{$(-1;+\infty)$}
	{$(-2; 0)$}
	{$(-4;+\infty)$}
	}
	{
	\begin{tikzpicture}[scale=0.65, line join=round, line cap=round, >=stealth,
		declare function={a=1; b=3; c=0; d=-4; f(\x)=a*(\x)^3 + b*(\x)^2 + c*(\x) + d; xmin=-3.5; xmax=2.5; ymin=-4.5; ymax=2;}]
		%		\draw[color=gray!50,thin] (xmin,ymin) grid (xmax,ymax);
		%		 Trục toạ độ
		\draw[->] (xmin,0)--(xmax,0) node[shift={(-100:7pt)},font=\normalsize]{$ x $};
		\draw[->] (0,ymin)--(0,ymax) node[shift={(-30:7pt)},font=\normalsize]{$ y $};
		\fill (0,0) circle (1pt) node[shift={(-45:9pt)},font=\normalsize]{$ O $};
		
		% Vẽ đồ thị hàm số
		\clip (xmin+0.1,ymin+0.1) rectangle (xmax-0.1,ymax-0.1);
		\draw[smooth, samples=100, line width=.5] plot[domain=xmin:xmax] (\x, {f(\x)});
		
		% Các điểm trên trục x và y
		\foreach \x/\goc in {-2/90,1/-60}{
		\fill (\x,0) circle (1pt) node[shift={(\goc:5pt)},font=\scriptsize]{$\x$};}
		\foreach \y/\goc in {-4/-150}{
		\fill (0,\y) circle (1pt) node[shift={(\goc:7pt)},font=\scriptsize]{$\y$};}
	\end{tikzpicture}
	}
	\loigiai{Quan sát đồ thị, ta thấy hàm số đã cho đồng biến trên khoảng $(0;+\infty)$.}
\end{ex}

\begin{ex}%[Nguồn: Bộ đề minh họa Moon 2024-2025]%[2D1N2-2]
	Cho hàm số đa thức bậc ba $y=f(x)$ có bảng biến thiên như hình vẽ dưới đây
	\begin{center}
		\begin{tikzpicture}
			\tkzTabInit[espcl=2.5,lgt=1.5,nocadre]
			{$x$/0.7,$y'$/0.7,$y$/2.1}
			{$-\infty$,$-1$,$1$,$+\infty$}
			\tkzTabLine{,+,0,-,0,+,}
			\tkzTabVar{-/$-\infty$,+/$2$,-/$-2$,+/$+\infty$}
		\end{tikzpicture}
	\end{center}
	Giá trị cực đại của hàm số $y=f(x)$ bằng
	\choice
	{$y_{\text{CĐ}} = -2$}
	{\True $y_{\text{CĐ}} = 2$}
	{$y_{\text{CĐ}} = -1$}
	{$y_{\text{CĐ}} = 1$}
	\loigiai{
	Từ bảng biến thiên ta thấy giá trị cực đại của hàm số là $y_{\text{CĐ}} = 2$.
	}
\end{ex}

\begin{ex}%[Nguồn: Bộ đề minh họa Moon 2024-2025]%[2D1N2-2]
	Cho hàm số $y=f(x)$ có bảng biến thiên như hình sau:
	\begin{center}
		\begin{tikzpicture}[scale=1, font=\footnotesize, line width=1pt]
			\tkzTabInit[nocadre=true, lgt=1.2, espcl=2, deltacl=0.6]
			{$x$/0.8,$f'(x)$/0.6,$f(x)$/2}
			{$-\infty$,$0$,$2$,$+\infty$};
			\tkzTabLine{,-,$0$,+,$0$,-,};
			\tkzTabVar{+/$+\infty$,-/$1$,+/$5$,-/$-\infty$};
		\end{tikzpicture}
	\end{center}
	Giá trị cực đại của hàm số đã cho bằng
	\choice
	{$0$}
	{$2$}
	{\True $5$}
	{$1$}
	\loigiai
	{
	Từ bảng biến thiên ta có giá trị cực đại của hàm số đã cho bằng $5$ tại $x=2$.
	}
\end{ex}

\begin{ex}%[Nguồn: Bộ đề minh họa Moon 2024-2025]%[2D1N2-2]
	Cho hàm số $y=f(x)$ có bảng xét dấu đạo hàm như sau
	\begin{center}
		\begin{tikzpicture}
			\tkzTabInit[lgt=1.5,espcl = 2]
			{$x$ /1, $f'(x)$ /1}
			{$-\infty$,$-2$,$0$,$1$, $3$,$+\infty$}
			\tkzTabLine{ ,+,z,-,z, +,z,-,z,- }
		\end{tikzpicture}
	\end{center}
	Hàm số đã cho có bao nhiêu điểm cực trị?
	\choice
	{$2$}
	{$4$}
	{\True $3$}
	{$1$}
	\loigiai{
	Từ bảng biến thiên ta thấy, $f'(x)$ đổi dấu $3$ lần nên hàm số $y=f(x)$ có $3$ điểm cực trị.
	}
\end{ex}

\begin{ex}%[Nguồn: Bộ đề minh họa Moon 2024-2025]%[2D1N2-2]
	Cho hàm số $y = f(x)$ có bảng biến thiên như hình vẽ dưới đây.
	\begin{center}
		\begin{tikzpicture}[>=stealth,line join=round,line cap=round,font=\footnotesize,scale=1]
			\tkzTabInit[espcl=2]{$x$ /.8 , $f'(x)$ / .7 , $f(x)$ / 1.8}{$-\infty$, $-1$, $0$, $1$, $+\infty$}
			\tkzTabLine{ , - , z , + , z , - , z , + , }
			\tkzTabVar{+/ $+\infty$ , -/ $0$ , +/ $3$ , -/ $0$ , +/ $+\infty$}
		\end{tikzpicture}
	\end{center}
	Hàm số có giá trị cực tiểu bằng
	\choice
	{$3$}
	{$-1$}
	{$1$}
	{\True $0$}
	\loigiai{
	Dựa vào bảng biến thiên, ta thấy hàm số đạt cực tiểu tại $x = -1$ và $x = 1$ và có giá trị cực tiểu là $0$.
	}
\end{ex}

\begin{ex}%[Nguồn: Bộ đề minh họa Moon 2024-2025]%[2D1N2-2]
	Cho hàm số $y = f(x)$ liên tục trên $\mathbb{R}$ và có đồ thị như hình vẽ dưới. Trên đoạn $\left[-3; 4 \right]$, hàm số đã cho có bao nhiêu cực trị?
	\begin{center}
		\begin{tikzpicture}[>=stealth,line join=round,line cap=round,font=\footnotesize,scale=0.7]
			\draw[->]
			(-3.5,0) -- (5,0) node[below]{$x$}
			(0,-1) -- (0,6) node[right]{$y$}
			;
			\draw
			(-3,4) -- (0,3)
			(3,0) -- (4,5)
			;
			\draw[dashed]
			(-3,0) -- (-3,4) -- (1,4) -- (1,0)
			(4,0) -- (4,5) -- (0,5)
			;
			\draw[samples=100,smooth,domain=0:3] plot(\x,{(-1)*(\x)^2 + 2*(\x)+3});
			\fill
			(0,0) circle(1pt) node[below left]{$O$}
			(-3,0) circle(1pt) node[below]{$-3$}
			(1,0) circle(1pt) node[below]{$1$}
			(4,0) circle(1pt) node[below]{$4$}
			(0,3) circle(1pt) node[below left]{$3$}
			(0,4) circle(1pt) node[below left]{$4$}
			(0,5) circle(1pt) node[left]{$5$}
			;
		\end{tikzpicture}
	\end{center}
	\choice
	{\True $4$}
	{$3$}
	{$2$}
	{$1$}
	\loigiai
	{Từ đồ thị của hàm số trên đoạn $\left[-3; 4\right
	]$ ta thấy hàm số đã cho có $4$ cực trị.
	}
\end{ex}

\begin{ex}%[Nguồn: Bộ đề minh họa Moon 2024-2025]%[2D1N2-2]
	Cho hàm số $f(x)$ liên tục trên đoạn $[-3;3]$ và có bảng xét dấu đạo hàm như sau
	\begin{center}
		\begin{tikzpicture}
			\tkzTabInit[nocadre=false,lgt=2,espcl=2.1]
			{$x$ /0.6,$f'(x)$ /0.6}
			{$-3$,$-1$,$0$,$1$,$2$,$3$}
			\tkzTabLine{,+,$0$,-,$0$,-,$0$,+,$0$,-}
		\end{tikzpicture}
	\end{center}
	Mệnh đề nào \textbf{sai}?
	\choice
	{Hàm số đạt cực tiểu tại $x=1$}
	{Hàm số đạt cực đại tại $x=2$}
	{Hàm số đạt cực đại tại $x=-1$}
	{\True Hàm số đạt cực tiểu tại $x=0$}
	\loigiai{
	Từ bảng xét dấu của $f'(x)$, ta  suy ra khẳng định \lq\lq Hàm số đạt cực tiểu tại $x=0$\rq\rq \, là sai.
	}
\end{ex}

%

\begin{ex}%[Nguồn: Bộ đề minh họa Moon 2024-2025]%[2D1N2-2]
	Giá trị cực tiểu của hàm số $y=f(x)$ trên đoạn $[1 ; 5]$ là
	\choice
	{$y_{\text{CT}}=3$}
	{$y_{\text{CT}}=4$}
	{$y_{\text{CT}}=2$}
	{\True $y_{\text{CT}}=0$}
	\loigiai{}
\end{ex}

\begin{ex}%[Nguồn: Bộ đề minh họa Moon 2024-2025]%[2D1N2-2]
	Cho hàm số $y=f(x)$ xác định trên $\mathbb{R}\setminus\{0\}$, liên tục trên mỗi khoảng xác định và có bảng biến thiên như sau
	\begin{center}
		\begin{tikzpicture}
			\tkzTabInit[espcl=3,lgt=1.5,nocadre]
			{$x$/0.7,$f'(x)$/0.7,$f(x)$/2.1}
			{$-\infty$,$0$,$1$,$+\infty$}
			\tkzTabLine{,-,d,+,0,-,}
			\tkzTabVar{+/$+\infty$,-D-/$-1$/$-\infty$,+/$2$,-/$-\infty$}
		\end{tikzpicture}
	\end{center}
	Hàm số đã cho có bao nhiêu điểm cực trị?
	\choice
	{$3$}
	{$0$}
	{\True $1$}
	{$2$}
	\loigiai{
	Từ bảng biến thiên, ta thấy  hàm số xác định và có đạo hàm đổi dấu tại $x=1$ nên hàm số có $1$ điểm cực trị.
	}
\end{ex}

\begin{ex}%[Nguồn: Bộ đề minh họa Moon 2024-2025]%[2D1N3-1]
	Cho hàm số $y=f(x)$ có bảng biến thiên như sau
	\begin{center}
		\begin{tikzpicture}[font=\normalsize,t style/.style={style=solid}]
			\tkzTabInit[nocadre=true, lgt=1.2, espcl=2.5, deltacl=0.5]
			{$x$/0.75, $f'(x)$/0.75, $f(x)$/3}
			{$-\infty$, $-2$, $3$, $+\infty$}
			\tkzTabLine{ , +, d, -, 0, +}
			\path
			($(N13)!0.1!(N12)$) node (A1){$5$}
			($(N23)!0.8!(N22)$) node[xshift=-15pt] (A2){$+\infty$}
			($(N23)!0.8!(N22)$) node[xshift=10pt] (A22){$4$}
			($(N33)!0.1!(N32)$) node (A3){$0$}
			($(N43)!0.8!(N42)$) node (A4){$+\infty$};
			\foreach \x/\y in {A1/A2, A22/A3, A3/A4}{
			\draw[-stealth] (\x)--(\y);
			}
			\draw[double] (N22)--(N23);
		\end{tikzpicture}
	\end{center}
	Giá trị nhỏ nhất của hàm số $y=f(x)$ trên đoạn $[0;6]$ bằng
	\choice
	{\True $0$}
	{$4$}
	{$f(4)$}
	{$f(6)$}
	\loigiai{
	Giá trị nhỏ nhất của hàm số $y=f(x)$ trên đoạn $[0;6]$ bằng $0$.
	}
\end{ex}

\begin{ex}%[Nguồn: Bộ đề minh họa Moon 2024-2025]%[2D1N3-1]
	\immini[thm]{
	Cho hàm số $y = f(x)$ liên tục trên đoạn $[-1;3]$ và có đồ thị như hình vẽ bên.	Gọi $M$, $m$ lần lượt là giá trị lớn nhất và giá trị nhỏ nhất của hàm số trên đoạn $[-1;3]$. Khi đó $M + m$ bằng
	\choice[2]
	{$-6$}
	{\True $-2$}
	{$-5$}
	{$2$}
	}{
	\begin{tikzpicture}[>=stealth,line join=round,line cap=round,font=\footnotesize,scale=.6]
		\tikzset{declare function={
		x1=-1.2;x2=3.5;y1=-4.5;y2=2.5;
		a=-1;b=3;c=0;d=-4;
		f(\x)=a*(\x)^3+b*(\x)^2+c*(\x)+d;
		}}
		\begin{scope}
			\clip (x1,y1)rectangle(x2,y2);
			\draw
			(-1,2) .. controls ++(-80:1) and ++(165:.8) .. (0, -3)..controls ++(6:.5) and ++(-120:.5)..(1,-2)
			(1,-2)--(2,-4)--(3,1);
		\end{scope}
		\draw[->] (0,y1)--(0,y2)node[left]{$y$};
		\draw[->] (x1,0)--(x2,0)node[above]{$x$};
		\filldraw(0,0)circle(1.2pt)node[below left]{$O$};
		\draw[dashed]
		(1,0)|-(0,-2) (2,0)|-(0,-4) (3,0)|-(0,1) (-1,0)|-(0,2)
		;
		\foreach \x/\goc in {-1/-110,1/90,2/90,3/-90}{
		\draw[fill=black] (\x,0)circle(1.2pt) node[shift={(\goc:2.8mm)}]{$\x$};
		}
		\foreach \y/\goc in {2/0,-3/200,-4/-180,1/180}{
		\draw[fill=black] (0,\y)circle(1.2pt)node[shift={(\goc:2.8mm)}]{$\y$};
		}
		
	\end{tikzpicture}
	}
	\loigiai{
	Từ đồ thị, ta thấy giá trị lớn nhất của hàm số trên đoạn $[-1;3]$ là $M = 2$ (đạt được tại $x = -1$) và giá trị nhỏ nhất là $m = -4$ (đạt được tại $x = 3$). Vậy $M + m = 2 + (-4) = -2$.
	}
\end{ex}

%

\begin{ex}%[Nguồn: Bộ đề minh họa Moon 2024-2025]%[2D1N3-1]
	Trên đoạn $[1 ; 5]$, hàm số đã cho đạt giá trị lớn nhất tại điểm
	\choice
	{$x=4$}
	{$x=1$}
	{\True $x=2$}
	{$x=5$}
	\loigiai{}
\end{ex}

%

\begin{ex}%[Nguồn: Bộ đề minh họa Moon 2024-2025]%[2D1N4-1]
	Cho hàm số $y=\dfrac{a x+b}{c x+d}$, $(c \neq 0, a d-b c \neq 0)$ có đồ thị như hình vẽ dưới. Tiệm cận ngang của đồ thị hàm số là
	\begin{center}
		\begin{tikzpicture}[line cap=butt,line join=miter,>=stealth,xscale=0.62,yscale=0.62]
			\tikzset{declare function={xmin=-5;xmax=3;Xkxd=-1;
			ymin=-3.5;ymax=4.5;Ykxd=1/2;
			a=1; b=-1; c=2; d=2; f(\x)=(a*(\x)+b)/(c*(\x)+d);
			},
			smooth,samples=50
			}
			\draw[->] (xmin-0.25,0)--(xmax+0.5,0)
			node[shift={(-95:7pt)},font=\normalsize]{$ x $};
			\draw[->] (0,ymin-0.25)--(0,ymax+0.5)
			node[shift={(5:7pt)},font=\normalsize]{$ y $};
			\fill (0,0) node[shift={(225:9pt)},font=\normalsize]{$ O $};
			\foreach \x in {-5, -4, -3, -2, -1, 1, 2, 3}{
			\draw (\x,2pt)--(\x,-2pt) +(0,-9pt) node[font=\scriptsize,fill=white,inner sep=1pt]{$\x$};
			}
			\foreach \y in {-3, -2, -1, 1, 2, 3, 4}{
			\draw (2pt,\y)--(-2pt,\y) +(-3pt,0) node[font=\scriptsize,anchor=east,fill=white,inner sep=1pt]{$\y$};
			}
			\begin{scope}[thick]
				\clip (xmin,ymin) rectangle (xmax,ymax);
				\draw[blue!50!black] (xmin,Ykxd)--(xmax,Ykxd)
				(Xkxd,ymin)--(Xkxd,ymax);
				\draw[blue] plot[domain=xmin:{Xkxd-0.02}] (\x, {f(\x)});
				\draw[blue] plot[domain={Xkxd+0.02}:xmax] (\x, {f(\x)});
			\end{scope}
		\end{tikzpicture}
	\end{center}
	
	\choice
	{$x=-1$}
	{\True $y=\dfrac{1}{2}$}
	{$y=-1$}
	{$x=\dfrac{1}{2}$}
	\loigiai{
	Từ đồ thị ta có tiệm cận ngang của đồ thị hàm số đã cho là $y=\dfrac{1}{2}$.
	}
\end{ex}

\begin{ex}%[Nguồn: Bộ đề minh họa Moon 2024-2025]%[2D1N4-1]
	Cho hàm số $y=f(x)$ có bảng biến thiên như sau
	\begin{center}
		\begin{tikzpicture}[font=\normalsize,t style/.style={style=solid}]
			\tkzTabInit[nocadre=true, lgt=1.2, espcl=2.5, deltacl=0.5]
			{$x$/0.75, $f'(x)$/0.75, $f(x)$/3}
			{$-\infty$, $-2$, $3$, $+\infty$}
			\tkzTabLine{ , +, d, -, 0, +}
			\path
			($(N13)!0.1!(N12)$) node (A1){$5$}
			($(N23)!0.8!(N22)$) node[xshift=-15pt] (A2){$+\infty$}
			($(N23)!0.8!(N22)$) node[xshift=10pt] (A22){$4$}
			($(N33)!0.1!(N32)$) node (A3){$0$}
			($(N43)!0.8!(N42)$) node (A4){$+\infty$};
			\foreach \x/\y in {A1/A2, A22/A3, A3/A4}{
			\draw[-stealth] (\x)--(\y);
			}
			\draw[double] (N22)--(N23);
		\end{tikzpicture}
	\end{center}
	Tổng số đường tiệm cận đứng và tiệm cận ngang của đồ thị hàm số $y=f(x)$ là
	\choice
	{\True $2$}
	{$1$}
	{$3$}
	{$0$}
	\loigiai{
	Ta có $\lim\limits_{x\to -\infty}y=5 $ suy ra $y=5$ là tiệm cận ngang của đồ thị hàm số.\\
	$\lim\limits_{x\to -2^-}y=+\infty $ suy ra $x=-2$ là tiệm cận đứng của đồ thị hàm số.\\
	Vậy đồ thị hàm số có $2$ đường tiệm cận.
	}
\end{ex}

%Câu trắc nghiệm 3

\begin{ex}%[Nguồn: Bộ đề minh họa Moon 2024-2025]%[2D1N4-1]
	Cho hàm số $y=f(x)=\dfrac{ax+b}{cx+d}$ có đồ thị như hình vẽ sau
	\begin{center}
		\begin{tikzpicture}[font=\footnotesize, line join=round, line cap=round, >=stealth, scale=0.5]
			\begin{scope}
				\clip (-4,-6) rectangle (6,4);
				\draw (1,-6)--(1,4) (-4,-1)--(6,-1);
				\draw[smooth, samples=500] plot[domain=-4:0.9] (\x,{(-(\x)+2)/((\x)-1)});
				\draw[smooth, samples=500] plot[domain=1.1:6] (\x,{(-(\x)+2)/((\x)-1)});
			\end{scope}
			\draw[->] (-4,0)--(0,0)node[above left]{$O$}--(6,0)node[below]{$x$};
			\draw[->] (0,-6)--(0,4)node[left]{$y$};
			\foreach \x/\g/\n in {(1,0)/-45/1, (2,0)/45/2, (0,-1)/150/-1, (0,-2)/-150/-2}{
			\fill \x circle (2pt)+(\g:0.6)node{$\n$};
			}
		\end{tikzpicture}
	\end{center}
	Tâm đối xứng của đồ thị hàm số có tọa độ là
	\choice
	{$(1;0)$}
	{$(-1;1)$}
	{$(2;-2)$}
	{\True $(1;-1)$}
	\loigiai{
	Đồ thị hàm số có tiệm cận đứng là $x=1$, tiệm cận ngang là $y=-1$.\\
	Suy ra tâm đối xứng của đồ thị hàm số có tọa độ là $(1;-1)$.
	}
\end{ex}

\begin{ex}%[Nguồn: Bộ đề minh họa Moon 2024-2025]%[2D1N4-1]
	Cho hàm số $y=f(x)$ có bảng biến thiên như hình sau
	\begin{center}
		\begin{tikzpicture}[scale=1, line cap=round, line join=round, >=stealth,font=\footnotesize]
			\tkzTabInit[lgt=2.5,espcl=3] % Giãn cách cột
			{$x$ / 1 , $f'(x)$ / 1 , $f(x)$ / 2 }
			{$-\infty$, $-2$, $1$, $+\infty$}
			\tkzTabLine{,-,d,-,d,+,} % Đạo hàm f'(x)
			\tkzTabVar{+/ 4 ,-D+/1 /$+\infty$  , -D-/2 /2, +/$+\infty$ }
		\end{tikzpicture}
	\end{center}
	Số đường tiệm cận của đồ thị hàm số đã cho là
	\choice
	{$4$}
	{\True $2$}
	{$3$}
	{$1$}
	\loigiai{Dựa vào bảng biến thiên ta có
	$\displaystyle \lim _{x \rightarrow+\infty} f(x)=+\infty$, $\displaystyle \lim _{x \rightarrow-\infty} f(x)=4$ nên đồ thị có 1 TCN là $y=4$.\\
	Và $\displaystyle \lim _{x \rightarrow-2^{+}} f(x)=+\infty$ nên đồ thị hàm số có 1 TCĐ là  $x=-2$. Vậy đồ thị hàm số có tổng cộng 2 đường tiệm cận.
	}
\end{ex}

%

\begin{ex}%[Nguồn: Bộ đề minh họa Moon 2024-2025]%[2D1N5-1]
	Đồ thị hàm số $y=x^3-3x^2+2$ là đường cong trong hình nào dưới đây?
	\choice
	{\begin{tikzpicture}[thick,>=stealth,scale=0.6]
		\clip(-2.5,-1) rectangle (2.5,3.5);
		\draw[->,very thick,blue] (-2.5,0) -- (2.5,0) node[below left] {\small $x$};
		\draw[->,very thick,blue] (0,-1) -- (0,3.5) node[below left] {\small $y$};
		\draw [fill=white,draw=blue] (0,0) circle (1pt)node[below right] {\footnotesize $O$};
		\draw[very thick,black,smooth,samples=100,domain=-2.5:2.5] plot(\x,{(\x)^4-2*(\x)^2+2});
		%	\draw[dashed, thick,blue]
		%	(1,0) node[below]{1}|-(0,1) node[above left]{1}
		%	(-1,0) node[below]{-1}|-(0,1)node[left]{};
	\end{tikzpicture}}
	{\begin{tikzpicture}[thick,>=stealth,scale=0.6]
		\clip(-2.5,-1) rectangle (2.5,4.5);
		\draw[->,very thick,blue] (-2.5,0) -- (2.5,0) node[below left] {\small $x$};
		\draw[->,very thick,blue] (0,-1) -- (0,4.5) node[below left] {\small $y$};
		\draw [fill=white,draw=blue] (0,0) circle (1pt)node[below right] {\footnotesize $O$};
		\draw[very thick,black,smooth,samples=100,domain=-2.5:2.5] plot(\x,{(\x)^3-3*(\x)+2});
		%	\draw[dashed, thick,blue]
		%	(1,0) node[below]{1}|-(0,1) node[above left]{1}
		%	(-1,0) node[below]{-1}|-(0,1)node[left]{};
	\end{tikzpicture}}
	{\True \begin{tikzpicture}[thick,>=stealth,scale=0.6]
		\clip(-2,-2.5) rectangle (4.5,3.5);
		\draw[->,very thick,blue] (-2,0) -- (4.5,0) node[below left] {\small $x$};
		\draw[->,very thick,blue] (0,-2.5) -- (0,3.5) node[below left] {\small $y$};
		\draw [fill=white,draw=blue] (0,0) circle (1pt)node[below right] {\footnotesize $O$};
		\draw[very thick,black,smooth,samples=100,domain=-2:4.5] plot(\x,{(\x)^3-3*(\x)^2+2});
		%	\draw[dashed, thick,blue]
		%	(1,0) node[below]{1}|-(0,1) node[above left]{1}
		%	(-1,0) node[below]{-1}|-(0,1)node[left]{};
	\end{tikzpicture}}
	{\begin{tikzpicture}[thick,>=stealth,scale=0.6]
		\clip(-2,-1.5) rectangle (4.5,4.5);
		\draw[->,very thick,blue] (-2,0) -- (4.5,0) node[below left] {\small $x$};
		\draw[->,very thick,blue] (0,-1.5) -- (0,4.5) node[below left] {\small $y$};
		\draw [fill=white,draw=blue] (0,0) circle (1pt)node[below right] {\footnotesize $O$};
		\draw[very thick,black,smooth,samples=100,domain=-2:4.5] plot(\x,{-(\x)^3+3*(\x)^2});
		%	\draw[dashed, thick,blue]
		%	(1,0) node[below]{1}|-(0,1) node[above left]{1}
		%	(-1,0) node[below]{-1}|-(0,1)node[left]{};
	\end{tikzpicture}}
	\loigiai{
	Đồ thị hàm bậc ba $y=x^3-3x^2+2$ với $a>0$ nên loại
	\begin{center}
		\begin{tabular}{cc}
			\begin{tikzpicture}[thick,>=stealth,scale=0.6]
				\clip(-2.5,-1) rectangle (2.5,3.5);
				\draw[->,very thick,blue] (-2.5,0) -- (2.5,0) node[below left] {\small $x$};
				\draw[->,very thick,blue] (0,-1) -- (0,3.5) node[below left] {\small $y$};
				\draw [fill=white,draw=blue] (0,0) circle (1pt)node[below right] {\footnotesize $O$};
				\draw[very thick,black,smooth,samples=100,domain=-2.5:2.5] plot(\x,{(\x)^4-2*(\x)^2+2});
				%	\draw[dashed, thick,blue]
				%	(1,0) node[below]{1}|-(0,1) node[above left]{1}
				%	(-1,0) node[below]{-1}|-(0,1)node[left]{};
			\end{tikzpicture}
			&
			\begin{tikzpicture}[thick,>=stealth,scale=0.6]
				\clip(-2,-1.5) rectangle (4.5,4.5);
				\draw[->,very thick,blue] (-2,0) -- (4.5,0) node[below left] {\small $x$};
				\draw[->,very thick,blue] (0,-1.5) -- (0,4.5) node[below left] {\small $y$};
				\draw [fill=white,draw=blue] (0,0) circle (1pt)node[below right] {\footnotesize $O$};
				\draw[very thick,black,smooth,samples=100,domain=-2:4.5] plot(\x,{-(\x)^3+3*(\x)^2});
				%	\draw[dashed, thick,blue]
				%	(1,0) node[below]{1}|-(0,1) node[above left]{1}
				%	(-1,0) node[below]{-1}|-(0,1)node[left]{};
			\end{tikzpicture}
		\end{tabular}
	\end{center}
	Hàm số $y=x^3-3x^2+2$ có $y'=3x^2-6x$. Cho $y'=0 \Rightarrow \hoac{&x=0\\&x=2.}$\\
	Suy ra $x=0$ và $x=2$ là hai điểm cực trị nên chọn
	\begin{center}
		\begin{tikzpicture}[thick,>=stealth,scale=0.6]
			\clip(-2,-2.5) rectangle (4.5,3.5);
			\draw[->,very thick,blue] (-2,0) -- (4.5,0) node[below left] {\small $x$};
			\draw[->,very thick,blue] (0,-2.5) -- (0,3.5) node[below left] {\small $y$};
			\draw [fill=white,draw=blue] (0,0) circle (1pt)node[below right] {\footnotesize $O$};
			\draw[very thick,black,smooth,samples=100,domain=-2:4.5] plot(\x,{(\x)^3-3*(\x)^2+2});
		\end{tikzpicture}
	\end{center}
	}
\end{ex}

\begin{ex}%[Nguồn: Bộ đề minh họa Moon 2024-2025]%[2D1N5-1]
	Bảng biến thiên ở hình vẽ là của hàm số nào dưới đây?
	\begin{center}
		\begin{tikzpicture}
			\tkzTabInit[espcl=3,lgt=1.5,nocadre]
			{$x$/0.7,$y'$/0.7,$y$/2.1}
			{$-\infty$,$2$,$+\infty$}
			\tkzTabLine{,-,d,-,}
			\tkzTabVar{+/$1$,-D+/$-\infty$/$+\infty$,-/$1$}
		\end{tikzpicture}
	\end{center}
	\choice
	{$y=\dfrac{2x+1}{x-2}$}
	{$y=\dfrac{x-1}{2x+1}$}
	{\True $y=\dfrac{x+1}{x-2}$}
	{$y=\dfrac{x+3}{2+x}$}
	\loigiai{
	Dựa vào bảng biên thiên ta thấy hàm số đã cho có tiệm cận đứng $x=2$ và tiệm cận ngang $y=1$.
	}
\end{ex}

%

\begin{ex}%[Nguồn: Bộ đề minh họa Moon 2024-2025]%[2D1N5-7]
	Cho đồ thị hàm số $f(x)=\dfrac{3x^2-2x-5}{x-2}$ có tâm đối xứng là $I(a;b)$. Giá trị của biểu thức $a+3b$ là
	\choice
	{$11$}
	{\True $32$}
	{$31$}
	{$23$}
	\loigiai{ $f(x)=\dfrac{3x^2-2x-5}{x-2}=3x+4+\dfrac{3}{x-2}$.\\
	Tập xác định của hàm số là $\mathscr{D}=\mathbb{R}\setminus\left\{2\right\}$.\\
	Ta có \begin{itemize}
		\item $\heva{&\lim\limits_{x \to 2^+} f(x)=+\infty\\&\lim\limits_{x \to 2^-} f(x)=-\infty}$ nên đồ thị hàm số có đường tiệm cận đứng là $x=2$.
		\item $\heva{&\lim\limits_{x\to +\infty} \left[f(x)-(3x+4)\right]=\lim\limits_{x\to +\infty} \dfrac{3}{x-2}=0\\&\lim\limits_{x\to +\infty} \left[f(x)-(3x+4)\right]=\lim\limits_{x\to +\infty} \dfrac{3}{x-2}=0}$ nên đồ thị hàm số có đường tiệm cận xiên là $y=3x+4$.
	\end{itemize}
	Khi đó, đồ thị hàm số đã cho có tâm đối xứng $I$ là giao điểm của hai đường tiệm cận của đồ thị hàm số.\\
	Suy ra, $I(2;10)$. Vậy $a+3b=2+3\cdot 10=32$.}
\end{ex}

\begin{ex}%[Nguồn: Bộ đề minh họa Moon 2024-2025]%[2D1V2-1]
	Cho hàm số $f(x)=\ln \left(x^2-2x+1\right)-x$.
	\choiceTF
	{Tập xác định của hàm số là $\mathscr{D}=\mathbb{R}$}
	{\True Đạo hàm của hàm số $f(x)$ trên tập xác định của nó là $f'(x)=\dfrac{2}{x-1}-1$}
	{\True Số nghiệm của phương trình $f'(x)=0$ trên khoảng $(1;+\infty)$ là $1$}
	{Giá trị cực đại của $f(x)$ trên khoảng $(1;+\infty)$ là $a \ln 2+b$ với $a, b \in \mathbb{Z}$ thì $a+b=1$}
	\loigiai{
	\begin{itemchoice}
		\itemch Điều kiện $x^2-2x+1>0\Leftrightarrow x\neq 1$.\\
		Tập xác định là $\mathscr{D}=\mathbb{R}\setminus\{1\}$.
		\itemch Ta có
		\allowdisplaybreaks
		\begin{eqnarray*}
			f'(x)&=&\dfrac{2x-2}{x^2-2x+1}-1
			\\
			&=&\dfrac{-x^2+4x-3}{x^2-2x+1}
			\\
			&=&\dfrac{(x-1)(-x+3)}{(x-1)^2}
			\\
			&=&\dfrac{-x+3}{x-1}=\dfrac{2}{x-1}-1.
		\end{eqnarray*}
		\itemch Ta có $f'(x)=0\Leftrightarrow \dfrac{-x+3}{x-1}=0\Leftrightarrow -x+3=0\Leftrightarrow x=3\in (1;+\infty)$.\\
		Suy ra số nghiệm của phương trình $f'(x)=0$ trên khoảng $(1;+\infty)$ là $1$
		\itemch Ta có bảng biến thiên của hàm số trên $(1;+\infty)$
		\begin{center}
			\begin{tikzpicture}[font=\normalsize,t style/.style={style=solid}]
				\tkzTabInit[nocadre=true, lgt=1.2, espcl=2.5, deltacl=0.5]%,help]
				{$x$/0.75, $y'$/0.75, $y$/3}
				{$1$, $3$, $+\infty$}
				\tkzTabLine{ , +, 0, -}
				\path
				($(N13)!0.1!(N12)$) node (A1){$-\infty$}
				($(N23)!0.8!(N22)$) node (A2){$f(3)$}
				($(N33)!0.1!(N32)$) node (A3){$-\infty$};
				\foreach \x/\y in {A1/A2, A2/A3}{
				\draw[-stealth] (\x)--(\y);
				}
			\end{tikzpicture}
		\end{center}
		Hàm số đạt cực đại tại $x=3$, giá trị cực đại là $f(3)= 2\ln 2-3$.\\
		Suy ra $a=2$, $b=-3$.\\
		Vậy $a+b=-1$.
	\end{itemchoice}
	}
\end{ex}

\begin{ex}%[Nguồn: Bộ đề minh họa Moon 2024-2025]%[2D1V2-7]
	Trong hệ tọa độ $Oxy$, giả sử một chiếc thuyền vào sông tại điểm $(1; 0)$ và giữ hướng về phía gốc tọa độ $O$. Do dòng chảy mạnh, thuyền đi theo đường cong có phương trình $y=\dfrac{x^2-1}{2x}$, trong đó $x$ và $y$ tính bằng kilômet (km) (như hình vẽ bên dưới).
	\begin{center}
		\begin{tikzpicture}[scale=0.6, line cap=round, line join=round, >=stealth,font=\footnotesize]
			% Vẽ trục x và y
			\draw[->] (-0.5,0) -- (8,0) node[right] {$x$};
			\draw[->] (0,-4.5) -- (0,4) node[above] {$y$};
			\foreach \x in {1,2,3,4,5,6,7} {
			\draw[->,blue,decorate,decoration={coil,aspect=0}]  (\x,3) -- (\x,-4);
			}
			% Vẽ thuyền (chi tiết hơn)
			\begin{scope}[shift={(6,1.5)},yscale=-1] % Dịch chuyển thuyền đến vị trí (6,1)
				% Thân thuyền (hình thang)
				% Vẽ thuyền
				\draw [fill=blue!60] (1,1.1) -- (1.3,1.2) -- (1.8,1.2) -- (1.8,1.1) -- cycle;
				% Cột buồm
				\draw[fill=brown!60] (1.4,0.6) -- (1.40,1.1);
				% Đường chéo 1
				\draw (1.1,1) -- (1.4,0.6) -- (1.8,1)--cycle;
			\end{scope}
			% Vẽ điểm (1,0)
			\filldraw (7,0) circle (2pt) node[below] {(1,0)};
		\end{tikzpicture}
		
	\end{center}
	\choiceTF
	{Nếu duy trì hướng đi, thuyền sẽ đến được gốc tọa độ $O$}
	{Trên hướng chuyển động, con tàu đi qua điểm $N\left(\dfrac{1}{2};-\dfrac{3}{2}\right)$}
	{Trên hưởng chuyển động, tại một vị trí $M\left(a; \dfrac{a^2-1}{2a}\right)$ bất kì, khoảng cách giữa thuyền đến gốc tọa độ bằng $OM=a^2+\left(\dfrac{a^2-1}{2a}\right)^2(\mathrm{~km})$}
	{Chiếc thuyền gần gốc tọa độ nhất một khoảng bằng $\dfrac{\sqrt{5}-1}{2}(\mathrm{~km})$}
	\loigiai{
	\begin{itemchoice}
		\itemch Nếu duy trì hướng đi, thuyền sẽ đến được gốc tọa độ tức điểm
		$O(0 ; 0)$
		phải nằm trên phương trình đường đi của con thuyền.\\
		Vì hàm số không xác định tại điểm có toạ độ $x=0$ do đó hàm số sẽ không đi qua gốc toạ độ.
		\itemch Thay tọa độ của điểm $N$ vào phương trình $y=\dfrac{x^2-1}{2x}$, ta thấy không thỏa mãn. Do đó con tàu không đi qua điểm $N\left(\dfrac{1}{2};-\dfrac{3}{2}\right)$.
		\itemch Khoảng cách giữa con thuyền tại vị trí $M$ bất kì so với gốc tọa độ là
		$
		O M=\sqrt{a^2+\left(\dfrac{a^2-1}{2 a}\right)^2}
		$.
		\itemch Khoảng cách gần nhất  giữa chiếc thuyền và gốc tọa độ bằng min $OM$
		\begin{eqnarray*}
			O M&=&\sqrt{a^2+\left(\dfrac{a^2-1}{2 a}\right)^2} \\
			& =&\sqrt{a^2+\dfrac{\left(a^2-1\right)^2}{4 a^2}} \\
			& =&\sqrt{a^2+\dfrac{a^4-2 a^2+1}{4 a^2}} \\
			& =&\sqrt{a^2+\dfrac{1}{4} a^2-\dfrac{1}{2}+\dfrac{1}{4 a^2}} \\
			& =&\sqrt{\dfrac{5}{4} a^2+\dfrac{1}{4 a^2}-\dfrac{1}{2}} \\
			& \geq& \sqrt{2 \sqrt{\dfrac{5}{4} a^2 \cdot \dfrac{1}{4 a^2}}-\dfrac{1}{2}} \\
			&=&\sqrt{\dfrac{\sqrt{5}-1}{2}} .
		\end{eqnarray*}
		Đẳng thức xảy ra khi và chỉ khi $$ \dfrac{5}{4}a^2=\dfrac{1}{4a^2}  \Leftrightarrow 5 a^4=1 \Leftrightarrow a^4=\dfrac{1}{5} \Leftrightarrow a=\dfrac{1}{\sqrt[4]{5}} .
		$$
		Vậy chiếc thuyền gần gốc tọa độ nhất một khoảng $\sqrt{\dfrac{\sqrt{5}-1}{2}}$.
	\end{itemchoice}
	}
\end{ex}

\begin{ex}%[Nguồn: Bộ đề minh họa Moon 2024-2025]%[2D1V2-7]
	Sự phân huỷ của rác thải hữu cơ có trong nước sẽ làm tiêu hao oxygen hoà tan trong nước. Nồng độ oxygen (mg/l) trong một hồ nước sau $t$ giờ ($t\ge 0$) khi một lượng rác thải hữu cơ bị xả vào hồ được mô phỏng bởi hàm số $f(t)=5-\dfrac{at}{9t^2+1}$ ($a\in\mathbb{R}$, $a > 0$). Biết rằng nồng độ oxygen trong hồ nước thấp nhất đo được là $2{,}5$ (mg/l).
	\choiceTF
	{\True Tại thời điểm $t=0$ nồng độ oxygen trong hồ nước bằng $5$ (mg/l)}
	{Nồng độ oxygen trong hồ nước thấp nhất sau $30$ phút}
	{Giá trị của $a$ là $a=5$}
	{\True Nồng độ oxygen trong hồ nước sau $1$ giờ là $3{,}5$ (mg/l)}
	\loigiai{
	\begin{itemchoice}
		\itemch
		Tại thời điểm $t=0$ nồng độ oxygen trong hồ nước là
		\[f(0)=5-\dfrac{a\cdot 0}{9\cdot 0^2+1}=5 \text{ (mg/l).}\]
		\itemch
		Ta có $f'(x)=\dfrac{a(9t^2+1)-18at^2}{(9t^2+1)^2}=\dfrac{-9at^2+a}{(9t^2+1)^2}$.\\
		Xét $f'(x)=0 \Rightarrow -9at^2+a=0\Leftrightarrow t=\dfrac{1}{3}$ (giờ).\\
		Vậy nồng độ oxygen trong hồ nước thấp nhất sau $20$ phút.
		\itemch
		Theo bài ra ta có lượng oxygen trong nước thấp nhất là $2{,}5$ (mg/l) nên ta có
		\begin{center}
			\begin{tikzpicture}
				\tkzTabInit[nocadre=true,lgt=1.2,espcl=2,deltacl=0.6]
				{$t$/1,$f(t)$/2}
				{$0$,$\dfrac{1}{3}$,$+\infty$}
				\tkzTabVar{+/$+\infty$,-/$2{,}5$,+/$+\infty$}
			\end{tikzpicture}
		\end{center}
		Suy ra
		\begin{eqnarray*}
			&&2{,}5=f\left(\dfrac{1}{3}\right)\\
			&\Leftrightarrow& 2{,}5=5-\dfrac{\dfrac{1}{3}a}{9\cdot \left(\dfrac{1}{3}\right) ^2+1}\\
			&\Leftrightarrow&-2{,}5=\dfrac{-1}{6}a\\
			&\Leftrightarrow& a=15.
		\end{eqnarray*}
		\itemch
		Nồng độ oxygen trong hồ nước sau $1$ giờ là
		\begin{eqnarray*}
			f(1)&=5-\dfrac{15\cdot 1}{9\cdot 1^2 +1}\\
			&=5-\dfrac{15}{10}\\
			&=3{,}5.
		\end{eqnarray*}
	\end{itemchoice}
	}
\end{ex}

\begin{ex}%[Nguồn: Bộ đề minh họa Moon 2024-2025]%[2D1V2-7]
	Một phần đường chạy của tàu lượn siêu tốc khi gắn hệ trục tọa độ $Oxy$ được mô phỏng ở hình 2. Biết đường chạy của nó có dạng đồ thị hàm số bậc ba $y= ax^3 + bx^2 + cx + d$ $(0 \leq x \leq 90)$; tàu lượn xuất phát từ điểm $A$ đồng thời đi qua các điểm $B$, $C$, $D$ (như hình vẽ). Đơn vị mỗi trục là mét, dựa vào đồ thị hình 2, em hãy tính độ cao lớn nhất (theo đơn vị mét) mà tàu lượn siêu tốc đạt được so với mặt đất (xem trục $Ox$ là mặt đất). (Làm tròn kết quả đến hàng phần mười).
	\begin{center}
		%\includegraphics[scale=0.38]{hinhve/tauluon.jpg}
		\begin{tikzpicture}[line join=round, line cap=round,>=stealth,thick,scale=0.7]
			\tikzset{every node/.style={scale=0.9}}
			\draw[->] (-1.1,0)--(10.1,0) node[below left] {$x$};
			\draw[->] (0,-1.1)--(0,7.1) node[below left] {$y$};
			\draw (0,0) node [below left] {$O$};
			\draw [dashed](3,0)node[below]{$30$}--(3,1)--(0,1)node[left]{$10$};
			\draw [dashed](8,0)node[below]{$80$}--(8,3)--(0,3)node[left]{$30$}circle(3pt) (5,0)node[below]{$50$}--(5,3);
			\fill (0,3)node[above right]{$A$} circle (3pt) (3,1)node[above]{$B$} circle (3pt) (5,3)node[above]{$C$} circle (3pt) (8,3)node[above right]{$D$} circle (3pt);
			\draw (3,-1)node[below]{Hình 2};
			\begin{scope}
				\clip (-1,-1) rectangle (10,7);
				\draw[samples=200,domain=-1:9,smooth,variable=\x] plot (\x,{-1/15*((\x)^3)+13/15*((\x)^2)-8/3*(\x)+3});
			\end{scope}
		\end{tikzpicture}
	\end{center}
	\shortans[0]{$39{,}9$}
	\loigiai{Từ giả thiết ta thấy đồ thị hàm số bậc ba $y= ax^3 + bx^2 + cx + d$ đi qua các điểm $A(0; 30)$, $B(30; 10)$, $C(50; 30)$, $D(80; 30)$.\\
	Thay tọa độ các điểm trên vào hàm số rồi giải hệ phương trình ta tìm được $a=-\dfrac{1}{1500}$, $b=\dfrac{13}{150}$, $c=-\dfrac{8}{3}$, $d=30$.\\
	Vậy đường chạy có dạng đồ thị hàm số $y=-\dfrac{1}{1500}x^3+\dfrac{13}{150}x^2-\dfrac{8}{3}x+30$.\\
	Ta có $y'=-\dfrac{1}{500}x^2+\dfrac{13}{75}x-\dfrac{8}{3}$.\\
	$y'=0\Leftrightarrow  \hoac{&x=\dfrac{200}{3}\\&x=20.}$\\
	Từ đó suy ra độ cao lớn nhất của tàu lượn bằng $y\left(\dfrac{200}{3}\right)\approx 39{,}9$ m.
	}
\end{ex}

%

\begin{ex}%[Nguồn: Bộ đề minh họa Moon 2024-2025]%[2D1V3-2]
	Cho hàm số $f(x)=2\cos x+x$.
	\choiceTF
	{\True $f(0)=2; f\left(\dfrac{\pi}{2}\right)=\dfrac{\pi}{2}$}
	{Đạo hàm của hàm số đã cho là $f'(x)=2\sin x+1$}
	{\True Nghiệm của phương trình $f'(x)=0$ trên đoạn $\left[0; \dfrac{\pi}{2}\right]$ là $\dfrac{\pi}{6}$}
	{\True Giá trị lớn nhất của $f(x)$ trên đoạn $\left[0; \dfrac{\pi}{2}\right]$ là $\sqrt{3}+\dfrac{\pi}{6}$}
	\loigiai{
	\begin{enumerate}
		\item Đúng.
		
		Ta có $f(0)=2 \cos 0+0=2 \cdot 1=2$ và $f\left(\dfrac{\pi}{2}\right)=2 \cos \dfrac{\pi}{2}+\dfrac{\pi}{2}=2 \cdot 0+\dfrac{\pi}{2}=\dfrac{\pi}{2}$.
		\item Sai.
		
		$f'(x)=2(\cos x)'+x'=-2 \sin x+1$
		\item Đúng.
		
		Ta có $f'(x)=0 \Leftrightarrow-2 \sin x+1=0 \Leftrightarrow \sin x=\dfrac{1}{2} \Leftrightarrow\left[\begin{array}{l}x=\dfrac{\pi}{6}+k 2 \pi \\ x=\dfrac{5 \pi}{6}+k 2 \pi\end{array},(k \in \mathbb{Z})\right.$.
		Vì $x \in\left[0 ; \dfrac{\pi}{2}\right]$ nên $x=\dfrac{\pi}{6}$.
		
		\item Đúng.
		
		Ta có $f(0)=2$ ; $f\left(\dfrac{\pi}{2}\right)=\dfrac{\pi}{2}$ và $f\left(\dfrac{\pi}{6}\right)=\sqrt{3}+\dfrac{\pi}{6}$.
		Do đó, giá trị lớn nhất của $f(x)$ trên đoạn $\left[0 ; \dfrac{\pi}{2}\right]$ là $\sqrt{3}+\dfrac{\pi}{6}$.
	\end{enumerate}
	}
\end{ex}

\begin{ex}%[Nguồn: Bộ đề minh họa Moon 2024-2025] %[2D1V3-6]
	Khi thả một quả bóng từ đỉnh một toà tháp xuống, nó chạm đất sau 3 giây. Sau đó, quả bóng nảy lên trước khi chạm đất lần nữa 4 giây sau đó. Chiều cao tinh bằng mét của quả bóng so với mặt đất sau $t$ giây tuân theo một hàm số liên tục trên $[0; 7]$ như sau:
	$$
	H(t)=\left\{\begin{array}{lll}-5 t^2+c & \text{khi} & 0 \leq t < 3 \\-5 t^2+d t+e & \text{khi} & 3 \leq t \leq 7
	\end{array}(c, d, e \in \mathbb{R}).\right.
	$$
	\begin{center}
		\begin{tikzpicture}[scale=0.8,>=stealth, font=\footnotesize, line join=round, line cap=round]
			\def\a{-0.5} \def\b{0} \def\c{4.5} % Hệ số
			\def\xmin{-1} \def\xmax{9}
			\def\ymin{-1} \def\ymax{6}
			\draw[color=gray!50,dashed] (\xmin,\ymin) grid (\xmax,\ymax);
			\draw[->] (\xmin,0)--(\xmax,0) node [below]{$t(s)$};
			\draw[->] (0,\ymin)--(0,\ymax) node [left]{$H(m)$};
			\node at (0,0) [below left]{$O$};
			\clip (\xmin+0.1,\ymin+0.1) rectangle (\xmax-0.5,\ymax-0.1);
			\draw[smooth,samples=300,domain=0:3] plot(\x,{\a*(\x)^2+\b*(\x)+\c});
			\draw[smooth,samples=300,domain=3:7] plot(\x,{-0.5*(\x)^2+5*(\x)-10.5});
			\fill[blue] (1,4) circle (5pt);
			\fill[blue] (5,2) circle (5pt);
			\draw (3,0) circle (1pt);
			\draw (6,1.5)  node [above]{$H(t)$};
			\draw (3,0)  node [below]{$3$};
			\draw (7,0)  node [below]{$7$};
		\end{tikzpicture}
	\end{center}
	\choiceTF
	{\True $H(3)=H(7)=0$}
	{Quả bóng được thả từ độ cao $40$ m}
	{Giá trị của $d$ là $d=100$}
	{\True Độ cao lớn nhắt mà quả bóng đạt được sau lần nảy đầu tiên là $20$ m}
	\loigiai{
	\begin{itemchoice}
		\itemch Dựa vào đôg thị ta có $H(3)=H(7)=0$.
		\itemch Quả bóng được thả tại thời điểm $t=0$, nên để tìm độ cao của quả bóng ta tìm $H(0)$.\\
		Vì hàm số liên tục tại $x=3$ nên $\lim\limits_{x\to3}H(t)=H(3)\Leftrightarrow -5\cdot3^2+c=0\Leftrightarrow c=45
		$.\\
		Vậy $H(0)=45$. Do đó quả bóng được thả từ độ cao $45$ m.
		\itemch
		Ta có $\heva{&H(3)=0\\&H(7)=0}\Leftrightarrow \heva{&-5\cdot3^2+3d+e=0\\&-5\cdot7^2+7d+e=0}\Leftrightarrow \heva{&d=50\\&e=-105.}$\\
		Vậy $H(t)=-5t^2+50t-105$.
		\itemch
		Độ cao của quả bóng sau lần nảy đầu tiên trong khoảng thời gian $t\in[3;7]$ nên được mô tả bởi $H(t)=-5t^2+50t-105$.\\
		$H'(t)=-10t+50=0\Leftrightarrow t=5$ (tm).\\
		Ta có $H(3)=H(7)=0$; $H(5)=20$.\\
		Vậy độ cao lớn nhất của quả bóng sau lần nảy đầu tiên là $20$ m.
	\end{itemchoice}
	
	}
\end{ex}

\begin{ex}%[Nguồn: Bộ đề minh họa Moon 2024-2025]%[2D1V3-6]
	\immini{
	Độ cứng $S$ của một thanh gỗ hình chữ nhật tỉ lệ thuận với tích của chiều rộng $w$ và bình phương chiều dài $d$ của nó (theo đơn vị mét). Biết rằng nếu thanh gỗ có chiều dài là $6$ cm, chiều rộng là  $3$ cm thì độ cứng của nó bằng $108$.
	Một khúc gỗ hình tròn có đường kính là $24$ cm, người ta cắt thành một thanh gỗ hình chữ nhật như hình vẽ bên.
	}{
	\begin{tikzpicture}[line join=round,line cap=round,>=stealth]
		\def\r{sqrt(20)/2}
		\path
		(0,0)coordinate(A)
		(2,0)coordinate(B)
		(2,4)coordinate(C)
		($(A)+(C)-(B)$)coordinate(D)
		($(A)!1/2!(C)$)coordinate(O)
		;
		\draw[fill=black](O)circle(\r);
		\draw[fill=white](A)--(B)--(C)--(D)--cycle;
		\draw[black](A)--(C)node[sloped,above,pos=.5]{$24$ cm};
		\draw(A)--(B)node[above,midway]{w};
		\draw(C)--(B)node[left,midway]{d};
	\end{tikzpicture}
	}
	\choiceTF
	{\True Chiều dài và chiều rộng của hình chữ nhật liên hệ bởi công thức $d^2 + w^2 = 576$}
	{Độ cứng của thanh gỗ là $S=2dw^2$}
	{\True Độ cứng của thanh gỗ được mô phỏng bằng hàm số $S(w)=576w-w^3,\quad\forall w\in(0; 24)$}
	{Độ cứng lớn nhất của miếng gỗ cắt ra được là $S_{\text{max}} = 5321$ (làm tròn đến hàng đơn vị)}
	\loigiai{
	\begin{itemchoice}
		\itemch Đường kính của khúc gỗ hình tròn là $24$ cm, do đó bán kính là $12$ cm. Khi cắt thành một hình chữ nhật, hai cạnh của hình chữ nhật là các cạnh của tam giác vuông có đường chéo bằng đường kính của hình tròn.\\
		Do đó, $d^2+w^2=24^2=576$.
		\itemch Ta có độ cứng $S$ tỉ lệ thuận với tích $w\cdot d^2$. Từ đó, suy ra $S=k\cdot w\cdot d^2$.\\
		Biết $S=108$ khi $w = 0{,}03$ m và $d=0{,}06$ m, ta tính được $k = 1$.\\
		Vậy $S=wd^2$.
		\itemch Đây là công thức được suy ra từ các công thức liên quan. Theo công thức $S=wd^2$, và $d = \sqrt{576 - w^2}$ từ $d^2+w^2=576$, ta có $S=w\cdot(\sqrt{576-w2})^2=576w-w^3$.
		\itemch Xét hàm số $S(t)=-w^3+576ww$.\\
		$S'(w)=-3ww^2+576$, $S'(w)=0\Leftrightarrow\hoac{&w=8\sqrt{3}\\&w=-8\sqrt{3}.}$\\
		Vậy độ cứng lớn nhất của miếng gỗ cắt ra được là $\max S=3841$ khi $w=8\sqrt{3}$.
	\end{itemchoice}
	}
\end{ex}

\begin{ex}%[Nguồn: Bộ đề minh họa Moon 2024-2025]%[2D1V3-6]
	Một cuộc khảo sát thị trường cho thấy khi một loại máy lọc không khí mới được giới thiệu ra thị trường, hàm lợi nhuận $P(t)$ sau $t$ tháng là nguyên hàm của hàm tốc độ sinh lời $p(t) = \dfrac{500\left[1{,}4 - \ln(0{,}5t+1)\right]}{t+2}$ (chục triệu đồng mỗi tháng). Biết rằng ban đầu công ty chi $1$ tỷ đồng để sản xuất máy lọc không khí.
	\choiceTF
	{\True $P(t) = 700\ln(t+2) - 250\ln^2(0,5t+1) + C$}
	{\True Lợi nhuận cực đại là $3{,}9$ tỷ đồng}
	{Công ty bắt đầu thua lỗ vào tháng thứ $26$}
	{Một sản phẩm được gọi là \lq\lq trào lưu \rq\rq nếu lợi nhuận tại thời điểm cực đại bằng hai lần lợi nhuận sản phẩm sau khi đạt cực đại $12$ tháng. Như vậy, sản phẩm trên không phải một trào lưu}
	\loigiai{
	\begin{itemchoice}
		\itemch Để kiểm tra đáp áp, ta cần tính đạo hàm của $P(t)$ và so sánh với $p(t)$.\\
		Ta có
		\allowdisplaybreaks
		\begin{eqnarray*}
			P'(t)&=&\dfrac{700}{t+2}-\dfrac{250\cdot2\cdot 0{,}5\cdot\ln(0{,}5t+1)}{0{,}5t+1}\\
			&=&\dfrac{700}{t+2}-\dfrac{250\ln(0{,}5t+1)}{\tfrac{1}{2}t+1}\\&=&\dfrac{700}{t+2}-\dfrac{500\ln(0{,}5t+1)}{t+2}\\
			&=&\dfrac{500\left[1{,}4-\ln(0{,}5t+1)\right]}{t+2}=p(t).
		\end{eqnarray*}
		\itemch Ta cần tìm $C$. Do ban đầu công ty chi $1$ tỷ đồng để sản xuất máy lọc không khí nên $P(0)=-100$. Khi đó
		\allowdisplaybreaks
		\begin{eqnarray*}
			-100=700\ln2-250\ln1+C\Leftrightarrow C=-100-700\ln2.
		\end{eqnarray*}
		Do đó,
		\allowdisplaybreaks
		\begin{eqnarray*}
			P(t)&=&700\ln(t+2)-250\ln^2(0{,}5t+1)-100-700\ln2\\
			&=&700\ln[2(0{,}5t+1)]-250\ln^2(0{,}5t+1)-100-700\ln2\\
			&=&700\ln(0{,}5t+1)-250\ln^2(0{,}5t+1)-100\\
			&=&390-250\left(\ln(0{,}5t+1)-\dfrac{7}{5}\right)^2\le 390.
		\end{eqnarray*}
		Dấu "=" xảy ra khi, $$\Leftrightarrow \ln(0{,}5t+1)=\dfrac{7}{5}\Leftrightarrow t=\dfrac{\mathrm{e}^{\tfrac{7}{5}}-1}{0{,}5}\approx6{,1}.$$
		Vậy lợi cực đại là $3{,}9$ tỷ đồng.
		\itemch Công ty sẽ bị lỗ nếu và chỉ nếu
		\allowdisplaybreaks
		\begin{eqnarray*}
			&&P(t)<0\Leftrightarrow 700\ln(0{,}5t+1)-250\ln^2(0{,}5t+1)-100<0\\&\Leftrightarrow&\hoac{&\ln(0{,}5t+1)<\dfrac{7-\sqrt{39}}{5}\\&\ln(0{,}5t+1)>\dfrac{7+\sqrt{39}}{5}}
			\Leftrightarrow \hoac{&0{,}5t+1<\mathrm{e}^{\tfrac{7-\sqrt{39}}{5}}\\&0{,}5t+1>\mathrm{e}^{\tfrac{7+\sqrt{39}}{5}}}\\
			&\Leftrightarrow&\hoac{&t<\dfrac{\mathrm{e}^{\tfrac{7-\sqrt{39}}{5}}-1}{0{,}5}\approx0{,}3\\&t>\dfrac{\mathrm{e}^{\tfrac{7+\sqrt{39}}{5}-1}}{0.5}\approx26{,}3.}
		\end{eqnarray*}
		Vậy công ty bắt đầu thua lỗ từ tháng $27$.
		\itemch \textbf{Sai}. Lợi nhuận tại thời điểm cực đại là $P(6{,}1)\approx3{,}9$ tỷ.\\
		Sau thời điểm cực đại $12$ tháng, tức là vào tháng $t\approx6{,}1+12\approx18{,}1$. Khi đó lợi nhuận là $P(18{,}1)\approx 1{,}8$ tỷ.\\
		Ta thấy lợi nhuận cực đại không bằng hai lần lợi nhuận sau khi đạt cực đại $12$ tháng.\\
		Vậy sản phẩm trên không phải là một trào lưu.
	\end{itemchoice}
	}
\end{ex}

%

\begin{ex}%[Nguồn: Bộ đề minh họa Moon 2024-2025]%[2D1V3-6]
	Một tấm bia cứng có kích thước $60$ (cm) và $90$ (cm) được gấp đôi thành một hình chữ nhật $60$ (cm) và $45$ (cm) như hình vẽ. Sau đó, cắt ra từ các góc của hình chữ nhật vừa gấp bốn hình vuông bằng nhau có cạnh $x$ (cm). Tấm bìa được mở ra và sáu mép được gấp lên để tạo thành một hộp chữ nhật ($H$) có nắp và đáy (như hình vẽ). Thể tích lớn nhất của khối (H) bằng bao nhiêu lít? (làm tròn kết quả đến hàng phần mười)
	\begin{center}
		\begin{tikzpicture}[>=stealth,line join=round,line cap=round,font=\footnotesize,scale=1]
			\draw (0,0)--(0,3)--(4,3)--(4,0)--cycle;
			\draw [dashed] (2,0)--(2,3);
			\path (0,3)--(0,0)--([turn]-90:0.35) coordinate (xt)
			($(0,3)-(0,0)+(xt)$) coordinate (yt);
			\draw[dash pattern=on 2pt off 2pt] (0,3)--(yt) (0,0)--(xt);
			\draw[>=stealth,|<->|] (xt)--(yt) node[fill=white,inner sep=0pt,font=\scriptsize,midway,sloped]{$60$ cm};
			\path (0,0)--(4,0)--([turn]-90:0.35) coordinate (xt)
			($(0,0)-(4,0)+(xt)$) coordinate (yt);
			\draw[>=stealth,|<->|] (xt)--(yt) node[fill=white,inner sep=0pt,font=\scriptsize,midway,sloped]{$90$ cm};
		\end{tikzpicture}\hspace{1cm} \begin{tikzpicture}[>=stealth,line join=round,line cap=round,font=\footnotesize,scale=1]
			\draw (0,0)--(0,0.5)--(-0.5,0.5)--(-0.5,2.5)--(0,2.5)--(0,3)--(1,3)--(1,2.5)--(1.5,2.5)--(1.5,0.5)--(1,0.5)--(1,0)--cycle;
			\path (0,0)--(0,0.5)node[pos=0.3,sloped,black,below,scale=0.8]{$x$}
			(-0.5,0.5)--(0,0.5)node[pos=0.4,sloped,black,above,scale=0.8]{$x$}
			(-0.5,2.5)--(0,2.5)node[pos=0.3,sloped,black,below,scale=0.8]{$x$}
			(0,2.5)--(0,3)node[pos=0.5,sloped,black,below,scale=0.8]{$x$}
			(1,3)--(1,2.5)node[pos=0.5,sloped,black,below,scale=0.8]{$x$}
			(1,2.5)--(1.5,2.5) node[pos=0.5,sloped,black,below,scale=0.8]{$x$}
			(1.5,0.5)--(1,0.5) node[pos=0.5,sloped,black,above,scale=0.8]{$x$}
			(1,0.5)--(1,0) node[pos=0.5,sloped,black,below,scale=0.8]{$x$};
			\path (-0.5,0)--(1.5,0)--([turn]-90:0.35) coordinate (xt)
			($(-0.5,0)-(1.5,0)+(xt)$) coordinate (yt);
			\draw[>=stealth,|<->|] (xt)--(yt) node[fill=white,inner sep=0pt,font=\scriptsize,midway,sloped]{$45$ cm};
		\end{tikzpicture}\hspace{1cm} \begin{tikzpicture}[>=stealth,line join=round,line cap=round,font=\footnotesize,scale=1]
			\draw (0,0)--(0,0.5)--(-0.5,0.5)--(-0.5,2.5)--(0,2.5)--(0,3)--(0.5,3)--(0.5,2.5)--(2.5,2.5)--(2.5,3)--(3,3)--(3,2.5)--(3.5,2.5)--(3.5,0.5)--(3,0.5)--(3,0)--(2.5,0)--(2.5,0.5)--(0.5,0.5)--(0.5,0)--cycle;
			\draw[dashed] (0,0.5) rectangle (0.5,2.5);
			\draw[dashed] (2.5,0.5) rectangle (3,2.5);
			\path (0,3)--(0,0)--([turn]-90:0.75) coordinate (xt)
			($(0,3)-(0,0)+(xt)$) coordinate (yt);
			\draw[>=stealth,|<->|] (xt)--(yt) node[fill=white,inner sep=0pt,font=\scriptsize,midway,sloped]{$60$ cm};
			\path (-0.5,0)--(3.5,0)--([turn]-90:0.25) coordinate (xt)
			($(-0.5,0)-(3.5,0)+(xt)$) coordinate (yt);
			\draw[>=stealth,|<->|] (xt)--(yt) node[fill=white,inner sep=0pt,font=\scriptsize,midway,sloped]{$90$ cm};
		\end{tikzpicture}
	\end{center}
	\shortans[1]{$20{,}5$}
	\loigiai{
	Sau khi cắt bốn hình vuông cạnh $x, (0<x<\dfrac{45}{2})$ cm và gấp tấm bìa, kích thước của hình hộp là
	\begin{itemize}
		\item Chiều dài $60-2x$;
		\item Chiều rộng $45-2x$;
		\item Chiều cao $2x$.
	\end{itemize}
	Thể tích khối hộp là
	\[
	V(x)=2x(60-2x)(45-2x)=8x^3-420x^2+5\,400x.
	\]
	Ta có $V'(x)=24x^2-840x+5400; V'(x)=0 \Leftrightarrow \hoac{
	&x=\dfrac{35+5\sqrt{13}}{2}\approx 36{,}18\,\,(\text{loại})\\
	&x=\dfrac{35-5\sqrt{13}}{2}\approx 13{,}82\,\,(\text{nhận}).
	}$\\
	Bảng biến thiên
	\begin{center}
		\begin{tikzpicture}
			\tkzTabInit[nocadre,lgt=2.2,espcl=2.5,deltacl=0.5]
			{$x$/1.6,$V'(x)$/0.6,$V(x)$/2}
			{$0$,$\dfrac{35-5\sqrt{13}}{2}$,$\dfrac{45}{2}$}
			\tkzTabLine{,-,0,+,}
			\tkzTabVar{-/$ $,+/$V\left(\dfrac{35-5\sqrt{13}}{2}\right)$,-/$ $}
		\end{tikzpicture}
	\end{center}
	Vậy thể tích lớn nhất của khối hộp là $V\left(\dfrac{35-5\sqrt{13}}{2}\right)\approx 20\,468{,}04 \,\,\rm{cm^3}\approx 20{,}5$ lít.
	}
\end{ex}

\begin{ex}%[Nguồn: Bộ đề minh họa Moon 2024-2025]%[2D1V3-6]
	Một chiếc phà chạy giữa đất liền và đảo Dedlos. Phà có công suất tối đa là $1\,000$ xe hơi mỗi chuyến, nhưng việc tải gần hết công suất rất tốn thời gian. Biết rằng số lượng xe hơi đưa lên phà mỗi chuyến là $f(t)=\dfrac{2000t}{2t+1}$ và mất một khoảng thời gian là $1$ giờ. Mỗi xe cần trung bình $3{,}6$ giây để dỡ xuống khi đến điểm đích. Thời gian di chuyển đến đảo và thời gian vòng về đều mất $1{,}28$ giờ. Nên tải bao nhiêu xe lên phà cho mỗi chuyến đi để lượng xe trung bình di chuyển qua lại đảo mỗi giờ đạt lớn nhất? (làm tròn kết quả đến hàng đơn vị).
	\shortans[oly]{615}
	\loigiai{
	Để đưa được $f(t)$ xe lên phà cần $t$ giờ.\\
	Tổng thời gian đưa xe qua đảo hoặc từ đảo về là $t+\dfrac{3{,}6}{3\,600}f(t)+1{,}28$ giờ.\\
	Số xe di chuyển trung bình mỗi giờ là $g(t)$, với
	$$g(t)=\dfrac{f(t)}{t+\dfrac{3{,}6}{3\,600}f(t)+1{,}28}=\dfrac{\dfrac{2000t}{2t+1}}{t+\dfrac{3{,}6}{3\,600}\cdot\dfrac{2000t}{2t+1}+1{,}28}=\dfrac{2\,000t}{2t^2+5{,}56t+1{,}28}=\dfrac{2\,000}{2t+\dfrac{1{,}28}{t}+5{,}56}.$$
	Áp dụng bất đẳng thức Côsi cho $2$ số dương $2t$ và $\dfrac{1{,}28}{t}$, ta có $2t + \dfrac{1{,}28}{t} \ge 2\sqrt{2t\cdot\dfrac{1{,}28}{t}}$.\\
	Suy ra $\dfrac{2\,000}{2t + \dfrac{1{,}28}{t} + 5{,}56} \le \dfrac{2\,000}{2\sqrt{2t\cdot\dfrac{1{,}28}{t}} + 5{,}56} = \dfrac{50\,000}{219}$.\\
	Dấu bằng xảy ra khi $2t = \dfrac{1{,}28}{t} \Leftrightarrow t^2 = 0{,}64 \Leftrightarrow t = 0{,}8$.\\
	Vậy để lượng xe trung bình di chuyển qua lại đảo mỗi giờ đạt lớn nhất cần tải lên phà mỗi chuyến $ f(0,8) \approx 615$ xe.
	}
\end{ex}

\begin{ex}%[Nguồn: Bộ đề minh họa Moon 2024-2025]%[2D1V3-6]
	Tại sân vận động Quốc Gia Mỹ Đình có kích thước chiều dài là $100$ m và chiều rộng là $70$ m. Khung thành được đặt chính giữa với chiểu rộng là $8$ m. Cầu thủ Quang Hải muốn đá vào khung thành với quả bóng đặt ở điểm $A$ cách đường biên dọc một khoảng bằng $20$ m (tham khảo hình vẽ).
	\begin{center}
		\begin{tikzpicture}[line join = round, line cap=round,>=stealth,font=\footnotesize,scale=1]
			\coordinate (O) at (0,0);
			\coordinate (A) at (-5.5,-0.75);
			\coordinate (B) at (-6,0);
			\coordinate (C) at (5.5,-0.75);
			\coordinate (D) at (6,0);
			\coordinate (E) at (-4,2.5);
			\coordinate (M) at (-7,0.75);
			\coordinate (N) at (-7,-0.75);
			\draw[fill=green!70!blue]
			(-7,-4) rectangle (7,4);
			\draw[white,line width=2pt] (-7,-1.8) --(-5.5,-1.8)--(-5.5,1.8)--(-7,1.8) (0,-4)--(0,4);
			\draw[white,line width=2pt]  (-7,-0.75)--(-6.5,-0.75)--(-6.5,0.75)--(-7,0.75)
			(7,-0.75)--(6.5,-0.75)--(6.5,0.75)--(7,0.75)
			;
			\draw[red,line width=1pt](-7,-0.75)--   (-4,2.5)--(-7,0.75) (-4,2.5)--(-7,2.5) (-4,2.5)--(-4,4)
			;
			\draw[white,line width=2pt] (7,-1.8) --(5.5,-1.8)--(5.5,1.8)--(7,1.8) ;
			\draw[white,line width=2pt] (O) circle (2 cm);
			\draw[white,line width=2pt]
			let	\p1=($(A)-(B)$),
			\n1={veclen(\x1,\y1)},
			\n2={atan2(\y1,\x1)}
			in (A) arc (\n2:{\n2+110}:\n1);
			\draw[white,line width=2pt]
			let	\p1=($(C)-(D)$),
			\n1={veclen(\x1,\y1)},
			\n2={atan2(\y1,\x1)}
			in (C) arc (\n2:{\n2+-110}:\n1);
			\draw[fill=black]((-4,2.5) circle (1pt) ($(-4,2.5)+(30:3mm)$) node[scale=1]{$A$};
			\draw(-4,2.5)  ($(-4,2.5)+(-30:9mm)$) node[scale=1]{Quang Hải};
			\path (E)--(-7,2.5) node [above,sloped,pos=0.6] {$h$};
			\path (E)--(-4,4) node [left,pos=0.5] {$20$m};
			\draw pic["$\alpha$",draw,angle eccentricity=1.2,angle radius=0.7cm]{angle=M--E--N};
		\end{tikzpicture}
	\end{center}
	Tìm khoảng cách $h$ (m) từ vị tri đặt bóng $A$ đến đường biên ngang để góc sút $\alpha$ (xem hình vẽ) của Quang Hải là lớn nhất. Giả sử sân bóng phẳng (làm tròn kết quả đến hàng phần mười). \shortans[oly]{$14{,}5$}
	\loigiai{
	\begin{center}
		\begin{tikzpicture}[line join = round, line cap=round,>=stealth,font=\footnotesize,scale=1]
			\coordinate (O) at (0,0);
			\coordinate (A) at (-5.5,-0.75);
			\coordinate (B) at (-6,0);
			\coordinate (C) at (5.5,-0.75);
			\coordinate (D) at (6,0);
			\coordinate (E) at (-4,2.5);
			\coordinate (M) at (-7,0.75);
			\coordinate (N) at (-7,-0.75);
			\draw[fill=green!70!blue]
			(-7,-4) rectangle (7,4);
			\draw[white,line width=2pt] (-7,-1.8) --(-5.5,-1.8)--(-5.5,1.8)--(-7,1.8) (0,-4)--(0,4);
			\draw[white,line width=2pt]  (-7,-0.75)--(-6.5,-0.75)--(-6.5,0.75)--(-7,0.75)
			(7,-0.75)--(6.5,-0.75)--(6.5,0.75)--(7,0.75)
			;
			\draw[red,line width=1pt](-7,-0.75)--   (-4,2.5)--(-7,0.75) (-4,2.5)--(-7,2.5) (-4,2.5)--(-4,4)
			;
			\draw[white,line width=2pt] (7,-1.8) --(5.5,-1.8)--(5.5,1.8)--(7,1.8) ;
			\draw[white,line width=2pt] (O) circle (2 cm);
			\draw[white,line width=2pt] (O) circle (2pt);
			\draw[fill=white] (-6,0) circle (1.5pt);
			\draw[fill=white] (6,0) circle (1.5pt);
			\draw[white,line width=2pt]
			let	\p1=($(A)-(B)$),
			\n1={veclen(\x1,\y1)},
			\n2={atan2(\y1,\x1)}
			in (A) arc (\n2:{\n2+110}:\n1);
			\draw[white,line width=2pt]
			let	\p1=($(C)-(D)$),
			\n1={veclen(\x1,\y1)},
			\n2={atan2(\y1,\x1)}
			in (C) arc (\n2:{\n2+-110}:\n1);
			\draw[fill=black]((-4,2.5) circle (1pt) ($(-4,2.5)+(30:3mm)$) node[scale=1]{$A$};
			\draw(-7,2.5) circle (1pt)node[left,scale=1]{$B$};
			\draw(-7,4) circle (1pt)node[left,scale=1]{$M$};
			\draw(-7,-0.75)circle (1pt)node[left,scale=1]{$C$};
			\draw(-7,0.75)circle (1pt)node[left,scale=1]{$D$};
			\draw(-7,0)circle (1pt)node[left,scale=1]{$O$};
			\draw(-4,2.5)  ($(-4,2.5)+(-30:9mm)$) node[scale=1]{Quang Hải};
			\path (E)--(-7,2.5) node [above,sloped,pos=0.6] {$h$};
			\path (E)--(-4,4) node [left,pos=0.5] {$20$m};
			\draw pic["$\alpha$",draw,angle eccentricity=1.2,angle radius=0.7cm]{angle=M--E--N};
		\end{tikzpicture}
	\end{center}
	Ta có $\alpha =\widehat{DAC}$ suy ra $\alpha_{\max}$ khi $\left(\tan\alpha\right)_{\max}$.\\
	Lại có $OM=\dfrac{70}{2}=35$ m, suy ra $MB+BD+DO=35$ m.\\
	Khi đó $BD=35-MB-DO=35-20-4=11$ m.\\
	Suy ra
	\begin{eqnarray*}
		\tan \alpha=\tan\widehat{DAC}&=&\tan\left(\widehat{BAC}-\widehat{BAD}\right)\\
		&=&\dfrac{\tan\widehat{BAC}-\tan\widehat{BAD}}{1+\tan\widehat{BAC}\cdot\tan\widehat{BAD}}\\
		&=& \dfrac{\dfrac{BC}{BA}-\dfrac{BD}{BA}}{1+\dfrac{BC}{BA}\cdot\dfrac{BD}{BA}}\\
		&=& \dfrac{\dfrac{19}{h}-\dfrac{11}{h}}{1+\dfrac{19}{h}\cdot\dfrac{11}{h}}\\
		&=&\dfrac{8}{\dfrac{209}{h}+h}\\
		&\leq& \dfrac{8}{2\sqrt{\dfrac{209}{h}\cdot h}}=\dfrac{4\sqrt{209}}{209}.
	\end{eqnarray*}
	Dấu \lq\lq$=$\rq\rq\, xảy ra khi $h=\dfrac{209}{h}\Leftrightarrow h=\sqrt{209}\approx 14{,}5$.
	}
\end{ex}

\begin{ex}%[Nguồn: Bộ đề minh họa Moon 2024-2025]%[2D1V3-6]
	Từ một miếng bìa có độ dài hạ cánh lần lượt là $0{,}9$ m và $1{,}5$ m như hình bên dưới, bạn Duy cắt đi phần tô màu xám và gắp lại để được một hình hợp chữ nhật.
	\begin{center}
		\begin{tikzpicture}[>=stealth,line join=round,line cap=round,font=\footnotesize,scale=1,thick]
			\draw[fill=gray!60]
			(0,0)rectangle(5,4)
			;
			\draw[fill=white]
			(0,0)rectangle(1,1)
			(0,3)rectangle(1,4)
			;
			\foreach\i/\j in {0/1,1/2,2/3,3/4}{
			\draw[fill=white]
			(\i,1)rectangle(\j,3);
			}
			\path
			(0,0)--(0,1)node[pos=.5,left]{$x$ (m)}
			(0,0)--(1,0)node[pos=.5,below]{$x$ (m)}
			(0,4)--(0,3)node[pos=.5,left]{$x$ (m)}
			(0,4)--(1,4)node[pos=.5,above]{$x$ (m)}
			;
			\draw[<->]
			(0,-.5)--(5,-.5)node[pos=.5,below]{$1{,}5$ (m)}
			;
			\draw[<->]
			(5.5,0)--(5.5,4)node[pos=.5,right]{$0{,}9$ (m)}
			;
			\path
			(0,4)coordinate (A)(1,4)coordinate (D)(4,3)coordinate (G)
			(1,3)coordinate (F)(1,1)coordinate (E)(4,1)coordinate (H)
			(0,0)coordinate (B)(1,0)coordinate (C)
			;
		\end{tikzpicture}
	\end{center}
	Gọi $V$ là thể tích hình hợp chữ nhật được tạo thành. Tìm $x$ (m) để hình hợp tạo thành có thể tích lớn nhất.
	
	\shortans[oly]{0{,}3}
	\loigiai{
	\begin{center}
		\begin{tikzpicture}[>=stealth,line join=round,line cap=round,font=\footnotesize,scale=1,thick]
			\draw[fill=gray!60]
			(0,0)rectangle(5,4)
			;
			\draw[fill=white]
			(0,0)rectangle(1,1)
			(0,3)rectangle(1,4)
			;
			\foreach\i/\j in {0/1,1/2,2/3,3/4}{
			\draw[fill=white]
			(\i,1)rectangle(\j,3);
			}
			\path
			(0,0)--(0,1)node[pos=.5,left]{$x$ (m)}
			(0,0)--(1,0)node[pos=.5,below]{$x$ (m)}
			(0,4)--(0,3)node[pos=.5,left]{$x$ (m)}
			(0,4)--(1,4)node[pos=.5,above]{$x$ (m)}
			;
			\draw[<->]
			(0,-.5)--(5,-.5)node[pos=.5,below]{$1{,}5$ (m)}
			;
			\draw[<->]
			(5.5,0)--(5.5,4)node[pos=.5,right]{$0{,}9$ (m)}
			;
			\path
			(0,4)coordinate (A)(1,4)coordinate (D)(4,3)coordinate (G)
			(1,3)coordinate (F)(1,1)coordinate (E)(4,1)coordinate (H)
			(0,0)coordinate (B)(1,0)coordinate (C)
			;
			\foreach \point/\goc in {A/130,D/45,F/45,G/90,E/-45,H/-90,B/210,C/-90}{
			\draw[fill=black](\point)circle(.8pt)+(\goc:2mm)node[scale=.8]{$\point$};
			}
		\end{tikzpicture}
	\end{center}
	Đặt các điểm như  hình vẽ trên. Khi đó ta có $EF = DC - DF - EC = 0{,}9 - 2x $ (m).\\
	Lúc này, khi miếng bìa được gấp vào thành hình hộp chữ nhật có chiều cao là $x$ (m), chiều rộng đáy là $x$ (m) và chiều dài đáy là $0{,}9 - 2x$ (m).\\
	Suy ra  $	V = x^2 \cdot (0{,}9 - 2x) \, (\text{m}^3)$.\\
	Xét hàm số $V(x) = x^2 \cdot (0{,}9 - 2x)$.\\
	Khi đó $V'(x) = -6x^2 + 1{,}8x$.\\
	Do đó $V'(x) = 0 \Leftrightarrow  -6x^2 + 1{,}8x = 0 \Leftrightarrow  x = 0$ hoặc  $x = 0{,}3$.
	Mà điều kiện $0 < x < \dfrac{0{,}9}{2} = 0{,}45$ nên $x = 0{,}3$ thỏa mãn điều kiện.\\
	Bảng biến thiên của hàm số $V(x)$ trên khoảng $(0; 0{,}45)$ như sau
	\begin{center}
		\begin{tikzpicture}[>=stealth,line join=round,line cap=round,font=\footnotesize,scale=1]
			\tkzTabInit[nocadre=false,lgt=1.2,espcl=3,deltacl=.55]
			{$x$/0.7, $y'$/0.7, $y$/1.8}
			{$0$,$0{,}3$,$0{,}45$}
			\tkzTabLine{,+,$0$,-}
			\tkzTabVar{-/$0$,+/$0{,}027$,-/$0$}
		\end{tikzpicture}
	\end{center}
	Từ bảng biến thiên, ta có hàm số $V(x)$ đạt giá trị lớn nhất $0{,}027$ tại $x = 0{,}3$.
	}
\end{ex}

%Câu 3

\begin{ex}%[Nguồn: Bộ đề minh họa Moon 2024-2025]%[2D1V3-6]
	Chủ một nhà hàng muốn làm tường rào bao quanh $600$\,m$^2$ đất để làm bãi đỗ xe. Ba cạnh của khu đất sẽ được rào bằng một loại thép với chi phí $14\,000$ đồng một mét, riêng mặt thứ tư do tiếp giáp với mặt bên của nhà hàng nên được xây bằng tường gạch xi măng với chi phí là $28\,000$ đồng mỗi mét. Biết rằng cổng vào của khu đỗ xe là $5$\,m.
	\begin{center}
		%		\includegraphics[scale=0.3]{hinh/de22tlncau3}
	\end{center}
	Tìm chu vi của khu đất khi chi phí nguyên liệu bỏ ra là ít nhất, biết rằng khu đất rào được có dạng hình chữ nhật.
	\shortans[]{$100$}
	\loigiai
	{Gọi $x$\,(m) ($x>5$) là chiều rộng của khu đất.\\
	Chiều dài của khu đất là $\dfrac{600}{x}$\,(m).\\
	Chi phí để làm hàng rào bằng thép là $\left(2\cdot\dfrac{600}{x}+x-5\right)\cdot 14\,000=\left(\dfrac{1\,200}{x}+x-5\right)\cdot 14\,000 $\,(đồng).\\
	Chi phí để làm tường bê tông là $x\cdot28\,000=28\,000x$\,(đồng).\\
	Vậy tổng chi phí để làm tường rào cho bãi đỗ xe là \[T=\left(\dfrac{1200}{x}+x-5\right)\cdot 14\,000 + 28\,000x=\dfrac{16\,800\,000}{x}+42\,000x-70\,000\,\text{(đồng)}.\]\\
	Đặt $f(x)=\dfrac{16\,800\,000}{x}+42\,000x-70\,000$ với $x>5$.\\
	Ta có $f'(x)=-\dfrac{16\,800\,000}{x^2}+42\,000$.
	\begin{eqnarray*}
		& & f'(x)=0\\
		&\Leftrightarrow & -\dfrac{16\,800\,000}{x^2}+42\,000=0\\
		&\Leftrightarrow & \dfrac{16\,800\,000}{x^2}=42\,000\\
		&\Leftrightarrow & x^2=400\\
		&\Leftrightarrow & x=\pm 20.
	\end{eqnarray*}
	Vì $x>5$ nên $x=20$. Ta có $f(20)=1\,610\,000$.\\
	Xét dấu
	\begin{itemize}
		\item $f'(x)>0\Leftrightarrow x>20$.
		\item $f'(x)<0\Leftrightarrow 0<x<20$.
	\end{itemize}
	Bảng biến thiên
	\begin{center}
		\begin{tikzpicture}
			\tkzTabInit[nocadre=true,lgt=1,espcl=3,deltacl=0.5]
			{$x$/0.7,$y'$/0.7,$y$/2}
			{$0$,$20$,$+\infty$}
			\tkzTabLine{,-,0,+,}
			\tkzTabVar{+/$+\infty$,-/$1\,610\,000$,+/$+\infty$}
		\end{tikzpicture}
	\end{center}
	Từ bảng biến thiên ta có $\min\limits f(x)=1\,610\,000$ khi $x=20$.\\
	Vậy chi phí nguyên liệu bỏ ra ít nhất là  $T=1\,610\,000$\,(đồng) \\
	Khi đó chu vi của khu đất là $2\cdot\left(\dfrac{600}{20}+20\right)=100$\,(m).
	}
\end{ex}

\begin{ex}%[Nguồn: Bộ đề minh họa Moon 2024-2025]%[2D1V3-6]
	Người ta cần cắt một tấm tôn có hình dạng là một elip với độ dài trục lớn bằng $10$ , độ dài trục bé bằng $8$ để được một tấm tôn có dạng hình chữ nhật nội tiếp elip. Sau đó gò tấm tôn hình chữ nhật đó để thu được một hình trụ không có đáy như hình vẽ. Thể tích lớn nhất của khối trụ thu được bằng bao nhiêu? (làm tròn kết quả đến hàng phần mười).
	\begin{center}
		\begin{tikzpicture}[>=stealth,line join=round,line cap=round,font=\footnotesize,scale=1]
			%vẽ elip
			\draw
			(0,0) ellipse ({3} and {1.5})
			;
			\fill[gray!40]
			(2,{sqrt(5)/2}) -- (2,{-sqrt(5)/2}) -- (-2,-{sqrt(5)/2}) -- (-2,{sqrt(5)/2}) -- cycle
			;
			%vẽ mũi tên
			\draw
			(4,0.5) -- (5,0.5) -- (5,1) -- (5.75,0) -- (5,-1) -- (5,-0.5) -- (4,-0.5) -- cycle
			;
			%
			%vẽ hình trụ
			
			\draw
			(9,1.5) ellipse ({1.5} and {0.5})
			(10.5,1.5) -- (10.5,-1)
			(7.5,1.5) -- (7.5,-1)
			(10.5,-1) arc(0:-180:{1.5} and {0.5})
			;
			\draw[dashed]
			(10.5,-1) arc(0:180:{1.5} and {0.5})
			;
		\end{tikzpicture}
	\end{center}
	
	\shortans[oly]{$24{,}5$}
	\loigiai{
	\begin{itemize}
		\item Phương trình chính tắc của elip là
		$$\dfrac{x^2}{a^2} + \dfrac{y^2}{b^2} = 1$$
		Với $a = \dfrac{10}{2} = 5$ và $b = \dfrac{8}{2} = 4$, ta có
		$$\dfrac{x^2}{25} + \dfrac{y^2}{16} = 1$$
		\item Hình chữ nhật nội tiếp:
		Gọi $(x, y)$ là tọa độ của một đỉnh của hình chữ nhật nằm trong góc phẩn tư thứ nhất. Các đỉnh còn lại là $(-x, y),(x,-y)$ và $(-x,-y)$. \\
		- Chiều rộng hình chữ nhật: $2 x$
		- Chiều cao hình chữ nhật: $2 y$
		\item Hinh trụ: Khi gò hình chữ nhật thành hình trụ không đáy:\\
		- Chu vi đáy hình trụ: $2 \pi r=2 x \Rightarrow r=\frac{x}{\pi}$
		- Chiều cao hình trụ: $\mathrm{h}=2 \mathrm{y}$
		\item Thể tích hình trụ:
		$$V = \pi r^2 h = \pi \left(\frac{x}{\pi}\right)^2 (2y) = \frac{2x^2y}{\pi}$$
		\item \textbf{Rút $x^2$ theo $y$:}
		$$\frac{x^2}{25} = 1 - \frac{y^2}{16}$$
		$$x^2 = 25\left(1 - \frac{y^2}{16}\right) = 25 - \frac{25y^2}{16}$$
		\item \textbf{Biểu thức thể tích theo $y$:}
		Thay $x^2$ vào biểu thức thể tích:
		$$V(y) = \frac{2}{\pi}\left(25 - \frac{25y^2}{16}\right)y = \frac{50y}{\pi} - \frac{50y^3}{16\pi} = \frac{50}{\pi}\left(y - \frac{y^3}{16}\right)$$
		
		\item Ta có	$$V'(y) = \frac{50}{\pi}\left(1 - \frac{3y^2}{16}\right)$$
		và  $V'(y) = 0\Leftrightarrow 1 - \dfrac{3y^2}{16} = 0\Leftrightarrow \dfrac{3y^2}{16} = 1\Leftrightarrow y^2 = \dfrac{16}{3}$.\\
		Vì $y > 0$ nên $y = \sqrt{\dfrac{16}{3}} = \dfrac{4\sqrt{3}}{3}$.
		\item Bảng biến thiên của $V(y)$:
		\begin{center}
			\begin{tikzpicture}
				\tkzTabInit[espcl=2.5,lgt=1.5,nocadre]
				{$x$/0.7,$y'$/0.7,$y$/2.1}
				{$0$,$\tfrac{4\sqrt{3}}{3}$,$+\infty$}
				\tkzTabLine{,+,0,-,}
				\tkzTabVar{-/$0$,+/$24{,}5$,-/$-\infty$}
			\end{tikzpicture}
		\end{center}
		Vậy thể tích lớn nhất của khối trụ thu được là $24.5$.
	\end{itemize}
	}
\end{ex}

%

\begin{ex}%[Nguồn: Bộ đề minh họa Moon 2024-2025]%[2D1V5-7]
	Cho hàm số $f(x) = x^3 - 6x^2 + 9x - 1$ có đồ thị $(C)$. Gọi $A$, $B$ là hai điểm cực trị của $(C)$.
	\choiceTF
	{\True Tập xác định của hàm số đã cho là $\mathbb{R}$}
	{Hàm số đã cho đồng biến trên khoảng $(1; 3)$}
	{\True Phương trình đường thẳng đi qua hai điểm $A$, $B$ có dạng $x - by + c = 0$ thì $b + c = -3$}
	{Đường thẳng $AB$ tạo trục hoành một góc bằng $45^\circ$}
	\loigiai{\begin{itemchoice}
		\itemch Tập xác định của hàm số là $\mathscr{D}=\mathbb{R}$.
		\itemch Ta có $f'(x)=3x^2-12x+9$.\\
		$\Rightarrow f'(x)=0\Leftrightarrow 3x^2-12x+9=0\Leftrightarrow\hoac{&x=1\\&x=3.}$\\
		Bảng biến thiên
		\begin{center}
			\begin{tikzpicture}[scale=1]
				\tkzTabInit
				{$x$/1,$f'(x) $/1,$f(x)$/2}
				{$-\infty$,$1$,$3$,$+\infty$}
				\tkzTabLine{,+,0,-,0,+,} %
				\tkzTabVar{-/$-\infty$,+/$3$, -/$-1$,+/$+\infty$} %dấu mũi tên, + trên, -dưới
			\end{tikzpicture}
		\end{center}
		Dựa vào bảng biến thiên ta có hàm số nghịch biến trên $(1;3)$.
		\itemch Dựa vào bảng biến thiên ta có đồ thị hàm số có hai điểm cực trị là $A(1;3)$, $B(3;-1)$.\\
		Phương trình đường thẳng đi qua hai đưởng $A$, $B$ có dạng $y=mx+n$.\\
		Khi đó ta có hệ phương trình như sau $\heva{&3=m+n\\&-1=3m+n}\Leftrightarrow \heva{&m=-2\\&n=5.}$\\
		Suy ra phương trình đường thẳng $AB$ là $y=-2x+5$ hay $x+\dfrac{1}{2}y-\dfrac{5}{2}=0$.\\
		Vậy $b=-\dfrac{1}{2}$, $c=-\dfrac{5}{2}$ nên $b+c=-3$.
		\itemch Đường thẳng $AB$ có hệ số góc $k=\tan \alpha=-2$ (với $\alpha$ là góc tạo bởi đường thẳng $AB$ và chiều dương của trục $Ox$).\\
		Suy ra $\alpha\approx 116,6^\circ$.\\
		Vậy góc giữa đường thẳng $AB$ và trục hoành là $63,4^\circ$.
	\end{itemchoice}}
\end{ex}

\begin{ex}%[Nguồn: Bộ đề minh họa Moon 2024-2025]%[2D1V5-8]
	Hai nhà máy được đặt tại các vị trí $A$ và $B$ cách nhau $8$ km. Nhà máy xử lí nước thải được đặt ở vị trí $C$ trên đường trung trực của đoạn thẳng $AB$, cách trung điểm $M$ của đoạn thẳng $AB$ một khoảng là $3$ km. Người ta muốn làm đường ống dẫn nước thải từ hai nhà máy $A$, $B$ đến nhà máy xử lí nước thải $C$ gồm các đoạn thẳng $AI$, $BI$ và $IC$, với $I$ là vị trí nằm giữa $M$ và $C$. Đặt $I M=x$ km (với $0<x<3$).
	\begin{center}
		\begin{tikzpicture}[>=stealth,line cap=round,line join=round]
			\path(-4,0)node[left](A){$A$}
			++(4,0)node[below](M){$ M$} ++(4,0)node[right](B){$ B $}
			(M)++(0,2.2)node[above left](I){$I$}++(0,2.2)node[above](C){$ C $}
			;
			\draw(-4,0)--(4,0) (0,0)--(0,4) (-4,0)--(0,2)(4,0)--(0,2);
		\end{tikzpicture}
	\end{center}
	\choiceTF
	{$I A=I B=\sqrt{x^2+9}$ (km)}
	{ Tổng độ dài đường ống được biểu diễn qua hàm số $f(x)=2 \sqrt{x^2+9}+3-x$ (km)}
	{\True Tổng độ dài đường ống nhỏ nhất bằng $9,9$ (km) (làm tròn kết quả đến hàng phần chục)}
	{\True Khi tổng độ dài đường ống nhỏ nhất thì góc $\widehat{AIB}=120^{\circ}$}
	\loigiai{
	\begin{itemchoice}
		\itemch {\bf Sai.}\\
		Ta có $AM=MB=\dfrac{AB}{2}=4$ (km).\\
		Suy ra $IA=IB=\sqrt{x^2+16}$ (km).
		\itemch {\bf Sai.}\\
		Độ dài đường ống nước là $IA+IB+IC=2\sqrt{x^2+16}+3-x$.\\
		Vì vậy $f(x)=2\sqrt{x^2+16}+3-x$ (km).
		\itemch {\bf Đúng.}\\
		Xét hàm số $f(x)=2\sqrt{x^2+16}+3-x$ với $0<x<3$.\\
		Ta có $f'(x)=\dfrac{2x}{\sqrt{x^2+16}}-1=0\Leftrightarrow \sqrt{x^2+16}=2x\Rightarrow x=\dfrac{4\sqrt{3}}{3}$.\\
		Bảng biến thiên
		\begin{center}
			\begin{tikzpicture}
				\tkzTabInit[lgt=1.2,espcl=4]
				{$x$/1,$f’(x)$/1,$f(x)$/2.5}
				{$0$,$\dfrac{4\sqrt{3}}{3}$,$3$}
				\tkzTabLine{ ,-,z,+, }
				\tkzTabVar{+/$11$,-/$3+4\sqrt{3}$,+/$10$}
			\end{tikzpicture}
		\end{center}
		Suy ra $\min\limits_{(0; 3)}f(x) =f\left(\dfrac{4\sqrt{3}}{3}\right)=3+4\sqrt{3}\approx 9{,}9$ (km).
		\itemch {\bf Đúng.}\\
		Ta có $\tan \widehat{AIM}=\dfrac{AM}{IM}=\dfrac{4}{\tfrac{4\sqrt{3}}{3}}=\sqrt{3}\Rightarrow \widehat{AIM}=60^\circ\Rightarrow \widehat{AIB}=120^\circ$.
	\end{itemchoice}
	}
\end{ex}

% Câu 4

\begin{ex}%[Nguồn: Bộ đề minh họa Moon 2024-2025]%[2D1V5-8]
	Một máy bay đang bay ở độ cao $H$ khi bắt đầu hạ cánh xuống một đường băng cách mặt đất một khoảng $L$ theo phương ngang, như hình vẽ. Giả sử đường băng hạ cánh của máy bay là đồ thị của một hàm đa thức bậc ba $y = ax^3 + bx^2 + cx + d$, trong đó $M(-L; H)$ và $O(0; 0)$ là $2$ điểm cực trị của đồ thị hàm số.
	\begin{center}
		\begin{tikzpicture}[>=stealth,line join=round,line cap=round,font=\footnotesize,thick,yscale=.8,xscale=1.2]
			\tikzset{declare function={x1=-8;x2=2;y1=-1.5;y2=4;}}
			
			\draw[->](x1,0)--(x2,0)node[below,scale=.8]{$x$};
			\draw[->](0,y1)--(0,y2)node[right,scale=.8]{$y$};
			
			\draw[<->](-7,0)--(-7,3.5)node[scale=.8,pos=.4,right]{$H$}node[above]{$M$};
			
			\draw[|<->](-7,-.5)--(0,-.5)node[pos=.5,fill=white]{$L$};
			
			\draw
			(-6,3.4)--(-5.8,4)node[above,scale=.8]{Đường hạ cánh}
			(0,0)coordinate (O)circle(1.2pt)node[below right,scale=.8] {$O$}--+(120:.5)node[above left,scale=.8]{Sân bay};
			
			\draw
			(-8,3.45) .. controls ++(4:3) and ++(145:.8) .. (-3.6, 2.5)..controls++(-40:2) and ++(180:2)..(0,0);
		\end{tikzpicture}
	\end{center}
	\choiceTFt
	{$y' = x^2 + Lx$}
	{$y = H \left[ 4 \left( \dfrac{x}{L} \right)^3 + 6 \left( \dfrac{x}{L} \right)^2 \right]$}
	{\True Tại vị trí $x = -\dfrac{L}{3}$, máy bay có độ cao là $\dfrac{7H}{27}$}
	{Một đường thẳng tiếp xúc với đường băng tại vị trí $x = -\dfrac{L}{3}$ có hệ số góc bằng $-\dfrac{4H}{3}$}
	\loigiai{
	Giả sử hàm số mô tả đường bay là $y = ax^3 + bx^2 + cx + d$.\\
	Từ hình vẽ, ta thấy đồ thị hàm số đi qua gốc tọa độ $O(0,0)$, do đó $d = 0$.\\
	Ta có $y' = 3ax^2 + 2bx + c$.\\
	Hàm số đạt cực trị tại $x = -L$ và $x = 0$, do đó
	\allowdisplaybreaks
	\begin{eqnarray*}
		\heva{&y'(-L) = 0 \\ &y'(0) = 0}
		\Leftrightarrow
		\heva{&3a \cdot 0^2 + 2b \cdot 0 + c = 0 \\ &3a \cdot L^2 - 2b \cdot L + c = 0}
		\Leftrightarrow
		\heva{&c = 0 \\ &b = \dfrac{3L}{2}a.}
	\end{eqnarray*}
	Suy ra $y = ax^3 + \dfrac{3L}{2}ax^2$.\\
	Lại có
	\allowdisplaybreaks
	\begin{eqnarray*}
		y(-L) = H
		&\Leftrightarrow& a \cdot (-L)^3 + \dfrac{3L}{2} \cdot a \cdot L^2 = H \\
		&\Leftrightarrow& -aL^3 + \dfrac{3aL^3}{2} = H \\
		&\Leftrightarrow& \dfrac{aL^3}{2} = H \\
		&\Leftrightarrow& a = \dfrac{2H}{L^3}.
	\end{eqnarray*}
	Do đó, $b = \dfrac{3L}{2}a = \dfrac{3L}{2} \cdot \dfrac{2H}{L^3} = \dfrac{3H}{L^2}$.\\
	Phương trình đường bay của máy bay là $y = \dfrac{2H}{L^3}x^3 + \dfrac{3H}{L^2}x^2$.
	\begin{itemchoice}
		\itemch Ta có $y'= \dfrac{6H}{L^3}x^2+\dfrac{6H}{L^2}x$.
		\itemch Ta có $y = \dfrac{2H}{L^3}x^3 + \dfrac{3H}{L^2}x^2= \dfrac{H}{2}\left[4\left(\dfrac{x}{L}\right)^3 + 6\left(\dfrac{x}{L}\right)^2\right]$.
		\itemch Ta có \allowdisplaybreaks
		\begin{eqnarray*}
			y\left(-\dfrac{L}{3}\right) &=& \dfrac{2H}{L^3} \left(-\dfrac{L}{3}\right)^3 + \dfrac{3H}{L^2} \left(-\dfrac{L}{3}\right)^2 \\
			&=& -\dfrac{2H}{27} + \dfrac{3H}{9} \\
			&=& \dfrac{7H}{27}.
		\end{eqnarray*}
		\itemch Đường thẳng tiếp xúc với đường băng tại vị trí $x = -\dfrac{L}{3}$ có hệ số góc là $$k=y'\left(-\dfrac{L}{3}\right)= \dfrac{6H}{L^3}\left(-\dfrac{L}{3}\right)^2+\dfrac{6H}{L^2}\left(-\dfrac{L}{3}\right)=-\dfrac{4H}{3L}.$$
	\end{itemchoice}
	}
\end{ex}

\begin{ex}%[Nguồn: Bộ đề minh họa Moon 2024-2025]%[2D1V5-8]
	Một hòn đá được ném từ một cây cầu với quỹ đạo ban đầu là $25^\circ$ so với phương nằm ngang. Vào thời điểm nó di chuyển được $x$ (m) theo phương ngang, chiều cao của hòn đá so với mặt nước dưới cầu được cho bởi $h(x) = ax^2 + bx + c$ (mét) với $a$, $b$, $c \in \mathbb{R}$. Hòn đá được ném từ độ cao $3$ mét so với mặt nước và đạt độ cao cực đại khi $x = 5$
	\begin{center}
		\tikzset{nguoi/.pic={
		\begin{scope}[line cap=round,line join=round]
			\draw[fill=black] (0.2,0) circle (0.5cm);
			
			% Thân
			\draw[line width = 3pt] (0.2,-0.5) -- (0,-2);
			% Tay trái (giơ lên để ném)
			\draw[line width = 3pt] (0.2,-0.7) -- (-1.2,-1);
			% Tay phải (hướng về phía trước)
			\draw[line width = 3pt] (0.2,-0.7) -- (0.75,-0.8) -- (1.4,-0.3);
			% Chân trái
			\draw[line width = 3pt] (0,-2) -- (-1.5,-3.1);
			% Chân phải
			\draw[line width = 3pt] (0.1,-2) -- (0.3,-2.5) -- (0.2,-3.1);
		\end{scope}
		}}
		\begin{tikzpicture}[>=stealth,line join=round,line cap=round,font=\footnotesize,scale=1]
			\draw[->]
			(0,0) -- (5,0)node[below]{$x$};
			\draw[->]
			(0,0) -- (0,4)node[right]{$h(x)$};
			\fill
			(0,0) -- (0,2.45) -- (0.5,2.45) -- (0.5,2.9) -- (0.3,2.9) -- (0.3,2.55) -- (-1.5,2.55) -- (-1.5,2.45) -- (-0.5,2.45) -- (-0.5,0) -- (0,0)
			;
			\path
			(-0.3,3.2) pic[scale=.2]{nguoi}
			;
			\draw[samples=100,smooth,domain=0:4] plot(\x,{-(1/3)*(\x)^2 + (2/3)*(\x) + 3});
			\fill[gray!50] plot[samples=100,smooth,domain=-1.5:5] (\x,{(1/6)*sin((\x*180)/2*pi)-0.2}) -- (5,{(1/6)*sin(450/pi)-0.2}) -- (5,-0.7) -- (-1.5,-0.7) -- cycle;
			
			\draw[->] (0,3) -- (1.5,3.8);
			\draw[dashed] (0,3) -- (1.5,3);
		\end{tikzpicture}
	\end{center}
	\choiceTF
	{\True Giá trị $c = 3$}
	{\True Đạo hàm của $h(x)$ là $h'(x) = 2ax + b$}
	{Độ cao lớn nhất của hòn đá so với mặt nước là $4{,}16$ mét (làm tròn kết quả đến hàng phần trăm)}
	{Hòn đá rơi xuống nước tại vị trí $x = 4{,}45$ mét (làm tròn kết quả đến hàng phần trăm)}
	\loigiai{
	\begin{itemchoice}
		\itemch \textbf{Đúng}. Ta có $h(0) = 3 \Leftrightarrow c = 3$, khi đó $h(x) = ax^2 + bx + c$ (mét).\\
		Mặt khác hòn đá đạt độ cao cực đại khi $x = 5$ suy ra $\dfrac{-b}{2a} = 5 \Rightarrow a = \dfrac{-1}{10}b$.\\
		Mà ban đầu hòn đá được ném với góc $25^\circ$ so với phương nằm ngang.\\
		Suy ra $f'(0) = \tan{25^\circ} \Leftrightarrow b = \tan{25^\circ}$.\\
		Từ đó ta được $a = \dfrac{-1}{10} \cdot \tan{25^\circ}$.\\
		Suy ra $h(x) = \dfrac{-\tan{25^\circ}}{10}  x^2 + \tan{25^\circ} x+ 3$.
		\itemch \textbf{Đúng}. Ta có $h'(x) = 2ax + b = -\dfrac{\tan{25^\circ}}{5}x + \tan{25^\circ}$.
		\itemch \textbf{Sai}. Hòn đá đạt độ cao lớn nhất so với mặt nước khi $x = 5$, khi đó ta có\\
		$h(5) = \dfrac{-\tan{25^\circ}}{10}
		\cdot 5^2 + \tan{25^\circ}  \cdot 5+ 3 \approx 4{,}17$ mét.
		\itemch \textbf{Sai}. Xét $h(x) = 0 \Leftrightarrow \dfrac{-\tan{25^\circ}}{10}  x^2 + \tan{25^\circ} x+ 3 \Leftrightarrow \hoac{&x \approx -4{,}45 \quad \text{(loại)}\\&x \approx 14{,}45 \quad \text{(thỏa mãn)}.}$
		Vậy hòn đá rơi xuống mặt nước tại vị trí $x \approx 14{,}45$ mét.
	\end{itemchoice}
	}
\end{ex}

\begin{ex}%[Nguồn: Bộ đề minh họa Moon 2024-2025]%[2D1V5-8]
	Giả sử cường độ ánh sáng của một nguồn điểm tỉ lệ thuận với cường độ của nguồn sáng đó và tỉ lệ nghịch với bình phương khoảng cách từ điểm đó đến nguồn sáng. Hai nguồn điểm có cường độ lần lượt là $S$ và $8S$, cách nhau $90$ cm. Xét một điểm $M$ nằm trên đoạn thẳng nối hai nguồn, cường độ ánh sáng tại điểm đó nhỏ nhất thì điểm đó cách nguồn có cường độ $S$ bằng bao nhiêu centimet? (cho biết cường độ sáng tại điểm $M$ bằng tổng cường độ sáng mỗi nguồn tại điểm đó).
	\begin{center}
		\begin{tikzpicture}[scale=1, font=\footnotesize,line join=round, line cap=round, >=stealth]
			%			\draw[gray,xstep = 1, ystep = 1] (0,0) grid (5,5);
			\path
			(3,0) coordinate (M) node[above] {$M$}
			;
			\node[circle, line width = .2 mm, draw = black, anchor = center, minimum size = 1cm] (D1) at (0,0) {};
			\node[circle, line width = .2 mm, draw = black, anchor = center, minimum size = 1cm] (D2) at (9,0) {};
			
			\draw (D1.0)--(D2.180)
			(D1.45)--(D1.-135)
			(D1.135)--(D1.-45)
			(D2.45)--(D2.-135)
			(D2.135)--(D2.-45)
			;
			\fill
			(D1.center) circle (2pt)
			(D2.center) circle (2pt)
			(M) circle (2pt)
			;
			\draw[stealth-stealth] (0,1) -- (9,1) node[midway,above] {$90$ cm};
		\end{tikzpicture}
	\end{center}
	\par\shortans{30}
	\loigiai{
	Gọi $I$ là cường độ ánh sáng. Vì $I$ tỉ lệ thuận với cường độ của nguồn sáng và tỉ lệ nghịch với bình phương khoảng cách từ điểm đó đến nguồn sáng nên $I=k\dfrac{S}{r^2}$.\\
	Giả sử điểm $M$ nằm cách nguồn sáng $S$ (bên trái) một khoảng là $x$ thì $M$ cách nguồn sáng $8S$ một khoảng là $90-x$.\\
	Ta có cường độ ánh sáng tại điểm $M$ do nguồn sáng $1$ gây ra là $I_1=k\dfrac{S}{x^2}$.\\
	Tương tự cường độ ánh sáng tại điểm $M$ do nguồn sáng $1$ gây ra là $I_1=k\dfrac{8S}{(90-x)^2}$.\\
	Vậy cường độ sáng tổng hợp lại $M$ là
	$$I(x)=I_1+I_2=k\dfrac{S}{x^2}+k\dfrac{8S}{(90-x)^2}$$
	Ta có $I'(x)=-k\dfrac{2Sx}{x^4}-k\dfrac{-16S(90-x)}{(90-x)^4}=kS\left(\dfrac{16}{(90-x)^3}-\dfrac{2}{x^3}\right)$.\\
	$I'(x)=0 \Rightarrow (90-x)^3=8x^3 \Rightarrow 90-x=2x \Rightarrow x=30$ cm.\\
	
	}
\end{ex}

\begin{ex}%[Nguồn: Bộ đề minh họa Moon 2024-2025]%[2D3H1-3]
	Kết quả khảo sát năng suất (đơn vị: tấn/ha) của một số thửa ruộng được minh họa ở biểu đồ sau:
	\begin{center}
		\begin{tikzpicture}[line join=round, line cap=round,>=stealth,thick,scale=0.8]
			\tikzset{every node/.style={scale=0.9}}
			\draw[->] (0,0)--(9,0);
			\draw (12,0)node[below left] {Năng suất (tấn/ha)};
			\draw[->] (0,0)--(0,7) node[left] {Số thửa ruộng};
			\draw (5,6.7)node[above]{Năng suất lúa của một số thửa ruộng};
			\foreach \x in {1,...,6}{\draw (0,\x)node[left]{$\x$}--(9,\x);}
			\node[rotate=55] at (1.2,-1) {$[5.5,5.7)$};
			\node[rotate=55] at (2.2,-1) {$[5.7,5.9)$};
			\node[rotate=55] at (3.2,-1) {$[5.9,6.1)$};
			\node[rotate=55] at (4.2,-1) {$[6.1,6.3)$};
			\node[rotate=55] at (5.2,-1) {$[6.3,6.5)$};
			\node[rotate=55] at (6.2,-1) {$[6.5,6.7)$};
			\draw[fill = blue!80,opacity=0.5] (1,0) rectangle (2,3);
			\draw[fill = blue!80,opacity=0.5] (2,0) rectangle (3,4);
			\draw[fill = blue!80,opacity=0.5] (3,0) rectangle (4,6);
			\draw[fill = blue!80,opacity=0.5] (4,0) rectangle (5,5);
			\draw[fill = blue!80,opacity=0.5] (5,0) rectangle (6,5);
			\draw[fill = blue!80,opacity=0.5] (6,0) rectangle (7,2);
		\end{tikzpicture}
	\end{center}
	Lập bảng tần số ghép nhóm, ta tính được khoảng tứ phân vị của mẫu số liệu trên gần bằng giá trị nào dưới đây nhất?
	\choice
	{$0{,}3$}
	{$0{,}4$}
	{\True $0{,}5$}
	{$0{,}6$}
	\loigiai{Từ biểu đồ, ta có bảng tần số ghép nhóm của mẫu số liệu như sau:
	\begin{center}
		\begin{tabular}{|c|c|c|c|c|c|c|}
			\hline
			Năng suất (tấn/ha) &$[5{,}5;5{,}7)$& $[5{,}7;5{,}9)$ &$[5{,}9; 6{,}1)$ & $[6{,}1; 6{,}3)$ &$[6{,}3; 6{,}5)$ &$[6{,}5; 6{,}7)$\\
			\hline
			Giá trị đại diện (tấn/ha) & $5{,}6$ & $5{,}8$ & $6{,}0$ & $6{,}2$ & $6{,}4$ & $6{,}6$\\
			\hline
			Tần số & $3$ & $4$ & $6$ & $5$ & $5$ & $2$\\
			\hline
		\end{tabular}
	\end{center}
	Cỡ mẫu $n=25$.\\
	Gọi $x_1;\ldots;x_{25}$ là mẫu số liệu gốc về năng suất của một số thửa ruộng được khảo sát được xếp theo thứ tự không giảm.\\
	Ta có \begin{eqnarray*}
		&&x_1; x_2; x_3\in [5{,}5; 5{,}7).\\
		&&x_4;\ldots;x_7\in [5{,}7; 5{,}9).\\
		&&x_8;\ldots;x_{13}\in [5{,}9; 6{,}1).\\
		&&x_{14};\ldots;x_{18}\in [6{,}1; 6{,}3).\\
		&&x_{19};\ldots;x_{23}\in [6{,}3; 6{,}5).\\
		&&x_{24};{x_{25}}\in [6{,}5; 6{,}7).
	\end{eqnarray*}
	Tứ phân vị thứ nhất của mẫu số liệu gốc là $\dfrac{x_6+x_7}{2}\in [5{,}7; 5{,}9)$.\\
	Do đó, tứ phân vị thứ nhất của mẫu số liệu ghép nhóm là $Q_1=5{,}7+\dfrac{\dfrac{25}{4}-3}{4}\cdot \left(5{,}9-5{,}7\right)=5{,}8625$.\\
	Tứ phân vị thứ ba của mẫu số liệu gốc là $\dfrac{x_{19}+x_{20}}{2}\in [6{,}3; 6{,}5)$.\\
	Do đó, tứ phân vị thứ ba của mẫu số liệu ghép nhóm là\\
	\[Q_3=6{,}3+\dfrac{\dfrac{3\cdot 25}{4}-\left(3+4+6+5\right)}{5}\cdot \left(6{,}5-6{,}3\right)=6{,}33.\]
	Khoảng tứ phân vị của mẫu số liệu ghép nhóm là $\Delta_Q=Q_3-Q_1=6{,}33-5{,}8625=0{,}4675$.
	}
\end{ex}

\begin{ex}%[Nguồn: Bộ đề minh họa Moon 2024-2025]%[2D3H1-3]
	Một vườn thú ghi lại tuổi thọ (đơn vị: năm) của $20$ con hổ và thu được kết quả như sau
	\begin{center}
		\begin{tabular}{|l|c|c|c|c|c|}
			\hline
			Tuổi thọ, & {$[14; 15)$} & {$[15; 16)$} & {$[16; 17)$} & {$[17; 18)$} & {$[18; 19)$} \\
			\hline
			Số con hổ & 1 & 3 & 8 & 6 & 2 \\
			\hline
		\end{tabular}
	\end{center}
	Nhóm chứa tứ phân vị thứ nhất là
	\choice
	{$[14;15)$}
	{$[15;16)$}
	{\True $[16;17)$}
	{$[17;18)$}
	\loigiai{
	Sắp xếp mẫu số liệu theo thứ tự không giảm $x_1$; $x_2$; $\ldots$; $x_{20}$.\\
	Tứ phân vị thứ nhất $Q_1=\dfrac{x_5+x_6}{2}\in [16;17]$.\\
	Vậy nhóm chứa tứ phân vị thứ nhất là $[16;17)$.
	}
\end{ex}

\begin{ex}%[Nguồn: Bộ đề minh họa Moon 2024-2025]%[2D3H1-3]
	Kết quả điều tra về số giờ làm thêm trong $1$ tuần của một nhóm sinh viên được cho ở bảng sau:
	\begin{center}
		\begin{tikzpicture}[thick, line join=round, line cap=round, ybar, >=stealth, x=1mm, y=.3mm, scale=1.2]
			\color{black}
			\def\rong{10mm}
			\def\kcach{10}
			\def\cdai{65}
			\def\ccao{195}
			\def\bdau{5}
			\foreach \x/\xtext in {5/[2;4), 15/[4;6), 25/[6;8), 35/[8;10), 45/[10;12)}{
			\draw[thin, shift={(\x, 0)}] (0pt, 2pt)--(0pt,-2pt);
			\draw(\x, 0) node[below] {\footnotesize\color{black} \xtext};
			}
			\foreach \y/\ytext in {60/12, 100/20, 185/37, 105/21, 50/10}{
			\draw[thin, shift={(0, \y)}] (2pt, 0pt)--(-2pt, 0pt);
			\draw(0, \y) node[left] {\tiny\color{black} \ytext};
			}
			\draw[pattern=north east lines, pattern color=green, bar width=\rong]plot coordinates{(5, 60)};
			\draw[pattern=north east lines, pattern color=green, bar width=\rong]plot coordinates{(15, 100)};
			\draw[pattern=north east lines, pattern color=green, bar width=\rong]plot coordinates{(25, 185)};
			\draw[pattern=north east lines, pattern color=green, bar width=\rong]plot coordinates{(35, 105)};
			\draw[pattern=north east lines, pattern color=green, bar width=\rong]plot coordinates{(45, 50)};
			\draw(5, 60) node[above]{\color{black}12};
			\draw(15, 100) node[above]{\color{black}20};
			\draw(25, 185) node[above]{\color{black}37};
			\draw(35, 105) node[above]{\color{black}21};
			\draw(45, 50) node[above]{\color{black}10};
			\draw[->](-5, 0)--(\cdai, 0)node[below]{\footnotesize\color{black} (Số giờ)};
			\draw[->](0,-4)--(0, \ccao)node[above]{\color{black}\footnotesize\color{black} (Số sinh viên)};
			\node[below left](0, 0){\footnotesize\color{black} $O$};
		\end{tikzpicture}
	\end{center}
	Khoảng tứ phân vị của mẫu số liệu ghép nhóm cho bởi biểu đồ trên gần nhất với giá trị nào dưới đây?
	\choice
	{\True $3{,}2$}
	{$3{,}7$}
	{$5{,}3$}
	{$5{,}8$}
	\loigiai{
	Ta có $n=12+20+37+21+10=100$.\\
	Tứ phân vị thứ nhất là $\dfrac{1}{2}(x_{25}+x_{26})$.\\
	Do $x_{25}$ và $x_{26}$ thuộc nhóm $[4;6)$ nên tứ phân vị thứ nhất là
	$$Q_1=4+\dfrac{\dfrac{100}{4}-12}{20}\cdot(6-2)=5{,}3.$$
	Tứ phân vị thứ ba là $\dfrac{1}{2}(x_75+x_76)$.\\
	Do $x_{75}$ và $x_{76}$ thuộc nhóm $[8;10)$ nên tứ phân vị thứ nhất là
	$$Q_3=8+\dfrac{\dfrac{100\cdot 3}{4}-(12+20+37)}{21}\cdot(10-8)=\dfrac{60}{7}.$$
	Vậy $\Delta Q=Q_3-Q_1=\dfrac{60}{7}-5{,}3\approx3{,}2$.
	}
\end{ex}

\begin{ex}%[Nguồn: Bộ đề minh họa Moon 2024-2025]%[2D3H1-3]
	Chiều cao của các bạn học sinh nữ của lớp $12B$ được ghi lại ở bảng sau
	\begin{center}
		\begin{tabular}{|w{c}{3cm}|w{c}{2.5cm}|w{c}{2.5cm}|w{c}{2.5cm}|w{c}{2.5cm}|}
			\hline
			Chiều cao (cm) & $\left[155; 160 \right)$ & $\left[160; 165 \right)$ & $\left[165; 170 \right)$ & $\left[170; 175 \right)$ \\
			\hline
			Số học sinh & $2$ & $5$ & $8$ & $7$ \\
			\hline
		\end{tabular}
	\end{center}
	Khoảng tứ phân vị của mẫu số liệu ghép nhóm trên gần giá trị nào nhất sau đây?
	\choice
	{$5{,}5$}
	{$6{,}5$}
	{\True $7{,}5$}
	{$8{,}5$}
	\loigiai
	{
	Ta có cỡ của mẫu số liệu là $n = 2 + 5 + 8 + 7 = 22$.\\
	Gọi $x_1, x_2, \ldots x_{22}$ là các giá trị của mẫu số liệu đã sắp xếp theo thứ tự tăng dần.\\
	Khi đó $Q_1 = x_6 \in \left[160; 165 \right)$ và $Q_3 = x_{17} \in \left[170; 175 \right)$.\\
	Ta được \begin{align*}
		Q_1 = 160 + \dfrac{\dfrac{22}{4} - 2}{5} \cdot \left(165 - 160 \right) = 163{,}5.\\
		Q_3 = 170 + \dfrac{\dfrac{22 \cdot 3}{4} - 15}{7} \cdot \left(175 - 170 \right) = \dfrac{2395}{14}.
	\end{align*}
	Khoảng tứ phân vị của mẫu số liệu trên là $\Delta Q = Q_3 - Q_1 \approx 7{,}57$
	}
\end{ex}

\begin{ex}%[Nguồn: Bộ đề minh họa Moon 2024-2025]%[2D3H1-4]
	Khảo sát thời gian tập thể dục trong ngày của một số học sinh khối 11 thu được mẫu số liệu ghép nhóm sau
	\begin{center}
		\begin{tabular}{|c|c|c|c|c|c|}
			\hline
			Thời gian (phút) & $[0; 20)$ & $[20; 40)$ & $[40; 60)$ & $[60; 80)$ & $[80; 100)$ \\
			\hline
			Số học sinh ($f$) & $5$ & $9$ & $12$ & $10$ & $6$ \\
			\hline
		\end{tabular}
	\end{center}
	Tính $9Q_{1} - Q_{3}$.
	\choice
	{$217$}
	{\True $219$}
	{$220$}
	{$218$}
	\loigiai{
	Ta có bảng
	\begin{center}
		\begin{tabular}{|c|c|c|c|c|c|}
			\hline
			Thời gian (phút) & $[0; 20)$ & $[20; 40)$ & $[40; 60)$ & $[60; 80)$ & $[80; 100)$ \\
			\hline
			Số học sinh ($f$) & $5$ & $9$ & $12$ & $10$ & $6$ \\
			\hline
			Tần số tích lũy ($CF$) & $5$ & $14$ & $26$ & $36$ & $42$ \\ % Bảng tần số tích lũy
			\hline
		\end{tabular}
	\end{center}
	Cỡ mẫu $n=5+9+12+10+6=42$.\\
	Ta có $\dfrac{n}{4}=\dfrac{42}{4}=10{,}5$ nên nhóm chứa tứ phân vị thứ nhất  $Q_1$ là $[20; 40)$.\\
	$Q_1=20+\dfrac{10{,}5-5}{9} \cdot 20=20+\dfrac{5{,}5}{9} \cdot 20 \approx 32{,}22$.	\\
	Ta có $\dfrac{3n}{4}=\dfrac{3 \cdot 42}{4}=31{,}5$ nên nhóm chứa tứ phân vị thứ  ba $Q_3$ là $[60; 80)$.\\
	$Q_3=60+\dfrac{31{,}5-(5+9+12)}{10} \cdot 20=60+\dfrac{5{,}5}{10} \cdot 20=60+11=71$.\\
	Vậy $9Q_1-Q_3=9 \cdot 32{,}22-71 \approx 289{,}98-71 \approx 219$.
	}
\end{ex}

%

\begin{ex}%[Nguồn: Bộ đề minh họa Moon 2024-2025]%[2D3H1-4]
	Một cuộc khảo sát đã tiến hành xác định tuổi (theo năm) của 120 chiếc ô tô. Kết quả điều tra được cho trong bảng sau:\\
	\centerline{
	\begin{tabular}{|l|c|c|c|c|c|}
		\hline
		Số tuổi (theo năm) & {$[0;4)$} & {$[4;8)$} & {$[8;12)$} & {$[12;16)$} & {$[20;24)$} \\
		\hline
		Số ô tô & $23$ & $25$ & $27$ & $26$ & $19$ \\
		\hline
	\end{tabular}}
	Có bao nhiêu ô tô có độ tuổi dưới $12$ năm?
	\choice
	{$26$}
	{$37$}
	{$45$}
	{\True $75$}
	\loigiai{
	Số ô tô có độ tuổi dưới $12$ năm là $23+25+27=75$.}
\end{ex}

%

\begin{ex}%[Nguồn: Bộ đề minh họa Moon 2024-2025]%[2D3H2-2]
	Hai mẫu số liệu ghép nhóm $M_1$, $M_2$ có bảng tần số ghép nhóm như sau:
	
	\begin{center}
		\begin{tabular}{c c}
			$M_1$
			&
			\begin{tabular}{|c|c|c|c|c|c|}
			\hline
			Nhóm & {$[8; 10)$} & {$[10; 12)$} & {$[12; 14)$} & {$[14; 16)$} & {$[16; 18)$} \\
			\hline
			Tần số & 3 & 4 & 8 & 6 & 4 \\
			\hline
		\end{tabular}
		\\
		\end{tabular}
		
		\,\\
		
		\begin{tabular}{c c}
			$M_2$
			&
			\begin{tabular}{|c|c|c|c|c|c|}
			\hline
			Nhóm & {$[8; 10)$} & {$[10; 12)$} & {$[12; 14)$} & {$[14; 16)$} & {$[16; 18)$} \\
			\hline
			%		\cline {2-6}
			Tần số & 6 & 8 & 16 & 12 &8\\
			\hline
		\end{tabular}
		\\
		\end{tabular}
	\end{center}
	Gọi $s_1, s_2$ lần lượt là độ lệch chuẩn của mẫu số liệu ghép nhóm $M_1, M_2$. Phát biểu nào sau đây là đúng?
	\choice
	{\True $s_1=s_2$}
	{$s_1=2s_2$}
	{$2s_1=s_2$}
	{$4s_1=s_2$}
	\loigiai{
	Ta có giá trị đại diện của các nhóm lần lượt là $9$, $11$, $13$, $15$, $17$.\\
	Số trung bình của mẫu số liệu $M_1 $ là
	$$
	\overline{x}_1 =\dfrac{3 \cdot 9+4 \cdot 11+8 \cdot 13+6 \cdot 15+4 \cdot 17}{25}=13{,}32 .
	$$
	Phương sai của mẫu số liệu $M_1 $ là
	$$
	s_1 ^{2}=\dfrac{1}{25}\left[3 \cdot 9^{2}+4 \cdot 11^{2}+8 \cdot 13^{2}+6 \cdot 15^{2}+4 \cdot 17^{2}\right]-13{,}32^{2}=5{,}9776
	$$
	Số trung bình của mẫu số liệu $M_2 $ là
	$$
	\overline{x}_2 =\dfrac{6 \cdot 9+8 \cdot 11+16 \cdot 13+12 \cdot 15+8 \cdot 17}{50}=13{,}32
	$$
	Phương sai của mẫu số liệu $M_2 $ là
	$$
	s_2 ^{2}=\dfrac{1}{50}\left[6 \cdot 9^{2}+8 \cdot 11^{2}+16 \cdot 13^{2}+12 \cdot 15^{2}+8 \cdot 17^{2}\right]-13,32^{2}=5{,}9776 .
	$$
	Vậy $s_1 ^{2}=s_2 ^{2} \Leftrightarrow s_1 =s_2 $.
	Vậy hai mẫu có độ lệch chuẩn bằng nhau.
	}
\end{ex}

\begin{ex}%[Nguồn: Bộ đề minh họa Moon 2024-2025]%[2D3H2-2]
	Nhiệt độ trong 55 ngày của một địa phương được cho trong bảng ghép lớp sau:
	\begin{center}
		\begin{tabular}{|l|c|c|c|c|c|c|}
			\hline
			Nhiệt độ $\left(^{\circ} C\right)$ & {$[19; 22)$} & {$[22; 25)$} & {$[25; 28)$} & {$[28; 31)$} & {$[31; 34)$} & {$[34; 37)$} \\
			\hline
			Số ngày & 5 & 7 & 8 & 16 & 12 & 7 \\
			\hline
		\end{tabular}
	\end{center}
	Phương sai của mẫu số liệu được làm tròn đến chữ số thập phân thứ nhất nằm trong khoảng
	\choice
	{$(17; 19)$}
	{$(20; 21)$}
	{\True$(19; 20)$}
	{$(23; 25)$}
	\loigiai{
	Cỡ mẫu là $n=55$.\\
	Số trung bình của mẫu số liệu ghép nhóm là
	$$\overline{x}=\dfrac{1}{55}\left(5\cdot 20{,}5+7\cdot 23{,}5+8\cdot 26{,}5+16\cdot 29{,}5+12\cdot 32{,}5+7\cdot 35{,}5\right)=28{,}9.$$
	Phương sai của mẫu số liệu ghép nhóm là
	$$S^2=\dfrac{1}{55}\left(5\cdot 20{,}5^2+7\cdot 23{,}5^2+8\cdot 26{,}5^2+16\cdot 29{,}5^2+12\cdot 32{,}5^2+7\cdot 35{,}5^2\right)-(28{,}9)^2=19{,}4.$$
	Vậy phương sai của mẫu số liệu trên thuộc khoảng $(19; 20)$.
	}
\end{ex}

%

\begin{ex}%[Nguồn: Bộ đề minh họa Moon 2024-2025]%[2D3N1-1]
	Kết quả khảo sát cân nặng số táo ở lô hàng $B$ được cho ở bảng sau
	\begin{center}
		\begin{tabular}{|c|c|c|c|c|c|}
			\hline
			Cân nặng (g) & {$[150; 155)$} & {$[155; 160)$} & {$[160; 165)$} & {$[165; 170)$} & {$[170; 175)$} \\
			\hline
			Số quả táo ở lô hàng $B$ & 1 & 3 & 7 & 10 & 4 \\
			\hline
		\end{tabular}
	\end{center}
	Số táo được khảo sát trong bảng số liệu là
	\choice
	{$6$}
	{\True $25$}
	{$7$}
	{$5$}
	\loigiai{
	Số táo	được khảo sát trong bảng số liệu là $n=1+3+7+10+4=25$.
	}
\end{ex}

\begin{ex}%[Nguồn: Bộ đề minh họa Moon 2024-2025]%[2D3N1-1]
	Khảo sát thời gian xem ti vi trong một ngày của một số học sinh khối 12 thu được mẫu số liệu ghép nhóm sau
	\begin{center}
		\begin{tabular}{|l|c|c|c|c|c|}
			\hline
			Thời gian (phút) & {$[0; 20)$} & {$[20; 40)$} & {$[40; 60)$} & {$[60; 80)$} & {$[80; 100)$} \\
			\hline
			Số học sinh & $5$ & $9$ & $12$ & $10$ & $6$\\
			\hline
		\end{tabular}
	\end{center}
	Số học sinh xem ti vi từ $60$ phút đến dưới $80$ phút là
	\choice
	{$5$}
	{$12$}
	{$9$}
	{\True$10$}
	\loigiai{Dựa vào bảng trên ta thấy số học sinh thuộc nửa khoảng $[60 ; 80)$ là $10$ .}
\end{ex}

%\subsection{Câu trắc nghiệm nhiều phương án lựa chọn}tn
\cautn

\begin{ex}%[Nguồn: Bộ đề minh họa Moon 2024-2025]%[2D3N1-1]
	Cho mẫu số liệu ghép nhóm có bảng tần số ghép nhóm như sau:
	\begin{center}
		\begin{tabular}{|c|c|c|c|c|}
			\hline Nhóm & {$[0 ; 30)$} & {$[30 ; 60)$} & {$[60 ; 90)$} & {$[90 ; 120)$} \\
			\hline Tần số & $5$ & $9$ & $12$ & $8$ \\
			\hline
		\end{tabular}
	\end{center}
	Giá trị đại diện của nhóm $[60 ; 90)$ là
	\choice
	{$65$}
	{\True $75$}
	{$90$}
	{$60$}
	\loigiai{
	Giá trị đại diện của nhóm $[60; 90)$ cho bởi $\dfrac{60+90}{2}=75$.
	}
\end{ex}

\begin{ex}%[Nguồn: Bộ đề minh họa Moon 2024-2025]%[2D3N1-4]
	Cân nặng của một số quả mít trong khu vườn được thống kê ở bảng sau
	\begin{center}
		\begin{tabular}{|l|c|c|c|c|c|}
			\hline Cân nặng (kg)&{$[4;6)$}&{$[6;8)$}&{$[8;10)$}&{$[10;12)$}&{$[12;14)$}\\
			\hline Số quả mít&$6$&$12$&$19$&$9$&$4$\\
			\hline
		\end{tabular}
	\end{center}
	Số quả mít có cân nặng ít hơn $10$ kg trong bảng trên là
	\choice
	{$19$}
	{$46$}
	{$40$}
	{\True $37$}
	\loigiai
	{
	Từ bảng số liệu ta có số quả mít có cân nặng ít hơn $10$ kg là $$6+12+19=37.$$
	}
\end{ex}

%

\begin{ex}%[Nguồn: Bộ đề minh họa Moon 2024-2025]%[2D3N2-2]
	Một mẫu số liệu ghép nhóm về chiều cao của một lớp (đơn vị là centimét) có phương sai là $6,25$. Độ lệch chuẩn của mẫu số liệu đó bằng
	\choice
	{\True$2{,}5$ cm}
	{$12{,}5$ cm}
	{$3{,}125$ cm}
	{$42{,}25$ cm}
	\loigiai{Độ lệch chuẩn của mẫu số liệu đã cho là $s=\sqrt{6{,}25}=2{,}5$ cm.}
\end{ex}

%

\begin{ex}%[Nguồn: Bộ đề minh họa Moon 2024-2025]%[2D4C1-6]
	Tốc độ trao đổi chất cơ bản của sinh vật có thể tăng hoặc giảm tùy thuộc vào hoạt động của sinh vật. Cụ thể, sau khi hấp thụ chất dinh dưỡng, sinh vật thường trải qua một sự tăng đột biến trong tốc độ trao đổi chất của nó, sau đó dần dần trở lại mức cơ bản.
	Linh vừa kết thúc bữa tối trong buổi sinh nhật của mình và nạp vào mức năng lượng là $5\,120$ J. Sau đó cô đã tiêu hao hết số năng lượng đó trong vòng $12$ giờ tiếp theo. Giả sử $t$ giờ sau bữa ăn, Linh tiêu hao được $M(t) k J$ thì tốc độ tiêu hao năng lượng của cô được mô phỏng bởi hàm số $M'(t)=M_0+t \mathrm{e}^{-0{,}1t^2}$ (kJ/h), $t \in[0; 12]$.
	\choiceTF
	{\True $M(t)=M_0 t-5\mathrm{e}^{-0{,}1t^2}+C$ với $C$ là hằng số}
	{\True $M_0=0{,}01$ (làm tròn kết quả đến hàng phần trăm)}
	{Năng nượng còn lại sau $6$ giờ đầu là $197$ J (làm tròn kết quả đến hàng đơn vị)}
	{Tốc độ tiêu hao năng lượng trung bình trong khoảng thời gian từ $a$ (giờ) đến $b$ (giờ) được tính bởi công thức $v_{t b}=\dfrac{M(b)-M(a)}{b-a}$. Tốc độ tiêu hao năng lượng trung bình từ $6$ giờ đến $12$ giờ của Linh là $32{,}76$ J/h (làm tròn kết quả đến hàng phần trăm)}
	\loigiai{
	\begin{itemchoice}
		\itemch Ta có $M'(t)=\left(M_0 t-5\mathrm{e}^{-0{,}1t^2}+C\right)' = M_0+t\mathrm{e}^{-0{,}1t^2}$.\\
		Suy ra $\displaystyle \int M'(t)\mathrm{\,d}t=M(t)=M_0 t-5\mathrm{e}^{-0{,}1t^2}+C$.
		\itemch Tổng năng lượng tiêu hao trong khoảng thời gian $12$ giờ là $5\,120$ J $ =5{,}12$ kJ tức là
		\[
		M(12)-M(0)=5{,}12\,\,\text{kJ}.
		\]
		Khi $t=12$ ta có $M(12)=M_0\cdot 12-5\mathrm{e}^{-0{,}1\cdot 12^2}+C$.\\
		Khi $t=0$ ta có $M(0)=M_0\cdot 0-5\mathrm{e}^{-0{,}1\cdot 0^2}+C=-5+C$.\\
		Suy ra $M(12)-M(0)=12M_0-5\left( \mathrm{e}^{-0{,}1\cdot 12^2}-1\right)$.\\
		Thay $M(12)-M(0)=5\,120$ ta được
		\[
		12M_0-5\left( \mathrm{e}^{-0{,}1\cdot 12^2}-1\right) = 5{,}12
		\Leftrightarrow M_0=0{,}01\,\, \rm{kJ/h}.
		\]
		\itemch Năng lượng tiêu hao trong $6$ giờ
		\[
		M(6)-M(0)=6M_0-5\left(\mathrm{e}^{-0{,}1\cdot 6^2}-1\right)\approx 4{,}92\,\,\text{kJ}.
		\]
		Năng lượng còn lại sau $6$ giờ
		\[
		5120\,\,\text{J}-4920\,\,\text{J}=200\,\,{J}.
		\]
		\itemch	Tốc độ tiêu hao trung bình từ $6$ giờ đến $12$ giờ là
		\[
		v_{\text{tb}}=\dfrac{M(12)-M(6)}{12-6}=\dfrac{6M_0-5\left(\mathrm{e}^{-0{,}1\cdot 12^2}-\mathrm{e}^{-0{,}1\cdot 6^2}\right)}{6}\approx 0{,}0327695\,\,\rm{kJ/h}\approx 32{,}77 \,\,\rm{J/h}.
		\]
	\end{itemchoice}
	}
\end{ex}

%

\begin{ex}%[Nguồn: Bộ đề minh họa Moon 2024-2025]%[2D4C1-6]
	Một bể bơi hình trụ trên mặt đất có đường kính $5$ m, được bơm nước vào với tốc độ $v_0$, không đổi cho đến khi mực nước cao $1$ m. \begin{center}
		\begin{tikzpicture}
			% Tạo phần bể bơi (mặt trụ)
			\shade[top color=gray!60, bottom color=gray!60]
			(0,0) ellipse (2.5 and 0.5); % Đáy trên
			
			\draw[thick] (0,-1.5) ellipse (2.5 and 0.5); % Đáy dưới
			
			% Thành bể (cạnh bên)
			\shade[bottom color=gray!60, top color=gray!50]
			(2.5,0) -- (2.5,-1.5) arc (0:-180:2.5 and 0.5) -- (-2.5,0) arc (180:0:2.5 and 0.5);
			
			% Mặt nước
			\fill[blue!60, opacity=0.7] (0,0) ellipse (2.2 and 0.4); % Mặt nước trong bể
			
			% Đường kính
			\draw[<->, thick] (-2.2,0) -- (2.2,0);
			\node at (0,-0.2) {$5$ m};
			
			% Ghi chú
			\node at (0, 0.6) {Mặt nước};
			\node[above right] at (2.5, -0.8) {Thành bể};
		\end{tikzpicture}
	\end{center}
	Sau khi được làm đầy $1$ m, bể bơi bị thủng ở đáy và nước rò rỉ ra ngoài. Bể bơi rò hết toàn bộ nước trong $8$ giờ. Biết rằng lượng nước còn lại trong hồ được mô phỏng bởi hàm $h(t)$ và tốc độ nước chảy ra ngoài vào thời điểm $t$ giờ (tính từ lúc bắt đầu đỗ) được xác định bởi hàm số $h'(t) = at + b$ ($a$, $b \in \mathbb{R}$). Lúc nước chảy hết ra ngoài thì vận tốc nước chảy bằng $0$.
	\choiceTF
	{Thể tích của bể bơi sau khi được làm đầy $1$ m là $6,25$ m$^3$}
	{\True Tốc độ nước chảy ra ngoài tại thời điểm $8$ giờ kể từ lúc bể bị rò bằng $0$}
	{$a = \dfrac{25}{128}$}
	{Sau $4$ giờ kể từ lúc bể bị rò, lượng nước bị mất đi bằng $4,6875$ m$^3$}
	\loigiai{\begin{itemchoice}
		\itemch Thể tích nước sau khi bể làm đầy $1$ m là $$V_{\text{nước}}=\pi R^2\cdot 1=\pi\cdot \left(\dfrac{5}{2}\right)^2\cdot 1=\dfrac{25\pi}{4}\approx 19,63~ \mathrm{m}^3.$$
		\itemch Theo đề ta có bể rò hết toàn bộ nước trong $8$ giờ nên tại thời điểm $8$ giờ thì nước chảy hết ra ngoài.\\
		Vậy tốc độ nước tại thời điểm $8$ giờ kể từ lúc bể bị rò bằng $0$.
		\itemch Lượng nước còn lại trong bể sau $t$ giờ kể từ lúc bể bị rò là \[h(t)=\displaystyle\int (at+b)\mathrm{\,d}t=\dfrac{1}{2}at^2+bt+c.\]
		Tại thời điểm $t=0$ giờ thì  $h(0)=\dfrac{25\pi}{4}$ nên $c=\dfrac{25\pi}{4}$.\\
		Tại thời điểm $t=8$ giờ thì $\heva{&h(8)=0\\&h'(8)=0}\Leftrightarrow\heva{&32 a+8b+\dfrac{25\pi}{4}=0\\&8a+b=0}\Leftrightarrow\heva{&a=\dfrac{25\pi}{128}\\&b=-\dfrac{25\pi}{16}.}$
		\\Vậy $a=\dfrac{25\pi}{128}$.
		\itemch Sau $4$ giờ kể từ lúc bể bị rò, lượng nước bị mất đi là $$h(0)-h(4)=\dfrac{25\pi}{4}-\left(\dfrac{1}{2}\cdot \dfrac{25\pi}{128}\cdot 4^2-\dfrac{25\pi}{16}\cdot 4+\dfrac{25\pi}{4}\right)=\dfrac{75\pi}{16}\approx 14,72~\mathrm{m}^3.$$
	\end{itemchoice}}
\end{ex}

%

\begin{ex}%[Nguồn: Bộ đề minh họa Moon 2024-2025]%[2D4C1-6]
	\immini{
	Một quân nhân đang ở trong một chiếc xe tăng X (loại quân sự) di chuyển dọc theo trục $Oy$ về phía gốc tọa độ. Tại thời điểm $t=0$, xe cách gốc tọa độ $4$\,km, và $10$\,phút sau xe cách gốc tọa độ $2$\,km. Tốc độ của xe (km/h) tỉ lệ nghịch với khoảng cách của xe (km) đến gốc tọa độ. Biết rằng một xe tăng địch đang chờ ở tại điểm $A(3; 0)$ trên trục $Ox$, nhưng có một bức tường cao dọc theo đường cong $xy=1$ (tất cả khoảng cách tính bằng km) ngăn cản bạn nhìn thấy vị trí chính xác của nó. Tại thời điểm đầu tiên quân địch phát hiện xe tăng, tốc độ của xe tăng X là bao nhiêu km/h?
	}
	{
	\includegraphics[width=5cm]{hinh/Cau4P3De1}
	}
	\shortans{27}
	\loigiai{
	Ta có $10$\,phút $=\dfrac{1}{6}$\,giờ.\\
	Gọi $s(t)$\,km là hàm số biểu thị quãng đường đi của xe tăng X theo thời gian $t$\,giờ,\\
	$v(t)$\,km/h là vận tốc của xe tăng X theo thời gian $t$\,giờ,\\
	Theo giả thiết, ta có $v(t)=-\dfrac{k}{s(t)}$ \hfill $(1)$\\
	trong đó $k$ là hệ số cần tìm và dấu \lq\lq $-$\rq\rq\, biểu thị chiều chuyển động của xe tăng X.\\
	Mà $s'(t)=v(t)$.\hfill $(2)$\\
	Từ $(1)$ và $(2)$ ta có
	\allowdisplaybreaks
	\begin{eqnarray*}
		&&s'(t)=-\dfrac{k}{s(t)}\\
		&\Rightarrow& s(t)\cdot s'(t)=k\\
		&\Rightarrow& \displaystyle\int s(t)\cdot s'(t)\mathrm{\,d}t=-\displaystyle\int k\mathrm{\,d}t\\
		&\Rightarrow&\dfrac{1}{2}s^2(t)=-kt+C
	\end{eqnarray*}
	Do $\heva{&s(0)=4\\&s\left(\dfrac{1}{6}\right)=2}\Rightarrow \heva{&\dfrac{1}{2}\cdot 4^2=-k\cdot 0+C\\&\dfrac{1}{2}\cdot 2^2=-k\cdot \dfrac{1}{6}+C}\Rightarrow\heva{&C=8\\&k=36.}$\\
	Xét hàm số $xy=1\Rightarrow y=\dfrac{1}{x}\Rightarrow y'=-\dfrac{1}{x^2}$.\\
	Gọi $d$ là đường thẳng qua $(3;0)$ và tiếp xúc với đồ thị hàm số $y=\dfrac{1}{x}$ tại điểm $(x_0;y_0)$. Khi đó
	\allowdisplaybreaks
	\begin{eqnarray*}
		&&y_0=y'(x_0)(x_0-3)\\
		&\Rightarrow& \dfrac{1}{x_0}=-\dfrac{1}{x_0^2}(x_0-3)\\
		&\Rightarrow&\heva{&x_0=\dfrac{3}{2}\\&y_0=\dfrac{2}{3}.}
	\end{eqnarray*}
	Suy ra $d\colon y=-\dfrac{4}{9}(x-3)$.\\
	Khi xe tăng địch phát hiện xe tăng X tại điểm $B$, hay $B=d\cap Oy\Rightarrow B\left(0;\dfrac{4}{3}\right)\Rightarrow s(t)=\dfrac{4}{3}$.\\
	Vậy $v(t)=-\dfrac{k}{s(t)}=-\dfrac{36}{\dfrac{4}{3}}=-27$\,km/h.
	}
\end{ex}

%Điền đáp án 4

\begin{bt}%[Nguồn: Bộ đề minh họa Moon 2024-2025]%[2D4C2-6]
	Một chiếc xe đua Bugatti đang chuyển động trên đường đua. Đồ thị trên hình vẽ bên dưới biểu thị vận tốc $v$ (m/s) của chiếc xe đó trong $5$ giây đầu tiên.
	\begin{center}
		\begin{tikzpicture}[font=\footnotesize, line join=round, line cap=round, >=stealth, scale=1, y=0.166666666cm]
			\draw[smooth] plot[domain=0:2] (\x,{(\x)^3/2+(\x)});
			\draw (2,6)--(3,30)--(5,30);
			\draw[dashed] (2,0)|- (0,6) (3,0)|-(0,30) (5,0)--(5,30);
			\draw[->] (-0.5,0)--(6,0)node[below]{$t$ (s)};
			\draw (0,0)node[below left]{$O$};
			\draw[->] (0,-1)--(0,35)node[left]{$v$ (m/s)};
			\foreach \x/\g/\n in {(2,0)/below/2, (3,0)/below/3, (5,0)/below/5, (0,6)/left/6, (0,30)/left/30, (2,6)/above/\, , (3,30)/above/\, , (5,30)/above/\,}{
			\fill \x circle (1pt)node[\g]{$\n$};
			}
		\end{tikzpicture}
	\end{center}
	Đồ thị trong $2$ giây đầu tiên là một nhánh của hàm bậc ba nhận $O$ làm tâm đối xứng, trong $1$ giây tiếp theo xe tăng tốc với gia tốc $a$ (m/s$^2$) và đạt vận tốc $30$ m/s tại giây thứ $3$, sau đó duy trì vận tốc này đến giây thứ $5$. Biết quãng đường xe đi được trong $5$ giây đầu bằng $82$ m. Vận tốc của xe tại giây đầu tiên bằng bao nhiêu? (tính theo đơn vị km/h).
	\par\shortans{$5{,}4$}
	\loigiai{
	Gọi $v_{1}(t)=at^3+bt^2+ct+d$ ($a\ne 0$) là hàm số bậc ba biểu diễn vận tốc trong $2$ giây đầu.\\
	Ta có $v_{1}'(t)=3at^2+2bt+c$, $v_{1}''(t)=6at+2b$.\\
	Do $O$ thuộc đồ thị hàm số $v_{1}(t)$ nên ta có $d=0$.\\
	Điểm $O$ là điểm uốn nên $v''(0)=0 \Leftrightarrow b=0$.\\
	Do $(2;6)$ thuộc đồ thị hàm số $v_{1}(t)$ nên ta có $8a+2c=6$. \quad$(1)$.\\
	Trong thời gian từ giây thứ $2$ đến thứ $3$ vật tăng tốc với gia tốc $a$ m/s$^2$, suy ra $v_{2}(t)=\displaystyle\int a\mathrm{\,d}t=at+C$, với $C$ là hằng số.\\
	Để tránh nhầm lẫn gia tốc $a$ m/s$^2$ và hệ số $a$ của $v_{1}(t)$.\\
	Ta gọi $v_{2}(t)=et+g$ ($e\ne 0$) là hàm số biểu diễn vận tốc từ giây thứ $2$ đến thứ $3$.\\
	Do $(2;6)$ và $(3;30)$ thuộc đồ thị hàm số $v_{2}(t)$ nên ta có hệ phương trình
	$$ \heva{& 2e+g=6 \\& 3e+g=30} \Leftrightarrow \heva{& e=24 \\& g=-42.} $$
	Suy ra $v_{2}(t)=24t-42$.\\
	Từ giây thứ $3$ đến giây thứ $5$ vận tốc giữ ở mức $30$ m/s, suy ra hàm số biểu diễn vận tốc của xe trong thời gian này là $v_{3}(t)=30$.\\
	Gọi $v(t)$ là hàm số biểu diễn vận tốc của xe trong $5$ giây đầu tiên, khi đó
	$$ v(t)=\heva{
	& at^3+ct & \text{\,khi\,} & 0 \leq t < 2 \\
	& 24t-42 & \text{\,khi\,} & 2 \leq t < 3 \\
	& 30 & \text{\,khi\,} & 3 \leq t \leq 5.
	} $$
	Quãng đường xe đi được trong $5$ giây đầu bằng $82$ m, ta có
	\begin{eqnarray*}
		&& \displaystyle\int\limits_0^5 v(t) \mathrm{\,d}x= 82 \\
		&\Leftrightarrow& \displaystyle\int\limits_0^2 (at^3+ct) \mathrm{\,d}x + \displaystyle\int\limits_2^3 (24t-42) \mathrm{\,d}x + \displaystyle\int\limits_3^5 30 \mathrm{\,d}x = 82 \\
		&\Leftrightarrow& \left(\dfrac{at^4}{4}+\dfrac{ct^2}{2}\right)\Bigg|_0^2 + 18 + 60 = 82 \\
		&\Leftrightarrow& 4a+2c=4. \quad(2)
	\end{eqnarray*}
	Từ $(1)$ và $(2)$ ta có hệ phương trình $\heva{& 8a+2c=6 \\& 4a+2c=4} \Leftrightarrow \heva{&a=\dfrac{1}{2} \\& c=1.}$\\
	Suy ra $v_{1}(t)=\dfrac{1}{2}t^3+t$.\\
	Vậy trong giây đầu tiên, vận tốc của xe là $v_{1}(1)=1{,}5$ m/s $=5{,}4$ km/h.
	}
\end{bt}

%

\begin{ex}%[Nguồn: Bộ đề minh họa Moon 2024-2025]%[2D4C2-6]
	Trên hệ trục tọa độ $Oxy$, một khu vườn được giới hạn bởi parabol $\left(P_1\right)\colon y=-x^2+4$ và trục hoành. Biết rằng parabol $\left(P_1\right)$ cắt trục hoành tại hai điểm $A$, $B$ và đường thẳng $d\colon y=a $ $(0<a<4)$ cắt $\left(P_1\right)$ tại hai điểm $M$, $N$ như hình vẽ bên dưới. Xét parabol $\left(P_2\right)$ đi qua hai điểm $A$, $B$ và có đỉnh thuộc đường thẳng $y=a$. Hai phần tô đậm có diện tích bằng nhau. Cho đơn vị trên các trục tọa độ là $10$ mét, phần tô đậm dùng để trồng hoa với chi phí là $200\ 000$ đồng mỗi mét vuông và phần còn lại dùng để trồng cỏ với chi phí là $100\ 000$ đồng mỗi mét vuông. Số tiền mà người ta bỏ ra để trang trí khu vườn bằng bao nhiêu triệu đồng? (làm tròn kết quả đến hàng phần mười).
	\begin{center}
		\begin{tikzpicture}[>=stealth,x=1cm,y=1cm,scale=1]
			\draw[->] (-3,0) -- (3,0) node[below] {$x$};
			\draw[->] (0,-1) -- (0,5) node[left] {$y$};
			\filldraw (0,0) circle (1pt)node[below right]{$O$};
			\def\f[#1]{-(#1)^2+4};
			\def\g[#1]{-0.4*(#1)^2+1.6};
			\draw[domain=-2.2:2.2,samples=300] plot (\x,{\f[\x]});
			\draw[domain=-2.5:2.5,samples=300] plot (\x,{\g[\x]});
			\draw[pattern=north west lines,opacity=0.7]plot[domain=-2:2](\x,{\g[\x]})--plot[domain=2:-1](\x,{0});
			\draw[pattern=north west lines,opacity=0.7]plot[domain=-1.549:1.549](\x,{\f[\x]})--plot[domain=2:-1](\x,{1.6});
			\path (2,0) coordinate (B);
			\path (-2,0) coordinate (A);
			\path (-1.549,1.6) coordinate (M);
			\path (1.549,1.6) coordinate (N);
			\foreach \x/\g in {A/135,B/45,M/135,N/45}
			\fill [blue] (\x) circle (1.5pt)
			+(\g:3mm) node {$\x$};
			\draw (-3,1.6)--(3,1.6) node[above]{$y=a$};
		\end{tikzpicture}
	\end{center}
	\loigiai{
	Phương trình giao điểm của $(P_1)$ với trục hoành là $-x^2+4=0\Leftrightarrow x=\pm 2$.\\
	Suy ra $A(-2;0)$ và $B(2;0)$.\\
	Phương trình giao điểm của $(P_1)$ với $d$ là $-x^2+4=a\Leftrightarrow x=\pm \sqrt{4-a}$.\\
	$(P_2)$ có dạng $y=kx^2+l$. $(P_2)$ đi qua $A(-2;0)$ và $C(0;a)$
	\[\heva{&4k+l=0\\&l=a}\Rightarrow \heva{&k=-\dfrac{a}{4}\\&l=a}\Rightarrow (P_2)\colon y=-\dfrac{a}{4}x^2+a\]
	Ta có $\begin{aligned}[t]
		\displaystyle\int\limits_{-2}^2 \left(-\dfrac{a}{4}x^2+a\right) \mathrm{\,d}x=\displaystyle\int\limits_{-\sqrt{4-a}}^{\sqrt{4-a}} \left(-x^2+4-a\right) \mathrm{\,d}x&\Leftrightarrow \left(-\dfrac{a}{12}x^3+ax\right)\Bigg|^2_{-2}=\left(-\dfrac{x^3}{3}+(4-a)x\right)\Bigg|_{-\sqrt{4-a}}^{\sqrt{4-a}}\\
		&\Leftrightarrow \dfrac{8}{3}a=\dfrac{4}{3}(4-a)\sqrt{4-a}\\
		&\Rightarrow 4a^2=(4-a)^3\\
		&\Leftrightarrow a^3-8a^2+48a-64=0\\
		&\Leftrightarrow a\approx 1{,}72.
	\end{aligned}$\\
	Số tiền mà người ta bỏ ra để trang trí khu vườn bằng
	\begin{eqnarray*}
		&&2000\cdot 2\displaystyle\int\limits_{-2}^2 \left(-\dfrac{a}{4}x^2+a\right) \mathrm{\,d}x+1000\cdot\left[\displaystyle\int\limits_{-2}^2 \left(-x^2+4\right) \mathrm{\,d}x- 2\displaystyle\int\limits_{-2}^2 \left(-\dfrac{a}{4}x^2+a\right) \mathrm{\,d}x\right]\\
		&=&\displaystyle\int\limits_{-2}^2 \left(-1000x^2+4000-500ax^2+2000a\right) \mathrm{\,d}x=19840 \text{ (nghìn đồng).}
	\end{eqnarray*}
	Vậy số tiền mà người ta bỏ ra để trang trí khu vườn là $19{,}8$ triệu đồng.
	}
\end{ex}

%

\begin{bt}%[Nguồn: Bộ đề minh họa Moon 2024-2025]		%[2D4C3-2]
	Kiến trúc sư thiết kế một khu sinh hoạt cộng đồng có dạng hình chữ nhật với chiều rộng và chiều dài lần lượ là $60$ m và $80$ m. Trong đó, phần được tô màu đậm là sân chơi, phần còn lại đề trồng hoa. Mỗi phần trồng hoa có đường biên cong là một phần của parabol với đỉnh thuộc một trục đối xứng của hình chữ nhật và khoảng cách từ đỉnh đó đến trung điểm cạnh tương ứng của hình chữ nhật bằng $20$ m (xem hình minh họa). Diện tích của phần sân chơi là bao nhiêu mét vuông?
	\shortans{3200}
	\begin{center}
		\begin{tikzpicture}[yscale=0.8]
			% Bottom parabolic curve
			\draw[thick,domain=0:6,samples=100] plot (
			\x,{1.8 - 0.2*(\x - 3)^2});
			% Vertical dimensions
			\draw[<->] (3,0) -- (3,1.8) node[midway,left] {$20\,\mathrm{m}$};
			\draw[<->] (3,6.2) -- (3,8) node[midway,left] {$20\,\mathrm{m}$};
			\begin{scope}
				\draw[thick] (0,0) rectangle (6,8);
				\draw[thick,domain=0:6,samples=100] plot (\x,{6.2 + 0.2*(\x - 3)^2});
				
				\fill[pattern=north east lines] plot[domain=0:6,samples=100] (\x,{1.8 - 0.2*(\x - 3)^2}) --  (6,0)--(6,8)--
				plot[domain=6:0,samples=100] (\x,{6.2 + 0.2*(\x - 3)^2});
			\end{scope}
		\end{tikzpicture}
	\end{center}
	\loigiai{
	
	Gọi $S_h$, $S_c$, $S_k$ lần lượt là diện tích của phần trồng hoa, phần sân chơi và diện tích của khu sinh hoạt.
	
	Ta có công thức tính diện tích hình phẳng giới hạn bởi parabol và trục hoành với độ dài đáy $d$ và đường cao $h$ là $S=\dfrac{2}{3} d h$.
	
	Suy ra diện tích phần trồng hoa là $S_h=\dfrac{2}{3} \cdot 60\cdot 20=800\mathrm{~m}^2$.
	
	Khi đó diện tích phần sân chơi là $S_c=S_k-2S_h=60\cdot 80-2\cdot 800=3200\mathrm{~m}^2$.
	}
\end{bt}

\begin{ex}%[Nguồn: Bộ đề minh họa Moon 2024-2025]%[2D4C3-4]
	Một chiếc dàn Ghita có chiều cao $5$ cm, khi cắt một mặt cắt ngang của cây đàn ta thu được một mặt phẳng như hình vẽ.
	\begin{center}
		\begin{tikzpicture}[line join=round, line cap=round,>=stealth,thick, xscale=.4, yscale=.2]
			\tikzset{every node/.style={scale=0.9}}
			\draw (0,0) node [below left] {$O$};
			\begin{scope}[rotate=-10, yscale=1.4]
				\clip (0,-15) rectangle (11,15);
				\draw[fill =gray, line width = 1.2pt,draw=none] (0,0) plot[domain=0:10.08](\x,{1/24*((\x)^4)-7/9*((\x)^3)+14/3*((\x)^2)-32/3*(\x)+0})--(10.08,0)--cycle;
				\draw[samples=200,domain=0:10.08,smooth,variable=\x] plot (\x,{1/24*((\x)^4)-7/9*((\x)^3)+14/3*((\x)^2)-32/3*(\x)+0});
			\end{scope}
			\begin{scope}[rotate=-10]
				\draw[->] (-1.1,0)--(12,0) ;
				\draw[->] (0,-8.1)--(0,15.1);
				\clip (0,-15) rectangle (11,15);
				\draw[fill =white, line width = 1.2pt,draw=none] (0,0) plot[domain=0:10.08](\x,{1/24*((\x)^4)-7/9*((\x)^3)+14/3*((\x)^2)-32/3*(\x)+0})--(10.08,0)--cycle;
				\draw[samples=200,domain=0:10.08,smooth,variable=\x] plot (\x,{-1/24*((\x)^4)+7/9*((\x)^3)+-14/3*((\x)^2)+32/3*(\x)+0});
				\draw[samples=200,domain=0:10.08,smooth,variable=\x] plot (\x,{1/24*((\x)^4)-7/9*((\x)^3)+14/3*((\x)^2)-32/3*(\x)+0});
				\draw[samples=200,domain=0:10.08,smooth,variable=\x] plot (\x,{1/24*((\x)^4)-7/9*((\x)^3)+14/3*((\x)^2)-32/3*(\x)+0});
				
			\end{scope}
			
		\end{tikzpicture}
	\end{center}
	Đối với mỗi vị trí, người ta đo được chiều rộng $h$ của cái đàn và ghi lại qua bảng sau ($x$, $h$ có đơn vị cm)
	\begin{center}
		\begin{tabular}{|c|c|c|c|c|c|c|c|c|c|c|}
			\hline$x$ & $0$ & $1$ & $2$ & $3$ & $4$ & $5$ & $6$ & $7$ & $8$ & $9$ \\
			\hline$h$ & $0$ & $13{,}4$ & $16{,}4$ & $15{,}3$ & $14$ & $15{,}6$ & $20$ & $25{,}4$ & $28{,}4$ & $23{,}2$ \\
			\hline
		\end{tabular}
	\end{center}
	Bạn Nhâm nhận thấy rằng số liệu được để cập trên bảng gần giống với một hàm bậc bốn. Bằng cách mô phỏng hàm bậc bốn $y=f(x)=ax^4+bx^3+cx^2+dx+e$ trên hệ trục $Oxy$, đồ thị hàm số đi qua các điểm $O, A(6; 10)$ và có $3$ điểm cực trị có hoành độ là $2 ; 4 ; 8$. Dựa vào hàm số $f(x)$ tìm được, tính thể tích của cái đàn ghita (đơn vị $\text{cm}^3$, làm tròn đến hàng đơn vị). \shortans[oly]{$889$}
	%plot(\x,{1.0/3.0*((-(\x)^(4.0))/24.0+14.0*(\x)^(3.0)/18.0-14.0*(\x)^(2.0)/3.0+32.0*(\x)/3.0)});
	\loigiai{
	Đồ thị hàm số $y=f(x)$ có $3$ điểm cực trị có hoành độ là $2; 4; 8$ nên
	\[f'(x)=k(x-2)(x-4)(x-8),\,\, \text{với}\, k\neq 0,\]
	hay
	\[f'(x)=k\left(x^3-14x^2+56x-64\right).\]
	Suy ra
	\[f(x)=k\left(\dfrac{1}{4}x^4-\dfrac{14}{3}x^3+28x^2-64x\right)+e.\]
	Do $f(0)=0$ nên $e=0$.\\
	Vì vậy $f(x)=k\left(\dfrac{1}{4}x^4-\dfrac{14}{3}x^3+28x^2-64x\right)$.\\
	Lại có $f(6)=10\Rightarrow -60k=10\Leftrightarrow k=-\dfrac{1}{6}$.\\
	Suy ra $f(x)=-\dfrac{1}{6}\left(\dfrac{1}{4}x^4-\dfrac{14}{3}x^3+28x^2-64x\right)$.\\
	Ta có
	\begin{eqnarray*}
		f(x)=0&\Leftrightarrow &-\dfrac{1}{6}\left(\dfrac{1}{4}x^4-\dfrac{14}{3}x^3+28x^2-64x\right)\\
		&\Leftrightarrow& \hoac{&x=0	\\&\dfrac{1}{4}x^3-\dfrac{14}{3}x^2+28x-64=0. }
	\end{eqnarray*}
	Phương trình $\dfrac{1}{4}x^3-\dfrac{14}{3}x^2+28x-64=0$ có nghiệm duy nhất $x\approx 10{,}07$.\\
	Diện tích mặt cắt ngang của cây đàn là
	\[S=2\displaystyle\int\limits_{0}^{10{,}07}f(x)\mathrm{\, d}x =2\displaystyle\int\limits_{0}^{10{,}07}-\dfrac{1}{6}\left(\dfrac{1}{4}x^4-\dfrac{14}{3}x^3+28x^2-64x\right)\mathrm{\, d}x\approx 177{,}85\,\, \text{cm}^2.\]
	Vậy thể tích của cây đàn là
	\[V=\displaystyle\int\limits_{0}^{5}S\mathrm{\, d}x\approx 5\cdot 177{,}85\approx 889\,\, \text{cm}^3.\]
	}
\end{ex}

\begin{ex}%[Nguồn: Bộ đề minh họa Moon 2024-2025]%[2D4C3-4]
	Một mô hình của hệ tim mạch liên hệ thể tích $V(t)$ của máu trong động mạch chủ tại thời điểm $t$ trong thời kỳ co tâm thất (giai đoạn co) với áp suất $P(t)$ trong động mạch chủ tại cùng thời điểm được cho bởi phương trình $V(t)=[0{,}4+0{,}09 P(t)]\left(\dfrac{3 t^2}{T^2}-\dfrac{2 t^3}{T^3}\right)$
	(lít) với $T$ là chu kỳ của pha tâm thu $(T=0{,}27 \mathrm{~s})$. Giả sử áp suất động mạch chủ $P(t)$ tăng đều từ 15 mmHg tại thời điểm $t=0$ đến 30 mmHg tại thời điểm $t=T$. Tìm thể tích trung bình của máu trong động mạch chủ theo đơn vị lít trong suốt pha tâm thu $(0 \leq t \leq T)$. (Làm tròn kết quả đến hàng phần mười).
	\par\shortans[]{$1{,}3$} \par
	\loigiai{
	Gọi áp suất trong động mạch chủ tại thời điểm $t$ là $P(t)=a t+b, a \neq 0.$\\ Theo bài ta ta có $P(0)=b=15$, $P(0{,}27)=0{,}27 a+b=30 \Rightarrow P(t)=\dfrac{500}{9}t+15$.
	Khi đó \begin{eqnarray*}
		V(t)&=&[0{,}4+0{,}09 P(t)]\left(\dfrac{3 t^2}{T^2}-\dfrac{2 t^3}{T^3}\right) \\
		&=&\left[0{,}4+0{,}09\left(\dfrac{500}{9} t+15\right)\right]\left(\dfrac{3 t^2}{0{,}27^2}-\dfrac{2 t^3}{0{,}27^3}\right) .
	\end{eqnarray*}
	Thể tích trung bình của máu trong động mạch chủ trong suốt pha tâm thu là
	\begin{eqnarray*}
		V_{t b}&=&\dfrac{1}{0{,}27-0} \int_0^{0{,}27} V(t) d t \\
		&=&\dfrac{1}{0{,}27-0} \int_0^{027}\left[0{,}4+0{,}09\left(\dfrac{500}{9} t+15\right)\right]\left(\dfrac{3 t^2}{0{,}27^2}-\dfrac{2 b^3}{0{,}27^{7^3}}\right) d t \\
		&=&\dfrac{539}{400} \approx 1{,}3 .
	\end{eqnarray*}
	}
\end{ex}

\begin{ex}%[Nguồn: Bộ đề minh họa Moon 2024-2025]%[2D4C3-5]
	Để chuẩn bị cho đêm hội diễn văn nghệ chào đón năm mới, bạn An đã làm một chiếc mũ \lq\lq cách điệu\rq\rq \, cho ông già Noel có dáng một khối tròn xoay. Mặt cắt qua trục của chiếc mũ như hình vẽ bên dưới. Biết rằng $OO' = 5$ cm, $OA = 10$ cm, $OB = 20$ cm, đường cong $AB$ là một phần của Parabol có đỉnh là điểm $A$. Thể tích của chiếc mũ bằng bao nhiêu centimét khối? (kết quả làm tròn đến hàng đơn vị).
	\begin{center}
		\begin{tikzpicture}[>=stealth,line join=round,line cap=round,font=\footnotesize,scale=0.25]
			\path
			(0,-5) coordinate (O')
			(0,0) coordinate (O)
			(10,0) coordinate (A)
			(0,20) coordinate (B)
			;
			\draw
			(-10,0) -- (-10,-5) -- (10,-5) -- (10,0)
			;
			\draw[dashed]
			(-10,0) -- (10,0)
			(0,20) -- (0,-5)
			;
			\draw[samples=100,smooth,domain=0:10] plot(\x,{(1/5)*(\x)^2 - 4*(\x) + 20});
			\draw[samples=100,smooth,domain=0:-10] plot(\x,{(1/5)*(\x)^2 + 4*(\x) + 20});
			\fill
			(O') circle(1pt) node[below]{$O'$}
			(A) circle(1pt) node[right]{$A$}
			(B) circle(1pt) node[right]{$B$}
			(O) circle(1pt) node[above left]{$O$}
			;
		\end{tikzpicture}
	\end{center}
	\shortans{$2618$}
	\loigiai{
	Xét hệ trục toạ độ $Oxy$ với gốc toạ độ tại điểm $O$, tia $Ox$ trùng $OA$ và tia $Oy$ trùng $OB$ như hình dưới.
	\begin{center}
		\begin{tikzpicture}[>=stealth,line join=round,line cap=round,font=\footnotesize,scale=0.25]
			\path
			(0,-5) coordinate (O')
			(0,0) coordinate (O)
			(10,0) coordinate (A)
			(0,20) coordinate (B)
			;
			\draw
			(-10,0) -- (-10,-5) -- (10,-5) -- (10,0)
			;
			\draw[dashed]
			(-10,0) -- (10,0)
			(0,20) -- (0,-5)
			;
			\draw[->]
			(10,0) -- (12,0) node[below]{$x$}
			;
			\draw[->]
			(0,20) -- (0,22) node[left]{$y$}
			;
			\draw[samples=100,smooth,domain=0:10] plot(\x,{(1/5)*(\x)^2 - 4*(\x) + 20});
			\draw[samples=100,smooth,domain=0:-10] plot(\x,{(1/5)*(\x)^2 + 4*(\x) + 20});
			\fill
			(O') circle(1pt) node[below]{$O'$}
			(A) circle(1pt) node[above]{$A$}
			(B) circle(1pt) node[right]{$B$}
			(O) circle(1pt) node[above left]{$O$}
			;
		\end{tikzpicture}
	\end{center}
	Ta chia cái mũ thành hai phần: phần đế hình trụ và hình chóp có được khi quay Parabol quanh trục $Oy$.\\
	Dễ thấy Parabol là đồ thị của hàm số $y = \dfrac{1}{5}x^2 - 4x + 20$ ta có
	\begin{align*}
		y = \dfrac{1}{5}x^2 - 4x + 20 &\Leftrightarrow 5y = x^2 - 20x + 100\\
		&\Leftrightarrow 5y = \left(x-10 \right)^2 \\
		&\Leftrightarrow \sqrt{5y} = 10 - x \, (\text{do } x \leq 10).\\
		&\Leftrightarrow x = 10 - \sqrt{5y}.
	\end{align*}
	Khi đó thể tích của phần chóp phía trên là $\pi\displaystyle\int\limits_0^{20} \left(10 - \sqrt{5y} \right)^2\mathrm{\,d}y = \dfrac{1000\pi}{3}$ cm$^3$.\\
	Thể tích của phần phía dưới là $\pi \cdot 10^2 \cdot 5 = 500\pi$ cm$^3$.\\
	Vậy thể tích của chiếc mũ là $V = \dfrac{1000\pi}{3} + 500\pi = \dfrac{2500\pi}{3} \approx 2618$ cm$^3$.
	}
\end{ex}

\begin{ex}%[Nguồn: Bộ đề minh họa Moon 2024-2025]%[2D4H1-1]
	Cho hàm số $y = f(x)$ có đạo hàm là $f'(x) = 8x^3 + \sin{x}$, $\forall x \in \mathbb{R}$ và thỏa mãn $f(0) = 3$.
	\choiceTF
	{\True Hàm số $y = f(x)$ là một nguyên hàm của hàm số $y = f'(x)$}
	{$f(x) = 2x^4 - \cos{x} + C_1$ với $C_1$ là hằng số}
	{$\displaystyle\int\limits f\left(x \right)\mathrm{\,d}x = \dfrac{2}{5}x^5 - \sin{x} + 3x + C_2$, với $C_2$ là hằng số}
	{\True Biết $F(x)$ là nguyên hàm của $f(x)$ thỏa mãn $F(0) = 2$. Khi đó, $F(1) = \dfrac{32}{5} - \sin{1}$}
	\loigiai{
	\begin{itemchoice}
		\itemch \textbf{Đúng}. Ta có $\left( f(x) + C \right)' = f'(x)$, với $C$ là hằng số.\\
		Vậy hàm số $y = f(x)$ là một nguyên hàm của hàm số $y = f'(x)$
		\itemch \textbf{Sai.} ta có $f(x) = \displaystyle\int\limits f'\left(x \right)\mathrm{\,d}x = \displaystyle\int\limits \left(8x^3 + \sin{x} \right)\mathrm{\,d}x = 2x^4 - \cos{x} + C_1$, $C_1$ là hằng số.\\
		Mà $f(0) = 3$ nên ta có $2 \cdot 0^4 - \cos{0} + C_1 = 3 \Rightarrow C_1 = 4$.\\
		Suy ra $f(x) = 2x^4 - \cos{x} + 4$.
		\itemch \textbf{Sai.} Ta có $\displaystyle\int\limits f\left(x \right)\mathrm{\,d}x = \displaystyle\int\limits \left(2x^4 - \cos{x} + 4 \right)\mathrm{\,d}x = \dfrac{2}{5}x^5 - \sin{x} + 4x + C_2$, với $C_2$ là hằng số.
		\itemch \textbf{Đúng.} Ta có $F(0) = 2$ suy ra $\dfrac{2}{5} \cdot 0^5 - \sin{0} + 4 \cdot 0 + C_2 = 2 \Rightarrow C_2 = 2$.\\
		Khi đó $F(1) = \dfrac{2}{5} \cdot 1^5 - \sin{1} + 4 \cdot 1 + 2 = \dfrac{32}{5} - \sin{1}$.
	\end{itemchoice}
	}
\end{ex}

\begin{ex}%[Nguồn: Bộ đề minh họa Moon 2024-2025]%[2D4H1-2]
	Cho hàm số $f(x)$ liên tục trên $\mathbb{R}$. Biết $F(x)$ là nguyên hàm của $f(x)$ thoả mãn $F(2)=2$ và $F(x)=\displaystyle\int[x-f(x)] d x, \forall x \in \mathbb{R}$. Giá trị của $F(4)$ bằng
	\choice
	{\True $5$}
	{$6$}
	{$8$}
	{$9$}
	\loigiai{
	Vì $F(x)$ là nguyên hàm của $f(x)$  nên $F'(x)=f(x)$. $\qquad(1)$\\
	Vì $F(x)=\displaystyle\int[x-f(x)] d x, \forall x \in \mathbb{R}$ nên $F'(x)=x-f(x)$. $\qquad(2)$\\
	Từ $(1)$ và $(2)$ suy ra $f(x)=x-f(x)\Rightarrow f(x)=\dfrac{1}{2}x$.\\
	Suy ra $F(x)=\displaystyle\int f(x)\mathrm{\,d}x=\dfrac{x^2}{4}+C$.\\
	Ta có $F(2)=2\Leftrightarrow 1+C=2\Leftrightarrow C=1$.\\
	Suy ra $F(x)= \dfrac{x^2}{4}+1$.\\
	Vậy $F(4)=5$.
	}
\end{ex}

%
\Opensolutionfile{ansbook}[Ansbook/TenFileTF]

\begin{ex}%[Nguồn: Bộ đề minh họa Moon 2024-2025]%[2D4H1-3]
	Cho hàm số $f(x)=2x-3\cos x$. Gọi $F(x)$ là một nguyên hàm của hàm số $f(x)$ thoả mãn điều kiện $F\left(\dfrac{\pi}{2}\right)=3$.
	\choiceTF
	{\True $F'(x)=2x-3\cos x$}
	{$\displaystyle\int f(x)\mathrm{\, d}x=x^2+3\sin x+C$}
	{\True $F(x)=x^2-3\sin x+6-\dfrac{\pi^2}{4}$}
	{$F(0)=3-\dfrac{\pi^2}{4}$}
	\loigiai{
	\begin{itemchoice}
		\itemch {\bf Đúng}.\\
		$F(x)$ là nguyên hàm của $f(x)$ nên $F'(x)=f(x)=2x-3\cos x$.
		\itemch {\bf Sai}.\\
		$F(x)=\displaystyle\int f(x)\mathrm{\, d}x=\displaystyle\int 2x-3\cos x\mathrm{\, d}x=x^2-3\sin x +C$.
		\itemch {\bf Đúng}.\\
		$F\left(\dfrac{\pi}{2}\right)=3 \Rightarrow 3=\left(\dfrac{\pi}{2}\right)^2-3\sin \left(\dfrac{\pi}{2}\right)+C \Rightarrow C=6-\left(\dfrac{\pi^2}{4}\right)$.\\
		Vậy $F(x)=x^2-3\sin x+6-\dfrac{\pi^2}{4}$
		\itemch {\bf Sai}.\\
		$F(0)=6-\dfrac{\pi^2}{4}$.
	\end{itemchoice}
	
	}
\end{ex}

\begin{ex}%[Nguồn: Bộ đề minh họa Moon 2024-2025]%[2D4H1-3]
	Mệnh đề nào \textbf{sai} trong các mệnh đề sau?
	\choice
	{$\displaystyle\int \dfrac{1}{\sin^2 x} \mathrm{d}x = -\cot x + C$}
	{$\displaystyle\int \cos x \mathrm{d}x = \sin x + C$}
	{$\displaystyle\int \dfrac{1}{\cos^2 x} \mathrm{d}x = \tan x + C$}
	{\True $\displaystyle\int \sin x \mathrm{d}x = \cos x + C$}
	\loigiai{
	Ta có $\displaystyle\int \sin x \mathrm{d}x = \cos x + C$ là sai vì $\displaystyle\int \sin x \mathrm{d}x = -\cos x + C$.
	}
\end{ex}

%

\begin{ex}%[Nguồn: Bộ đề minh họa Moon 2024-2025]%[2D4H1-3]
	Họ tất cả các nguyên hàm của hàm số $f(x)=4x+\sin x$ là
	\choice
	{\True $2x^2-\cos x+C$}
	{$2x^2+\cos x+C$}
	{$2x^2-\sin x+C$}
	{$2x^2+\sin x+C$}
	\loigiai{
	Ta có
	\begin{align*}
		\displaystyle \int f(x)\mathrm{\,d}x &= \int \left(4x+\sin x\right) \mathrm{\,d}x \\
		&= \int 4x\mathrm{\,d}x+\int \sin x \mathrm{\,d}x\\
		&=4\cdot \dfrac{x^2}{2}-\cos x+C\\
		&=2x^2-\cos x+C.
	\end{align*}
	}
\end{ex}

\begin{ex}%[Nguồn: Bộ đề minh họa Moon 2024-2025]%[2D4H1-3]
	Hàm số $F(x)=\sin 2x$ là một nguyên hàm của hàm số nào sau đây?
	\choice
	{$f_3(x)=\cos 2x$}
	{\True $f_2(x)=2\cos 2x$}
	{$f_1(x)=\dfrac{1}{2}\cos 2x$}
	{$f_4(x)=-\dfrac{1}{2}\cos 2x$}
	\loigiai
	{
	Ta có $f(x)=F'(x) =(\sin 2x)'=2\cos 2x$.
	}
\end{ex}

%

\begin{ex}%[Nguồn: Bộ đề minh họa Moon 2024-2025]%[2D4H1-4]
	Họ tất cả các nguyên hàm của hàm số $f(x) = \mathrm{e}^{x} - \dfrac{3}{x}$ trên khoảng $(-\infty;0)$ là
	\choice
	{$\mathrm{e}^{x+1} - 3\ln(-x) + C$}
	{\True $\mathrm{e}^{x} - 3\ln x + C$}
	{$\mathrm{e}^{x+1} - 3\ln x + C$}
	{$\mathrm{e}^{x} - 3\ln(-x) + C$}
	\loigiai{Ta có $\displaystyle\int f(x)\mathrm{\,d}x=\displaystyle\int \left(\mathrm{e}^x-\dfrac{3}{x}\right)\mathrm{\,d}x=\mathrm{e}^x-3\ln x+C$.}
\end{ex}

%

\begin{ex}%[Nguồn: Bộ đề minh họa Moon 2024-2025]%[2D4H1-5]
	Họ tất cả các nguyên hàm của hàm số $f(x)=\mathrm{e}^{4x+3}$ là
	\choice
	{$\mathrm{e}^{4x+3}+C$}
	{$4\mathrm{e}^{4x+3}+C$}
	{$(4x+3) \mathrm{e}^{4x+2}$}
	{\True $\dfrac{1}{4} \mathrm{e}^{4x+3}+C$}
	\loigiai{
	Ta có
	$\displaystyle\int f(x)\mathrm{\,d}x=\displaystyle\int \mathrm{e}^{4x+3}\mathrm{\,d}x=\dfrac{1}{4}\displaystyle\int \mathrm{e}^{4x+3}\mathrm{\,d}(4x+3)=\dfrac{1}{4} \mathrm{e}^{4x+3}+C$.
	}
\end{ex}

\begin{ex}%[Nguồn: Bộ đề minh họa Moon 2024-2025]%[2D4H1-6]
	Ở nhiệt độ $37^{\circ}$ C, một phản ứng hóa học từ chất đầu A, chuyển hóa thành chất B theo phương trình A $\longrightarrow$ B. Giả sử $y(x)$ là nồng độ chất $A$ (đơn vị mol $L^{-1}$) tại thời điểm $x$ (giây), $y(x)>0$ với $x \geq 0$, thỏa mãn hệ thức $y'(x)=-7 \cdot 10^{-4} y(x)$ với $x \geq 0$. Biết rằng tại $x=0$, nồng độ (đầu) của A là $0{,}05$ mol $L^{-1}$. Xét hàm số $f(x)=\ln y(x)$ với $x \geq 0$. Các phát biểu sau đây đúng hay sai?
	\choiceTF
	{\True $f'(x)=-7 \cdot 10^{-4}$}
	{\True $f(x)=-7 \cdot 10^{-4} x+\ln (0{,}05)$}
	{$y(30)-y(15)=-6\cdot 10^{-4}$}
	{\True Nồng độ trung bình của chất A từ thời điểm $15$ giây đến thời điểm $30$ giây gần bằng $0{,}05$}
	\loigiai{
	\begin{itemchoice}
		\itemch {\bf Đúng.}\\
		Ta có $f'(x)=\left[\ln y(x)\right]'=\dfrac{y'(x)}{y(x)}=-7\cdot 10^{-4}$.
		\itemch {\bf Đúng.}\\
		Ta có $f(x)=\displaystyle\int f'(x)\mathrm{\,d}x=\displaystyle\int -7\cdot 10^{-4}\mathrm{\,d}x =-7\cdot 10^{-4}x+C$.\\
		Theo giả thiết, $y(0)=0{,}05$ nên $f(0)=\ln y(0)=\ln 0{,}05$.\\
		Vì vậy $C=\ln 0{,}05$, suy ra $f(x)=-7\cdot 10^{-4}x+\ln0{,}05$.
		\itemch {\bf Sai.}\\
		Từ $f(x)=\ln y(x)$, suy ra $y(x)=\mathrm{e}^{f(x)}=\mathrm{e}^{-7\cdot 10^{-4}x+\ln0{,}05}=\dfrac{1}{20}\cdot \mathrm{e}^{-7\cdot 10^{-4}x}$.\\
		Do đó $y(30)-y(15)=\dfrac{1}{20}\left(\mathrm{e}^{-7\cdot 10^{-4}\cdot 30}-\mathrm{e}^{-7\cdot 10^{-4}\cdot15}\right)\approx-5{,}2\cdot 10^{-4}$.
		\itemch {\bf Đúng.}\\
		Nồng độ trung bình của chất A từ thời điểm $15$ giây đến thời điểm $30$ giây là
		\[\dfrac{1}{30-15}\displaystyle\int\limits_{15}^{30}y(x)\mathrm{\, d}x=\dfrac{1}{15}\displaystyle\int\limits_{15}^{30}\left(-\dfrac{1}{7\cdot 10^{-4}}\right)y'(x)\mathrm{\, d}x=-\dfrac{10^4}{105}\cdot y(x)\Bigg|_{15}^{30}\approx 0{,}05.\]
	\end{itemchoice}
	}
\end{ex}

\begin{ex}%[Nguồn: Bộ đề minh họa Moon 2024-2025]%[2D4H2-1]
	Nếu $\displaystyle\int\limits_0^2f(x)\mathrm{\, d}x=4$ thì $\displaystyle\int\limits_0^2\left[f(x)-3\right]\mathrm{\, d}x$ bằng
	\choice
	{$-4$}
	{$1$}
	{\True $-2$}
	{$3$}
	\loigiai
	{
	$\displaystyle\int\limits_0^2\left[f(x)-3\right]\mathrm{\, d}x=\displaystyle\int\limits_0^2f(x)\mathrm{\, d}x-\displaystyle\int\limits_0^2 3\mathrm{\, d}x=4-6=-2$.
	}
\end{ex}

\begin{ex}%[Nguồn: Bộ đề minh họa Moon 2024-2025]%[2D4H2-1]
	Cho $\displaystyle\int\limits_{-2}^{5}f(x)\mathrm{\, d}x=8$ và $\displaystyle\int\limits_{-2}^{5}g(x)\mathrm{\, d}x=-3$. Tính $I=\displaystyle\int\limits_{-2}^{5}\left[f(x)-4g(x)-1\right]\mathrm{\, d}x$.
	\choice
	{$I=3$}
	{\True $I=13$}
	{$I=-11$}
	{$I=27$}
	\loigiai{
	Ta có $I=\displaystyle\int\limits_{-2}^{5}f(x)\mathrm{\, d}x-4\displaystyle\int\limits_{-2}^{5}g(x)\mathrm{\, d}x-\displaystyle\int\limits_{-2}^{5}1\mathrm{\, d}x=8-4\cdot (-3)-7=13$.
	}
\end{ex}

\begin{ex}%[Nguồn: Bộ đề minh họa Moon 2024-2025]%[2D4H2-2]
	Cho $\displaystyle\int_{0}^{3}f(x)\mathrm{\,d}x=2$. Tính giá trị của tích phân $L=\displaystyle\int_{0}^{3}\left[ 2f(x)-x^{2}\right] \mathrm{\,d}x$.
	\choice
	{$L=0$}
	{\True $L=-5$}
	{$L=-23$}
	{$L=-7$}
	\loigiai{
	$L=\displaystyle\int_{0}^{3}\left[ 2f(x)-x^{2}\right] \mathrm{\,d}x=\displaystyle\int_{0}^{3} 2f(x)\mathrm{\,d}x-\displaystyle\int_{0}^{3}x^2\mathrm{\,d}x=2\displaystyle\int_{0}^{3} f(x)\mathrm{\,d}x-\dfrac{x^3}{3}\Big|_0^3=2\cdot2-\dfrac{3^3}{3}=-5$.
	}
\end{ex}

%

\begin{ex}%[Nguồn: Bộ đề minh họa Moon 2024-2025]%[2D4H2-2]
	Cho $F(x)$ là một nguyên hàm của hàm số $f(x)$ trên $\left[-1;3\right]$ và thỏa mãn $F(-1) = 2$, $F(3) = \dfrac{11}{2}$. Giá trị của $\displaystyle\int_{-1}^{3}\left[2f(x)-x\right]\mathrm{\,d}x$ bằng
	\choice
	{$11$}
	{$\dfrac{7}{2}$}
	{$\dfrac{19}{2}$}
	{\True $3$}
	\loigiai{Ta có \[\displaystyle\int_{-1}^{3}\left[2f(x)-x\right]\mathrm{\,d}x= \left.\left(2F(x)-\dfrac{1}{2}x^2\right)\right|_{-1}^3=\left(2\cdot\dfrac{11}{2}-\dfrac{9}{2}\right)-\left(2\cdot 2-\dfrac{1}{2}\right)=3.\]}
\end{ex}

%

\begin{ex}%[Nguồn: Bộ đề minh họa Moon 2024-2025]%[2D4H2-3]
	Biết $F(x)=\cos x$ là một nguyên hàm của hàm số $f(x)$ trên $\mathbb{R}$. Giá trị của $\displaystyle\int\limits_0^{\tfrac{\pi}{3}} 2f(x)\mathrm{\,d}x$ bằng
	\choice
	{\True $-1$}
	{$3$}
	{$1$}
	{$\sqrt{3}$}
	\loigiai{Ta có
	$\displaystyle\int\limits_0^{\tfrac{\pi}{3}} 2f(x)\mathrm{\,d}x=2\displaystyle\int\limits_0^{\tfrac{\pi}{3}} f(x)\mathrm{\,d}x=2F(x)\bigg|_0^{\tfrac{\pi}{3}}=2\left(\cos \dfrac{\pi}{3}-\cos 0\right)=-1$.
	}
\end{ex}

\begin{ex}%[Nguồn: Bộ đề minh họa Moon 2024-2025]%[2D4H2-6]
	Nam đang tham gia một bài học từ mới tiếng Anh trong vòng $60$ phút. Biết rằng $M(t)$ là số từ mới mà Nam có thể ghi nhớ trong $t$ phút. Tốc độ ghi nhớ từ mới của Nam được xác định bởi hàm số $M'(t) = at - bt^2$ (với $a, b \in \mathbb{R}$) (từ/phút) và đạt cao nhất tại thời điểm $40$ phút. Biết rằng Nam ghi nhớ được $18$ từ mới trong $10$ phút đầu tiên của bài học.
	\choiceTF
	{$a = 0{,}4$}
	{Khả năng ghi nhớ của Nam tại thời điểm $20$ phút là $6$ từ/phút}
	{Trong cả bài học Nam ghi nhớ được tổng cộng $427$ từ mới}
	{Biết rằng tốc độ học trung bình (từ/phút) tại thời điểm $n$ đến thời điểm $m$ được tính bởi công thức
	$\dfrac{1}{m-n}\displaystyle\int_n^m M'(t)\mathrm{\,d}t$.
	Tốc độ học trung bình của Nam trong cả bài học là $6$ từ/phút}
	\loigiai{
	\begin{itemchoice}
		\itemch Tốc độ ghi nhớ từ mới của Nam đạt cao nhất tại thời điểm $40$ phút nên $\dfrac{a}{2b} = 40$ hay $a = 80b$.\\
		Số từ mới mà Nam có thể ghi nhớ trong $t$ phút là
		$$M(t) = \displaystyle\int M'(t) \mathrm{\,d}t = \displaystyle\int (at - bt^2) \mathrm{\,d}t = \dfrac{a}{2}t^2 - \dfrac{b}{3}t^3 + C.$$
		Mặt khác, $M(0) = 0 \Rightarrow C = 0 \Rightarrow M(t) = \dfrac{a}{2}t^2 - \dfrac{b}{3}t^3$.
		
		Nam có thể ghi nhớ được $18$ từ mới trong $10$ phút đầu tiên của bài học nên $M(10) = 18$ hay
		$$\dfrac{a}{2} \cdot 10^2 - \dfrac{b}{3} \cdot 10^3 = 18 \Rightarrow \dfrac{80b}{2} \cdot 10^2 - \dfrac{b}{3} \cdot 10^3 = 18 \Rightarrow \dfrac{11000}{3}b = 18 \Rightarrow b = \dfrac{27}{5500}.$$
		
		Suy ra $a = \dfrac{108}{275} \approx 0,39$.\\
		Vậy $M(t) = \dfrac{54}{275}t^2 - \dfrac{9}{5500}t^3$, $M'(t) = \dfrac{108}{275}t - \dfrac{27}{5500}t^2$.
		\itemch Khả năng ghi nhớ của Nam tại thời điểm 20 phút là
		$$M'(20) = \dfrac{108}{275} \cdot 20 - \dfrac{27}{5500} \cdot 20^2 \approx 5{,}9 \text{ (từ/phút)}.$$
		\itemch Trong cả tiết học Nam ghi nhớ được tổng cộng số từ mới là
		$$\displaystyle\int_0^{60} \left(\dfrac{108}{275}t - \dfrac{9}{5500}t^2\right) \mathrm{\,d}t \approx 353 \text{ (từ)}.$$
		\itemch Tốc độ học trung bình của Nam trong cả tiết học là
		$$\dfrac{1}{60-0} \displaystyle\int_0^{60} M'(t) \mathrm{\,d}t = \dfrac{1}{60} \displaystyle\int_0^{60} \left(\dfrac{54}{275}t - \dfrac{9}{5500}t^2\right) \mathrm{\,d}t = \dfrac{324}{55} \approx 5{,}9 \text{ (từ/phút)}.$$
		
	\end{itemchoice}
	}
\end{ex}

\begin{ex}%[Nguồn: Bộ đề minh họa Moon 2024-2025]%[2D4H3-1]
	Cho hình phẳng $(H)$ giới hạn bởi các đồ thị hàm số $f(x)=x^2-2x-1$ và $g(x)=2x-4$. Xét hàm số $Q(x)=\dfrac{x^3}{3}-2x^2+3x$.
	\choiceTF
	{\True Phương trình $f(x)-g(x)=0$ có hai nghiệm phân biệt}
	{Hiệu $f(x)-g(x) > 0$ với mọi $x \in(1; 3)$}
	{\True Hàm số $Q(x)$ là một nguyên hàm của hàm số $f(x)-g(x)$}
	{\True Diện tích hình phẳng $(H)$ bằng $Q(1)-Q(3)$}
	\loigiai{
	\begin{itemchoice}
		\itemch Phương trình $f(x)=g(x)\Leftrightarrow x^2-2x-1=2x-4\Leftrightarrow x^2-4x+3=0\Leftrightarrow \hoac{&x=1\\&x=3}$.
		\itemch Ta có bảng xét dấu sau
		\begin{center}
			\begin{tikzpicture}
				\tkzTabInit[nocadre=false,lgt=2.5,espcl=2,deltacl=0.6]
				{$x$ /0.6,$f(x)-g(x)$ /1.1}
				{$-\infty$,$1$,$3$,$+\infty$}
				\tkzTabLine{,+,$0$,-,$0$,+,}
			\end{tikzpicture}
		\end{center}
		Từ bảng xét dấu suy ra $f(x)-(g(x)<0$ với $x\in(1;3)$.
		\itemch
		Ta có $\displaystyle\int (f(x)-g(x)) \mathrm{\, d}x=\displaystyle\int (x^2-4x+3) \mathrm{\, d}x=\dfrac{x^3}{3}-2x^2+3x+C$.\\
		Vậy $Q(x)=\dfrac{x^3}{3}-2x^2+3x$ là một nguyên hàm của $f(x)-g(x)$.
		\itemch
		Diện tích hình phẳng $(H)$ là $S=\displaystyle\int\limits_1^3 |x^2-4x+3| \mathrm{\, d}x=\displaystyle\int\limits_1^3 -(x^2-4x+3) \mathrm{\, d}x=Q(1)-Q(3)$.
	\end{itemchoice}
	
	}
\end{ex}

\begin{ex}%[Nguồn: Bộ đề minh họa Moon 2024-2025]%[2D4H3-1]
	Hình thang cong $ABCD$ ở hình vẽ có diện tích bằng
	\begin{center}
		\begin{tikzpicture}[line join=round, line cap=round,>=stealth,thick]
			\tikzset{every node/.style={scale=0.9}}
			\draw[->] (-1.1,0)--(4.1,0) node[below left] {$x$};
			\draw[->] (0,-1.8)--(0,4.1) node[below left] {$y$};
			\draw (0,0) node [below left] {$O$};
			\draw [](1,1)--(1,3) (3,-1)--(3,1) (3.3,-1.4)node[left]{$y=-x+2$} (2,1.9)node[above right]{$y=\dfrac{3}{x}$};
			\draw [dashed](1,0)node[below]{$1$}--(1,1)--(0,1)node[left]{$1$} (3,-1)--(0,-1)node[left]{$-1$} (1,3)--(0,3)node[left]{$3$};
			\begin{scope}
				\clip (-1,-1.3) rectangle (4,4);
				\draw[samples=200,domain=-1:4,smooth,variable=\x] plot (\x,{-1*(\x)+2});
				\draw[samples=200,domain=0.5:4,smooth,variable=\x] plot (\x,{3/(\x)});
				\fill[pattern=north east lines]plot[samples=200,domain=1:3,smooth,variable=\x] (\x,{-(\x)+2})--plot[samples=200,domain=3:1,smooth,variable=\x] (\x,{3/(\x)})--cycle;
			\end{scope}
		\end{tikzpicture}
	\end{center}
	\choice
	{$\displaystyle\int\limits_1^3 \left(\dfrac{3}{x} - x + 2\right) \mathrm{d}x$}
	{$\displaystyle\int\limits_1^3 \left(\dfrac{3}{x} - x - 2\right) \mathrm{d}x$}
	{$\displaystyle\int\limits_1^3 \left(\dfrac{3}{x} + x + 2\right) \mathrm{d}x$}
	{\True $\displaystyle\int\limits_1^3 \left(\dfrac{3}{x} + x - 2\right) \mathrm{d}x$}
	\loigiai{
	Từ hình vẽ ta thấy diện tích hình thang cong $ABCD$ bằng
	\[\displaystyle\int\limits_1^3 \left|\dfrac{3}{x} -(-x+2)\right| \mathrm{d}x=\displaystyle\int\limits_1^3 \left(\dfrac{3}{x} + x - 2\right) \mathrm{d}x\]
	}
\end{ex}

%

\begin{ex}%[Nguồn: Bộ đề minh họa Moon 2024-2025]%[2D4H3-1]
	\immini{
	Diện tích hình thang cong ở hình vẽ bên là $S=10$. Tích phân $\displaystyle\int\limits_0^4[4x+f(x)]\mathrm{\,d}x$ bằng
	}{
	\begin{tikzpicture}[thick,>=stealth,scale=0.8]
		\clip(-1.5,-1) rectangle (5.5,4.0);
		\draw[->,very thick,blue] (-1.5,0) -- (5.5,0) node[below left] {\small $x$};
		\draw[->,very thick,blue] (0,-1) -- (0,4.0) node[below left] {\small $y$};
		\draw [fill=white,draw=blue] (0,0) circle (1pt)node[below right] {\footnotesize $O$};
		\draw[very thick,black,smooth,samples=100,domain=-0.3:5.5] plot(\x,{(1/3)*(\x)^3-2*(\x)^2+3*(\x)+2});
		\draw[pattern = north east lines, line width = 1.2pt,draw=none] (0,0)--(0,2)--
		plot[domain=0:4] (\x,{(1/3)*(\x)^3-2*(\x)^2+3*(\x)+2})--(4,0)--cycle;
		\draw[dashed, thick,blue]
		(0,2) node[left]{$2$}
		(4,0) node[below]{$4$}
		(3.5,3.5) node[left]{$y=f(x)$}
		(2,1) node[left]{$S$};
		%	(1,0) node[below]{1}|-(0,3) node[above left]{3};
	\end{tikzpicture}
	}
	\choice
	{$14$}
	{\True $42$}
	{$32$}
	{$26$}
	\loigiai{
	Ta có $S=\displaystyle\int\limits_0^4 \left|f(x)\right|\mathrm{\,d}x=\displaystyle\int\limits_0^4 f(x)	\mathrm{\,d}x=10$.\\
	Do đó,
	\[
	\displaystyle\int\limits_0^4 \left[4x+f(x)\right]\mathrm{\,d}x=\displaystyle\int\limits_0^4 4x \mathrm{\,d}x+\displaystyle\int\limits_0^4 f(x)\mathrm{\,d}x
	=\left( 2x^2\right)\bigg|_0^4+10=32+10=42.
	\]
	}
\end{ex}

\begin{ex}%[Nguồn: Bộ đề minh họa Moon 2024-2025]%[2D4H3-1]
	Gọi $S$ là diện tích của hình phẳng giới hạn bởi các đường $y=2^x$, $y=0$, $x=0$, $x=2$. Mệnh đề nào dưới đây đúng?
	\choice
	{\True$S=\displaystyle\int\limits_0^2 2^x \mathrm{\, d}x$}
	{$S=\pi \displaystyle\int\limits_0^2 2^{2x} \mathrm{~d} x$}
	{$S=\displaystyle\int\limits_0^2 2^{2x} \mathrm{\, d}x$}
	{$S=\pi \displaystyle\int\limits_0^2 2^x \mathrm{~d} x$}
	\loigiai{
	Diện tích của hình phẳng giới hạn bởi các đường $y=2^x$, $y=0$, $x=0$, $x=2$ là $S=\displaystyle\int\limits_0^2 2^x \mathrm{\, d}x$.
	}
\end{ex}

\begin{ex}%[Nguồn: Bộ đề minh họa Moon 2024-2025]%[2D4H3-2]
	\immini{Cho hàm số $y=f(x)$ liên tục trên $\mathbb{R}$ và có đồ thị như hình vẽ bên. Biết rằng các diện tích $S_1$, $S_2$ thỏa mãn $S_1=2S_2=3$. Tích phân $\displaystyle\int_0^4f(x) \mathrm{d}x$ bằng}{
	\begin{tikzpicture}[scale=1, font= \footnotesize, line join=round, line cap=round, >=stealth]
		\draw[->] (-1,0) -- (4.5,0) node[above] {$x$};
		\draw[->] (0,-1) -- (0,2) node[right] {$y$};
		\draw[fill=black] (0,0) circle (1.5pt);
		\draw[line width=1pt,smooth,samples=100,domain=-0.35:4.25] plot(\x,{0.35*(-1*(\x)^3+6*(\x)^2-8*(\x))});
		\node[below left] at (0,0) {$O$};
		\node[below left] at (4,0) {$4$};
		\fill[pattern=north west lines]
		plot[domain=0:4] (\x,{0.35*(-1*(\x)^3+6*(\x)^2-8*(\x))}) --
		plot[domain=4:0] (\x,{0}) -- cycle;
		\node at (1,-0.5) {$S_1$};
		\node at (3,0.5) {$S_2$};
	\end{tikzpicture}}
	\choice
	{$3$}
	{$\dfrac{3}{2}$}
	{\True $-\dfrac{3}{2}$}
	{$\dfrac{9}{2}$}
	\loigiai{Ta có
	$$
	\int_0^{4} f(x) \mathrm{d} x=\int_0^a f(x) \mathrm{d} x+\int_a^4 f(x) \mathrm{d} x=-S_1+S_2=-\dfrac{3}{2}
	$$}
\end{ex}

\begin{ex}%[Nguồn: Bộ đề minh họa Moon 2024-2025]%[2D4H3-3]
	\immini{
	Cho hình phẳng $(H)$ giới hạn bởi đồ thị hàm số $y=4-x^2$ và trục hoành.	Thể tích khối tròn xoay được tạo thành khi quay $(H)$ xung quanh trục $Ox$ bằng
	\choice
	{$\dfrac{32\pi}{3}$}
	{$\dfrac{512}{15}$}
	{\True $\dfrac{512\pi}{15}$}
	{$\dfrac{32}{3}$}
	}
	{
	\begin{tikzpicture}[line cap=butt,line join=miter,>=stealth,scale=0.7]
		\draw[->] (-3,0)--(3,0)
		node[shift={(-100:7pt)},font=\normalsize]{$x$};
		\draw[->] (0,-1)--(0,5)
		node[shift={(180:7pt)},font=\normalsize]{$y$};
		\draw (5pt,0) |- (0,5pt) (0,0) node[shift={(225:9pt)},font=\normalsize]{$O$};
		\tikzset{declare function={f(\x)=4-(\x)^2;
		}}
		\begin{scope}[overlay]
			\path[pattern = north east lines, pattern color = black] plot[domain=-2:2] (\x, {f(\x)});
		\end{scope}
		\begin{scope}
			\clip (-3,-1) rectangle (3,5);
			\draw[samples=100] plot[domain=-5:5] (\x, {f(\x)});
		\end{scope}
		\node at (-1,1) {$(H)$};
	\end{tikzpicture}
	}
	\loigiai{
	Giao điểm của đồ thị hàm số $y=4-x^2$ và trục hoành là $4-x^2=0\Leftrightarrow\hoac{&x=2\\&x=-2.}$\\
	Thể tích khối tròn xoay được tạo thành khi quay $(H)$ xung quanh trục $Ox$ bằng
	\allowdisplaybreaks
	\begin{eqnarray*}
		V&=&\pi\displaystyle\int\limits_{-2}^2(4-x^2)^2\mathrm{\,d}x
		\\
		&=&\pi\displaystyle\int\limits_{-2}^2(16-8x^2+x^4)\mathrm{\,d}x
		\\
		&=&\pi\left(16x-8\cdot\dfrac{x^3}{3}+\dfrac{x^5}{5}\right)\bigg|_{-2}^2
		\\
		&=&\dfrac{512\pi}{15}.
	\end{eqnarray*}
	}
\end{ex}

\begin{ex}%[Nguồn: Bộ đề minh họa Moon 2024-2025]%[2D4H3-3]
	Cho hình phẳng $\left(H \right)$ giới hạn bởi các đường $y = 2x - x^2$, $y = 0$. Quay $\left(H \right)$ quanh trục hoành tạo thành một khối tròn xoay có thể tích là
	\choice
	{$\displaystyle\int\limits_0^2 \left(2x - x^2 \right)\mathrm{\,d}x$}
	{\True $\pi\displaystyle\int\limits_0^2 \left(2x - x^2 \right)^2\mathrm{\,d}x$}
	{$\displaystyle\int\limits_0^2 \left(2x - x^2 \right)^2\mathrm{\,d}x$}
	{$\pi\displaystyle\int\limits_0^2 \left(2x - x^2 \right)\mathrm{\,d}x$}
	\loigiai
	{Quay $\left(H \right)$ quanh trục hoành tạo thành một khối tròn xoay có thể tích là $\pi\displaystyle\int\limits_0^2 \left(2x - x^2 \right)^2\mathrm{\,d}x$.
	}
\end{ex}

\begin{ex}%[Nguồn: Bộ đề minh họa Moon 2024-2025]%[2D4H3-3]
	\immini{Tính thể tích của khối tròn xoay sinh ra khi quay quanh trục $Ox$ hình phẳng giới hạn bởi đồ thị hàm số $y=\sqrt{x}$, trục hoành và hai đường thẳng $x=0$, $x=1$ (xem hình bên) .
	\choice
	{$V=\dfrac{2\pi}{3}$}
	{$V=\dfrac{\pi}{3}$}
	{$V=\dfrac{3\pi}{2}$}
	{\True $V=\dfrac{\pi}{2}$}}
	{
	\begin{tikzpicture}[scale=1.8, font=\footnotesize, line join=round, line cap=round,>=stealth]
		%\fill[color=cyan!70, draw=none] (0,-3/2)--(0,3/2)--plot[domain=0:4](\x,{sqrt(\x)+3/2})--(4,3.5)--(4,-3.5)--plot[domain=4:0](\x,{-sqrt(\x)-3/2})--cycle;
		\draw[->] (-1,0) --(1.4,0) node[below]{$x$};
		\draw[->] (0,-1.1) --(0,1.3) node[left]{$y$};
		\draw (0,0) node[above left]{$O$};
		\draw[smooth,samples=200,domain=0:1] plot(\x,{sqrt(\x)});
		\draw[smooth,samples=200,domain=0:1] plot(\x,{-sqrt(\x)});
		\draw (1,0) ellipse (0.2 and 1);
		\draw[dashed] (0,-1)--(1,-1)--(1,1)--(0,1);
		\draw[below right](1,0)node{$1$}circle (0.3pt);
		\draw[left](0,1)node{$1$}circle (0.3pt);
		\draw[left](0,-1)node{$-1$}circle (0.3pt);
		\draw(0.5,0.75)node[above]{$y=\sqrt{x}$};
	\end{tikzpicture}}
	\loigiai{Thể tích khối tròn xoay là
	\[V=\pi \displaystyle\int\limits_0^1 \left(\sqrt{x}\right)^2\mathrm{\,d}x=\pi \cdot \dfrac{x^2}{2}\bigg|_0^1=\dfrac{\pi}{2}.\]
	}
\end{ex}

%

\begin{ex}%[Nguồn: Bộ đề minh họa Moon 2024-2025]%[2D4N1-1]
	Nguyên hàm của hàm số $f(x)=\mathrm{e}x$ là
	\choice
	{$\dfrac{\mathrm{e}{x+1}}{x+1}+C$}
	{\True $\mathrm{e}x+C$}
	{$\dfrac{\mathrm{e}x}{x}+C$}
	{$x \mathrm{e}{x-1}+C$}
	\loigiai{
	Ta có $\displaystyle \int \mathrm{e}{x} \mathrm{~d} x=\mathrm{e}{x}+C$.
	}
\end{ex}

\begin{ex}%[Nguồn: Bộ đề minh họa Moon 2024-2025]%[2D4N1-1]
	Cho các hàm số $f(x)$ và $g(x)$ có đạo hàm trên $\mathbb{R}$. Mệnh đề nào sau đây \textbf{sai}?
	\choice
	{$\displaystyle\int[f(x)+g(x)] \mathrm{\, d}x=\displaystyle\int f(x) \mathrm{\, d}x+\displaystyle\int g(x) \mathrm{\, d}x$}
	{$\displaystyle\int k f(x) \mathrm{\, d}x=k \displaystyle\int f(x) \mathrm{\, d}x$ (số thực $k \neq 0$)}
	{\True$\displaystyle\int f(x) g(x) \mathrm{\, d}x=\displaystyle\int f(x) \mathrm{\, d}x\cdot \displaystyle\int g(x) \mathrm{\, d}x$}
	{$\displaystyle\int f'(x) \mathrm{\, d}x=f(x)+C(C\in \mathbb{R})$}
	\loigiai{
	Mệnh đề $\displaystyle\int f(x) g(x) \mathrm{\, d}x=\displaystyle\int f(x) \mathrm{\, d}x\cdot \displaystyle\int g(x) \mathrm{\, d}x$ là sai.
	}
\end{ex}

\begin{ex}%[Nguồn: Bộ đề minh họa Moon 2024-2025]%[2D4N1-1]
	Tìm họ nguyên hàm của hàm số $f(x) = \cos x - 2x$.
	\choice
	{\True $\displaystyle\int\limits f(x)\mathrm{\,d}x =\sin x-x^2+C$}
	{$\displaystyle\int\limits f(x)\mathrm{\,d}x =-\sin x-x^2+C$}
	{$\displaystyle\int\limits f(x)\mathrm{\,d}x =\sin x-x^2$}
	{$\displaystyle\int\limits f(x)\mathrm{\,d}x =-\sin x-x^2$}
	\loigiai{
	Họ nguyên hàm của hàm số $f(x) = \cos x - 2x$ là  $\displaystyle\int\limits f(x)\mathrm{\,d}x =\sin x-x^2+C$.
	}
\end{ex}

%Câu đúng sai 1

\begin{ex}%[Nguồn: Bộ đề minh họa Moon 2024-2025]%[2D4N1-2]
	Biết $F(x)$ là một nguyên hàm của hàm số $f(x)=\dfrac{x^2+1}{x}$ trên khoảng $(0;+\infty)$.
	\choiceTF
	{\True $f(x)=x+\dfrac{1}{x}$}
	{$F(x)=f'(x), \forall x \in (0;+\infty)$}
	{$F(x)=\dfrac{1}{2}x^2-\dfrac{1}{x^2}+C$, với $C$ là hằng số}
	{Biết rằng đồ thị của hàm số $F(x)$ đi qua $M\left(\mathrm{e};\dfrac{\mathrm{e}^2}{2}\right)$. Khi đó $F(1)=\dfrac{1}{2}$}
	\loigiai{
	\begin{itemchoice}
		\itemch Trên khoảng $(0;+\infty)$, ta có $f(x)=\dfrac{x^2+1}{x}=x+\dfrac{1}{x}$.
		\itemch Ta có $F(x)=\displaystyle\int f(x)\mathrm{\,d}x$, $\forall x\in(0;+\infty)$.
		\itemch Với $x\in(0;+\infty)$, ta có $\left(\dfrac{1}{2}x^2-\dfrac{1}{x^2}+C\right)'=x+\dfrac{2}{x^3}=\dfrac{x^4+2}{x^3}\ne f(x)$.
		\itemch Trên khoảng $(0;+\infty)$, ta có $\displaystyle\int f(x)\mathrm{\,d}x = \displaystyle\int \left(x+\dfrac{1}{x}\right) \mathrm{\,d}x = \dfrac{1}{2}x^2+\ln x + C$, với $C$ là hằng số.\\
		Do $M\left(\mathrm{e};\dfrac{\mathrm{e}^2}{2}\right)$ thuộc đồ thị hàm số $F(x)$, suy ra $\dfrac{\mathrm{e}^2}{2} = \dfrac{1}{2}\cdot \mathrm{e}^2 + \ln \mathrm{e} + C \Leftrightarrow C =-1$.\\
		Suy ra $F(x)=\dfrac{1}{2}x^2+\ln x - 1$.\\
		Khi đó $F(1)=-\dfrac{1}{2}.$
	\end{itemchoice}
	}
\end{ex}

\begin{ex}%[Nguồn: Bộ đề minh họa Moon 2024-2025]%[2D4N1-2]
	Hàm số $F(x)=x^2+x+C$ là một nguyên hàm của hàm số nào dưới đây?
	\choice
	{\True $f_3(x)=2x+1$}
	{$f_1(x)=\dfrac{x^3}{3}+\dfrac{x^2}{2}$}
	{$f_2(x)=x^3+x^2$}
	{$f_4(x)=x+1$}
	\loigiai{Ta có $F'(x)=2x+1$.}
\end{ex}

\begin{ex}%[Nguồn: Bộ đề minh họa Moon 2024-2025]%[2D4N1-2]
	Họ nguyên hàm của hàm số $y = x^2 - x$ là
	\choice
	{\True $\dfrac{x^3}{3} - \dfrac{x^2}{2} + C$}
	{$2x - 1 + C$}
	{$x^2 + x + C$}
	{$\dfrac{x^3}{3} - x + C$}
	\loigiai{
	Ta có	$\displaystyle\int(x^2-x) \, \mathrm{d}x = \dfrac{x^3}{3} - \dfrac{x^2}{2} + C$.
	}
\end{ex}

\begin{ex}%[Nguồn: Bộ đề minh họa Moon 2024-2025]%[2D4N1-2]
	Trên khoảng $\left(0; +\infty \right)$, họ nguyên hàm của hàm số $f(x) = x^{-\dfrac{4}{5}}$ là
	\choice
	{$-\dfrac{5}{9}x^{-\dfrac{9}{5}} + C$}
	{$\dfrac{1}{5}x^{\dfrac{1}{5}} + C$}
	{\True $5x^{\dfrac{1}{5}} + C$}
	{$-\dfrac{9}{5}x^{-\dfrac{9}{5}} + C$}
	\loigiai
	{Ta có $\pi\displaystyle\int\limits x^{-\dfrac{4}{5}}\mathrm{\,d}x = x^{-\dfrac{4}{5} + 1} : \left(-\dfrac{4}{5} + 1 \right) + C = 5x^{\dfrac{1}{5}} + C$, với $C$ là hằng số.
	}
\end{ex}

\begin{ex}%[Nguồn: Bộ đề minh họa Moon 2024-2025]%[2D4N1-2]
	Cho $\displaystyle\int\limits f(x){\mathrm{\,d}}x=4x+C$. Họ nguyên hàm của hàm số $(x-1)f(x)$ là
	\choice
	{$x^{2}-4x+C$}
	{\True $2x^{2}-4x+C$}
	{$\dfrac{1}{2}x^{4}-\dfrac{2}{3}x^{3}+C$}
	{$\dfrac{1}{2}x^{4}-\dfrac{1}{3}x^{3}+C$}
	\loigiai{
	Từ $\displaystyle\int\limits f(x){\mathrm{\,d}}x=4x+C$, suy ra $f(x) = 4$.\\
	Do đó $(x-1)f(x) = (x-1) \cdot 4 = 4x - 4$.\\
	Vậy $\displaystyle\int\limits (4x - 4) {\mathrm{\,d}}x = 2x^2 - 4x + C$.
	}
\end{ex}

\begin{ex}%[Nguồn: Bộ đề minh họa Moon 2024-2025]%[2D4N1-3]
	Họ nguyên hàm $\displaystyle\int\limits \cos 2x\mathrm{\,d}x$ bằng
	\choice
	{$-\dfrac{1}{2}\sin 2x+C$}
	{$-\sin 2x+C$}
	{$\sin 2x+C$}
	{\True $\dfrac{1}{2}\sin 2x+C$}
	\loigiai{
	Ta có  $\displaystyle\int\limits \cos 2x\mathrm{\,d}x=\dfrac{1}{2}\sin 2x+C$.
	}
\end{ex}

%

\begin{ex}%[Nguồn: Bộ đề minh họa Moon 2024-2025]%[2D4N1-3]
	Cho hàm số $f(x)=1-\dfrac{1}{\cos ^2 x}$. Khẳng định nào sau đây đúng?
	\choice
	{$\displaystyle\int f(x) \mathrm{\,d}x=x+\tan x+C$}
	{\True $\displaystyle\int f(x) \mathrm{\,d}x=x-\tan x+C$}
	{$\displaystyle\int f(x) \mathrm{\,d}x=x+\cot x+C$}
	{$\displaystyle\int f(x) \mathrm{\,d}x=x-\cot x+C$}
	\loigiai{
	$\displaystyle\int f(x) \mathrm{\,d}x=\displaystyle\int \left(1-\dfrac{1}{\cos ^2 x}\right) \mathrm{\,d}x=x-\tan x+C$.}
\end{ex}

\begin{ex}%[Nguồn: Bộ đề minh họa Moon 2024-2025]%[2D4N1-4]
	Hàm số $f(x)=\mathrm{e}^{3x}$ có họ các nguyên hàm là
	\choice
	{$3\mathrm{e}^{3x}+C$}
	{$\dfrac{1}{3}\mathrm{e}^x+C$}
	{\True $\dfrac{1}{3}\mathrm{e}^{3x}+C$}
	{$3\mathrm{e}^x+C$}
	\loigiai{
	Ta có $\displaystyle\int \mathrm{e}^{3x}\mathrm{\,d}x=\dfrac{1}{3}\mathrm{e}^{3x}+C$.
	}
\end{ex}

%

\begin{ex}%[Nguồn: Bộ đề minh họa Moon 2024-2025]%[2D4N2-1]
	Nếu $\displaystyle\int\limits_1^3 f(x) \mathrm{\,d}x=5$ thì $\displaystyle\int\limits_1^3 2f(x) \mathrm{\,d}x$ bằng
	\choice
	{$\dfrac{5}{2}$}
	{$7$}
	{$4$}
	{\True $10$}
	\loigiai{
	$\displaystyle\int\limits_1^3 2f(x) \mathrm{\,d}x=2\cdot5=10$.}
\end{ex}

\begin{ex}%[Nguồn: Bộ đề minh họa Moon 2024-2025]%[2D4N2-2]
	Cho hàm số $y=f(x)$ có đạo hàm là hàm liên tục trên $\mathbb{R}$ thỏa mãn $\displaystyle\int\limits_0^2 f'(x) \mathrm{\,d} x=45$ và $f(0)=3$. Giá trị của biểu thức $f(2)$ bằng
	\choice
	{$42$}
	{$15$}
	{\True $48$}
	{$135$}
	\loigiai{
	Ta có $\displaystyle\int\limits_0^2 f'(x) \mathrm{\,d}x=f(2)-f(0)\Leftrightarrow 45=f(2)-f(0)$.\\
	Vậy $f(2)=45+f(0)=45+3=48$.
	}
\end{ex}

\begin{ex}%[Nguồn: Bộ đề minh họa Moon 2024-2025]%[2D4N2-2]
	Tích phân $\displaystyle\int\limits_{-1}^1x^{2020}dx$ bằng
	\choice
	{$\dfrac{1}{2021}$}
	{\True $\dfrac{2}{2021}$}
	{$\dfrac{2}{2020}$}
	{$0$}
	\loigiai{
	$\displaystyle\int\limits_{-1}^1x^{2020}dx=\dfrac{x^{2021}}{2021}\Bigg|_{-1}^1=\dfrac{1^{2021}}{2021}-\dfrac{(-1)^{2021}}{2021}=\dfrac{2}{2021}$.
	}
\end{ex}

\begin{ex}%[Nguồn: Bộ đề minh họa Moon 2024-2025]%[2D4N2-2]
	Biết $\displaystyle\int\limits_1^3f(x)\mathrm{\,d}x=3$. Giá trị của $\displaystyle\int\limits_1^3[f(x)+3]\mathrm{\,d}x$ bằng
	\choice
	{$6$}
	{$3$}
	{\True $9$}
	{$5$}
	\loigiai{
	Ta có \begin{eqnarray*}
		&&\displaystyle\int\limits_1^3[f(x)+3]\mathrm{\,d}x\\
		&=& \displaystyle\int\limits_1^3f(x)\mathrm{\,d}x + \displaystyle\int\limits_1^3 3\mathrm{\,d}x\\
		&=&\displaystyle\int\limits_1^3f(x)\mathrm{\,d}x + 3x\bigg|_1^3\\
		&=& 3+ 6 \\
		&=&9.
	\end{eqnarray*}
	
	}
\end{ex}

%Câu trắc nghiệm 7

\begin{ex}%[Nguồn: Bộ đề minh họa Moon 2024-2025]%[2D4N2-3]
	$\displaystyle\int\limits_{\tfrac{\pi}{7}}^{\tfrac{\pi}{3}} \cos x \mathrm{\,d}x$ bằng
	\choice
	{\True $\sin\dfrac{\pi}{3}-\sin\dfrac{\pi}{7}$}
	{$\sin\dfrac{\pi}{7}-\sin\dfrac{\pi}{3}$}
	{$\cos\dfrac{\pi}{7}-\cos\dfrac{\pi}{3}$}
	{$\cos\dfrac{\pi}{3}-\cos\dfrac{\pi}{7}$}
	\loigiai{
	Ta có $\displaystyle\int\limits_{\tfrac{\pi}{7}}^{\tfrac{\pi}{3}} \cos x \mathrm{\,d}x = \sin x\Bigg|_{\tfrac{\pi}{7}}^{\tfrac{\pi}{3}}=\sin\dfrac{\pi}{3}-\sin\dfrac{\pi}{7}$.
	}
\end{ex}

\begin{ex}%[Nguồn: Bộ đề minh họa Moon 2024-2025]%[2D4N3-1]
	Diện tích $S$ của hình phẳng giới hạn bởi các đường $y=\mathrm{e}^x$, $y=-1$, $x=0$ và $x=1$ được tính bởi công thức nào dưới đây?
	\choice
	{$S=\pi\displaystyle\int\limits_0^1(\mathrm{e}^x+1)\mathrm{\,d}x$}
	{$S=\pi\displaystyle\int\limits_0^1(\mathrm{e}^{2x}+1)\mathrm{\,d}x$}
	{\True $S=\displaystyle\int\limits_0^1(\mathrm{e}^x+1)\mathrm{\,d}x$}
	{$S=\displaystyle\int\limits_0^1(\mathrm{e}^x-1)\mathrm{\,d}x$}
	\loigiai{
	Diện tích $S$ của hình phẳng giới hạn bởi các đường $y=\mathrm{e}^x$, $y=-1$, $x=0$ và $x=1$ được tính bởi công thức
	$$S=\displaystyle\int\limits_0^1|\mathrm{e}^x-(-1)|\mathrm{\,d}x=\displaystyle\int\limits_0^1(\mathrm{e}^x+1)\mathrm{\,d}x.$$
	}
\end{ex}

%

\begin{ex}%[Nguồn: Bộ đề minh họa Moon 2024-2025]%[2D4N3-3]
	Cho hàm số $y=f(x)$ liên tục, nhận giá trị dương trên đoạn $[a; b]$. Xét hình phẳng $(H)$ giới hạn bởi đồ thị hàm số $y=f(x)$, trục hoành và hai đường thẳng $x=a$, $x=b$. Khối tròn xoay được tạo thành khi quay hình phẳng $(H)$ quanh trục $Ox$ có thể tích là
	\choice
	{$V=\pi \displaystyle\int\limits_a^b|f(x)|d x$}
	{$V=\pi^2\displaystyle\int\limits_a^b f(x) d x$}
	{$V=\pi^2\displaystyle\int\limits_a^b[f(x)]^2\mathrm{~d} x$}
	{\True $V=\pi \displaystyle\int\limits_a^b[f(x)]^2\mathrm{~d} x$}
	\loigiai{
	Thể tích khối tròn xoay tạo thành khi quay hình phẳng $(H)$ quanh trục $Ox$ là $\displaystyle V=\pi \int\limits_{a}^{b}[f(x)]^{2} \mathrm{~d} x$.
	}
\end{ex}

\begin{ex}%[Nguồn: Bộ đề minh họa Moon 2024-2025]%[2D4V1-6]
	Mặt cắt ngang của một ống dẫn nước nóng là hình vành khuyên như hình vẽ.
	\begin{center}
		\begin{tikzpicture}[>=stealth, line join=round, line cap=round, font=\footnotesize, scale=1]
			\definecolor{xam}{rgb}{0.2,0.2,0.2}
			\definecolor{xam2}{rgb}{0.4,0.4,0.4}
			\tikzset{declare function={r1=1; r2=r1*1.8;}}
			\shade[inner color=xam2!80, outer color=xam2] (0,0) coordinate (O) circle (r2);
			\fill[white] (0,0) circle (r1);
			\node[text width=1.6cm, below, scale=.9, align=center] at (O) {Nước ở $100^\circ\mathrm{C}$};
			\draw[line width=1pt, xam] (0,0) circle (r1);
			\draw[line width=1pt, xam] (0,0) circle (r2);
			\path ($(O)+(-90:r2*0.75)$) node[text=white, font=\bfseries] {Kim loại};
			\fill (O) circle (1pt);
			\draw[->] (O) -- ($(O)+(45:r1*1.5)$) node[above, pos=.5, sloped] {$x$};
			\path (current bounding box.south) node[below=2mm] {Mặt cắt ống};
		\end{tikzpicture}
	\end{center}
	Bán kính ngoài là $r$ cm và bán kính trong là $2$ cm ($r > 2$). Bên trong ống, nhiệt độ nước được duy trì ở mức $100^\circ\mathrm{C}$. Bên trong kim loại, nhiệt độ giảm dần từ bên trong ra bên ngoài. Biết rằng nhiệt độ $T$ ($^\circ$C) tại điểm $A$ trên thành ống là hàm số của khoảng cách $x$ (cm) từ $A$ đến tâm của mặt cắt và thoả mãn $T'(x) = -\dfrac{10}{x}$, với $2 \le x \le r$.
	\choiceTFt
	{\True $T(x) = -10 \ln x + C$ với $C$ là hằng số}
	{\True $T(x) = 100 -10 \ln \dfrac{x}{2}$}
	{Nếu bán kính ngoài là $r = 4$ cm thì nhiệt độ bề mặt ngoài của ống là $93{,}1^\circ$C (kết quả làm tròn đến hàng phần trăm của $^\circ\mathrm{C}$)}
	{Để nhiệt độ bề mặt ngoài của ống không vượt quá $90^\circ\mathrm{C}$ thì nên thiết kế ống với bán kính ngoài tối thiểu là $5{,}5$ cm (làm tròn kết quả đến hàng phần mười của cm)}
	\loigiai{
	\begin{itemchoice}
		\itemch Ta có  $\displaystyle\int\left(-\dfrac{10}{x}\right) \, \mathrm{d}x =-10\ln x +C$ (với $2\le x \le r$).\\
		Do đó $T(x) = -10 \ln x + C$.
		\itemch Theo giả thiết ta có $T(2) = 100$, suy ra $100 = -10 \ln 2 + C \Rightarrow C = 100 + 10 \ln 2$.\\
		Vậy $T(x) = -10 \ln x + 100 + 10 \ln 2 = 100 - 10 \ln \dfrac{x}{2}$.
		\itemch	Ta có
		\[
		T(4) = 100 - 10 \ln \dfrac{4}{2} = 100 - 10 \ln 2 \approx 100 - 10 \cdot 0{,}693 = 93{,}07^\circ\text{C}.
		\]
		\itemch Sai.
	\end{itemchoice}
	
	Ta có
	\allowdisplaybreaks
	\begin{eqnarray*}
		&&T(x) \le 90\\
		&\Rightarrow &100 - 10 \ln \dfrac{x}{2} \le 90 \\
		&\Leftrightarrow &-10 \ln \dfrac{x}{2} \le -10\\
		&\Leftrightarrow &\ln \dfrac{x}{2} \ge 1\\
		&\Leftrightarrow &\dfrac{x}{2}\ge e\\
		&\Leftrightarrow &x\ge 2\mathrm{e}\approx 5{,}4.
	\end{eqnarray*}
	Vậy bán kính ngoài tối thiểu là $r \approx 5{,}4$ cm.
	}
\end{ex}

%

\begin{ex}%[Nguồn: Bộ đề minh họa Moon 2024-2025]%[2D4V1-6]
	Một xe ô tô đang chạy trên một đoạn đường với tốc độ $72 \mathrm{~km} / \mathrm{h}$ thì người lái xe bất ngờ phát hiện chướng ngại vật trên đường cách đó $80 \mathrm{~m}$. Người lái xe phản ứng một giây sau đó bằng cách đạp phanh khẩn cấp. Kể từ thời điểm này, ô tô chuyển động chậm dần đều với tốc độ $v(t)=a t+b$ $(\mathrm{m} / \mathrm{s})$, trong đó $t$ là thời gian tính bằng giây kể từ lúc đạp phanh. Rất may mắn xe đã dừng hẳn cách chướng ngại vật một khoảng $40 \mathrm{~m}$ nên không xảy ra va chạm.
	\choiceTF
	{\True Quãng đường xe ô tô đã di chuyển kể từ khi đạp phanh đến khi dừng hẳn là $40$ mét}
	{\True Giá trị của $b$ là $20$}
	{Thời gian kể từ lúc đạp phanh đến khi xe ô tô dừng hẳn là $3$ giây}
	{Vận tốc trung bình của ô tô từ lúc phát hiện chướng ngại vật đến khi dừng hẳn là $15{,}3$ $(\mathrm{m} / \mathrm{s})$ (làm tròn kết quả đến hàng phần trăm)}
	\loigiai{
	\begin{enumerate}[a)]
		\item Quãng đường xe ô tô đã di chuyển kể từ khi đạp phanh đến khi dừng hẳn là $80-40=40$ mét.
		\item Ta có $s(t)=\displaystyle\int v(t) \mathrm{\,d}t=\dfrac{at^2}{2}+bt+c$.\\
		Ta có $72 \mathrm{~km} / \mathrm{h}=20\mathrm{~m} / \mathrm{s}$.\\
		Giải hệ $\heva{&s(0)=0\\&v(0)=20\\&s\left(-\dfrac{b}{a}\right)=40}\Leftrightarrow \heva{&c=0\\&b=20\\&\dfrac{a\cdot\left(-\dfrac{20}{a}\right)^2}{2}+20\cdot\left(-\dfrac{20}{a}\right)=40}\Leftrightarrow \heva{&c=0\\&b=20\\&a=-5.}$
		\item Suy ra $v(t)=-5t+20=0\Leftrightarrow t=4$ $(s)$.
		\item Vận tốc trung bình là $\dfrac{40}{4}=10$ $(\mathrm{m} / \mathrm{s})$.
	\end{enumerate}}
\end{ex}

%

\begin{ex}%[Nguồn: Bộ đề minh họa Moon 2024-2025]%[2D4V1-6]
	Một tấm ván gỗ chỉ được hỗ trợ ở hai đầu $O$ và $P$, cách nhau $4$ (m). Tấm ván võng xuống dưới do trọng lượng của nó tạo thành một đường cong. Xét trên hệ trục $Oxy$ như hình vẽ dưới, đơn vị mỗi trục là (mét), đường cong trong hình vẽ có phương trình $y=f(x)$.
	\begin{center}
		\begin{tikzpicture}[>=stealth,line join=round,line cap=round,font=\footnotesize,scale=1]
			\draw[line width=1pt, ->] (-1,0)--(4.5,0) node [below] {$x$};
			\draw[line width=1pt, ->] (0,-1)--(0,1) node [right] {$y$};
			\coordinate[label=above left:$O$] (O)at(0,0);
			\coordinate[label=above :$P$] (P)at(4,0);
			\coordinate[label=above left:$ $] (B)at(2,-0.3);
			\coordinate[label=above left:$ $] (C)at(4.2,0.1);
			\coordinate[label=above left:$ $] (A)at(-0.3,0.1);
			\draw[line width=1.3pt] plot[smooth,tension=0.8] coordinates{(A) (B) (C)};
			\fill[black] (0,0) circle (3pt) ;
			\fill[black] (0,0)--(-0.3,-0.5)--(0.3,-0.5)--cycle;
			\fill[black] (3.9,0) circle (3pt) ;
			\fill[black] (3.9,0)--(3.6,-0.5)--(4.2,-0.5)--cycle;
			\draw (2,-0.3) node[below] {$y=f(x)$};
		\end{tikzpicture}
	\end{center}
	Người ta chứng minh được $f''(x)=\dfrac{1}{100}\left(2x-\dfrac{x^2}{2}\right)$ với $0\leq x \leq 4$. Tại điểm cách điểm $P$ một khoảng $1$ (mét), tấm ván bị võng xuống bao nhiêu (cm)? (làm tròn kết quả đến hàng phần trăm).
	\shortans[1]{$2{,}38$}
	\loigiai{
	Ta có $\displaystyle  f'(x)=\int f''(x)\mathrm{\,d}x=\int \dfrac{1}{100}\left(2x-\dfrac{x^2}{2}\right)\mathrm{\,d}x=\dfrac{1}{100}\left(x^2-\dfrac{x^3}{6}\right)+C_1$.\\
	Suy ra $\displaystyle f(x)=\int f'(x)\mathrm{\,d}x=\dfrac{1}{100}\left(\dfrac{x^3}{3}-\dfrac{x^4}{24}\right)+C_1x+C_2$.\\
	Vì $f(0)=0$ và $f(4)=0$ nên ta có
	\[
	\heva{
	&\dfrac{1}{100}\left(\dfrac{0^3}{3}-\dfrac{0^4}{24}\right)+C_1\cdot 0+C_2=0\\
	&\dfrac{1}{100}\left(\dfrac{4^3}{3}-\dfrac{4^4}{24}\right)+C_1\cdot 4+C_2=0
	}\Leftrightarrow \heva{
	&C_2=0\\
	&C_1=-\dfrac{2}{75}.
	}
	\]
	Vậy $f(x)=\dfrac{1}{100}\left(\dfrac{x^3}{3}-\dfrac{x^4}{24}\right)-\dfrac{2}{75}x$.\\
	Suy ra $f(3)=-\dfrac{19}{800}$ (m).\\
	Tại điểm cách $P$ $1$ (m), tấm ván bị võng xuống khoảng $2{,}38$ (cm).
	}
\end{ex}

\begin{ex}%[Nguồn: Bộ đề minh họa Moon 2024-2025]%[2D4V1-6]
	Một người bình thường với chiều cao $h$ cm, nặng $w$ kilogram có diện tích bề mặt cơ thể $S$ được mô hình hoá bởi công thức $S=\dfrac{1}{60}\cdot w^{0{,}5}\cdot h^{0{,}5}$ (m$^2$) (công thức Mosteller).\\
	Một đối tượng có chiều cao bằng $168$ cm, nặng $62$ kg tham gia một cuộc nghiên cứu về sức khỏe trong $5$ năm. Người ta nhận thấy cân nặng của đối tượng quan sát thay đổi với tốc độ $w'(t)=0{,}02t^2+0{,}2t$ kg/năm ($0\le t\le 5$) và chiều cao tăng đều mỗi năm $0{,}5$ cm. Sau $5$ năm quan sát, diện tích bề mặt cơ thể của đối tượng trên tăng thêm bao nhiêu centimet vuông so với ban đầu? (làm tròn kết quả đến hàng đơn vị).
	\par\shortans{581}
	\loigiai{
	Chiều cao của người này sau $5$ năm là
	\begin{center}
		$h_5=h_0+0{,}5t=168+5\times 0{,}5=170{,}5$ cm $=1{,}705$ m.
	\end{center}
	Cân nặng của người ngày sau $5$ năm là
	$$w_5=w_0++\displaystyle\int\limits_0^5\left(0{,}02t^2+0{,}2t\right)\mathrm{\, d}t=62+\dfrac{10}{3}=\dfrac{196}{3}~\mathrm{kg}.$$
	Diện tích bề mặt cơ thể sau $5$ năm tăng thêm là
	$$\begin{aligned}
		\Delta S &= S_5-S_0=\dfrac{1}{60}w_5^{0{,}5}\cdot h_5^{0{,}5}-\dfrac{1}{60}w_0^{0{,}5}\cdot h_0^{0{,}5}\\&=\dfrac{1}{60}\cdot\left(\dfrac{196}{3}\right)^{0{,}5}\cdot 1{,}705^{0{,}5}-\dfrac{1}{60}\cdot62^{0{,}5}\cdot 1{,}68^{0{,}5}=58{,}1\cdot 10^{-3}~\mathrm{m^2}\approx 581~\mathrm{cm^2}
	\end{aligned}$$
	}
\end{ex}

\begin{ex}%[Nguồn: Bộ đề minh họa Moon 2024-2025]%[2D4V2-6]
	Một quần thể vi khuẩn $(A)$ có số lượng cá thể là $P(t)$ sau $t$ phút quan sát được phát hiện thay đổi với tốc độ là $P'(t) = a\mathrm{e}^{0{,}1t} + 150\mathrm{e}^{-0{,}03t}$ (vi khuẩn/phút) $(a \in \mathbb{R})$. Biết rằng lúc bắt đầu quan sát, quần thể có $200\,000$ vi khuẩn và đạt tốc độ tăng trưởng là $350$ vi khuẩn/phút.
	\choiceTF
	{\True Giá trị của $a = 200$}
	{$P(t)=2000\mathrm{e}^{0.1 t} - 5000\mathrm{e}^{-0.03t}+ 200\,000$}
	{\True Sau $12$ phút, số lượng vi khuẩn trong quần thể là $206\,152$ con (làm tròn kết quả đến hàng đơn vị)}
	{Sau $12$ phút, một quần thể vi khuẩn $(B)$ có tốc độ tăng trưởng là $G'(t) = 500\mathrm{e}^{0{,}2t}$ (vi khuẩn/phút) bắt đầu cạnh tranh nguồn thức ăn trực tiếp với quần thể $(A)$, một cá thể tại quần thể $(B)$ triệt tiêu một cá thể tại quần thể $(A)$. Sau 5 phút cạnh tranh, quần thể $(A)$ bị triệt tiêu hoàn toàn. Số lượng vi khuẩn của quần thể $(B)$ ở thời điểm bắt đầu cạnh tranh là $191\,967$ con (làm tròn kết quả đến hàng đơn vị)}
	\loigiai{Vì tốc độ thay đổi là $P'(t) = a\mathrm{e}^{0{,}1t} + 150\mathrm{e}^{-0{,}03t}$ nên số lượng cá thể vi khuẩn là \[P(t)=\displaystyle\int\limits_0^t\left(a\mathrm{e}^{0{,}1t} + 150\mathrm{e}^{-0{,}03t}\right)\mathrm{d}t=10a\mathrm{e}^{0{,}1t} + 5000\mathrm{e}^{-0{,}03t}+C.\]
	
	\begin{itemchoice}
		
		\itemch \textbf{Đúng}.\\
		Lúc đầu quan sát vi khuẩn có tốc độ tăng trưởng là $350$ vi khuẩn /phút nên ta có $P'(0)=350\Leftrightarrow a=200$.
		\itemch \textbf{Sai}.\\
		Với $a=200$ và lúc đầu có $200\,000$ con vi khuẩn nên ta có \[P(0)=200\,000\Leftrightarrow 2000-5000+C=200\,000\Leftrightarrow C=203\,000.\]
		Vậy ta có $P(t)=2000\mathrm{e}^{0.1 t} - 5000\mathrm{e}^{-0.03t}+ 203\,000$.
		\itemch \textbf{Đúng}.\\
		Với $P(t)=2000\mathrm{e}^{0.1 t} - 5000\mathrm{e}^{-0.03t}+ 203\,000$ nên $P(12)=206\,152$.
		\itemch \textbf{Sai}.\\
		Số lượng vi khuẩn ở quần thể B là $G(t)=\displaystyle\int 500\mathrm{e}^{0{,}2t}\mathrm{d}t=2500\mathrm{e}^{0{,}2t}+C_1$.\\
		Sau $5$ phút cạnh tranh số lượng hai quần thể bằng nhau nên ta có $$P(17)=G(5)\Leftrightarrow 2500\mathrm{e}+C_2=210\,945\Leftrightarrow C_2=204\,149.$$
		Vậy số lượng vi khuẩn ở quần thể B lúc bắt đầu cạnh tranh là $G(0)=206\,649$.
	\end{itemchoice}
	}
\end{ex}

\begin{ex}%[Nguồn: Bộ đề minh họa Moon 2024-2025]%[2D4V2-6]
	\immini{
	Hình vẽ bên mô tả hiệu suất làm việc của hai công nhân trong một nhà máy trong thời gian 6 giờ. Công nhân $A$ đang sàn xuất với hiệu suất $Q_1'(t)=-2t^2+4t+58$ sản phẩm mỗi giờ, trong khi công nhân $B$ đang sản xuất với hiệu suất $Q_2'(t)=53+a t$ sản phẩm mỗi giờ $(a \in \mathbb{R})$. Biết rằng hàm $Q_1(t)$ và $Q_2(t)$ mô phỏng số lượng sản phẩm mới làm được của công nhân $A$.
	}
	{
	\begin{tikzpicture}[line cap=butt,line join=miter,>=stealth]
		\tikzset{declare function={xmin=-2;xmax=5;ymin=-1;ymax=5;
		f(\x)=-0.42*(\x)^2+0.25*(\x)+4; g(\x)=-0.33*(\x)+3;
		},
		smooth,samples=450
		}
		\draw[->] (xmin,0)--(xmax,0) node[shift={(-100:12pt)},font=\normalsize]{$t(h)$};
		\draw[->] (0,ymin)--(0,ymax) node[shift={(190:15pt)},font=\normalsize]{$Q'(t)$};
		\fill (0,0) node[shift={(225:7pt)},font=\normalsize]{$ O $};
		\fill (3,1) circle (1pt); \fill (3,2) circle (1pt); \fill (2.4,2.2) circle (1pt); \fill (0,3) circle (1pt); \fill (0,4) circle (1pt);
		\draw [dashed] (3,0)node[below]{$6$}--(3,1) (2.4,0)node[below]{$5$}--(2.4,2.2);
		\fill [pattern=north east lines] plot[domain=0:3] (\x,{f(\x)})--plot[domain=3:0](\x,{g(\x)}) ;
		\clip (xmin,ymin) rectangle (xmax,ymax);
		\draw  plot[domain=0:3] (\x, {f(\x)})node[right]{$Q_1'(t) $};
		\draw  plot[domain=0:3] (\x, {g(\x)})node[right]{$Q_2'(t) $};
	\end{tikzpicture}
	}
	\choiceTF
	{\True Hiệu suất cực đại của công nhân $A$ là $60$ sản phẩm mỗi giờ}
	{Phần diện tích tô đậm biểu diễn cho tổng số lượng sản phẩm mới mà $2$ công nhân làm được trong $6$ giờ}
	{\True Sau $5$ giờ số lượng sản phẩm mới mà công nhân $A$ hoàn thành nhiều hơn công nhân $B$ là $54$ sản phẩm}
	{Sau $6$ giờ làm việc tổng số lượng sản phẩm mới mà $2$ công nhân hoàn thành là $502$ sản phẩm}
	\loigiai{
	\begin{itemchoice}
		\itemch Ta có $(Q_1'(t))'=-4t+4$.\\
		$(Q_1'(t))'=0\Leftrightarrow t=1$.
		\begin{center}
			\begin{tikzpicture}[font=\normalsize,t style/.style={style=solid}]
				\tkzTabInit[nocadre=true, lgt=1.8, espcl=2.5, deltacl=0.5]%,help]
				{$t$/0.75, $(Q_1'(t))'$/0.75, $Q_1'(t)$/3}
				{$0$, $1$, $6$}
				\tkzTabLine{ , +, 0,-}
				\path
				($(N13)!0.1!(N12)$) node (A1){$58$}
				($(N23)!0.8!(N22)$) node (A2){$60$}
				($(N33)!0.1!(N32)$) node (A3){$10$};
				\foreach \x/\y in {A1/A2, A2/A3}{
				\draw[-stealth] (\x)--(\y);
				}
			\end{tikzpicture}
		\end{center}
		Vậy hiệu suất cực đại của công nhân $A$ là $60$ sản phẩm mỗi giờ.
		\itemch Diện tích phần tô đậm là $S=\displaystyle\int\limits_0^6\left|Q_1'(t)-Q_2'(t)\right|\mathrm{\,d}t$ nên nó không biểu diễn cho tổng số lượng sản phẩm mới mà $2$ công nhân làm được trong $6$ giờ.
		\itemch
		Dựa vào đồ thị, ta có $Q'_1(5)=Q'_2(5)\Leftrightarrow -2\cdot 5^2+4\cdot 5+58=53+5a\Leftrightarrow a=-5$.\\
		Suy ra $Q'_2(5)=53-5t$.\\
		Sau $5$ giờ, số lượng sản phẩm mới của công nhân $A$ hoàn thành nhiều hơn số lượng sản phẩm mới của công nhân $B$ bằng
		\allowdisplaybreaks
		\begin{eqnarray*}
			\displaystyle\int\limits_0^5 Q_1'(t)\mathrm{\,d}t-\displaystyle\int\limits_0^5 Q_2'(t)\mathrm{\,d}t&=&\displaystyle\int\limits_0^5 (-2t^2 + 4t + 58)\mathrm{\,d}t-\displaystyle\int\limits_0^5 (53-5t)\mathrm{\,d}t
			\\
			&\approx& 54.
		\end{eqnarray*}
		\itemch
		Sau $6$ giờ làm việc tổng số lượng sản phẩm mới mà $2$ công nhân hoàn thành bằng
		\allowdisplaybreaks
		\begin{eqnarray*}
			\displaystyle\int\limits_0^5 Q_1'(t)\mathrm{\,d}t+\displaystyle\int\limits_0^5 Q_2'(t)\mathrm{\,d}t&=&\displaystyle\int\limits_0^5 (-2t^2 + 4t + 58)\mathrm{\,d}t+\displaystyle\int\limits_0^5 (53-5t)\mathrm{\,d}t
			\\
			&=& 276+228=505.
		\end{eqnarray*}
	\end{itemchoice}
	}
\end{ex}

%

\begin{ex}%[Nguồn: Bộ đề minh họa Moon 2024-2025]%[2D4V3-2]
	Một người điều khiển ô tô đang ở đường dẫn muốn nhập làn vào đường cao tốc. Khi ô tô cách điểm nhập làn $200$ m, tốc độ của ô tô là $36\mathrm{~km} / h$. Hai giây sau đó, ô tô bắt đầu tăng tốc với tốc độ $v(t)=a t+b(\mathrm{~m}/s)$ $(a, b \in \mathbb{R}; a > 0)$, trong đó $t$ là thời gian tính bằng giây kể từ khi bắt đầu tăng tốc. Biết rằng ô tô nhập làn cao tốc sau $12$ giây và duy trì sự tăng tốc trong $24$ giây kể từ khi bắt đầu tăng tốc.
	\choiceTF
	{\True Quãng đường ô tô đi được từ khi bắt đầu tăng tốc đến khi nhập làn là $180$ m}
	{\True Giá trị của $b$ là $10$}
	{Quãng đường $S(t)$ (đơn vị: mét) mà ô tô đi được trong thời gian $t$ giây $(0\leq t \leq 24)$ kể từ khi tăng tốc được tính theo công thức $S(t)=\displaystyle\int\limits_0^{24} v(t) \mathrm{d} t$}
	{Sau 24 giây kể từ khi tăng tốc, tốc độ của ô tô không vượt quá tốc độ tối đa cho phép là $100\mathrm{~km} / h$}
	\loigiai{
	\begin{enumerate}
		\item Đúng.
		
		Ta có $36$ km/h $=10 $ m/s.
		
		Quãng đường ô tô đi được trước khi bắt đầu tăng tốc là $10 \cdot 2=20$ m. Suy ra quãng đường ô tô đi được từ khi bắt đầu tăng tốc đến khi nhập làn là $200-20=180$ m.
		
		\item Đúng.
		
		Ta có $v(0)=a \cdot 0+b=b$ mà $v(0)=10$.
		Do đó, $b=10$.
		
		\item Sai.
		
		Quãng đường di chuyển trong khoảng thời gian từ $t_1 $ đến $t_2 $ là
		$S(t)=\displaystyle\int\limits_{t_1}^{t_2} v(t) \mathrm{d} t$.
		Vậy quãng đường $S(t)$ mà ô tô đi được trong thời gian $t$ giây kể từ khi tăng tốc là $S(t)=\displaystyle\int\limits_{0}^{t} v(t) \mathrm{d} t$.
		
		\item Sai.
		
		Vì ô tô nhập làn cao tốc sau $12$ giây nên ta được
		$$
		\begin{aligned}
			&\displaystyle\int\limits_{0}^{12} v(t) \mathrm{d} t  =180 \\
			\Leftrightarrow& \displaystyle\int\limits_{0}^{12} a t+10 \mathrm{~d} t  =180 \\
			\Leftrightarrow& \left.\left(a \cdot \dfrac{t^{2}}{2}+10 t\right) \right|_{0} ^{12}  =180 \\
			\Leftrightarrow& 72 a+120  =180 \\
			\Leftrightarrow& a  =\dfrac{5}{6}.
		\end{aligned}
		$$
		Khi đó, $v(t)=\dfrac{5}{6} t+10$. Suy ra $v(24)=\dfrac{5}{6} \cdot 24+10=30$ km/s.
		
		Vậy sau $24$ giây kể từ khi tăng tốc, tốc độ của ô tô là $30$ m/s $=108$ km/h.
	\end{enumerate}
	}
\end{ex}

\begin{ex}%[Nguồn: Bộ đề minh họa Moon 2024-2025]%[2D4V3-2]
	\immini[thm]{Kiến trúc sư thiết kế một khu sinh hoạt cộng đồng có dạng hình vuông $ABCD$ có độ dài đường chéo $AC = 120 \, \text{m}$. Trong đó, phần được tô màu đậm là sân chơi, phần còn lại để trồng hoa. Mỗi phần trồng hoa có đường biên cong là một phần của parabol với đỉnh thuộc mặt trục đối xứng của hình vuông, khoảng cách từ đỉnh đến đường biên cong của hình vuông bằng $40$ m và $AM = MN = NB$ (xem hình minh họa).
	Diện tích của phần sân chơi là bao nhiêu mét vuông? (Làm tròn kết quả đến hàng đơn vị).}{\begin{tikzpicture}[line join=round, line cap=round,>=stealth,thick,scale=0.5]
		\tikzset{every node/.style={scale=0.9}}
		\fill[color=blue!40](0,-6)--(6,0)--(0,6)--(-6,0)--cycle;
		\fill[color=white] (0,-6)--(-2,-4)--plot[samples=200,domain=-2:2,smooth,variable=\x] (\x,{-0.5*(\x)^2-2})--(2,-4)--cycle;
		\fill[color=white] (0,6)--(-2,4)--plot[samples=200,domain=-2:2,smooth,variable=\x] (\x,{0.5*(\x)^2+2})--(2,4)--cycle;
		\fill[color=white] (6,0)--(4,-2)--plot[samples=200,domain=-2:2,smooth,variable=\x] ({0.5*(\x)^2+2},\x)--(4,2)--cycle;
		\fill[color=white] (-6,0)--(-4,-2)--plot[samples=200,domain=-2:2,smooth,variable=\x] ({-0.5*(\x)^2-2},\x)--(-4,2)--cycle;
		\draw[<->] (0,-6)--(0,-2)node[midway,right] {\small$40$m};
		\draw[<->] (0,6)node[above]{$A$}--(0,2)node[midway,right] {\small$40$m};
		\draw[<->] (-6,0)node[left]{$D$}--(-2,0)node[midway,above] {\small$40$m};
		\draw[<->] (6,0)node[right]{$B$}--(2,0)node[midway,above] {\small$40$m};
		\draw [](0,6)--(6,0)--(0,-6)node[below]{$C$}--(-6,0)--cycle;
		\draw (4,2)node[above right]{$N$} (2,4)node[above right]{$M$};
		\begin{scope}
			\clip (-6,-6) rectangle (6,6);
			\draw[samples=200,domain=-2:2,smooth,variable=\x] plot (\x,{0.5*(\x)^2+0*(\x)+2});
			\draw[samples=200,domain=-2:2,smooth,variable=\x] plot (\x,{-0.5*(\x)^2-2});
			\draw[samples=200,domain=-2:2,smooth,variable=\x] plot ({0.5*(\x)^2+0*(\x)+2},\x);
			\draw[samples=200,domain=-2:2,smooth,variable=\x] plot ({-0.5*(\x)^2-2},\x);
		\end{scope}
	\end{tikzpicture}}
	\shortans[0]{$3467$}
	\loigiai{
	Thiết lập hệ trục tọa độ $Oxy$ với $O$ là trung điểm của $BD$, tia $Ox$ trùng với tia $OB$, tia $Oy$ trùng với tia $OA$ với mỗi đơn vị là $1$ m.\\
	\begin{center}
		\begin{tikzpicture}[line join=round, line cap=round,>=stealth,thick,scale=0.5]
			\tikzset{every node/.style={scale=0.9}}
			\fill[color=blue!40](0,-6)--(6,0)--(0,6)--(-6,0)--cycle;
			\fill[color=white] (0,-6)--(-2,-4)--plot[samples=200,domain=-2:2,smooth,variable=\x] (\x,{-0.5*(\x)^2-2})--(2,-4)--cycle;
			\fill[color=white] (0,6)--(-2,4)--plot[samples=200,domain=-2:2,smooth,variable=\x] (\x,{0.5*(\x)^2+2})--(2,4)--cycle;
			\fill[color=white] (6,0)--(4,-2)--plot[samples=200,domain=-2:2,smooth,variable=\x] ({0.5*(\x)^2+2},\x)--(4,2)--cycle;
			\fill[color=white] (-6,0)--(-4,-2)--plot[samples=200,domain=-2:2,smooth,variable=\x] ({-0.5*(\x)^2-2},\x)--(-4,2)--cycle;
			\draw [](0,6)--(6,0)--(0,-6)node[below]{$C$}--(-6,0)--cycle;
			\draw (4,2)node[above right]{$N$} (2,4)node[above right]{$M$};
			\begin{scope}
				\clip (-6,-6) rectangle (7.8,7.2);
				\draw[samples=200,domain=-2.6:2.6,smooth,variable=\x] plot (\x,{0.5*(\x)^2+0*(\x)+2})node[above right]{$y=\dfrac{1}{20}\cdot x^2+20$};
				\draw[samples=200,domain=-2:2,smooth,variable=\x] plot (\x,{-0.5*(\x)^2-2});
				\draw[samples=200,domain=-2:2,smooth,variable=\x] plot ({0.5*(\x)^2+0*(\x)+2},\x);
				\draw[samples=200,domain=-2:2,smooth,variable=\x] plot ({-0.5*(\x)^2-2},\x);
			\end{scope}
			\draw [->] (0,-6)--(0,8)node[left]{$y$};
			\draw [->] (-6,0)--(8,0)node[below]{$x$};
			\draw (0,2)node[above left]{$I$} (0,6)node[above right]{$A$} (6,0)node[above right]{$B$} (-6,0) node[above left]{$D$};
			\draw [dashed] (2,0)node[below left]{$20$}--(2,4)--(0,4)node[left]{$40$};
		\end{tikzpicture}
	\end{center}
	Từ giả thiết ta suy ra $A(0; 60)$, $B(60; 0)$, $M(20;40)$.\\
	Vì $I(0; 20)$ là đỉnh của parabol có bề lõm hướng lên trên nên ta suy ra parabol đi qua điểm $M$ và có đỉnh $I$ có phương trình $y=\dfrac{1}{20}\cdot x^2+20$.\\
	Phần diện tích hình phẳng giới hạn bởi parabol trên và hai đường thẳng $AD$ và $AB$ là
	\[S=2\displaystyle\int\limits_0^{20}\left(60-x-\dfrac{1}{20}x^2-20\right)=\dfrac{2800}{3}.\]
	Từ đó suy ra phần diện tích sân chơi bằng $$AB^2-4S=7200-4\cdot \dfrac{2800}{3}=\dfrac{10400}{3}\approx 3467\, \mathrm{m}^2.$$
	}
\end{ex}

\begin{ex}%[Nguồn: Bộ đề minh họa Moon 2024-2025]%[2D4V3-2]
	\immini{Từ một tấm tôn phẳng hình chữ nhật có chiều dài $8$ cm, chiều rộng $5$ cm có gắn hệ toạ độ $Oxy$ như hình vẽ bên. Thầy Tuấn cắt miếng tôn theo ba đường: Đường cong $AIB$ là một phần của Parabol, các đường cong $AE$, $EB$ là một phần đồ thị hàm số bậc ba. Trang trí phần còn lại để tạo thành một chiếc mặt nạ đồ chơi có trục đối xứng là trục $Oy$. Biết đường cong $EB$ đi qua các điểm $(1;-2)$ và $(3;-3)$.}{\begin{tikzpicture}[very thick,>=stealth',scale=0.9]
		\tikzset{declare function={xmin=-4;xmax=4;
		ymin=-4;ymax=1;
		f(\x)=1 - (\x)^2/16;
		g(\x)=1/2*(\x)^3-17*(\x)^2/6+13*(\x)/3-4;
		},
		smooth,samples=50
		}
		\draw[black,thick] (-4,-4) rectangle (4,1);
		\draw (-2,-1) circle (0.5 cm);
		\draw (2,-1) circle (0.5 cm);
		\draw[->] (xmin-0.25,0)--(xmax+0.5,0)
		node[shift={(-100:7pt)},font=\normalsize]{$x$};
		\draw[->] (0,ymin-0.25)--(0,ymax+0.5)
		node[shift={(170:7pt)},font=\normalsize]{$y$};
		\fill (0,0) node[shift={(135:9pt)},font=\normalsize]{$O$};
		\foreach \x in {-4, 4}{
		\draw (\x,2pt)--(\x,-2pt) +(0,-9pt) node[shift={(-10:5pt)},font=\footnotesize,fill=white,inner sep=1pt]{$\x$};
		}
		\foreach \y in {-4, 1}{
		\draw (2pt,\y)--(-2pt,\y) +(-3pt,0) node[shift={(135:9pt)},font=\footnotesize,fill=white,inner sep=1pt]{$\y$};
		}
		\begin{scope}
			\clip (xmin,ymin) rectangle (xmax,ymax);
			\draw[black,thick] plot[domain=xmin:xmax] (\x, {f(\x)});
			\draw[black,thick] plot[domain=xmin:xmax] (\x, {g(\x)});
			\draw[black,thick] plot[domain=xmin:xmax] (\x, {g(-\x)});
		\end{scope}
		\fill
		(4,0)circle(1.5pt)node[above right]{$B$}
		(-4,0)circle(1.5pt)node[above left]{$A$}
		(0,1)circle(1.5pt)node[above right]{$I$}
		(0,-4)circle(1.5pt)node[above right]{$E$}
		;
	\end{tikzpicture}}
	\noindent Tính diện tích chiếc mặt nạ đồ chơi của thầy Tuấn (làm tròn đến hàng phần mười theo đơn vị cm$^2$).
	
	\shortans[oly]{24{,}9}
	\loigiai{
	Giả sử đường cong $AIB$ có phương trình là $y = f(x) = mx^2 + nx + p$.\\
	Đường cong $EB$ có phương trình là $y = g(x) = ax^3 + bx^2 + cx + d$. \\
	Vì mặt nạ đối xứng qua trục $Oy$ nên diện tích của mặt nạ bằng $ 2\displaystyle\int_{0}^{4} |f(x) - g(x)|\mathrm{\,d}x$. \\
	\textbf{Viết phương trình của $f(x)$: }\\
	Ta có đường cong $AIB$ đi qua các điểm $A(-4; 0)$; $I(0; 1)$ và $B(4; 0)$. Từ đó ta có hệ phương trình
	$$\heva{&m\cdot(-4)^2-4n+p=0\\&p=1\\&m\cdot4^2+4n+p=0}\Leftrightarrow\heva{&m=-\dfrac{1}{16}&\\&n=0\\&p=1}\Rightarrow f(x)=-\dfrac{1}{16}x^2+1.$$
	\textbf{Viết phương trình của $g(x)$: }\\
	Vì đồ thị $g(x)$ đi qua $E(0;-4)$ nên $d=-4$. Suy ra $g(x)=ax^3+bx^2+cx-4$.\\
	Lại có đường cong $EB$ qua $B(4;0)$ và các điểm $(1;-2)$ và $(3;-3)$. Từ đó ta có hệ phương trình
	$$\heva{&a\cdot4^3+b\cdot4^2+4c-4=0\\&a+b+c-4=-2\\&a\cdot3^3+b\cdot3^2+3c-4=-3}\Leftrightarrow\heva{&64a+16b+4c=4\\&a+b+c=2\\&27a+9b+3c=1}\Leftrightarrow \heva{&a=\dfrac{1}{2}\\&b=-\dfrac{17}{6}\\&c=\dfrac{13}{3}.}$$
	Suy ra $g(x)=\dfrac{1}{2}x^3-\dfrac{17}{6}x^2+\dfrac{13}{3}x-4$.\\
	Suy ra diện tích mặt nạ là
	$$2\displaystyle\int_{0}^{4} \left| -\dfrac{1}{16}x^2+1 - \dfrac{1}{2}x^3+\dfrac{17}{6}x^2-\dfrac{13}{3}x+4\right| \mathrm{\,d}x\approx24{,}9.$$
	}
\end{ex}

\begin{ex}%[Nguồn: Bộ đề minh họa Moon 2024-2025]%[2D4V3-2]
	\immini[thm]{
	Một bức tường hình chữ nhật $ABCD$ có kích thước $6$ m $\times$ $4$ m, được bạn An trang trí bằng cách vẽ hai đồ thị $f(x)=a^x$ và $g(x)=\log_a x$ ($a$ là số dương và khác $1$) đối xứng nhau qua đường thẳng $d: y=x$ và chia thành ba phần (tham khảo hình vẽ bên dưới). Phần $H_1$ được sơn màu xanh da trời, phần $H_2$ được sơn màu vàng, phần $H_3$ được sơn màu xanh lá cây. Biết rằng mỗi hộp sơn các màu chỉ sơn được $3$ m$^2$ tường, đồng thời giá của hộp sơn màu xanh da trời là $100\,000$ đồng/hộp, hộp sơn vàng là $140\,000$ đồng/hộp, hộp sơn xanh lá cây là $130\,000$ đồng/hộp. Tính giá tiền bạn Hà mua để sơn bức tường này (đơn vị là triệu đồng và cửa hàng sơn chỉ bán số nguyên hộp).
	}{
	\begin{tikzpicture}[>=stealth, line join=round, line cap=round, font=\footnotesize, scale=0.6]
		% Định nghĩa hàm số
		\tikzset{declare function={
		x1=-3.5; x2=4.5; y1=-3.5; y2=3.5; % Giới hạn trục
		a=sqrt(3); % Cơ số
		f(\x)=a^(\x); %
		g(\x)=ln(\x)/ln(a);
		h(\x)=\x;
		}}
		
		\fill[yellow](-3,-2)rectangle(3,2);
		\begin{scope}
			\clip (x1,y1) rectangle (x2,y2);
			\fill[smooth, samples=100, domain=-3:1, thick, cyan] plot(\x,{f(\x)})--(1.2,2)--(-3,2);
			\fill[smooth, samples=100, domain=.33:3, thick, green] plot(\x,{g(\x)})--(3,2)--(3,-2);
			
			\draw[smooth, samples=100, domain=x1:x2, thick, cyan] plot(\x,{f(\x)});
			\draw[smooth, samples=100, domain=0.01:x2, thick, green] plot(\x,{g(\x)}) ;
			\draw[smooth, samples=100, domain=x1:x2, thick, red] plot(\x,{h(\x)});
			\draw[thick](-3,-2)rectangle(3,2);
		\end{scope}
		
		\draw[->] (0,y1) -- (0,y2) node[left] {$y$};
		\draw[->] (x1,0) -- (x2,0) node[above] {$x$};
		\filldraw (0,0) circle (1.2pt) node[above left=-1mm] {$O$};
	\end{tikzpicture}
	}
	
	\shortans[oly]{1150}
	\loigiai{
	Từ hình vẽ ta thấy đồ thị hàm số $g(x) = \log_b x$ đi qua $ (3; 2)$ nên
	$$
	2 = \log_b 3 \Leftrightarrow  b^2 = 3 \Leftrightarrow  b = \sqrt{3} \,\, (\text{do } b > 0).
	$$
	Khi đó $g(x) = \log_{\sqrt{3}} x$.\\
	Mặt khác đồ thị hàm số $f(x)$ và $g(x)$ đối xứng nhau qua $y = x$ nên $	f(x) = \left(\sqrt{3}\right)^x$.\\
	Hoành độ giao điểm của đồ thị hàm số $y = f(x)$ và $y = 2$ là nghiệm của phương trình:
	$$
	f(x) = 2 \Leftrightarrow  \left(\sqrt{3}\right)^x = 2 \Leftrightarrow  x = \log_{\sqrt{3}} 2.
	$$
	Hoành độ giao điểm của đồ thị hàm số $y = g(x)$ và $y = -2$ là nghiệm của phương trình:
	$$
	g(x) = -2 \Leftrightarrow  \log_{\sqrt{3}} x = -2 \Leftrightarrow  x = \left(\sqrt{3}\right)^{-2} = \dfrac{1}{3}.
	$$
	Diện tích phần $H_1$ là $S_1$:
	$$
	S_1 = \displaystyle\int\limits_{-3}^{\log_{\sqrt{3}} 2} \left(2 - \left(\sqrt{3}\right)^x\right) \,\mathrm{d}x \approx 5{,}23 \, (\text{m}^2).
	$$
	Như vậy ta cần dùng $2$ hộp sơn màu xanh da trời.\\
	Diện tích phần $H_3$ là $S_3$:
	$$
	S_3 = \displaystyle\int\limits_{\tfrac{1}{3}}^{3} \left(\log_{\sqrt{3}} x + 2\right) \,\mathrm{d}x \approx 7{,}15 \, (\text{m}^2).
	$$
	Như vậy ta cần dùng $3$ hộp sơn màu xanh lá cây.\\
	Diện tích phần $H_2$ là $S_2$:
	$$
	S_2 = 24 - \displaystyle\int\limits_{-3}^{\log_{\sqrt{3}} 2} \left(2 - \left(\sqrt{3}\right)^x\right) \,\mathrm{d}x - \displaystyle\int\limits_{\tfrac{1}{3}}^{3} \left(\log_{\sqrt{3}} x + 2\right) \,\mathrm{d}x \approx 11{,}62 \, (\text{m}^2).
	$$
	Như vậy ta cần dùng $4$ hộp sơn vàng.\\
	Vậy tổng số tiền là
	$$
	100\,000 \cdot 2 + 140\,000 \cdot 3 + 130\,000 \cdot 4 = 1\,150\,000 \, (\text{đồng}).
	$$
	}
\end{ex}

\begin{ex}%[Nguồn: Bộ đề minh họa Moon 2024-2025]%[2D4V3-2]
	Nhà thiết kế muốn trang trí một biển quảng cáo có dạng hình tròn tâm $O$ bán kính $5$ dm, phía trong được trang trí bởi hai hình chữ nhật $ABCD$ và $MNPQ$ có độ dài cạnh là $AB=2MN=8$ dm và hai đường parabol đối xứng nhau chung đỉnh $O$, trong đó hình chữ nhật $ABCD$ nội tiếp đường tròn tâm $O$ còn Parabol nằm phía trên trục hoành đi qua các điểm $A$, $M$, $Q$ và $D$ (như hình vẽ). Hỏi diện tích phần tô đậm bằng bao nhiêu dm$^2$? (\textit{làm tròn kết quả đến hàng phần mười}).
	\shortans[]{$25{,}5$}
	\loigiai{
	Gọi điểm $E(0;5)$ và $F(-5;0)$.\\
	Điểm $I$ là giao điểm của $AB$ và $OF$ nên ta suy ra $OI=\sqrt{AO^2-AI^2}=3$, suy ra $I(-3;4)$.\\
	Phương trình parabol qua các điểm $A$, $M$, $Q$ và $D$ có dạng $y=ax^2+bx+c$.\\
	Do Parabol qua điểm $O(0;0)$ nên suy ra $\heva{&b=0\\&c=0.}$\\
	Parabol qua điểm $A(-3;4)$ nên suy ra $4=a\cdot (-3)^2 \Leftrightarrow a=\dfrac{4}{9}$.\\
	Suy ra hàm số parabol cần tìm là $y=\dfrac{4}{9}x^2$.\\
	Suy ra tọa độ điểm $M\left(\dfrac{3}{\sqrt{2}};2\right)$.\\
	Do đó diện tích hình phẳng được giới hạn bởi các parabol và đoạn $MN$ là
	\begin{eqnarray*}
		S_1=\displaystyle\int\limits_{\frac{-3}{\sqrt{2}}}^{\frac{3}{\sqrt{2}}}\left( 1-\frac{4}{9}x^2\right) \mathrm{\,d}x=\sqrt{2} \text{ (dm$^2$)}.
	\end{eqnarray*}
	Diện tích phần hình phẳng giới hạn bởi parabol, đoạn $AI$, đường $MN$ và trục hoành là
	\begin{eqnarray*}
		S_2=\displaystyle\int\limits_{-3}^{\frac{-3}{\sqrt{2}}}\left(\frac{4}{9}x^2\right) \mathrm{\,d}x=4-\sqrt{2} \text{  (dm$^2$)}.
	\end{eqnarray*}
	Diện tích tam giác $\triangle ADE$ là $S_{\triangle ADE}=\dfrac{1}{2}\cdot AD\cdot \mathrm{d}(E;AD)=\dfrac{1}{2}\cdot 6\cdot 1= 3$ (dm$^2$).\\
	Ta có $\tan \widehat{AOI}=\dfrac{4}{3}\Rightarrow \widehat{AOI} \approx53{,}13^\circ$.\\
	Diện tích phần hình phẳng giới hạn bởi cung $\overset\frown {AF}$ và dây cung $AF$ là
	\[S_{\overset\frown {AF}}=\dfrac{\pi R^2\cdot 53{,}13^\circ}{360^\circ}-S_{\triangle AFO}=\dfrac{\pi \cdot 5^2\cdot 53{,}13^\circ}{360^\circ}-\dfrac{1}{2}\cdot 4\cdot 5\approx 1{,}591 \text{ (dm$^2$)}.\]
	Vậy diện tích phần tô đậm trên hình là\\
	$S= 2S_1+4S_2+2S_{\triangle ADE}+4S_{\overset\frown {AF}}=2\sqrt{2}+4\cdot \left(4-\sqrt{2} \right)+2\cdot 3+ 4\cdot 1{,}591\approx25{,}5 $ (dm$^2)$.
	}
\end{ex}

\begin{ex}%[Nguồn: Bộ đề minh họa Moon 2024-2025]%[2D4V3-3]
	\immini{
	Bên trong hình vuông cạnh $4$, dựng hình sao bốn cánh đều như hình vẽ bên (các kích thước cần thiết cho như ở trong hình bên). Tính thể tích $V$ của khối tròn xoay sinh ra khi quay hình sao đó quanh trục $Ox$ (làm tròn kết quả đến hàng phần mười).
	}
	{
	\begin{tikzpicture}[line cap=butt,line join=miter,>=stealth,scale=0.7]
		\tikzset{declare function={xmin=-3;xmax=3;ymin=-3;ymax=3;},
		smooth,samples=450
		}
		\path
		(2,2) coordinate (A)
		(2,-2) coordinate (B)
		(-2,-2) coordinate (C)
		(-2,2) coordinate (D);
		\fill [pattern=north east lines] (A)--(B)--(C)--(D);
		\fill[white] (A)--(1,0)--(B) (B)--(0,-1)--(C) (C)--(-1,0)--(D) (D)--(0,1)--(A);
		\draw[->] (xmin,0)--(xmax,0) node[shift={(-100:7pt)},font=\normalsize]{$ x $};
		\draw[->] (0,ymin)--(0,ymax) node[shift={(190:7pt)},font=\normalsize]{$ y $};
		\fill (0,0) node[shift={(225:7pt)},font=\normalsize]{$ O $};
		\clip (xmin,ymin) rectangle (xmax,ymax);
		\fill (2,2) circle (1pt);	\fill (2,-2) circle (1pt);
		\fill (-2,-2) circle (1pt);	\fill (-2,2) circle (1pt);
		\draw[dashed] (A)--(B)--(C)--(D)--cycle
		(-2,0)node[below left]{$-2$} (-1,0)node[above right]{$-1$} (1,0)node[below left]{$1$} (2,0)node[below right]{$2$}
		(0,2)node[above left]{$2$} (0,-2)node[below left]{$-2$}
		;
		\draw (A)--(1,0)--(B)--(0,-1)--(C)--(-1,0)--(D)--(0,1)--cycle;
	\end{tikzpicture}
	}
	
	\shortans{$20{,}9$}
	\loigiai{
	\begin{center}
		\begin{tikzpicture}[line cap=butt,line join=miter,>=stealth]
			\tikzset{declare function={xmin=-3;xmax=3;ymin=-3;ymax=3;},
			smooth,samples=450
			}
			\path
			(2,2) coordinate (B)
			(2,-2) coordinate (C)
			(-2,-2) coordinate (D)
			(-2,2) coordinate (A)
			(0,1)  coordinate (M)
			(1,0)  coordinate (N)
			(-1,0)  coordinate (Q)
			(0,-1)  coordinate (P)
			;
			\fill [pattern=north east lines] (B)--(C)--(D)--(A);
			\fill[white] (B)--(1,0)--(C) (C)--(0,-1)--(D) (D)--(-1,0)--(A) (A)--(0,1)--(B);
			\draw[->] (xmin,0)--(xmax,0) node[shift={(-100:7pt)},font=\normalsize]{$ x $};
			\draw[->] (0,ymin)--(0,ymax) node[shift={(190:7pt)},font=\normalsize]{$ y $};
			\fill (0,0) node[shift={(225:7pt)},font=\normalsize]{$ O $};
			\clip (xmin,ymin) rectangle (xmax,ymax);
			\fill (2,2) circle (1pt);	\fill (2,-2) circle (1pt);
			\fill (-2,-2) circle (1pt);	\fill (-2,2) circle (1pt);
			\draw[dashed] (A)--(B)--(C)--(D)--cycle
			(-2,0)node[below left]{$-2$} (-1,0)node[above right]{$-1$} (1,0)node[below left]{$1$} (2,0)node[below right]{$2$}
			(0,2)node[above left]{$2$} (0,-2)node[below left]{$-2$}
			;
			\draw (B)--(1,0)--(C)--(0,-1)--(D)--(-1,0)--(A)--(0,1)--cycle;
			\foreach \t/\g in {A/90,B/0,C/-90,D/-90,N/120,M/60,P/-60,Q/160}{
			\draw[fill=black] (\t) circle (1pt) node[shift={(\g:7pt)},font=\scriptsize]{$ \t $};
			}
		\end{tikzpicture}
	\end{center}
	Ta có khối tròn xoay đó được tạo thành khi quay hình phẳng $QAMBN$ quanh trục $Ox$.\\
	Mà $S_{OQAM}=S_{ONBM} $ nên thể tích của khối tròn xoay đó sẽ bằng $2$ lần thể tích của khối tròn xoay khi quay hình phẳng $ONBM$ quanh trục $Ox$.\\
	Suy ra ta có thể tích
	\[V=2\left(\pi\displaystyle\int\limits_0^2 MB^2\mathrm{\,d}x-\pi\displaystyle\int\limits_1^2 NB^2\mathrm{\,d}x\right).\]
	Ta có $M(0;1)$, $B(2;2)$, $\vec{MB}=(2;1)$, suy ra một vectơ pháp tuyến của đường thẳng $MB$ là \break$\vec{n}_1=(1;-2)$.\\
	Ta có $MB\colon 1\cdot (x-0)-2\cdot (y-1)=0\Leftrightarrow x-2y+2=0\Leftrightarrow y=\dfrac{1}{2}x+1$.\\
	Ta có $N(1;0)$, $B(2;2)$, $\vec{NB}=(1;2)$, suy ra một vectơ pháp tuyến của đường thẳng $NB$ là \break$\vec{n}_2=(2;-1)$.\\
	Ta có $NB\colon 2\cdot (x-1)-1\cdot (y-0)=0\Leftrightarrow 2x-y-2=0\Leftrightarrow y=2x-2$.\\
	Suy ra \[V=2\left(\pi\displaystyle\int\limits_0^2 \left(\dfrac{1}{2}x+1\right)^2\mathrm{\,d}x-\pi\displaystyle\int\limits_1^2 (2x-2)^2\mathrm{\,d}x\right)=\dfrac{20}{3}\pi\approx 20{,}9.\]
	}
\end{ex}

%

\begin{ex}%[Nguồn: Bộ đề minh họa Moon 2024-2025]%[2D4V3-3]
	\immini{Cho hình vuông $ABCD$ tâm $O$, có cạnh bằng $2$ dm. Gọi $I$, $K$ lần lượt là trung điểm cạnh $AB$, $CD$. Đồ thị $(C)$ của hàm số bậc ba nhận $O$ làm tâm đối xứng và $A$, $C$ là hai điểm cực trị tạo với các đoạn thẳng $IK$, $IA$, $CK$ một miền phẳng $(H)$ (phần gạch chéo trong hình bên). Một chiếc đồng hồ có các số dạng một khối tròn xoay được tạo thành khi quay $(H)$ quanh trục $IK$. Thể tích của chiếc đồng hồ đó bằng bao nhiêu dm$^3$? (làm tròn kết quả đến hàng phần trăm).}{
	\begin{tikzpicture}[line join=round, line cap=round,>=stealth,thick,scale=1.8]
		\draw
		(0,0) node [above right] {$O$} circle(0.5pt)
		;
		\draw (1,1) -- (1,-1) -- (-1,-1)--(-1,1)--cycle (-1,0)--(1,0);
		\draw[samples=200,domain=-1:1,smooth,variable=\x] plot (\x,{(1/2)*(\x)^3 - (3/2)*(\x)});
		\fill
		(-1,1)node[above left]{$A$}
		(-1,-1)node[below left]{$B$}
		(1,-1)node[below right]{$C$}
		(1,1)node[above right]{$D$}
		(1,0)node[ right]{$K$}
		(-1,0)node[left]{$I$}
		;
		\draw[pattern = north east lines, line width = 1.2pt,draw=none]plot[domain=0:1](\x,{(1/2)*(\x)^3 - (3/2)*(\x)})--(1,0);
		\draw[pattern = north east lines, line width = 1.2pt,draw=none]plot[domain=-1:0](\x,{(1/2)*(\x)^3 - (3/2)*(\x)})--(-1,0);
	\end{tikzpicture}}
	\shortans{$3{,}05$}
	
	\loigiai{\begin{center}
		\begin{tikzpicture}[line join=round, line cap=round,>=stealth,thick,scale=1.8]
			\draw[->] (-1.5,0)--(1.5,0)node[below]{$x$};
			\draw[->] (0,-1.5)--(0,1.5)node[right]{$y$};
			\draw
			(0,0) node [above right] {$O$} circle(0.5pt)
			;
			\draw (1,1) -- (1,-1) -- (-1,-1)--(-1,1)--cycle (-1,0)--(1,0);
			\draw[samples=200,domain=-1:1,smooth,variable=\x] plot (\x,{(1/2)*(\x)^3 - (3/2)*(\x)});
			\fill
			(-1,1)node[above left]{$A$}
			(-1,-1)node[below left]{$B$}
			(1,-1)node[below right]{$C$}
			(1,1)node[above right]{$D$}
			(1,0)node[above right]{$K$}
			(-1,0)node[above left]{$I$}
			(-1,0)node[below left]{$-1$}
			(1,0)node[below right]{$1$}
			(0,1)node[above right]{$1$}
			(0,-1)node[above left]{$-1$}
			;
			\draw[pattern = north east lines, line width = 1.2pt,draw=none]plot[domain=0:1](\x,{(1/2)*(\x)^3 - (3/2)*(\x)})--(1,0);
			\draw[pattern = north east lines, line width = 1.2pt,draw=none]plot[domain=-1:0](\x,{(1/2)*(\x)^3 - (3/2)*(\x)})--(-1,0);
		\end{tikzpicture}
	\end{center}
	Gắng hệ trục $Oxy$ như hình vẽ, ta có $A(-1;1)$ và $C(1;-1)$ là hai điểm cực trị của đồ thị hàm số $y=f(x)=ax^3+bx^2+cx+d$ $(a\neq 0)$.\\
	Ta có $y'=f'(x)=3ax^2+2bx+c$.\\
	Vì $x=1$ và $x=-1$ là hai điểm cực trị của hàm số nên $f'(x)=3a(x^2-1)=3ax^2-3a$.\\
	Suy ra $f(x)=\displaystyle\int (3ax^2-3a)\mathrm{\,d}x=ax^3-3ax+d$.\\
	Mặt khác đồ thị đi qua $O(0;0)$ và $C(1;-1)$ nên $\heva{&c=0\\&a-3a=-1}\Leftrightarrow\heva{&c=0\\&a=\dfrac{1}{2}.}$.\\
	Suy ra $f(x)=\dfrac{1}{2}x^3-\dfrac{3}{2}x$.\\
	Vậy thể tích của chiếc đồng hồ là
	\begin{eqnarray*}
		V&=&\pi\displaystyle\int_{-1}^1\left(\dfrac{1}{2}x^3-\dfrac{3}{2}x\right)^2\mathrm{\,d}x\\
		&=& \pi\displaystyle\int_{-1}^1\left(\dfrac{1}{4}x^6-\dfrac{3}{2}x^4+\dfrac{9}{4}x^2\right)\mathrm{\,d}x\\
		&=&\pi\left.\left(\dfrac{1}{28}x^7-\dfrac{3}{10}x^5+\dfrac{3}{4}x^3\right)\right|_{-1}^{1}\\
		&=&\dfrac{34\pi}{35}\\
		&\approx& 3{,}05 ~\mathrm{dm}^3.
	\end{eqnarray*}
	}
\end{ex}

%Câu trắc nghiệm 12

\begin{ex}%[Nguồn: Bộ đề minh họa Moon 2024-2025]%[2D4V3-3]
	\immini[thm]
	{Đường cong trong hình bên dưới có tên gọi là đường Lemmiscate. Trong mặt phẳng $Oxy$, phương trình của đường Lemmiscate đã cho là $16y^2=x^2(25-x^2)$.
	Thể tích vật thể tròn xoay tạo thành khi cho hình phẳng giới hạn bởi đường cong đó quay quanh trục $Ox$ bằng}
	{\begin{tikzpicture}[font=\footnotesize, line join=round, line cap=round, >=stealth, scale=0.5]
		\begin{scope}
			\draw[smooth, samples=500] plot[domain=-5:0] (\x,{sqrt(((\x)^2*(25-(\x)^2))/(16))});
			\draw[smooth, samples=500] plot[domain=5:0] (\x,{sqrt(((\x)^2*(25-(\x)^2))/(16))});
			\draw[smooth, samples=500] plot[domain=-5:0] (\x,{-sqrt(((\x)^2*(25-(\x)^2))/(16))});
			\draw[smooth, samples=500] plot[domain=5:0] (\x,{-sqrt(((\x)^2*(25-(\x)^2))/(16))});
		\end{scope}
		\draw[->] (-6,0)--(6,0)node[below]{$x$};
		\draw (-170:0.4)node[below left=-1mm]{$O$};
		\draw[->] (0,-4)--(0,4)node[left]{$y$};
	\end{tikzpicture}}
	\choice
	{$\dfrac{625}{6}\pi$}
	{\True $\dfrac{625}{12}\pi$}
	{$\dfrac{1\,250}{3}\pi$}
	{$\dfrac{625}{3}\pi$}
	\loigiai{
	Ta có $16y^2=x^2(25-x^2) \Leftrightarrow y^2=\dfrac{x^2(25-x^2)}{16}$.\\
	Cho $y^2=0 \Leftrightarrow x^2(25-x^2)=0\Leftrightarrow \hoac{& x=0 \\& x=\pm 5.}$\\
	Ta có đồ thị hàm số đối xứng qua hai trục $Ox$ và $Oy$.\\
	Khi đó thể tích của vật thể tròn xoay tạo thành khi cho hình phẳng giới hạn bởi đường cong đó quay quanh trục $Ox$ bằng
	$$ V= 2\pi\displaystyle\int\limits_0^5 y^2\mathrm{\,d}x = 2\pi\displaystyle\int\limits_0^5 \dfrac{x^2(25-x^2)}{16} \mathrm{\,d}x = \dfrac{625}{12}\pi.$$
	}
\end{ex}

\begin{ex}%[Nguồn: Bộ đề minh họa Moon 2024-2025]%[2D4V3-3]
	\immini{Chướng ngại vật \lq \lq tường cong \rq \rq  trong một sân thi đấu X-Game là một khối bê tông có chiều cao từ mặt đất lên là $3$ m. Giao của mặt tường cong và mặt đất là đoạn thằng $A B=2 \mathrm{~m}$. Thiết diện của khối tường cong cắt bời mặt phẳng vuông góc với $A B$ tại $A$ là một hình tam giác vuông cong $A C E$ với $A C=4$ m, $C E=3$ m và cạnh cong $A E$ nằm trên một đường Parabol có trục đối xứng vuông góc với mặt đất. Tại vị trí $M$ là trung điểm của $A C$ thì tường cong có độ cao $1$ m. Thể tích bê tông cần sử dụng để tạo nên khối tường cong đó là bao nhiêu m$^3$. Viết kết quả làm tròn đến hàng phần chục.}
	{
	\begin{tikzpicture}[line join=round,line cap=round,>=stealth,font=\footnotesize,scale=.8, declare function={f(\x)=0.2*((\x)^2);}]
		\path
		(0,0) coordinate (A)
		(5,5) coordinate (E)
		(5,0) coordinate (C)
		(-1,3) coordinate (B)
		(2.5,0) coordinate (M)
		($(E)+(-1,3)$) coordinate (F)
		($(A)+(-1,3)$) coordinate (B)
		($(C)+(-1,3)$) coordinate (C1)
		
		;
		\draw [pattern=north east lines,samples=200,domain=0:5,smooth,variable=\x] (A)--plot (\x,{f(\x)})--(F);
		\draw[pattern=north east lines, samples=200,domain=-1:4,smooth,variable=\x] (A)--(B)--plot (\x,{f(\x+1)+3})--(F);
		\draw[] (B)--(A)--(C)--(E)--(F) (M)--(2.5,{f(2.5)}) coordinate (M1) node [midway] {   $\quad \quad  1$m};
		\draw[dashed] (C)--(C1)--(F) (B)--(C1);
		\foreach \p/\r in {A/-90,B/90,E/0,F/0,C/0,M/-90}
		\fill (\p) circle (1pt) node[shift={(\r:3mm)}]{$\p$};
		
	\end{tikzpicture}
	}
	\shortans[oly]{$9{,}3$}
	\loigiai{
	\begin{center}
		
		\begin{tikzpicture}[line join=round,line cap=round,>=stealth,font=\footnotesize,scale=.8, declare function={f(\x)=0.2*((\x)^2);}]
			\path
			(0,0) coordinate (A)
			(5,5) coordinate (E)
			(5,0) coordinate (C)
			(-1,3) coordinate (B)
			(2.5,0) coordinate (M)
			($(E)+(-1,3)$) coordinate (F)
			($(A)+(-1,3)$) coordinate (B)
			($(C)+(-1,3)$) coordinate (C1)
			
			;
			\draw [pattern=north east lines,samples=200,domain=0:5,smooth,variable=\x] (A)--plot (\x,{f(\x)})--(F);
			\draw[pattern=north east lines, samples=200,domain=-1:4,smooth,variable=\x] (A)--(B)--plot (\x,{f(\x+1)+3})--(F);
			\draw[] (B)--(A)--(C)--(E)--(F) (M)--(2.5,{f(2.5)}) coordinate (M1) node [midway] {   $\quad \quad  1$m};
			\draw[dashed] (C)--(C1)--(F) (B)--(C1);
			\foreach \p/\r in {A/-90,B/90,E/0,F/0,C/0,M/-90}
			\fill (\p) circle (1pt) node[shift={(\r:3mm)}]{$\p$};
			
		\end{tikzpicture}
	\end{center}
	\begin{itemize}
		\item \textbf{Xác định phương trình Parabol của cạnh cong AE:} \\
		Parabol đi qua các điểm $A(0,0)$, $E(4,3)$ và trung điểm $M(2,1)$.
		Giả sử phương trình có dạng $y=a x^2+b x+c$.
		Thay các điểm vào, giải hệ phương trình
		$$
		\heva{&16 a+4 b=3 \\
		&4 a+2 b=1 \\
		&c=0.}$$
		Kết quả $a=\dfrac{1}{8}$, $b=\dfrac{1}{4}$, $c=0$.\\
		Phương trình Parabol $y=\dfrac{1}{8} x^2+\dfrac{1}{4} x$.
		\item \textbf{Tính diện tích mặt cắt $ACE$}$$
		\text { Diện tích }=\int_0^4\left(\dfrac{1}{8} x^2+\dfrac{1}{4} x\right) d x=\left[\dfrac{1}{24} y^3+\dfrac{1}{8} y^2\right]_0^4=\dfrac{14}{3} \mathrm{~m}^2.
		$$
		\item \textbf{Tính thể tích khối tường cong}\\
		$$\text { Thể tích }=\dfrac{14}{3}\cdot 2=\dfrac{28}{3} \approx 9{,}333 \mathrm{~m}^3.$$
	\end{itemize}
	}
\end{ex}

\begin{ex}%[Nguồn: Bộ đề minh họa Moon 2024-2025]%[2D4V3-5]
	Một ly trà sữa dạng hình nón cụt, có đường kính đáy ly là $6$ cm,  đường kính miệng ly là $9$ cm, chiều cao $13{,}4$ cm. Ở miệng ly có sử dụng một nắp đậy có hình dạng nửa mặt cầu, và ở đỉnh của nửa mặt cầu này có một hình tròn có đường kính $2$ cm để cắm ống hút,  mặt phẳng chứa hình tròn này song song với mặt phẳng chứa miệng ly (tham khảo hình vẽ).\\
	\begin{tabular}{ccc}
		\begin{tikzpicture}[line join=round,line cap=round,>=stealth,scale=.6]
			\path
			(0,0)coordinate(O)
			(2,0)coordinate(A)
			(3,4)coordinate(B)
			(-1,4)coordinate(D)
			;
			\draw(O)--(D)(A)--(B);
			\draw[dashed](A) arc (0:180:1 cm and 0.5 cm);
			\draw(A) arc (0:-180:1 cm and 0.5 cm);
			\draw[dashed](B) arc (0:180:2 cm and 0.5 cm);
			\draw(B) arc (0:-180:2 cm and 0.5 cm);
			\draw(B) arc (0:180:2 cm and 2 cm);
			\draw(2,5.75) arc (0:-180:1 cm and 0.5 cm);
			\fill[black] (1,1) circle(0.3);
			\fill[black] (1,2) circle(0.3);
			\fill[black] (0,2) circle(0.3);
			\fill[black] (2,2) circle(0.3);
		\end{tikzpicture}
		&\hspace*{3cm}
		\begin{tikzpicture}[line join=round,line cap=round,>=stealth,scale=.6]
			\path
			(0,0)coordinate(O)
			(2,0)coordinate(A)
			(3,4)coordinate(B)
			(-1,4)coordinate(D)
			;
			\draw(O)--(D)(A)--(B);
			\draw[dashed](A) arc (0:180:1 cm and 0.5 cm);
			\draw(A) arc (0:-180:1 cm and 0.5 cm);
			\draw[dashed](B) arc (0:180:2 cm and 0.5 cm);
			\draw(B) arc (0:-180:2 cm and 0.5 cm);
			\draw(B) arc (0:180:2 cm and 2 cm);
			\draw(2,5.75) arc (0:-180:1 cm and 0.5 cm);
			\draw[<->](0,-1)--(2,-1)node[below,midway]{$6$ cm};
			\draw[<->](-1,4)--(3,4)node[above,midway]{$9$ cm};
			\draw[<->](0,6.3)--(2,6.3)node[above,midway]{$2$ cm};
			\draw[<->](4,0)--(4,4)node[below=3mm,pos=.5,sloped]{$13{,}4$ cm};
		\end{tikzpicture}
		&\hspace*{2cm}
		\begin{tikzpicture}[line join=round,line cap=round,>=stealth,rotate=-90,scale=.6]
			\path
			(0,0)coordinate(O)node[above right]{$B$}
			(2,0)coordinate(A)
			(3,4)coordinate(B)
			(-1,4)coordinate(D)
			(1,0)coordinate(E)node[above right]{$A$}
			(1,4)coordinate(F)node[above right]{$O$}
			(-1,4)coordinate(G)node[above right]{$C$}
			(1,5.75)coordinate(H)
			(0,5.75)coordinate(I)node[above right]{$D$}
			;
			\draw(O)--(D)(A)--(B);
			\draw[dashed](A) arc (0:180:1 cm and 0.5 cm);
			\draw(A) arc (0:-180:1 cm and 0.5 cm);
			\draw[dashed](B) arc (0:180:2 cm and 0.5 cm);
			\draw(B) arc (0:-180:2 cm and 0.5 cm);
			\draw(B) arc (0:180:2 cm and 2 cm);
			\draw(2,5.75) arc (0:-180:1 cm and 0.5 cm);
			\draw[->](1,-1)--(1,7)node[below]{$x$};
			\draw[->](4,4)--(-2,4)node[right]{$y$};
			\draw(O)--(E)(H)--(I);
		\end{tikzpicture}\\
	\end{tabular}\\
	Chọn hệ trục $Oxy$ (đơn vị trên trục là centimet) với trục $Ox$ đi qua tâm của $2$ đáy hình nón cụt và gốc tọa độ $O$ trùng với tâm của đáy lớn như hình vẽ.
	\choiceTF
	{\True Phương trình đường thẳng $BC$ là $y=\dfrac{15}{134}x+4{,}5$}
	{\True Tọa độ điểm $D$ là $\left(\dfrac{\sqrt{77}}{2}; 1 \right)$}
	{\True Thể tích bên trong của ly không bao gồm nắp là $600$ ml (làm tròn kết quả đến hàng đơn vị)}
	{\True Thể tích bên trong của ly bao gồm cả thể tích của nắp là $790{,}6$ ml (làm tròn kết quả đến hàng phần mười)}
	\loigiai{
	\begin{itemchoice}
		\itemch Giả sử rằng phương trình là $y=ax+b$, vì điểm $B(-13{,}4;3)$ và $C(0;4{,}5)$ cùng nằm trên đường thẳng này, ta có hệ phương trình:
		$$\heva{&-13{,}4a+b=3\\&b=4{,}5}\Leftrightarrow\heva{&a=\dfrac{15}{135}\\&b=4{,}5.}$$
		Vậy phương trình đường thẳng $BC$ là $y=\dfrac{15}{134}x+4{,}5$.
		\itemch Phương trình đường tròn tâm $O(0;0)$ bán kính $OC=4{,}5$ cm là $x^2+y^2=4{,}5^2$.\\
		Do đường tròn đi qua điểm $D(x_0;1)$ nên $x_0^1+1=4{,}5^2\Rightarrow x_0=\dfrac{\sqrt{77}}{2}$.\\
		Vậy tọa độ điểm $D\left(\dfrac{\sqrt{77}}{2};1\right)$.
		\itemch Thể tích bên trong ly không bao gồm nắp là
		$$
		\pi \int\limits_{-13,4}^{0} \left(\frac{15}{134}x + 4,5 \right)^2 dx \approx 600 \text{ cm}^3= 600 \text{ ml}.
		$$
		\itemch Đường tròn nắp ly có phương trình $x^2+y^2=4{,}5^2\Rightarrow y=\pm\sqrt{20{,}25 - x^2}$.\\
		Ta xét phần dương của nắp ly, ta có $y=\sqrt{20{,}25 -x^2}$.\\
		Thể tích nắp ly là:
		$$\pi \int\limits_{0}^{\frac{\sqrt{77}}{2}}\left(\sqrt{20{,}25-x^2} \right)^2\mathrm{\,d}x.$$
		Vậy thể tích bên trong ly bao gồm cả thể tích nắp là:
		$$V = \pi \int\limits_{-13{,}4}^{0}\left(\frac{15}{134}x+4{,}5\right)^2\mathrm{\,d}x+\pi\int\limits_{0}^{\frac{\sqrt{77}}{2}}\left(\sqrt{20{,}25 -x^2}\right)^2\mathrm{\,d}x\approx 790{,}6 \text{ ml}.$$
	\end{itemchoice}
	}
\end{ex}

\begin{ex}%[Nguồn: Bộ đề minh họa Moon 2024-2025]%[2D4V3-5]
	Một cái hồ nước ước tính có hình tròn, đường kính $400$ dm (xem hình). Bắt đầu từ tâm, độ sâu của nước được đo mỗi $25$ dm và ghi lại trong bảng với $x$ có đơn vị dm.\\
	\begin{minipage}[c]{8cm}
		\begin{tabular}{|c|>{\centering\arraybackslash}p{0.8cm}|*{4}{>{\centering\arraybackslash}p{0.8cm}|}}
			\hline
			$x$ & $0$ & $25$ & $50$ & $75$ & $100$ \\ \hline
			\text{Độ sâu} & $20$ & $19$ & $19$ & $17$ & $15$ \\ \hline
		\end{tabular}\vspace{0.5cm}
		\begin{tabular}{|c|>{\centering\arraybackslash}p{0.8cm}|*{4}{>{\centering\arraybackslash}p{0.8cm}|}}
			\hline
			$x$ & $125$ & $150$ & $175$ & $200$ \\ \hline
			\text{Độ sâu} & $14$ & $10$ & $6$ & $0$ \\ \hline
		\end{tabular}
	\end{minipage}\hspace{0.5cm}
	\begin{minipage}[c]{7cm}
		\begin{tikzpicture}[scale=1]
			\def\r{4.5cm}
			\draw[->] (-1,0) -- (\r+1cm,0)node[below]{$x$\,(dm)}; % Trục x
			\draw[->] (0,1) -- (0,-\r-1cm)node[right]{$y$\,(dm)}; % Trục y
			\fill[black!30] (0,-\r)
			to[out=0, in=200] (2,-3.8)
			to[out=20, in=-130] (3.8,-1.25)
			to[out=10, in=180] (\r,0)
			-- (\r,-\r) -- (0,-\r) -- cycle;
			\fill
			(\r,0) circle (1pt) node[above]{$200$}
			(0,-\r) circle (1pt) node[left]{$20$}
			(0,0) circle (1pt) node[shift={(135:3mm)}]{$O$}
			(1.5,-2) node[above]{Hồ nước}
			(-0.3,-2.5) node[rotate=90] {Độ sâu}
			;
		\end{tikzpicture}
	\end{minipage}\\
	Người ta xây dựng một hàm số bậc hai $f(x)=ax^2+bx+c$ ($a\ne 0$) để mô phỏng tương đối độ sâu của cái ao theo số liệu ở bảng $1$, tức là trên hệ trục tọa độ $Oxy$, đồ thị của hàm số đó đi qua các điểm $A(0;20)$, $B(100;15)$, $C(200;0)$ và đơn vị trên các trục tọa độ là dm. Gọi $V_0$ là thể tích của hồ nước. Dựa vào hàm số $f(x)$ tìm được, ta có
	\choiceTF
	{Độ sâu lớn nhất của cái ao là $20$\,m}
	{Thể tích của hồ nước được xác định thông qua công thức $V_0=\pi\displaystyle\int\limits_0^{200}f^2(x)\mathrm{\,d}x$}
	{Thể tích của hồ nước là $V_c=8378$ lít (làm tròn đến hàng đơn vị của dm$^3$)}
	{Hồ nước đang đầy thì nước bắt đầu bốc hơi khỏi hồ với lượng nước bốc hơi phụ thuộc vào hàm số $V'(t)=0{,}03V\cdot \mathrm{e}^{-0{,}03t}$ ($t$ tính theo ngày). Vậy sau $10$ ngày không mưa, lượng nước còn lại trong hồ là $465$ lít (làm tròn kết quả đến hàng đơn vị)}
	\loigiai{
	\begin{itemchoice}
		\itemch
		Gọi hàm số bậc hai mô phỏng độ sâu hồ có dạng $y=ax^2+bx+c\,\, (a\neq 0)$.
		Hàm số cắt trục tung tại điểm $A(0;20)$ nên suy ra $c= 20$.\\
		Đồng thời điểm $A(0;20)$ là đỉnh của parabol nên ta có $b=0$.\\
		Hàm số $y=ax^2+20$ đi qua điểm $C(200;0)$ nên ta có
		\begin{eqnarray*}
			&&0=a\cdot 200^2+20\\
			&\Leftrightarrow&a=-\dfrac{1}{2\,000}.
		\end{eqnarray*}
		Suy ra hàm số bậc hai mô phỏng độ sâu hồ là $y=-\dfrac{1}{2\,000}x^2+20$.\\
		Độ sâu lớn nhất của cái ao là $20$ dm.
		\itemch
		Vì hàm số bậc hai mô phỏng độ sâu hồ là $y=-\dfrac{1}{2\,000}x^2+20$.\\
		Suy ra $x=\sqrt{40\,000-2\,000y}$.\\
		Do đó thể tích của hồ nước được xác định bởi công thức
		\[V_0=\pi\displaystyle\int_0^{20}(40\,000-2\,000y)\mathrm{\,d}y.\]
		\itemch
		Thể tích của hồ nước là
		\[V_c=\pi\displaystyle\int_0^{20}(40\,000-2\,000y)\mathrm{\,d}y\approx1\,256\,637\text{ (lít).}\]
		\itemch
		Sau $10$ ngày không mưa, lượng nước trong hồ đã bốc hơi là
		\[V'(10)=0{,}03\cdot 1\,256\,637\cdot \mathrm{e}^{-0{,}03\cdot 10}\approx 27\,928\text{ (lít)}.\]
		Vậy sau $10$ ngày không mưa, lượng nước còn lại trong hồ là
		\[1\,256\,637- 27\,928=1\,228\,709\text{ (lít)}.\]
	\end{itemchoice}
	}
\end{ex}

\begin{ex}%[Nguồn: Bộ đề minh họa Moon 2024-2025]%[2D4V3-5]
	\immini{Ông Duy có một mảnh vườn hình vuông cạnh bằng $8$ m. Ông dự định xây một cái bể bơi đặc biệt (phần kẻ sọc trong hình vẽ bên). Biết $AM=\dfrac{AB}{4}$, phần đường cong đi qua các điểm $C$, $M$, $N$ là một phần của đường Parabol có trục đối xứng là $MP(MP\parallel AD)$ và chi phí để làm bể bơi là $5$ triệu đồng $/ $1$\mathrm{~m}^2$. Số tiền ông Duy phải trả để xây cái bể bơi đó là bao nhiêu triệu đồng? (làm tròn kết quả đến hàng đơn vị).}{\begin{tikzpicture}[line join=round, line cap=round,>=stealth,thick,scale=0.6]
		\path
		(0,8)coordinate (A)
		(8,8)coordinate (B)
		(8,0)coordinate (C)
		(0,0)coordinate (D)
		(2,0)coordinate (P)
		(2,8)coordinate (M)
		(0,64/9)coordinate (N)
		;
		\begin{scope}
			\clip (-2,-2) rectangle (8,8);
			\fill[pattern=north west lines]plot[samples=200,domain=0:8,smooth,variable=\x] (\x,{-2/9*(\x)^2+8/9*(\x)+64/9})--plot[samples=200,domain=8:0,smooth,variable=\x] (\x,{-8/9*(\x)+64/9})--cycle;
			\draw[dashed] plot[samples=200,domain=0:8,smooth,variable=\x] (\x,{-2/9*(\x)^2+8/9*(\x)+64/9});
			\draw plot[samples=200,domain=0:8.1,smooth,variable=\x] (\x,{-2/9*(\x)^2+8/9*(\x)+64/9});
			\draw plot[samples=200,domain=0:8.1,smooth,variable=\x] (\x,{-8/9*(\x)+64/9});
			\draw[samples=200,domain=0:8,smooth,variable=\x] plot (\x,{-2/9*(\x)^2+8/9*(\x)+64/9});
			\draw[samples=200,domain=0:8,smooth,variable=\x] plot (\x,{-8/9*(\x)+64/9});
		\end{scope}
		\draw (A)--(B)--(C)--(D)--cycle;
		\draw [dashed](M)--(P);
		\foreach \x/\g in {A/180,B/0,C/0,P/-90,M/90,N/180,D/180} \fill[black] (\x) circle (1pt) ($(\x)+(\g:3mm)$)node{$\x$};
	\end{tikzpicture}}
	\shortans[]{$95$}
	\loigiai{
	\begin{center}
		\begin{tikzpicture}[line join=round, line cap=round,>=stealth,thick,scale=0.6]
			\draw[->] (-2.1,0)--(9.1,0) node[below left] {$x$};
			\draw[->] (0,-2.1)--(0,9.1) node[below left] {$y$};
			\draw (0,0) node [below left] {$O$};
			\foreach \x/\nx in {8/8}
			\draw[thin] (\x,1pt)--(\x,-1pt) node [below] {$\nx$};
			\foreach \y/\ny in {8/8}
			\draw[thin] (1pt,\y)--(-1pt,\y) node [left] {$\ny$};
			\draw[dashed,thin](2,0)--(2,8)--(0,8);
			\path
			(0,8)coordinate (A)
			(8,8)coordinate (B)
			(8,0)coordinate (C)
			(0,0)coordinate (D)
			(2,0)coordinate (P)
			(2,8)coordinate (M)
			(0,64/9)coordinate (N)
			;
			\begin{scope}
				\clip (-2,-2) rectangle (8,8);
				\fill[pattern=north west lines]plot[samples=200,domain=0:8,smooth,variable=\x] (\x,{-2/9*(\x)^2+8/9*(\x)+64/9})--plot[samples=200,domain=8:0,smooth,variable=\x] (\x,{-8/9*(\x)+64/9})--cycle;
				\draw[dashed] plot[samples=200,domain=0:8,smooth,variable=\x] (\x,{-2/9*(\x)^2+8/9*(\x)+64/9});
				\draw plot[samples=200,domain=0:8.1,smooth,variable=\x] (\x,{-2/9*(\x)^2+8/9*(\x)+64/9});
				\draw plot[samples=200,domain=0:8.1,smooth,variable=\x] (\x,{-8/9*(\x)+64/9});
				\draw[samples=200,domain=0:8,smooth,variable=\x] plot (\x,{-2/9*(\x)^2+8/9*(\x)+64/9});
				\draw[samples=200,domain=0:8,smooth,variable=\x] plot (\x,{-8/9*(\x)+64/9});
			\end{scope}
			\draw (A)--(B)--(C);
			\foreach \x/\g in {A/40,B/0,C/40,P/-90,M/90,N/180,D/45} \fill[black] (\x) circle (1pt) ($(\x)+(\g:3mm)$)node{$\x$};
		\end{tikzpicture}
	\end{center}
	Gắn trục tọa độ như hình vẽ.
	Gọi phương trinh trình của Parabol là $(P)\colon y=ax^2+bx+c$.\\
	Ta có $(P)$ đi qua điểm $C(8;0)$, $M(2;8)$ và có hoành độ đỉnh $x=2$ nên ta có hệ phương trình sau
	$$\heva{&64a+8b+c=0\\&4a+2b+c=8\\&\dfrac{-b}{2a}=2}\Leftrightarrow \heva{&a=-\dfrac{2}{9}\\&b=\dfrac{8}{9}\\&c=\dfrac{64}{9}}\Rightarrow (P)\colon y=-\dfrac{2}{9}x^2+\dfrac{8}{9}x+\dfrac{64}{9}.$$
	Giao điểm của $(P)$ với trục $Oy$ là điểm $N\left(0;\dfrac{64}{9}\right)$.\\
	Gọi $d\colon y=ax+b$ là đường thẳng đi qua $N$ và $C$. Khi đó phương trình của $d$ là $y=-\dfrac{8}{9}x+\dfrac{64}{9}$.\\
	Diện tích hình phẳng giới hạn bởi đồ thị $(P)$ và đường thẳng $d$ là
	$$S=\displaystyle\int\limits_0^8 \left(-\dfrac{2}{9}x^2+\dfrac{8}{9}x+\dfrac{64}{9}+\dfrac{8}{9}x-\dfrac{64}{9}\right) \mathrm{\, d}x=\displaystyle\int\limits_0^8 \left(-\dfrac{2}{9}x^2+\dfrac{16}{9}x\right) \mathrm{\, d}x=\dfrac{512}{27}.$$
	Vậy số tiền ông Duy phải trả để xây bể bơi là $\dfrac{512}{27}\cdot 5\approx 95$ triệu đồng.
	}
\end{ex}

\begin{ex}%[Nguồn: Bộ đề minh họa Moon 2024-2025]%[2D4V3-5]
	Một người thiết kế mô hình một cái đèn ngủ bằng nhựa có hình dạng như hình về 3D ở hình 1. Hình 2 là mặt cắt bởi mặt phẳng cắt đi qua trục của đèn, hình 3 là bản vẽ toán học. Tam giác $ABC$ trong hình 3 là tam giác đều cạnh $4\sqrt{3}$ dm và $M$, $N$, $P$ là trung điểm các cạnh, $G$ là trọng tâm của tam giác. Phần tô đậm trong hình 3 được tạo bởi giao nhau của các cặp cung tròn đi qua $3$ điểm là $AGB$ và $AGC$, $PGN$ và $PGM$, $PGN$ và $NGM$ (xem hình vẽ). Biết rằng chiếc đèn ngủ được tạo thành khi xoay phần tô đậm trong hình 3 quanh trục là đường thẳng $AM$. Chi phí trung bình để sản xuất chiếc đèn ngủ là $40 000$ đồng trên mỗi dm$^3$. Chi phí để sản xuất đèn ngủ trên bằng bao nhiêu nghìn đồng? (làm tròn đến hàng đơn vị).
	\begin{center}
		%		\includegraphics[scale=0.3]{hinh/de22tlncau6}
	\end{center}\shortans[]{$236$}
	\loigiai{Gọi hệ trục tọa độ có $MC\equiv Ox$ và $MA\equiv Oy$.\\
	Ta được $M(0;0)$, $A(0;6)$, $C(2\sqrt{3};0)$, $B(-2\sqrt{3};0)$, $G(0;2)$, $P(-\sqrt{2};3)$, $N\left( \sqrt{2};3\right) $\\
	Phương trình đường tròn đi qua $3$ điểm $A$, $B$, $G$ là
	$$ \left(x+2\sqrt{3} \right)^2+(y-4)^2=16\Rightarrow x=-2\sqrt{3}\pm \sqrt{16-(y-4)^2}.$$
	Phương trình đường tròn đi qua $3$ điểm $P$, $G$, $M$ là
	$$ \left(x+\sqrt{3} \right)^2+(y-1)^2=4\Rightarrow x=-\sqrt{3}\pm \sqrt{4-(y-1)^2}.$$
	Phương trình đường tròn đi qua $3$ điểm $P$, $G$, $N$ là
	$$ x^2+(y-4)^2=4\Rightarrow x=\pm \sqrt{4-(y-4)^2}.$$
	Thể tích của chiếc đèn ngủ là
	$$V=\pi \displaystyle\int\limits_{2}^{6}\left( -2\sqrt{3}+\sqrt{16-(y-4)^2}\right)^2 \mathrm{\,d}y+\pi \displaystyle\int\limits_{2}^{3}\left(4-(y-4)^2 \right)-\left( -\sqrt{3}+ \sqrt{4-(y-1)^2}\right)^2  \mathrm{\,d}y\approx 5{,}907.$$
	Chi phí để sản xuất đèn ngủ là $V\cdot40\approx236$ (nghìn đồng)
	}
\end{ex}

%

\begin{ex}%[Nguồn: Bộ đề minh họa Moon 2024-2025]%[2D5C1-4]
	Một công ty có hai chi nhánh. Sản phẩm của chi nhánh I chiếm $60\%$ còn chi nhánh II chiếm $40\%$ tổng sản phẩm của công ty. Tỉ lệ sản phẩm bị lỗi của chi nhánh I chiếm $1\%$ còn của chi nhánh II chiếm $2\%$ tổng sản phẩm công ty. Chọn ngẫu nhiên một sản phẩm của công ty.
	\choiceTF
	{Xác suất để sản phẩm của chi nhánh I được chọn là $0,4$}
	{Xác suất để lấy ra sản phẩm bị lỗi ở chi nhánh II là $0,02$}
	{Xác suất lấy ra sản phẩm bị lỗi là $0,015$}
	{Biết rằng sản phẩm lấy ra bị lỗi. Xác suất sản phẩm đó do chi nhánh I sản xuất là $\dfrac{4}{7}$}
	\loigiai{Gọi $A$ là biến cố \lq\lq Sản phẩm được chọn từ chi nhánh I\rq\rq.\\
	Gọi $B$ là biến cố \lq\lq Sản phẩm được chọn từ chi nhánh II\rq\rq.\\
	Gọi $C$ là biến cố \lq\lq Sản phẩm được chọn bị lỗi\rq\rq.\\
	Gọi $C_1$ là biến cố \lq\lq Sản phẩm bị lỗi do chi nhánh I sản xuất\rq\rq.\\
	Gọi $C_2$ là biến cố \lq\lq Sản phẩm bị lỗi do chi nhánh II sản xuất\rq\rq.\\
	Theo đề ta có $\mathrm{P}(A)=0{,}6$, $\mathrm{P}(B)=0{,}4$, $\mathrm{P}(C|A)=0{,}01$, $\mathrm{P}(C|B)=0{,}02$.
	\begin{itemchoice}
		\itemch Theo đề ta có xác suất để sản phầm của chi nhánh I được chọn là $\mathrm{P}(A)=0{,}6$.
		\itemch Xác suất để lấy ra sản phẩm bị lỗi ở chi nhánh II là \[\mathrm{P}(C_2)=\mathrm{P}(CB)=\mathrm{P}(B)\cdot\mathrm{P}(C|B)=0{,}4\cdot0{,}02=0{,}008.\]
		\itemch Xác suất sản phẩm lấy ra bị lỗi là \begin{eqnarray*}
			\mathrm{P}(C)&=&\mathrm{P}(C_1)+\mathrm{P}(C_2)\\
			&=&\mathrm{P}(A)\cdot\mathrm{P}(C|A)+\mathrm{P}(B)\cdot\mathrm{P}(C|B)\\
			&=&0{,}6\cdot 0{,}01+0{,}4\cdot 0{,}02\\
			&=&0{,}006+0{,}008\\
			&=&0{,}014.
		\end{eqnarray*}
		\itemch Áp dụng công thức Bayes ta có \[\mathrm{P}(A|C)=\dfrac{\mathrm{P}(CA)}{\mathrm{P}(C)}=\dfrac{\mathrm{P}(A)\cdot\mathrm{P}(C|A)}{\mathrm{P}(C)}=\dfrac{0{,}6\cdot 0{,}01}{0{,}014}=\dfrac{3}{7}.\]
	\end{itemchoice}}
\end{ex}

\begin{ex}%[Nguồn: Bộ đề minh họa Moon 2024-2025]%[2D5C1-4]
	Ở thị trấn của tôi, trời mưa một phần ba số ngày. Nếu trời mua, sẽ có khả năng xảy ra ùn tắc giao thông với xác suất $\dfrac{1}{2}$, nếu trời không mưa, sẽ có khả năng xảy ra ùn tắc giao thông là $\dfrac{1}{4}$. Nếu trời mưa và có ùn tắc giao thông, tôi sẽ đến muộn làm việc với xác suất $\dfrac{1}{2}$. Mặt khác, xác suất đến muộn là $\dfrac{1}{8}$ nếu trời không mưa và không có ùn tắc giao thông. Trong các tình huống khác (mưa và không có ùn tắc giao thông, không mưa và có ùn tắc giao thông), xác suất đến muộn của tôi đều là $0{,}25$. Chọn một ngày ngẫu nhiên mà tôi đi làm muộn, vậy xác suất trời mưa vào ngày hôm đó là bao nhiêu \%? Kết quả làm tròn đến hàng phần mười.
	\shortans[oly]{$54{,}5$}
	\loigiai{
	Để giải bài toán này, chúng ta sẽ sử dụng định lý Bayes để tính xác suất trời mưa vào ngày bạn đi làm muộn. Chúng ta xác định các sự kiện và xác suất tương ứng:
	
	\begin{itemize}
		\item $R$ : Trời mưa.
		\item \( \overline{R} \): Trời không mưa.
		\item \( T \): Có ùn tắc giao thông.
		\item \( \overline{T} \): Không có ùn tắc giao thông.
		\item \( L \): Đi làm muộn.
	\end{itemize}
	
	Các xác suất đã cho:
	\begin{align*}
		P(R) &= \frac{1}{3}, \\
		P(\overline{R}) &= \frac{2}{3}, \\
		P(T|R) &= \frac{1}{2}, \\
		P(T|\overline{R}) &= \frac{1}{4}, \\
		P(L|R \cap T) &= \frac{1}{2}, \\
		P(L|\overline{R} \cap \overline{T}) &= \frac{1}{8}, \\
		P(L|R \cap \overline{T}) &= 0.25, \\
		P(L|\overline{R} \cap T) &= 0.25.
	\end{align*}
	
	Chúng ta cần tính \( P(R|L) \), xác suất trời mưa khi bạn đi làm muộn.
	
	Đầu tiên, tính xác suất đi làm muộn \( P(L) \):
	
	\[
	P(L) = P(L|R \cap T)P(R \cap T) + P(L|R \cap \overline{T})P(R \cap \overline{T}) + P(L|\overline{R} \cap T)P(\overline{R} \cap T) + P(L|\overline{R} \cap \overline{T})P(\overline{R} \cap \overline{T})
	\]
	
	Tính các xác suất giao:
	
	\begin{align*}
		P(R \cap T) &= P(R)P(T|R) = \frac{1}{3} \times \frac{1}{2} = \frac{1}{6}, \\
		P(R \cap \overline{T}) &= P(R)P(\overline{T}|R) = \frac{1}{3} \times \frac{1}{2} = \frac{1}{6}, \\
		P(\overline{R} \cap T) &= P(\overline{R})P(T|\overline{R}) = \frac{2}{3} \times \frac{1}{4} = \frac{1}{6}, \\
		P(\overline{R} \cap \overline{T}) &= P(\overline{R})P(\overline{T}|\overline{R}) = \frac{2}{3} \times \frac{3}{4} = \frac{1}{2}.
	\end{align*}
	
	Thay vào công thức \( P(L) \):
	
	\[
	P(L) = \frac{1}{2} \times \frac{1}{6} + 0.25 \times \frac{1}{6} + 0.25 \times \frac{1}{6} + \frac{1}{8} \times \frac{1}{2} = \frac{1}{12} + \frac{1}{24} + \frac{1}{24} + \frac{1}{16} = \frac{11}{48}
	\]
	
	Tiếp theo, tính \( P(R|L) \):
	
	\[
	P(R|L) = \frac{P(L|R)P(R)}{P(L)}
	\]
	
	Tính \( P(L|R) \):
	
	\[
	P(L|R) = P(L|R \cap T)P(T|R) + P(L|R \cap \overline{T})P(\overline{T}|R) = \frac{1}{2} \times \frac{1}{2} + 0.25 \times \frac{1}{2} = \frac{1}{4} + \frac{1}{8} = \frac{3}{8}
	\]
	
	Thay vào công thức Bayes:
	
	\[
	P(R|L) = \frac{\frac{3}{8} \times \frac{1}{3}}{\frac{11}{48}} = \frac{\frac{1}{8}}{\frac{11}{48}} = \frac{1}{8} \times \frac{48}{11} = \frac{6}{11} \approx 0.5455
	\]
	
	Vậy xác suất trời mưa vào ngày bạn đi làm muộn là khoảng \( 54.5\% \).
	}
\end{ex}

\begin{ex}%[Nguồn: Bộ đề minh họa Moon 2024-2025]%[2D5C1-4]
	Một công ty kinh doanh $2$ mặt hàng là $A$ và $B$. Xác suất có lãi của mặt hàng $A$ là $0{,}6$ và xác suất có lãi của mặt hàng $B$ là $0{,}7$. Xác suất chỉ có mặt hàng $A$ có lãi là $0{,}2$.
	\begin{itemize}
		\item Gọi $A$ là biến cố \lq \lq Mặt hàng $A$ có lãi \rq \rq.
		\item Gọi $B$ là biến cố \lq \lq Mặt hàng $B$ có lãi \rq \rq.
		
	\end{itemize}
	\choiceTF
	{\True $\mathrm{P}(A\overline{B})=0{,}2$}
	{Xác suất để cả $2$ mặt hàng đều có lãi là $0{,}5$}
	{\True Xác suất để có đúng một mặt hàng có lãi là $0{,}5$}
	{\True Xác suất để mặt hàng $B$ có lãi biết mặt hàng $A$ không có lãi là $0{,}75$}
	\loigiai{
	\begin{itemchoice}
		\itemch Theo đề, ta có $\mathrm{P}(A)= 0{,}6$; $ \mathrm{P}(B) = 0{,}7$ và $\mathrm{P}(A\overline{B}) = 0{,}2$.\\
		\itemch Ta có
		\[\mathrm{P}(A)=(AB)+\mathrm{P}(A\overline{B})\]
		Do đó, xác suất để cả hai mặt hàng đều có lãi là \[\mathrm{P}(AB) = \mathrm{P}(A)-\mathrm{P}(A\overline{B})=0{,}6 - 0{,}2=0{,}4.\]
		\itemch Xác suất để chỉ có một mặt hàng có lãi là
		\begin{align*}
			\mathrm{P}(A\overline{B}) + \mathrm{P}(\overline{A}B) &= \mathrm{P}(A) - \mathrm{P}(AB) + \mathrm{P}(B) - P(AB) \\
			&= 0{,}6 - 0{,}4 + 0{,}7 - 0{,}4 = 0{,}5.
		\end{align*}
		\itemch Xác suất để mặt hàng $B$ có lãi biết mặt hàng $A$ không có lãi là
		\begin{align*}
			\mathrm{P}(B\mid\overline{A}) = \dfrac{\mathrm{P}(B\overline{A})}{\mathrm{P}(\overline{A})} = \dfrac{\mathrm{P}(B) - \mathrm{P}(AB)}{1 - \mathrm{P}(A)} = \dfrac{0{,}7 - 0{,}4}{1 - 0{,}6} = \dfrac{0{,}3}{0{,}4} = 0{,}75.
		\end{align*}
	\end{itemchoice}
	}
\end{ex}

\begin{ex}%[Nguồn: Bộ đề minh họa Moon 2024-2025]%[2D5H2-4]
	Khi trả lời câu hỏi trong một bài thi trắc nghiệm, học sinh có thể biết đáp án hoặc dự đoán đáp án. Các câu hỏi trắc nghiệm có $4$ đáp án nhưng chỉ có $1$ đáp án đúng. Giả sử với mỗi câu hỏi bạn Tuấn có xác suất biết đáp án đúng là $0{,}6$ và xác suất Tuấn không biết đáp án đúng là $0{,}4$. Trong trường hợp không biết đáp án Tuấn sẽ dự đoán đáp án đúng bằng cách chọn ngẫu nhiên một trong $4$ đáp án của đề thi. Giả sử Tuấn gặp một câu hỏi trắc nghiệm.\\
	Gọi $A$ là biến cố \lq\lq Câu trả lời của Tuấn là đúng\rq\rq.\\
	Gọi $B$ là biến cố \lq\lq Câu hỏi đó Tuấn đã biết đáp án\rq\rq.\\
	\choiceTF
	{\True $\mathrm{P}(B)=0{,}6$, $\mathrm{P}\left(\overline{B} \right)=0{,}4$}
	{Xác suất có điều kiện: $\mathrm{P}(A\mid B)=0{,}5$; $\mathrm{P}\left( A\mid\overline{B}\right)=0{,}25 $}
	{Xác suất để câu trả lời của Tuấn đúng là $72\%$}
	{\True Với câu trắc nghiệm mà câu trả lời của Tuấn là đúng, xác suất để câu đó là câu mà Tuấn biết đáp án đúng là $\dfrac{6}{7}$}
	\loigiai{
	\begin{itemchoice}
		\itemch Vì câu hỏi bạn Tuấn có xác suất biết đáp án đúng là $0{,}6$ và xác suất Tuấn không biết đáp án đúng là $0{,}4$ nên $\mathrm{P}(B)=0{,}6$, $\mathrm{P}\left(\overline{B} \right)=0{,}4$.
		\itemch Xác suất câu trả lời của Tuấn là đúng khi Tuấn đã biết đáp án bằng $\mathrm{P}(A\mid B)= 1$.\\
		Xác suất câu trả lời của Tuấn là đúng khi Tuấn chưa biết đáp án bằng $\mathrm{P}\left( A\mid\overline{B}\right)=0{,}25$.
		\itemch Áp dụng công thức xác suất toàn phần $\mathrm{P}(A) = \mathrm{P}(B)\mathrm{P}(A\mid B) + \mathrm{P}\left( \overline{B}\right) \mathrm{P}\left( A\mid\overline{B}\right) $ ta có
		$$\mathrm{P}(A) = 0{,}6 \cdot 1 + 0{,}4 \cdot 0{,}25 = 0{,}7.$$
		Vậy xác suất để câu trả lời của Tuấn đúng là $70\%$.
		\itemch Áp dụng công thức Bayes $\mathrm{P}(B|A) = \dfrac{\mathrm{P}(B)\mathrm{P}(A\mid B)}{\mathrm{P}(A)}$ ta có
		$$\mathrm{P}\left( B\mid A\right)  = \dfrac{0{,}6 \cdot 1}{0{,}7} = \dfrac{6}{7}.$$
		Vậy với câu trắc nghiệm mà câu trả lời của Tuấn là đúng thì xác suất để câu đó là câu mà Tuấn biết đáp án đúng là $\dfrac{6}{7}$.
	\end{itemchoice}
	}
\end{ex}

\begin{ex}%[Nguồn: Bộ đề minh họa Moon 2024-2025]%[2D5V1-3]
	Một hồi cứu về một bệnh nhân ung thư vú đã phẫu thuật cho kết quả với tỉ lệ sống trên $5$ năm là $60\%$ và tỉ lệ di căn là $30\%$. Biết rằng số bệnh nhân vừa sống trên $5$ năm vừa di căn chỉ bằng một nửa số bệnh nhân vừa không di căn vừa sống không quá $5$ năm. Một bệnh nhân bị ung thư vú và không di căn, tính xác suất để người này sống trên $5$ năm. Làm tròn kết quả đến chữ số thập phân thứ hai.
	\shortans[oly]{0{,}71}
	\loigiai{\begin{center}
		\begin{tikzpicture}[very thick,>=stealth',scale=1]
			\def\gocm{20}
			\def\gocn{10}
			\def\r{4.5}
			\path(0,0)node(O){}++(\gocm:\r)node(A1){$A$}++(\gocn:\r)node(A2){$\mathrm{P}\left( \overline{B}\mid A\right)=a $};
			\path(A1)++({-\gocn}:\r)node(a2){$\mathrm{P}\left( B\mid A\right) =1-a$};
			\path(O)++(-\gocm:\r)node(B1){$\overline{A}$}++(\gocn:\r)node(B2){$\mathrm{P}\left( B\mid \overline{A}\right) =b$};
			\path(B1)++({-\gocn}:\r)node(b2){$\mathrm{P}\left( \overline{B}\mid \overline{A}\right) =1-b$};
			%%Node dòng trên
			\path (A1)++(0,2.5)node(T1)[above]{\textbf{Sống trên 5 năm}} (A2)++(0,1.7)node(T2)[above]{\textbf{Người di căn}};
			\foreach \x/\y in {
			O/A1,A1/A2,
			O/B1,B1/B2,
			A1/a2,
			B1/b2}
			\draw[-stealth](\x.east)--(\y.west);
			\foreach \x/\y in {T1/A1,T2/A2}
			\draw[-stealth](\x.south)--(\y.north);
			\path(O)--(A1.west)node[pos=0.5,above,sloped]{$0{,}6$}(O)--(B1.west)node[pos=0.5,below,sloped]{$0{,}4$}(B1.east)--(B2.west)node[pos=0.5,above,sloped]{$a$}(A1.east)--(A2.west)node[pos=0.5,above,sloped]{$b$}
			(A1.east)--(a2.west)
			(B1.east)--(b2.west);
		\end{tikzpicture}
	\end{center}
	Gọi $A$ là biến cố: \lq\lq Người này sống trên $5$ năm\rq\rq, $B$ là biến cố: \lq\lq Người này có di căn\rq\rq.\\
	Khi đó $\mathrm{P}(A)=0{,}6$; $\mathrm{P}\left(\overline{A} \right)=0{,}4$; $\mathrm{P}(B)=0{,}3$; $\mathrm{P}\left(\overline{B} \right)=0{,}7$.\\
	Gọi $a$ là số người di căn trong số những người sống trên $5$ năm, $b$ là số người di căn trong số những người sống không quá $5$ năm.\\
	Ta có $\mathrm{P}(B) = \mathrm{P}(A B) + \mathrm{P}\left( \overline{A} B\right)  = 0{,}6a + 0{,}4b = 0{,}3$ \quad (1).\\
	Số bệnh nhân vừa sống trên $5$ năm vừa di căn chỉ bằng một nửa số bệnh nhân vừa không di căn vừa sống không quá 5 năm nên $\mathrm{P}(A  B) = \dfrac{1}{2} \mathrm{P}\left( \overline{A} \overline{B}\right)$. \\
	Suy ra $0{,}6a=\dfrac{1}{2}\cdot0{,}4\cdot(1-b)\Leftrightarrow0{,}6a+0{,}2b=0{,}2$ \quad (2).\\
	Từ (1) và (2) ta có hệ phương trình, giải hệ ta được $\heva{&a=\dfrac{1}{6}\\&b=\dfrac{1}{2}.}$\\
	Xác suất bệnh nhân bị ung thư vú và không di căn sống trên $5$ năm là
	$$\mathrm{P}\left(A\mid \overline{B} \right)=\dfrac{\mathrm{P}\left(A\overline{B} \right) }{\mathrm{P}\left( \overline{B}\right) }=\dfrac{0{,}6\cdot(1-a)}{0{,7}}\approx0{,}71.$$
	}
\end{ex}

%

\begin{ex}%[Nguồn: Bộ đề minh họa Moon 2024-2025]%[2D5V1-4]
	Email rác hay spam email là những email hàng loạt được gửi đi mà không có sự đồng ý của người nhận, dẫn đến những trải nghiệm không tốt cho người dùng. Nội dung của spam email thường chứa các thông tin lừa đảo, quảng bá hoặc rao bán các sản phẩm, dịch vụ có vấn đề. Người ta đã sử dụng một thuật toán để phân loại thư rác, biết rằng thuật toán này có thể phân loại đến $99\%$ thư rác và tỉ lệ sai sót khi phân loại thư bình thường thành thư rác là $5\%$. Thống kê cho thấy rằng, bình quân cứ $1\,000$ thư được phân loại đúng thì có $410$ thư rác. Tỉ lệ thư điện tử (email) gửi đến một địa chỉ là thư rác là bao nhiêu \%? (kết quả làm tròn đến hàng đơn vị).
	
	\shortans{$41$}
	
	\loigiai{
	Gọi $A$ là biến cố \lq\lq Email là thư rác\rq\rq.\\
	Gọi $B$ là biến cố \lq\lq Email được phân loại đúng\rq\rq.\\
	$\overline{A}$ là biến cố \lq\lq Email là thư bình thường (không phải thư rác)\rq\rq.\\
	Ta có\\
	Xác suất phân loại đúng thư rác là $\mathrm{P}(B|A) = 99\% = 0,99$.\\
	Xác suất phân loại đúng thư bình thường là  $\mathrm{P}(B|\overline{A}) = 95\% = 0{,}95$.\\
	Mặt khác, số thư phân loại đúng là $1\,000$, trong đó có  $410$ thư rác phân loại đúng. Suy ra
	\[
	\mathrm{P}(B|A)\cdot\mathrm{P}(A) = \dfrac{410}{1\,000} = 0.41
	\]
	Thay $\mathrm{P}(B|A) = 0,99$, ta tìm  $\mathrm{P}(A)$
	\[
	0,99\cdot\mathrm{P}(A) = 0,41\Rightarrow \mathrm{P}(A) = \dfrac{0,41}{0,99} \approx 0,4141.
	\]
	Vậy tỉ lệ thư rác trong tổng số email gửi đến là $41\%$.
	}
\end{ex}

%

\begin{ex}%[Nguồn: Bộ đề minh họa Moon 2024-2025]%[2D5V2-2]
	Hộp I có $5$ bi trắng và $5$ bi đen. Hộp II có $6$ bi trắng và $4$ bi đen. Bỏ hai viên bi từ hộp I sang hộp II. Sau đó lấy ra ngẫu nhiên $1$ viên bi từ hộp II.
	\begin{itemize}
		\item Gọi $K$ là biến cố: \lq\lq bi lấy ra từ hộp II là của hộp I\rq\rq.
		\item Gọi $A$ là biến cố: \lq\lq bi lấy ra là bi trắng\rq\rq.
	\end{itemize}
	\choiceTF
	{\True Xác suất để lấy được bi ra từ hộp II của hộp I là $\dfrac{1}{6}$}
	{Xác suất có điều kiện $\mathrm{P}(A \mid \overline{K})=\dfrac{2}{5}$}
	{Xác suất để lấy được bi trắng là $\mathrm{P}(A)=\dfrac{5}{6}$}
	{Giả sử lấy được bi trắng từ hộp II. Xác suất để lấy được bi trắng của hộp I là $\dfrac{1}{7}$}
	\loigiai{
	Gọi các biến cố
	\begin{itemize}
		\item $B_1$: \lq\lq hai viên bi lấy ra từ hộp I có màu trắng\rq\rq;
		\item $B_2$: \lq\lq hai viên bi lấy ra từ hộp I có 1 màu trắng và 1 màu đen\rq\rq;
		\item $B_3$: \lq\lq hai viên bi lấy ra từ hộp I có màu đen\rq\rq.
	\end{itemize}
	Khi đó
	\[\mathrm{P}(B_1)=\dfrac{\mathrm{C}^2_5}{\mathrm{C}^2_{10}}=\dfrac{2}{9}; \quad \mathrm{P}(B_2)=\dfrac{5\cdot5}{\mathrm{C}^2_{10}}=\dfrac{5}{9};\quad \mathrm{P}(B_3)=\dfrac{\mathrm{C}^2_5}{\mathrm{C}^2_{10}}=\dfrac{2}{9}.\]
	\begin{enumerate}[a)]
		\item $\mathrm{P}(K)=\dfrac{2}{12}=\dfrac{1}{6}$.
		\item $\mathrm{P}(A \mid \overline{K})=\dfrac{\mathrm{P}(A \cap \overline{K})}{\mathrm{P}(\overline K)}=\dfrac{\dfrac{6}{12}}{\dfrac{5}{6}}=\dfrac{3}{5}$.
		\item
		$\mathrm{P}(A)=\mathrm{P}(B_1)\cdot \mathrm{P}(A\mid B_1)+\mathrm{P}(B_2)\cdot \mathrm{P}(A\mid B_2)+\mathrm{P}(B_3)\cdot \mathrm{P}(A\mid B_3)=\dfrac{2}{9}\cdot\dfrac{8}{12}+\dfrac{5}{9}\cdot\dfrac{7}{12}+\dfrac{2}{9}\cdot\dfrac{6}{12}=\dfrac{7}{12}$.
		\item $\mathrm{P}(B_1\mid A)+\mathrm{P}(B_2\mid A)=\dfrac{\mathrm{P}(B_1)\cdot \mathrm{P}(A\mid B_1)}{\mathrm{P}(A)}+\dfrac{\mathrm{P}(B_2)\cdot \mathrm{P}(A\mid B_2)}{\mathrm{P}(A)}=\dfrac{\dfrac{2}{9}\cdot\dfrac{8}{12}+\dfrac{5}{9}\cdot\dfrac{7}{12}}{\dfrac{7}{12}}=\dfrac{17}{21}$.
	\end{enumerate}}
\end{ex}

%

\begin{bt}%[Nguồn: Bộ đề minh họa Moon 2024-2025]	%[2D6C1-4]
	Có hai chiếc hộp, hộp I có $6$ quả bóng màu đỏ và $4$ quả bóng màu vàng, hộp II có $7$ quả bóng màu đỏ và $3$ quả bóng màu vàng, các quả bóng có cùng kích thước và khối lượng. Lấy ngẫu nhiên một quả bóng từ hộp I bỏ vào hộp II. Sau đó, lấy ra ngẫu nhiên một quả bóng từ hộp II. Tính xác suất để quả bóng được lấy ra từ hộp II là quả bóng được chuyển từ hộp I sang, biết rằng quả bóng đó có màu đỏ (làm tròn kết quả đến hàng phần trăm).
	\shortans{0,08}
	\loigiai{
	
	Gọi $A$ là xác suất quả bóng lấy ra từ hộp II là bóng chuyển từ hộp I sang.
	
	Gọi $B$ là xác suất quả bóng lấy ra từ hộp II là bóng màu đỏ. Ta cần tính $P(A\mid B)$.
	
	Nếu ta lấy được một quả bóng đỏ từ hộp $I$ (xác suất là $0{,}6$) rồi bỏ vào hộp II thì hộp II lúc này sẽ có $8$ bóng đỏ và $3$ bóng vàng.
	
	Nếu ta lấy được một quả bóng vàng từ hộp $I$ (xác suất là $0{,}4$) rồi bỏ vào hộp II thì hộp II lúc này sẽ có $7$ bóng đỏ và $4$ bóng vàng. Suy ra Xác suất để lấy được quả bóng màu đỏ ở hộp II là
	$$
	P(B)=0{,}6 \cdot \dfrac{8}{11}+0{,}4 \cdot \dfrac{7}{11}=\dfrac{38}{55}.
	$$
	
	Xác suất để lấy được quả bóng lấy được từ hộp II là quả bóng chuyển từ hộp I sang và có màu đỏ là
	$$
	P(A \cap B)=0{,}6 \cdot \dfrac{1}{11}+0{,}4 \cdot 0=\dfrac{3}{55}.
	$$
	Theo công thức xác suất có điều kiện
	$$
	P(A\mid B)=\dfrac{P(A \cap B)}{P(B)}=\dfrac{3}{38} \approx 0{,}08.
	$$
	Vậy xác suất cần tìm là $0{,}08$.
	}
\end{bt}

\begin{ex}%[Nguồn: Bộ đề minh họa Moon 2024-2025]%[2D6C2-3]
	Có một kho chứa bia kém chất lượng chứa các thùng giống nhau (24 lon/thùng) gồm $2$ loại: loại $I$ để lẫn mỗi thùng $5$ lon quá hạn sử dụng, loại $II$ để lẫn mỗi thùng $3$ lon quá hạn. Biết số lượng thùng loại $I$ gấp $2$ lần số lượng thùng loại $II$. Chọn ngẫu nhiên $1$ thùng từ trong kho, từ thùng đó chọn ngẫu nhiên $10$ lon thì thấy trong $10$ lon đó có hai lon quá hạn sử dụng. Tính xác suất $10$ lon được lấy là bia loại $I$ (làm tròn kết quả đến hàng phần trăm).
	\loigiai{
	Gọi $A_1$ là biến cố chọn được thùng loại I.\\
	$A_2$ là biến cố chọn được thùng loại II.\\
	$B$ là biến cố chọn được 10 lon bia trong đó có hai lon quá hạn sử dụng từ thùng được chọn ra.\\
	Từ đó ta có $\mathrm{P}(A_1)=\dfrac{2}{3}$; $\mathrm{P}(A_1)=\dfrac{1}{3}$.\\
	Áp dụng công thức xác suất có điều kiện ta có
	$\mathrm{P}(B|A_1)=\dfrac{\mathrm{C}_5^2\mathrm{C}_{19}^8}{\mathrm{C}_{24}^{10}}=\dfrac{195}{506}$;	$\mathrm{P}(B|A_2)=\dfrac{\mathrm{C}_3^2\mathrm{C}_{21}^8}{\mathrm{C}_{24}^{10}}=\dfrac{315}{1012}$.\\
	Xác suất để chọn được hai lon quá hạn là
	\allowdisplaybreaks
	\begin{eqnarray*}
		\mathrm{P}(B)&=&\mathrm{P}(A_1)\cdot\mathrm{P}(B|A_1)+\mathrm{P}(A_2)\cdot\mathrm{P}(B|A_2)\\
		&=&\dfrac{2}{3}\cdot\dfrac{195}{506}+\dfrac{1}{3}\cdot\dfrac{315}{1012}\\
		&=&\dfrac{365}{1012}.
	\end{eqnarray*}
	Suy ra xác suất lấy được bia thuộc loại $I$ là\\
	$\mathrm{P}(A_1|B)=\dfrac{\dfrac{2}{3}\cdot\dfrac{195}{506}}{\dfrac{365}{1012}}=\dfrac{52}{73}\approx0{,}71$.
	}
\end{ex}

\begin{ex}%[Nguồn: Bộ đề minh họa Moon 2024-2025]%[2D6C2-4]
	Trong một báo cáo, xét nghiệm Mammography người mắc bệnh ung thư vú cho kết quả dương tính với xác suất là $90 \%$, người không mắc bệnh ung thư vú cho kết quả âm tính với xác suất $97 \%$. Nghiên cứu dịch tễ học chỉ ra tỉ lệ mắc ung thư vú của phụ nữ trong độ tuổi 55 là $1 \%$. Một phụ nữ 55 tuổi, không có tiền sử ung thư vú thực hiện xét nghiệm Mammography hai lần độc lập nhau đều nhận được kết quả là dương tính. Xác suất người phụ nữ đó mắc bệnh ung thư vú gần nhất với giá trị nào sau đây?
	\shortans[]{$90$}
	\loigiai{
	Xét hai biến cố\\
	$X$ : \lq\lq~Người phụ nữ mắc bệnh ung thư vú~\rq\rq.\\
	$Y$: \lq\lq~Kết quả xét nghiệm của người phụ nữ là dương tính~\rq\rq.\\
	Do người phụ nữ đó 55 tuổi nên $P(X)=1 \%=0{,}01 \Rightarrow P(\bar{X})=1-P(X)=0{,}99$. Khi đó
	\begin{itemize}
		\item  $P(X \mid Y)$ là xác suất người phụ nữ bị mắc bệnh ung thư vú khi có kết quả xét nghiệm dương tính.
		\item  $P(Y \mid X)$ là xác suất người phụ nữ có kết quả xét nghiệm là dương tính với điều kiện có mắc bệnh $\Rightarrow P(Y \mid X)=90 \%=0{,}9$.
		\item  $P(Y \mid \bar{X})$ là xác suất người phụ nữ có kết quả xét nghiệm là dương tính với điều kiện không mắc bệnh $\Rightarrow P(Y \mid \bar{X})=1-97 \%=0{,}03$.
	\end{itemize}
	Áp dụng công thức Bayes, ta có xác suất người phụ nữ bị mắc bệnh ung thư vú khi có kết quả xét nghiệm dương tính lần thứ nhất là
	$$
	P(X \mid Y)=\dfrac{P(X) \cdot P(Y \mid X)}{P(X) \cdot P(Y \mid X)+P(\bar{X}) \cdot P(Y \mid X)}=\dfrac{0{,}01 \cdot 0{,}9}{0{,}01 \cdot 0{,}9+0{,}99.0{,}03}=\dfrac{10}{43}
	$$
	Vì người phụ nữ xét nghiệm 2 lần đều ra kết quả dương tính nên ở lần thứ hai, tỉ lệ mắc ung thư vú của người phụ nữ này được chuyển thành $P(X)=\dfrac{10}{43}$.\\
	Do đó hoàn toàn tương tự thì xác suất người phụ nữ bị mắc bệnh ung thư vú khi có kết quả xét nghiệm dương tính lần thứ hai là
	$$
	P(X \mid Y)=\dfrac{P(X) \cdot P(Y \mid X)}{P(X) \cdot P(Y \mid X)+P(\bar{X}) \cdot P(Y \mid \bar{X})}=\dfrac{\dfrac{10}{43} \cdot 0{,}9}{\dfrac{10}{43} \cdot 0{,}9+\left(1-\dfrac{10}{43}\right) \cdot 0{,}03}=\dfrac{100}{111} \approx 90 \%
	$$
	Vậy xác suất người phụ nữ bị mắc bệnh ung thư vú gần với $90 \%$.
	}
\end{ex}

\begin{ex}%[Nguồn: Bộ đề minh họa Moon 2024-2025]%[2D6H1-2]
	Cho hai biến cố $A$ và $B$, biết $\mathrm{P}(A)=0{,}6$, $\mathrm{P}\left(\overline{B}\right)=0{,}2$, $\mathrm{P}(AB)=0{,}42$.
	\choiceTF
	{\True $\mathrm{P}(B)=0{,}8$}
	{$A$ và $B$ là hai biến cố độc lập}
	{$\mathrm{P}\left(\overline{A}B\right)=0{,}48$}
	{\True $\mathrm{P}\left(B\mid \overline{A}\right)=0{,}95$}
	\loigiai{
	\begin{itemchoice}
		\itemch {\bf Đúng.}\\ Ta có $\mathrm{P}(B)=1-\mathrm{P}\left(\overline{B}\right)=0{,}8$.
		\itemch  {\bf Sai.}\\ Ta có $\mathrm{P}(A)\cdot \mathrm{P}(B)=0{,}6\cdot 0{,}8 =0{,}48$.\\
		Suy ra $\mathrm{P}(AB)\neq \mathrm{P}(A)\cdot \mathrm{P}(B)$ nên hai biến cố $A$ và $B$ không phải là hai biến cố độc lập.
		\itemch  {\bf Sai.}\\ Ta có sơ đồ cây như sau
		\begin{center}
			\begin{tikzpicture}[>=stealth,line cap=round,line join=round]
				\path(0,0)node(a){Xác suất }
				++(3,2)node(b){$ A $} ++(3,1.5)node(d){$ B $} (b)++(3,-1.5)node(e){$ \overline{B} $}
				(a)++(3,-2)node(c){$\overline{A} $}++(3,1.5)node(f){$ B $}(c)++(3,-1.5)node(g){$ \overline{B} $}
				;
				\draw(a)--(b) (b)--(e) (b)--(d) (a)--(c) --(f) (c)--(g);
			\end{tikzpicture}
		\end{center}
		Do đó
		$\mathrm{P}(B)=\mathrm{P}\left(\overline{A}B\right)+ \mathrm{P}\left(AB\right)$.\\
		Suy ra $\mathrm{P}\left(\overline{A}B\right)=\mathrm{P}(B)-\mathrm{P}\left(AB\right)=0{,}8-0{,}42 =0{,}38$.
		\itemch {\bf Đúng.}\\ Ta có $\mathrm{P}\left(B\mid \overline{A}\right) =\dfrac{\mathrm{P}\left(\overline{A}B\right)}{\mathrm{P}\left(\overline{A}\right)}=\dfrac{0{,}38}{1-0{,}6}=0{,}95$.
	\end{itemchoice}
	
	}
\end{ex}

%\indapan{10}{ans/ans-TN-DE 22}

\cauds
\Opensolutionfile{ans}[ans/ans-DS-DE 22]

\begin{ex}%[Nguồn: Bộ đề minh họa Moon 2024-2025]%[2D6H1-2]
	Khảo sát $100$ người trong đó có $49$ nam và $51$ nữ về việc có nuôi thú cưng không thì được bảng sau:
	\begin{center}
		\begin{tabular}{|c|c|c|c|}
			\hline
			&Có nuôi thú cưng&Không có thú cưng&Tổng\\
			\hline
			Nam& 41 & 8 & 49 \\
			\hline
			Nữ& 45 & 6 & 51 \\
			\hline
			Tổng& 86 & 14 & 100 \\
			\hline
		\end{tabular}
	\end{center}
	Chọn ngẫu nhiên một người trong số người được khảo sát.
	\choiceTF
	{Xác suất người được chọn là nam là $0{,}41$}
	{\True Xác suất người được chọn nuôi thú cưng là $0{,}86$}
	{Xác suất người được chọn là nam và nuôi thú cưng là $0{,}45$}
	{\True Biết người được chọn là nam, xác suất người đó nuôi thú cưng là $\dfrac{41}{49}$}
	\loigiai{
	\begin{itemize}
		\item Gọi $A$ là biến cố \lq\lq Người đó nuôi thú cưng\rq\rq.
		\item Gọi $B$ là biến cố \lq\lq Người đó là nam\rq\rq.
	\end{itemize}
	\begin{itemchoice}
		\itemch Xác suất người được chọn là nam bằng $\mathrm{P}(B)=\dfrac{49}{100}=0{,}49$.
		\itemch Xác suất người được chọn nuôi thú cưng là $\mathrm{P}(A)=\dfrac{86}{100}=0{,}86$.
		\itemch Xác suất người được chọn là nam và nuôi thú cưng bằng $\mathrm{P}(AB)=\dfrac{41}{100}=0{,}41$.
		\itemch Biết người được chọn là nam, xác suất người đó nuôi thú cưng là $\mathrm{P}(A|B)=\dfrac{\mathrm{P}(AB)}{\mathrm{P}(B)}=\dfrac{41}{49}$.
	\end{itemchoice}
	}
\end{ex}

%Câu 2

\begin{ex}%[Nguồn: Bộ đề minh họa Moon 2024-2025]%[2D6H1-2]
	Gieo hai con xúc xắc cân đối và đồng chất. Tính xác suất để tổng số chấm xuất hiện trên hai con xúc xắc đó không nhỏ hơn $10$ nếu biết rằng ít nhất một con xúc xắc xuất hiện mặt $5$ chấm (kết quả làm tròn đến hàng phần trăm).
	\shortans[]{$0{,}27$}
	\loigiai{
	Gọi $A$ là biến cố  \lq\lq Tổng số chấm trên $2$ con xúc xắc không nhỏ hơn $10$\rq\rq.\\
	Gọi $B$ là biến cố \lq\lq Có ít nhất $1$ con xúc xắc xuất hiện mặt $5$ chấm\rq\rq.\\
	Khi đó:\\
	$A=\{(4;6);(6;4); (5;5); (5; 6); (6;5);(6;6)\}$ $\Rightarrow n(A)=6$.\\
	$B=\{(1;5); (2;5); (3;5); (4;5); (5;5); (6;5); (5;1); (5; 2); (5;3); (5;4); (5;6)\}$ $\Rightarrow n(B)=11$.\\
	$A \cap B=\{(5;5); (5;6); (6;5)\}$ $\Rightarrow n(A \cap B)=3$.\\
	$\mathrm{P}(A \mid B)=\dfrac{n(A B)}{n(B)}=\dfrac{3}{11}\approx 0{,}27$.
	}
\end{ex}

\begin{ex}%[Nguồn: Bộ đề minh họa Moon 2024-2025]%[2D6H1-2]
	Trong một giải đấu bóng đá, đội bóng của Hưng thi đấu $30$ trận với đúng một trong hai chiến thuật hoặc tấn công hoặc phòng ngự xuyên suốt cả trận đấu. Biết rằng, số trận thi đấu theo chiến thuật tấn công gấp đôi số trận thi đấu theo chiến thuật phòng ngự. Khi chơi theo chiến thuật tấn công, họ thắng $30\%$ số trận và thua $20\%$ số trận. Khi họ chơi theo chiến thuật phòng ngự, họ thắng $20\%$ số trận và thua $40\%$ số trận. Mỗi trận thắng đội được cộng $3$ điểm, mỗi trận hoà đội được cộng $1$ điểm và không cộng điểm cho trận thua. Tổng số điểm đội bóng đạt được trong giải là bao nhiêu điểm?
	\loigiai{
	Gọi số trận thi đấu theo chiến thuật tấn công, phòng ngự lần lượt là $x$, $y$.\\
	Ta có hệ phương trình $\heva{&x+y=30\\&x-2y=0}\Leftrightarrow \heva{&x=20\\&y=10.}$\\
	Số điểm mà đội bóng đạt được từ chơi theo chiến thuật tấn công là \\
	$20\cdot 0{,}3\cdot3+20\cdot0{,}5\cdot1+20\cdot0{,}2\cdot0=28$.\\
	Số điểm mà đội bóng đạt được khi chơi theo chiến thuật phòng ngự là\\
	$10\cdot 0{,}2\cdot3+10\cdot0{,}4\cdot1+10\cdot0{,}4\cdot0=10$.\\
	Vậy tổng số điểm mà đội bóng đạt được trong giải là $28+10=38$ điểm.
	}
\end{ex}

\begin{ex}%[Nguồn: Bộ đề minh họa Moon 2024-2025]%[2D6H2-2]
	Một doanh nghiệp có $45\%$ nhân viên là nữ. Tỉ lệ nhân viên nữ và tỉ lệ nhân viên nam mua bảo hiểm nhân thọ lần lượt là $7\%$ và $5\%$. Chọn ngẫu nhiên một nhân viên của doanh nghiệp.
	\choiceTF
	{Xác suất nhân viên được chọn là nam là $0{,}45$}
	{Xác suất nhân viên được chọn có mua bảo hiểm nhân thọ là $0{,}061$}
	{\True Biết rằng nhân viên được chọn có mua bảo hiểm nhân thọ. Xác suất nhân viên đó là nữ là $\dfrac{63}{118}$}
	{Biết rằng nhân viên được chọn có mua bảo hiểm nhân thọ. Xác suất nhân viên đó là nam cao hơn là nữ}
	\loigiai{ Gọi $A$ là biến cố \lq~Nhân viên được chọn là nữ~\rq và $B$ là biến cố \lq~Nhân viên được chọn có mua bảo hiểm nhân thọ~\rq
	\begin{itemchoice}
		\itemch Theo đề ta có $P(A)=0{,}45$; $P(B \mid A)=0{,}07$;  $P(B \mid \bar{A})=0{,}05$; $P(\bar{A})=0{,}55$. Vậy xác suất nhân viên được chọn là nam là $0{,}55$.
		\itemch
		Áp dụng công thức xác suất đầy đủ ta có
		\begin{eqnarray*}
			P(B)&=&P(A) \cdot P(B \mid A)+P(\bar{A}) \cdot P(B \mid \bar{A}) \\
			&=&0{,}45 \cdot 0{,}07+0{,}55 \cdot 0{,}05=0{,}059 .
		\end{eqnarray*}
		Vậy xác suất nhân viên được chọn có mua báo hiểm nhân thọ là $0{,}059$ .
		\itemch Ta có $P(A \mid B)=\dfrac{P(A) \cdot P(B \mid A)}{P(B)}=\dfrac{0{,}45 \cdot 0{,}07}{0{,}059}=\dfrac{63}{118}$.\\
		Vậy xác suất nhân viên được chọn có mua bảo hiểm nhân thọ là nữ là $\dfrac{63}{118}$.
		\itemch  Ta có
		$P(\bar{A} \mid B)=\dfrac{P(\bar{A}) \cdot P(B \mid \bar{A})}{P(B)} =\dfrac{0{,}55 \cdot 0{,}05}{0{,}059}=\dfrac{55}{118} .$
		Vậy xác suất nhân viên được chọn có mua  bảo hiểm nhân thọ là nam thấp hơn nữ.
	\end{itemchoice}
	
	}
\end{ex}

\begin{ex}%[Nguồn: Bộ đề minh họa Moon 2024-2025]%[2D6H2-2]
	Hai bạn An và Bình chơi một trò chơi như sau
	\begin{itemize}
		\item  Mỗi người lần lượt bốc một viên bi từ một hộp đựng 2 bi đỏ và 4 bi xanh (bi khi được bốc ra, không hoàn lại vào hộp).
		\item Người nào bốc được bi đỏ trước thì là thắng cuộc.
	\end{itemize}
	Biết người được bốc trước giành chiến thắng, xác suất người đó giành chiến thắng ở lượt bốc thứ hai là bao nhiêu \%? (kết quả làm tròn đến hàng phần mười). \shortans[oly]{$33{,}3$}
	\loigiai{
	Gọi $A$ là biến cố \lq\lq An là người thắng cuộc\rq\rq, và $B$ là biến cố \lq\lq An thắng cuộc ở lần bốc thứ $2$\rq\rq.
	\begin{itemize}
		\item Xác suất An thắng cuộc ở lượt bốc thứ nhất là $\dfrac{2}{6}=\dfrac{1}{3}$.
		\item Xác suất An thắng cuộc ở lượt bốc thứ hai là $\dfrac{4}{6}\cdot \dfrac{3}{5}\cdot\dfrac{2}{4}=\dfrac{1}{5}$.
		\item Xác suất An thắng cuộc ở lượt bốc thứ $3$ là $\dfrac{4}{6}\cdot\dfrac{3}{5}\cdot\dfrac{2}{4}\dfrac{1}{3}\cdot 1=\dfrac{1}{15}$.
	\end{itemize}
	Suy ra $\mathrm{P}(A)=\dfrac{1}{3}+\dfrac{1}{5}+\dfrac{1}{15}=\dfrac{3}{5}$.\\
	Suy ra $\mathrm{P}\left(B\mid A\right)=\dfrac{\mathrm{P}\left(BA\right)}{\mathrm{P}(A)}=\dfrac{\tfrac{1}{5}}{\tfrac{3}{5}}=\dfrac{1}{3}\approx 33{,}3\%$.
	}
\end{ex}

%Câu đúng sai 3

\begin{ex}%[Nguồn: Bộ đề minh họa Moon 2024-2025]%[2D6H2-4]
	Một thùng chứa $100$ quả táo trong đó có $80\%$ số quả táo được dán nhãn, số còn lại không được dán nhãn. Bạn Hoàng lấy ra một quả trong thùng, sau đó bạn Hà lấy ra một quả thứ hai.
	\begin{itemize}
		\item Gọi $A$ là biến cố: \lq\lq Quả táo bạn Hoàng lấy ra có dán nhãn\rq\rq.
		\item Gọi $B$ là biến cố: \lq\lq Quả táo bạn Hà lấy ra có dán nhãn\rq\rq.
	\end{itemize}
	\choiceTF
	{\True $\mathrm{P}(A)=\dfrac{4}{5}$}
	{Xác suất có điều kiện $\mathrm{P}(B\mid A)=\dfrac{79}{100}$}
	{\True Xác xuất bạn Hà lấy ra quả táo có dán nhãn bằng $0{,}8$}
	{Biết rằng bạn Hà lấy ra quả táo có dán nhãn. Xác suất để Hoàng cũng lấy ra quả táo có dán nhãn là $20{,}2\%$ (làm tròn kết quả đến hàng phần mười theo đơn vị phần trăm)}
	\loigiai{
	Số quả táo được dán nhãn là $80\%\cdot 100=80$ quả; số táo không được dán nhãn là $20$ quả.\\
	Khi đó $\mathrm{P}(A)=\dfrac{80}{100}=\dfrac{4}{5}$; $\mathrm{P}(\overline{A})=1-\dfrac{4}{5}=\dfrac{1}{5}$; $\mathrm{P}(B\mid A)=\dfrac{79}{99}$; $\mathrm{P}(B\mid\overline{A})=\dfrac{80}{99}$.
	\begin{itemchoice}
		\itemch Ta có $n(A)=80$, suy ra $\mathrm{P}(A)=\dfrac{80}{100}=\dfrac{4}{5}$.
		\itemch Ta có $\mathrm{P}(B\mid A)=\dfrac{79}{99}$.
		\itemch Áp dụng công thức xác suất toàn phần, ta có
		$$ \mathrm{P}(B)=\mathrm{P}(A)\cdot \mathrm{P}(B\mid A) + \mathrm{P}(\overline{A})\cdot \mathrm{P}(B\mid\overline{A})=\dfrac{4}{5}\cdot\dfrac{79}{99}+\dfrac{1}{5}\cdot\dfrac{80}{99}=0{,}8.$$
		\itemch Xác suất để Hoàng cũng lấy ra quả táo có dán nhãn biết bạn Hà lấy ra quả táo có dán nhãn là $\mathrm{P}(A\mid B)$. Áp dụng công thức Bayes ta có
		$$ \mathrm{P}(A\mid B) =\dfrac{\mathrm{P}(A)\cdot \mathrm{P}(B\mid A)}{\mathrm{P}(B)} = \dfrac{\dfrac{4}{5}\cdot\dfrac{79}{99}}{0{,}8} \approx 79,{80}\%.$$
	\end{itemchoice}
	}
\end{ex}

\begin{ex}%[Nguồn: Bộ đề minh họa Moon 2024-2025]%[2D6V1-2]
	Một mạch điện gồm $2$ bộ phận mắc nối tiếp, với xác suất làm việc tốt trong một khoảng thời gian nào đó của mỗi bộ phận lần lượt là $0{,}95$ và $0{,}98$.
	\begin{itemize}
		\item Gọi $A$ là biến cố: \lq\lq Bộ phận thứ nhất hỏng\rq\rq.
		\item Gọi $B$ là biến cố: \lq\lq Bộ phận thứ hai hỏng\rq\rq.
		\item Gọi $H$ là biến cố: \lq\lq Mạch không hoạt động\rq\rq.
	\end{itemize}
	\choiceTFt
	{$\mathrm{P}\left(AB\right)=0{,}921$}
	{\True Xác suất có điều kiện $\mathrm{P}\left(H|A\right) = \mathrm{P}\left(H|B\right)$}
	{ Xác suất để mạch ngừng làm việc là $0{,}068$}
	{\True Ở một thời điểm trong khoảng thời gian trên người ta thấy mạch điện ngừng làm việc (do bộ phận nào đó hỏng). Xác suất để chỉ bộ phận thứ hai hỏng là $\dfrac{19}{69}$}
	\loigiai{
	\begin{itemchoice}
		\itemch Ta có $\mathrm{P}(A) = 1 - 0{,}95 = 0{,}05$;   $\mathrm{P}(B) = 1 - 0{,}98 = 0{,}02$.\\
		Do đó $\mathrm{P}\left(AB\right)=0{,}05 \cdot 0{,}02 = 0{,}001$.
		\itemch Đúng. Vì hai bộ phận mắc nối tiếp, mạch ngừng hoạt động khi ít nhất một trong hai bộ phận hỏng.\\
		Do đó, $H = A \cup B$.\\
		Xác suất có điều kiện $\mathrm{P}\left(H|A\right)$ là xác suất mạch hỏng khi bộ phận thứ nhất đã hỏng và $\mathrm{P}\left(H|B\right)$ là xác suất mạch hỏng khi bộ phận thứ hai đã hỏng.\\
		Vì mạch nối tiếp, nếu bộ phận thứ nhất hỏng hoặc thứ hai hỏng thì chắc chắn mạch hỏng.\\
		Vậy $\mathrm{P}\left(H|A\right)=\mathrm{P}\left(H|B\right)= 1$.
		\itemch Xác suất để mạch ngừng hoạt động là
		$$
		\mathrm{P}(H) = \mathrm{P}(A \cup B) = \mathrm{P}(A) + \mathrm{P}(B) - \mathrm{P}(A \cap B) = 0{,}05 + 0{,}02 - 0{,}001 = 0{,}069.
		$$
		\itemch Xác suất để chỉ bộ phận thứ hai hỏng khi mạch đã hỏng là
		$$
		\mathrm{P}\left(\overline{A}B|H\right) = \dfrac{\mathrm{P}\left(\overline{A}BH\right)}{\mathrm{P}(H)} = \dfrac{\mathrm{P}\left(\overline{A}B\right)}{\mathrm{P}(H)} =\dfrac{\mathrm{P}\left(\overline{A}\right)\mathrm{P}(B)}{\mathrm{P}(H)}= \dfrac{0{,}95\cdot 0{,}02}{0{,}069} = \dfrac{19}{69}.
		$$
	\end{itemchoice}
	}
\end{ex}

\begin{ex}%[Nguồn: Bộ đề minh họa Moon 2024-2025]%[2D6V1-2]
	Sách ID của Moon.vn được in tại hai phân xưởng $A$ và $B$ và đưa về kho sau khi in xong. Xưởng $A$ có nhiệm vụ in $60\%$ tổng số lượng sách, xưởng $B$ sẽ in số lượng sách còn lại. Biết rằng số lượng sách tại xưởng $A$ và $B$ in đạt yêu cầu về chất lượng và chuyển về kho lần lượt là $95\%$ và $90\%$. Nhân viên kiểm kho chọn ngẫu nhiên một cuốn sách ID để kiểm tra thì thấy cuốn sách này không đạt yêu cầu về chất lượng. Xác suất để cuốn sách ID đó được in ở xưởng A là bao nhiêu phần trăm? (Làm tròn kết quả đến hàng phần chục).
	\shortans[0]{$42{,}9$}
	\loigiai{
	Gọi $A$ là biến cố sách in tại xưởng $A$.\\
	Gọi $X$ là biến cố sách in đạt chất lượng.\\
	Ta có $\mathrm{P}(A)=0{,}6$, $\mathrm{P}(\overline{A})=0{,}4$.\\
	và $\mathrm{P}(X)=0{,}6\cdot 0{,}95+0{,}4\cdot0{,}9=0{,}93$.\\
	Xác suất cần tìm là $\mathrm{P}(A\mid \overline{X})=\dfrac{\mathrm{P}(A\overline{X})}{\mathrm{P}(\overline{X})}=\dfrac{0{,}6\cdot 0{,}05}{0{,}07}\approx 42{,}9\%$.
	}
\end{ex}

%

\begin{ex}%[Nguồn: Bộ đề minh họa Moon 2024-2025]%[2D6V1-4]
	Trước khi đưa một loại sản phẩm ra thị trường, người ta đã phỏng vấn ngẫu nhiên $200$ khách hàng về sản phẩm đó. Kết quả thống kê như sau: có $105$ người trả lời \lq\lq sẽ mua\rq\rq\; có $95$ người trả lời \lq\lq không mua\rq\rq. Kinh nghiệm cho thấy tỉ lệ khách hàng thực sự sẽ mua sản phẩm tương ứng với những cách trả lời \lq\lq sẽ mua\rq\rq\ và \lq\lq không mua\rq\rq\ lần lượt là $\mathbf{70\%}$ và $30\%$.
	\begin{itemize}
		\item Gọi $A$ là biến cố \lq\lq Người được phỏng vấn thực sự sẽ mua sản phẩm\rq\rq.
		\item Gọi $B$ là biến cố \lq\lq Người được phỏng vấn trả lời sẽ mua sản phẩm\rq\rq.
	\end{itemize}
	\choiceTF
	{\True Xác suất $P(B)=\dfrac{21}{40}$ và $P\left(\overline{B}\right) =\dfrac{19}{40}$}
	{Xác suất có điều kiện $P(A\mid B)=0{,}3$}
	{\True Xác suất $P(A)=0{,}51$}
	{Trong số những người được phỏng vấn thực sự sẽ mua sản phẩm có $70\%$ người đã trả lời \lq\lq sẽ mua\rq\rq\ khi được phỏng vấn (kết quả tính theo phần trăm được làm tròn đến hàng đơn vị)}
	\loigiai{
	\begin{enumerate}
		\item Đúng.
		
		Theo đề bài, có $200$ khách hàng được phỏng vấn, trong đó 105 người trả lời \lq\lq sẽ mua\rq\rq. Vậy xác suất người được phỏng vấn trả lời \lq\lq sẽ mua\rq\rq\ là $P(B)=\dfrac{105}{200}=\dfrac{21}{40}$.
		Xác suất người được phỏng vấn trả lời \lq\lq không mua\rq\rq\ là
		$$
		P(\overline{B})=\dfrac{95}{200}=\dfrac{19}{40} .
		$$
		\item Sai.
		
		Xác suất có điều kiện $P(A \mid B)$ là xác suất người thực sự mua sản phẩm (biến cố $A$ ) khi biết người đó trả lời \lq\lq sẽ mua\rq\rq\ (biến cố $B$ ). Theo đề bài, tỉ lệ khách hàng thực sự mua sản phẩm khi trả lời \lq\lq sẽ mua\rq\rq\ là $70 \%$, tức là $P(A \mid B)=0{,}7$.
		\item Đúng.
		
		Ta có $P(B)=\dfrac{21}{40}$ và $P(\overline{B})=\dfrac{19}{40}$.
		
		Theo giả thiết thì $P(A \mid B)=0{,}7$ và $P(A \mid \overline{B})=0{,}3$.
		Suy ra
		$$
		P(A)=P(A \mid B) \cdot P(B)+P(A \mid \overline{B}) \cdot P(\overline{B})=0{,}7 \cdot \dfrac{21}{40}+0{,}3 \cdot \dfrac{19}{40}=0{,}51 .
		$$
		
		\item Sai.
		
		Ta có $P(B \mid A) \cdot P(A)=P(A \cap B)=P(A \mid B) \cdot P(B)=\dfrac{7}{10} \cdot \dfrac{21}{40}$.
		
		Suy ra $P(B \mid A)=\dfrac{147}{400 \cdot P(A)}=\dfrac{147}{400 \cdot 0{,}51} \approx 72 \%$.
		
		Vậy trong số những người được phỏng vấn thực sự sẽ mua sản phẩm có khoảng $72 \%$ đã trả lời \lq\lq sẽ mua\rq\rq\ khi được phỏng vấn.
	\end{enumerate}
	}
\end{ex}

%

\begin{ex}%[Nguồn: Bộ đề minh họa Moon 2024-2025]%[2D6V1-4]
	Trong quán giải khát có bán các chai nước hoa quả với giả 15000 đồng một chai, các chai rỗng sẽ đối lấy 5000 đồng. Bạn có thể uống được nhiều nhất bao nhiêu chai nếu trong túi bạn có 100000 đồng?
	\shortans[]{$9$}
	\loigiai{
	Đầu tiên dùng 100000 đồng ta sẽ mua được 6 chai nước và dư 10000 đồng.\\
	Uống 6 chai ta sẽ được 6 chai rỗng, sau đó đem đi đổi ta sẽ được 30000 đồng.\\
	Dùng 30000 đồng trên mua được 2 chai nước mới.\\
	Uống 2 chai nước mới ta được 2 chai rỗng, đem đi đổi ta được 10000 đồng.\\
	Cộng với 10000 đồng còn dư ở phía trên ta được 20000 đồng và mua được 1 chai nước mới.\\
	Vậy ta có thể uống được nhiều nhất là 9 chai nước.
	}
\end{ex}

\begin{ex}%[Nguồn: Bộ đề minh họa Moon 2024-2025]%[2D6V2-2]
	Trong một ngôi làng có $500$ người thì $240$ người là nam. Thống kê cho thấy rằng, khả năng mắc bệnh hô hấp ở nam trong làng là $0{,}6\%$ và ở nữ trong làng là $0{,}35\%$. Giả sử gặp  một người trong làng.\\
	Gọi $A$ là biến cố \lq\lq gặp người mắc bệnh trong làng\rq\rq.\\
	Gọi $B$ là biến cố \lq\lq gặp người nam trong làng\rq\rq.
	\choiceTF
	{\True $P(B) = \dfrac{12}{25} \Rightarrow P(\overline{B}) = \dfrac{13}{25}$}
	{Xác suất có điều kiện $P(A \mid \overline{B}) = 0{,}0006$}
	{Tỉ lệ mắc bệnh hô hấp chung của cả làng là $0{,}42\%$}
	{Giả sử có một người trong làng không mắc bệnh. Xác suất để người đó là nữ là $47{,}94\%$}
	\loigiai{
	\begin{itemchoice}
		\itemch \textbf{Đúng}.\\
		Từ giả thiết ta có $P(B)=\dfrac{240}{500}=\dfrac{12}{25}\Rightarrow P(\overline{B}) = \dfrac{13}{25}$.
		\itemch \textbf{Sai}.\\
		Xác suất có điều kiện $P(A \mid \overline{B}) =0{,}35\%=0{,}0035$.
		\itemch \textbf{Sai}.\\
		Tỷ lệ mắc bệnh chung của cả làng là $$P(A)=P(B)\cdot P(A\mid B)+P(\overline{B})\cdot P(A\mid \overline{B})=\dfrac{12}{25}\cdot 0{,}6\%+\dfrac{13}{25}\cdot 0{,}35\%=0{,}47\%.$$
		\itemch \textbf{Sai}.\\
		Ta có $P(\overline{B}\mid\overline{A})=\dfrac{P(\overline{B}\cap\overline{A})}{P(\overline{A})}\approx 52\%$.
	\end{itemchoice}
	}
	
\end{ex}

\begin{ex}%[Nguồn: Bộ đề minh họa Moon 2024-2025]%[2D6V2-2]
	Một nhà máy sàn xuất bóng đèn có tỉ lệ bóng đèn đạt tiêu chuẩn là $82\%$. Trước khi xuất ra thị trường, mỗi bóng đèn được sản xuất ra đều phải qua một khâu kiểm tra chất lượng tự động. Vì sự kiểm tra này không chính xác tuyệt đối nên một bóng đèn tốt chi có xác suất $92\%$ được công nhận, và một bóng đèn hỏng có xác suất $96\%$ được loại bỏ.
	Gọi $A$ là biến cố \lq\lq bóng được công nhận đạt tiêu chuẩn sau khi qua kiểm tra chất lượng\rq\rq.\\
	Gọi $B$ là biến cố \lq\lq Sản phầm đạt tiêu chuẩn\rq\rq.
	\choiceTF
	{$\mathrm{P}(B)=0{,}18$; $\mathrm{P}(\overline{B})=0{,}82$}
	{Xác suất có điều kiện $\mathrm{P}(A\mid \overline{B})=0{,}92$}
	{\True Ti lệ bóng được công nhận đạt tiêu chưẫn sau khi qua kiểm tra chất lượng là $76{,}16\%$}
	{Tỷ lệ bóng đèn tốt trong số những bóng đèn được công nhận là $98,01\%$ (kết quả làm tròn đến hàng phần trăm)}
	\loigiai{
	\begin{itemchoice}
		\itemch Tỉ lệ bóng đèn đạt tiêu chuẩn (bóng tốt) là
		\[\mathrm{P}(B) = 82\% = 0{,}82,~\mathrm{P}(\overline{B}) = 1 - \mathrm{P}(B) = 18\% = 0{,}18.\]
		\itemch Xác suất kiểm tra đúng:
		\begin{itemize}
			\item Bóng tốt được công nhận: $\mathrm{P}(A \mid B) = 92\% = 0{,}92$.
			\item Bóng hỏng bị loại: $\mathrm{P}(\overline{A} \mid \overline{B}) = 96\% = 0{,}96$, do đó
			\[\mathrm{P}(A \mid \overline{B}) = 1 - \mathrm{P}(\overline{A} \mid \overline{B}) = 1 - 0{,}96 = 0{,}04.\]
			Ta có sơ đồ cây sau
			\begin{center}
				\tikzstyle{rect} = [rectangle, draw, rounded corners, text centered, minimum width=2.5cm, minimum height=1cm]
				\tikzstyle{arrow} = [thick, ->, >=stealth]
				\begin{tikzpicture}[node distance=2cm and 2.5cm] % Increase vertical spacing
					
					% Root node
					\node[rect,fill=red!50] (root) {Bóng đèn};
					
					% Level 1 nodes (adjusted heights)
					\node[rect,fill=purple!30, right=of root, yshift=3cm] (pass) {Đạt chuẩn};
					\node[rect,fill=red!30, right=of root, yshift=-3cm] (fail) {Không đạt chuẩn};
					
					% Level 2 nodes (adjusted positions for symmetry)
					\node[rect,fill=blue!30, above right=1cm and 1.5cm of pass] (pass-recog) {Được công nhận};
					\node[rect, fill=pink!30, below right=1cm and 1.5cm of pass] (pass-notrecog) {Được loại bỏ};
					
					\node[rect,fill=blue!30, above right=1cm and 1.5cm of fail] (fail-recog) {Được công nhận};
					\node[rect, fill=pink!30, below right=1cm and 1.5cm of fail] (fail-notrecog) {Được loại bỏ};
					
					% Draw arrows with labels
					\draw[arrow] (root) -- node[pos=0.3, above] {B} node[pos=0.5, below,yshift=-5pt] {$0{,}82$} (pass);
					\draw[arrow] (root) -- node[pos=0.3, above] {$\overline{B}$} node[pos=0.4, below,yshift=-5pt] {$0{,}18$} (fail);
					
					% Arrows for "Đạt chuẩn" subtree
					\draw[arrow] (pass) -- node[pos=0.5, left] {$A\mid B $} node[pos=0.7, below,yshift=-5pt] {$0{,}92$} (pass-recog);
					\draw[arrow] (pass) -- node[pos=0.5, above] {$\overline{A}\mid B $} node[pos=0.3, below] {} (pass-notrecog);
					
					% Arrows for "Không đạt chuẩn" subtree
					\draw[arrow] (fail) -- node[pos=0.5, left] {$A\mid \overline{B} $} node[pos=0.7, below,yshift=-5pt] {$0{,}04 $} (fail-recog);
					\draw[arrow] (fail) -- node[pos=0.5,above] {$\overline{A}\mid \overline{B}$} node[pos=0.7, left,yshift=-5pt] {$0{,}96 $} (fail-notrecog);
					
				\end{tikzpicture}
			\end{center}
		\end{itemize}
		\itemch  Sử dụng công thức xác suất toàn phần,ta có
		\allowdisplaybreaks
		\begin{eqnarray*}
			\mathrm{P}(A)& =& \mathrm{P}(A \mid B)\mathrm{P}(B) + \mathrm{P}(A \mid \overline{B})\mathrm{P}(\overline{B})
			\\
			&=&0{,}92\cdot 0{,}82 +0{,}04\cdot 0{,}18 = 0{,}7544 + 0{,}0072 = 0{,}7616.
		\end{eqnarray*}
		Tỉ lệ bóng đèn được công nhận đạt tiêu chuẩn là $76{,}16\%$.
		\itemch Xác suất có điều kiện $\mathrm{P}(B \mid A)$ (tỉ lệ bóng tốt trong số bóng được công nhận):\\
		Áp dụng công thức Bayes:
		\allowdisplaybreaks
		\begin{eqnarray*}
			\mathrm{P}(B \mid A)& =& \frac{\mathrm{P}(A \mid B)\mathrm{P}(B)}{\mathrm{P}(A)}
			\\
			&=&\dfrac{0{,}92\cdot 0{,}82}{0{,}7616} = \dfrac{943}{952} \approx 0{,}9905.
		\end{eqnarray*}
		Làm tròn đến hàng phần trăm, $\mathrm{P}(B \mid A) \approx 99{,}05\%$.
	\end{itemchoice}
	}
\end{ex}

%

\begin{ex}%[Nguồn: Bộ đề minh họa Moon 2024-2025]%[2D6V2-3]
	Ở một khu rừng nọ có $7$ chú lùn, trong đó có $5$ chú luôn nói thật, $2$ chú còn lại nói thật với xác suất $0{,}5$. Một nàng Bạch Tuyết lạc vào trong rừng và gặp một chú lùn.\\
	Gọi $A$ là biến cố: \lq\lq Chú lùn gặp được luôn nói thật\rq\rq.\\
	Gọi $B$ là biến cố: \lq\lq Chú lùn đó nhận mình là người luôn nói thật\rq\rq.
	\choiceTF
	{\True $\mathrm{P}(A)=\dfrac{5}{7}$; $\mathrm{P}(\overline{A})=\dfrac{2}{7}$}
	{Xác xuất có điều kiện $\mathrm{P}(B\mid A)=0{,}5$}
	{\True $\mathrm{P}(B)=\dfrac{6}{7}$}
	{\True Nàng Bạch Tuyết gặp ngẫu nhiên một chú lùn. Biết rằng chú lùn mà bạn Tuyết gặp tự nhận mình là người luôn nói thật. Xác suất để chú lùn đó luôn nói thật là $\dfrac{5}{6}$}
	\loigiai{
	\begin{itemchoice}
		\itemch Ta có $\mathrm{P}(A)=\dfrac{5}{7}\Rightarrow \mathrm{P}(\overline{A}) = \dfrac{2}{7}$
		\itemch Xác xuất có điều kiện $P(B\mid A)=1$.
		\itemch Theo công thức Bayes, ta có
		\[\mathrm{P}(B)=\mathrm{P}(A) \cdot \mathrm{P}(B|A)+\mathrm{P}(\overline{A})\cdot\mathrm{P}(B|\overline{A}) \Leftrightarrow \mathrm{P}(B)=1\cdot \dfrac{5}{7}+ 0{,}5\cdot \dfrac{2}{7}=\dfrac{6}{7}.\]
		\itemch Ta có
		\[\mathrm{P}(A|B)=\dfrac{\mathrm{P}(AB)}{\mathrm{P(B)}}=\dfrac{\mathrm{P}(A)\cdot \mathrm{P}(B|A)}{\mathrm{P}(B)}= \dfrac{\dfrac{5}{7}\cdot 1}{\dfrac{6}{7}}=\dfrac{5}{6}.\]
	\end{itemchoice}
	}
\end{ex}

\begin{ex}%[Nguồn: Bộ đề minh họa Moon 2024-2025]%[2D6V2-3]
	Nobita và Shizuka chuẩn bị đi tham quan hòn đảo Honshu trong hai ngày thứ Bảy và Chủ nhật tuần này. Ở hòn đảo Honshu này, mỗi ngày chi có nắng hoặc mưa, nếu một ngày là nắng thì khả năng xảy ra mưa ở ngày ngày tiếp theo là $20\%$, còn nếu một ngày là mưa thì khả năng ngày hôm sau vẫn mưa là $30\%$. Theo dự báo thời tiết, xác suất trời sẽ nắng vào thứ Bảy tuần này là $0{,}7$. Gọi $A$ là biến cố \lq\lq Ngày thứ Bảy tuần này trời nắng\rq\rq\, và $B$ là biến cố \lq\lq Ngày Chủ nhật tuần này trời mưa\rq\rq\,.
	\choiceTF
	{\True $P(A)=0{,}7$}
	{Xác suất có điều kiện $P(\overline{B} \mid A)=0{,}77$}
	{Xác suất ngày chủ nhật tuần này trời nắng là $80\%$}
	{Bạn mèo máy Doraemon có thể đến được tương lai nhưng lại chỉ đến hòn đảo vào ngày Chủ nhật và báo cho Nobita biết rằng Chủ nhật tuần này trời mưa, khi đó xác suất ngày thứ 7 trời nắng là $62\%$ (làm tròn đến hàng đơn vị theo đơn vị phần trăm)}
	\loigiai{
	\begin{center}
		\begin{tikzpicture}
			\def\gocm{20}
			\def\gocn{10}
			\def\r{4}
			\tikzset{s/.style={outer sep=0.5 mm,draw=magenta,rectangle,minimum width=2.75cm,rounded corners=1mm}}
			\path(0,0)node(O){}++(\gocm:\r)node[s](A1){Nắng (A)}++(\gocn:\r)node[s](A2){Mưa (B)};
			\path(A1)++({-\gocn}:\r)node[s](a2){Nắng $(\overline{B})$};
			\path(O)++(-\gocm:\r)node[s](B1){Mưa $(\overline{A})$}++(\gocn:\r)node[s](B2){Mưa (B)};
			\path(B1)++({-\gocn}:\r)node[s](b2){Nắng $(\overline{B})$};
			\foreach \x/\y in {
			O/A1,A1/A2,
			O/B1,B1/B2,
			A1/a2,
			B1/b2}
			\draw[-stealth](\x.east)--(\y.west);
			\path(O)--(A1.west)node[pos=0.5,above,sloped]{$\mbox{0{,}7}$}(O)--(B1.west)node[pos=0.5,below]{$\mbox{0{,}3}$}(B1.east)--(B2.west)node[pos=0.5,above]{$\mbox{0{,}3}$}(A1.east)--(A2.west)node[pos=0.5,above]{$\mbox{0{,}2}$}
			(A1.east)--(a2.west)node[pos=0.5,below,sloped]{$\mbox{0{,}8}$}
			(B1.east)--(b2.west)node[pos=0.5,below,sloped]{$\mbox{0{,}7}$};
			%%Node dòng trên
			\path(A2)++(0,1)node{\textbf{Chủ nhật}}++(180:4)node{\textbf{Thứ bảy}};
		\end{tikzpicture}
	\end{center}
	Ta có sơ đồ hình cây như hình vẽ.\\
	Ta có $A$ là biến cố \lq\lq Ngày thứ Bảy tuần này trời nắng\rq\rq\, và $B$ là biến cố \lq\lq Ngày Chủ nhật tuần này trời mưa\lq\lq.
	\begin{itemchoice}
		\itemch Theo giả thiết ta có $P(A)=0{,}7$.
		\itemch Ta có $ \mathrm{P}(\overline{B} \mid A)=\dfrac{ \mathrm{P}(\overline{B} \cap A)}{ \mathrm{P}(A)}=\dfrac{ \mathrm{P}(\overline{B})\cdot  \mathrm{P}(A \mid \overline{B})}{ \mathrm{P}(A)}=0{,}8$.
		\itemch Theo công thức xác suất toàn phần ta có
		\begin{eqnarray*}
			\mathrm{P}(\overline{B}) &=& \mathrm{P}(A) \cdot \mathrm{P}(\overline{B}|A) +\mathrm{P}\left(\overline{A}\right) \cdot  \mathrm{P}\left(\overline{B}|\overline{A}\right)\\
			&=& 0{,}7 \cdot 0{,}8+ 0{,}3\cdot 0{,}7= 0{,}77.
		\end{eqnarray*}
		\itemch Ta có $ \mathrm{P}(A\mid B)=\dfrac{ \mathrm{P}(A \cap B)}{ \mathrm{P}(B)}=\dfrac{ \mathrm{P}(A)\cdot  \mathrm{P}(B \mid A)}{ \mathrm{P}(B)}$.\\
		$ \mathrm{P}(B)=\mathrm{P}(A \cap B)+\mathrm{P}(\overline{A} \cap B)=\mathrm{P}(A)\cdot\mathrm{P}(B\mid A)+\mathrm{P}(\overline{A})\cdot\mathrm{P}(B\mid \overline{A})=0{,}23$.\\
		Do đó $ \mathrm{P}(A\mid B)=\dfrac{ \mathrm{P}(A \cap B)}{ \mathrm{P}(B)}=\dfrac{ \mathrm{P}(A)\cdot  \mathrm{P}(B \mid A)}{ \mathrm{P}(B)}=\dfrac{0{,}7 \cdot 0{,}2}{0{,}23}=0{,}6087\approx 61\%$.
	\end{itemchoice}
	}
\end{ex}

%

\begin{ex}%[Nguồn: Bộ đề minh họa Moon 2024-2025]%[2D6V2-3]
	Một tờ tiền giả lần lượt bị hai người $A$ và $B$ kiểm tra. Xác suất để người $A$ phát hiện ra tờ này giả là $0{,}7$. Nếu người $A$ cho rằng tờ này tiền giả, thì xác suất để người $B$ cũng nhận định như thế là $0{,}8$. Ngược lại, nếu người $A$ cho rằng tờ này là tiền thật thì xác suất để người $B$ cũng nhận định như thế là $0{,}4$.%[2D6V2-3]
	\choiceTF
	{Xác suất để $A$ không phát hiện ra tờ tiền đó giả là $0{,}2$}
	{\True Xác suất để hai người này đều không phát hiện đây là tờ tiền giả là $0{,}12$}
	{\True Xác suất để ít nhất một trong hai người này phát hiện ra tờ tiền đó là giả là $0{,}88$}
	{\True Biết tờ tiền đó đã bị ít nhất một trong hai người này phát hiện là giả, xác suất để $A$ phát hiện ra nó giả là $79{,}5\%$ (làm tròn đến hàng phần chục)}
	\loigiai{
	\begin{itemchoice}
		\itemch {\bf Sai}.\\
		Gọi $A$ là biến cố \lq\lq người $A$ phát hiện tờ tiền này là giả\rq\rq.\\
		Vậy $\overline{A}$ là biến cố \lq\lq người $A$ không phát hiện tờ tiền này là giả\rq\rq.\\
		Gọi $B$ là biến cố \lq\lq người $B$ phát hiện tờ tiền này là giả\rq\rq.\\
		Vậy $\overline{B}$ là biến cố \lq\lq người $B$ không phát hiện tờ tiền này là giả\rq\rq.\\
		Theo đề bài $\mathrm{P}(A)=0{,}7 \Rightarrow \mathrm{P}\left(\overline{A}\right)=1-\mathrm{P}(A)=1-0{,}7=0{,}3$.
		\itemch {\bf Đúng}.\\
		Nếu người $A$ cho rằng tờ tiền này là thật, xác suất để người $B$ cũng nhận định như thế là $\mathrm{P}\left(B\mid\overline{A}\right)=0{,}4$.\\
		Xác suất để cả hai người đều không phát hiện ra tờ tiền giả:
		$$\mathrm{P}\left(\overline{A} \cap\overline{B}\right)=\mathrm{P}(\overline{A}) \cdot \mathrm{P}(B \mid\overline{A})=0{,}3 \cdot 0{,}4=0{,}12.$$
		\itemch {\bf Đúng}.\\
		Gọi $C$ là biến cố \lq\lq Ít nhất một trong hai người này phát hiện ra tờ tiền đó là giả\rq\rq.\\
		Xác suất để ít nhất một trong hai người này phát hiện ra tờ tiền đó là giả là
		$$\mathrm{P}(C)=1-\mathrm{P}\left(\overline{A} \cap\overline{B}\right)=1-0{,}12=0{,}88.$$
		\itemch {\bf Đúng}.\\
		Ta có
		$\mathrm{P}(A \mid C)=\dfrac{\mathrm{P}(C\mid A)\cdot \mathrm{P}(A)}{\mathrm{P}(C)}=\dfrac{1\cdot 0{,}7}{0{,}88}=79{,}5\%$.
	\end{itemchoice}
	
	}
\end{ex}

%

\begin{ex}%[Nguồn: Bộ đề minh họa Moon 2024-2025]%[2D6V2-3]
	Một xạ thủ bắn hai viên đạn vào một bia. Xác suất bắn trúng viên thứ nhất là $0{,}7$. Nếu bắn trúng viên thứ nhất thì khả năng bắn trúng viên thứ hai là $0{,}8$ nhưng nếu bắn trượt viên thứ nhất thì khả năng bắn trúng viên thứ hai chỉ còn $0{,}3$. Biết rằng viên thứ hai xạ thủ bắn trúng, xác suất xạ thủ bắn trúng viên thứ nhất là bao nhiêu $\%$? (làm tròn kết quả đến hàng phần chục).
	\shortans[1]{$86{,}2$}
	\loigiai{
	Gọi $A$ là biến cố: \lq\lq xạ thủ bắn trúng viên thứ nhất\rq\rq.\\
	Gọi $B$ là biến cố: \lq\lq xạ thủ bắn trúng viên thứ hai\rq\rq.\\
	Ta có $\mathrm{P}(A)=0{,}7$; $\mathrm{P}(\overline{A})=0{,}3$;
	$\mathrm{P}(B\mid A)=0{,}8$; $\mathrm{P}(B\mid \overline{A})=0{,}3$; \\
	Theo công thức Bayes
	\[
	\mathrm{P}(A\mid B)=\dfrac{P(AB)}{P(B)}.
	\]
	Với $\mathrm{P}(AB)=\mathrm{P}(A)\cdot \mathrm{P}(B\mid A)=0{,}7\cdot 0{,}8=0{,}56$.\\
	Đồng thời $\mathrm{P}(B)=\mathrm{P}(AB)+\mathrm{P}(\overline{A}B)$\\
	Với $\mathrm{P}(AB)=0{,}56$ và $\mathrm{P}(\overline{A}B)=\mathrm{P}(\overline{A})\cdot \mathrm{P}(B\mid \overline{A})=0{,}3\cdot 0{,}3=0{,}09$.\\
	Suy ra $\mathrm{P}(B)=0{,}56+0{,}09=0{,}65$.\\
	Vậy
	\[
	\mathrm{P}(A\mid B)=\dfrac{0{,}56}{0{,}65}\approx 0{,}8615 = 86{,} 2\%.
	\]
	}
\end{ex}

\begin{ex}%[Nguồn: Bộ đề minh họa Moon 2024-2025]%[2D6V2-3]
	Ở vùng $A$ có hai nhóm, nhóm $1$ là nhóm người có thu nhập tốt (trên $15$ triệu đồng/tháng) và nhóm $2$ là nhóm có thu nhập không tốt. Ở vùng $A$ có $40\%$ người có thu nhập tốt và $58\%$ người không gửi tiết kiệm. Khảo sát độc lập những người thuộc nhóm $1$ và nhóm $2$ và tính tỉ lệ phần trăm số người gửi tiết kiệm của từng nhóm thì thấy rằng: Tỉ lệ người gửi tiết kiệm của nhóm $1$ gấp đôi tỉ lệ người tiết kiệm của nhóm $2$. Giả sử một người ở vùng $A$ không gửi tiết kiệm. Xác suất để người ấy có thu nhập tốt là bao nhiêu $\%$? (kết quả làm tròn đến hàng phần chục).
	\par\shortans{28}
	\loigiai{
	Gọi $A$ là biến cố \lq\lq Thu nhập tốt\rq\rq.\\
	Gọi $B$ là biến cố \lq\lq Không gửi tiết kiệm\rq\rq.\\
	Gọi $x$ là tỉ lệ người gửi tiết kiệm ở nhóm $1$, $y$ là tỷ lệ người gửi tiết kiệm ở nhóm $2$.\\
	Theo đề bài ta có $x=2y\quad (1)$.\\
	Gọi $N$ là tổng số người ở vùng $A$, thì số người không gửi tiết kiệm ở nhóm $1$ là $(1-x)\cdot 0{,}4 N$.\\
	Số người không gửi tiết kiệm ở nhóm $2$ là $(1-y)\cdot 0{,}6 N$.\\
	Vì tổng số người không gửi tiết kiệm ở vùng $A$ là $0{,}58N$ nên ta có
	$$(1-x)\cdot 0{,}4 N + (1-y)\cdot 0{,}6 N = 0{,}58N \Leftrightarrow (1-x)\cdot 0{,}4  + (1-y)\cdot 0{,}6  = 0{,}58.\quad(2)$$
	Giải $(1)$ và $(2)$ ta có $x=0{,}6$; $y=0{,}3$.\\
	Theo đề bài ta có
	$P(A)=0{,}4$ và $P(B)=0{,}58$.\\
	Xác suất người không gửi tiết kiệm, biết người đó thu nhập tốt là
	$\mathrm{P}(B\mid A)= 1-0{,}6=0{,}4$.\\
	Vậy xác suất người có thu nhập tốt khi biết người đó không gửi tiết kiệm là
	$$P(A\mid B)=\dfrac{\mathrm{P}(A)\cdot \mathrm{P}(B\mid A)}{\mathrm{P}(B)}=\dfrac{0{,}4\cdot 0{,}4}{0{,}58}=28\%.$$
	}
\end{ex}

%Điền đáp án 3

\begin{bt}%[Nguồn: Bộ đề minh họa Moon 2024-2025]%[2D6V2-3]
	Vắc xin AstraZeneca (AZD1222) được Tổ chức Y tế Thế giới (WHO) cấp phép sử dụng khẩn cấp giúp ngăn ngừa các triệu chứng nghiêm trọng và giảm tử vong do COVID-19. Vắc xin này được tiêm ở tỉnh X, thống kê cho thấy rằng: Với người có bệnh nền thì xác suất xảy ra phản ứng phụ sau tiêm là $28\%$, với người không có bệnh nền thì xác suất xảy ra phản ứng phụ sau tiêm là $17\%$. Chọn ngẫu nhiên một người được tiêm và thấy người này có phản ứng phụ. Tính xác suất để người này bị bệnh nền. Biết tỷ lệ người có bệnh nền ở tỉnh X là $12\%$. (làm tròn kết quả đến hàng phần trăm).
	\par\shortans{$0{,}18$}
	\loigiai{
	Gọi
	\begin{itemize}
		\item $B$ là biến cố \lq\lq Người được chọn có bệnh nền\rq\rq.
		\item $A$ là biến cố \lq\lq Người được chọn có phản ứng phụ\rq\rq.
	\end{itemize}
	Ta có sơ đồ cây
	\begin{center}
		\begin{tikzpicture}[font=\footnotesize, line join=round, line cap=round, >=stealth, scale=1]
			\draw[->] (0,0)--++(10:3) coordinate (B) node[sloped, above, pos=0.5]{$0{,}12$}node[sloped, right]{$B$};
			\draw[->] (B)+(0:0.5)--+(0:3) coordinate (A1) node[sloped, above=-1mm, pos=0.5]{$0{,}72$} node[sloped, right]{$\overline{A}$};
			\draw[->] (B)+(0:0.5)--+(20:3) coordinate (A2) node[sloped, above, pos=0.5]{$0{,}28$} node[sloped, right]{$A$};
			\draw[->] (0,0)--++(-10:3) coordinate (B') node[sloped, below, pos=0.5]{$0{,}88$}node[sloped, right]{$\overline{B}$};
			\draw[->] (B')+(0:0.5)--+(0:3) coordinate (A3) node[sloped, above=-1mm, pos=0.5]{$0{,}17$} node[sloped, right]{$A$};
			\draw[->] (B')+(0:0.5)--+(-20:3) coordinate (A4) node[sloped, below, pos=0.5]{$0{,}83$} node[sloped, right]{$\overline{A}$};
		\end{tikzpicture}
	\end{center}
	Biến cố \lq\lq Chọn một người bị bệnh nền biết người này có phản ưng phụ\rq\rq\, là $B\mid A$. Áp dụng công thức Bayes ta có
	$$ \mathrm{P}(B\mid A)
	=\dfrac{\mathrm{P}(B) \cdot \mathrm{P}(A\mid B)}{\mathrm{P}(B) \cdot \mathrm{P}(A\mid B) + \mathrm{P}(\overline{B}) \cdot \mathrm{P}(A\mid \overline{B})}
	=\dfrac{0{,}12\cdot0{,}28}{0{,}12\cdot0{,}28 + 0{,}88\cdot0{,}17}
	=\dfrac{42}{229}
	\approx 0{,}18.
	$$
	}
\end{bt}

\begin{ex}%[Nguồn: Bộ đề minh họa Moon 2024-2025]%[2D6V2-3]
	Một hộp chứa $10$ viên bi xanh và $5$ viên bi đỏ. Bạn An lấy ra ngẫu nhiên $1$ viên bi từ hộp, xem màu, rồi bỏ ra ngoài. Nếu viên bi An lấy ra có màu xanh, bạn Bình sẽ lấy ra ngẫu nhiên $2$ viên bi từ hộp; còn nếu viên bi An lấy ra có màu đỏ, bạn Bình sẽ lấy ra ngẫu nhiên $3$ viên bi từ hộp. \\
	Tính xác suất để An lấy được viên bi màu xanh, biết rằng tất cả các viên bi được hai bạn chọn ra đều có đủ cả hai màu. (Kết quả làm tròn đến hàng phần trăm).\\
	\shortans[oly]{0{,}70}
	\loigiai{
	Gọi $A$ là biến cố: \lq\lq An lấy được viên bi màu xanh\rq\rq. Khi đó $\overline{A}$ là biến cố: \lq\lq An lấy được viên bi màu đỏ\rq\rq.\\
	Gọi $B$ là biến cố: \lq\lq Bình lấy được các viên bi cùng màu với viên bi của An\rq\rq. Khi đó $\overline{B}$ là biến cố: \lq\lq Bình lấy được các viên bi có chứa viên bi có màu khác với viên bi của An\rq\rq.\\
	Suy ra $\mathrm{P}(A) = \dfrac{10}{15} = \dfrac{2}{3}$, nên $\mathrm{P}(\overline{A}) = \dfrac{1}{3}$.\\
	Yêu cầu bài toán tương đương với việc tính $\mathrm{P}(A|\overline{B})$.\\
	Áp dụng công thức Bayes, ta có $\mathrm{P}(A|\overline{B}) = \dfrac{\mathrm{P}(A) \mathrm{P}(\overline{B}|A)}{\mathrm{P}(\overline{B})}$.\\
	Khi An lấy được viên bi màu xanh, thì số cách lấy $2$ viên bi của Bình sao cho tất cả các viên bi được hai ban chọn có đúng cách hai màu là $C_9^1 \cdot C_5^1 + C_5^2 = 55$ cách.\\
	Suy ra $\mathrm{P}(\overline{B}|A) = \dfrac{55}{C_{14}^2} = \dfrac{55}{91}$.\\
	\textbf{Trường hợp} 1: An lấy được bóng xanh, thì cách lấy $2$ viên bi của Bình sao cho $2$ viên bi này cùng màu của An là $C_9^2$ cách.\\
	\textbf{Trường hợp} 2: An lấy được bóng đỏ, thì cách lấy $3$ viên bi của Bình sao cho $3$ viên bi này cùng màu của An là $C_5^3$ cách.\\
	Do đó  $\mathrm{P}(B) = \dfrac{C_9^2}{C_{14}^2} + \dfrac{C_5^3}{C_{14}^3} = \dfrac{11}{26}$, suy ra $\mathrm{P}(\overline{B}) = 1 - \dfrac{11}{26} = \dfrac{15}{26}$.\\
	Suy ra $\mathrm{P}(A|\overline{B}) = \dfrac{\mathrm{P}(A) \mathrm{P}(\overline{B}|A)}{\mathrm{P}(\overline{B})} = \dfrac{\dfrac{2}{3} \cdot \dfrac{55}{91}}{\dfrac{15}{26}} \approx 0{,}70$.
	}
\end{ex}

\begin{ex}%[Nguồn: Bộ đề minh họa Moon 2024-2025]%[2D6V2-4]
	Tỉ lệ học sinh tiêm vắc xin phòng bệnh Thủy Đậu trong một trường $M$ là $70\%$. Trong số những học sinh đã được tiêm phòng, tỉ lệ mắc Thủy Đậu là $4\%$, còn trong số học sinh chưa tiêm, tỉ lệ mắc bệnh là $20\%$. Gặp ngẫu nhiên một học sinh ở trường đó.
	\begin{itemize}
		\item Gọi $A$ là biến cố: \lq\lq Gặp học sinh mắc bệnh thủy đậu\rq\rq.
		\item Gọi $B$ là biến cố: \lq\lq Gặp học sinh đã tiêm vắc xin phòng bệnh Thủy Đậu \rq\rq.
	\end{itemize}
	\choiceTF
	{$P(B) = 0{,}3$, $P\left( \overline{B}\right) = 0{,}7$}
	{\True Xác suất có điều kiện $P \left(A | \overline{B} \right) = 0{,}2$}
	{Xác suất gặp học sinh bị bệnh thủy đậu là $7{,}8\%$}
	{\True Biết học sinh đó bị bệnh Thủy Đậu. Xác suất học sinh đó không tiêm vắc xin phòng bệnh Thủy Đậu là $0{,}68$ (kết quả làm tròn đến hàng phần trăm)}
	\loigiai{
	\begin{itemchoice}
		\itemch \textbf{Sai}. Theo bài ra ta có $P\left(B \right) = 0{,}7$ và $P \left(\overline{B} \right) = 0{,}3$.
		\itemch \textbf{Đúng}. Xác suất học sinh chưa tiêm phòng bị mắc bệnh Thuỷ Đậu là $P \left(A| \overline{B} \right) = 0{,}2$.
		\itemch \textbf{Sai}. Xác suất học sinh đã được tiêm phòng mắc Thuỷ Đậu là $P \left(A \big| B \right) = 0{,}04$.\\
		Xác suất gặp học sinh bị Thuỷ Đậu là $P \left(A \right) = P\left(B \right) \cdot P \left(A| B \right) + P \left(\overline{B} \right) \cdot P \left(A| \overline{B} \right) = 0{,}088 = 8{,}8\%$.
		\itemch \textbf{Đúng}. Xác suất để học sinh bị bệnh Thủy Đậu đó không tiêm vắc xin phòng bệnh Thủy Đậu là
		\begin{align*}
			P \left(\overline{B} \big| A \right) = \dfrac{P \left(\overline{B} \right) \cdot P \left(A \big| \overline{B} \right)}{P \left(A \right)} = \dfrac{0{,}3 \cdot 0{,}2}{0{,}088} \approx 0{,}68.
		\end{align*}
	\end{itemchoice}
	}
\end{ex}

\begin{ex}%[Nguồn: Bộ đề minh họa Moon 2024-2025]%[2D6V2-4]
	Một cặp trẻ sinh đôi có thể do cùng một trứng (sinh đôi thật) hay do hai trứng khác nhau sinh ra (sinh đôi giả). Các cặp sinh đôi thật luôn có cùng giới tính. Cặp sinh đôi giả thì giới tính của mỗi đứa độc lập với nhau và có xác suất $40{,}5$ là con trai. Thống kê cho thấy $34\%$ cặp sinh đôi đều là trai, $30\%$ cặp sinh đôi đều là gái và $36\%$ cặp sinh đôi có giới tính khác nhau. Chọn ngẫu nhiên một cặp sinh đôi.\\
	Gọi $A$ là biến cố \lq\lq Cặp sinh đôi thật\rq\rq, $B$ là biến cố \lq\lq Cặp sinh đôi có cùng giới tính\rq\rq.
	\choiceTF
	{\True $\mathrm{P}(B)=64\%$}
	{\True Xác suất có điều kiện $\mathrm{P}(B\mid A)=1$}
	{\True Tỷ lệ sinh đôi thật là $28\%$}
	{\True Tỷ lệ cặp sinh đôi thật trong số cặp sinh đôi cùng giới tính là $43{,}75\%$}
	\loigiai{
	\begin{itemchoice}
		\itemch
		Xác suất của biến cố $B$ là
		\[\mathrm{P}(B)=34\%+30\%=64\%.\]
		\itemch
		Xác suất có điều kiện $\mathrm{P}(B\mid A)$ là xác suất của biến cố \lq\lq Cặp sinh đôi có cùng giới tính\rq\rq\,với điều kiện \lq\lq Cặp sinh đôi thật\rq\rq\,có xác suất là $100\%$.
		\itemch
		Ta có xác suất $\mathrm{P}\left( B\mid\overline{A}\right)=\dfrac{1}{2}$.\\
		Theo công thức xác suất toàn phần ta có
		\begin{eqnarray*}
			&&\mathrm{P}(B)=P(A)\cdot \mathrm{P}(B\mid A)+\mathrm{P}(\overline A)\cdot \mathrm{P}(B\mid A)\\
			&\Leftrightarrow& 0{,}64 = \mathrm{P}(A)\cdot 1+\left( 1-\mathrm{P}(A)\right) \cdot \dfrac{1}{2}\\
			&\Leftrightarrow& 0{,}64 = \mathrm{P}(A)-\dfrac{1}{2}\mathrm{P}(A)+\dfrac{1}{2}\\
			&\Leftrightarrow& \mathrm{P}(A)=0{,}28.
		\end{eqnarray*}
		Tỷ lệ sinh đôi thật là $28\%$.
		\itemch
		Xác suất để chọn được cặp sinh đôi thật biết rằng cặp sinh đôi đó cùng giới tính là $\mathrm{P}(A\mid B)$.\\
		Theo công thức nhân xác suất, ta có $\mathrm{P}(AB) = \mathrm{P}(A)\cdot \mathrm{P}(B \mid A)$.\\
		Ta có, $\mathrm{P}(A)= 0{,}28$ và $\mathrm{P}(B\mid A)=1$.\\
		Do đó, $\mathrm{P}(AB)= \mathrm{P}(A)\cdot \mathrm{P}(B\mid A)=0{,}28$.\\
		Lại có $\mathrm{P}(B)= 0{,}34 + 0{,}3 =0{,}64$.\\
		Như vậy, $\mathrm{P}(A \mid B) = \dfrac{\mathrm{P}(AB)}{\mathrm{P}(B)}=\dfrac{0{,}28}{0{,}64}=0{,}4375$.
	\end{itemchoice}
	}
\end{ex}

%

\begin{ex}%[Nguồn: Bộ đề minh họa Moon 2024-2025]%[2D6V2-4]
	Hai công nhân cần phải hoàn thành số sản phẩm nhất định. Công nhân thứ nhất phải làm $45\%$ số sản phẩm, công nhân thứ hai phải làm $55\%$ số sản phẩm. Khả năng xảy ra sai sót của công nhân thứ nhất là $3\%$ và của công nhân thứ hai là $1\%$. Chọn ngẫu nhiên 1 sản phẩm.\\
	Gọi $A$ là biến cố \lq\lq Sản phẩm được chọn là của công nhân thứ nhất\rq\rq.\\
	Gọi $B$ là biến cố \lq\lq Sản phẩm được chọn bị lỗi\rq\rq.
	\choiceTF
	{$P(A)=0{,}5$}
	{\True Xác suất có điều kiện $P(B\mid A)=0{,}03$}
	{$P(B)=0{,}02$}
	{Biết sản phẩm chọn được bị lỗi, xác suất sản phẩm được chọn là của công nhân thứ nhất bằng $\dfrac{27}{38}$}
	\loigiai{
	\begin{itemchoice}
		\itemch \textbf{Sai.} \\
		Vì xác suất sản phẩm được chọn là của công nhân thứ nhất là $P(A)=\dfrac{45}{100}=0{,}45$.
		\itemch \textbf{Đúng.} \\
		Vì công nhân thứ nhất có tỉ lệ sai sót là $3\%$ hay $P(B\mid A)=0{,}03$.
		\itemch \textbf{Sai.} \\
		Ta có $P(A)=0{,}45$; $P(\overline{A})=1-0{,}45=0{,}65$;
		$P(B\mid A)=0{,}03$; $P(B\mid \overline{A})=0{,}01$.\\
		Áp dụng công thức xác suất toàn phần\\
		$P(B)=P(B\mid A)\cdot P(A)+P(B\mid \overline{A})\cdot P(\overline{A})=0{,}03\cdot 0{,}45+0{,}01\cdot 0{,}55=0{,}019$.
		\itemch \textbf{Sai.} \\
		Áp dụng công thức Bayes, ta có\\
		$P(A\mid B)=\dfrac{P(B\mid A)\cdot P(A)}{P(B)}=\dfrac{0{,}03\cdot 0{,}55}{0{,}019}=\dfrac{33}{38}$.
	\end{itemchoice}
	}
\end{ex}

\begin{ex}%[Nguồn: Bộ đề minh họa Moon 2024-2025]%[2D6V2-4]
	Tại một phòng khám chuyên khoa tỷ lệ người đến khám có bệnh là $0{,}8$. Người ta áp dụng phương pháp chẩn đoán mới thì thấy nếu khẳng định có bệnh thì đúng $9$ trên $10$ trường hợp; còn nếu khẳng định không bệnh thì đúng $5$ trên $10$ trường hợp. Hỏi xác suất chuẩn đoán đúng là bao nhiêu $\%$? Biết rằng chuẩn đoán đúng là khi người bị bệnh được chuẩn đoán có bệnh hoặc người không bị bệnh được chuẩn đoán không bị bệnh.
	\shortans{$82$}
	\loigiai{
	Gọi $H_1$ là biến cố \lq\lq Người đến khám có bệnh\rq\rq.\\
	Suy ra $P \left(H_1 \right) = 0{,}8$.\\
	$H_2$ là biến cố \lq\lq Người đến khám không có bệnh\rq\rq.\\
	Suy ra $P\left(H_2 \right) = 0{,}2$.\\
	$A$ là biến cố \lq\lq Chuẩn đoán đúng\rq\rq.\\
	Theo đề bài ta có
	\begin{align*}
		P \left(A \big| H_1 \right) = 0{,}9 \text{ và } P \left(A \big| H_2 \right) = 0{,}5.
	\end{align*}
	Theo công thức xác suất đầy đủ ta có
	\begin{align*}
		P(A) = P\left(H_1 \right) \cdot P \left(A \big| H_1 \right) + P\left(H_2 \right) \cdot P \left(A \big| H_2 \right) = 0{,}8 \cdot 0{,}9 + 0{,}2 \cdot 0{,}5 = 0{,}82.
	\end{align*}
	Vậy xác suất chuẩn đoán đúng của phòng khám là $82\%$.
	}
\end{ex}

%

\begin{ex}%[Nguồn: Bộ đề minh họa Moon 2024-2025]%[2D6V2-4]
	Hộp thứ nhất có $3$ viên bi xanh và $6$ viên vi đỏ. Hộp thứ hai có $3$ viên vi xanh và $7$ viên bi đỏ. Các viên bi có cùng kích thước và khối lượng. Lấy ngẫu nhiên ra một viên bi từ hộp thứ nhất chuyển sang hộp thứ hai. Sau đó lại lấy ngẫu nhiên đồng thời hai viên từ hộp thứ hai, biết rằng hai bi lấy ra từ hộp thứ hai đều là bi màu đỏ, xác suất viên bi lấy ra từ hộp thứ nhất cũng là bi màu đỏ là bao nhiêu? (viết kết quả làm tròn đến hàng phần chục).
	\shortans{72,7}
	\loigiai{
	Gọi $A$ là biến cố \lq\lq Viên bi lấy từ hộp thứ nhất là bi đỏ\rq\rq,\\
	$B$ là biến cố \lq\lq Hai viên bi lấy từ hộp thứ hai đều là bi đỏ\rq\rq.\\
	Do hộp thứ nhất có tổng công $9$ viên bi, trong đó có $6$ bi đỏ nên $P(A)=\dfrac{6}{9}=\dfrac{2}{3}$.\\
	Vì có $3$ bi xanh trong hộp thứ nhất nên $P(\overline{A})=\dfrac{3}{9}=\dfrac{1}{3}$.\\
	Nếu viên bi lấy từ hộp thứ nhất là bi đỏ thì hộp thứ hai có $3$ bi xanh và $7+1=8$ bi đỏ.\\ Do đó khi lấy hai bi từ hộp thứ hai, xác suất để cả hai viên đều là bi đỏ \[P(B\mid A)=\dfrac{\mathrm{C}_8^2}{\mathrm{C}_{11}^2}=\dfrac{28}{55}.\]
	Nếu viên bi lấy từ hộp thứ nhất là bi xanh thì hộp thứ hai có $3+1=4$ bi xanh và $7$ bi đỏ.\\ Do đó khi lấy hai bi từ hộp thứ hai, xác suất để cả hai viên đều là bi đỏ \[P(B\mid \overline{A})=\dfrac{\mathrm{C}_7^2}{\mathrm{C}_{11}^2}=\dfrac{21}{55}.\]
	Áp dụng định lý xác suất toàn phần\\
	$P(B)=P(B\mid A)\cdot P(A)+P(B\mid \overline{A})\cdot P(\overline{A})=\dfrac{28}{55}\cdot \dfrac{2}{3}+\dfrac{21}{55}\cdot \dfrac{1}{3}=\dfrac{7}{15}$.\\
	Áp dụng công thức Bayes, ta có\\
	$P(A\mid B)=\dfrac{P(B\mid A)\cdot P(A)}{P(B)}=\dfrac{\dfrac{28}{55}\cdot \dfrac{2}{3}}{\dfrac{7}{15}}=\dfrac{8}{11}\approx 72{,}7$.
	}
\end{ex}

%

\begin{bt}%[Nguồn: Bộ đề minh họa Moon 2024-2025]	%[2H2C2-6]
	Hệ thống định vị toàn cầu GPS là một hệ thống cho phép xác định vị trí của một vật thể trong không gian. Trong cùng một thời điểm, vị trí của một điểm $M$ trong không gian sẽ được xác định bởi bốn vệ tinh cho trước nhờ các bộ thu phát tín hiệu đặt trên các vệ tinh. Giả sử trong không gian với hệ tọa độ $Oxyz$, có bốn vệ tinh lần lượt đặt tại các điểm $A(3; 1; 0)$, $B(3; 6; 6)$, $C(4; 6; 2)$, $D(6; 2; 14)$, vị trí $M(a; b; c)$ thỏa mãn $MA=3$, $MB=6$, $MC=5$, $MD=13$. Khoảng cách từ điểm $M$ đến điểm $O$ bằng bao nhiêu?
	\shortans{3}
	\loigiai{
	Ta có
	$$
	\begin{aligned}
		& M A=\sqrt{(a-3)^2+(b-1)^2+c^2}=3, \\
		& M B=\sqrt{(a-3)^2+(b-6)^2+(c-6)^2}=6, \\
		& M C=\sqrt{(a-4)^2+(b-6)^2+(c-2)^2}=5, \\
		& M D=\sqrt{(a-6)^2+(b-2)^2+(c-14)^2}=13.
	\end{aligned}
	$$
	
	Bình phương hai vế của các phương trình trên, ta được
	$$
	\begin{aligned}
		(a-3)^2+(b-1)^2+c^2	&=9\\
		(a-3)^2+(b-6)^2+(c-6)^2	&=36\\
		(a-4)^2+(b-6)^2+(c-2)^2	&=25\\
		(a-6)^2+(b-2)^2+(c-14)^2&=169
	\end{aligned}
	$$
	
	Trừ phương trình thứ hai cho phương trình thứ nhất, ta được
	$$
	(b-6)^2-(b-1)^2+(c-6)^2-c^2=27.
	$$
	
	Rút gọn, ta có $10b-35+12c-36=-27$ hay $5b+6c=22$.
	Tương tự, trừ phương trình thứ ba cho phương trình thứ nhất, ta được
	$$
	(a-4)^2-(a-3)^2+(b-6)^2-(b-1)^2+(c-2)^2-c^2=16.
	$$
	
	Rút gọn, ta có $2a-7+10b-35+4c-4=-16$, hay $a+5b+2c=15$.
	
	Cuối cùng, trừ phương trình thứ tư cho phương trình thứ nhất, ta được:
	$$
	(a-6)^2-(a-3)^2+(b-2)^2-(b-1)^2+(c-14)^2-c^2=160.
	$$
	
	Rút gọn, ta có $6a-27+2b-3+28c-196=-160$ hay $3a+b+14c=33$. Ta có hệ phương trình
	$$
	\left\{\begin{array} {r} {5 b+6 c=2 2,} \\
		{a+5 b+2 c=1 5,} \\
		{3 a+b+1 4 c=3 3.}
	\end{array} \Leftrightarrow \left\{\begin{array}{l}
		a=1 \\
		b=2 \\
		c=2.
	\end{array}\right.\right.
	$$
	
	Vậy điểm $M(1; 2; 2)$ và khoảng cách từ điểm $M$ đến điểm $O$ bằng $OM=\sqrt{1^2+2^2+2^2}=3$.
	}
\end{bt}

\begin{ex}%[Nguồn: Bộ đề minh họa Moon 2024-2025]%[2H2C2-6]
	Một ống phun nước có hình dạng như hình vẽ dưới. Để giữ cho ống nước được cân bằng không bị nghiêng kỹ sư sử dụng ba đoạn thép để nối các điểm $C$, $A$, $G$ với mặt đất, các đoạn thép $CD$, $GF$, $AE$ có độ lớn lực căng lần lượt bằng $1\,200$ N, $800$ N và $600$ N. Trong hệ tọa độ $Oxyz$, coi gốc tọa độ là chân ống nước, trục $Oz$ hướng lên trời, mặt đất là mặt phẳng $(Oxy)$ các thông số được cho như hình vẽ, đơn vị trên các hệ trục tọa độ tính bằng mét. Coi đường kính ống không đáng kề, độ lớn vectơ hợp lực của ba sợi thép tác động lên ông nước là bao nhiêu Niutơn (làm tròn kết quả đến hàng đơn vị).
	%\begin{center}
		%	\includegraphics[scale=0.5]{cau-5-TLN-De-5}
	%\end{center}
	
	\shortans{$2311$}
	\loigiai{
	Giả sử lực tác dụng lên $3$ đoạn dây $CD$, $GF$, $AE$ lần lượt là $\vec{T}_1$, $\vec{T}_2$, $\vec{T}_3$.\\
	Suy ra hợp lực $\vec{T}=\vec{T}_1+\vec{T}_2+\vec{T}_3$. $\qquad (*) $\\
	(Áp dụng công thức: Cho $2$ vectơ $\vec{u}$, $\vec{v}$ cùng hướng, ta có $\vec{u}=\dfrac{\left|\vec{u}\right|}{\left|\vec{v}\right|}\cdot \vec{v}) $.
	\begin{itemize}
		\item Vì $\vec{T}_1$, $\vec{CD}$ cùng hướng nên ta suy ra
		$\vec{T}_1=\dfrac{\left|\vec{T}_1\right|}{\left|\vec{CD}\right|}\cdot \vec{CD} $.
		\item Vì $\vec{T}_2$, $\vec{GF}$ cùng hướng nên ta suy ra $\vec{T}_2=\dfrac{\left|\vec{T}_2\right|}{\left|\vec{GF}\right|}\cdot \vec{GF}$.
		\item Vì $\vec{T}_3$, $\vec{AE}$ cùng hướng nên ta suy ra $\vec{T}_3=\dfrac{\left|\vec{T}_3\right|}{\left|\vec{AE}\right|}\cdot \vec{AE}$.
	\end{itemize}
	Suy ra $\vec{T}=\dfrac{T_1}{CD}\cdot \vec{CD}+\dfrac{T_2}{GF}\cdot \vec{GF}+\dfrac{T_3}{AE}\cdot \vec{AE}$.\\
	Ta có
	\begin{itemize}
		\item $ C(-1{,}5;0;4{,}5)$, $D(0;3;0) \Rightarrow\vec{CD}=(1{,}5;3;-4{,}5)\Rightarrow CD=1{,}5\sqrt{14}$.
		\item $ G(0;-1;3)$, $F(2;-1;0) \Rightarrow\vec{GF}=(2;0;-3)\Rightarrow GF=\sqrt{13}$.
		\item $ A(0;0;3)$, $ E(-1{,}5;0;0) \Rightarrow\vec{AE}=(-1{,}5;0;-3)\Rightarrow AE=1{,}5\sqrt{5}$.
	\end{itemize}
	Suy ra
	\allowdisplaybreaks
	\begin{eqnarray*}
		\vec{T}&=&\dfrac{1\,200}{1{,}5\sqrt{14}}(1{,}5;3;-4{,}5)+\dfrac{800}{\sqrt{13}}(2;0;-3)+\dfrac{600}{1{,}5\sqrt{5}}(-1{,}5;0;-3)
		\\
		&=&(496{,}145;641{,}427;-2164{,}437).
	\end{eqnarray*}
	Suy ra độ lớn của vectơ hợp lực là
	$\left|\vec{T}\right|=\sqrt{496{,}145^2+641{,}427^2+(-2164{,}437)^2}\approx 2311$ N.
	}
\end{ex}

%

\begin{ex}%[Nguồn: Bộ đề minh họa Moon 2024-2025]%[2H2H1-2]
	Cho hình hộp $ABCD. A' B' C' D'$ (minh họa như hình dưới). Phát biểu nào sau đây là đúng?
	\begin{center}
		\begin{tikzpicture}[line join=round, line cap=round,>=stealth,thick,scale=0.8]
			\path
			(0,0) coordinate (A)
			(-2,-2) coordinate (B)
			(4,0) coordinate (D)
			(0.8,3.2) coordinate (A')
			($(B)+(D)-(A)$) coordinate (C)
			($(B)+(A')-(A)$) coordinate (B')
			($(D)+(A')-(A)$) coordinate (D')
			($(D')+(B')-(A')$) coordinate (C')
			;
			\draw (B')--(B)--(C)--(D)--(D')--(A')--cycle
			(B')--(C')--(C) (D')--(C')
			;
			\draw[dashed] (B)--(A)--(D) (C')--(A)--(A');
			\foreach \x/\g in {A/160,B/180,C/-30,D/0,A'/90,B'/140,D'/90,C'/90} \draw[fill=white] (\x) circle (.03)+(\g:0.3)node{$\x$};
		\end{tikzpicture}
	\end{center}
	\choice
	{$\overrightarrow{AB}+\overrightarrow{BB'}+\overrightarrow{B' A'}=\overrightarrow{AC'}$}
	{$\overrightarrow{AB}+\overrightarrow{BC'}+\overline{C' D'}=\overline{AC'}$}
	{$\overrightarrow{AB}+\overrightarrow{AC}+\overrightarrow{AA'}=\overrightarrow{AC'}$}
	{\True $\overrightarrow{AB}+\overrightarrow{AA'}+\overrightarrow{AD}=\overrightarrow{AC'}$}
	\loigiai{
	Theo qui tắc hình hộp ta có $\overrightarrow{A B}+\overrightarrow{A A'} +\overrightarrow{A D} =\overrightarrow{A C'}$.
	}
\end{ex}

\begin{ex}%[Nguồn: Bộ đề minh họa Moon 2024-2025]%[2H2H1-2]
	Cho hình chóp $S.ABCD$ có đáy là hình bình hành (hình vẽ minh họa). Hãy chọn khẳng định đúng
	\begin{center}
		\begin{tikzpicture}[scale=0.8,>=stealth, font=\footnotesize, line join=round, line cap=round]
			\def\a{4}
			\path (0:0) coordinate (A)
			++(0:\a) coordinate (D)
			++(-130:\a/2) coordinate (C)
			($(A)+(C)-(D)$) coordinate (B)
			($(A)+(80:\a)$) coordinate (S)
			(intersection of A--C and B--D) coordinate (O);%giao điểm O
			\draw[dashed,thick] (B)--(A)--(D)(A)--(S);
			\draw[thick] (B)-- (C)--(D)
			(B)--(S)(C)--(S)(D)--(S);
			\foreach \x/\g in {A/135,B/-135,C/-45,D/45,S/90}
			\fill[black] (\x) circle (1pt)
			($(\g:3mm)+(\x)$) node {$\x$};
		\end{tikzpicture}
	\end{center}
	\choice
	{\True $\overrightarrow{SA} + \overrightarrow{SC} = \overrightarrow{SB} + \overrightarrow{SD}$}
	{$\overrightarrow{SA} + \overrightarrow{AB} = \overrightarrow{SD} + \overrightarrow{DC}$}
	{$\overrightarrow{SA} + \overrightarrow{AD} = \overrightarrow{SB} + \overrightarrow{BC}$}
	{$\overrightarrow{SA} + \overrightarrow{SB} = \overrightarrow{SC} + \overrightarrow{SD}$}
	\loigiai{
	\begin{center}
		\begin{tikzpicture}[scale=0.8,>=stealth, font=\footnotesize, line join=round, line cap=round]
			\def\a{4}
			\path (0:0) coordinate (A)
			++(0:\a) coordinate (D)
			++(-130:\a/2) coordinate (C)
			($(A)+(C)-(D)$) coordinate (B)
			($(A)+(80:\a)$) coordinate (S)
			(intersection of A--C and B--D) coordinate (O);%giao điểm O
			\draw[dashed,thick] (B)--(A)--(D)(A)--(S) (A)--(C) (B)--(D) (S)--(O);
			\draw[thick] (B)-- (C)--(D)
			(B)--(S)(C)--(S)(D)--(S);
			\foreach \x/\g in {A/135,B/-135,C/-45,D/45,S/90,O/-90}
			\fill[black] (\x) circle (1pt)
			($(\g:3mm)+(\x)$) node {$\x$};
		\end{tikzpicture}
	\end{center}
	Gọi $O$ là giao điểm của $AC$ và $BD$, vì $ABCD$ là hình bình hành nên $O$ là trung điểm của $AC$ và $BD$. Từ đó ta có \[\overrightarrow{SA} + \overrightarrow{SC} = 2\overrightarrow{SO}=\overrightarrow{SB} + \overrightarrow{SD}\]
	}
\end{ex}

\begin{ex}%[Nguồn: Bộ đề minh họa Moon 2024-2025]%[2H2H1-2]
	\immini{
	Cho hình lập phương $ABCD.A' B' C' D'$ có cạnh bằng $a$. Độ dài của vectơ \break$\vec{u}=\overrightarrow{A' C'}-\overrightarrow{A' A}$ bằng
	\choice
	{$\sqrt{2} a$}
	{$\dfrac{\sqrt{3} a}{2}$}
	{$\sqrt{6} a$}
	{\True $\sqrt{3} a$}
	}
	{
	\begin{tikzpicture}[declare function={r=3;}]
		\path (0:0) coordinate (A)
		(0:r) coordinate (B)
		++(37:{0.65*r}) coordinate (C)
		++(180:r) coordinate (D)
		\foreach \x in {A,B,C,D}{(\x)++(90:r) coordinate (\x')};
		\draw[dash pattern=on 2pt off 2 pt] (D')--(D)--(A) (D)--(C) (A)--(C');
		\draw (A)--(B)--(C) (A)--(A') (B)--(B') (C)--(C')(A')--(B')--(C')--(D')--cycle (A')--(C');
		\foreach \t/\g in {A/180,B/0,C/0,D/40,A'/180,B'/120,C'/0,D'/180}{
		\draw[fill=black] (\t) circle (1pt) node[shift={(\g:7pt)},font=\scriptsize]{$ \t $};
		}
	\end{tikzpicture}
	}
	\loigiai{
	Ta có	$\vec{u}=\overrightarrow{A' C'}-\overrightarrow{A' A}=\overrightarrow{AC'}$. Suy ra $\left|\overrightarrow{u}\right|=AC' =\sqrt{3} a$.
	}
\end{ex}

\begin{ex}%[Nguồn: Bộ đề minh họa Moon 2024-2025]%[2H2H1-2]
	\immini[thm]{Cho hình chóp $S.ABCD$ đáy là hình vuông cạnh $a$, $SA=a\sqrt{3}$. Đường thẳng $SA\perp (ABCD)$ (hình vẽ). Tổng $\vv{AB}+\vv{AD}+\vv{SA}$ bằng}{\begin{tikzpicture}[scale=.8, font=\footnotesize, line join=round, line cap=round, >=stealth]
		\def\bc{3} \def\ba{1.5} \def\h{3} \def\gocB{30}
		\coordinate[label=below left:$B$] (B) at (0,0);
		\coordinate[label=above left:$A$] (A) at (\gocB:\ba);
		\coordinate[label=below:$C$] (C) at (\bc,0);
		\coordinate[label=right:$D$] (D) at ($(C)-(B)+(A)$);
		\coordinate[label=above:$S$] (S) at ($(A)+(90:\h)$);
		\draw (B)--(C)--(D)--(S)--cycle (S)--(C);
		\draw[dashed] (A)--(D) (S)--(A)--(B);
		\foreach \diem in {A,B,C,D,S}
		\fill (\diem)circle(1pt);
		\draw (A)--++(90:0.2)--++(0:0.2)--++(-90:0.2);
	\end{tikzpicture}}
	\choice
	{\True $\vv{SC}$}
	{$\vv{SD}$}
	{$\vv{SB}$}
	{$\vv{BS}$}
	\loigiai{
	Ta có $\vv{AB}+\vv{AD}+\vv{SA}=\vv{AC}+\vv{SA}=\vv{SA}+\vv{AC}=\vv{SC}$.
	}
\end{ex}

%Câu trắc nghiệm 10

\begin{ex}%[Nguồn: Bộ đề minh họa Moon 2024-2025]%[2H2H1-2]
	\immini[thm]
	{Cho tứ diện $ABCD$ có $G$ là trọng tâm của tam giác $BCD$. Đặt $\overrightarrow{AB}=\overrightarrow{x}$, $\overrightarrow{AC}=\overrightarrow{y}$, $\overrightarrow{AD}=\overrightarrow{z}$. Phát biểu nào sau đây là đúng?
	\choice
	{\True $\overrightarrow{AG}=\dfrac{1}{3}(\overrightarrow{x}+\overrightarrow{y}+\overrightarrow{z})$}
	{$\overrightarrow{AG}=\dfrac{2}{3}(\overrightarrow{x}+\overrightarrow{y}+\overrightarrow{z})$}
	{$\overrightarrow{AG}=-\dfrac{1}{3}(\overrightarrow{x}+\overrightarrow{y}+\overrightarrow{z})$}
	{$\overrightarrow{AG}=\dfrac{2}{3}(\overrightarrow{x}+\overrightarrow{y}+\overrightarrow{z})$}
	}
	{\begin{tikzpicture}[font=\footnotesize, line join=round, line cap=round, >=stealth, scale=1]
		\path (0,0) coordinate (B) (2,2) coordinate (A) (-40:1.5) coordinate (C) (3.5,0) coordinate (D);
		\draw (A)--(B)--(C)--(D)--cycle (A)--(C);
		\draw[dashed] (B)--(D);
		\foreach \x/\g in {A/90, B/180, C/-90, D/0}{
		\fill (\x) circle (1pt)+(\g:0.3)node{$\x$};
		}
	\end{tikzpicture}}
	\loigiai{
	Do $G$ là trọng tâm $\triangle BCD$ nên $\overrightarrow{GB}+\overrightarrow{GC}+\overrightarrow{GD}=\overrightarrow{0}$.\\
	Suy ra $3\overrightarrow{AG}=\overrightarrow{AB}+\overrightarrow{AC}+\overrightarrow{AD} \Leftrightarrow \overrightarrow{AG}=\dfrac{1}{3}(\overrightarrow{AB}+\overrightarrow{AC}+\overrightarrow{AD})$.\\
	Vậy $\overrightarrow{AG}=\dfrac{1}{3}(\overrightarrow{x}+\overrightarrow{y}+\overrightarrow{z})$.
	}
\end{ex}

\begin{ex}%[Nguồn: Bộ đề minh họa Moon 2024-2025]%[2H2H1-2]
	Cho hình lập phương $ABCD.A'B'C'D'$. Khẳng định nào sau đây {\bf sai}?
	\choice
	{$\left(\overrightarrow{AB}; \overrightarrow{A'D'}\right)=90^\circ$}
	{$\left(\overrightarrow{AB}; \overrightarrow{A'C'}\right)=45^\circ$}
	{$\left(\overrightarrow{AC}; \overrightarrow{B'D'}\right)=90^\circ	$}
	{\True $\left(\overrightarrow{A'A}; \overrightarrow{CB'}\right)=45^\circ$}
	\loigiai{
	\immini{
	Ta có
	\begin{itemize}
		\item $\overrightarrow{A'D'} =\overrightarrow{AD} \Rightarrow \left(\overrightarrow{AB}; \overrightarrow{A'D'}\right)=\left(\overrightarrow{AB}; \overrightarrow{AD}\right)=90^\circ$.
		\item $\overrightarrow{A'C'}=\overrightarrow{AC}	\Rightarrow \left(\overrightarrow{AB}; \overrightarrow{A'C'}\right) =\left(\overrightarrow{AB}; \overrightarrow{AC}\right)=45^\circ$.
		\item $\heva{&B'D'\parallel BD	\\&AC\perp BD}\Rightarrow \left(\overrightarrow{AC}; \overrightarrow{B'D'}\right)=90^\circ$.
		\item $\overrightarrow{A'A}=\overrightarrow{C'C}\Rightarrow \left(\overrightarrow{A'A}; \overrightarrow{CB'}\right)=\left(\overrightarrow{C'C}; \overrightarrow{CB'}\right)=135^\circ$.
	\end{itemize}
	}
	{\begin{tikzpicture}[line join=round, line cap = round, >=stealth, scale=.8,font=\footnotesize,transform shape]
		\def\a{3.5}
		\path	(0:0) coordinate (A)
		++(0:\a) coordinate (D)
		++(-140:\a/2) coordinate (C)
		($(A)+(C)-(D)$) coordinate (B)
		($(A)+(90:\a)$) coordinate (A')
		($(B)+(90:\a)$) coordinate (B')
		($(C)+(90:\a)$) coordinate (C')
		($(D)+(90:\a)$) coordinate (D');
		\draw[dashed]	(B)--(A)--(D)	(A)--(A');
		\draw	(C)--(C')		(B)--(B')	(C)--(D) (A')--(B')--(C')--(D')--cycle;
		\foreach \x/\g in {A/180,B/180,C/0,D/0,A'/180,B'/180,C'/0,D'/0}
		\fill[black]	(\x) circle (1pt)
		($(\g:4mm)+(\x)$) node {$\x$};
		\draw[->] (B)--(C) (D)--(D');
	\end{tikzpicture}}
	}
\end{ex}

\begin{ex}%[Nguồn: Bộ đề minh họa Moon 2024-2025]%[2H2H1-2]
	Cho hình lăng trụ tam giác đều $ABC.A'B'C'$ có cạnh $AB = 3$, $AA' = 4$. Độ dài của vectơ $\overrightarrow{u} = \overrightarrow{A'B'} + \overrightarrow{C'A}$ bằng
	\begin{center}
		\begin{tikzpicture}[>=stealth,line join=round,line cap=round,font=\footnotesize,scale=1]
			\path
			(0,0) coordinate (A)
			(1,-1.5) coordinate (B)
			(4,0) coordinate (C)
			($(A)+(0,3)$) coordinate (A')
			($(B)+(0,3)$) coordinate (B')
			($(C)+(0,3)$) coordinate (C')
			;
			\draw
			(B') -- (A') -- (A) -- (B) -- (C) -- (C') -- (B') -- (B)
			(A') -- (C')
			;
			\draw[dashed]
			(C') -- (A) -- (C)
			;
			\fill
			(A) circle(1pt) node[left]{$A$}
			(B) circle(1pt) node[left]{$B$}
			(C) circle(1pt) node[below]{$C$}
			(A') circle(1pt) node[above]{$A'$}
			(B') circle(1pt) node[above]{$B'$}
			(C') circle(1pt) node[above]{$C'$}
			;
		\end{tikzpicture}
	\end{center}
	\choice
	{$7$}
	{$12$}
	{\True $5$}
	{$10$}
	\loigiai
	{Ta có $\overrightarrow{u} = \overrightarrow{A'B'} + \overrightarrow{C'A} = \overrightarrow{C'A} + \overrightarrow{AB} = \overrightarrow{C'B}$.\\
	Ta có $\left| \overrightarrow{u} \right| = C'B = \sqrt{BC^2 + C'C^2} = \sqrt{3^2 + 4^2} = 5$.}
\end{ex}

\begin{ex}%[Nguồn: Bộ đề minh họa Moon 2024-2025]%[2H2H1-2]
	Cho hình lăng trụ $ABC.A'B'C'$. Gọi $M$ là trung điểm của cạnh $BB'$. Đặt $\overrightarrow{CA}=\overrightarrow{a}$, $\overrightarrow{CB}=\overrightarrow{b}$, $\overrightarrow{AA'}=\overrightarrow{c}$. Khẳng định nào dưới đây là đúng?
	\choice
	{$\overrightarrow{AM}=\overrightarrow{a}+\overrightarrow{c}-\dfrac{1}{2}\overrightarrow{b}$}
	{$\overrightarrow{AM}=\overrightarrow{b}+\overrightarrow{c}-\dfrac{1}{2}\overrightarrow{a}$}
	{\True $\overrightarrow{AM}=\overrightarrow{b}-\overrightarrow{a}+\dfrac{1}{2}\overrightarrow{c}$}
	{$\overrightarrow{AM}=\overrightarrow{a}-\overrightarrow{c}+\dfrac{1}{2}\overrightarrow{b}$}
	\loigiai{\immini{Vì $M$ là trung điểm của $BB'$ nên $\overrightarrow{AM}=\dfrac{1}{2}\overrightarrow{AB}+\dfrac{1}{2}\overrightarrow{AB'}.\quad (1)$\\
	Ta có $\overrightarrow{AB}=\overrightarrow{CB}-\overrightarrow{CA}=\overrightarrow{b}-\overrightarrow{a}$.\\
	Vì $ABB'A'$ là hình bình hành nên
	\[\overrightarrow{AB'}=\overrightarrow{AB}+\overrightarrow{AA'}=\overrightarrow{CB}-\overrightarrow{CA}+\overrightarrow{AA'}=\overrightarrow{b}-\overrightarrow{a}+\overrightarrow{c}.\]
	Thay vào $(1)$, ta có
	\[\overrightarrow{AM}=\dfrac{1}{2}\left(\overrightarrow{b}-\overrightarrow{a}\right)+\dfrac{1}{2}\left(\overrightarrow{b}-\overrightarrow{a}+\overrightarrow{c}\right)=\overrightarrow{AM}=\overrightarrow{b}-\overrightarrow{a}+\dfrac{1}{2}\overrightarrow{c}.\]
	}
	{\begin{tikzpicture}[>=stealth,line join=round,line cap=round,font=\footnotesize,scale=0.5]
		\clip (-1,-4) rectangle (11,7);
		\path
		(0,0) coordinate (A)
		(3,-3) coordinate (B)
		(8,0) coordinate (C)
		(2,5) coordinate (A')
		(5,2) coordinate (B')
		(10,5) coordinate (C')
		(4,-0.5) coordinate (M)
		;
		\draw  (A)--(B)--(C)--(C')--(A')--(B')--(B)(B')--(C')(A')--(A);
		\fill[black] (A)node[above left]{$A$}(C) circle (1.5pt) node[below right]{$C$}(B)  node[below left]{$B$}(A') node[below left]{$A'$}(B') circle (1.5pt) node[below right]{$B'$}(C') circle (1.5pt) node[below right]{$C'$}(M)node[below right]{$M$};
		\draw[->,dashed] (C)--(A);
		\draw[->] (C)--(B);
		\draw[->](A)--(A');
		\draw[->] (A)--(M);
		\draw[->] (A)--(B');
		\draw(2,0)[above right]node{$\overrightarrow{a}$};
		\draw(4,-2.8)[right]node{$\overrightarrow{b}$};
		\draw(1,2)[above right]node{$\overrightarrow{c}$};
	\end{tikzpicture}}
	}
\end{ex}

\begin{ex}%[Nguồn: Bộ đề minh họa Moon 2024-2025]%[2H2H1-2]
	Cho hình lăng trụ $ABC.A'B'C'$. Gọi $M$ là trung điểm của cạnh $BC$. Khẳng định nào sau đây đúng?
	\choice
	{\True $\overrightarrow{AM} + \dfrac{1}{2}\overrightarrow{B'C'} = \overrightarrow{A'C'}$}
	{$\overrightarrow{AM} + \dfrac{1}{2}\overrightarrow{B'C'} = \overrightarrow{AB}$}
	{$\overrightarrow{AM} + \dfrac{1}{2}\overrightarrow{B'C'} = \overrightarrow{AC'}$}
	{$\overrightarrow{AM} + \dfrac{1}{2}\overrightarrow{B'C'} = \overrightarrow{A'C}$}
	\loigiai{
	\immini{
	Do $M$ là trung điểm $BC$ nên $\dfrac{1}{2}\overrightarrow{B'C'}=\overrightarrow{MC}$.\\
	Khi đó $\overrightarrow{AM}+\dfrac{1}{2}\overrightarrow{B'C'}=\overrightarrow{AM}+\overrightarrow{MC}=\overrightarrow{AC}=\overrightarrow{A'C'}$. }{\begin{tikzpicture}[line join=round, line cap=round,thick,scale=0.5]
		\coordinate (A') at (0,4);
		\coordinate (A) at (-2,0);
		\coordinate (B) at (0,-2);
		\coordinate (C) at (3,0);
		\coordinate (B') at ($(B) + (2,4)$);
		\coordinate (C') at ($(C) + (2,4)$);
		\coordinate (M) at ($(B)!0.5!(C)$);
		\draw(A')--(A)  (B)--(C) (A)--(B)--(B')--(A')--(C')--(C) (B')--(C');
		\draw[dashed,thin](A)--(C);
		\foreach \i/\g in {A'/90,A/180,B/-90,C/0,B'/0,C'/0,M/0}{\draw[fill=black](\i) circle (1.5pt) ($(\i)+(\g:4mm)$) node[scale=1]{$\i$};}
	\end{tikzpicture}}
	}
\end{ex}

%

\begin{ex}%[Nguồn: Bộ đề minh họa Moon 2024-2025]%[2H2H1-3]
	Cho hình chóp tứ giác đều $S.ABCD$ có tất cả các cạnh bằng $2$. Gọi $M$ là trung điểm $SD$ (minh hoạ như hình vẽ). Giá trị của $\vec{CM} \cdot \vec{BA}$ bằng
	\begin{center}
		\begin{tikzpicture}[line cap=round,line join=round,x=1.0cm,y=1.0cm,>=stealth,scale=1]
			\path
			(0:0) coordinate (A)
			+(0:5) coordinate (D)
			+(-140:2.5) coordinate (B)
			($(B)+(D)-(A)$) coordinate (C)
			(intersection of A--C and B--D) coordinate (O)
			++(90:5) coordinate (S);
			\path ($(S)!.5!(D)$) coordinate (M);
			\draw[dashed]
			(A)--(B) (A)--(D) (A)--(S);
			\draw
			(D)--(C)--(B)
			(S)--(B) (S)--(C)--(M) (S)--(D);
			\foreach \x/\g in {A/135,D/0,C/-45,B/-135,M/45,S/90}
			\fill [blue] (\x) circle (1.5pt)
			+(\g:3mm) node {$\x$};
		\end{tikzpicture}
	\end{center}
	\choice
	{$-\sqrt{3}$}
	{$-3$}
	{\True $3$}
	{$\sqrt{3}$}
	\loigiai{
	$\vec{CM} \cdot \vec{BA}=\vec{CM}\cdot\vec{CD}=CM\cdot CD\cdot \cos(\vec{CM},\vec{CD})=\dfrac{\sqrt{3}}{2}\cdot2\cdot2\cos30^\circ=3$.
	}
\end{ex}

%

\begin{ex}%[Nguồn: Bộ đề minh họa Moon 2024-2025]%[2H2H1-3]
	Góc giữa đường thẳng $AA'$ và $BC'$ bằng
	\choice
	{$30^{\circ}$}
	{$90^{\circ}$}
	{\True $45^{\circ}$}
	{$60^{\circ}$}
	\loigiai{
	Từ giả thiết ta có $BCC'B'$ là hình vuông.\\ Khi đó
	$(A'A,BC')=(B'B,BC')=\widehat{B'BC}=45^{\circ}$.}
\end{ex}

%

\begin{ex}%[Nguồn: Bộ đề minh họa Moon 2024-2025]%[2H2H1-3]
	Tính giá trị của $\overrightarrow{AA'} \cdot \overrightarrow{BC'}$.
	\choice
	{$8$}
	{$16$}
	{$16\sqrt{2}$}
	{\True $8\sqrt{2}$}
	\loigiai{
	Hình vuông $BCC'B'$ có cạnh bằng $4$ nên $BC'=4\sqrt{2}$.\\
	Khi đó $\overrightarrow{AA'} \cdot \overrightarrow{BC'}=AA'\cdot BC' \cos\big(\overrightarrow{AA'},\overrightarrow{BC'}\big)=4\cdot 4\cos 45^\circ= 8\sqrt{2}$.
	}
\end{ex}

\begin{ex}%[Nguồn: Bộ đề minh họa Moon 2024-2025]%[2H2H2-2]
	Trong không gian với hệ trục toạ độ $Oxyz$, cho điểm $M\left(1; 2; 3\right)$. Hình chiếu vuông góc của $M$ lên trục $Ox$ là điểm
	\choice
	{\True $R\left(1; 0; 0 \right)$}
	{$S\left(0; 0; 3 \right)$}
	{$P\left(1; 0; 3 \right)$}
	{$Q\left(0; 2; 0 \right)$}
	\loigiai
	{Hình chiếu vuông góc của $M$ lên trục $Ox$ là điểm $R\left(1; 0; 0 \right)$.
	}
\end{ex}

\begin{ex}%[Nguồn: Bộ đề minh họa Moon 2024-2025]%[2H2H2-3]
	Trong không gian $Oxyz$, cho điểm $M(2;0;1)$. Gọi $A$, $B$ lần lượt là hình chiếu vuông góc của $M$ trên trục $O x$ và trên mặt phẳng $(Oyz)$. Đường thẳng $AB$ có một vectơ chỉ phương là vectơ nào sau đây?
	\choice
	{$\vec{u}_1=(2;0;1)$}
	{\True $\vec{u}_2=(-2;0;1)$}
	{$\vec{u}_3=(1;0;-2)$}
	{$\vec{u}_4=(1;0;2)$}
	\loigiai
	{
	$A$ là hình chiếu vuông góc của $M$ trên trục $Ox$ nên $A(2;0;1)$.\\
	$B$ là hình chiếu vuông góc của $M$ trên mặt phẳng $Oyz$ nên $B(0;0;1)$.\\
	Véctơ chỉ phương của đường thẳng $AB$ là véctơ $\overrightarrow{AB}=(-2;0;1)$.
	}
\end{ex}

%

\begin{ex}%[Nguồn: Bộ đề minh họa Moon 2024-2025]%[2H2N1-2]
	Cho hình hộp $ABCD. A' B' C' D'$. Tính tổng $\overrightarrow{AB}+\overrightarrow{AD}+\overrightarrow{A' C'}$
	\begin{center}
		\begin{tikzpicture}[declare function={a=2;b=4;h=3;},line join=round]
			\path (0,0) coordinate (B)
			(35:a) coordinate (A)
			(b,0) coordinate (C)
			($(C)-(B)+(A)$) coordinate (D)
			($(A)+(90:h)$) coordinate (A')
			($(B)-(A)+(A')$) coordinate (B')
			($(C)-(A)+(A')$) coordinate (C')
			($(D)-(A)+(A')$) coordinate (D');
			\draw ( B')--(B)--(C)--(D)--(D')--(A')--(B')--(C')--(D')  (C)--(C');
			\draw[dashed]  (A')--(A)--(D)  (A)--(B);
			\foreach \t/\g in {A/180,B/-90,C/-90,A'/90,B'/90,C'/90,D'/0,D/0}{
			\draw[fill=black] (\t) circle (1pt) node[shift={(\g:7pt)},font=\scriptsize]{$ \t $};
			}
		\end{tikzpicture}
	\end{center}
	\choice
	{$2\overrightarrow{AA'}$}
	{$\overrightarrow{0}$}
	{\True $2\overrightarrow{AC}$}
	{$2\overrightarrow{C'A'}$}
	\loigiai{
	Ta có $\overrightarrow{A'C'}=\overrightarrow{AC}$ và $\overrightarrow{AB}+\overrightarrow{AD}=\overrightarrow{AC}$ (quy tắc hình bình hành).\\
	Suy ra $\overrightarrow{AB}+\overrightarrow{AD}+\overrightarrow{A' C'}=\left(\overrightarrow{AB}+\overrightarrow{AD}\right)+\overrightarrow{A' C'}=\overrightarrow{AC}+\overrightarrow{AC}=2\overrightarrow{AC}$.
	}
\end{ex}

\begin{ex}%[Nguồn: Bộ đề minh họa Moon 2024-2025]%[2H2N1-2]
	Cho tứ diện đều $ABCD$ cạnh $a$. Độ dài của vectơ $\vec{u}=\overrightarrow{AB}-\overrightarrow{AC}$ là
	\choice
	{$0$}
	{$2a$}
	{$\sqrt{2} a$}
	{\True $a$}
	\loigiai{Ta có $\vec{u}=\overrightarrow{A B}-\overrightarrow{A C}=\overrightarrow{C B}$\\
	$\Rightarrow|\overrightarrow{B C}|=BC=a$.}
\end{ex}

%

\begin{ex}%[Nguồn: Bộ đề minh họa Moon 2024-2025]%[2H2N1-2]
	\immini{	Cho hình lập phương $ABCD.A'B'C'D'$. Tổng $\overrightarrow{BA'} + \overrightarrow{D'C'}$ là vectơ nào sau đây?
	\choice
	{$\overrightarrow{BC}$}
	{$\overrightarrow{AC}$}
	{\True $\overrightarrow{CC'}$}
	{$\overrightarrow{BA}$}}{
	\begin{tikzpicture}[scale=0.7]
		\def\a{3.5}
		\path	(0:0) coordinate (A)
		++(0:\a) coordinate (D)
		++(-130:\a/2) coordinate (C)
		($(A)+(C)-(D)$) coordinate (B)
		($(A)+(90:\a)$) coordinate (A')
		($(B)+(90:\a)$) coordinate (B')
		($(C)+(90:\a)$) coordinate (C')
		($(D)+(90:\a)$) coordinate (D');
		\draw[dashed]	(B)--(A)--(D)	(A)--(A');
		\draw	(C)--(C')	(D)--(D')	(B)--(B')	(B)--(C)--(D) (A')--(B')--(C')--(D')--cycle;
		\foreach \x/\g in {A/180,B/180,C/0,D/0,A'/180,B'/180,C'/0,D'/0}
		\fill[black]	(\x) circle (1pt)
		($(\g:4mm)+(\x)$) node {$\x$};
	\end{tikzpicture}}
	\loigiai{Ta có $\overrightarrow{BA'}+\overrightarrow{D'C'}=\overrightarrow{CD'}+\overrightarrow{D'C'}=\overrightarrow{CC'}$.}
\end{ex}

\begin{ex}%[Nguồn: Bộ đề minh họa Moon 2024-2025]%[2H2N1-3]
	Trong không gian $Oxyz$, cho hai véc-tơ $\overrightarrow{a} = (2;-1;3)$, $\overrightarrow{b} = (0;3;5)$. Tính $\overrightarrow{a} \cdot \overrightarrow{b}$.
	\choice
	{\True$12$}
	{$7$}
	{ $9$}
	{$6$}
	\loigiai{
	$\overrightarrow{a} \cdot \overrightarrow{b} = 2 \cdot 0+ (-1 )\cdot 3+ 3 \cdot 5= 0 - 3 + 15 = 12$.
	}
\end{ex}

\begin{ex}%[Nguồn: Bộ đề minh họa Moon 2024-2025]%[2H2N2-1]
	Trong không gian $Oxyz$, cho điểm $M(-4;2;-3)$. Tìm toạ độ điểm $N$ đối xứng với $M$ qua trục $Oy$.
	\choice
	{$N(-4;-2;-3)$}
	{\True $N(4;2;3)$}
	{$N(-4;2;3)$}
	{$N(0;2;0)$}
	\loigiai{
	Điểm $N$ đối xứng với $M$ qua trục $Oy$ có tọa độ là $N(4;2;3)$.
	}
\end{ex}

%

\begin{ex}%[Nguồn: Bộ đề minh họa Moon 2024-2025]%[2H2N2-2]
	Trong không gian $Oxyz$, điểm $M'$ đối xứng với điểm $M(2 ; 3 ;-4)$ qua gốc toạ độ $O$ có toạ độ là
	\choice
	{$M'(-2;-3;-4)$}
	{\True $M'(-2;-3;4)$}
	{$M'(-2;3;-4)$}
	{$M'(2;3;4)$}
	\loigiai{
	$M'(-2;-3;4)$ đối xứng với $M(2 ; 3 ;-4)$ qua gốc toạ độ $O$.}
\end{ex}

\begin{ex}%[Nguồn: Bộ đề minh họa Moon 2024-2025]%[2H2N2-2]
	Trong không gian $Oxyz$, cho hai vectơ $\overrightarrow{u}=(2;1;0)$ và $\overrightarrow{v}=(-1;0;-2)$. Tính $\cos\left(\overrightarrow{u};\overrightarrow{v}\right)$.
	\choice{$\cos\left(\overrightarrow{u};\overrightarrow{v}\right)=\dfrac{2}{25}$}
	{\True $\cos\left(\overrightarrow{u};\overrightarrow{v}\right)=-\dfrac{2}{5}$}
	{$\cos\left(\overrightarrow{u};\overrightarrow{v}\right)=-\dfrac{2}{25}$}
	{$\cos\left(\overrightarrow{u};\overrightarrow{v}\right)=\dfrac{2}{5}$}
	\loigiai{Ta có:
	
	\[\cos\left(\overrightarrow{u};\overrightarrow{v}\right) = \dfrac{\overrightarrow{u} \cdot \overrightarrow{v}}{|\overrightarrow{u}| \cdot |\overrightarrow{v}|}=\dfrac{(2)\cdot(-1) + (1)\cdot(0) + (0)\cdot(-2)}{\sqrt{2^2 + 1^2 + 0^2}\cdot \sqrt{(-1)^2 + 0^2 + (-2)^2}}=-\dfrac{2}{5}.\]
	}
\end{ex}

\begin{ex}%[Nguồn: Bộ đề minh họa Moon 2024-2025]%[2H2N2-4]
	Trong không gian $Oxyz$, cho $A(1;-2;3)$, $B(2;-4;1)$, $C(2;0;2)$, khi đó tích vô hướng $\overrightarrow{AB}\cdot\overrightarrow{AC}$ bằng
	\choice
	{$7$}
	{$-5$}
	{$4$}
	{\True $-1$}
	\loigiai{Ta có $\overrightarrow{AB}=(1;-2;-2)$ và $\overrightarrow{AC}=(1;2;-1)$.\\
	Khi đó $\overrightarrow{AB}\cdot\overrightarrow{AC}=1\cdot1+(-2)\cdot2+(-2)\cdot(-1)=-1$.}
\end{ex}

%Câu trắc nghiệm 4

\begin{ex}%[Nguồn: Bộ đề minh họa Moon 2024-2025]%[2H2N2-4]
	Trong không gian $Oxyz$, cho vectơ $\overrightarrow{a}=(1;0;3)$. Độ dài của vectơ $\overrightarrow{a}$ là
	\choice
	{$1$}
	{$2$}
	{$4$}
	{\True $\sqrt{10}$}
	\loigiai{
	Ta có $|\overrightarrow{a}|=\sqrt{1^2+0^2+3^2}=\sqrt{10}$.
	}
\end{ex}

\begin{ex}%[Nguồn: Bộ đề minh họa Moon 2024-2025]%[2H2N2-4]
	Trong không gian $O x y z$, cho điểm $A(2 ; 2 ; 1)$. Tính độ dài đoạn thẳng $O A$.
	\choice
	{\True $O A=3$}
	{$O A=9$}
	{$O A=\sqrt{5}$}
	{$O A=5$}
	\loigiai{
	Ta có $OA=\sqrt{x_A^2+y_A^2+z_A^2}=\sqrt{2^2+2^2+1^2}=3$.
	}
\end{ex}

\begin{ex}%[Nguồn: Bộ đề minh họa Moon 2024-2025]%[2H2V1-4]
	Trong không gian với hệ tọa độ $Oxyz$, một cabin cáp treo xuất phát từ điểm $A(-20;12;0)$, chuyển động thẳng đều theo đường cáp và cùng chiều với véc-tơ $\vec{u}=(2;-2;1)$ với tốc độ là $6$ m/s (đơn vị trên mỗi trục tọa độ là mét); giả sử sau $t$ (s) kể từ lúc xuất phát $(t\geq 0)$, cabin đến điểm $M$.
	\begin{center}
		\begin{tikzpicture}[line join=round,line cap=round,>=stealth]
			% Cáp treo
			\draw[thick] (-1,5) -- (10,5);
			% Các cabin
			\foreach \x in {0, 4, 8} {
			% Cabin chính
			\draw[thick] (\x,3) -- (\x+2,3) -- (\x+1.8,4.5) -- (\x+0.2,4.5) -- cycle; % Cabin ngoài
			\fill[gray!30] (\x,3) -- (\x+2,3) -- (\x+1.8,4.5) -- (\x+0.2,4.5) -- cycle; % Màu cabin
			% Dải đen dưới thân cabin
			\fill[black] (\x+0.2,3) rectangle (\x+1.8,3.5);
			% Cửa sổ
			\fill[white] (\x+0.4,3.6) rectangle (\x+0.8,4.3); % Cửa sổ trái
			\fill[white] (\x+1.2,3.6) rectangle (\x+1.6,4.3); % Cửa sổ phải
			% Thanh treo cabin
			\draw[thick] (\x+1,4.5) -- (\x+1,5);
			\draw[thick] (\x+0.8,4.8) -- (\x+1.2,4.8); % Thanh ngang treo
			% Chi tiết nối
			\draw[thick] (\x+0.9,4.7) -- (\x+1.1,4.7);
			\draw[thick] (\x+1,4.8) -- (\x+1,4.7);
			}
		\end{tikzpicture}
	\end{center}
	\choiceTF
	{\True Độ dài đoạn $AM$ là $AM = 6t$ (m)}
	{\True Véc-tơ $\overrightarrow{AM}=(2t;-2t;t)$}
	{\True Tọa độ điểm $M$ sau $t$ giây là $M(-20+2t; 12-2t; t)$}
	{Một người đứng tại điểm $O$ quan sát cabin chạy trên cáp treo, sau thời gian $3{,}56$ giây (làm tròn kết quả đến hàng phần trăm) thì khoảng cách giữa người quan sát và cabin là gần nhất}
	\loigiai{
	\begin{itemchoice}
		\itemch Cabin chuyển động thẳng đều với tốc độ $6$ m/s. Sau $t$ giây, khoảng cách mà cabin đã di chuyển là $6t$ m.\\
		Do đó $AM = 6t$ (m).
		\itemch Ta có $\overrightarrow{AM}=t\overrightarrow{u}=(2t;-2t;t)$.
		\itemch Ta có $\overrightarrow{AM}=(2t;-2t;t)\Leftrightarrow\heva{&x_M+20=2t\\&y_M-12=-2t\\&z_M=t}\Leftrightarrow\heva{&x_M=-20+2t\\&y_M=12\\&z_M=t.}$\\
		Tọa độ điểm $M$ sau $t$ giây là $M(-20+2t; 12-2t; t)$.
		\itemch Khoảng cách giữa người quan sát và cabin là độ dài $OM$.\\
		$OM=\sqrt{(-20t+2t)^2+(12-2t)^2+t^2}\Rightarrow OM^2=9t^2-128t+544$.\\
		Ta thấy $OM^2$ đạt giá trị nhỏ nhất khi $t\approx7{,}11$.
	\end{itemchoice}
	}
\end{ex}

%

\begin{ex}%[Nguồn: Bộ đề minh họa Moon 2024-2025]%[2H2V1-4]
	Cho hình lăng trụ đứng $ABC.A'B'C'$ có đáy $ABC$ là tam giác đều cạnh $\sqrt{2}$, $BA'=\sqrt{6}$. Khoảng cách giữa hai đường thẳng $A'B$ và $B'C$ là bao nhiêu? (làm tròn kết quả đến hàng phần trăm).
	\shortans{0{,}67}
	\loigiai{
	\begin{center}
		\begin{tikzpicture}[>=stealth,declare function={a=3; b=0.6*a; h=a;goc=-50;}]
			\path
			(0,0) coordinate (A)
			(a,0) coordinate (C)
			(goc:b) coordinate (B)
			\foreach \p in {A,B,C} {
			($(\p)+(90:h)$) coordinate (\p')}
			($(B)!0.5!(A)$)	coordinate (O)
			;
			\draw[dashed] (A)--(C)--(O);
			\draw (A')--(B')--(C')--cycle (C)--(B')--(B)--(C)--(C') (A')--(A)--(B)--cycle;
			\draw[->](B)--($(B)+(goc:1cm)$)node[shift={(180:7pt)},font=\small]{$x$};
			\draw[->](C)--($(C)!-0.4!(O)$)node[shift={(-90:7pt)},font=\small]{$y$};
			\draw[->](O)--+($(90:{h+1.4})$)node[shift={(200:7pt)},font=\small]{$z$};
			\foreach \x/\goc in {A/160,B/180,C/-40,A'/160,B'/220,C'/0,O/200}{
			\draw[fill] (\x) circle (1pt) node[shift={(\goc:7pt)},font=\small]{$\x$};
			}
		\end{tikzpicture}
	\end{center}
	Gọi $O$ là trung điểm của $AB$, do $\triangle ABC$ đều cạnh $\sqrt{2}$ nên $OC=\dfrac{\sqrt{6}}{2}$.\\
	$\triangle A'AB$ vuông tại $A$ nên $A'A=\sqrt{A'B^2-AB^2}=\sqrt{\big(\sqrt{6}\big)^2-\big(\sqrt{2}\big)^2}=2$.\\
	Chọn hệ trục tọa độ $O xyz$ như hình vẽ. Khi đó\\
	$A\left(-\dfrac{\sqrt{2}}{2};0;0\right)$, $B\left(\dfrac{\sqrt{2}}{2};0;0\right)$,
	$C\left(0;\dfrac{\sqrt{6}}{2};0\right)$,
	$A'\left(-\dfrac{\sqrt{2}}{2};0;2\right)$,
	$B'\left(\dfrac{\sqrt{2}}{2};0;2\right)$,
	$C'\left(0;\dfrac{\sqrt{6}}{2};2\right)$.\\
	Ta có $\overrightarrow{A'B}=\big(\sqrt{2};0;-2\big)$, $\overrightarrow{B'C}=\left(-\dfrac{\sqrt{2}}{2};\dfrac{\sqrt{6}}{2};-2\right)$ suy ra $\dfrac{\sqrt{6}}{2}\left[\overrightarrow{A'B},\overrightarrow{B'C}\right]=\big(2;2\sqrt{3};\sqrt{2}\big)$.\\
	Mặt phẳng $(P)$ chứa $B'C$ và song song với $A'B$ có phương trình $2x+2\sqrt{3}y+\sqrt{2}z-3\sqrt{2}=0$.\\
	Từ đó\\
	$\mathrm{d}\big(A'B,B'C\big)=\mathrm{d}\big(A'B,(P)\big)=\mathrm{d}\big(B,(P)\big)=\dfrac{\left|2\cdot \dfrac{\sqrt{2}}{2}+2\sqrt{3}\cdot 0+\sqrt{2}\cdot 0-3\sqrt{2}\right|}{\sqrt{4+12+2}}=\dfrac{2}{3}\approx 0{,}67$.
	}
\end{ex}

%Câu đúng sai 2

\begin{ex}%[Nguồn: Bộ đề minh họa Moon 2024-2025]%[2H2V2-6]
	Một chiếc điện thoại được đặt trên một giá đỡ có ba chân với điểm đặt $S(0;0;20)$ và các điểm chạm mặt đất của ba chân lần lượt là $A(0;-6;0)$, $B(3\sqrt{3};3;0)$, $C(-3\sqrt{3};3;0)$ (đơn vị cm). Cho biết điện thoại có trọng lượng là $2$ N và ba lực tác dụng lên giá đỡ được phân bố như hình vẽ là ba lực $\overrightarrow{F}_1$, $\overrightarrow{F}_2$, $\overrightarrow{F}_3$ có độ lớn bằng nhau và đo bằng đơn vị N.
	\begin{center}
		\tdplotsetmaincoords{75}{115}
		\begin{tikzpicture}[font=\footnotesize, line join=round, line cap=round, >=stealth, scale=0.4, tdplot_main_coords]
			\pgfmathsetmacro\bancanba{3*sqrt(3)}
			\draw[->] (-7,0,0) -- (9,0,0) node[anchor=north east] {$x$};
			\draw[->] (0,-7,0) -- (0,7,0) node[anchor=north west] {$y$};
			\draw[->] (0,0,10.8) -- (0,0,12) node[anchor=south] {$z$};
			\path
			(0,0,0) coordinate (O)
			(0,-6,0) coordinate (A)
			(\bancanba,3,0) coordinate (B)
			(-\bancanba,3,0) coordinate (C)
			(0,0,9) coordinate (S);
			\draw[dashed] (0,0,0) circle [radius=6] (A)--(B)--(C)--cycle (S)--(O);
			\draw (S)--(A) (S)--(B) (S)--(C);
			\foreach \x [count=\i from 1] in {A,B,C}{
			\path ($(S)!1/3!(\x)$) coordinate (f\i);
			\draw[->] (S)--(f\i)node[right=-1mm]{$\overrightarrow{F}_\i$};
			}
			\draw[rounded corners=1] (0,-2,8.9) rectangle (0,2,11);
			\fill (0,-2,9.2) rectangle (0,-1.8,10.2);
			\foreach \x/\g in {C/above right, A/above left, B/below, S/above right, O/below}{
			\fill (\x) circle (3.3pt)node[\g]{$\x$};
			}
		\end{tikzpicture}
	\end{center}
	\choiceTF
	{\True $\overrightarrow{SA}=(0;-6;-20)$}
	{$\overrightarrow{F}_1+\overrightarrow{F}_2+\overrightarrow{F}_3=\overrightarrow{F}(0;0;2)$}
	{$\left|\overrightarrow{F}_1\right|=\dfrac{1}{20}\left|\overrightarrow{SA}\right|$}
	{\True Biết tọa độ của lực $\overrightarrow{F}_1=(a;b;c)$, khi đó $T=2a+5b+6c=-5$}
	\loigiai{
	\begin{itemchoice}
		\itemch Ta có $\overrightarrow{SA}=(0;-6;-20)$.
		\itemch Ta có $\overrightarrow{F}_1+\overrightarrow{F}_2+\overrightarrow{F}_3=\overrightarrow{F}$, do điện thoại có trọng lượng là $2$ N nên $\left|\overrightarrow{F}\right|=2$.\\
		Lại có $\left|\overrightarrow{SO}\right|=20$, $\overrightarrow{F}$ và $\overrightarrow{SO}=(0;0;-20)$ cùng hướng nên $\overrightarrow{F}=\dfrac{1}{10}\overrightarrow{SO}$.\\
		Suy ra $\overrightarrow{F}=(0;0;-2)$.
		\itemch Do ba lực $\overrightarrow{F}_1$, $\overrightarrow{F}_2$, $\overrightarrow{F}_3$ có độ lớn bằng nhau nên với cùng số $k$, ta có $\overrightarrow{F}_1=k\overrightarrow{SA}$, $\overrightarrow{F}_2=k\overrightarrow{SB}$ và $\overrightarrow{F}_3=k\overrightarrow{SC}$.\\
		Ta có $\overrightarrow{SB}=(3\sqrt3;3;-20)$, $\overrightarrow{SC}=(-3\sqrt3;3;-20)$. Khi đó
		$$ \overrightarrow{F}_1+\overrightarrow{F}_2+\overrightarrow{F}_3=\overrightarrow{F} \Leftrightarrow \heva{& k\cdot0+k\cdot3\sqrt3+k\cdot(-3\sqrt3) =0 \\& k\cdot(-6)+k\cdot3+k\cdot3=0 \\& k\cdot(-20)+k\cdot(-20)+k\cdot(-20)=-2} \Leftrightarrow k=\dfrac{1}{30}. $$
		Suy ra $\overrightarrow{F}_1=\dfrac{1}{30}\overrightarrow{SA}$.\\
		Vậy $\left|\overrightarrow{F}_1\right|=\dfrac{1}{30}\left|\overrightarrow{SA}\right|$.
		\itemch Ta có $\overrightarrow{F}_1=\dfrac{1}{30}\overrightarrow{SA}$ suy ra $\overrightarrow{F}_1=\left(0;-\dfrac{1}{5};-\dfrac{2}{3}\right)$. Do đó $a=0$, $b=-\dfrac{1}{5}$, $c=-\dfrac{2}{3}$.\\
		Vậy $T=2a+5b+6c=-5$.
	\end{itemchoice}
	}
\end{ex}

\begin{ex}%[Nguồn: Bộ đề minh họa Moon 2024-2025]%[2H2V2-6]
	Hình vẽ sau mô tả vị trí của máy bay vào thời điểm $9$h$30$ phút. Biết các đơn vị trên hình tính theo đơn vị km.
	\begin{center}
		\begin{tikzpicture}[line join = round, line cap=round,>=stealth,font=\footnotesize,scale=1]
			\path
			(0,0) coordinate (O)
			(5,0) coordinate (B)
			(-3,-2) coordinate (A)
			(0,4) coordinate (C)
			($(A)+(B)-(O)$) coordinate (N)
			($(N)+(0,4)$) coordinate (M)
			;
			\draw	(O)--(B)(O)--(A) (O)--(C);
			
			\foreach \x/\r/\p in{A/180/x,B/90/y,C/90/z}
			\draw[->,line width=2pt] (O)--($(O)!1.2!(\x)$)node[scale=1.5,shift={(\r:3mm)}]{$\p$};
			;
			
			%\draw pic[draw,angle radius=7mm] {angle = M--O--C};
			\draw (A) node[shift={(150:4mm)}]{$150$};
			\draw (B) node[shift={(90:4mm)}]{$300$};
			\draw (C) node[shift={(180:4mm)}]{$9$};
			\draw (B) node[shift={(-70:4mm)}]{(Đông)};
			\draw (A) node[shift={(-30:4mm)}]{(Nam)};
			
			\draw[dashed] (O)--(N)--(B) (A)--(N)--(M)--(C);
			\draw[->,line width=2pt] (O)--(M);
			
			\draw (M) node[yshift=.4cm]{	\begin{tikzpicture}[line join = round, line cap=round,>=stealth,font=\footnotesize,scale=0.15]
			
			\draw[cyan,line width=3pt]
			(0,0)--(0.2,-0.5)coordinate (A)--(5,-1.2)coordinate (O)
			(10,0)--(9.8,-0.5)coordinate (A')--(5,-1.2)
			(5,-0.8) circle(0.7cm)
			($(A)!0.6!(O)+(0,-0.3)$) circle(0.3cm)
			($(A)!0.6!(O)+(0,-0.3)$) circle(0.2cm)
			($(A')!0.6!(O)+(0,-0.3)$) circle(0.3cm)
			($(A')!0.6!(O)+(0,-0.3)$) circle(0.2cm)
			(7,0)--(5,-0.3) -- (3,0)
			(5,-0.3)--(5,1.2)
			;
			\fill[cyan] (5,-0.8) circle(0.7cm);
			
			\fill[black,xshift=-0.05cm] (4.5,-0.7) rectangle (4.6,-0.5)
			(4.7,-0.7) rectangle (4.8,-0.5)
			(4.9,-0.7) rectangle (5,-0.5)
			(5.1,-0.7) rectangle (5.2,-0.5)
			(5.3,-0.7) rectangle (5.4,-0.5)
			(5.5,-0.7) rectangle (5.6,-0.5)
			;
			
			\draw[line width=2pt]
			($(A')!0.75!(O)$)--($(A')!0.75!(O)+(0,-0.65)$)
			($(A')!0.75!(O)+(0,-0.65)$)--++(0:0.2)
			($(A')!0.75!(O)+(0,-0.65)$)--++(180:0.2)--++(-90:0.1)
			($(A')!0.75!(O)+(0,-0.65)$)--++(180:0.2)--++(90:0.1)
			($(A')!0.75!(O)+(0,-0.65)$)--++(0:0.2)--++(-90:0.1)
			($(A')!0.75!(O)+(0,-0.65)$)--++(0:0.2)--++(90:0.1)
			($(A)!0.75!(O)$)--($(A)!0.75!(O)+(0,-0.65)$)
			($(A)!0.75!(O)+(0,-0.65)$)--++(0:0.2)
			($(A)!0.75!(O)+(0,-0.65)$)--++(180:0.2)--++(-90:0.1)
			($(A)!0.75!(O)+(0,-0.65)$)--++(180:0.2)--++(90:0.1)
			($(A)!0.75!(O)+(0,-0.65)$)--++(0:0.2)--++(-90:0.1)
			($(A)!0.75!(O)+(0,-0.65)$)--++(0:0.2)--++(90:0.1)
			;
		\end{tikzpicture}	}
		;
		\end{tikzpicture}
	\end{center}
	\choiceTF
	{\True Máy bay đang ở độ cao $9$ km}
	{ Tọa độ của máy bay lúc này là $(300; 150; 9)$}
	{\True Phi công để máy bay ở chế độ tự động với vận tốc theo hướng đông là $750$ km/h, độ cao không đổi. Biết rằng gió thổi theo hướng đông với vận tốc $10$ m/s. Giả sử vận tốc và hướng gió không đổi thì lúc $10$h$30$ phút máy bay ở tọa độ là $(150; 1086; 9)$}
	{Sau khi bay đến vị trí lúc $10$h$30$ thì máy bay bay ngược lại (hướng tây) với vận tốc $800$ km/h với độ cao không đổi, biết lúc đó trời lặng gió thì lúc $11$h máy bay ở tọa độ $(686; 150; 9)$}
	\loigiai{
	\begin{itemchoice}
		\itemch  {\bf Đúng.}\\
		Dựa vào hình vẽ ta thấy máy bay đang ở độ cao $9$ km.
		\itemch  {\bf Sai.}\\
		Máy bay đang ở tọa độ $(150; 300; 9)$.
		\itemch  {\bf Đúng.}\\
		Vận tốc gió $10$ m/s $= 36$ km/h.\\
		Máy bay bay tự động trong khoảng thời gian từ $9$h$30$ đến $10$h$30$ với quãng đường $750$ km.\\
		Quãng đường thực tế máy bay bay được là $750+36=786$ km.\\
		Do đó tọa độ máy bay là $(150; 1\,086; 9)$.
		\itemch  {\bf Sai.}\\
		Quãng đường máy bay bay được trong khoảng thời gian từ $10$h$30$ đến $11$h là \[800\cdot \dfrac{1}{2}=400\,\, \text{km}.\]
		Do đó tọa độ máy bay là $(150; 868; 9)$.
	\end{itemchoice}
	}
\end{ex}

\begin{ex}%[Nguồn: Bộ đề minh họa Moon 2024-2025]%[2H2V2-6]
	Cho hình lập phương $ABCD.A'B'C'D'$ có cạnh bằng $20$ cm.
	\begin{center}
		\begin{tikzpicture}[>=stealth,line join=round,line cap=round,font=\footnotesize,scale=.8]
			\definecolor{cinnamon}{rgb}{0.82, 0.41, 0.12}
			\definecolor{bistre}{rgb}{0.24, 0.17, 0.12}
			\definecolor{amber}{rgb}{1.0, 0.75, 0.0}
			% Định nghĩa "kien" pic
			\tikzset{
			pics/kien/.style n args={2}{
			code={
			\def\mycolorone{#1}
			\def\mycolortwo{#2}
			\clip (-3,-3) rectangle (3,3);
			\draw[line width=.2mm]
			(-.58,.77)
			..controls +(-160:.05) and +(20:.05) ..(-.85,.8)
			..controls +(-140:.1) and +(130:.2) ..(-1.4,.6)
			(-.6,.7)
			..controls +(130:.3) and +(20:.2) ..(-1.9,1.05)
			(-.75,.2)
			..controls +(-160:.05) and +(20:.05) ..(-.85,.2)
			..controls +(-140:.1) and +(-10:.3) ..(-1.4,-.1)
			(-.42,0)
			..controls +(-140:.1) and +(10:.2) ..(-.8,-.4)
			(-.38,0)
			..controls +(-100:.2) and +(10:.2) ..(-.83,-.43);
			
			% Các thành phần
			\def\D{
			(-.75,.45)
			..controls +(-150:.1) and +(95:.15) ..(-.92,.22)
			--(-.85,.3)--(-.8,.23)
			..controls +(90:.1) and +(-160:.15) ..(-.6,.45)--cycle;
			}
			\fill[\mycolorone] \D;
			\draw[black] \D;
			
			\def\C2{
			(.6,-.1)
			..controls +(150:.1) and +(-150:.1) ..(.8,.5)
			..controls +(30:.1) and +(80:.1) ..(.85,0)
			..controls +(30:.1) and +(-30:.1) ..cycle;
			}
			\fill[\mycolorone] \C2;
			\draw[black] \C2;
			
			\def\N{
			(-.05,.65)
			..controls +(40:.2) and +(82:.35) ..(.45,.35)
			..controls +(-30:.1) and +(80:.25) ..(.7,0)
			..controls +(70:.1) and +(82:.15) ..(.9,0)
			..controls +(-98:.15) and +(-130:.55) ..(.6,-.1)
			..controls +(-130:.25) and +(-150:.4) ..(.3,.3)
			..controls +(-130:.25) and +(-90:.4) ..(-.1,.55)--(-.15,.55)
			..controls +(-150:.25) and +(10:.3) ..(-.65,.4)
			..controls +(160:.25) and +(-40:.2) ..(-.8,.5)
			..controls +(65:.3) and +(100:.5) ..cycle;
			}
			\fill[\mycolortwo!20!\mycolorone] \N;
			\draw[black] \N;
			
			% Các phần khác
			\def\C1{
			(1,-.1)
			..controls +(150:.2) and +(-150:.3) ..(1.45,.4)
			..controls +(30:.1) and +(130:.1) ..(1.8,-1)
			..controls +(-70:.1) and +(-150:.05) ..(2.1,-1.4)--(2.07,-1.43)
			..controls +(150:.45) and +(-85:.2) ..(1.45,.3)
			..controls +(-110:.1) and +(-30:.05) ..cycle;
			}
			\fill[\mycolorone] \C1;
			\draw[black] \C1;
			
			\def\D{
			(1.2,-1.25)
			..controls +(40:.9) and +(15:.6) ..(.7,0)
			..controls +(-120:.65) and +(-140:.1) ..cycle;
			}
			\fill[\mycolorone] \D;
			\draw[black] \D;
			
			\def\C{
			(.2,0)
			..controls +(150:.2) and +(-40:.35) ..(-.2,.5)
			..controls +(150:.2) and +(40:.2) ..(-.8,.2)
			..controls +(-10:.05) and +(-150:.1) ..(-.7,.2)
			..controls +(30:.1) and +(-150:.2) ..(-.3,.45)
			..controls +(30:.1) and +(150:.2) ..(0,0)
			..controls +(-40:.1) and +(-50:.2) ..cycle;
			}
			\fill[\mycolorone] \C;
			\draw[\mycolorone] \C;
			
			\def\C4{
			(.2,-.05)
			..controls +(50:.2) and +(-30:.35) ..(.2,-.15)
			..controls +(160:.1) and +(-70:.05) ..(.15,-.1)
			..controls +(160:.1) and +(-70:.05) ..(.05,.44)
			..controls +(-140:.1) and +(50:.05) ..(-.35,0)
			..controls +(-140:.05) and +(50:.05) ..(-.45,0)
			..controls +(60:.05) and +(-130:.05) ..(.05,.57)
			..controls +(50:.2) and +(130:.05) ..cycle;
			}
			\fill[\mycolorone] \C4;
			\draw[\mycolorone] \C4;
			
			\def\C3{
			(.5,-.3)
			..controls +(150:.2) and +(-140:.25) ..(.8,.25)
			..controls +(40:.2) and +(85:1.5) ..(.75,-1.5)
			..controls +(-95:.1) and +(30:.05) ..(.65,-1.7)
			..controls +(120:.05) and +(-50:.05) ..(.63,-1.67)
			..controls +(30:.25) and +(-110:.25) ..(.8,.15)
			..controls +(-110:.1) and +(-30:.1) ..cycle;
			}
			\fill[\mycolorone] \C3;
			\draw[black] \C3;
			
			% Mắt
			\def\D{
			(-.4,.8)
			..controls +(0:.1) and +(15:.1) ..(-.5,.65)
			..controls +(-165:.1) and +(180:.1) ..cycle;
			}
			\fill[\mycolorone] \D;
			\draw[black] \D;
			}
			}
			}
			
			\tikzset{
			pics/hhChuNhat/.style n args={8}{
			code={
			\tikzset{
			declare function={a=3;b=1.5;goc=-130;h=2;}
			}
			\path
			(0,0)coordinate (#1)+(0:a)coordinate (#2)+(goc:b)coordinate (#4)+(90:h)coordinate (#5)
			($(#2)+(#4)-(#1)$)coordinate (#3)
			;
			\foreach \pone/\pname in {#2/#6,#3/#7,#4/#8}{
			\path
			($(\pone)+(#5)-(#1)$)coordinate (\pname)
			;
			}
			\foreach \pointo/\pointt in {#1/#2,#1/#4,#1/#5}{
			\draw[fill=black,dashed](\pointo)--(\pointt);
			}
			\foreach \pointo/\pointt in {#2/#3,#3/#4,#5/#6,#6/#7,#7/#8,#8/#5,#2/#6,#3/#7,#4/#8}{
			\draw[fill=black](\pointo)--(\pointt);
			}
			}
			}
			}
			
			\path
			(0,0) pic{hhChuNhat={A}{D}{C}{B}{A'}{D'}{C'}{B'}}
			;
			\foreach \pointo/\pointt in {A/D',D/B}{
			\draw[fill=black,dashed](\pointo)--(\pointt);
			}
			\path ($(A)+(60:.1)$) pic[xscale=-1,scale=.1,rotate=-10] {kien={amber}{amber}} (D) pic[scale=.1,rotate=45] {kien={bistre}{cinnamon}};
			\foreach \point/\goc in {A/-90,D/-80,C/-60,B/190,A'/90,B'/120,C'/120,D'/80}{
			\draw[fill=black](\point)circle(.8pt)+(\goc:2mm)node[scale=.8]{$\point$};
			}
		\end{tikzpicture}
	\end{center}
	Hai chú kiến vàng và đen xuất phát cùng một lúc tại các vị trí $A$ và $D$, kiến vàng đi từ $A$ đến $D'$ với vận tốc $2\,\text{cm/s}$ và kiến đen đi từ $D$ đến $B$ với vận tốc $3\,\text{cm/s}$. Hỏi khoảng cách ngắn nhất giữa hai chú kiến là bao nhiêu $\text{cm}$? (Viết kết quả làm tròn đến hàng phần chục).
	
	\shortans[oly]{19{,}3}
	\loigiai{
	\begin{center}
		\begin{tikzpicture}[>=stealth,line join=round,line cap=round,font=\footnotesize,scale=.8]
			\definecolor{cinnamon}{rgb}{0.82, 0.41, 0.12}
			\definecolor{bistre}{rgb}{0.24, 0.17, 0.12}
			\definecolor{amber}{rgb}{1.0, 0.75, 0.0}
			% Định nghĩa "kien" pic
			\tikzset{
			pics/kien/.style n args={2}{
			code={
			\def\mycolorone{#1}
			\def\mycolortwo{#2}
			\clip (-3,-3) rectangle (3,3);
			\draw[line width=.2mm]
			(-.58,.77)
			..controls +(-160:.05) and +(20:.05) ..(-.85,.8)
			..controls +(-140:.1) and +(130:.2) ..(-1.4,.6)
			(-.6,.7)
			..controls +(130:.3) and +(20:.2) ..(-1.9,1.05)
			(-.75,.2)
			..controls +(-160:.05) and +(20:.05) ..(-.85,.2)
			..controls +(-140:.1) and +(-10:.3) ..(-1.4,-.1)
			(-.42,0)
			..controls +(-140:.1) and +(10:.2) ..(-.8,-.4)
			(-.38,0)
			..controls +(-100:.2) and +(10:.2) ..(-.83,-.43);
			
			% Các thành phần
			\def\D{
			(-.75,.45)
			..controls +(-150:.1) and +(95:.15) ..(-.92,.22)
			--(-.85,.3)--(-.8,.23)
			..controls +(90:.1) and +(-160:.15) ..(-.6,.45)--cycle;
			}
			\fill[\mycolorone] \D;
			\draw[black] \D;
			
			\def\C2{
			(.6,-.1)
			..controls +(150:.1) and +(-150:.1) ..(.8,.5)
			..controls +(30:.1) and +(80:.1) ..(.85,0)
			..controls +(30:.1) and +(-30:.1) ..cycle;
			}
			\fill[\mycolorone] \C2;
			\draw[black] \C2;
			
			\def\N{
			(-.05,.65)
			..controls +(40:.2) and +(82:.35) ..(.45,.35)
			..controls +(-30:.1) and +(80:.25) ..(.7,0)
			..controls +(70:.1) and +(82:.15) ..(.9,0)
			..controls +(-98:.15) and +(-130:.55) ..(.6,-.1)
			..controls +(-130:.25) and +(-150:.4) ..(.3,.3)
			..controls +(-130:.25) and +(-90:.4) ..(-.1,.55)--(-.15,.55)
			..controls +(-150:.25) and +(10:.3) ..(-.65,.4)
			..controls +(160:.25) and +(-40:.2) ..(-.8,.5)
			..controls +(65:.3) and +(100:.5) ..cycle;
			}
			\fill[\mycolortwo!20!\mycolorone] \N;
			\draw[black] \N;
			
			% Các phần khác
			\def\C1{
			(1,-.1)
			..controls +(150:.2) and +(-150:.3) ..(1.45,.4)
			..controls +(30:.1) and +(130:.1) ..(1.8,-1)
			..controls +(-70:.1) and +(-150:.05) ..(2.1,-1.4)--(2.07,-1.43)
			..controls +(150:.45) and +(-85:.2) ..(1.45,.3)
			..controls +(-110:.1) and +(-30:.05) ..cycle;
			}
			\fill[\mycolorone] \C1;
			\draw[black] \C1;
			
			\def\D{
			(1.2,-1.25)
			..controls +(40:.9) and +(15:.6) ..(.7,0)
			..controls +(-120:.65) and +(-140:.1) ..cycle;
			}
			\fill[\mycolorone] \D;
			\draw[black] \D;
			
			\def\C{
			(.2,0)
			..controls +(150:.2) and +(-40:.35) ..(-.2,.5)
			..controls +(150:.2) and +(40:.2) ..(-.8,.2)
			..controls +(-10:.05) and +(-150:.1) ..(-.7,.2)
			..controls +(30:.1) and +(-150:.2) ..(-.3,.45)
			..controls +(30:.1) and +(150:.2) ..(0,0)
			..controls +(-40:.1) and +(-50:.2) ..cycle;
			}
			\fill[\mycolorone] \C;
			\draw[\mycolorone] \C;
			
			\def\C4{
			(.2,-.05)
			..controls +(50:.2) and +(-30:.35) ..(.2,-.15)
			..controls +(160:.1) and +(-70:.05) ..(.15,-.1)
			..controls +(160:.1) and +(-70:.05) ..(.05,.44)
			..controls +(-140:.1) and +(50:.05) ..(-.35,0)
			..controls +(-140:.05) and +(50:.05) ..(-.45,0)
			..controls +(60:.05) and +(-130:.05) ..(.05,.57)
			..controls +(50:.2) and +(130:.05) ..cycle;
			}
			\fill[\mycolorone] \C4;
			\draw[\mycolorone] \C4;
			
			\def\C3{
			(.5,-.3)
			..controls +(150:.2) and +(-140:.25) ..(.8,.25)
			..controls +(40:.2) and +(85:1.5) ..(.75,-1.5)
			..controls +(-95:.1) and +(30:.05) ..(.65,-1.7)
			..controls +(120:.05) and +(-50:.05) ..(.63,-1.67)
			..controls +(30:.25) and +(-110:.25) ..(.8,.15)
			..controls +(-110:.1) and +(-30:.1) ..cycle;
			}
			\fill[\mycolorone] \C3;
			\draw[black] \C3;
			
			% Mắt
			\def\D{
			(-.4,.8)
			..controls +(0:.1) and +(15:.1) ..(-.5,.65)
			..controls +(-165:.1) and +(180:.1) ..cycle;
			}
			\fill[\mycolorone] \D;
			\draw[black] \D;
			}
			}
			}
			\tikzset{
			pics/hhChuNhat/.style n args={8}{
			code={
			\tikzset{
			declare function={a=3;b=1.5;goc=-130;h=2;}
			}
			\path
			(0,0)coordinate (#1)+(0:a)coordinate (#2)+(goc:b)coordinate (#4)+(90:h)coordinate (#5)
			($(#2)+(#4)-(#1)$)coordinate (#3)
			;
			\foreach \pone/\pname in {#2/#6,#3/#7,#4/#8}{
			\path
			($(\pone)+(#5)-(#1)$)coordinate (\pname)
			;
			}
			\foreach \pointo/\pointt in {#1/#2,#1/#4,#1/#5}{
			\draw[fill=black,dashed](\pointo)--(\pointt);
			}
			\foreach \pointo/\pointt in {#2/#3,#3/#4,#5/#6,#6/#7,#7/#8,#8/#5,#2/#6,#3/#7,#4/#8}{
			\draw[fill=black](\pointo)--(\pointt);
			}
			}
			}
			}
			
			\path
			(0,0) pic{hhChuNhat={D}{C}{B}{A}{D'}{C'}{B'}{A'}}
			($(A)!2/6!(D')$)coordinate (M)
			($(D)!3/6!(B)$)coordinate (N)
			;
			\foreach \po/\truc/\pos in{D'/z/right,A/x/left,C/y/above}{
			\draw[->] (\po)--($(D)!1.3!(\po)$)node[\pos]{$\truc$};
			}
			\foreach \pointo/\pointt in {A/D',D/B}{
			\draw[fill=black,dashed](\pointo)--(\pointt);
			}
			\path ($(A)+(60:.1)$) pic[xscale=-1,scale=.1,rotate=-10] {kien={amber}{amber}} (D) pic[scale=.1,xscale=-1,rotate=40] {kien={bistre}{cinnamon}};
			\foreach \point/\goc in {A/-90,D/-80,C/-60,B/-90,A'/160,B'/120,C'/120,D'/170,M/170,N/45}{
			\draw[fill=black](\point)circle(.8pt)+(\goc:2mm)node[scale=.8]{$\point$};
			}
		\end{tikzpicture}
	\end{center}
	Giả sử trong khoảng thời gian $x$ (s) ($x>0$) kiến vàng đi được quãng đường là $AM=2x$ cm, khi đó kiến đen đi được quãng đường là $DN=3x$, cm.\\
	Không mất tính tổng quát, ta giả sử rằng hình lập phương có độ dài cạnh bằng $1$ cm và ta đặt hệ trục tọa độ $Dxyz$ như hình vẽ trên.\\
	Từ đó suy ra $AD'=DB=\sqrt{2}$ cm;  $D=(0;0;0)$, $B=(1;1;0)$, $A(1,0,0)$ và $D'(0;0,1)$.\\
	Suy ra $\overrightarrow{AD'}=(1;0;-1)$, $\overrightarrow{DB}=(1;1;0)$; $\overrightarrow{AM}=(x_M-1;y_M,z_M)$; $\overrightarrow{DN}=(x_N;y_N,z_N)$.\\
	Ta có $\heva{
	&\overrightarrow{AM}=\dfrac{2x}{AD'}\overrightarrow{AD'}=\dfrac{2x}{\sqrt{2}}\overrightarrow{AD'}=\left(\dfrac{2x}{\sqrt{2}};0;-\dfrac{2x}{\sqrt{2}}\right)\\
	&\overrightarrow{DN}=\dfrac{3x}{DB}\overrightarrow{DB}=\dfrac{3x}{\sqrt{2}}\overrightarrow{DB}=\left(\dfrac{3x}{\sqrt{2}};\dfrac{3x}{\sqrt{2}};0\right)
	}\Rightarrow \heva{&M=\left(\dfrac{2x}{\sqrt{2}}+1;0;-\dfrac{2x}{\sqrt{2}}\right)\\&N=\left(\dfrac{3x}{\sqrt{2}};\dfrac{3x}{\sqrt{2}};0\right)}\Rightarrow \overrightarrow{MN}=\left(\dfrac{x}{\sqrt{2}}-1;-\dfrac{3x}{\sqrt{2}};\dfrac{2x}{\sqrt{2}}\right)$.\\
	Khi đó $MN=\sqrt{\left(\dfrac{x}{\sqrt{2}}-1\right)^2+\left(-\dfrac{3x}{\sqrt{2}}\right)^2+\left(\dfrac{2x}{\sqrt{2}}\right)^2}=\sqrt{7x^2-\sqrt{2}x+1}=f(x)$.\\
	Suy ra $MN_{\min}=f\left(\dfrac{\sqrt{2}}{14}\right)=\sqrt{\dfrac{13}{14}}$.\\
	Vậy khoảng cách ngắn nhất giữa hai chú kiến bằng $20\cdot \sqrt{\dfrac{13}{14}}\approx 19{,}3$ cm.
	}
\end{ex}

\begin{ex}%[Nguồn: Bộ đề minh họa Moon 2024-2025]%[2H2V2-6]
	Trong không gian $Oxyz$ (đơn vị đo là km), bốn chiếc máy bay ở bốn hướng khác nhau khi vừa bay vào vùng phủ sóng của một chiếc radar thì trên radar cùng một lúc báo tín hiệu phát hiện ra mục tiêu. Tại thời điểm radar phát hiện mục tiêu thì $4$ chiếc máy bay ở vị trí có toạ độ lần lượt là $A \left(30; 25; 33 \right)$, $B \left(14; 1; 49 \right)$, $C \left(40; -29; 1\right)$, $D \left(0; 31; 41 \right)$. Hỏi bán kính vùng phủ sóng của radar là bao nhiêu km?
	\shortans{$50$}
	\loigiai{
	Gọi $M \left(x; y; z \right)$ là địa điểm đặt rađa.\\
	Theo đề bài ra ta có
	\begin{align*}
		\heva{&MA =MB\\&MA = MC\\&MA = MD} & \Rightarrow \heva{&MA^2 = MB^2\\&MA^2 = MC^2\\&MA^2 = MD^2}\\
		& \Rightarrow \heva{&\left(x-30 \right)^2 + \left(y -25 \right)^2 + \left(z - 33 \right)^2 = \left(x- 14 \right)^2 + \left(y-1 \right)^2 + \left(z-49 \right)^2\\
		&\left(x-30 \right)^2 + \left(y -25 \right)^2 + \left(z - 33 \right)^2 = \left(x-40 \right)^2 + \left(y + 29 \right)^2 + \left(z-1 \right)^2 \\
		&\left(x-30 \right)^2 + \left(y -25 \right)^2 + \left(z - 33 \right)^2 = x^2 + \left(y - 31 \right)^2 + \left(z - 41 \right)^2
		}\\
		&\Rightarrow \heva{&-32x - 48y + 32z = -16\\&-20x -108y -64z = -172\\&-60x + 12y + 16z = 28}\\
		&\Rightarrow \heva{&x=0\\&y=1\\&z=1.}
	\end{align*}
	Suy ra, bán kính vùng phủ sóng $R = OA = \sqrt{\left(-30 \right)^2 + \left(-24 \right)^2 + \left(-32 \right)^2} = 50$ km.
	}
\end{ex}

%

\begin{ex}%[Nguồn: Bộ đề minh họa Moon 2024-2025]%[2H5C1-7]
	\immini{Một vật trang trí dạng hình chóp tam giác đều có chiều cao $110$ mm và đáy là tam giác đều cạnh $120$ mm như hình vẽ bên. Chọn hệ trục tọa độ $Oxyz$ sao cho gốc tọa độ $O$ trùng với trung điểm của cạnh đáy $BC$, đỉnh $B$ thuộc tia $Ox$ và đỉnh $A$ thuộc tia $Oy$ (đơn vị mỗi trục là mm, xem hình vẽ)}{
	\begin{tikzpicture}[scale=0.8]
		\def\a{5}
		\def\h{5}
		\path	(0:0) coordinate (A)
		++(0:\a) coordinate (B)
		++(-150:4*\a/5) coordinate (C)
		($(C)!0.5!(B)$) coordinate (O)
		($(A)!2/3!(O)$) coordinate (H)
		($(H)+(90:\h)$) coordinate (S)
		(B)++(30:\a/5)coordinate (x)
		($(A)!-1/3!(O)$) coordinate (y)
		($(O)+(90:5/4*\h)$) coordinate (z);
		\draw (C)--(A) (C)--(B)
		(A)--(S)	(B)--(S)	(C)--(S);
		\draw[dashed] (A)--(B) (S)--(H)	(A)--(O);
		\draw[->] (O)--(z);
		\draw[->] (B)--(x);
		\draw[->] (A)--(y);
		\foreach \x / \goc in {A/120,B/-30,C/-135,H/30,S/90,O/-45}
		\fill (\x) circle (1.5pt)
		($(\x)+(\goc:3mm)$) node {$\x$};
		\foreach \y / \goc in {x/0,y/-170,z/60}
		\fill (\y) circle (0.1pt)
		($(\y)+(\goc:3mm)$) node {$\y$};
	\end{tikzpicture}}
	\choiceTF
	{\True $S(0; 20\sqrt{3}; 110)$, $A(0; 60\sqrt{3}; 0)$}
	{$\overrightarrow{SB} = (-60; -20\sqrt{3}; 110)$, $\overrightarrow{AB} = (60; -60\sqrt{3}; 0)$}
	{ Một vectơ pháp tuyến của mặt phẳng $(SAB)$ là $\overrightarrow{n} = \left(11\sqrt{3}; -11; -4\sqrt{3}\right)$}
	{Cosin của góc giữa hai mặt bên $(SAB)$ và $(SAC)$ của vật trang trí đó bằng $0,34$ (làm tròn kết quả đến hàng phần trăm)}
	\loigiai{\begin{itemchoice}
		\itemch Vì $BC\perp (SAH)$ tại điểm $O$ nên $(SAH)\equiv (Oyz)$.\\
		Ta có $SH=110$, $OA=60\sqrt{3}$, $OH=\dfrac{1}{3}OA=\dfrac{1}{3}\cdot\dfrac{120\sqrt{3}}{2}=20\sqrt{3}$.\\
		Vậy $S\left(0;20\sqrt{3};110\right)$, $A\left(0;60\sqrt{3};0\right)$.
		\itemch Ta có $B\left(60;0;0\right)$.\\
		Vậy $\overrightarrow{SB}=\left(60;-20\sqrt{3};-110\right)$, $\overrightarrow{AB}=\left(60;-60\sqrt{3};0\right)$.
		\itemch Gọi $\overrightarrow{n}_1$ là vectơ pháp tuyến của mặt phẳng $(SAB)$.\\
		Ta có $\heva{&\overrightarrow{n}_1\perp\overrightarrow{SB}\\&\overrightarrow{n}_1\perp\overrightarrow{AB}}\Rightarrow \overrightarrow{n}_1=\left[\overrightarrow{SB},\overrightarrow{AB}\right]=\left(-6600\sqrt{3};-6600;-2400\sqrt{3}\right)$.\\
		Vì $\overrightarrow{n}=\left(11\sqrt{3};-11;-4\sqrt{3}\right)\neq \dfrac{1}{600}\overrightarrow{n}_1$ nên $\overrightarrow{n}$ không là vectơ pháp tuyến của mặt phẳng $(SAB)$.
		\itemch Gọi $\alpha$ là góc giữa hai mặt phẳng $(SAB)$ và $(SAC)$.\\
		Ta có $(SAB)$ có vectơ pháp tuyến là $\overrightarrow{n}=\left(-11\sqrt{3};-11;-4\sqrt{3}\right)$.\\
		Mặt khác, $C\left(-60;0;0\right)$ nên $\overrightarrow{AC}=\left(-60;-60\sqrt{3};0\right)$, $\overrightarrow{SA}=\left(0;40\sqrt{3};-110\right)$.\\
		Khi đó, $(SAC)$ có vectơ pháp tuyến $\overrightarrow{n}_2=\left[\overrightarrow{SA},\overrightarrow{AC}\right]=\left(-6600\sqrt{3};6600;2400\sqrt{3}\right)$ hay có vectơ pháp tuyến là $\overrightarrow{n'}=\left(-11\sqrt{3};11;4\sqrt{3}\right)$.\\
		Suy ra, \begin{eqnarray*}
			\cos\alpha&=&\left|\cos\left(\overrightarrow{n},\overrightarrow{n'}\right)\right|\\&=&\dfrac{\left|-11\sqrt{3}\cdot (-11\sqrt{3})+(-11)\cdot11+(-4\sqrt{3})\cdot(4\sqrt{3})\right|}{\sqrt{(-11\sqrt{3})^2+(-11)^2+(-4\sqrt{3})^2}\cdot\sqrt{(-11\sqrt{3})^2+11^2+(4\sqrt{3})^2}}\\
			&=&\dfrac{50}{83}\approx 0,60.
		\end{eqnarray*}
	\end{itemchoice}}
\end{ex}

\begin{ex}%[Nguồn: Bộ đề minh họa Moon 2024-2025]%[2H5C2-6]
	Trong không gian với hệ tọa độ $O x y z$, cho hai điểm $A(2 ;-1 ; 3), B(-2 ; 1 ;-1)$ và đường thẳng $d\colon \dfrac{x+1}{1}=\dfrac{y-1}{-4}=\dfrac{z-1}{-1}$. Gọi $M$ là điểm chạy trên đường thẳng $d$ và $N$ là chân đường cao hạ từ $B$ lên đường thẳng $A M$. Biết điểm $N$ luôn chạy trên một đường cong cố định. Độ dài đường cong đó bằng bao nhiêu? (làm tròn kết quả đến hàng phần chục).
	\shortans[]{$18{,}7$}
	\loigiai{
	Ta có $\widehat{ANB}=90^\circ$ nên $N$ thuộc mặt cầu $(S)$ có đường kính $AB=6$, bán kính $R=3$, tâm $I(0;0;1)$.\\
	Lại có $N \in  MA \Rightarrow N \in (A;d)$, hay $N$ chạy trên mặt phẳng $(P)$  chứa  $A$ và $d$. Chọn điểm $K(-1;1;1) \in d$ $$\Rightarrow \overrightarrow{n_P}=\left[\overrightarrow{u_d} ; \overrightarrow{A K}\right]=(10 ; 5 ;-10)=5(2 ; 1 ;-2) .$$
	Phương trình mặt phẳng $(P)$ chứa
	$A(2;-1;3)$ có $\overrightarrow{n_P}=(2 ; 1 ;-2)$ là $(P)\colon  2(x-2)+(y+1)-2(z-3)=0$  hay $2 x+y-2 z+3=0$. Ta có
	$$d=d(I ; (P))=\dfrac{|2\cdot 0+0-2\cdot1+3|}{\sqrt{4+1+4}}=\dfrac{1}{3} .$$
	Điểm $N$ vừa thuộc $(S)$ vừa thuộc $(P)$ nên $N$ thuộc đường tròn giao tuyến của $(S)$ và $(P)$. Bán kính đường tròn giao tuyến là  $$r=\sqrt{R^2-d^2}  =\sqrt{3^2-\left(\dfrac{1}{3}\right)^2}=\dfrac{4 \sqrt{5}}{3}.$$
	Vậy độ dài đường cong là $2 \pi r=\dfrac{8 \pi \sqrt{5}}{3} \approx 18,7.$
	}
\end{ex}

%

\begin{ex}%[Nguồn: Bộ đề minh họa Moon 2024-2025]%[2H5H1-2]
	Trong không gian với hệ trục tọa độ $Oxyz$, cho mặt phẳng $(P)$ có phương trình $x-3y-z+8=0$. véc-tơ nào sau đây là một véc-tơ pháp tuyến của mặt phẳng $(P)$?
	\choice
	{$\overrightarrow{n_1}(1;-3; 1)$}
	{\True $\overrightarrow{n_2}(1;-3;-1)$}
	{$\overrightarrow{n_3}(1;-3; 8)$}
	{$\overrightarrow{n_4}(1; 3; 8)$}
	\loigiai{
	Mặt phẳng $(P)\colon x-3 y-z+8=0$ có một véc-tơ pháp tuyến là $\overrightarrow{n_3}=(1 ;-3 ;-1)$.
	}
\end{ex}

\begin{ex}%[Nguồn: Bộ đề minh họa Moon 2024-2025]%[2H5H1-2]
	Trong không gian $Oxyz$, cho mặt phẳng $(P)\colon x-4y+3z-2=0$. Véc-tơ nào dưới đây là một véc-tơ pháp tuyến của $(P)$?
	\choice
	{$\vec{n}=(0;-4; 3)$}
	{$\vec{n}=(1; 4; 3)$}
	{\True$\vec{n}=(-1; 4;-3)$}
	{$\vec{n}=(-4; 3;-2)$}
	\loigiai{
	Véc-tơ pháp tuyến của mặt phẳng $(P)$ là $\vec{n}=(-1; 4;-3)$.
	}
\end{ex}

\begin{ex}%[Nguồn: Bộ đề minh họa Moon 2024-2025]%[2H5H1-3]
	Trong không gian với hệ tọa độ $Oxyz$, cho các điểm $A(1;-1; 3)$, $B(0;-4;-1)$, $ C(2;-1; 2)$ và $D(3;-1;-8)$.
	\choiceTF
	{\True $\overrightarrow{AB}=(-1;-3;-4)$}
	{\True Một vectơ pháp tuyến của mặt phẳng $(ABC)$ là $\bar{n}(6;-10; 6)$}
	{\True Phương trình của mặt phẳng $(ABC)$ là $3x-5y+3z-17=0$}
	{Bốn điểm $A, B, C, D$ cùng thuộc một mặt phẳng}
	\loigiai{
	\begin{itemchoice}
		\itemch Ta có $\overrightarrow{A B}=(-1 ;-3 ;-4)$
		\itemch Mặt phẳng $(ABC)$ có 2 véc tơ chỉ phương là  $\overrightarrow{A B}=(-1 ;-3 ;-4)$, $ \overrightarrow{B C}=(2 ; 3 ; 3)$
		$\Rightarrow \overrightarrow{n_{A B C}}=[\overrightarrow{A B}, \overrightarrow{B C}]=(3 ;-5 ; 3).$\\
		Do đó véc tơ $(6 ;-10 ; 6)=2 \overrightarrow{n_{A B C}}$ cũng là một véc tơ pháp tuyến của mặt phẳng $(ABC)$.
		\itemch Mặt phẳng $(ABC)$ đi qua  $A(1 ;-1 ; 3)$ và nhận
		$\overrightarrow{n_{A B C}}=(3 ;-5 ; 3)$  làm  véc tơ pháp tuyến có phương trình là
		$3(x-1)-5(y+1)+3(z-3)=0\Leftrightarrow 3 x-5 y+3 z-17=0$.
		\itemch Thay tọa độ  điểm $D(3;-1;-8)$ vào phương trình mặt phẳng $(ABC)$ ta thấy $D\not\in (ABC)$ nên 4 điểm $A,B,C,D$ không đồng phẳng.
		
	\end{itemchoice}
	}
\end{ex}

\begin{ex}%[Nguồn: Bộ đề minh họa Moon 2024-2025]%[2H5H1-3]
	Trong không gian $Oxyz$, cho điểm $A(1; 2; -1)$. Mặt phẳng $(P)$ đi qua $A$ và chứa trục $Oy$ có phương trình là
	\choice
	{$x-z=0$}
	{$x-2z=0$}
	{$y=2$}
	{\True $x+z=0$}
	\loigiai{
	Đường thẳng chứa trục $Oy$ đi qua điểm $O(0;0;0)$ và nhận vectơ $\overrightarrow{j}=(0; 1; 0)$ làm vectơ chỉ phương.\\
	Ta có $\overrightarrow{OA}=(1; 2; -1)$, suy ra $\left[\overrightarrow{OA}; \overrightarrow{j}\right]=(1; 0; 1)$.\\
	Mặt phẳng $(P)$ đi qua $A$ và nhận vectơ $\overrightarrow{n}=(1; 0; 1)$ làm vectơ pháp tuyến có phương trình $x+z=0$.
	}
\end{ex}

%

\begin{ex}%[Nguồn: Bộ đề minh họa Moon 2024-2025]%[2H5H1-3]
	Trong không gian $Oxyz$, cho hai điểm $A(1;-1;0)$ và $B(1;2;1)$. Phương trình của mặt phẳng đi qua điểm $A$ và vuông góc với $AB$ là
	\choice
	{\True $3y+z+3=0$}
	{$y+z+1=0$}
	{$3y+z-3=0$}
	{$x+y+z=0$}
	\loigiai{Ta có mặt phẳng $(\alpha)$ vuông góc với $AB$ nên có vectơ pháp tuyến là $\overrightarrow{n}=\overrightarrow{AB}=\left(0;3;1\right)$.\\
	Mà mặt phẳng $(\alpha)$ đi qua điểm $A(1;-1;0)$ nên phương trình mặt phẳng là $3y+z+3=0$.}
\end{ex}

%================================PHẦN II
\subsubsection*{Phần II. Câu trắc nghiệm đúng sai. Thí sinh trả lời từ câu 1 đến câu 4. Trong mỗi ý a), b), c), d) ở mỗi câu, thí sinh chọn đúng hoặc sai.}
\setcounter{ex}{0}
\Opensolutionfile{ans}[ans/ans-De4-phanII]

\begin{ex}%[Nguồn: Bộ đề minh họa Moon 2024-2025]%[2H5H1-5]
	Trong không gian  $Oxyz$, cho các điểm $A(0;-1; 1)$, $B(-2; 1;-1)$ và $\overrightarrow{OC}=-\overrightarrow{i}+3\overrightarrow{j}+2\overrightarrow{k}$.
	\choiceTF
	{\True Toạ độ điểm $C$ là $C(-1; 2; 3)$}
	{\True Toạ độ các vectơ $\overrightarrow{AB}=(-2; 2;-2); \overrightarrow{AC}=(-1; 4; 1)$}
	{Một vectơ pháp tuyến của mặt phẳng $(ABC)$ là $(5;-2;-3)$}
	{Khoảng cách từ gốc toạ độ $O$ đến mặt phẳng $(ABC)$ bằng $\dfrac{5}{\sqrt{33}}$}
	\loigiai{
	\begin{itemchoice}
		\itemch	Ta có $\overrightarrow{OC}=-\overrightarrow{i}+3\overrightarrow{j}+2\overrightarrow{k}$ nên tọa độ điểm $C(-1;3;2)$.
		\itemch Tọa độ $\overrightarrow{AB}=(-2;2;-2)$ và $\overrightarrow{AC}=(-1;4;1)$.
		\itemch	Một vectơ pháp tuyến của mặt phẳng $(ABC)$ là \[\overrightarrow{n}=[\overrightarrow{AB},\overrightarrow{AC}] = (10;4;-6)=2(5;2;-3).\]
		\itemch Phương trình mặt phẳng $(ABC)$ là
		\[
		5(x-0)+2(y+1)-3(z-1)=0 \Leftrightarrow 5x+2y-3z+5=0.
		\]
		Khoảng cách từ $O$ đến $(ABC)$ là
		\[
		\mathrm{d}(O,(ABC))=\dfrac{|5\cdot 0+2\cdot 0-3\cdot 0+5|}{\sqrt{5^2+2^2+(-3)^2}}=\dfrac{5}{\sqrt{38}}=\dfrac{5\sqrt{38}}{38}.
		\]
	\end{itemchoice}
	}
\end{ex}

\begin{ex}%[Nguồn: Bộ đề minh họa Moon 2024-2025]%[2H5H1-5]
	Trong không gian với hệ tọa độ $Oxyz$, khoảng cách từ điểm $M(1;0;1)$ đến mặt phẳng $(P)\colon  2x - y + 2z - 3 = 0$ bằng
	\choice
	{ $\dfrac{4}{3}$}
	{$\dfrac{2}{3}$}
	{$1$}
	{\True$\dfrac{1}{3}$}
	\loigiai{
	Khoảng cách từ điểm $M(x_0; y_0; z_0)$ đến mặt phẳng $(P)\colon  Ax + By + Cz + D = 0$ được tính bởi công thức
	$$
	\mathrm{d}(M, (P)) = \dfrac{|Ax_0 + By_0 + Cz_0 + D|}{\sqrt{A^2 + B^2 + C^2}}
	$$
	Khi đó
	$$
	\mathrm{d}(M, (P)) = \dfrac{|2 \cdot 1 - 0 + 2 \cdot 1 - 3|}{\sqrt{2^2 + (-1)^2 + 2^2}} = \dfrac{|2 - 0 + 2 - 3|}{\sqrt{4 + 1 + 4}} = \dfrac{1}{3}.
	$$
	}
\end{ex}

%

\begin{ex}%[Nguồn: Bộ đề minh họa Moon 2024-2025]%[2H5H2-3]
	Trong không gian với hệ tọa độ $Oxyz$, cho ba điểm $A(1 ; 0 ;-2)$, $B(-1 ; 2 ; 4)$, $C(2 ; 0 ; 1)$. Gọi $(P)$ là mặt phẳng đi qua $A$ và vuông góc với $BC$.
	\choiceTF
	{\True $\vec{BC}=(3 ;-2 ;-3)$}
	{Đường thẳng $BC$ có phương trình là $\dfrac{x-2}{3}=\dfrac{y}{-2}=\dfrac{z+1}{-3}$}
	{\True Phương trình của mặt phẳng $(P)$ là $3 x-2 y-3 z-9=0$}
	{Giao điểm của đường thẳng $BC$ với mặt phẳng $(P)$ có tọa độ là $I(a ; b ; c)$. Khi đó $a+b+c=0$}
	\loigiai{
	\begin{enumerate}[a)]
		\item $\vec{BC}=(3 ;-2 ;-3)$.
		\item Phương trình đường thẳng $BC$ đi qua $C(2;0;1)$ có vectơ chỉ phương $\vec{BC}=(3 ;-2 ;-3)$ là
		\[\dfrac{x-2}{3}=\dfrac{y}{-2}=\dfrac{z-1}{-3}.\]
		\item Phương trình của mặt phẳng $(P)$ đi qua $A(1 ; 0 ;-2)$ có vectơ pháp tuyến $\vec{BC}=(3 ;-2 ;-3)$ là
		\[3(x-1)-2y-3(z+2)=0\Leftrightarrow 3x-2y-3z-9=0.\]
		\item Phương trình tham số của $BC$ là $\heva{&x=2+3t\\&y=-2t\\&z=1-3t}$ $(t\in \mathbb{R})$.\\
		$I\in BC \Rightarrow I(2+3t;-2t;1-3t)$. Thay vào phương trình của $(P)$ ta có
		\begin{eqnarray*}
			&&3(2+3t)-2(-2t)-3(1-3t)-9=0\Leftrightarrow t=\dfrac{3}{11}\\
			&\Rightarrow& a+b+c=3-2t=3-2\cdot \dfrac{3}{11}=\dfrac{27}{11}.
		\end{eqnarray*}
	\end{enumerate}}
\end{ex}

\begin{ex}%[Nguồn: Bộ đề minh họa Moon 2024-2025]%[2H5H2-3]
	Trong không gian $Oxyz$, cho mặt phẳng $(P)\colon 2x - y - 2z + 3 = 0$. Đường thẳng $\Delta$ đi qua điểm $M(4; 1; -3)$ và vuông góc với $(P)$ có phương trình chính tắc là
	\choice
	{$\dfrac{x+4}{2} = \dfrac{y+1}{-1} = \dfrac{z-3}{-2}$}
	{$\dfrac{x-2}{4} = \dfrac{y+1}{1} = \dfrac{z+2}{-3}$}
	{$\dfrac{x+2}{2} = \dfrac{y+2}{1} = \dfrac{z-3}{-2}$}
	{\True $\dfrac{x-4}{2} = \dfrac{y-1}{-1} = \dfrac{z+3}{-2}$}
	\loigiai{
	Đường thẳng $\Delta$ qua điểm $M(4; 1; -3)$ và vuông góc với $(P)\colon 2x-y-2z+3=0$ nên có véc-tơ chỉ phương là $\overrightarrow{u}=(2; -1; -2)$ từ đó có phương trình chính tắc là \[\dfrac{x-4}{2} = \dfrac{y-1}{-1} = \dfrac{z+3}{-2}.\]
	}
\end{ex}

%

\begin{ex}%[Nguồn: Bộ đề minh họa Moon 2024-2025]%[2H5H2-3]
	Trong không gian $Oxyz$, cho mặt phẳng $(P)\colon 2x-3y+6z-5=0$ và điểm $A(2;-3; 1)$. Đường thẳng $d$ đi qua điểm $A$ và vuông góc với mặt phẳng $(P)$ có phương trình tham số là
	\choice
	{$d \colon\heva{&x=2+2t \\ &y=-3-3t\\ &z=1-5t}$}
	{$d \colon\heva{&x=2+2t \\ &y=-3-3t\\ &z=6+t}$}
	{\True $d \colon\heva{&x=2+2t \\ &y=-3-3t \\  &z=1+6t}$}
	{$d \colon\heva{&x=-2+2t \\ &y=3-3t \\ & z=-1+6t}$}
	\loigiai{
	Ta có đường thẳng $d\perp (P)$  nên có vectơ chỉ phương là $\overrightarrow{u}_d=\overrightarrow{n}_{(P)}=(2;-3;6)$.\\
	Đường thẳng $d$ đi qua $A$ và có vectơ chỉ phương là $\overrightarrow{u}_d==(2;-3;6)$ có phương trình tham số là
	\[\heva{&x=2+2t\\&y=-3-3t\\&z=1+6t.}\]
	}
\end{ex}

\begin{ex}%[Nguồn: Bộ đề minh họa Moon 2024-2025]%[2H5H2-3]
	Trong không gian với hệ tọa độ $Oxyz$, đường thẳng đi qua hai điểm $A(3;1;-6)$, $B(5;3;-2)$ có phương trình tham số là
	\choice
	{\True$ \heva{&x=6+t\\& y=4+t\\& z=2t} $}
	{$ \heva{&x=5+2t\\ &y=3+2t\\ &z=-2-4t} $}
	{ $ \heva{&x=3+t\\& y=1+t\\& z=-6-2t} $}
	{$ \heva{&x=6+2t\\ &y=4+2t\\ &z=-1+4t} $}
	\loigiai{
	Ta có $\overrightarrow{AB} = (2; 2; 4)$. Khi đó véc-tơ chỉ phương của đường thẳng $AB$ là $\overrightarrow{n_{AB}}=(1;1;2)$.\\
	Phương trình tham số của đường thẳng đi qua $A(3;1;-6)$ và có véc-tơ chỉ phương $\overrightarrow{AB}$ là
	$ \heva{&x=3+t\\& y=1+t\\& z=-6+2t.} $
	}
\end{ex}

\begin{ex}%[Nguồn: Bộ đề minh họa Moon 2024-2025]%[2H5H2-7]
	Trong không gian với hệ tọa độ $Oxyz$, cho mặt phẳng $(P)\colon 2x+y+2z-1=0$ và đường thẳng $d\colon\dfrac{x-1}{1}=\dfrac{y+2}{-2}=\dfrac{z-2}{-2}$. Gọi $\alpha$ là góc giữa đường thẳng $d$ và mặt phẳng $(P)$.
	\choiceTFt
	{Một véc-tơ pháp tuyến của mặt phẳng $(P)$ là $(2;2;1)$}
	{\True Một véc-tơ chỉ phương của đường thẳng $d$ là $(-1;2;2)$}
	{Tích vô hướng của hai véc-tơ $\overrightarrow{n}(2;1;2)$ và $\overrightarrow{u}(1;-2;-2)$ bằng $4$}
	{Góc giữa đường thẳng $d$ và mặt phẳng $(P)$ là $60^\circ$ (làm tròn kết quả đến hàng đơn vị của độ)}
	\loigiai{
	\begin{itemchoice}
		\itemch Mặt phẳng $(P)\colon 2x+y+2z-1=0$ có véc-tơ pháp tuyến là $\overrightarrow{n}=(2;1;2)$.
		\itemch Đường thẳng $d\colon\dfrac{x-1}{1}=\dfrac{y+2}{-2}=\dfrac{z-2}{-2}$ có véc-tơ chỉ phương là $\overrightarrow{u}=(1;-2;-2)$, do đó véc-tơ $(-1;2;2)$ cũng là véc-tơ chỉ phương của $d$.
		\itemch Ta có $\overrightarrow{n}=(2;1;2)$ và $\overrightarrow{u}=(1;-2;-2)$ nên $\overrightarrow{n} \cdot \overrightarrow{u} = 2\cdot 1 + 1\cdot (-2) + 2\cdot (-2) = 2 - 2 - 4 = -4$.
		\itemch Ta có $\sin\left(d,(P)\right)=\cos \left(\overrightarrow{n},\overrightarrow{u}\right) = \dfrac{|\overrightarrow{n} \cdot \overrightarrow{u}|}{|\overrightarrow{n}| \cdot |\overrightarrow{u}|} = \dfrac{|-4|}{\sqrt{2^2+1^2+2^2}\cdot \sqrt{1^2+(-2)^2+(-2)^2}} = \dfrac{4}{3 \cdot 3} = \dfrac{4}{9}$.\\
		Suy ra $\left(d,(P)\right)=29^{\circ}$.
	\end{itemchoice}
	}
\end{ex}

%Điền đáp án 1

\begin{bt}%[Nguồn: Bộ đề minh họa Moon 2024-2025]%[2H5H2-7]
	Cho hình lăng trụ đứng $ABC.A'B'C'$ có đáy $ABC$ là tam giác vuông tại $C$, $AC=3a$, $BC=4a$ và góc giữa đường thẳng $B'C$ và mặt phẳng $(ABC)$ bằng $45^\circ$. Tính sin của góc giữa đường thẳng $B'C$ và mặt phẳng $(ABC')$ (làm tròn kết quả đến hàng phần trăm).
	\par\shortans{$0{,}73$}
	\loigiai{
	Ta có $CC'\perp (ABC)$ nên $C$ là hình chiếu của $C'$ trên $(ABC)$.\\
	Khi đó $\left(BC',(ABC)\right)=\left(BC',BC\right)=\widehat{C'BC}=45^\circ$.\\
	Xét $\triangle C'BC$ vuông tại $C$, ta có $CC'=BC\tan\widehat{C'BC}=4a\cdot\tan 45^\circ=4a$.\\
	Chọn hệ trục tọa độ $Oxyz$, với $C \equiv O$, $A\in Ox$, $B\in Oy$ và $C'\in Oz$, đơn vị trên các trục là $a$.
	\begin{center}
		\tdplotsetmaincoords{75}{110}
		\begin{tikzpicture}[font=\footnotesize, >=stealth, tdplot_main_coords]
			\path
			(0,0,0) coordinate (C)+(0,0,4) coordinate (C')
			(3,0,0) coordinate (A)+(0,0,4) coordinate (A')
			(0,4,0) coordinate (B)+(0,0,4) coordinate (B')
			;
			\draw (A')--(C')--(B')--cycle (A)--(A') (B)--(B') (A)--(B);
			\draw[dashed] (A)--(C)--(B) (C)--(C') (B')--(C) (A)--(C')--(B);
			\draw[->] (A)--++(3,0,0)node[below]{$x$};
			\draw[->] (B)--++(0,1,0)node[right]{$y$};
			\draw[->] (C')--++(0,0,1)node[left]{$z$};
			\foreach \x/\g in {C/left, A/below, B/below, C'/left, A'/left, B'/right}{
			\fill (\x) circle (1pt)node[\g]{$\x$};
			}
		\end{tikzpicture}
	\end{center}
	Khi đó tọa độ các điểm là $C(0;0;0)$, $A(3;0;0)$, $B(0;4;0)$, $C'(0;0;4)$, $A'(3;0;4)$, $B'(4;0;4)$.\\
	Ta có $\overrightarrow{B'C}=(-4;0;-4)$.\\
	Mặt phẳng $(ABC')$ có $\overrightarrow{AB}=(-3;4;0)$, $\overrightarrow{AC'}=(-3;0;4)$ là cặp vectơ chỉ phương.\\
	Do đó $\overrightarrow{n}=\left[\overrightarrow{AB},\overrightarrow{AC'}\right]=(16;12;12)$ là một vectơ pháp tuyến của $(ABC')$.\\
	Chọn $\overrightarrow{n}'=\dfrac{1}{4}\overrightarrow{n}=(4;3;3)$ là vectơ pháp tuyến của $(ABC')$.\\
	Khi đó
	$$\sin\left(BC',(ABC')\right)
	=\left|\cos\left(\overrightarrow{BC'},\overrightarrow{n}'\right)\right|
	=\dfrac{|0\cdot4+(-4)\cdot3+(-4)\cdot3|}{\sqrt{0^2+(-4)^2+(-4)^2}\cdot\sqrt{4^2+3^2+3^2}}
	=\dfrac{3}{\sqrt{17}} \approx 0{,}73.$$
	}
\end{bt}

\begin{ex}%[Nguồn: Bộ đề minh họa Moon 2024-2025]%[2H5H2-7]
	\immini{Trong không gian hệ trục tọa độ $Oxyz$ (đơn vị trên mỗi trục là kilômét), đài kiểm soát không lưu của một sân bay ở vị trí $O(0;0;0)$ và được thiết kế phát hiện máy bay ở khoảng cách tối đa $417$ km. Một máy bay đang chuyển động theo đường thẳng $d$ từ điểm $A(-688;-185;8)$ đến điểm $D(222;565;8)$ và hướng về đài kiểm soát không lưu (như hình vẽ).}
	{	\begin{tikzpicture}[scale=.5]
		\def\r{3}
		\def \x{3}  %bán kính trục lớn elip
		\def \y{1.2}
		\coordinate (o1) at (-\r,0);
		\coordinate (o2) at (\r,0);
		\coordinate (O) at (0,0);
		
		% ve mp
		\coordinate (P) at (-5,2);
		\coordinate (Q) at ($(P)+(240:6)$);
		\coordinate (R) at ($(Q)+(12,0)$);
		\coordinate (S) at ($(R)+(60:6)$);
		\fill[cyan!20] (P)--(Q)--(R)--(S)--cycle;
		
		\draw (O) circle(\r);
		\path[fill=orange!20] (O) circle(\r);
		\fill[orange!80] (o2) arc (0:-180:\x cm and \y cm) arc (180:360:\x cm and \x cm);
		\draw[dashed] (o2) arc (0:180:\x cm and \y cm);
		\draw (o2) arc (0:-180:\x cm and \y cm);
		
		% truc oz
		\coordinate (Oz) at (0,4.5);
		\coordinate (Oz') at (0,2.5);
		\coordinate (Oz'') at (0,-3.5);
		\draw[dashed] (Oz'')--(Oz');
		\draw[->] (Oz')--(Oz);
		% tru  ox
		\def\t{240}
		\coordinate (Ox) at ({\x*cos(\t)},{\y*sin(\t)});
		\coordinate (Ox') at ($(O)!2!(Ox)$);
		\coordinate (Ox'') at ($(Ox)!2.2!(O)$);
		\draw[dashed] (O)--(Ox);
		\draw[->] (Ox)--(Ox');
		\draw[dashed] (O)--(Ox'');
		% truc oy
		\coordinate (O4) at (\r,0);
		\coordinate (O5) at (1.5*\r,0);
		\coordinate (O2) at (-\r,0);
		\coordinate (O1) at (-1.5*\r,0);
		\draw (O1)--(O2);
		\draw[dashed] (O2)--(O4);
		\draw[->] (O4)--(O5);
		% duong d
		\coordinate (A) at (-4.52,1.7);
		\coordinate (A3) at (4.5,0.3);
		\coordinate (A1) at ($(A)!0.25!(A3)$);
		\coordinate (A4) at ($(A)!0.7!(A3)$);
		\coordinate (A2) at ($(A)!0.82!(A3)$);
		\coordinate (A2) at ($(A)!0.82!(A3)$);
		\draw (A)--(A1);
		\draw[dashed] (A1)--(A2);
		\draw (A2)--(A3);
		\foreach \x in {O}{
		\fill (\x) circle(1pt);
		}
		\path
		node at (O) [above left]{$O$}
		node at (A) [above] {$A$}
		node at (A4) [below] {$d$}
		node at (A1) [above right]{B}
		node at (A2) [above right]{$C$}
		node at (A3) [above]{$D$}
		node at (Oz) [left]{$z$}
		node at (O5) [below]{$y$}
		node at (Ox')[below]{$x$}
		;
	\end{tikzpicture}
	}
	\choiceTF
	{Phương trình đường thẳng $d\colon\heva{&x = -688 + 91t \\&y = -185 + 75t \\&z = 8},\,(t \in \mathbb{R})$}
	{\True Tọa độ vị trí sớm nhất máy bay xuất hiện trên radar là $B(-415; 40; 8)$}
	{Khoảng cách ngắn nhất giữa máy bay và đài kiểm soát là $590$ km}
	{Biết vận tốc máy bay là $900$ km/h thì thời gian di chuyển của máy bay trong phạm vi theo dõi của hệ thống quan sát là $19{,}7$ phút (làm tròn kết quả đến hàng phần mười)}
	\loigiai{
	\begin{itemchoice}
		\itemch Đường thẳng $d$ qua $A(-688; -185; 8)$ và có vectơ chỉ phương là $\overrightarrow{AD} = (910; 750; 0)$, nên có phương trình tham số đã cho là
		$\heva{&x = -688 + 910t \\&y = -185 + 750t \\&z = 8},\, (t \in \mathbb{R}).$
		\itemch Phương trình mặt cầu $(S)$ có tâm $O$, bán kính $R=417$ là $x^2+y^2+z^2=173\,889$.\\
		Do $B=d\cap (S)$ nên tọa độ $B$ thỏa
		\[\heva{&x^2+y^2+z^2=173\,889\quad (1)\\&x=-688+910t\quad(2)\\&y=-185+750t\quad (3)\\&z=8\quad (4)}\]
		Thay (2), (3) và (4) vào (1), ta được
		\allowdisplaybreaks
		\begin{eqnarray*}
			&&(-688+910t)^2+(-185+750t)^2+8^2=173\,889\\ &\Leftrightarrow&1\,390\,600t^2-1\,529\,660t+33\,3744=0
			\Leftrightarrow\hoac{&t=\dfrac{4}{5}\\&t=\dfrac{3}{10}.}
		\end{eqnarray*}
		\begin{itemize}
			\item Với $t=\dfrac{4}{5}$ thì $B_1(40;415;8)$.
			\item  Với $t=\dfrac{3}{10}$ thì $B_2(-415;40;8)$.
		\end{itemize}
		Ta thấy $\overrightarrow{AB_1}$ và $\overrightarrow{B_1B_2}$ ngược hướng nên điểm $B(-415;40;8)$ và $C(40;415;8)$.
		\itemch Khoảng cách nhắn nhất giữa máy bay và đài kiểm soát bằng với khoảng cách từ điểm $O$ đến đường thẳng $d$ và cho bởi
		\[\mathrm{d}[O,d]=\dfrac{\big|\left[\overrightarrow{AD},\overrightarrow{OA}\right]\big|}{\big|\overrightarrow{AD}\big|}\approx 295 \,\text{km}.\]
		
		\itemch Quãng đường máy bay di chuyển trong phạm vi theo dõi là $s=BC=\sqrt{347\,650}$.\\ Nên thời gian di chuyển của máy bay trong phạm vi theo dõi là\\ $$t=\dfrac{s}{v}=\dfrac{\sqrt{347\,650}}{900}=0{,}65513.. \text{giờ}  \approx39{,}3 \,\text{phút}.$$
	\end{itemchoice}
	}
\end{ex}

\begin{ex}%[Nguồn: Bộ đề minh họa Moon 2024-2025]%[2H5H3-2]
	Trong không gian $Oxy	z$, cho mặt cầu $(S)\colon x^2 + y^2 + z^2 - 4x + 2y - 2z - 3 = 0$. Tìm tọa độ tâm $I$ và bán kính $R$ của $(S)$.
	\choice
	{\True $I(2; -1; 1)$ và $R = 3$}
	{$I(-2; 1; -1)$ và $R = 3$}
	{$I(2; -1; 1)$ và $R = 9$}
	{$I(-2; 1; -1)$ và $R = 9$}
	\loigiai{
	Ta có $x^2 + y^2 + z^2 - 4x + 2y - 2z - 3 = 0\Leftrightarrow (x-2)^2+(y+1)^2+(z-1)^2=9$.\\
	Do đó đường mặt cầu có tâm $I(2; -1; 1)$ và bán kính $R = 3$.
	}
\end{ex}

\begin{ex}%[Nguồn: Bộ đề minh họa Moon 2024-2025]%[2H5H3-2]
	Trong không gian $Oxyz$, cho mặt cầu $\left(S \right) \colon x^2 + y^2 + z^2 - 2x + 2y - 4z = 0$. Tâm của mặt cầu $\left(S \right)$ có toạ độ là
	\choice
	{$\left(2; -2; -4 \right)$}
	{\True $\left(1; -1; 2 \right)$}
	{$\left(-2; 2; 4\right)$}
	{$\left(-1; 1; -2 \right)$}
	\loigiai
	{Mặt cầu $\left(S \right)$ có toạ độ tâm là $\left(1; -1; 2 \right)$.
	}
\end{ex}

\begin{ex}%[Nguồn: Bộ đề minh họa Moon 2024-2025]%[2H5H3-3]
	Trong không gian $Oxyz$, cho hai điểm $A(2;1;1)$ và $B(0;3;-1)$. Mặt cầu đường kính $AB$ có phương trình là
	\choice
	{$(x-1)^2+(y-2)^2+z^2=9$}
	{$x^2+(y-2)^2+z^2=3$}
	{\True$(x-1)^2+(y-2)^2+z^2=3$}
	{$(x-1)^2+(y-2)^2+(z+1)^2=9$}
	\loigiai{
	Tâm của mặt cầu là trung điểm $I$ của $AB$. Tọa độ tâm $I(1;2;0)$.\\
	Bán kính của mặt cầu $R=IA=\sqrt{(2-1)^2+(1-2)^2+(1-0)^2}=\sqrt{3}$.\\
	Phương trình mặt cầu đường kính $AB$ là $(x-1)^2+(y-2)^2+z^2=3$.
	}
\end{ex}

\begin{ex}%[Nguồn: Bộ đề minh họa Moon 2024-2025]%[2H5H3-3]
	Trong không gian với hệ tọa độ $Oxyz$, cho hai điểm $A(1;-2;3)$ và $B(5;4;7)$. Phương trình mặt cầu đường kính $AB$ là
	\choice
	{$\left(x-1\right)^2+\left(y+2\right)^2+\left(z-3\right)^2=17$}
	{\True $\left(x-3\right)^2+\left(y-1\right)^2+\left(z-5\right)^2=17$}
	{$\left(x-5\right)^2+\left(y-4\right)^2+\left(z-7\right)^2=17$}
	{$\left(x-6\right)^2+\left(y-2\right)^2+\left(z-10\right)^2=17$}
	\loigiai{Mặt cầu đường kính $AB$ nên tâm mặt cầu là trung điểm $I\left(3;1;5\right)$ của đoạn thẳng $AB$.\\
	Bán kính mặt cầu $R=IA=\sqrt{(1-3)^2+(-2-1)^2+(3-5)^2}=\sqrt{17}$.\\
	Phương trình mặt cầu là $\left(x-3\right)^2+\left(y-1\right)^2+\left(z-5\right)^2=17$.
	}
\end{ex}

%

\begin{ex}%[Nguồn: Bộ đề minh họa Moon 2024-2025]%[2H5H3-3]
	Trong không gian $Oxyz$, mặt cầu $(S)$ có tâm $I(1; 2;-1)$ và tiếp xúc với mặt phẳng $(P)\colon 2x-2y-z-8=0$ có phương trình là
	\choice
	{$(S)\colon (x+1)^2+(y+2)^2+(z-1)^2=9$}
	{$(S)\colon (x+1)^2+(y+2)^2+(z-1)^2=3$}
	{$(S)\colon (x-1)^2+(y-2)^2+(z+1)^2=3$}
	{\True $(S)\colon (x-1)^2+(y-2)^2+(z+1)^2=9$}
	\loigiai{
	Mặt cầu $(S)$ có tâm $I(1; 2;-1)$ và tiếp xúc với mặt phẳng $(P)\colon 2x-2y-z-8=0$ nên có bán kính $R=\mathrm{d}\big(I,(P)\big)=\dfrac{|2.1-2.2-(-1)-8|}{4+4+1}=3$.\\
	Do đó có phương trình $(S)\colon (x-1)^2+(y-2)^2+(z+1)^2=9$.
	}
\end{ex}

%

\begin{ex}%[Nguồn: Bộ đề minh họa Moon 2024-2025]%[2H5N1-2]
	Trong không gian $Oxyz$, cho mặt phẳng $(P)\colon 2x-3y+6z-5=0$ và điểm $A(2;-3; 1)$. Một vectơ pháp tuyến của mặt phẳng $(P)$ là
	\choice
	{$\overrightarrow{n}_1=(2;-3;-5)$}
	{\True $\overrightarrow{n}_2=(2;-3; 6)$}
	{$\overrightarrow{n}_3=(2; 3;-5)$}
	{$\overrightarrow{n}_4=(2;-3; 5)$}
	\loigiai{
	Một vectơ pháp tuyến của mặt phẳng $(P)$ là $\overrightarrow{n}=(2;-3;6)$.
	}
\end{ex}

\begin{ex}%[Nguồn: Bộ đề minh họa Moon 2024-2025]%[2H5N1-2]
	Trong không gian $Oxyz$, cho mặt phẳng $\left(\alpha\right)\colon 2x-y+3z+5=0$. Véc-tơ nào dưới đây là một véc-tơ pháp tuyến của $\left(\alpha\right)$?
	\choice
	{$\overrightarrow{n}_3\left(-2;1;3\right)$}
	{$\overrightarrow{n}_4\left(2;1;-3\right)$}
	{\True $\overrightarrow{n}_2\left(2;-1;3\right)$}
	{$\overrightarrow{n}_1\left(2;1;3\right)$}
	\loigiai{
	Một véc-tơ pháp tuyến của mặt phẳng $\left(\alpha\right)$ là $\overrightarrow{n}_2\left(2;-1;3\right)$.
	}
\end{ex}

\begin{ex}%[Nguồn: Bộ đề minh họa Moon 2024-2025]%[2H5N1-3]
	Phương trình nào dưới đây là phương trình của mặt phẳng $(ABC)$?
	\choice
	{$\dfrac{x}{1}+\dfrac{y}{3}+\dfrac{z}{4}=1$}
	{$\dfrac{x}{1}-\dfrac{y}{3}-\dfrac{z}{4}=1$}
	{$\dfrac{x}{4}+\dfrac{y}{3}+\dfrac{z}{-1}=1$}
	{\True $\dfrac{x}{1}-\dfrac{y}{3}-\dfrac{z}{4}=-1$}
	\loigiai{Mặt phẳng $(ABC)$ đi qua ba điểm $A(-1; 0; 0)$, $B(0; 3; 0)$, $C(0; 0; 4)$ là $(A B C)\colon \dfrac{x}{-1}+\dfrac{y}{3}+\dfrac{z}{4}=1\Leftrightarrow \dfrac{x}{1}-\dfrac{y}{3}-\dfrac{z}{4}=-1$. }
\end{ex}

\begin{ex}%[Nguồn: Bộ đề minh họa Moon 2024-2025]%[2H5N2-1]
	Trong không gian $Oxyz$, cho đường thẳng $d\colon  \dfrac{x-2}{2}=\dfrac{y+1}{-1}=\dfrac{z+2}{2}$. Đường thẳng $d$ đi qua điểm nào dưới đây?
	\choice
	{$A(2;-1; 2)$}
	{$B(-2; 1; 2)$}
	{$C(2; 1; 2)$}
	{\True $D(2;-1;-2)$}
	\loigiai{
	Ta có
	\begin{itemize}
		\item$ \dfrac{2-2}{2}=\dfrac{-1+1}{-1}\neq\dfrac{2+2}{2}\Rightarrow A\notin d$.
		\item$ \dfrac{-2-2}{2}=\dfrac{1+1}{-1}\neq\dfrac{2+2}{2}\Rightarrow B\notin d$.
		\item$ \dfrac{2-2}{2}\neq\dfrac{1+1}{-1}\neq\dfrac{2+2}{2}\Rightarrow C\notin d$.
		\item$ \dfrac{2-2}{2}=\dfrac{-1+1}{-1}=\dfrac{-2+2}{2}\Rightarrow D\in d$.
	\end{itemize}
	Vậy $d$ qua điểm $D$.
	}
\end{ex}

%

\begin{ex}%[Nguồn: Bộ đề minh họa Moon 2024-2025]%[2H5N2-2]
	Trong không gian với hệ trục tọa độ $Oxyz$, phương trình của đường thẳng đi qua điểm $M(1;-3; 5)$ và có một véc-tơ chỉ phương $\overrightarrow{u}(2;-1; 1)$ là
	\choice
	{$\dfrac{x-1}{2}=\dfrac{y-3}{-1}=\dfrac{z-5}{1}$}
	{$\dfrac{x-1}{2}=\dfrac{y-3}{-1}=\dfrac{z+5}{1}$}
	{\True $\dfrac{x-1}{2}=\dfrac{y+3}{-1}=\dfrac{z-5}{1}$}
	{$\dfrac{x+1}{2}=\dfrac{y+3}{-1}=\dfrac{z-5}{1}$}
	\loigiai{
	Phương trình của đường thẳng đi qua điểm $M(1 ;-3 ; 5)$ và có một véc-tơ chỉ phương $\overrightarrow{u}=(2 ;-1 ; 1)$ là $\dfrac{x-1}{2}=\dfrac{y+3}{-1}=\dfrac{z-5}{1}$.
	}
\end{ex}

%

\begin{ex}%[Nguồn: Bộ đề minh họa Moon 2024-2025]%[2H5N2-2]
	Trong không gian với hệ tọa độ $Oxyz$, đường thẳng nào sau đây có vectơ chỉ phương là $\vec{u}=(2; 3;-1)$?
	\choice
	{\True $\heva{&x=1-4t\\&y=2-6t\\&z=-1+2t}$}
	{$\heva{&x=1+4t\\&y=2+6t\\&z=-1-4t}$}
	{$\heva{&x=1-2t\\&y=2-3t\\&z=-1-t}$}
	{$\heva{&x=1+2t\\&y=2-3t\\&z=-1-t}$}
	\loigiai{Đườngthẳng $\heva{&x=1-4t\\&y=2-6t\\&z=-1+2t}$ có một vectơ chỉ phương là $\vec{a}=(-4;-6;2)=-2(2; 3;-1)$.
	}
\end{ex}

\begin{ex}%[Nguồn: Bộ đề minh họa Moon 2024-2025]%[2H5N2-3]
	Cho đường thẳng $\Delta$ đi qua điểm $M(2;0;-1)$ và vectơ chỉ phương là $\overrightarrow{a}=(4;-6;2)$. Phương trình tham số của đường thẳng $\Delta$ là
	\choice
	{$\heva{&x=-2+4t\\&y=-6t\\&z=1+2t} (t\in\mathbb{R})$}
	{$\heva{&x=-2+2t\\&y=-3t\\&z=1+t} (t\in\mathbb{R})$}
	{\True $\heva{&x=2+2t\\&y=-3t\\&z=-1+t} (t\in\mathbb{R})$}
	{$\heva{&x=-2+2t\\&y=-3t\\&z=-1-t} (t\in\mathbb{R})$}
	\loigiai{
	Vectơ chỉ phương $\overrightarrow{a}=(2;-3;1)$ nên $\Delta\colon\heva{&x=2+2t\\&y=-3t\\&z=-1+t} (t\in\mathbb{R})$.
	}
\end{ex}

\begin{ex}%[Nguồn: Bộ đề minh họa Moon 2024-2025]%[2H5N2-3]
	Phương trình tham số của đường thẳng $AB$ là
	\choice
	{$\heva{&x=-1+t \\&y=3t \\&z=t} \quad(t \in \mathbb{R})$}
	{$\heva{&x=1+t \\ &y=3t \\ &z=0} \quad(t \in \mathbb{R})$}
	{$\heva{&x=-1-t \\&y=3t \\&z=0} \quad(t \in \mathbb{R})$}
	{\True $\heva{&x=-1+t\\&y=3t\\&z=0} \quad(t \in \mathbb{R})$}
	\loigiai{Phương trình tham số của đường thẳng $AB$ nhận $\overrightarrow{A B}=(1 ; 3 ; 0)$ làm véc tơ chỉ phương và đi qua điểm $A(-1;0;0)$ là
	$
	\heva{
	&x=-1+t \\
	&y=3 t \\
	&z=0
	} \quad(t \in \mathbb{R})$.}
\end{ex}

%

\begin{ex}%[Nguồn: Bộ đề minh họa Moon 2024-2025]%[2H5N2-3]
	Trong không gian $Oxyz$, đường thẳng đi qua hai điểm $P(1 ;-1 ; 2)$ và $Q(2 ; 0 ; 1)$ có phương trình tham số là
	\choice
	{\True $\heva{&x=2+t \\ &y=t \\ &z=1-t}$}
	{$\heva{&x=1+t \\ &y=-1+t \\ &z=2+t}$}
	{$\heva{&x=2-t \\ &y=t \\ &z=1+t}$}
	{$\heva{&x=1+t \\ &y=-1-t \\ &z=2+t}$}
	\loigiai{
	Phương trình đường thẳng đi qua điểm $Q(2 ; 0 ; 1)$ có vectơ chỉ phương $\vec{PQ}=(1;1;-1)$ là
	\[\heva{&x=2+t \\ &y=t \\ &z=1-t}\quad (t\in\mathbb{R}).\]
	}
\end{ex}

\begin{ex}%[Nguồn: Bộ đề minh họa Moon 2024-2025]%[2H5N2-5]
	Trong không gian $Oxyz$, mặt phẳng $(\alpha)\colon x+2y+3z-6=0$ cắt trục tung tại điểm có tung độ bằng
	\choice
	{$2$}
	{$6$}
	{\True $3$}
	{$1$}
	\loigiai{
	Với $x=0$, $z=0$ suy ra $0+2y+3\cdot 0-6=0\Rightarrow y=3$.\\
	Vậy $(\alpha)\colon x+2y+3z-6=0$ cắt trục tung tại điểm có tung độ bằng $3$.
	}
\end{ex}

%Câu trắc nghiệm 11

\begin{ex}%[Nguồn: Bộ đề minh họa Moon 2024-2025]%[2H5N2-5]
	Trong không gian $Oxyz$, đường thẳng $\Delta\colon \dfrac{x-1}{1}=\dfrac{y-1}{-1}=\dfrac{z}{-1}$ song song với mặt phẳng nào sau đây?
	\choice
	{$(P)\colon x+y-z=0$}
	{\True $(\beta)\colon x+z=0$}
	{$(Q)\colon x+y+2z=0$}
	{$(\alpha)\colon x-y+1=0$}
	\loigiai{
	Đường thẳng $\Delta$ có $\overrightarrow{u}=(1;-1;-1)$ là một vectơ chỉ phương và $M(1;1;0)\in\Delta$.\\
	Mặt phẳng $(\beta)$ có $\overrightarrow{n}_{(\beta)}=(1;0;1)$ là một vectơ pháp tuyến.\\
	Ta có $\overrightarrow{u}\cdot\overrightarrow{n}_{(\beta)}=0$ nên $\overrightarrow{u}\perp\overrightarrow{n}_{(\beta)}$.\\
	Đồng thời điểm $M(1;1;0)\notin(\beta)$.\\
	Vậy $\Delta\parallel(\beta)$.
	}
\end{ex}

\begin{ex}%[Nguồn: Bộ đề minh họa Moon 2024-2025]%[2H5N3-2]
	Trong không gian $Oxyz$, cho mặt cầu $(S)\colon(x-2)^2+(y+1)^2+(z-3)^2=4$. Tâm của $(S)$ có tọa độ là
	\choice
	{$(-4;2;-6)$}
	{$(-2;1;-3)$}
	{\True $(2;-1;3)$}
	{$(4;-2;6)$}
	\loigiai
	{
	Tâm của $(S)$ có tọa độ là $(2,-1,3)$.
	}
\end{ex}

\begin{ex}%[Nguồn: Bộ đề minh họa Moon 2024-2025]%[2H5N3-2]
	Trong không gian $Oxyz$, cho mặt cầu $(S)\colon(x-1)^{2}+(y-2)^{2}+(z+1)^{2}=4$. Tọa độ tâm $I$ và bán kính $R$ của mặt cầu $(S)$ là
	\choice
	{\True $I(1;2;-1)$ và $R=2$}
	{$I(1;2;-1)$ và $R=4$}
	{$I(-1;-2;1)$ và $R=4$}
	{$I(-1;-2;1)$ và $R=2$}
	\loigiai{Theo định nghĩa phương trình mặt cầu, tọa độ tâm $I$ và bán kính $R$ của mặt cầu $(S)$ là $I(1;2;-1)$ và $R=\sqrt{4}=2$.}
\end{ex}

\begin{ex}%[Nguồn: Bộ đề minh họa Moon 2024-2025]%[2H5N3-2]
	Trong không gian $Oxyz$, mặt cầu $(S)\colon x^2+y^2+z^2-2x+4y+2z-3=0$ có bán kính bằng
	\choice
	{$3\sqrt{3}$}
	{\True $3$}
	{$\sqrt{3}$}
	{$9$}
	\loigiai{
	Mặt cầu $(S)$ có bán kính $R=\sqrt{1^2+(-2)^2+(-1)^2+3}=3$.
	}
\end{ex}

\begin{ex}%[Nguồn: Bộ đề minh họa Moon 2024-2025]%[2H5N3-2]
	Trong không gian $Oxyz$ cho mặt cầu $(S)\colon x^2+y^2+z^2+8x-4y+10z-4=0$. Khi đó $(S)$ có tâm $I$ và bán kính $R$ lần lượt là
	\choice
	{\True $I(-4;2;-5)$, $R=7$}
	{$I(-4;2;-5)$, $R=4$}
	{$I(-4;2;-5)$, $R=49$}
	{$I(-4;2;5)$, $R=7$}
	\loigiai{
	Mặt cầu $(S)$ có tâm $I(-4;2;-5)$ và có bán kính là $R=\sqrt{(-4)^2+2^2+(-5)^2-(-4)}=7$.
	}
\end{ex}

%

\begin{ex}%[Nguồn: Bộ đề minh họa Moon 2024-2025]%[2H5N3-3]
	Trong không gian $Oxyz,$ phương trình mặt cầu có tâm $O$ và đi qua điểm $M(1;2;-2)$ là
	\choice
	{\True $x^2 + y^2 + z^2 = 9$}
	{$x^2 + y^2 + z^2 = 3$}
	{$x^2 + y^2 + z^2 = 0$}
	{$x^2 + y^2 + z^2 = 1$}
	\loigiai{Mặt cầu $(S)$ có tâm $O$ và đi qua $I$ nên $R=OI=\sqrt{1^2+2^2+(-2)^2}=3$.\\
	Vậy phương trình mặt cầu $(S)$ là $x^2+y^2+z^2=9$.}
\end{ex}

\begin{ex}%[Nguồn: Bộ đề minh họa Moon 2024-2025]%[2H5V1-2]
	Trong không gian $Oxyz$, cho hình lăng trụ tam giác đều $A_1B_1C_1$ có $A_1(\sqrt{3};-1;1)$, hai đỉnh $B$, $C$ thuộc trục $Oz$ và $AA_1=1$ ($C$ không trùng với $O$). Biết $\vec{u}=(a;b;10)$ là một vectơ chỉ phương của đường thẳng $A_1C$. Giá trị của $a^2 + b^2$ bằng bao nhiêu?
	\shortans[oly]{400}
	\loigiai{
	\begin{center}
		\begin{tikzpicture}[scale=0.8, font=\footnotesize,line join=round, line cap=round, >=stealth]
			\path
			(0,0) coordinate (A)
			++(-120:2) coordinate (B)
			(3,0) coordinate (C)
			($(B)!.5!(C)$)coordinate (M)
			;
			\foreach \i in{A,B,C}{
			\coordinate (\i_1) at ($(\i)+(0,3)$);
			};
			\draw (B)--(B_1) (A)--(B)--(C)--(C_1) (A_1)--(B_1)--(C_1)--(A_1);
			\draw[dashed] (A)--(C) (A)--(M)--(A_1)--(A);
			\foreach \i/\g in {A/-180,B/-90,C/-90,A_1/90,B_1/120,C_1/90,M/-90}
			\fill[black] (\i) circle(1pt)+(\g:4mm)node[scale=1]{$\i$};
		\end{tikzpicture}
	\end{center}
	Gọi $M$ là trung điểm $BC$ khi đó $AM$ vuông góc với $BC$.\\
	Ta có $\heva{&AA_1\perp BC\\&AM\perp BC}\Rightarrow BC\perp (AA_1M)$.\\
	Mặt phẳng $(A_1AM)$ qua $A_1$ và nhận $\overrightarrow{k}=(0;0;1)$ làm VTPT nên $(A_1AM)\colon z-1=0$.\\
	Mà $M=(A_1AM)\cap Oz$ nên $M(0;0;1)$ suy ra $A_1M=2$.\\
	Trong tam giác $A_1AM$ có $AM=\sqrt{A_1M^2-AA_1^2}=\sqrt{3}$.\\
	Ta có tam giác $ABC$ đều nên $AM=\dfrac{BC\sqrt{3}}{2}\Rightarrow BC=\dfrac{2AM}{\sqrt{3}}=2$.\\
	Gọi $B(0;0;m)$ mà $M$ là trung điểm $BC$ nên $C(0;0;2-m)$.\\
	Ta có $BC=|2-2m|=2\Leftrightarrow\hoac{&m=0\\&m=2}\Rightarrow B(0;0;0)$, $C(0;0;2)$ vì $C$ không trùng với $O$.\\
	Do đó $\overrightarrow{A_1C}=\left(-\sqrt{3};1;1 \right)=\dfrac{1}{10}\left( -10\sqrt{3};10;10\right)\Rightarrow\heva{&a=-10\sqrt{3}\\&b=10}$.\\
	Vậy $a^+b^2=400$.
	}
\end{ex}

\begin{ex}%[Nguồn: Bộ đề minh họa Moon 2024-2025]%[2H5V1-7]
	\immini{Một mái nhà hình tròn được đặt trên 3 cây cột trụ. Các cây cột trụ vuông góc với mặt sàn nhà phẳng và có độ cao lần lượt là $8$m, $9$m, $10$m. Ba chân cột là ba đỉnh của một tam giác đều trên mặt sàn nhà với cạnh dài $8$m. Chọn hệ trục tọa độ như hình vẽ, với $B$ thuộc tia $Ox$, $C$ thuộc tia $Oy$, tia $Oz$ cùng hướng với véc-tơ $\vv{AA'}$; gốc tọa độ $O$ trùng với trung điểm của $AC$ và mỗi đơn vị trên trục có độ dài 1 mét (xem hình vẽ).}{\begin{tikzpicture}[>=stealth,line join=round,line cap=round,font=\footnotesize,scale=1]
		\coordinate[label=center:$I$] (I)at(0,2.3);
		\begin{scope}[rotate=-18]
			\draw[red,fill=blue!30] (I) ellipse (2cm and 1.2cm);
		\end{scope}
		\coordinate[label=below right:$O$] (O)at(0,0);
		\draw[dashed,black]  (-1.4,3.2)coordinate(A')--(1.5,1.4)coordinate(C')--(-0.8,1.8)coordinate(B')--(A') (-1.4,0)coordinate(A)--(1.5,0)coordinate(C) (0,2.33)coordinate(T)--(O)--(-0.8,-0.8)coordinate(B) ($(B')!2/3!(T)$)coordinate(I)--($(B)!2/3!(O)$)coordinate(J);
		\draw (-1.8,0)--(A)--(B)--(C);
		\draw[->] (1.5,0)--(2,0)node[below]{$y$};
		\draw[->] (0,2.33)--(0,4)node[right]{$z$};
		\draw[->] (-0.8,-0.8)--(-1.2,-1.2)node[right]{$x$};
		\draw[line width=3pt,orange!70!black] (-1.4,0.05)--(-1.4,3.15) (1.5,0.05)--(1.5,1.35) (-0.8,-0.75)--(-0.8,1.75);
		\foreach \diem in {T,I,J,O} \fill (\diem)circle(1pt);
		\foreach \diem/\vitri in {A/below left,B/left,C/below,A'/above,B'/left,C'/right} \fill (\diem)circle(1pt)node[\vitri]{$\diem$};
	\end{tikzpicture}}
	\choiceTF
	{\True Tọa độ các điểm $A'(0;-4; 10)$, $B'(4\sqrt{3}; 0; 9)$, $C'(0; 4; 8)$}
	{\True Phương trình mặt phẳng $\left(A' B' C'\right)$ là $y+4z-36=0$}
	{Tọa độ các điểm $A'(0;-4; 10)$, $B'(4\sqrt{3}; 0; 9)$, $C'(0; 4; 8)$}
	{\True Phương trình mặt phẳng $\left(A' B' C'\right)$ là $y+4z-36=0$}
	\loigiai{
	\begin{itemchoice}
		\itemch Tọa độ các điểm $A(0;-4; 0)$, $B(4\sqrt{3};0; 0)$, $C(0;4; 0)$, $A'(0;-4; 10)$, $B'(4\sqrt{3}; 0; 9)$, $C'(0; 4; 8)$
		\itemch Ta có $\vv{A'B'}=\left(4\sqrt{3};4;-1\right)$; $\vv{A'C'}=\left(-4\sqrt{3};4;-1\right)$.\\
		Véc-tơ pháp tuyến của mặt phẳng $(A'B'C')$ là $\vv{n}=\left[\vv{A'B'},\vv{A'C'}\right]=\left(0;8\sqrt{3};32\sqrt{3}\right)=8\sqrt{3}\left(0;1;4\right)$.\\
		Phương trình mặt phẳng $(A'B'C')$ là $y+z-36=0$.
		\itemch Vec-tơ pháp tuyến của mặt phẳng $(ABC)$ là $\vv{k}=(0;0;1)$.\\
		Khi đó $\cos\left((ABC),(A'B'C')\right)=\dfrac{|5|}{\sqrt{4^2+1^2}}=\dfrac{4}{\sqrt{17}}\Rightarrow \left((ABC),(A'B'C')\right)\approx 14^\circ$.\\
		Vậy độ dốc của mái khoảng $14^\circ$, mái nhà trên không đạt tiêu chuẩn.
		\itemch Gọi $I(a;b;c)$. Suy ra
		$\heva{&b+c=36\\&a^2+(b+4)^2+(c-10)^2=\left(a-4\sqrt{3}\right)^2+b^2+(c-9)^2\\&a^2+(b+4)^2+(c-10)^2=a^2+(b-4)^2+(c-8)^2}\Leftrightarrow \heva{&a=\sqrt{5}\\&b=0\\&c=9.}$\\
		Vậy $I(\sqrt{5};0;9)$, điểm $I$ cách mặt sàn khoảng $9$ mét.
	\end{itemchoice}
	
	}
\end{ex}

%

\begin{ex}%[Nguồn: Bộ đề minh họa Moon 2024-2025]%[2H5V1-7]
	Trong không gian $Oxyz$, cho hai điểm $A(5; 0; 6)$ và $B(3; 5; 0)$. Điểm $M$ di động trên trục $Oz$, điểm $N$ di động trên trục $Oy$. Độ dài đường gấp khúc $AMNB$ có độ dài nhỏ nhất bằng bao nhiêu? (kết quả làm tròn đến hàng phần mười)
	\shortans[1]{$13{,}5$}
	\loigiai{
	Để tìm độ dài ngắn nhất của đường gấp khúc $AMNB$ ta sẽ \lq\lq trải\rq\rq \, các điểm $A$, $B$ về cùng $1$ mặt phẳng $(Oyz)$ với các điểm $M$, $N$ và thoả mãn đoạn thẳng mới bằng với đoạn thẳng ban đầu (tức
	$AM=A'M; BM=BM'$) và đoạn gấp khúc ngắn nhất khi $4$ điểm trên thẳng hàng.\\
	Ta quay vuông góc mặt phẳng chứa điểm $A(5; 0; 6)$; (tức mặt phẳng màu xanh) xuống mặt phẳng $(Oyz$) ta được điểm $A'(0;-5; 6)$.\\
	Giả sử điểm $M(0; 0; z) \in Oz$. \\
	Suy ra $\heva{&AM=\sqrt{5^2+(z-6)^2}\\& A'M=\sqrt{(-5)^2+(z-6)^2}} \Rightarrow AM=A'M$.\\
	Tương tự, ta quay vuông góc mặt phẳng chứa điểm $B(3; 5; 0)$ (tức mặt phẳng màu hồng) xuống mặt phẳng $(Oyz)$ ta được điểm $B'(0;5;-3)$.\\
	Giả sử điểm $N(0; y; 0) \in Oy$. Suy ra $BN=BN'$.\\
	Ta có độ dài đường gấp khúc $AMNB=A'M+MN+NB'$.\\
	Suy ra $\left(AM'+MN+N'B\right)_{min}$ xảy ra khi $A'$, $M$, $N$, $B'$ thẳng hàng và bằng $A'B'$.
	\[A'B'=\sqrt{(5+5)^2+(-3-6)^2}=\sqrt{181}\approx 13{,}5.\]
	}
\end{ex}

\begin{ex}%[Nguồn: Bộ đề minh họa Moon 2024-2025]%[2H5V2-2]
	Trong không gian $Oxyz$, cho ba đường thẳng $d\colon \dfrac{x}{1}=\dfrac{y}{1}=\dfrac{z+1}{-2}$, $\Delta_1\colon \dfrac{x-3}{2}=\dfrac{y}{1}=\dfrac{z-1}{1}$ và $\Delta_2\colon\dfrac{x-1}{1}=\dfrac{y-2}{2}=\dfrac{z}{1}$. Đường thẳng $\Delta$ vuông góc với $d$ đồng thời cắt $\Delta_1$, $\Delta_2$ tương ứng tại $H$, $K$ sao cho độ dài $HK$ nhỏ nhất. Biết rằng $\Delta$ có một và chỉ một vectơ chỉ phương là $\vec{u}=(h,k,5)$. Giá trị của $h^2+k^2$ bằng bao nhiêu?
	\shortans[]{$50$}
	\loigiai{
	Gọi $H(3+2t;t;1+t)\in\Delta_1$ và $K(1+a;2+2a;a)\in \Delta_2$.\\
	Ta có $\vec{HK}=(a-2t-2;2a-t+2;a-t-1)$.\\
	Đường thẳng $d$ có một vectơ chỉ phương là $\vec{b}=(1;1;-2)$.\\
	Vì $d\perp\Delta \Rightarrow \vec{b}\cdot\vec{HK}=0\Leftrightarrow a-2t-2+2a-t+2-2(a-t-1)=0\Leftrightarrow a=t-2$.\\
	Từ đó suy ra $\vec{HK}=(-t-4;t-2;-3)\Rightarrow HK^2=(t+4)^2+(t-2)^2+9=2(t+1)^2+27$.\\
	Từ đó suy ra giá trị nhỏ nhất của $HK$ là $\sqrt{27}$ khi $t=1$.\\
	Suy ra $\vec{HK}=(-3;-3;-3)$ suy ra một vectơ chỉ phương của $\Delta$ là $\vec{u}=(5;5;5)$.\\
	Vậy $h=k=5$ nên $h^2+k^2=50$.
	}
\end{ex}

\begin{ex}%[Nguồn: Bộ đề minh họa Moon 2024-2025]%[2H5V2-5]
	Trong không gian $Oxyz$, cho mặt phẳng $(\alpha)\colon x+y-z+6=0$ có vectơ pháp tuyến $\overrightarrow{n}$ và đường thẳng $d\colon\dfrac{x-1}{2}=\dfrac{y+4}{3}=\dfrac{z}{5}$ có vectơ chỉ phương $\overrightarrow{u}$.
	\choiceTF
	{$\overrightarrow{n}(1;1;-1)$ và $\overrightarrow{u}(1;-4;0)$}
	{\True Đường thẳng $d$ cắt mặt phẳng $(\alpha)$}
	{\True Lấy $A(1;-4;0)\in d$. Gọi $\Delta$ là đường thẳng đi qua $A$ và vuông góc với $(\alpha)$ thì phương trình tham số của đường thẳng $\Delta$ là $\heva{&x=1+t\\&y=-4+t\quad(t\in\mathbb{R})\\&z=-t}$}
	{\True Hình chiếu vuông góc của $d$ trên $(\alpha)$ có phương trình là $\dfrac{x}{2}=\dfrac{y+5}{3}=\dfrac{z-1}{5}$}
	\loigiai{
	\begin{itemchoice}
		\itemch
		Ta có vectơ pháp tuyến của mặt phẳng $(\alpha)$ là $\overrightarrow{n}_\alpha =(1;1;-1)$.\\
		Ta có vectơ chỉ phương của đường thẳng $d$ là $\overrightarrow{u}_d =(2;3;5)$.
		\itemch
		Xét $\overrightarrow{u}_d\cdot \overrightarrow{n}_\alpha=2+3-5=0$.\\
		Lại có điểm $A(1;-4;0)\in d$ nhưng $A\notin (\alpha)$ nên suy ra $d\perp (\alpha)$. \\Hay đường thẳng $d$ cắt mặt phẳng $(\alpha)$.
		\itemch
		Đường thẳng $\Delta$ có vectơ chỉ phương chính là vectơ pháp tuyến của mặt phẳng $\alpha$, $\overrightarrow{u}_{\Delta}=\overrightarrow{n}_\alpha =(1;1;-1)$.\\
		Suy ra phương trình tham số của đường thẳng $\Delta$ là
		$\heva{&x=1+t\\&y=-4+t\\&z=-t}(t\in\mathbb{R})$.
		\itemch
		Gọi đường thẳng $a$ là đường thẳng qua điểm $A$ và vuông góc với mặt phẳng  $\alpha$. \\Ta có phương trình đường thẳng $a$ là
		$\heva{&x=1+u\\&y=-4+u\\&z=-u.}$\\
		Cho $a\cap (\alpha)=B$ ta được điểm $B$ là hình chiếu vuông góc của điểm $A$ lên ($\alpha$).\\
		Tọa độ điểm $B$ thỏa mãn hệ pt $\heva{&x=1+u\\&y=-4+u\\&z=-u\\&x+y-z+6=0} \Rightarrow B(0;-5;1)$.\\
		Gọi hình chiếu vuông góc của đường thẳng $d$ lên mặt phẳng  ($\alpha$) là đường thẳng $d'$. \\Ta có $d'$ qua $B$ và $\overrightarrow{u}_{d'}=\overrightarrow{u}_d$.\\
		Phương trình chính tắc của đường thẳng $d'$ là
		$$\dfrac{x}{2}=\dfrac{y+5}{3}=\dfrac{z-1}{5}.$$
	\end{itemchoice}
	}
\end{ex}

%

\begin{ex}%[Nguồn: Bộ đề minh họa Moon 2024-2025]%[2H5V2-5]
	Trong không gian với hệ tọa độ $Oxyz$, cho tam giác $ABC$ với toạ độ 3 điểm là $A(1; 2;-1)$; $B(2;-1; 3)$; $C(-4; 7; 5)$.
	\choiceTF
	{\True Toạ độ $\overrightarrow{AB}=(1;-3; 4)$}
	{\True Độ dài $BC=2AB$}
	{Gọi điểm $D(a; b; c)$ là chân đường phân giác trong kẻ từ đỉnh $B$ xuống cạnh $AC$ thì ta có $a+b+c=3$}
	{\True Đường phân giác của góc $\widehat{ABC}$ cắt mặt phẳng $(Oxy)$ tại điểm $(-2; 6; 0)$}
	\loigiai{
	\begin{itemchoice}
		\itemch \textbf{Đúng.} \\
		
		\itemch \textbf{Đúng.} \\
		Vì $\overrightarrow{AB}=(1;-3; 4)\Rightarrow AB=\sqrt{26}$;
		$\overrightarrow{BC}=(-6;8; 2)\Rightarrow BC=2\sqrt{26}$.\\
		Do đó $BC=2AB$.
		\itemch \textbf{Sai.} \\
		\begin{center}
			\begin{tikzpicture}[declare function={a = 5; goc=50; b = 0.65*a;}]
				\path
				(goc:b) coordinate (B)
				(0,0) coordinate (A)
				(a,0) coordinate (C)
				($(B)!1cm!(A)$) coordinate (Nt)
				($(B)!1cm!(C)$) coordinate (Mt)
				($(Mt)!0.5!(Nt)$) coordinate (Xt)
				(intersection of B--Xt and A--C) coordinate (D);
				
				\path pic[draw,angle radius=7pt]{angle= A--B--D};
				\path pic[draw,angle radius=9pt]{angle= D--B--C};
				
				\draw (B)--(D) (B)--(A)--(C)--cycle;
				
				\foreach \t/\g in {B/90,A/-90,C/-90,D/-90}{
				\fill (\t) circle (1pt) node[shift={(\g:7pt)},font=\footnotesize]{$ \t $};
				}
			\end{tikzpicture}
		\end{center}
		Ta có $\overrightarrow{DA}=(1-a;2-b;-1-c)$; $\overrightarrow{DC}=(-4-a;7-b;5-c)$.\\
		Theo tính chất đường phân giác, ta có\\ $\dfrac{DA}{DC}=\dfrac{AB}{BC}=\dfrac{1}{2}\Rightarrow \overrightarrow{DA}=-\dfrac{1}{2}\overrightarrow{DC}\Rightarrow\heva{&1-a=-\dfrac{1}{2}(-4-a)\\&2-b=-\dfrac{1}{2}(7-b)\\&-1-c=-\dfrac{1}{2}(5-c)}\Rightarrow\heva{&a=-\dfrac{2}{3}\\&b=\dfrac{11}{3}\\&c=1}\Rightarrow D\left(-\dfrac{2}{3};\dfrac{11}{3}; 1\right)$.\\
		Do đó $a+b+c=-\dfrac{2}{3}+\dfrac{11}{3}+1=4$.
		\itemch \textbf{Đúng.} \\
		Đường phân giác của góc $\widehat{ABC}$ đi qua $B$ và $D$ có một vectơ chỉ phương là $\overrightarrow{BD}=\left(-\dfrac{8}{3};\dfrac{14}{3}; -2\right)=-\dfrac{2}{3}(4;-7;3)$ nên có phương trình $\heva{&x=2+4t\\&y=-1-7t\\&z=3+3t.}$\\
		Do đó $(BD)\cap (Oxy)=(-2;6;0)$.
	\end{itemchoice}
	}
\end{ex}

\begin{ex}%[Nguồn: Bộ đề minh họa Moon 2024-2025]%[2H5V2-7]
	Một radar có thể quay $180^\circ$ để quan sát máy bay quanh vùng phủ sóng của nó. Một máy bay cất cánh từ điểm $A$ nằm trên mặt đất theo chiều cùng chiều với vectơ $\vec{AB}$. Trong hệ tọa độ $Oxyz$ mặt đất là mặt phẳng $(Oxy)$, trục $O z$ hướng lên trời, điểm $A$ nằm trên trục $O y$ cách gốc tọa độ $0{,}6$ km; điểm $B$ nằm trên trục $O z$ có cao độ bằng $0{,}3$ km; radar đang nằm trên trục $Ox$ có hoành độ bằng $0{,}4$ km. Máy bay đang ở điểm $B$ bay theo hướng bay như cũ đến điểm $C(a;b;c)$ thì radar quay một góc bằng $60^\circ$. Tính $a+b+c$ theo đơn vị mét (làm tròn đến hàng đơn vị).
	\begin{center}
		\tikzset{rada/.pic={
		\definecolor{c191716}{RGB}{25,23,22}
		
		\begin{scope}[line cap=round,line join=round]
			\path[fill=white,nonzero rule] (0, 13.21) -- (10.42, 13.21) -- (10.42, 0.02) -- (0, 0.02) --cycle
			(0, 13.21);
			
			\path[fill=white,nonzero rule] (0, 13.21) -- (10.42, 13.21) -- (10.42, 0.02) -- (0, 0.02) --cycle
			(0, 13.21);
			
			\path[fill=c191716,nonzero rule] (0.03, 1.5) .. controls (0, 1.74) and (0.05, 1.99) ..
			(0.18, 2.21) .. controls (0.18, 2.21) and (0.18, 2.21) ..
			(0.18, 2.22) .. controls (0.2, 2.23) and (0.22, 2.23) ..
			(0.24, 2.22) -- (1.73, 0.74) -- (1.73, 0.74) .. controls (1.73, 0.74) and (1.73, 0.73) ..
			(1.73, 0.73) .. controls (1.75, 0.71) and (1.74, 0.69) ..
			(1.72, 0.67) .. controls (1.58, 0.59) and (1.42, 0.54) ..
			(1.25, 0.53) .. controls (1.09, 0.51) and (0.92, 0.53) ..
			(0.77, 0.59) .. controls (0.69, 0.61) and (0.62, 0.65) ..
			(0.55, 0.69) .. controls (0.48, 0.74) and (0.41, 0.79) ..
			(0.35, 0.85) .. controls (0.17, 1.03) and (0.06, 1.26) ..
			(0.03, 1.5);
			
			\path[fill=c191716,even odd rule] (0.39, 0.71) -- (0.06, 0.03) -- (1.2, 0.03) -- (1.2, 0.03) -- (1.22, 0.03) -- (1.04, 0.44) .. controls (0.94, 0.45) and (0.84, 0.47) ..
			(0.74, 0.51) .. controls (0.66, 0.54) and (0.58, 0.58) ..
			(0.5, 0.62) .. controls (0.46, 0.65) and (0.42, 0.68) ..
			(0.39, 0.71);
			
			\path[fill=c191716,even odd rule] (1.21, 1.76) .. controls (1.24, 1.79) and (1.28, 1.79) ..
			(1.31, 1.76) .. controls (1.34, 1.73) and (1.34, 1.69) ..
			(1.31, 1.66) -- (1.03, 1.38) .. controls (1, 1.35) and (0.95, 1.35) ..
			(0.93, 1.38) .. controls (0.9, 1.41) and (0.9, 1.45) ..
			(0.93, 1.48) --
			(1.21, 1.76);
			
			\path[fill=c191716,nonzero rule] (1.33, 1.94) .. controls (1.37, 1.94) and (1.41, 1.92) ..
			(1.45, 1.89) .. controls (1.48, 1.86) and (1.49, 1.82) ..
			(1.49, 1.78) .. controls (1.49, 1.74) and (1.48, 1.7) ..
			(1.45, 1.67) .. controls (1.41, 1.64) and (1.37, 1.62) ..
			(1.33, 1.62) .. controls (1.29, 1.62) and (1.25, 1.64) ..
			(1.22, 1.67) .. controls (1.19, 1.7) and (1.17, 1.74) ..
			(1.17, 1.78) .. controls (1.17, 1.82) and (1.19, 1.86) ..
			(1.22, 1.89) .. controls (1.25, 1.92) and (1.29, 1.94) ..
			(1.33, 1.94);
			
			\path[fill=c191716,even odd rule] (1.49, 2.47) .. controls (1.56, 2.49) and (1.65, 2.49) ..
			(1.72, 2.47) .. controls (1.8, 2.45) and (1.87, 2.42) ..
			(1.93, 2.36) .. controls (1.99, 2.3) and (2.03, 2.23) ..
			(2.05, 2.15) .. controls (2.06, 2.08) and (2.06, 1.99) ..
			(2.04, 1.92) .. controls (2.03, 1.88) and (2.06, 1.84) ..
			(2.1, 1.83) .. controls (2.13, 1.82) and (2.17, 1.85) ..
			(2.18, 1.88) .. controls (2.2, 1.98) and (2.21, 2.09) ..
			(2.18, 2.19) .. controls (2.16, 2.29) and (2.11, 2.38) ..
			(2.03, 2.46) .. controls (1.96, 2.53) and (1.86, 2.58) ..
			(1.76, 2.61) .. controls (1.66, 2.63) and (1.55, 2.63) ..
			(1.45, 2.61) .. controls (1.42, 2.6) and (1.39, 2.56) ..
			(1.4, 2.52) .. controls (1.41, 2.48) and (1.45, 2.46) ..
			(1.49, 2.47) --cycle
			(1.43, 2.26) .. controls (1.49, 2.28) and (1.55, 2.28) ..
			(1.6, 2.26) .. controls (1.66, 2.25) and (1.71, 2.22) ..
			(1.75, 2.18) .. controls (1.8, 2.14) and (1.82, 2.09) ..
			(1.84, 2.03) .. controls (1.85, 1.98) and (1.85, 1.92) ..
			(1.83, 1.86) .. controls (1.83, 1.82) and (1.85, 1.78) ..
			(1.89, 1.78) .. controls (1.93, 1.77) and (1.96, 1.79) ..
			(1.97, 1.83) .. controls (1.99, 1.91) and (1.99, 1.99) ..
			(1.97, 2.07) .. controls (1.96, 2.15) and (1.91, 2.22) ..
			(1.85, 2.28) .. controls (1.79, 2.34) and (1.72, 2.38) ..
			(1.64, 2.4) .. controls (1.56, 2.42) and (1.48, 2.42) ..
			(1.4, 2.4) .. controls (1.36, 2.39) and (1.34, 2.35) ..
			(1.34, 2.32) .. controls (1.35, 2.28) and (1.39, 2.25) ..
			(1.43, 2.26) --cycle
			(1.47, 2) .. controls (1.43, 2) and (1.39, 2.02) ..
			(1.38, 2.06) .. controls (1.38, 2.09) and (1.4, 2.13) ..
			(1.44, 2.14) .. controls (1.47, 2.15) and (1.51, 2.15) ..
			(1.55, 2.14) .. controls (1.59, 2.13) and (1.63, 2.11) ..
			(1.66, 2.08) .. controls (1.68, 2.06) and (1.7, 2.02) ..
			(1.71, 1.98) .. controls (1.72, 1.94) and (1.72, 1.9) ..
			(1.71, 1.87) .. controls (1.7, 1.83) and (1.67, 1.81) ..
			(1.63, 1.82) .. controls (1.59, 1.82) and (1.57, 1.86) ..
			(1.57, 1.9) .. controls (1.58, 1.92) and (1.58, 1.93) ..
			(1.58, 1.95) .. controls (1.57, 1.96) and (1.57, 1.97) ..
			(1.56, 1.98) .. controls (1.54, 1.99) and (1.53, 2) ..
			(1.52, 2.01) .. controls (1.5, 2.01) and (1.49, 2.01) ..
			(1.47, 2);
			
			\path[fill=c191716,even odd rule] (10.37, 13.11) .. controls (10.4, 13.05) and (10.41, 12.96) ..
			(10, 12.72) .. controls (9.98, 12.72) and (9.97, 12.71) ..
			(9.96, 12.7) -- (9.77, 11.86) -- (9.63, 11.78) -- (9.62, 12.52) .. controls (9.39, 12.41) and (9.29, 12.38) ..
			(9.15, 12.33) -- (9.09, 11.97) -- (8.96, 11.9) -- (8.93, 12.27) -- (8.63, 12.48) -- (8.75, 12.56) -- (9.08, 12.43) .. controls (9.2, 12.53) and (9.27, 12.61) ..
			(9.48, 12.74) -- (8.86, 13.12) -- (9, 13.2) -- (9.81, 12.95) .. controls (9.83, 12.96) and (9.84, 12.96) ..
			(9.85, 12.97) .. controls (10.27, 13.21) and (10.34, 13.16) ..
			(10.37, 13.11) --cycle
			(10.37, 13.11);
			
		\end{scope}
		
		}}
		\begin{tikzpicture}[scale=.7,transform shape]
			\path (0,0) pic[scale=.3]{rada};
			\path (2.9,2) coordinate (O)
			+(-30:3) coordinate (y) node[above] {$y$}
			+(90:3) coordinate (z) node[above] {$z$}
			+(90:1.8) coordinate (B)
			(.2,0) coordinate (M)
			($(O)!1.3!(M)$) coordinate (x) node[above] {$x$}
			($(O)!-1.5!(y)$) coordinate (y')
			($(O)!-1!(y)$) coordinate (A)
			($(A)!1.5!(B)$) coordinate (v) node[above] {$\overrightarrow{v}$}
			($(A)!1.9!(B)$) coordinate (v')
			;
			%			\draw[gray,xstep = 1, ystep = 1] (0,0) grid (5,5);
			
			\draw[-stealth,dashed] (O)--(z);
			\draw[-stealth,dashed] (O)--(y);
			\draw[-stealth,dashed] (O)--(x);
			\draw[dashed] (O)--(y')
			(A)--(v')
			;
			\draw (0.4,.55)--(B) node[midway,left] {$R$};
			\draw[-stealth] (B)--(v);
			\draw[-stealth] (.5,1) to[bend left = 30] (1.5,0.1) node[right] {$\theta$};
			
			\foreach \x/\g in {B/-45,M/-90,O/-90,A/-90}\fill (\x) circle (1.5pt)+(\g:3mm) node{$\x$};
			
		\end{tikzpicture}
		
	\end{center}
	\par\shortans{2630}
	\loigiai{
	\begin{center}
		\tikzset{rada/.pic={
		\definecolor{c191716}{RGB}{25,23,22}
		
		\begin{scope}[line cap=round,line join=round]
			\path[fill=white,nonzero rule] (0, 13.21) -- (10.42, 13.21) -- (10.42, 0.02) -- (0, 0.02) --cycle
			(0, 13.21);
			
			\path[fill=white,nonzero rule] (0, 13.21) -- (10.42, 13.21) -- (10.42, 0.02) -- (0, 0.02) --cycle
			(0, 13.21);
			
			\path[fill=c191716,nonzero rule] (0.03, 1.5) .. controls (0, 1.74) and (0.05, 1.99) ..
			(0.18, 2.21) .. controls (0.18, 2.21) and (0.18, 2.21) ..
			(0.18, 2.22) .. controls (0.2, 2.23) and (0.22, 2.23) ..
			(0.24, 2.22) -- (1.73, 0.74) -- (1.73, 0.74) .. controls (1.73, 0.74) and (1.73, 0.73) ..
			(1.73, 0.73) .. controls (1.75, 0.71) and (1.74, 0.69) ..
			(1.72, 0.67) .. controls (1.58, 0.59) and (1.42, 0.54) ..
			(1.25, 0.53) .. controls (1.09, 0.51) and (0.92, 0.53) ..
			(0.77, 0.59) .. controls (0.69, 0.61) and (0.62, 0.65) ..
			(0.55, 0.69) .. controls (0.48, 0.74) and (0.41, 0.79) ..
			(0.35, 0.85) .. controls (0.17, 1.03) and (0.06, 1.26) ..
			(0.03, 1.5);
			
			\path[fill=c191716,even odd rule] (0.39, 0.71) -- (0.06, 0.03) -- (1.2, 0.03) -- (1.2, 0.03) -- (1.22, 0.03) -- (1.04, 0.44) .. controls (0.94, 0.45) and (0.84, 0.47) ..
			(0.74, 0.51) .. controls (0.66, 0.54) and (0.58, 0.58) ..
			(0.5, 0.62) .. controls (0.46, 0.65) and (0.42, 0.68) ..
			(0.39, 0.71);
			
			\path[fill=c191716,even odd rule] (1.21, 1.76) .. controls (1.24, 1.79) and (1.28, 1.79) ..
			(1.31, 1.76) .. controls (1.34, 1.73) and (1.34, 1.69) ..
			(1.31, 1.66) -- (1.03, 1.38) .. controls (1, 1.35) and (0.95, 1.35) ..
			(0.93, 1.38) .. controls (0.9, 1.41) and (0.9, 1.45) ..
			(0.93, 1.48) --
			(1.21, 1.76);
			
			\path[fill=c191716,nonzero rule] (1.33, 1.94) .. controls (1.37, 1.94) and (1.41, 1.92) ..
			(1.45, 1.89) .. controls (1.48, 1.86) and (1.49, 1.82) ..
			(1.49, 1.78) .. controls (1.49, 1.74) and (1.48, 1.7) ..
			(1.45, 1.67) .. controls (1.41, 1.64) and (1.37, 1.62) ..
			(1.33, 1.62) .. controls (1.29, 1.62) and (1.25, 1.64) ..
			(1.22, 1.67) .. controls (1.19, 1.7) and (1.17, 1.74) ..
			(1.17, 1.78) .. controls (1.17, 1.82) and (1.19, 1.86) ..
			(1.22, 1.89) .. controls (1.25, 1.92) and (1.29, 1.94) ..
			(1.33, 1.94);
			
			\path[fill=c191716,even odd rule] (1.49, 2.47) .. controls (1.56, 2.49) and (1.65, 2.49) ..
			(1.72, 2.47) .. controls (1.8, 2.45) and (1.87, 2.42) ..
			(1.93, 2.36) .. controls (1.99, 2.3) and (2.03, 2.23) ..
			(2.05, 2.15) .. controls (2.06, 2.08) and (2.06, 1.99) ..
			(2.04, 1.92) .. controls (2.03, 1.88) and (2.06, 1.84) ..
			(2.1, 1.83) .. controls (2.13, 1.82) and (2.17, 1.85) ..
			(2.18, 1.88) .. controls (2.2, 1.98) and (2.21, 2.09) ..
			(2.18, 2.19) .. controls (2.16, 2.29) and (2.11, 2.38) ..
			(2.03, 2.46) .. controls (1.96, 2.53) and (1.86, 2.58) ..
			(1.76, 2.61) .. controls (1.66, 2.63) and (1.55, 2.63) ..
			(1.45, 2.61) .. controls (1.42, 2.6) and (1.39, 2.56) ..
			(1.4, 2.52) .. controls (1.41, 2.48) and (1.45, 2.46) ..
			(1.49, 2.47) --cycle
			(1.43, 2.26) .. controls (1.49, 2.28) and (1.55, 2.28) ..
			(1.6, 2.26) .. controls (1.66, 2.25) and (1.71, 2.22) ..
			(1.75, 2.18) .. controls (1.8, 2.14) and (1.82, 2.09) ..
			(1.84, 2.03) .. controls (1.85, 1.98) and (1.85, 1.92) ..
			(1.83, 1.86) .. controls (1.83, 1.82) and (1.85, 1.78) ..
			(1.89, 1.78) .. controls (1.93, 1.77) and (1.96, 1.79) ..
			(1.97, 1.83) .. controls (1.99, 1.91) and (1.99, 1.99) ..
			(1.97, 2.07) .. controls (1.96, 2.15) and (1.91, 2.22) ..
			(1.85, 2.28) .. controls (1.79, 2.34) and (1.72, 2.38) ..
			(1.64, 2.4) .. controls (1.56, 2.42) and (1.48, 2.42) ..
			(1.4, 2.4) .. controls (1.36, 2.39) and (1.34, 2.35) ..
			(1.34, 2.32) .. controls (1.35, 2.28) and (1.39, 2.25) ..
			(1.43, 2.26) --cycle
			(1.47, 2) .. controls (1.43, 2) and (1.39, 2.02) ..
			(1.38, 2.06) .. controls (1.38, 2.09) and (1.4, 2.13) ..
			(1.44, 2.14) .. controls (1.47, 2.15) and (1.51, 2.15) ..
			(1.55, 2.14) .. controls (1.59, 2.13) and (1.63, 2.11) ..
			(1.66, 2.08) .. controls (1.68, 2.06) and (1.7, 2.02) ..
			(1.71, 1.98) .. controls (1.72, 1.94) and (1.72, 1.9) ..
			(1.71, 1.87) .. controls (1.7, 1.83) and (1.67, 1.81) ..
			(1.63, 1.82) .. controls (1.59, 1.82) and (1.57, 1.86) ..
			(1.57, 1.9) .. controls (1.58, 1.92) and (1.58, 1.93) ..
			(1.58, 1.95) .. controls (1.57, 1.96) and (1.57, 1.97) ..
			(1.56, 1.98) .. controls (1.54, 1.99) and (1.53, 2) ..
			(1.52, 2.01) .. controls (1.5, 2.01) and (1.49, 2.01) ..
			(1.47, 2);
			
			\path[fill=c191716,even odd rule] (10.37, 13.11) .. controls (10.4, 13.05) and (10.41, 12.96) ..
			(10, 12.72) .. controls (9.98, 12.72) and (9.97, 12.71) ..
			(9.96, 12.7) -- (9.77, 11.86) -- (9.63, 11.78) -- (9.62, 12.52) .. controls (9.39, 12.41) and (9.29, 12.38) ..
			(9.15, 12.33) -- (9.09, 11.97) -- (8.96, 11.9) -- (8.93, 12.27) -- (8.63, 12.48) -- (8.75, 12.56) -- (9.08, 12.43) .. controls (9.2, 12.53) and (9.27, 12.61) ..
			(9.48, 12.74) -- (8.86, 13.12) -- (9, 13.2) -- (9.81, 12.95) .. controls (9.83, 12.96) and (9.84, 12.96) ..
			(9.85, 12.97) .. controls (10.27, 13.21) and (10.34, 13.16) ..
			(10.37, 13.11) --cycle
			(10.37, 13.11);
			
		\end{scope}
		
		}}
		\begin{tikzpicture}[scale=.7,transform shape]
			\path (0,0) pic[scale=.3]{rada};
			\path (2.9,2) coordinate (O)
			+(-30:3) coordinate (y) node[above] {$y$}
			+(90:3) coordinate (z) node[above] {$z$}
			+(90:1.8) coordinate (B)
			(.2,0) coordinate (M)
			($(O)!1.3!(M)$) coordinate (x) node[above] {$x$}
			($(O)!-1.5!(y)$) coordinate (y')
			($(O)!-1!(y)$) coordinate (A)
			($(A)!1.5!(B)$) coordinate (v) node[above] {$\overrightarrow{v}$}
			($(A)!1.9!(B)$) coordinate (v') node[above] {$C$}
			;
			%			\draw[gray,xstep = 1, ystep = 1] (0,0) grid (5,5);
			
			\draw[-stealth,dashed] (O)--(z);
			\draw[-stealth,dashed] (O)--(y);
			\draw[-stealth,dashed] (O)--(x);
			\draw[dashed] (O)--(y')
			(A)--(v')
			(M)--(v')
			;
			\draw (0.4,.55)--(B) node[midway,left] {$R$};
			\draw[-stealth] (B)--(v);
			%			\draw[-stealth] (.5,1) to[bend left = 30] (1.5,0.1) node[right] {$\theta$};
			
			\foreach \x/\g in {B/-45,M/-90,O/-90,A/-90}\fill (\x) circle (1.5pt)+(\g:3mm) node{$\x$}
			;
			\fill (v') circle(1.5pt);
			\draw[thin,fill=gray] pic[angle radius = 40pt, "$60^\circ$", angle eccentricity = 1.3]{ angle = v'--M--B};
		\end{tikzpicture}
	\end{center}
	Ta có điểm $A(0;-0{,}6;0)$; $B(0;0;0{,}3)$, $M(0{,}4;0;0)$.\\
	Vậy $\overrightarrow{AB}=(0;0{,}6;0{,}3)$ nên véctơ $\overrightarrow{u}=(0;2;1)$.\\
	Do đó phương trình đường thẳng $AB$ là $\heva{&x=0\\&y=2t\\&z=0{,}3t.}$\\
	Điểm $C$ thuộc đường thẳng $AB$ nên có tọa độ là $C(0;2t;0{,}3t)$.
	Vì $C$ ứng với góc quay $60^\circ$ của radar nên $\widehat{BMC}=60^\circ$.\\
	$\overrightarrow{MB}=(-0{,}4;0;0{,}3)$; $\overrightarrow{MC}=(-0{,}4;2t;0{,}3+t)$.\\
	Ta có $$\cos \widehat{BMC}=\dfrac{\overrightarrow{MB}\cdot \overrightarrow{MC}}{\left|\overrightarrow{MB}\right|\cdot \left|\overrightarrow{MC}\right|}=\dfrac{0{,}25+0{,}3t}{0{,}5\cdot \sqrt{5t^2+0{,}6t+0{,}25}}=\dfrac{1}{2}.\quad (*)$$
	Rút gọn $(*)$ ta thu được phương trình bậc hai
	$$
	3{,}56t^2-1{,}8t-0{,}75=0.
	$$
	Phương trình trên có hai nghiệm là $t\approx -0{,}27120$ (loại) và $t\approx 0{,}776819$.\\
	Vậy $a+b+c=0+2t+0{,}3+t=0{,}3+3t=2{,630457}$ (km) $\approx 2630$ (m).
	}
\end{ex}

\begin{ex}%[Nguồn: Bộ đề minh họa Moon 2024-2025]%[2H5V2-7]
	Một lều trại có mặt trước và mặt sau rộng $4$ m, hai mặt bên rộng $3$ m gồm sáu thanh cọc tre, vải bạt chống thấm nước, dây dù hoặc dây thừng để cố định lều tại sáu cọc sắt cắm sát đất như hình vẽ. Biết rằng, hai thanh $AF$, $OC$ có chiều dài $2{,}2$ m, bốn thanh còn lại có chiều dài $1{,}7$ m và đoạn dây thừng $IF=2{,}75$ m.  Chọn hệ trục tọa độ $Oxyz$ như hình vẽ và cho biết góc giữa đường thẳng chứa dây thừng $IF$ và mặt phẳng chứa tấm bạt $(CDEF)$ là $\alpha$.
	\begin{center}
		\begin{tikzpicture}[line join = round, line cap=round,>=stealth,font=\footnotesize,scale=1]
			\path
			(0,0) coordinate (O)
			(-5,-1) coordinate (x)
			(2,-3) coordinate (y)
			(0,5) coordinate (z)
			%	($(A)+(B)-(O)$) coordinate (N)
			%	($(N)+(0,4)$) coordinate (M)
			;
			\coordinate (A) at ($(O)!5/7!(x)$);
			\coordinate (I) at ($(O)!6.3/7!(x)$);
			\coordinate (B) at ($(O)!3/5!(y)$);
			\coordinate (a1) at	($(A)+(B)-(O)$);
			\coordinate (F) at ($(A)+(0,3)$);
			\coordinate (C) at ($(O)+(0,3)$);
			\coordinate (D) at ($(B)+(0,2.5)$);
			\coordinate (E) at ($(a1)+(0,2.5)$);
			\coordinate (b1) at ($(A)!-1!(a1)$);
			\coordinate (b2) at ($(b1)+(0,2)$);
			\coordinate (c1) at	($(b1)+(O)-(A)$);
			\coordinate (c2) at ($(c1)+(0,2)$);
			\draw[->] (O)--(x)node[below left]{$x$};
			\draw[->] (O)--(y)node[below left]{$y$};
			\draw[->] (O)--(z)node[above left]{$z$};
			\draw(O)--(c1)-- (b1)--(A)--(a1)--(B) (c2)--(b2)--(F)--(E)--(D)--(C)--(c2) (F)--(C) (F)--(I);
			\draw[line width=2pt] (A)--(F) (a1)--(E) (O)--(C) (B)--(D) (b1)--(b2) (c1)--(c2);
			\draw(E)--(-3.5,-3)  (b2)--(-5.5,2) (D)--(1.8,-1.7) (C)--(1.5,1.2) (c2)--(-.5,2);
			\path (A)--(b1) node [below left ,sloped,pos=0.1] {$4$m};
			\path (a1)--(B) node [below  ,sloped,pos=0.5] {$3$m};
			\foreach \i/\g in {A/60,B/-90,C/30,D/30,E/70,F/90,I/-90}{\draw[fill=black](\i) circle (1.5pt) ($(\i)+(\g:3mm)$) node[scale=1]{$\i$};}
		\end{tikzpicture}
	\end{center}
	Tính giá trị của $\alpha$ (tính theo đơn vị độ và làm tròn kết quả đến hàng đơn vị của độ). \shortans[oly]{$51^\circ$}
	\loigiai{
	Ta có $AI=\sqrt{IF^2-AF^2}=\sqrt{2{,}75^2-2{,}2^2}=1{,}65$ (m).\\
	Do đó $A(3; 0; 0)$, $I(4{,}65; 0; 0)$, $B(0; 2; 0)$, $E(3; 2; 1{,}7)$, $F(3; 0; 2{,}2)$ và $C(0; 0; 2{,}2)$.\\
	Suy ra $\overrightarrow{IF}=(-1{,}65; 0; 2{,}2)$, $\overrightarrow{EF}=(0; -2; 0{,}5)$ và $\overrightarrow{EC}=(-3; -2; 0{,}5)$.\\
	Lại có $\left[\overrightarrow{EF}, \overrightarrow{EC}\right]=(0; -1{,}5; -6)$ nên $\overrightarrow{n}=(0; 1; 4)$ là một vectơ pháp tuyến của mặt phẳng $(CDEF)$.\\
	Suy ra
	\[\sin\left(IF, (CDEF)\right)=\dfrac{\left|\overrightarrow{IF}\cdot \overrightarrow{n}\right|}{\left|\overrightarrow{IF}\right|\cdot \left|\overrightarrow{n}\right|}=\dfrac{\left|-1{,}65\cdot 0+0\cdot 1+2{,}2\cdot4\right|}{\sqrt{(-1{,}65)^2+0^2+2{,}2^2}\cdot\sqrt{0^2+1^2+4^2}}=\dfrac{16\sqrt{17}}{85}.\]
	Do đó $\left(IF, (CDEF)\right)\approx 51^\circ\Rightarrow \alpha\approx 51^\circ$.
	}
\end{ex}

\begin{ex}%[Nguồn: Bộ đề minh họa Moon 2024-2025]%[2H5V2-8]
	Một nắp bể nước hình chữ nhật $ABCD$ nằm cạnh bờ tường có kích thước $9$ dm $\times 12$ dm được kéo ra khỏi mặt sàn, do tác dụng của trọng lực nên nắp bể không thể mở ra được nếu không có người giữ. Người ta dùng một sợi dây xích dài $15$ dm và kéo căng nối đỉnh $C$ của hình chữ nhật với điểm $M$ nằm phía trên bờ tường sao cho $AM = 9$ dm và $AM$ vuông góc với mặt sàn. Chọn hệ trục $Oxy$ như hình vẽ, khi đó nắp bể mở ra và tạo với mặt sàn một góc $\alpha$ (đơn vị trên mỗi trục tọa độ tính bằng dm). Bỏ qua độ dày của nắp bể.
	\begin{center}
		\includegraphics[scale=0.8]{hinh/napbe.png}
	\end{center}
	\choiceTF
	{Điểm $M$ thuộc mặt phẳng có phương trình $z = 0$}
	{Tọa độ điểm $C$ là $C(9 \sin \alpha; 12; 9 \cos \alpha)$}
	{Góc giữa nắp bể và mặt sàn sau khi kéo lên là $\alpha = 60^\circ$}
	{\True Phương trình mặt phẳng chứa nắp bể sau khi kéo lên là $x - \sqrt{3} z = 0$}
	\loigiai{
	\begin{center}
		\begin{tikzpicture}[scale=1,>=stealth, font=\footnotesize, line join=round, line cap=round]
			\def\a{4}
			\path	(0:0) coordinate (O)
			++(-10:\a) coordinate (x)
			($(O)+(-140:3)$) coordinate (y)
			($(O)+(-140:\a/2)$) coordinate (D)
			($(O)+(90:3)$) coordinate (z)
			($(O)+(90:\a/2)$) coordinate (M)
			($(O)+(90:\a/4)$) coordinate (I)
			($(O)+(5:3)$) coordinate (B)
			($(D)+(5:3)$) coordinate (C);
			\coordinate (G) at (intersection of B--C and O--x);
			\draw[thick] (I)--(C);
			\draw[dashed,thick] (O)--(G);
			\draw[->] (G)--(x)node[below]{$x$};
			\draw[->] (O)--(y)node[below]{$y$};
			\draw[->] (O)--(z)node[left]{$z$};
			\draw [](O)--(D)--(C)--(B)--(O) (M)--(C)--(O);
			\foreach \x/\g in {O/160,B/0,C/-45,D/160,M/45,I/180}
			\fill[black]	(\x) circle (1pt)
			($(\g:3mm)+(\x)$) node {$\x$};
			%Hình chóp S.ABC có SA vuông góc đáy
		\end{tikzpicture}
	\end{center}
	Thiết lập hệ trục tọa độ như hình vẽ trên ta có $D(0; 12; 0)$.\\
	Ta có $OC=\sqrt{OB^2+OD^2}=15=MC$ nên $\triangle OMC$ là tam giác cân đỉnh $C$.\\
	Gọi $\alpha$ là góc giữa nắp bể và mặt sàn thì ta có $\alpha=\widehat{xOB}$, do đó điểm $B\left(9\cos \alpha; 0;\dfrac{9}{2}\right)$.\\
	Từ đó suy ra $C\left(9\cos \alpha; 12;\dfrac{9}{2}\right)$.\\
	Vì $OC=15$ nên ta có $\sqrt{81\cos^2\alpha+12^2+\dfrac{81}{4}}=15\Leftrightarrow \cos^2\alpha=\dfrac{3}{4}\Rightarrow \cos\alpha=\dfrac{\sqrt{3}}{2}$.\\
	Từ đó suy ra $\alpha=30^\circ$.
	\begin{itemchoice}
		\itemch \textbf{Sai}.\\
		Điểm $M(0; 0; 9)$ nên không thuộc mặt phẳng $z=0$.
		\itemch \textbf{Sai}.\\
		Tọa độ điểm $C$ là $C\left(9\cos \alpha; 12;\dfrac{9}{2}\right)$.
		\itemch \textbf{Sai}.\\
		Góc giữa nắp bể và mặt sàn sau khi kéo lên là $\alpha = 30^\circ$
		\itemch \textbf{Đúng}.\\
		Đặt $\overrightarrow{u}=\dfrac{2}{9}\overrightarrow{OB}=(\sqrt{3}; 0; 1)$ thì $\overrightarrow{n}=\left[\overrightarrow{j}; \overrightarrow{u}\right]=(1; 0; -\sqrt{3})$ là véc-tơ pháp tuyến của mặt phẳng $(OBCD)$ nên phương trình mặt phẳng chứa nắp bể là $x-\sqrt{3}z=0$.
	\end{itemchoice}
	}
\end{ex}

\begin{ex}%[Nguồn: Bộ đề minh họa Moon 2024-2025]%[2H5V2-8]
	Xét hai chiếc khinh khí cầu bay lên từ cùng một điểm trong cùng một ngày. Lúc $9$ h sáng, chiếc thứ nhất đang ở vị trí $A$ cách điểm xuất phát $2$ km về phía Nam và $1$ km về phía Đông, đồng thời cách mặt đất $0{,}5$ km. Chiếc thứ hai đang ở vị trí $B$ nằm cách điểm xuất phát $1$ km về phía Bắc và $1{,}5$ km về phía Tây đồng thời cách mặt đất $0{,}8$ km. Chọn hệ trục tọa độ $Oxyz$ với gốc $O$ đặt tại điểm xuất phát của hai khinh khí cầu, mặt phẳng $(Oxy)$ trùng với mặt đất, trục $Ox$ hướng về phía Nam, trục $Oy$ hướng về phía Đông và trục $Oz$ hướng thẳng đưng lên trời (như hình vẽ). Lấy đơn vị đo trên mỗi trục là km
	\begin{center}
		\begin{tikzpicture}[smooth,samples=300,scale=0.8,>=stealth,font=\footnotesize]
			\draw[->] (-4,0)node[above right]{Bắc}--(6,0) node[below]{$x$} node[above left]{Nam};
			\draw[->] (3,3)node[above right]{Tây}--(-2.3,-2.3) node[left]{$y$}node[below right]{Đông};
			\draw[->] (0,0)--(0,4.5) node[right]{$z$};
			\draw (0,0) node[below]{$O$};
			\fill [scale=.1,black,yshift=12 cm,xshift=25 cm]
			(-1,0) rectangle (1,1)
			(-1,2).. controls +(135:1) and +(180:4) .. (0,7)
			.. controls +(180:2) and +(135:2).. (-.5,2)
			.. controls +(120:2) and +(180:1) .. (0,6.99)
			.. controls +(0:1) and +(60:2) .. (.5,2)
			.. controls +(45:2) and +(0:2) .. (0,7)
			.. controls +(0:4) and +(45:1).. (1,2)
			;
			\draw[scale=.1,yshift=12 cm,xshift=25 cm] (-1,0) rectangle (1,2) (0,1)--(0,2);
			;
			\fill [scale=.1,blue,yshift=37 cm,xshift=-20 cm]
			(-1,0) rectangle (1,1)
			(-1,2).. controls +(135:1) and +(180:4) .. (0,7)
			.. controls +(180:2) and +(135:2).. (-.5,2)
			.. controls +(120:2) and +(180:1) .. (0,6.99)
			.. controls +(0:1) and +(60:2) .. (.5,2)
			.. controls +(45:2) and +(0:2) .. (0,7)
			.. controls +(0:4) and +(45:1).. (1,2)
			;
			\draw[blue,scale=.1,yshift=37 cm,xshift=-20 cm] (-1,0) rectangle (1,2) (0,1)--(0,2);
			;
			\draw[dashed] (4,0)--(2.5,-1.5)--(-1.5,-1.5)
			(2.5,1)--(0,0)--(2.5,-1.5)--(2.5,1);
			\draw[dashed] (-3,0)--(-2,1)--(1,1) (-2,1)--(0,0)--(-2,3.5)--(-2,1)
			;
			\draw[fill=black] (2.5,1) circle(2pt) (-2,3.5) circle(2pt);
		\end{tikzpicture}
	\end{center}
	\choiceTF
	{\True Tọa độ của khinh khi cầu thứ nhất lúc 9 h sáng là $A(2;1;0{,}5)$}
	{Phương trình chính tắc của đường thẳng $AB$ là $\dfrac{x-2}{30}=\dfrac{y-1}{25}=\dfrac{z-0,5}{3}$}
	{\True Lúc $9$ h sáng, khoảng cách giữa hai chiếc khinh khí cầu là $3{,}92$ km (làm tròn đến hàng phần trăm theo đơn vị km)}
	{Từ $9$ h sáng đến $9$ h $10'$ sáng, khinh khí cầu thứ nhất đi thẳng đều về huớng Nam với vận tốc $50$ km/h và độ cao không đổi để đến điểm $M$, khinh khí cầu thứ hai chuyển động thẳng đều đến điểm $N$ với vận tốc $60$ km/h, biết vectơ $\overrightarrow{BN}$ cùng hướng với vectơ $\vec{u}(2; 2; 1)$. Bỏ qua lực cản của gió, khoảng cách $MN$ là $4{,}66$ km (làm tròn đến hàng phần trăm theo đơn vị km)}
	\loigiai{
	Hệ trục tọa độ:
	\begin{itemize}
		\item Gốc $O$ tại điểm xuất phát.
		\item Trục $Ox$ hướng về phía Nam.
		\item Trục $Oy$ hướng về phía Đông.
		\item Trục $Oz$ hướng thẳng đứng lên trên.
		\item Đơn vị: km.
	\end{itemize}
	\begin{itemchoice}
		\itemch Vị trí ban đầu của hai khinh khí cầu lúc $9$ h sáng:\\
		Khinh khí cầu thứ nhất $(A)$:
		\begin{itemize}
			\item Cách điểm xuất phát $2$ km về phía Nam: \( x = 2 \) km.
			\item Cách điểm xuất phát $1$ km về phía Đông: \( y = 1 \) km.
			\item  Cách mặt đất $0{,}5$ km: \( z = 0.5 \) km.
		\end{itemize}
		Suy ra $A(2;1;0{,}5)$.\\
		Khinh khí cầu thứ hai $(B)$:
		\begin{itemize}
			\item Cách điểm xuất phát $1$ km về phía Bắc: \( x = -1 \) km (vì Bắc ngược hướng $Ox$).
			\item Cách điểm xuất phát $1{,}5$ km về phía Tây: \( y = -1.5 \) km (vì Tây ngược hướng $Oy$).
			\item Cách mặt đất $0{,}8$ km: \( z = 0.8 \) km.
		\end{itemize}
		Suy ra $ B(-1; -1{,}5;0{,}8)$.
		\itemch Ta có $\overrightarrow{AB} = (-3;-2{,}5;0{,}3)=-\dfrac{1}{10}(30;25;-3)$.\\
		$AB\colon \dfrac{x-2}{30} = \dfrac{y-1}{25} =\dfrac{z-0{,}5}{-3}$.
		\itemch Ta có $AB = \sqrt{(-3)^2 + (-2{,}5)^2 + (0{,}3)^2} =  \sqrt{15{,}34} \approx 3{,}92$  km.
		\itemch Thời gian di chuyển: $10' = \dfrac{1}{6}$ giờ.\\
		Khinh khí cầu thứ nhất $(A)$ di chuyển về hướng Nam:
		\begin{itemize}
			\item Vận tốc: \( 50 \) km/h.
			\item  Quãng đường di chuyển: \( 50\cdot \dfrac{1}{6}=\dfrac{25}{3} \) km.
		\end{itemize}
		Tọa độ $M:
		\heva{&x_M = 2 +\dfrac{25}{3} = \dfrac{31}{3}~ \text{ km} \\&	y_M = 1~ \text{ km} \\&z_M = 0{,}5~ \text{ km}.}$\\
		Suy ra $M\left(\dfrac{31}{3};1;0{,}5\right)$.\\
		Khinh khí cầu thứ hai $(B)$ di chuyển với vectơ \( \overrightarrow{BN} \) cùng hướng với \( \vec{u}(2; 2; 1) \):
		\begin{itemize}
			\item Vận tốc: \( 60 \) km/h.
			\item  Quãng đường di chuyển: \( 60 \cdot \dfrac{1}{6} = 10 \) km.
		\end{itemize}
		Định hướng di chuyển:
		\begin{itemize}
			\item Vectơ hướng: \( \vec{u} = (2; 2; 1) \).
			\item  $\left| \vec{u}\right| = \sqrt{2^2 + 2^2 + 1^2} = 3$.
			\item Vectơ đơn vị	$\vec{e} = \left( \dfrac{2}{3}; \dfrac{2}{3}; \dfrac{1}{3} \right)$.
			\item  Di chuyển:
			$\Delta \vec{BN} = 10\cdot \vec{e} = \left( \dfrac{20}{3}; \dfrac{20}{3}; \dfrac{10}{3} \right)$.
			\item  Tọa độ $N:
			\heva{&	x_N = -1 + \dfrac{20}{3} =\dfrac{17}{3} \text{ km} \\&
			y_N = -1{,}5 +\dfrac{20}{3} = \dfrac{31}{6} \text{ km} \\&
			z_N = 0{,}8 + \dfrac{10}{3} = \dfrac{62}{15} \text{ km}}$
			Suy ra $N\left(\dfrac{17}{3}; \dfrac{31}{6};\dfrac{62}{15}\right)$.
			\item  Khoảng cách \( MN \):\\
			$\overrightarrow{MN} = \left(-\dfrac{14}{3};\dfrac{25}{6}; \dfrac{109}{30}\right)$.
		\end{itemize}
		Suy ra \[
		\left| \overrightarrow{MN}\right| = \sqrt{\left(-\dfrac{14}{3}\right)^2 +\left(\dfrac{25}{6}\right)^2 + \left( \dfrac{109}{30}\right)^2}  \approx 7{,}23 \text{ km}.
		\]
	\end{itemchoice}
	}
\end{ex}

%

\begin{ex}%[Nguồn: Bộ đề minh họa Moon 2024-2025]%[2H5V2-8]
	\immini[thm]{
	Một tháp kiểm soát không lưu ở sân bay cao $109$ m đặt một đài kiểm soát không lưu ở độ cao $105$ m. Máy bay trong phạm vi cách đài kiểm soát $450$ km sẽ hiển thị trên màn hình ra đa. Chọn hệ trục tọa độ $Oxyz$ có gốc $O$ trùng với vị trí chân tháp, mặt phẳng ($Oxy$) trùng với mặt đất sao cho trục $Ox$ là hướng Tây, trục $Oy$ là hướng Nam và trục $Oz$ là trục thẳng đứng (Hình vẽ), đơn vị trên mỗi trục là kilômét.
	}
	{
	\tikzset{khongluu/.pic={
	\definecolor{cdce1eb}{RGB}{220,225,235}
	\definecolor{cafb4c8}{RGB}{175,180,200}
	\definecolor{c82d1f5}{RGB}{130,209,245}
	\definecolor{c8ab0e0}{RGB}{138,176,224}
	\definecolor{ce9edf5}{RGB}{233,237,245}
	\definecolor{caaccfa}{RGB}{170,204,250}
	\definecolor{cbec3d2}{RGB}{190,195,210}
	\begin{scope}
		\path[fill=cdce1eb,nonzero rule] (5.13, 5.58) -- (13.02, 5.58) -- (13.02, 0.21) -- (5.13, 0.21) --cycle
		(5.13, 5.58);
		\path[fill=cafb4c8,nonzero rule] (5.13, 5.58) -- (13.02, 5.58) -- (13.02, 4.44) -- (5.13, 4.44) --cycle
		(5.13, 5.58);
		\path[fill=c82d1f5,nonzero rule] (5.14, 4.44) -- (13.03, 4.44) -- (13.03, 3.3) -- (5.14, 3.3) --cycle
		(5.14, 4.44);
		\path[fill=cdce1eb,nonzero rule] (5.13, 5.58) -- (13.02, 5.58) -- (13.02, 0.21) -- (5.13, 0.21) --cycle
		(5.13, 5.58);
		\path[fill=cdce1eb,nonzero rule] (5.13, 5.58) -- (13.02, 5.58) -- (13.02, 4.44) -- (5.13, 4.44) --cycle
		(5.13, 5.58);
		\path[fill=c8ab0e0,nonzero rule] (5.14, 4.44) -- (13.03, 4.44) -- (13.03, 3.3) -- (5.14, 3.3) --cycle
		(5.14, 4.44);
		\path[fill=ce9edf5,nonzero rule] (5.5, 3.3) -- (13.02, 3.3) -- (13.02, 0.7) -- (6.19, 0.7) .. controls (6.17, 0.7) and (6.14, 0.7) ..
		(6.12, 0.7) .. controls (6.1, 0.7) and (6.08, 0.7) ..
		(6.05, 0.71) .. controls (6.03, 0.71) and (6.01, 0.72) ..
		(5.99, 0.73) .. controls (5.97, 0.73) and (5.94, 0.74) ..
		(5.92, 0.75) .. controls (5.9, 0.76) and (5.88, 0.77) ..
		(5.86, 0.78) .. controls (5.84, 0.79) and (5.82, 0.8) ..
		(5.8, 0.81) .. controls (5.79, 0.82) and (5.77, 0.84) ..
		(5.75, 0.85) .. controls (5.73, 0.87) and (5.72, 0.88) ..
		(5.7, 0.9) .. controls (5.68, 0.91) and (5.67, 0.93) ..
		(5.65, 0.95) .. controls (5.64, 0.97) and (5.63, 0.98) ..
		(5.61, 1) .. controls (5.6, 1.02) and (5.59, 1.04) ..
		(5.58, 1.06) .. controls (5.57, 1.08) and (5.56, 1.1) ..
		(5.55, 1.12) .. controls (5.54, 1.14) and (5.53, 1.16) ..
		(5.53, 1.19) .. controls (5.52, 1.21) and (5.52, 1.23) ..
		(5.51, 1.25) .. controls (5.51, 1.27) and (5.5, 1.3) ..
		(5.5, 1.32) .. controls (5.5, 1.34) and (5.5, 1.36) ..
		(5.5, 1.39) --cycle
		(5.5, 3.3);
		\path[fill=ce9edf5,nonzero rule] (13.02, 5.58) -- (13.02, 4.8) -- (6.07, 4.8) .. controls (6.06, 4.8) and (6.04, 4.8) ..
		(6.02, 4.8) .. controls (6, 4.8) and (5.98, 4.8) ..
		(5.96, 4.81) .. controls (5.94, 4.81) and (5.93, 4.82) ..
		(5.91, 4.82) .. controls (5.89, 4.83) and (5.87, 4.83) ..
		(5.85, 4.84) .. controls (5.84, 4.85) and (5.82, 4.86) ..
		(5.8, 4.86) .. controls (5.79, 4.87) and (5.77, 4.88) ..
		(5.75, 4.89) .. controls (5.74, 4.9) and (5.72, 4.92) ..
		(5.71, 4.93) .. controls (5.69, 4.94) and (5.68, 4.95) ..
		(5.67, 4.97) .. controls (5.65, 4.98) and (5.64, 4.99) ..
		(5.63, 5.01) .. controls (5.62, 5.02) and (5.61, 5.04) ..
		(5.6, 5.05) .. controls (5.58, 5.07) and (5.58, 5.08) ..
		(5.57, 5.1) .. controls (5.56, 5.12) and (5.55, 5.13) ..
		(5.54, 5.15) .. controls (5.53, 5.17) and (5.53, 5.19) ..
		(5.52, 5.21) .. controls (5.52, 5.22) and (5.51, 5.24) ..
		(5.51, 5.26) .. controls (5.51, 5.28) and (5.5, 5.3) ..
		(5.5, 5.32) .. controls (5.5, 5.34) and (5.5, 5.35) ..
		(5.5, 5.37) -- (5.5, 5.58) --cycle
		(13.02, 5.58);
		\path[fill=caaccfa,nonzero rule] (13.03, 4.44) -- (13.03, 3.65) -- (6.08, 3.65) .. controls (6.07, 3.65) and (6.05, 3.66) ..
		(6.03, 3.66) .. controls (6.01, 3.66) and (5.99, 3.66) ..
		(5.97, 3.67) .. controls (5.95, 3.67) and (5.94, 3.67) ..
		(5.92, 3.68) .. controls (5.9, 3.69) and (5.88, 3.69) ..
		(5.86, 3.7) .. controls (5.85, 3.71) and (5.83, 3.71) ..
		(5.81, 3.72) .. controls (5.8, 3.73) and (5.78, 3.74) ..
		(5.76, 3.75) .. controls (5.75, 3.76) and (5.73, 3.77) ..
		(5.72, 3.79) .. controls (5.7, 3.8) and (5.69, 3.81) ..
		(5.68, 3.82) .. controls (5.66, 3.84) and (5.65, 3.85) ..
		(5.64, 3.87) .. controls (5.63, 3.88) and (5.62, 3.9) ..
		(5.6, 3.91) .. controls (5.59, 3.93) and (5.58, 3.94) ..
		(5.58, 3.96) .. controls (5.57, 3.98) and (5.56, 3.99) ..
		(5.55, 4.01) .. controls (5.54, 4.03) and (5.54, 4.05) ..
		(5.53, 4.06) .. controls (5.53, 4.08) and (5.52, 4.1) ..
		(5.52, 4.12) .. controls (5.52, 4.14) and (5.51, 4.16) ..
		(5.51, 4.18) .. controls (5.51, 4.19) and (5.51, 4.21) ..
		(5.51, 4.23) -- (5.51, 4.44) --cycle
		(13.03, 4.44);
		\path[fill=cafb4c8,nonzero rule] (11.51, 9.57) -- (10.8, 8.77) .. controls (10.73, 8.77) and (10.21, 8.8) ..
		(9.54, 8.85) -- (10.47, 10.1) -- (10, 10.12) .. controls (9.94, 10.12) and (9.87, 10.11) ..
		(9.81, 10.09) .. controls (9.75, 10.07) and (9.7, 10.04) ..
		(9.65, 9.99) -- (8.48, 8.92) .. controls (8.17, 8.94) and (7.83, 8.96) ..
		(7.72, 8.97) .. controls (7.25, 8.99) and (6.92, 8.5) ..
		(6.92, 8.5) .. controls (6.92, 8.5) and (7.09, 8.09) ..
		(7.54, 8.07) .. controls (7.91, 8.06) and (10, 8.06) ..
		(10.79, 8.06) .. controls (10.84, 8.06) and (10.89, 8.06) ..
		(10.93, 8.07) .. controls (10.98, 8.08) and (11.02, 8.1) ..
		(11.06, 8.12) .. controls (11.1, 8.14) and (11.14, 8.17) ..
		(11.17, 8.2) .. controls (11.21, 8.23) and (11.24, 8.27) ..
		(11.26, 8.31) -- (12.09, 9.54) --cycle
		(11.51, 9.57);
		\path[fill=cbec3d2,nonzero rule] (11, 8.26) .. controls (11.1, 8.26) and (11.2, 8.29) ..
		(11.28, 8.34) -- (12.09, 9.54) -- (11.51, 9.57) -- (10.8, 8.77) .. controls (10.73, 8.77) and (10.21, 8.8) ..
		(9.54, 8.85) -- (10.47, 10.1) -- (10, 10.12) .. controls (9.94, 10.12) and (9.87, 10.11) ..
		(9.81, 10.09) .. controls (9.75, 10.07) and (9.7, 10.04) ..
		(9.65, 9.99) -- (8.48, 8.92) .. controls (8.17, 8.94) and (7.83, 8.96) ..
		(7.71, 8.97) .. controls (7.49, 8.98) and (7.31, 8.87) ..
		(7.17, 8.76) .. controls (7.14, 8.73) and (7.12, 8.71) ..
		(7.12, 8.71) .. controls (7.12, 8.71) and (7.29, 8.29) ..
		(7.75, 8.27) .. controls (8.12, 8.26) and (10.21, 8.26) ..
		(11, 8.26) --cycle
		(11, 8.26);
		\path[fill=cdce1eb,nonzero rule] (11.61, 4.44) -- (11.82, 4.44) -- (11.82, 3.3) -- (11.61, 3.3) --cycle
		(11.61, 4.44);
		\path[fill=cdce1eb,nonzero rule] (10.3, 4.44) -- (10.5, 4.44) -- (10.5, 3.3) -- (10.3, 3.3) --cycle
		(10.3, 4.44);
		\path[fill=cdce1eb,nonzero rule] (8.98, 4.44) -- (9.19, 4.44) -- (9.19, 3.3) -- (8.98, 3.3) --cycle
		(8.98, 4.44);
		\path[fill=cdce1eb,nonzero rule] (7.67, 4.44) -- (7.87, 4.44) -- (7.87, 3.3) -- (7.67, 3.3) --cycle
		(7.67, 4.44);
		\path[fill=cdce1eb,nonzero rule] (6.35, 4.44) -- (6.56, 4.44) -- (6.56, 3.3) -- (6.35, 3.3) --cycle
		(6.35, 4.44);
		\path[fill=cafb4c8,nonzero rule] (2.28, 13.23) .. controls (2.27, 13.23) and (2.25, 13.23) ..
		(2.24, 13.23) .. controls (2.23, 13.22) and (2.21, 13.22) ..
		(2.2, 13.21) .. controls (2.19, 13.21) and (2.18, 13.2) ..
		(2.16, 13.19) .. controls (2.15, 13.19) and (2.14, 13.18) ..
		(2.13, 13.17) .. controls (2.12, 13.16) and (2.11, 13.15) ..
		(2.11, 13.14) .. controls (2.1, 13.13) and (2.09, 13.11) ..
		(2.09, 13.1) .. controls (2.08, 13.09) and (2.08, 13.08) ..
		(2.08, 13.06) .. controls (2.07, 13.05) and (2.07, 13.04) ..
		(2.07, 13.02) -- (2.07, 11.74) -- (2.49, 11.74) -- (2.49, 13.02) .. controls (2.49, 13.04) and (2.48, 13.05) ..
		(2.48, 13.06) .. controls (2.48, 13.08) and (2.48, 13.09) ..
		(2.47, 13.1) .. controls (2.46, 13.11) and (2.46, 13.13) ..
		(2.45, 13.14) .. controls (2.44, 13.15) and (2.43, 13.16) ..
		(2.43, 13.17) .. controls (2.42, 13.18) and (2.41, 13.19) ..
		(2.39, 13.19) .. controls (2.38, 13.2) and (2.37, 13.21) ..
		(2.36, 13.21) .. controls (2.35, 13.22) and (2.33, 13.22) ..
		(2.32, 13.23) .. controls (2.31, 13.23) and (2.29, 13.23) ..
		(2.28, 13.23) --cycle
		(2.28, 13.23);
		\path[fill=cdce1eb,nonzero rule] (3.93, 10.95) -- (3.93, 10.9) -- (0.63, 10.9) -- (0.63, 10.96) .. controls (0.63, 10.98) and (0.63, 11.01) ..
		(0.63, 11.04) .. controls (0.63, 11.06) and (0.64, 11.09) ..
		(0.64, 11.12) .. controls (0.65, 11.14) and (0.65, 11.17) ..
		(0.66, 11.2) .. controls (0.67, 11.22) and (0.68, 11.25) ..
		(0.69, 11.27) .. controls (0.7, 11.3) and (0.71, 11.32) ..
		(0.72, 11.34) .. controls (0.74, 11.37) and (0.75, 11.39) ..
		(0.76, 11.41) .. controls (0.78, 11.44) and (0.79, 11.46) ..
		(0.81, 11.48) .. controls (0.83, 11.5) and (0.85, 11.52) ..
		(0.87, 11.54) .. controls (0.89, 11.56) and (0.91, 11.58) ..
		(0.93, 11.59) .. controls (0.95, 11.61) and (0.97, 11.63) ..
		(0.99, 11.64) .. controls (1.01, 11.66) and (1.04, 11.67) ..
		(1.06, 11.68) .. controls (1.08, 11.69) and (1.11, 11.71) ..
		(1.13, 11.72) .. controls (1.16, 11.73) and (1.18, 11.74) ..
		(1.21, 11.74) .. controls (1.23, 11.75) and (1.26, 11.76) ..
		(1.29, 11.76) .. controls (1.31, 11.77) and (1.34, 11.77) ..
		(1.37, 11.78) .. controls (1.39, 11.78) and (1.42, 11.78) ..
		(1.45, 11.78) -- (3.11, 11.78) .. controls (3.14, 11.78) and (3.16, 11.78) ..
		(3.19, 11.78) .. controls (3.22, 11.77) and (3.24, 11.77) ..
		(3.27, 11.76) .. controls (3.3, 11.76) and (3.32, 11.75) ..
		(3.35, 11.74) .. controls (3.37, 11.74) and (3.4, 11.73) ..
		(3.42, 11.72) .. controls (3.45, 11.71) and (3.47, 11.69) ..
		(3.5, 11.68) .. controls (3.52, 11.67) and (3.54, 11.66) ..
		(3.57, 11.64) .. controls (3.59, 11.63) and (3.61, 11.61) ..
		(3.63, 11.59) .. controls (3.65, 11.58) and (3.67, 11.56) ..
		(3.69, 11.54) .. controls (3.71, 11.52) and (3.73, 11.5) ..
		(3.75, 11.48) .. controls (3.76, 11.46) and (3.78, 11.44) ..
		(3.79, 11.41) .. controls (3.81, 11.39) and (3.82, 11.37) ..
		(3.84, 11.34) .. controls (3.85, 11.32) and (3.86, 11.3) ..
		(3.87, 11.27) .. controls (3.88, 11.25) and (3.89, 11.22) ..
		(3.9, 11.2) .. controls (3.9, 11.17) and (3.91, 11.14) ..
		(3.92, 11.12) .. controls (3.92, 11.09) and (3.93, 11.06) ..
		(3.93, 11.04) .. controls (3.93, 11.01) and (3.93, 10.98) ..
		(3.93, 10.96) --cycle
		(3.93, 10.95);
		\path[fill=ce9edf5,nonzero rule] (3.11, 11.78) -- (1.45, 11.78) .. controls (1.39, 11.78) and (1.34, 11.77) ..
		(1.29, 11.76) .. controls (1.24, 11.7) and (1.21, 11.64) ..
		(1.19, 11.56) .. controls (1.19, 11.55) and (1.18, 11.54) ..
		(1.18, 11.53) .. controls (1.18, 11.52) and (1.18, 11.51) ..
		(1.18, 11.5) .. controls (1.18, 11.48) and (1.18, 11.47) ..
		(1.18, 11.46) .. controls (1.18, 11.45) and (1.18, 11.44) ..
		(1.18, 11.43) .. controls (1.18, 11.42) and (1.18, 11.4) ..
		(1.19, 11.39) .. controls (1.19, 11.38) and (1.19, 11.37) ..
		(1.2, 11.36) .. controls (1.2, 11.35) and (1.21, 11.34) ..
		(1.21, 11.33) .. controls (1.22, 11.32) and (1.22, 11.31) ..
		(1.23, 11.3) .. controls (1.24, 11.29) and (1.25, 11.28) ..
		(1.25, 11.27) .. controls (1.26, 11.27) and (1.27, 11.26) ..
		(1.28, 11.25) .. controls (1.29, 11.24) and (1.3, 11.24) ..
		(1.31, 11.23) .. controls (1.31, 11.22) and (1.32, 11.22) ..
		(1.33, 11.21) .. controls (1.35, 11.21) and (1.36, 11.2) ..
		(1.37, 11.2) .. controls (1.38, 11.19) and (1.39, 11.19) ..
		(1.4, 11.19) .. controls (1.41, 11.19) and (1.42, 11.18) ..
		(1.43, 11.18) .. controls (1.44, 11.18) and (1.46, 11.18) ..
		(1.47, 11.18) -- (3.9, 11.18) .. controls (3.89, 11.22) and (3.87, 11.26) ..
		(3.86, 11.31) .. controls (3.84, 11.35) and (3.81, 11.38) ..
		(3.79, 11.42) .. controls (3.76, 11.46) and (3.74, 11.49) ..
		(3.71, 11.52) .. controls (3.67, 11.56) and (3.64, 11.59) ..
		(3.61, 11.61) .. controls (3.57, 11.64) and (3.53, 11.66) ..
		(3.49, 11.68) .. controls (3.45, 11.7) and (3.41, 11.72) ..
		(3.37, 11.74) .. controls (3.33, 11.75) and (3.29, 11.76) ..
		(3.24, 11.77) .. controls (3.2, 11.78) and (3.15, 11.78) ..
		(3.11, 11.78) --cycle
		(3.11, 11.78);
		\path[fill=cdce1eb,nonzero rule] (0.83, 7.63) .. controls (0.83, 7.63) and (1.41, 6.21) ..
		(1.41, 4.53) -- (1.41, 0.21) -- (3.15, 0.21) -- (3.15, 4.53) .. controls (3.15, 6.21) and (3.73, 7.63) ..
		(3.73, 7.63) -- (3.73, 7.69) -- (0.83, 7.69) --cycle
		(0.83, 7.63);
		\path[fill=ce9edf5,nonzero rule] (1.93, 5.02) -- (1.93, 1.54) .. controls (1.93, 1.51) and (1.93, 1.48) ..
		(1.93, 1.46) .. controls (1.93, 1.43) and (1.94, 1.4) ..
		(1.94, 1.38) .. controls (1.95, 1.35) and (1.95, 1.32) ..
		(1.96, 1.29) .. controls (1.97, 1.27) and (1.98, 1.24) ..
		(1.99, 1.22) .. controls (2, 1.19) and (2.01, 1.17) ..
		(2.03, 1.14) .. controls (2.04, 1.12) and (2.05, 1.09) ..
		(2.07, 1.07) .. controls (2.08, 1.05) and (2.1, 1.03) ..
		(2.12, 1) .. controls (2.14, 0.98) and (2.15, 0.96) ..
		(2.17, 0.94) .. controls (2.19, 0.92) and (2.21, 0.9) ..
		(2.23, 0.89) .. controls (2.26, 0.87) and (2.28, 0.85) ..
		(2.3, 0.84) .. controls (2.32, 0.82) and (2.35, 0.81) ..
		(2.37, 0.8) .. controls (2.4, 0.78) and (2.42, 0.77) ..
		(2.45, 0.76) .. controls (2.47, 0.75) and (2.5, 0.74) ..
		(2.53, 0.73) .. controls (2.55, 0.72) and (2.58, 0.72) ..
		(2.61, 0.71) .. controls (2.63, 0.71) and (2.66, 0.7) ..
		(2.69, 0.7) .. controls (2.71, 0.7) and (2.74, 0.7) ..
		(2.77, 0.7) -- (3.15, 0.7) -- (3.15, 4.53) .. controls (3.15, 6.21) and (3.73, 7.63) ..
		(3.73, 7.63) -- (3.73, 7.69) -- (1.5, 7.69) .. controls (1.67, 7.15) and (1.93, 6.14) ..
		(1.93, 5.02) --cycle
		(1.93, 5.02);
		
		\path[fill=cafb4c8,nonzero rule] (13.02, -0) -- (0.21, -0) .. controls (0.2, -0) and (0.18, 0) ..
		(0.17, 0) .. controls (0.16, 0.01) and (0.15, 0.01) ..
		(0.13, 0.02) .. controls (0.12, 0.02) and (0.11, 0.03) ..
		(0.1, 0.03) .. controls (0.09, 0.04) and (0.08, 0.05) ..
		(0.07, 0.06) .. controls (0.06, 0.07) and (0.05, 0.08) ..
		(0.04, 0.09) .. controls (0.03, 0.1) and (0.03, 0.12) ..
		(0.02, 0.13) .. controls (0.02, 0.14) and (0.01, 0.15) ..
		(0.01, 0.17) .. controls (0.01, 0.18) and (0.01, 0.19) ..
		(0.01, 0.21) .. controls (0.01, 0.22) and (0.01, 0.23) ..
		(0.01, 0.25) .. controls (0.01, 0.26) and (0.02, 0.27) ..
		(0.02, 0.29) .. controls (0.03, 0.3) and (0.03, 0.31) ..
		(0.04, 0.32) .. controls (0.05, 0.33) and (0.06, 0.34) ..
		(0.07, 0.35) .. controls (0.08, 0.36) and (0.09, 0.37) ..
		(0.1, 0.38) .. controls (0.11, 0.39) and (0.12, 0.39) ..
		(0.13, 0.4) .. controls (0.15, 0.4) and (0.16, 0.41) ..
		(0.17, 0.41) .. controls (0.18, 0.41) and (0.2, 0.41) ..
		(0.21, 0.41) -- (13.02, 0.41) .. controls (13.03, 0.41) and (13.04, 0.41) ..
		(13.06, 0.41) .. controls (13.07, 0.41) and (13.08, 0.4) ..
		(13.1, 0.4) .. controls (13.11, 0.39) and (13.12, 0.39) ..
		(13.13, 0.38) .. controls (13.14, 0.37) and (13.15, 0.36) ..
		(13.16, 0.35) .. controls (13.17, 0.34) and (13.18, 0.33) ..
		(13.19, 0.32) .. controls (13.2, 0.31) and (13.2, 0.3) ..
		(13.21, 0.29) .. controls (13.21, 0.27) and (13.22, 0.26) ..
		(13.22, 0.25) .. controls (13.22, 0.23) and (13.22, 0.22) ..
		(13.22, 0.21) .. controls (13.22, 0.19) and (13.22, 0.18) ..
		(13.22, 0.17) .. controls (13.22, 0.15) and (13.21, 0.14) ..
		(13.21, 0.13) .. controls (13.2, 0.12) and (13.2, 0.1) ..
		(13.19, 0.09) .. controls (13.18, 0.08) and (13.17, 0.07) ..
		(13.16, 0.06) .. controls (13.15, 0.05) and (13.14, 0.04) ..
		(13.13, 0.03) .. controls (13.12, 0.03) and (13.11, 0.02) ..
		(13.1, 0.02) .. controls (13.08, 0.01) and (13.07, 0.01) ..
		(13.06, 0) .. controls (13.04, 0) and (13.03, -0) ..
		(13.02, -0) --cycle
		(13.02, -0);
		
		\path[fill=c8ab0e0,nonzero rule] (0.63, 10.95) -- (3.93, 10.95) -- (3.93, 10.09) -- (0.63, 10.09) --cycle
		(0.63, 10.95);
		
		\path[fill=caaccfa,nonzero rule] (1.26, 10.38) -- (3.93, 10.38) -- (3.93, 10.95) -- (1.01, 10.95) -- (1.01, 10.64) .. controls (1.01, 10.62) and (1.01, 10.6) ..
		(1.02, 10.59) .. controls (1.02, 10.57) and (1.02, 10.56) ..
		(1.03, 10.54) .. controls (1.04, 10.53) and (1.04, 10.51) ..
		(1.05, 10.5) .. controls (1.06, 10.48) and (1.07, 10.47) ..
		(1.08, 10.46) .. controls (1.1, 10.45) and (1.11, 10.44) ..
		(1.12, 10.43) .. controls (1.14, 10.42) and (1.15, 10.41) ..
		(1.17, 10.4) .. controls (1.18, 10.4) and (1.2, 10.39) ..
		(1.21, 10.39) .. controls (1.23, 10.39) and (1.25, 10.38) ..
		(1.26, 10.38) --cycle
		(1.26, 10.38);
		
		\path[fill=cdce1eb,nonzero rule] (2.18, 10.96) -- (2.38, 10.96) -- (2.38, 10.08) -- (2.18, 10.08) --cycle
		(2.18, 10.96);
		
		\path[fill=cdce1eb,nonzero rule] (1.35, 10.96) -- (1.56, 10.96) -- (1.56, 10.08) -- (1.35, 10.08) --cycle
		(1.35, 10.96);
		
		\path[fill=cdce1eb,nonzero rule] (3, 10.96) -- (3.21, 10.96) -- (3.21, 10.08) -- (3, 10.08) --cycle
		(3, 10.96);
		
		\path[fill=c8ab0e0,nonzero rule] (0.83, 9.34) -- (3.73, 9.34) -- (3.73, 7.63) -- (0.83, 7.63) --cycle
		(0.83, 9.34);
		
		\path[fill=caaccfa,nonzero rule] (1.61, 7.91) -- (3.73, 7.91) -- (3.73, 9.34) -- (1.25, 9.34) -- (1.25, 8.27) .. controls (1.25, 8.24) and (1.25, 8.22) ..
		(1.26, 8.2) .. controls (1.26, 8.17) and (1.27, 8.15) ..
		(1.28, 8.13) .. controls (1.29, 8.11) and (1.3, 8.09) ..
		(1.31, 8.07) .. controls (1.33, 8.05) and (1.34, 8.03) ..
		(1.36, 8.01) .. controls (1.37, 8) and (1.39, 7.98) ..
		(1.41, 7.97) .. controls (1.43, 7.95) and (1.45, 7.94) ..
		(1.48, 7.93) .. controls (1.5, 7.92) and (1.52, 7.92) ..
		(1.54, 7.91) .. controls (1.57, 7.91) and (1.59, 7.91) ..
		(1.61, 7.91) --cycle
		(1.61, 7.91);
		
		\path[fill=cdce1eb,nonzero rule] (3.73, 8.57) -- (2.38, 8.57) -- (2.38, 9.34) -- (2.18, 9.34) -- (2.18, 8.57) -- (0.83, 8.57) -- (0.83, 8.36) -- (2.18, 8.36) -- (2.18, 7.63) -- (2.38, 7.63) -- (2.38, 8.36) -- (3.73, 8.36) --cycle
		(3.73, 8.57);
		
		\path[fill=cdce1eb,nonzero rule] (0.63, 10.12) -- (3.93, 10.12) .. controls (3.96, 10.12) and (3.99, 10.12) ..
		(4.01, 10.11) .. controls (4.04, 10.11) and (4.07, 10.1) ..
		(4.09, 10.09) .. controls (4.12, 10.08) and (4.14, 10.06) ..
		(4.16, 10.05) .. controls (4.18, 10.03) and (4.21, 10.02) ..
		(4.22, 10) .. controls (4.24, 9.98) and (4.26, 9.96) ..
		(4.28, 9.94) .. controls (4.29, 9.91) and (4.3, 9.89) ..
		(4.31, 9.86) .. controls (4.32, 9.84) and (4.33, 9.81) ..
		(4.34, 9.79) .. controls (4.34, 9.76) and (4.35, 9.73) ..
		(4.35, 9.71) .. controls (4.35, 9.68) and (4.34, 9.65) ..
		(4.34, 9.63) .. controls (4.33, 9.6) and (4.32, 9.57) ..
		(4.31, 9.55) .. controls (4.3, 9.52) and (4.29, 9.5) ..
		(4.28, 9.48) .. controls (4.26, 9.45) and (4.24, 9.43) ..
		(4.22, 9.41) .. controls (4.21, 9.39) and (4.18, 9.38) ..
		(4.16, 9.36) .. controls (4.14, 9.35) and (4.12, 9.33) ..
		(4.09, 9.32) .. controls (4.07, 9.31) and (4.04, 9.31) ..
		(4.01, 9.3) .. controls (3.99, 9.3) and (3.96, 9.29) ..
		(3.93, 9.29) -- (0.63, 9.29) .. controls (0.6, 9.29) and (0.57, 9.3) ..
		(0.54, 9.3) .. controls (0.52, 9.31) and (0.49, 9.31) ..
		(0.47, 9.32) .. controls (0.44, 9.33) and (0.42, 9.35) ..
		(0.4, 9.36) .. controls (0.37, 9.38) and (0.35, 9.39) ..
		(0.33, 9.41) .. controls (0.31, 9.43) and (0.3, 9.45) ..
		(0.28, 9.48) .. controls (0.27, 9.5) and (0.25, 9.52) ..
		(0.24, 9.55) .. controls (0.23, 9.57) and (0.23, 9.6) ..
		(0.22, 9.63) .. controls (0.21, 9.65) and (0.21, 9.68) ..
		(0.21, 9.71) .. controls (0.21, 9.73) and (0.21, 9.76) ..
		(0.22, 9.79) .. controls (0.23, 9.81) and (0.23, 9.84) ..
		(0.24, 9.86) .. controls (0.25, 9.89) and (0.27, 9.91) ..
		(0.28, 9.94) .. controls (0.3, 9.96) and (0.31, 9.98) ..
		(0.33, 10) .. controls (0.35, 10.02) and (0.37, 10.03) ..
		(0.4, 10.05) .. controls (0.42, 10.06) and (0.44, 10.08) ..
		(0.47, 10.09) .. controls (0.49, 10.1) and (0.52, 10.11) ..
		(0.54, 10.11) .. controls (0.57, 10.12) and (0.6, 10.12) ..
		(0.63, 10.12) --cycle
		(0.63, 10.12);
		
		\path[fill=ce9edf5,nonzero rule] (3.93, 10.12) -- (0.63, 10.12) .. controls (0.56, 10.12) and (0.49, 10.1) ..
		(0.43, 10.07) .. controls (0.43, 10.06) and (0.43, 10.04) ..
		(0.42, 10.03) .. controls (0.42, 10.01) and (0.42, 9.99) ..
		(0.42, 9.98) .. controls (0.42, 9.96) and (0.42, 9.95) ..
		(0.42, 9.93) .. controls (0.42, 9.91) and (0.42, 9.9) ..
		(0.43, 9.88) .. controls (0.43, 9.87) and (0.43, 9.85) ..
		(0.44, 9.84) .. controls (0.44, 9.82) and (0.45, 9.81) ..
		(0.45, 9.79) .. controls (0.46, 9.78) and (0.47, 9.76) ..
		(0.48, 9.75) .. controls (0.48, 9.74) and (0.49, 9.72) ..
		(0.5, 9.71) .. controls (0.51, 9.7) and (0.52, 9.68) ..
		(0.53, 9.67) .. controls (0.54, 9.66) and (0.56, 9.65) ..
		(0.57, 9.64) .. controls (0.58, 9.63) and (0.59, 9.62) ..
		(0.61, 9.61) .. controls (0.62, 9.6) and (0.63, 9.6) ..
		(0.65, 9.59) .. controls (0.66, 9.58) and (0.68, 9.58) ..
		(0.69, 9.57) .. controls (0.71, 9.56) and (0.72, 9.56) ..
		(0.74, 9.56) .. controls (0.75, 9.55) and (0.77, 9.55) ..
		(0.78, 9.55) .. controls (0.8, 9.55) and (0.82, 9.55) ..
		(0.83, 9.55) -- (4.14, 9.55) .. controls (4.21, 9.55) and (4.27, 9.56) ..
		(4.33, 9.59) .. controls (4.34, 9.63) and (4.35, 9.67) ..
		(4.35, 9.71) .. controls (4.35, 9.73) and (4.34, 9.76) ..
		(4.34, 9.79) .. controls (4.33, 9.81) and (4.32, 9.84) ..
		(4.31, 9.86) .. controls (4.3, 9.89) and (4.29, 9.91) ..
		(4.28, 9.94) .. controls (4.26, 9.96) and (4.24, 9.98) ..
		(4.22, 10) .. controls (4.21, 10.02) and (4.18, 10.03) ..
		(4.16, 10.05) .. controls (4.14, 10.06) and (4.12, 10.08) ..
		(4.09, 10.09) .. controls (4.07, 10.1) and (4.04, 10.11) ..
		(4.01, 10.11) .. controls (3.99, 10.12) and (3.96, 10.12) ..
		(3.93, 10.12) --cycle
		(3.93, 10.12);
	\end{scope}
	}}
	\begin{tikzpicture}[scale=.7,transform shape]
		\path (0,0) pic[scale=.3]{khongluu};
		\path (.685,0) coordinate (O)
		+(0:4) coordinate (y) node[above] {$y$}
		+(90:5) coordinate (z) node[above] {$z$}
		+(-145:3) coordinate (x) node[above] {$x$}
		;
		\draw[-stealth] (O)--(z);
		\draw[-stealth] (O)--(y);
		\draw[-stealth] (O)--(x);
	\end{tikzpicture}
	}
	Một máy bay đang ở vị trí $A$ cách mặt đất $8$ km, cách $268$ km về phía Đông, $185$ km về phía Nam so với tháp kiểm soát không lưu và đang chuyển động theo đường thẳng $d$ có vectơ chỉ phương là $\overrightarrow{u}=(82;76;0)$ hướng về đài kiểm soát không lưu.
	\choiceTF
	{Vị trí $A$ có tọa độ là $(268;-185;-8)$}
	{\True Đài kiểm soát không lưu có phát hiện được máy bay tại vị trí $A$}
	{\True Phương trình tham số của đường thẳng $d$ là $\heva{&x=-268+82t\\&y=185+76t\\&z=8}$ ($t$ là tham số)}
	{Khoảng cách gần nhất giữa máy bay và đài kiểm soát không lưu là $217{,}98$ km (làm tròn kết quả đến hàng phần trăm)}
	\loigiai{
	\begin{center}
		\tikzset{khongluu/.pic={
		\definecolor{cdce1eb}{RGB}{220,225,235}
		\definecolor{cafb4c8}{RGB}{175,180,200}
		\definecolor{c82d1f5}{RGB}{130,209,245}
		\definecolor{c8ab0e0}{RGB}{138,176,224}
		\definecolor{ce9edf5}{RGB}{233,237,245}
		\definecolor{caaccfa}{RGB}{170,204,250}
		\definecolor{cbec3d2}{RGB}{190,195,210}
		\begin{scope}
			\path[fill=cdce1eb,nonzero rule] (5.13, 5.58) -- (13.02, 5.58) -- (13.02, 0.21) -- (5.13, 0.21) --cycle
			(5.13, 5.58);
			\path[fill=cafb4c8,nonzero rule] (5.13, 5.58) -- (13.02, 5.58) -- (13.02, 4.44) -- (5.13, 4.44) --cycle
			(5.13, 5.58);
			\path[fill=c82d1f5,nonzero rule] (5.14, 4.44) -- (13.03, 4.44) -- (13.03, 3.3) -- (5.14, 3.3) --cycle
			(5.14, 4.44);
			\path[fill=cdce1eb,nonzero rule] (5.13, 5.58) -- (13.02, 5.58) -- (13.02, 0.21) -- (5.13, 0.21) --cycle
			(5.13, 5.58);
			\path[fill=cdce1eb,nonzero rule] (5.13, 5.58) -- (13.02, 5.58) -- (13.02, 4.44) -- (5.13, 4.44) --cycle
			(5.13, 5.58);
			\path[fill=c8ab0e0,nonzero rule] (5.14, 4.44) -- (13.03, 4.44) -- (13.03, 3.3) -- (5.14, 3.3) --cycle
			(5.14, 4.44);
			\path[fill=ce9edf5,nonzero rule] (5.5, 3.3) -- (13.02, 3.3) -- (13.02, 0.7) -- (6.19, 0.7) .. controls (6.17, 0.7) and (6.14, 0.7) ..
			(6.12, 0.7) .. controls (6.1, 0.7) and (6.08, 0.7) ..
			(6.05, 0.71) .. controls (6.03, 0.71) and (6.01, 0.72) ..
			(5.99, 0.73) .. controls (5.97, 0.73) and (5.94, 0.74) ..
			(5.92, 0.75) .. controls (5.9, 0.76) and (5.88, 0.77) ..
			(5.86, 0.78) .. controls (5.84, 0.79) and (5.82, 0.8) ..
			(5.8, 0.81) .. controls (5.79, 0.82) and (5.77, 0.84) ..
			(5.75, 0.85) .. controls (5.73, 0.87) and (5.72, 0.88) ..
			(5.7, 0.9) .. controls (5.68, 0.91) and (5.67, 0.93) ..
			(5.65, 0.95) .. controls (5.64, 0.97) and (5.63, 0.98) ..
			(5.61, 1) .. controls (5.6, 1.02) and (5.59, 1.04) ..
			(5.58, 1.06) .. controls (5.57, 1.08) and (5.56, 1.1) ..
			(5.55, 1.12) .. controls (5.54, 1.14) and (5.53, 1.16) ..
			(5.53, 1.19) .. controls (5.52, 1.21) and (5.52, 1.23) ..
			(5.51, 1.25) .. controls (5.51, 1.27) and (5.5, 1.3) ..
			(5.5, 1.32) .. controls (5.5, 1.34) and (5.5, 1.36) ..
			(5.5, 1.39) --cycle
			(5.5, 3.3);
			\path[fill=ce9edf5,nonzero rule] (13.02, 5.58) -- (13.02, 4.8) -- (6.07, 4.8) .. controls (6.06, 4.8) and (6.04, 4.8) ..
			(6.02, 4.8) .. controls (6, 4.8) and (5.98, 4.8) ..
			(5.96, 4.81) .. controls (5.94, 4.81) and (5.93, 4.82) ..
			(5.91, 4.82) .. controls (5.89, 4.83) and (5.87, 4.83) ..
			(5.85, 4.84) .. controls (5.84, 4.85) and (5.82, 4.86) ..
			(5.8, 4.86) .. controls (5.79, 4.87) and (5.77, 4.88) ..
			(5.75, 4.89) .. controls (5.74, 4.9) and (5.72, 4.92) ..
			(5.71, 4.93) .. controls (5.69, 4.94) and (5.68, 4.95) ..
			(5.67, 4.97) .. controls (5.65, 4.98) and (5.64, 4.99) ..
			(5.63, 5.01) .. controls (5.62, 5.02) and (5.61, 5.04) ..
			(5.6, 5.05) .. controls (5.58, 5.07) and (5.58, 5.08) ..
			(5.57, 5.1) .. controls (5.56, 5.12) and (5.55, 5.13) ..
			(5.54, 5.15) .. controls (5.53, 5.17) and (5.53, 5.19) ..
			(5.52, 5.21) .. controls (5.52, 5.22) and (5.51, 5.24) ..
			(5.51, 5.26) .. controls (5.51, 5.28) and (5.5, 5.3) ..
			(5.5, 5.32) .. controls (5.5, 5.34) and (5.5, 5.35) ..
			(5.5, 5.37) -- (5.5, 5.58) --cycle
			(13.02, 5.58);
			\path[fill=caaccfa,nonzero rule] (13.03, 4.44) -- (13.03, 3.65) -- (6.08, 3.65) .. controls (6.07, 3.65) and (6.05, 3.66) ..
			(6.03, 3.66) .. controls (6.01, 3.66) and (5.99, 3.66) ..
			(5.97, 3.67) .. controls (5.95, 3.67) and (5.94, 3.67) ..
			(5.92, 3.68) .. controls (5.9, 3.69) and (5.88, 3.69) ..
			(5.86, 3.7) .. controls (5.85, 3.71) and (5.83, 3.71) ..
			(5.81, 3.72) .. controls (5.8, 3.73) and (5.78, 3.74) ..
			(5.76, 3.75) .. controls (5.75, 3.76) and (5.73, 3.77) ..
			(5.72, 3.79) .. controls (5.7, 3.8) and (5.69, 3.81) ..
			(5.68, 3.82) .. controls (5.66, 3.84) and (5.65, 3.85) ..
			(5.64, 3.87) .. controls (5.63, 3.88) and (5.62, 3.9) ..
			(5.6, 3.91) .. controls (5.59, 3.93) and (5.58, 3.94) ..
			(5.58, 3.96) .. controls (5.57, 3.98) and (5.56, 3.99) ..
			(5.55, 4.01) .. controls (5.54, 4.03) and (5.54, 4.05) ..
			(5.53, 4.06) .. controls (5.53, 4.08) and (5.52, 4.1) ..
			(5.52, 4.12) .. controls (5.52, 4.14) and (5.51, 4.16) ..
			(5.51, 4.18) .. controls (5.51, 4.19) and (5.51, 4.21) ..
			(5.51, 4.23) -- (5.51, 4.44) --cycle
			(13.03, 4.44);
			\path[fill=cafb4c8,nonzero rule] (11.51, 9.57) -- (10.8, 8.77) .. controls (10.73, 8.77) and (10.21, 8.8) ..
			(9.54, 8.85) -- (10.47, 10.1) -- (10, 10.12) .. controls (9.94, 10.12) and (9.87, 10.11) ..
			(9.81, 10.09) .. controls (9.75, 10.07) and (9.7, 10.04) ..
			(9.65, 9.99) -- (8.48, 8.92) .. controls (8.17, 8.94) and (7.83, 8.96) ..
			(7.72, 8.97) .. controls (7.25, 8.99) and (6.92, 8.5) ..
			(6.92, 8.5) .. controls (6.92, 8.5) and (7.09, 8.09) ..
			(7.54, 8.07) .. controls (7.91, 8.06) and (10, 8.06) ..
			(10.79, 8.06) .. controls (10.84, 8.06) and (10.89, 8.06) ..
			(10.93, 8.07) .. controls (10.98, 8.08) and (11.02, 8.1) ..
			(11.06, 8.12) .. controls (11.1, 8.14) and (11.14, 8.17) ..
			(11.17, 8.2) .. controls (11.21, 8.23) and (11.24, 8.27) ..
			(11.26, 8.31) -- (12.09, 9.54) --cycle
			(11.51, 9.57);
			\path[fill=cbec3d2,nonzero rule] (11, 8.26) .. controls (11.1, 8.26) and (11.2, 8.29) ..
			(11.28, 8.34) -- (12.09, 9.54) -- (11.51, 9.57) -- (10.8, 8.77) .. controls (10.73, 8.77) and (10.21, 8.8) ..
			(9.54, 8.85) -- (10.47, 10.1) -- (10, 10.12) .. controls (9.94, 10.12) and (9.87, 10.11) ..
			(9.81, 10.09) .. controls (9.75, 10.07) and (9.7, 10.04) ..
			(9.65, 9.99) -- (8.48, 8.92) .. controls (8.17, 8.94) and (7.83, 8.96) ..
			(7.71, 8.97) .. controls (7.49, 8.98) and (7.31, 8.87) ..
			(7.17, 8.76) .. controls (7.14, 8.73) and (7.12, 8.71) ..
			(7.12, 8.71) .. controls (7.12, 8.71) and (7.29, 8.29) ..
			(7.75, 8.27) .. controls (8.12, 8.26) and (10.21, 8.26) ..
			(11, 8.26) --cycle
			(11, 8.26);
			\path[fill=cdce1eb,nonzero rule] (11.61, 4.44) -- (11.82, 4.44) -- (11.82, 3.3) -- (11.61, 3.3) --cycle
			(11.61, 4.44);
			\path[fill=cdce1eb,nonzero rule] (10.3, 4.44) -- (10.5, 4.44) -- (10.5, 3.3) -- (10.3, 3.3) --cycle
			(10.3, 4.44);
			\path[fill=cdce1eb,nonzero rule] (8.98, 4.44) -- (9.19, 4.44) -- (9.19, 3.3) -- (8.98, 3.3) --cycle
			(8.98, 4.44);
			\path[fill=cdce1eb,nonzero rule] (7.67, 4.44) -- (7.87, 4.44) -- (7.87, 3.3) -- (7.67, 3.3) --cycle
			(7.67, 4.44);
			\path[fill=cdce1eb,nonzero rule] (6.35, 4.44) -- (6.56, 4.44) -- (6.56, 3.3) -- (6.35, 3.3) --cycle
			(6.35, 4.44);
			\path[fill=cafb4c8,nonzero rule] (2.28, 13.23) .. controls (2.27, 13.23) and (2.25, 13.23) ..
			(2.24, 13.23) .. controls (2.23, 13.22) and (2.21, 13.22) ..
			(2.2, 13.21) .. controls (2.19, 13.21) and (2.18, 13.2) ..
			(2.16, 13.19) .. controls (2.15, 13.19) and (2.14, 13.18) ..
			(2.13, 13.17) .. controls (2.12, 13.16) and (2.11, 13.15) ..
			(2.11, 13.14) .. controls (2.1, 13.13) and (2.09, 13.11) ..
			(2.09, 13.1) .. controls (2.08, 13.09) and (2.08, 13.08) ..
			(2.08, 13.06) .. controls (2.07, 13.05) and (2.07, 13.04) ..
			(2.07, 13.02) -- (2.07, 11.74) -- (2.49, 11.74) -- (2.49, 13.02) .. controls (2.49, 13.04) and (2.48, 13.05) ..
			(2.48, 13.06) .. controls (2.48, 13.08) and (2.48, 13.09) ..
			(2.47, 13.1) .. controls (2.46, 13.11) and (2.46, 13.13) ..
			(2.45, 13.14) .. controls (2.44, 13.15) and (2.43, 13.16) ..
			(2.43, 13.17) .. controls (2.42, 13.18) and (2.41, 13.19) ..
			(2.39, 13.19) .. controls (2.38, 13.2) and (2.37, 13.21) ..
			(2.36, 13.21) .. controls (2.35, 13.22) and (2.33, 13.22) ..
			(2.32, 13.23) .. controls (2.31, 13.23) and (2.29, 13.23) ..
			(2.28, 13.23) --cycle
			(2.28, 13.23);
			\path[fill=cdce1eb,nonzero rule] (3.93, 10.95) -- (3.93, 10.9) -- (0.63, 10.9) -- (0.63, 10.96) .. controls (0.63, 10.98) and (0.63, 11.01) ..
			(0.63, 11.04) .. controls (0.63, 11.06) and (0.64, 11.09) ..
			(0.64, 11.12) .. controls (0.65, 11.14) and (0.65, 11.17) ..
			(0.66, 11.2) .. controls (0.67, 11.22) and (0.68, 11.25) ..
			(0.69, 11.27) .. controls (0.7, 11.3) and (0.71, 11.32) ..
			(0.72, 11.34) .. controls (0.74, 11.37) and (0.75, 11.39) ..
			(0.76, 11.41) .. controls (0.78, 11.44) and (0.79, 11.46) ..
			(0.81, 11.48) .. controls (0.83, 11.5) and (0.85, 11.52) ..
			(0.87, 11.54) .. controls (0.89, 11.56) and (0.91, 11.58) ..
			(0.93, 11.59) .. controls (0.95, 11.61) and (0.97, 11.63) ..
			(0.99, 11.64) .. controls (1.01, 11.66) and (1.04, 11.67) ..
			(1.06, 11.68) .. controls (1.08, 11.69) and (1.11, 11.71) ..
			(1.13, 11.72) .. controls (1.16, 11.73) and (1.18, 11.74) ..
			(1.21, 11.74) .. controls (1.23, 11.75) and (1.26, 11.76) ..
			(1.29, 11.76) .. controls (1.31, 11.77) and (1.34, 11.77) ..
			(1.37, 11.78) .. controls (1.39, 11.78) and (1.42, 11.78) ..
			(1.45, 11.78) -- (3.11, 11.78) .. controls (3.14, 11.78) and (3.16, 11.78) ..
			(3.19, 11.78) .. controls (3.22, 11.77) and (3.24, 11.77) ..
			(3.27, 11.76) .. controls (3.3, 11.76) and (3.32, 11.75) ..
			(3.35, 11.74) .. controls (3.37, 11.74) and (3.4, 11.73) ..
			(3.42, 11.72) .. controls (3.45, 11.71) and (3.47, 11.69) ..
			(3.5, 11.68) .. controls (3.52, 11.67) and (3.54, 11.66) ..
			(3.57, 11.64) .. controls (3.59, 11.63) and (3.61, 11.61) ..
			(3.63, 11.59) .. controls (3.65, 11.58) and (3.67, 11.56) ..
			(3.69, 11.54) .. controls (3.71, 11.52) and (3.73, 11.5) ..
			(3.75, 11.48) .. controls (3.76, 11.46) and (3.78, 11.44) ..
			(3.79, 11.41) .. controls (3.81, 11.39) and (3.82, 11.37) ..
			(3.84, 11.34) .. controls (3.85, 11.32) and (3.86, 11.3) ..
			(3.87, 11.27) .. controls (3.88, 11.25) and (3.89, 11.22) ..
			(3.9, 11.2) .. controls (3.9, 11.17) and (3.91, 11.14) ..
			(3.92, 11.12) .. controls (3.92, 11.09) and (3.93, 11.06) ..
			(3.93, 11.04) .. controls (3.93, 11.01) and (3.93, 10.98) ..
			(3.93, 10.96) --cycle
			(3.93, 10.95);
			\path[fill=ce9edf5,nonzero rule] (3.11, 11.78) -- (1.45, 11.78) .. controls (1.39, 11.78) and (1.34, 11.77) ..
			(1.29, 11.76) .. controls (1.24, 11.7) and (1.21, 11.64) ..
			(1.19, 11.56) .. controls (1.19, 11.55) and (1.18, 11.54) ..
			(1.18, 11.53) .. controls (1.18, 11.52) and (1.18, 11.51) ..
			(1.18, 11.5) .. controls (1.18, 11.48) and (1.18, 11.47) ..
			(1.18, 11.46) .. controls (1.18, 11.45) and (1.18, 11.44) ..
			(1.18, 11.43) .. controls (1.18, 11.42) and (1.18, 11.4) ..
			(1.19, 11.39) .. controls (1.19, 11.38) and (1.19, 11.37) ..
			(1.2, 11.36) .. controls (1.2, 11.35) and (1.21, 11.34) ..
			(1.21, 11.33) .. controls (1.22, 11.32) and (1.22, 11.31) ..
			(1.23, 11.3) .. controls (1.24, 11.29) and (1.25, 11.28) ..
			(1.25, 11.27) .. controls (1.26, 11.27) and (1.27, 11.26) ..
			(1.28, 11.25) .. controls (1.29, 11.24) and (1.3, 11.24) ..
			(1.31, 11.23) .. controls (1.31, 11.22) and (1.32, 11.22) ..
			(1.33, 11.21) .. controls (1.35, 11.21) and (1.36, 11.2) ..
			(1.37, 11.2) .. controls (1.38, 11.19) and (1.39, 11.19) ..
			(1.4, 11.19) .. controls (1.41, 11.19) and (1.42, 11.18) ..
			(1.43, 11.18) .. controls (1.44, 11.18) and (1.46, 11.18) ..
			(1.47, 11.18) -- (3.9, 11.18) .. controls (3.89, 11.22) and (3.87, 11.26) ..
			(3.86, 11.31) .. controls (3.84, 11.35) and (3.81, 11.38) ..
			(3.79, 11.42) .. controls (3.76, 11.46) and (3.74, 11.49) ..
			(3.71, 11.52) .. controls (3.67, 11.56) and (3.64, 11.59) ..
			(3.61, 11.61) .. controls (3.57, 11.64) and (3.53, 11.66) ..
			(3.49, 11.68) .. controls (3.45, 11.7) and (3.41, 11.72) ..
			(3.37, 11.74) .. controls (3.33, 11.75) and (3.29, 11.76) ..
			(3.24, 11.77) .. controls (3.2, 11.78) and (3.15, 11.78) ..
			(3.11, 11.78) --cycle
			(3.11, 11.78);
			\path[fill=cdce1eb,nonzero rule] (0.83, 7.63) .. controls (0.83, 7.63) and (1.41, 6.21) ..
			(1.41, 4.53) -- (1.41, 0.21) -- (3.15, 0.21) -- (3.15, 4.53) .. controls (3.15, 6.21) and (3.73, 7.63) ..
			(3.73, 7.63) -- (3.73, 7.69) -- (0.83, 7.69) --cycle
			(0.83, 7.63);
			\path[fill=ce9edf5,nonzero rule] (1.93, 5.02) -- (1.93, 1.54) .. controls (1.93, 1.51) and (1.93, 1.48) ..
			(1.93, 1.46) .. controls (1.93, 1.43) and (1.94, 1.4) ..
			(1.94, 1.38) .. controls (1.95, 1.35) and (1.95, 1.32) ..
			(1.96, 1.29) .. controls (1.97, 1.27) and (1.98, 1.24) ..
			(1.99, 1.22) .. controls (2, 1.19) and (2.01, 1.17) ..
			(2.03, 1.14) .. controls (2.04, 1.12) and (2.05, 1.09) ..
			(2.07, 1.07) .. controls (2.08, 1.05) and (2.1, 1.03) ..
			(2.12, 1) .. controls (2.14, 0.98) and (2.15, 0.96) ..
			(2.17, 0.94) .. controls (2.19, 0.92) and (2.21, 0.9) ..
			(2.23, 0.89) .. controls (2.26, 0.87) and (2.28, 0.85) ..
			(2.3, 0.84) .. controls (2.32, 0.82) and (2.35, 0.81) ..
			(2.37, 0.8) .. controls (2.4, 0.78) and (2.42, 0.77) ..
			(2.45, 0.76) .. controls (2.47, 0.75) and (2.5, 0.74) ..
			(2.53, 0.73) .. controls (2.55, 0.72) and (2.58, 0.72) ..
			(2.61, 0.71) .. controls (2.63, 0.71) and (2.66, 0.7) ..
			(2.69, 0.7) .. controls (2.71, 0.7) and (2.74, 0.7) ..
			(2.77, 0.7) -- (3.15, 0.7) -- (3.15, 4.53) .. controls (3.15, 6.21) and (3.73, 7.63) ..
			(3.73, 7.63) -- (3.73, 7.69) -- (1.5, 7.69) .. controls (1.67, 7.15) and (1.93, 6.14) ..
			(1.93, 5.02) --cycle
			(1.93, 5.02);
			
			\path[fill=cafb4c8,nonzero rule] (13.02, -0) -- (0.21, -0) .. controls (0.2, -0) and (0.18, 0) ..
			(0.17, 0) .. controls (0.16, 0.01) and (0.15, 0.01) ..
			(0.13, 0.02) .. controls (0.12, 0.02) and (0.11, 0.03) ..
			(0.1, 0.03) .. controls (0.09, 0.04) and (0.08, 0.05) ..
			(0.07, 0.06) .. controls (0.06, 0.07) and (0.05, 0.08) ..
			(0.04, 0.09) .. controls (0.03, 0.1) and (0.03, 0.12) ..
			(0.02, 0.13) .. controls (0.02, 0.14) and (0.01, 0.15) ..
			(0.01, 0.17) .. controls (0.01, 0.18) and (0.01, 0.19) ..
			(0.01, 0.21) .. controls (0.01, 0.22) and (0.01, 0.23) ..
			(0.01, 0.25) .. controls (0.01, 0.26) and (0.02, 0.27) ..
			(0.02, 0.29) .. controls (0.03, 0.3) and (0.03, 0.31) ..
			(0.04, 0.32) .. controls (0.05, 0.33) and (0.06, 0.34) ..
			(0.07, 0.35) .. controls (0.08, 0.36) and (0.09, 0.37) ..
			(0.1, 0.38) .. controls (0.11, 0.39) and (0.12, 0.39) ..
			(0.13, 0.4) .. controls (0.15, 0.4) and (0.16, 0.41) ..
			(0.17, 0.41) .. controls (0.18, 0.41) and (0.2, 0.41) ..
			(0.21, 0.41) -- (13.02, 0.41) .. controls (13.03, 0.41) and (13.04, 0.41) ..
			(13.06, 0.41) .. controls (13.07, 0.41) and (13.08, 0.4) ..
			(13.1, 0.4) .. controls (13.11, 0.39) and (13.12, 0.39) ..
			(13.13, 0.38) .. controls (13.14, 0.37) and (13.15, 0.36) ..
			(13.16, 0.35) .. controls (13.17, 0.34) and (13.18, 0.33) ..
			(13.19, 0.32) .. controls (13.2, 0.31) and (13.2, 0.3) ..
			(13.21, 0.29) .. controls (13.21, 0.27) and (13.22, 0.26) ..
			(13.22, 0.25) .. controls (13.22, 0.23) and (13.22, 0.22) ..
			(13.22, 0.21) .. controls (13.22, 0.19) and (13.22, 0.18) ..
			(13.22, 0.17) .. controls (13.22, 0.15) and (13.21, 0.14) ..
			(13.21, 0.13) .. controls (13.2, 0.12) and (13.2, 0.1) ..
			(13.19, 0.09) .. controls (13.18, 0.08) and (13.17, 0.07) ..
			(13.16, 0.06) .. controls (13.15, 0.05) and (13.14, 0.04) ..
			(13.13, 0.03) .. controls (13.12, 0.03) and (13.11, 0.02) ..
			(13.1, 0.02) .. controls (13.08, 0.01) and (13.07, 0.01) ..
			(13.06, 0) .. controls (13.04, 0) and (13.03, -0) ..
			(13.02, -0) --cycle
			(13.02, -0);
			
			\path[fill=c8ab0e0,nonzero rule] (0.63, 10.95) -- (3.93, 10.95) -- (3.93, 10.09) -- (0.63, 10.09) --cycle
			(0.63, 10.95);
			
			\path[fill=caaccfa,nonzero rule] (1.26, 10.38) -- (3.93, 10.38) -- (3.93, 10.95) -- (1.01, 10.95) -- (1.01, 10.64) .. controls (1.01, 10.62) and (1.01, 10.6) ..
			(1.02, 10.59) .. controls (1.02, 10.57) and (1.02, 10.56) ..
			(1.03, 10.54) .. controls (1.04, 10.53) and (1.04, 10.51) ..
			(1.05, 10.5) .. controls (1.06, 10.48) and (1.07, 10.47) ..
			(1.08, 10.46) .. controls (1.1, 10.45) and (1.11, 10.44) ..
			(1.12, 10.43) .. controls (1.14, 10.42) and (1.15, 10.41) ..
			(1.17, 10.4) .. controls (1.18, 10.4) and (1.2, 10.39) ..
			(1.21, 10.39) .. controls (1.23, 10.39) and (1.25, 10.38) ..
			(1.26, 10.38) --cycle
			(1.26, 10.38);
			
			\path[fill=cdce1eb,nonzero rule] (2.18, 10.96) -- (2.38, 10.96) -- (2.38, 10.08) -- (2.18, 10.08) --cycle
			(2.18, 10.96);
			
			\path[fill=cdce1eb,nonzero rule] (1.35, 10.96) -- (1.56, 10.96) -- (1.56, 10.08) -- (1.35, 10.08) --cycle
			(1.35, 10.96);
			
			\path[fill=cdce1eb,nonzero rule] (3, 10.96) -- (3.21, 10.96) -- (3.21, 10.08) -- (3, 10.08) --cycle
			(3, 10.96);
			
			\path[fill=c8ab0e0,nonzero rule] (0.83, 9.34) -- (3.73, 9.34) -- (3.73, 7.63) -- (0.83, 7.63) --cycle
			(0.83, 9.34);
			
			\path[fill=caaccfa,nonzero rule] (1.61, 7.91) -- (3.73, 7.91) -- (3.73, 9.34) -- (1.25, 9.34) -- (1.25, 8.27) .. controls (1.25, 8.24) and (1.25, 8.22) ..
			(1.26, 8.2) .. controls (1.26, 8.17) and (1.27, 8.15) ..
			(1.28, 8.13) .. controls (1.29, 8.11) and (1.3, 8.09) ..
			(1.31, 8.07) .. controls (1.33, 8.05) and (1.34, 8.03) ..
			(1.36, 8.01) .. controls (1.37, 8) and (1.39, 7.98) ..
			(1.41, 7.97) .. controls (1.43, 7.95) and (1.45, 7.94) ..
			(1.48, 7.93) .. controls (1.5, 7.92) and (1.52, 7.92) ..
			(1.54, 7.91) .. controls (1.57, 7.91) and (1.59, 7.91) ..
			(1.61, 7.91) --cycle
			(1.61, 7.91);
			
			\path[fill=cdce1eb,nonzero rule] (3.73, 8.57) -- (2.38, 8.57) -- (2.38, 9.34) -- (2.18, 9.34) -- (2.18, 8.57) -- (0.83, 8.57) -- (0.83, 8.36) -- (2.18, 8.36) -- (2.18, 7.63) -- (2.38, 7.63) -- (2.38, 8.36) -- (3.73, 8.36) --cycle
			(3.73, 8.57);
			
			\path[fill=cdce1eb,nonzero rule] (0.63, 10.12) -- (3.93, 10.12) .. controls (3.96, 10.12) and (3.99, 10.12) ..
			(4.01, 10.11) .. controls (4.04, 10.11) and (4.07, 10.1) ..
			(4.09, 10.09) .. controls (4.12, 10.08) and (4.14, 10.06) ..
			(4.16, 10.05) .. controls (4.18, 10.03) and (4.21, 10.02) ..
			(4.22, 10) .. controls (4.24, 9.98) and (4.26, 9.96) ..
			(4.28, 9.94) .. controls (4.29, 9.91) and (4.3, 9.89) ..
			(4.31, 9.86) .. controls (4.32, 9.84) and (4.33, 9.81) ..
			(4.34, 9.79) .. controls (4.34, 9.76) and (4.35, 9.73) ..
			(4.35, 9.71) .. controls (4.35, 9.68) and (4.34, 9.65) ..
			(4.34, 9.63) .. controls (4.33, 9.6) and (4.32, 9.57) ..
			(4.31, 9.55) .. controls (4.3, 9.52) and (4.29, 9.5) ..
			(4.28, 9.48) .. controls (4.26, 9.45) and (4.24, 9.43) ..
			(4.22, 9.41) .. controls (4.21, 9.39) and (4.18, 9.38) ..
			(4.16, 9.36) .. controls (4.14, 9.35) and (4.12, 9.33) ..
			(4.09, 9.32) .. controls (4.07, 9.31) and (4.04, 9.31) ..
			(4.01, 9.3) .. controls (3.99, 9.3) and (3.96, 9.29) ..
			(3.93, 9.29) -- (0.63, 9.29) .. controls (0.6, 9.29) and (0.57, 9.3) ..
			(0.54, 9.3) .. controls (0.52, 9.31) and (0.49, 9.31) ..
			(0.47, 9.32) .. controls (0.44, 9.33) and (0.42, 9.35) ..
			(0.4, 9.36) .. controls (0.37, 9.38) and (0.35, 9.39) ..
			(0.33, 9.41) .. controls (0.31, 9.43) and (0.3, 9.45) ..
			(0.28, 9.48) .. controls (0.27, 9.5) and (0.25, 9.52) ..
			(0.24, 9.55) .. controls (0.23, 9.57) and (0.23, 9.6) ..
			(0.22, 9.63) .. controls (0.21, 9.65) and (0.21, 9.68) ..
			(0.21, 9.71) .. controls (0.21, 9.73) and (0.21, 9.76) ..
			(0.22, 9.79) .. controls (0.23, 9.81) and (0.23, 9.84) ..
			(0.24, 9.86) .. controls (0.25, 9.89) and (0.27, 9.91) ..
			(0.28, 9.94) .. controls (0.3, 9.96) and (0.31, 9.98) ..
			(0.33, 10) .. controls (0.35, 10.02) and (0.37, 10.03) ..
			(0.4, 10.05) .. controls (0.42, 10.06) and (0.44, 10.08) ..
			(0.47, 10.09) .. controls (0.49, 10.1) and (0.52, 10.11) ..
			(0.54, 10.11) .. controls (0.57, 10.12) and (0.6, 10.12) ..
			(0.63, 10.12) --cycle
			(0.63, 10.12);
			
			\path[fill=ce9edf5,nonzero rule] (3.93, 10.12) -- (0.63, 10.12) .. controls (0.56, 10.12) and (0.49, 10.1) ..
			(0.43, 10.07) .. controls (0.43, 10.06) and (0.43, 10.04) ..
			(0.42, 10.03) .. controls (0.42, 10.01) and (0.42, 9.99) ..
			(0.42, 9.98) .. controls (0.42, 9.96) and (0.42, 9.95) ..
			(0.42, 9.93) .. controls (0.42, 9.91) and (0.42, 9.9) ..
			(0.43, 9.88) .. controls (0.43, 9.87) and (0.43, 9.85) ..
			(0.44, 9.84) .. controls (0.44, 9.82) and (0.45, 9.81) ..
			(0.45, 9.79) .. controls (0.46, 9.78) and (0.47, 9.76) ..
			(0.48, 9.75) .. controls (0.48, 9.74) and (0.49, 9.72) ..
			(0.5, 9.71) .. controls (0.51, 9.7) and (0.52, 9.68) ..
			(0.53, 9.67) .. controls (0.54, 9.66) and (0.56, 9.65) ..
			(0.57, 9.64) .. controls (0.58, 9.63) and (0.59, 9.62) ..
			(0.61, 9.61) .. controls (0.62, 9.6) and (0.63, 9.6) ..
			(0.65, 9.59) .. controls (0.66, 9.58) and (0.68, 9.58) ..
			(0.69, 9.57) .. controls (0.71, 9.56) and (0.72, 9.56) ..
			(0.74, 9.56) .. controls (0.75, 9.55) and (0.77, 9.55) ..
			(0.78, 9.55) .. controls (0.8, 9.55) and (0.82, 9.55) ..
			(0.83, 9.55) -- (4.14, 9.55) .. controls (4.21, 9.55) and (4.27, 9.56) ..
			(4.33, 9.59) .. controls (4.34, 9.63) and (4.35, 9.67) ..
			(4.35, 9.71) .. controls (4.35, 9.73) and (4.34, 9.76) ..
			(4.34, 9.79) .. controls (4.33, 9.81) and (4.32, 9.84) ..
			(4.31, 9.86) .. controls (4.3, 9.89) and (4.29, 9.91) ..
			(4.28, 9.94) .. controls (4.26, 9.96) and (4.24, 9.98) ..
			(4.22, 10) .. controls (4.21, 10.02) and (4.18, 10.03) ..
			(4.16, 10.05) .. controls (4.14, 10.06) and (4.12, 10.08) ..
			(4.09, 10.09) .. controls (4.07, 10.1) and (4.04, 10.11) ..
			(4.01, 10.11) .. controls (3.99, 10.12) and (3.96, 10.12) ..
			(3.93, 10.12) --cycle
			(3.93, 10.12);
		\end{scope}
		}}
		\begin{tikzpicture}[scale=.7,transform shape]
					\path (0,0) pic[scale=.3]{khongluu};
					\path (.685,0) coordinate (O) node[below right] {$O$}
					+(0:4) coordinate (y) node[below] {$y$ (Hướng Nam)}
					+(90:5) coordinate (z) node[above] {$z$}
					+(-145:3) coordinate (x) node[below] {$x$ (Hướng Tây)}
					;
					\draw[-stealth] (O)--(z);
					\draw[-stealth] (O)--(y);
					\draw[-stealth] (O)--(x);
		\end{tikzpicture}
	\end{center}
	\begin{itemchoice}
			\itemch {\bf Sai}.\\
			Vị trí $A$ cách mặt đất $8$ km, cách $268$ km về phía Đông, $185$ km về phía Nam nên ta có $A\left(-268;185;8\right)$.
			\itemch {\bf Đúng}.\\
			Tọa độ của đài kiểm soát là $M(0;0;0{,}105)$.\\
			Khoảng cách từ đài kiểm soát đến máy bay là
			$$
			MA=\sqrt{(-268)^2+185^2+(8-0{,}105)^2}\approx 325{,}7.
			$$
			Vậy $MA<450$ nên đài kiểm soát có phát hiện được máy bay tại vị trí $A$.
			\itemch {\bf Đúng}.\\
			Đường thẳng $d$ đi qua điểm $A(-268;185;8)$ và có véctơ chỉ phương là $\overrightarrow{u}=(86;76;0)$ nên phương trình tham số của đường thẳng $d$ là $\heva{&x=-268+82t\\&y=185+76t\\&z=8}$ ($t$ là tham số)
			\itemch {\bf Sai}.\\
			Khoảng cách gần nhất giữa máy bay và đài kiểm soát không lưu chính là khoảng cách từ đài kiểm soát không lưu đến quỹ đạo chuyển động $d$ của máy bay.\\
			Ta có $\overrightarrow{MA}=(-268;185;7{,}895)$.\\
			Vậy $h=\dfrac{\left|\left[\overrightarrow{MA},\overrightarrow{u}\right]\right|}{|\overrightarrow{u}|}=317{,}96$ (km).
	\end{itemchoice}
	}
\end{ex}

\begin{ex}%[Nguồn: Bộ đề minh họa Moon 2024-2025]%[2H5V2-8]
	Hình vẽ minh họa đường bay của một chiếc trực thăng $H$ cất cánh từ một sân bay. Xét hệ trục tọa độ $Oxyz$ có gốc tọa độ $O$ là chân tháp điều khiển của sân bay; trục $Ox$ là hướng đông, trục $Oy$ là hướng bắc và trục $Oz$ là trục thẳng đứng, đơn vị trên mỗi trục là kilômét. Trực thăng cất cánh từ điểm $G$. Vectơ $ \overrightarrow{r}$ chỉ vị trí của trực thăng tại thời điểm $t$ phút sau khi cất cánh $\left(t \geq 0 \right)$ có tọa độ là $ \overrightarrow{r} = \left(1+t; 0{,}5 + 2t; 2t \right)$.
	\begin{center}
		
		\definecolor{c565859}{RGB}{86,88,89}
		\definecolor{c211c1d}{RGB}{33,28,29}
		\definecolor{c7a7973}{RGB}{122,121,115}
		\tikzset{radar/.pic={
		\begin{scope}[line cap=round,line join=round]
			
			\path[fill=black,nonzero rule] (11.05, 12.66) .. controls (11.05, 12.65) and (11.05, 12.64) ..
			(11.05, 12.63) .. controls (11.05, 12.61) and (11.04, 12.6) ..
			(11.04, 12.59) .. controls (11.03, 12.58) and (11.03, 12.57) ..
			(11.02, 12.56) .. controls (11.01, 12.55) and (11.01, 12.54) ..
			(11, 12.53) .. controls (10.99, 12.52) and (10.98, 12.51) ..
			(10.97, 12.51) .. controls (10.96, 12.5) and (10.95, 12.49) ..
			(10.94, 12.49) .. controls (10.92, 12.48) and (10.91, 12.48) ..
			(10.9, 12.48) .. controls (10.89, 12.47) and (10.88, 12.47) ..
			(10.86, 12.47) .. controls (10.85, 12.47) and (10.84, 12.47) ..
			(10.83, 12.48) .. controls (10.82, 12.48) and (10.8, 12.48) ..
			(10.79, 12.49) .. controls (10.78, 12.49) and (10.77, 12.5) ..
			(10.76, 12.51) .. controls (10.75, 12.51) and (10.74, 12.52) ..
			(10.73, 12.53) .. controls (10.72, 12.54) and (10.71, 12.55) ..
			(10.71, 12.56) .. controls (10.7, 12.57) and (10.7, 12.58) ..
			(10.69, 12.59) .. controls (10.69, 12.6) and (10.68, 12.61) ..
			(10.68, 12.63) .. controls (10.68, 12.64) and (10.68, 12.65) ..
			(10.68, 12.66) .. controls (10.68, 12.67) and (10.68, 12.69) ..
			(10.68, 12.7) .. controls (10.68, 12.71) and (10.69, 12.72) ..
			(10.69, 12.73) .. controls (10.7, 12.75) and (10.7, 12.76) ..
			(10.71, 12.77) .. controls (10.71, 12.78) and (10.72, 12.79) ..
			(10.73, 12.79) .. controls (10.74, 12.8) and (10.75, 12.81) ..
			(10.76, 12.82) .. controls (10.77, 12.83) and (10.78, 12.83) ..
			(10.79, 12.84) .. controls (10.8, 12.84) and (10.82, 12.84) ..
			(10.83, 12.85) .. controls (10.84, 12.85) and (10.85, 12.85) ..
			(10.86, 12.85) .. controls (10.88, 12.85) and (10.89, 12.85) ..
			(10.9, 12.85) .. controls (10.91, 12.84) and (10.92, 12.84) ..
			(10.94, 12.84) .. controls (10.95, 12.83) and (10.96, 12.83) ..
			(10.97, 12.82) .. controls (10.98, 12.81) and (10.99, 12.8) ..
			(11, 12.79) .. controls (11.01, 12.79) and (11.01, 12.78) ..
			(11.02, 12.77) .. controls (11.03, 12.76) and (11.03, 12.75) ..
			(11.04, 12.73) .. controls (11.04, 12.72) and (11.05, 12.71) ..
			(11.05, 12.7) .. controls (11.05, 12.69) and (11.05, 12.67) ..
			(11.05, 12.66) --cycle
			(11.05, 12.66);
			
			\path[fill=black,nonzero rule] (12.89, 12.19) .. controls (12.86, 12.19) and (12.84, 12.21) ..
			(12.84, 12.24) .. controls (12.84, 12.31) and (12.57, 12.66) ..
			(12.18, 12.66) .. controls (12.15, 12.66) and (12.13, 12.68) ..
			(12.13, 12.71) .. controls (12.13, 12.74) and (12.15, 12.76) ..
			(12.18, 12.76) .. controls (12.6, 12.76) and (12.93, 12.38) ..
			(12.93, 12.24) .. controls (12.93, 12.21) and (12.91, 12.19) ..
			(12.89, 12.19) --cycle
			(12.89, 12.19);
			
			\path[fill=black,nonzero rule] (12.89, 12.29) -- (12.46, 12.29) .. controls (12.4, 12.29) and (12.34, 12.33) ..
			(12.33, 12.39) -- (12.22, 12.73) -- (12.13, 12.92) .. controls (12.12, 12.93) and (12.1, 12.94) ..
			(12.09, 12.94) -- (11.95, 12.94) .. controls (11.93, 12.94) and (11.92, 12.94) ..
			(11.91, 12.93) -- (11.74, 12.76) -- (11.01, 12.76) .. controls (10.98, 12.76) and (10.96, 12.74) ..
			(10.96, 12.71) -- (10.96, 12.61) .. controls (10.96, 12.59) and (10.97, 12.57) ..
			(11, 12.57) .. controls (11.17, 12.54) and (11.46, 12.47) ..
			(11.53, 12.4) -- (11.53, 12.39) .. controls (11.62, 12.22) and (11.73, 12) ..
			(11.95, 12) -- (12.6, 12) .. controls (12.79, 12) and (12.93, 12.1) ..
			(12.93, 12.24) .. controls (12.93, 12.26) and (12.91, 12.29) ..
			(12.89, 12.29) --cycle
			(12.89, 12.29);
			
			\path[fill=black,nonzero rule] (10.82, 12.8) -- (10.86, 12.8) -- (10.87, 12.78) .. controls (10.88, 12.76) and (10.9, 12.75) ..
			(10.91, 12.75) .. controls (10.92, 12.75) and (10.93, 12.75) ..
			(10.93, 12.76) .. controls (10.96, 12.77) and (10.97, 12.8) ..
			(10.95, 12.82) -- (10.86, 13.01) .. controls (10.85, 13.03) and (10.82, 13.04) ..
			(10.8, 13.03) -- (10.7, 12.99) .. controls (10.68, 12.98) and (10.67, 12.95) ..
			(10.68, 12.93) -- (10.73, 12.79) .. controls (10.73, 12.76) and (10.76, 12.75) ..
			(10.78, 12.76) .. controls (10.8, 12.76) and (10.82, 12.78) ..
			(10.82, 12.8) --cycle
			(10.82, 12.8);
			
			\path[fill=black,nonzero rule] (10.93, 12.56) .. controls (10.91, 12.57) and (10.88, 12.57) ..
			(10.87, 12.54) -- (10.86, 12.53) .. controls (10.86, 12.55) and (10.84, 12.57) ..
			(10.82, 12.57) .. controls (10.79, 12.57) and (10.77, 12.55) ..
			(10.77, 12.52) -- (10.77, 12.43) .. controls (10.77, 12.4) and (10.79, 12.38) ..
			(10.82, 12.38) -- (10.86, 12.38) .. controls (10.88, 12.38) and (10.9, 12.39) ..
			(10.91, 12.41) -- (10.95, 12.5) .. controls (10.96, 12.52) and (10.96, 12.55) ..
			(10.93, 12.56) --cycle
			(10.93, 12.56);
			
			\path[fill=black,nonzero rule] (12.04, 13.09) .. controls (12.01, 13.09) and (11.99, 13.11) ..
			(11.99, 13.13) -- (11.99, 13.18) .. controls (11.99, 13.21) and (12.01, 13.23) ..
			(12.04, 13.23) .. controls (12.07, 13.23) and (12.09, 13.21) ..
			(12.09, 13.18) -- (12.09, 13.13) .. controls (12.09, 13.11) and (12.07, 13.09) ..
			(12.04, 13.09) --cycle
			(12.04, 13.09);
			
			\path[fill=black,nonzero rule] (12.89, 13.04) -- (12.04, 13.04) .. controls (12.01, 13.04) and (11.99, 13.06) ..
			(11.99, 13.09) .. controls (11.99, 13.11) and (12.01, 13.13) ..
			(12.04, 13.13) -- (12.89, 13.13) .. controls (12.91, 13.13) and (12.93, 13.11) ..
			(12.93, 13.09) .. controls (12.93, 13.06) and (12.91, 13.04) ..
			(12.89, 13.04) --cycle
			(12.89, 13.04);
			
			\path[fill=black,nonzero rule] (12.04, 13.04) -- (11.19, 13.04) .. controls (11.17, 13.04) and (11.15, 13.06) ..
			(11.15, 13.09) .. controls (11.15, 13.11) and (11.17, 13.13) ..
			(11.19, 13.13) -- (12.04, 13.13) .. controls (12.07, 13.13) and (12.09, 13.11) ..
			(12.09, 13.09) .. controls (12.09, 13.06) and (12.07, 13.04) ..
			(12.04, 13.04) --cycle
			(12.04, 13.04);
			
			\path[fill=black,nonzero rule] (12.23, 11.82) .. controls (12.2, 11.82) and (12.18, 11.84) ..
			(12.18, 11.86) -- (12.18, 12.05) .. controls (12.18, 12.08) and (12.2, 12.1) ..
			(12.23, 12.1) .. controls (12.25, 12.1) and (12.28, 12.08) ..
			(12.28, 12.05) -- (12.28, 11.86) .. controls (12.28, 11.84) and (12.25, 11.82) ..
			(12.23, 11.82) --cycle
			(12.23, 11.82);
			
			\path[fill=black,nonzero rule] (12.6, 11.82) .. controls (12.58, 11.82) and (12.56, 11.84) ..
			(12.56, 11.86) -- (12.56, 12.05) .. controls (12.56, 12.08) and (12.58, 12.1) ..
			(12.6, 12.1) .. controls (12.63, 12.1) and (12.65, 12.08) ..
			(12.65, 12.05) -- (12.65, 11.86) .. controls (12.65, 11.84) and (12.63, 11.82) ..
			(12.6, 11.82) --cycle
			(12.6, 11.82);
			
			\path[fill=black,nonzero rule] (11.76, 11.82) .. controls (11.73, 11.82) and (11.71, 11.84) ..
			(11.71, 11.86) .. controls (11.71, 11.89) and (11.73, 11.91) ..
			(11.76, 11.91) -- (12.7, 11.91) .. controls (12.73, 11.91) and (12.82, 11.94) ..
			(12.85, 11.98) .. controls (12.86, 12) and (12.89, 12.01) ..
			(12.91, 12) .. controls (12.93, 11.98) and (12.94, 11.96) ..
			(12.93, 11.93) .. controls (12.88, 11.84) and (12.72, 11.82) ..
			(12.7, 11.82) --cycle
			(11.76, 11.82);
			
			\path[fill=c565859,nonzero rule] (0.58, 0.12) -- (2.17, 0.12) -- (2.17, 0.03) -- (0.58, 0.03) --cycle
			(0.58, 0.12);
			
			\path[fill=c565859,nonzero rule] (1.19, 4.02) -- (0.57, 4.02) .. controls (0.55, 4.02) and (0.54, 4.03) ..
			(0.53, 4.04) .. controls (0.52, 4.05) and (0.52, 4.06) ..
			(0.52, 4.07) -- (0.57, 4.5) .. controls (0.57, 4.52) and (0.59, 4.54) ..
			(0.61, 4.54) -- (1.19, 4.54) .. controls (1.21, 4.54) and (1.23, 4.52) ..
			(1.23, 4.49) -- (1.23, 4.07) .. controls (1.23, 4.05) and (1.21, 4.02) ..
			(1.19, 4.02) --cycle
			(0.62, 4.12) -- (1.14, 4.12) -- (1.14, 4.45) -- (0.65, 4.45) --cycle
			(0.62, 4.12);
			
			\path[fill=c565859,nonzero rule] (1.19, 3.52) -- (0.51, 3.52) .. controls (0.5, 3.52) and (0.48, 3.53) ..
			(0.48, 3.54) .. controls (0.47, 3.55) and (0.46, 3.56) ..
			(0.46, 3.57) -- (0.52, 4.08) .. controls (0.52, 4.1) and (0.54, 4.12) ..
			(0.56, 4.12) -- (1.19, 4.12) .. controls (1.21, 4.12) and (1.23, 4.1) ..
			(1.23, 4.07) -- (1.23, 3.57) .. controls (1.23, 3.54) and (1.21, 3.52) ..
			(1.19, 3.52) --cycle
			(0.56, 3.61) -- (1.14, 3.61) -- (1.14, 4.03) -- (0.61, 4.03) --cycle
			(0.56, 3.61);
			
			\path[fill=c565859,nonzero rule] (1.19, 2.91) -- (0.44, 2.91) .. controls (0.43, 2.91) and (0.41, 2.92) ..
			(0.41, 2.93) .. controls (0.4, 2.94) and (0.39, 2.95) ..
			(0.39, 2.97) -- (0.46, 3.57) .. controls (0.47, 3.59) and (0.48, 3.61) ..
			(0.51, 3.61) -- (1.19, 3.61) .. controls (1.21, 3.61) and (1.23, 3.59) ..
			(1.23, 3.57) -- (1.23, 2.96) .. controls (1.23, 2.94) and (1.21, 2.91) ..
			(1.19, 2.91) --cycle
			(0.49, 3.01) -- (1.14, 3.01) -- (1.14, 3.52) -- (0.55, 3.52) --cycle
			(0.49, 3.01);
			
			\path[fill=c565859,nonzero rule] (2.26, 0.03) .. controls (2.25, 0.03) and (2.23, 0.03) ..
			(2.22, 0.04) -- (1.15, 1.25) .. controls (1.14, 1.26) and (1.13, 1.28) ..
			(1.14, 1.3) .. controls (1.15, 1.32) and (1.17, 1.33) ..
			(1.18, 1.33) -- (2.12, 1.33) .. controls (2.14, 1.33) and (2.16, 1.31) ..
			(2.16, 1.29) -- (2.3, 0.08) .. controls (2.3, 0.06) and (2.29, 0.04) ..
			(2.27, 0.03) .. controls (2.27, 0.03) and (2.27, 0.03) ..
			(2.26, 0.03) --cycle
			(1.29, 1.24) -- (2.2, 0.21) -- (2.08, 1.24) --cycle
			(1.29, 1.24);
			
			\path[fill=c565859,nonzero rule] (1.19, 0.03) -- (0.11, 0.03) .. controls (0.1, 0.03) and (0.09, 0.03) ..
			(0.08, 0.04) .. controls (0.07, 0.05) and (0.06, 0.06) ..
			(0.06, 0.08) -- (0.2, 1.29) .. controls (0.2, 1.31) and (0.22, 1.33) ..
			(0.25, 1.33) -- (1.19, 1.33) .. controls (1.21, 1.33) and (1.23, 1.31) ..
			(1.23, 1.28) -- (1.23, 0.07) .. controls (1.23, 0.05) and (1.21, 0.03) ..
			(1.19, 0.03) --cycle
			(0.16, 0.12) -- (1.14, 0.12) -- (1.14, 1.24) -- (0.29, 1.24) --cycle
			(0.16, 0.12);
			
			\path[fill=c565859,nonzero rule] (1.19, 2.17) -- (0.35, 2.17) .. controls (0.34, 2.17) and (0.33, 2.18) ..
			(0.32, 2.19) .. controls (0.31, 2.2) and (0.31, 2.21) ..
			(0.31, 2.22) -- (0.39, 2.97) .. controls (0.4, 2.99) and (0.41, 3.01) ..
			(0.44, 3.01) -- (1.19, 3.01) .. controls (1.21, 3.01) and (1.23, 2.98) ..
			(1.23, 2.96) -- (1.23, 2.22) .. controls (1.23, 2.19) and (1.21, 2.17) ..
			(1.19, 2.17) --cycle
			(0.41, 2.26) -- (1.14, 2.26) -- (1.14, 2.91) -- (0.48, 2.91) --cycle
			(0.41, 2.26);
			
			\path[fill=c565859,nonzero rule] (2.12, 1.24) -- (1.19, 1.24) .. controls (1.16, 1.24) and (1.14, 1.26) ..
			(1.14, 1.28) -- (1.14, 2.22) .. controls (1.14, 2.24) and (1.16, 2.26) ..
			(1.19, 2.26) -- (2.02, 2.26) .. controls (2.04, 2.26) and (2.06, 2.25) ..
			(2.06, 2.22) -- (2.17, 1.29) .. controls (2.17, 1.27) and (2.17, 1.26) ..
			(2.16, 1.25) .. controls (2.15, 1.24) and (2.13, 1.24) ..
			(2.12, 1.24) --cycle
			(1.23, 1.33) -- (2.07, 1.33) -- (1.98, 2.17) -- (1.23, 2.17) -- (1.23, 1.33) --cycle
			(1.23, 1.33);
			
			\path[fill=c565859,nonzero rule] (1.86, 3.52) -- (1.19, 3.52) .. controls (1.16, 3.52) and (1.14, 3.54) ..
			(1.14, 3.57) -- (1.14, 4.07) .. controls (1.14, 4.1) and (1.16, 4.12) ..
			(1.19, 4.12) -- (1.8, 4.12) .. controls (1.83, 4.12) and (1.85, 4.1) ..
			(1.85, 4.07) -- (1.91, 3.57) .. controls (1.91, 3.56) and (1.9, 3.55) ..
			(1.89, 3.54) .. controls (1.89, 3.53) and (1.87, 3.52) ..
			(1.86, 3.52) --cycle
			(1.23, 3.61) -- (1.81, 3.61) -- (1.76, 4.03) -- (1.23, 4.03) --cycle
			(1.23, 3.61);
			
			\path[fill=c565859,nonzero rule] (1.93, 2.91) -- (1.19, 2.91) .. controls (1.16, 2.91) and (1.14, 2.94) ..
			(1.14, 2.96) -- (1.14, 3.56) .. controls (1.14, 3.59) and (1.16, 3.61) ..
			(1.19, 3.61) -- (1.86, 3.61) .. controls (1.88, 3.61) and (1.9, 3.59) ..
			(1.91, 3.57) -- (1.98, 2.96) .. controls (1.98, 2.95) and (1.97, 2.94) ..
			(1.96, 2.93) .. controls (1.96, 2.92) and (1.94, 2.91) ..
			(1.93, 2.91) --cycle
			(1.23, 3.01) -- (1.88, 3.01) -- (1.82, 3.52) -- (1.23, 3.52) --cycle
			(1.23, 3.01);
			
			\path[fill=c565859,nonzero rule] (2.02, 2.17) -- (1.19, 2.17) .. controls (1.16, 2.17) and (1.14, 2.19) ..
			(1.14, 2.22) -- (1.14, 2.96) .. controls (1.14, 2.98) and (1.16, 3) ..
			(1.19, 3) -- (1.93, 3) .. controls (1.95, 3) and (1.97, 2.99) ..
			(1.98, 2.96) -- (2.06, 2.22) .. controls (2.06, 2.21) and (2.06, 2.19) ..
			(2.05, 2.19) .. controls (2.04, 2.18) and (2.03, 2.17) ..
			(2.02, 2.17) --cycle
			(1.23, 2.26) -- (1.96, 2.26) -- (1.89, 2.91) -- (1.23, 2.91) --cycle
			(1.23, 2.26);
			
			\path[fill=c565859,nonzero rule] (1.19, 1.24) -- (0.25, 1.24) .. controls (0.24, 1.24) and (0.22, 1.24) ..
			(0.21, 1.25) .. controls (0.21, 1.26) and (0.2, 1.28) ..
			(0.2, 1.29) -- (0.31, 2.22) .. controls (0.31, 2.25) and (0.33, 2.26) ..
			(0.35, 2.26) -- (1.19, 2.26) .. controls (1.21, 2.26) and (1.23, 2.24) ..
			(1.23, 2.22) -- (1.23, 1.28) .. controls (1.23, 1.26) and (1.21, 1.24) ..
			(1.19, 1.24) --cycle
			(0.3, 1.33) -- (1.14, 1.33) -- (1.14, 2.17) -- (0.4, 2.17) --cycle
			(0.3, 1.33);
			
			\path[fill=c565859,nonzero rule] (1.8, 4.02) -- (1.19, 4.02) .. controls (1.16, 4.02) and (1.14, 4.05) ..
			(1.14, 4.07) -- (1.14, 4.49) .. controls (1.14, 4.52) and (1.16, 4.54) ..
			(1.19, 4.54) -- (1.76, 4.54) .. controls (1.78, 4.54) and (1.8, 4.52) ..
			(1.8, 4.5) -- (1.85, 4.07) .. controls (1.85, 4.06) and (1.85, 4.05) ..
			(1.84, 4.04) .. controls (1.83, 4.03) and (1.82, 4.02) ..
			(1.8, 4.02) --cycle
			(1.23, 4.12) -- (1.75, 4.12) -- (1.71, 4.45) -- (1.23, 4.45) -- (1.23, 4.12) --cycle
			(1.23, 4.12);
			
			\path[fill=c565859,nonzero rule] (1.19, 0.03) .. controls (1.17, 0.03) and (1.16, 0.03) ..
			(1.15, 0.04) -- (0.21, 1.25) .. controls (0.2, 1.27) and (0.2, 1.3) ..
			(0.22, 1.32) .. controls (0.24, 1.33) and (0.27, 1.33) ..
			(0.28, 1.31) -- (1.22, 0.1) .. controls (1.24, 0.08) and (1.23, 0.05) ..
			(1.21, 0.04) .. controls (1.2, 0.03) and (1.2, 0.03) ..
			(1.19, 0.03) --cycle
			(2.26, 0.03) .. controls (2.25, 0.03) and (2.23, 0.03) ..
			(2.22, 0.04) -- (0.32, 2.19) .. controls (0.3, 2.21) and (0.3, 2.24) ..
			(0.32, 2.25) .. controls (0.34, 2.27) and (0.37, 2.27) ..
			(0.39, 2.25) -- (1.22, 1.31) -- (2.29, 0.1) .. controls (2.31, 0.08) and (2.31, 0.05) ..
			(2.29, 0.04) .. controls (2.28, 0.03) and (2.27, 0.03) ..
			(2.26, 0.03) --cycle
			(2.12, 1.24) .. controls (2.11, 1.24) and (2.1, 1.24) ..
			(2.09, 1.25) -- (1.15, 2.19) -- (0.41, 2.93) .. controls (0.39, 2.95) and (0.39, 2.97) ..
			(0.41, 2.99) .. controls (0.42, 3.01) and (0.45, 3.01) ..
			(0.47, 2.99) -- (1.22, 2.25) -- (2.15, 1.31) .. controls (2.17, 1.3) and (2.17, 1.27) ..
			(2.15, 1.25) .. controls (2.14, 1.24) and (2.13, 1.24) ..
			(2.12, 1.24) --cycle
			(2.02, 2.17) .. controls (2, 2.17) and (1.99, 2.18) ..
			(1.99, 2.18) -- (1.15, 2.93) -- (0.48, 3.53) .. controls (0.46, 3.55) and (0.46, 3.58) ..
			(0.48, 3.6) .. controls (0.49, 3.62) and (0.52, 3.62) ..
			(0.54, 3.6) -- (1.21, 2.99) -- (2.05, 2.25) .. controls (2.06, 2.23) and (2.06, 2.21) ..
			(2.05, 2.19) .. controls (2.04, 2.18) and (2.03, 2.17) ..
			(2.02, 2.17) --cycle
			(1.93, 2.91) .. controls (1.92, 2.91) and (1.91, 2.92) ..
			(1.9, 2.92) -- (1.16, 3.53) -- (0.54, 4.03) .. controls (0.52, 4.05) and (0.51, 4.08) ..
			(0.53, 4.1) .. controls (0.55, 4.12) and (0.58, 4.12) ..
			(0.6, 4.1) -- (1.21, 3.6) -- (1.96, 3) .. controls (1.98, 2.98) and (1.98, 2.95) ..
			(1.97, 2.93) .. controls (1.96, 2.92) and (1.94, 2.91) ..
			(1.93, 2.91) --cycle
			(1.86, 3.52) .. controls (1.85, 3.52) and (1.84, 3.53) ..
			(1.83, 3.53) -- (0.59, 4.46) .. controls (0.57, 4.48) and (0.56, 4.5) ..
			(0.58, 4.52) .. controls (0.59, 4.54) and (0.62, 4.55) ..
			(0.64, 4.53) -- (1.89, 3.61) .. controls (1.91, 3.59) and (1.91, 3.56) ..
			(1.9, 3.54) .. controls (1.89, 3.53) and (1.88, 3.52) ..
			(1.86, 3.52) --cycle
			(1.8, 4.02) .. controls (1.79, 4.02) and (1.79, 4.03) ..
			(1.78, 4.03) -- (1.16, 4.46) .. controls (1.14, 4.47) and (1.13, 4.5) ..
			(1.15, 4.52) .. controls (1.16, 4.54) and (1.19, 4.55) ..
			(1.21, 4.53) -- (1.83, 4.11) .. controls (1.85, 4.09) and (1.86, 4.06) ..
			(1.84, 4.04) .. controls (1.83, 4.03) and (1.82, 4.02) ..
			(1.8, 4.02) --cycle
			(1.8, 4.02);
			
			\path[fill=c565859,nonzero rule] (1.19, 0.03) .. controls (1.18, 0.03) and (1.17, 0.03) ..
			(1.16, 0.04) .. controls (1.14, 0.05) and (1.13, 0.08) ..
			(1.15, 0.1) -- (2.09, 1.31) .. controls (2.1, 1.33) and (2.13, 1.33) ..
			(2.15, 1.32) .. controls (2.17, 1.3) and (2.17, 1.27) ..
			(2.16, 1.25) -- (1.22, 0.04) .. controls (1.21, 0.03) and (1.2, 0.03) ..
			(1.19, 0.03) --cycle
			(0.11, 0.03) .. controls (0.1, 0.03) and (0.09, 0.03) ..
			(0.08, 0.04) .. controls (0.06, 0.05) and (0.06, 0.08) ..
			(0.08, 0.1) -- (1.15, 1.31) -- (1.98, 2.25) .. controls (2, 2.27) and (2.03, 2.27) ..
			(2.05, 2.25) .. controls (2.06, 2.23) and (2.07, 2.21) ..
			(2.05, 2.19) -- (1.22, 1.25) -- (0.14, 0.04) .. controls (0.13, 0.03) and (0.12, 0.03) ..
			(0.11, 0.03) --cycle
			(0.25, 1.24) .. controls (0.24, 1.24) and (0.23, 1.24) ..
			(0.22, 1.25) .. controls (0.2, 1.27) and (0.2, 1.3) ..
			(0.22, 1.32) -- (1.15, 2.25) -- (1.9, 2.99) .. controls (1.91, 3.01) and (1.94, 3.01) ..
			(1.96, 2.99) .. controls (1.98, 2.98) and (1.98, 2.95) ..
			(1.96, 2.93) -- (0.28, 1.25) .. controls (0.27, 1.24) and (0.26, 1.24) ..
			(0.25, 1.24) --cycle
			(0.35, 2.17) .. controls (0.34, 2.17) and (0.33, 2.18) ..
			(0.32, 2.19) .. controls (0.3, 2.21) and (0.31, 2.23) ..
			(0.33, 2.25) -- (1.16, 2.99) -- (1.83, 3.6) .. controls (1.85, 3.62) and (1.88, 3.62) ..
			(1.9, 3.6) .. controls (1.91, 3.58) and (1.91, 3.55) ..
			(1.89, 3.53) -- (1.22, 2.93) -- (0.39, 2.18) .. controls (0.38, 2.18) and (0.37, 2.17) ..
			(0.35, 2.17) --cycle
			(0.44, 2.91) .. controls (0.43, 2.91) and (0.41, 2.92) ..
			(0.41, 2.93) .. controls (0.39, 2.95) and (0.39, 2.98) ..
			(0.41, 3) -- (1.16, 3.6) -- (1.77, 4.11) .. controls (1.79, 4.12) and (1.82, 4.12) ..
			(1.84, 4.1) .. controls (1.85, 4.08) and (1.85, 4.05) ..
			(1.83, 4.04) -- (1.21, 3.53) -- (0.47, 2.93) .. controls (0.46, 2.92) and (0.45, 2.91) ..
			(0.44, 2.91) --cycle
			(0.51, 3.52) .. controls (0.5, 3.52) and (0.48, 3.53) ..
			(0.47, 3.54) .. controls (0.46, 3.56) and (0.46, 3.59) ..
			(0.48, 3.6) -- (1.16, 4.11) -- (1.73, 4.53) .. controls (1.75, 4.55) and (1.78, 4.54) ..
			(1.79, 4.52) .. controls (1.81, 4.5) and (1.8, 4.47) ..
			(1.78, 4.46) -- (1.21, 4.03) -- (0.54, 3.53) .. controls (0.53, 3.52) and (0.52, 3.52) ..
			(0.51, 3.52) --cycle
			(0.57, 4.02) .. controls (0.55, 4.02) and (0.54, 4.03) ..
			(0.53, 4.04) .. controls (0.52, 4.06) and (0.52, 4.09) ..
			(0.54, 4.11) -- (1.16, 4.53) .. controls (1.18, 4.55) and (1.21, 4.54) ..
			(1.22, 4.52) .. controls (1.24, 4.5) and (1.23, 4.47) ..
			(1.21, 4.46) -- (0.59, 4.03) .. controls (0.58, 4.03) and (0.58, 4.02) ..
			(0.57, 4.02) --cycle
			(0.57, 4.02);
			
			\path[fill=c565859,nonzero rule] (1.19, 4.45) -- (0.62, 4.45) .. controls (0.59, 4.45) and (0.57, 4.47) ..
			(0.57, 4.49) -- (0.57, 5.23) .. controls (0.57, 5.25) and (0.59, 5.27) ..
			(0.62, 5.27) -- (1.19, 5.27) .. controls (1.21, 5.27) and (1.23, 5.25) ..
			(1.23, 5.23) -- (1.23, 4.49) .. controls (1.23, 4.47) and (1.21, 4.45) ..
			(1.19, 4.45) --cycle
			(0.66, 4.54) -- (1.14, 4.54) -- (1.14, 5.18) -- (0.66, 5.18) --cycle
			(0.66, 4.54);
			
			\path[fill=c565859,nonzero rule] (1.19, 5.19) -- (0.62, 5.19) .. controls (0.59, 5.19) and (0.57, 5.21) ..
			(0.57, 5.23) -- (0.57, 5.96) .. controls (0.57, 5.99) and (0.59, 6.01) ..
			(0.62, 6.01) -- (1.19, 6.01) .. controls (1.21, 6.01) and (1.23, 5.99) ..
			(1.23, 5.96) -- (1.23, 5.23) .. controls (1.23, 5.2) and (1.21, 5.19) ..
			(1.19, 5.19) --cycle
			(0.66, 5.28) -- (1.14, 5.28) -- (1.14, 5.92) -- (0.66, 5.92) --cycle
			(0.66, 5.28);
			
			\path[fill=c565859,nonzero rule] (1.76, 4.45) -- (1.19, 4.45) .. controls (1.16, 4.45) and (1.14, 4.47) ..
			(1.14, 4.49) -- (1.14, 5.23) .. controls (1.14, 5.25) and (1.16, 5.27) ..
			(1.19, 5.27) -- (1.76, 5.27) .. controls (1.78, 5.27) and (1.8, 5.25) ..
			(1.8, 5.23) -- (1.8, 4.49) .. controls (1.8, 4.47) and (1.78, 4.45) ..
			(1.76, 4.45) --cycle
			(1.23, 4.54) -- (1.71, 4.54) -- (1.71, 5.18) -- (1.23, 5.18) -- (1.23, 4.54) --cycle
			(1.23, 4.54);
			
			\path[fill=c565859,nonzero rule] (1.76, 5.92) -- (1.19, 5.92) .. controls (1.16, 5.92) and (1.14, 5.94) ..
			(1.14, 5.96) -- (1.14, 6.7) .. controls (1.14, 6.72) and (1.16, 6.74) ..
			(1.19, 6.74) -- (1.76, 6.74) .. controls (1.78, 6.74) and (1.8, 6.72) ..
			(1.8, 6.7) -- (1.8, 5.96) .. controls (1.8, 5.94) and (1.78, 5.92) ..
			(1.76, 5.92) --cycle
			(1.23, 6.01) -- (1.71, 6.01) -- (1.71, 6.65) -- (1.23, 6.65) -- (1.23, 6.01) --cycle
			(1.23, 6.01);
			
			\path[fill=c565859,nonzero rule] (1.76, 5.19) -- (1.19, 5.19) .. controls (1.16, 5.19) and (1.14, 5.21) ..
			(1.14, 5.23) -- (1.14, 5.96) .. controls (1.14, 5.99) and (1.16, 6.01) ..
			(1.19, 6.01) -- (1.76, 6.01) .. controls (1.78, 6.01) and (1.8, 5.99) ..
			(1.8, 5.96) -- (1.8, 5.23) .. controls (1.8, 5.2) and (1.78, 5.19) ..
			(1.76, 5.19) --cycle
			(1.23, 5.28) -- (1.71, 5.28) -- (1.71, 5.92) -- (1.23, 5.92) -- (1.23, 5.28) --cycle
			(1.23, 5.28);
			
			\path[fill=c565859,nonzero rule] (1.19, 5.92) -- (0.62, 5.92) .. controls (0.59, 5.92) and (0.57, 5.94) ..
			(0.57, 5.96) -- (0.57, 6.7) .. controls (0.57, 6.72) and (0.59, 6.74) ..
			(0.62, 6.74) -- (1.19, 6.74) .. controls (1.21, 6.74) and (1.23, 6.72) ..
			(1.23, 6.7) -- (1.23, 5.96) .. controls (1.23, 5.94) and (1.21, 5.92) ..
			(1.19, 5.92) --cycle
			(0.66, 6.01) -- (1.14, 6.01) -- (1.14, 6.65) -- (0.66, 6.65) --cycle
			(0.66, 6.01);
			
			\path[fill=c565859,nonzero rule] (1.76, 4.45) .. controls (1.74, 4.45) and (1.73, 4.46) ..
			(1.72, 4.47) -- (1.15, 5.2) .. controls (1.15, 5.2) and (1.15, 5.2) ..
			(1.15, 5.2) -- (0.58, 5.94) .. controls (0.56, 5.96) and (0.56, 5.98) ..
			(0.58, 6) .. controls (0.6, 6.02) and (0.63, 6.01) ..
			(0.65, 5.99) -- (1.22, 5.26) .. controls (1.22, 5.26) and (1.22, 5.26) ..
			(1.22, 5.26) -- (1.79, 4.52) .. controls (1.81, 4.5) and (1.8, 4.48) ..
			(1.78, 4.46) .. controls (1.77, 4.45) and (1.77, 4.45) ..
			(1.76, 4.45) --cycle
			(1.19, 4.45) .. controls (1.17, 4.45) and (1.16, 4.46) ..
			(1.15, 4.47) -- (0.58, 5.2) .. controls (0.56, 5.22) and (0.57, 5.25) ..
			(0.59, 5.26) .. controls (0.61, 5.28) and (0.63, 5.28) ..
			(0.65, 5.26) -- (1.22, 4.52) .. controls (1.24, 4.5) and (1.23, 4.48) ..
			(1.21, 4.46) .. controls (1.2, 4.45) and (1.2, 4.45) ..
			(1.19, 4.45) --cycle
			(1.76, 5.19) .. controls (1.74, 5.19) and (1.73, 5.19) ..
			(1.72, 5.2) -- (1.15, 5.94) -- (0.58, 6.67) .. controls (0.56, 6.69) and (0.57, 6.72) ..
			(0.59, 6.73) .. controls (0.61, 6.75) and (0.63, 6.75) ..
			(0.65, 6.73) -- (1.79, 5.26) .. controls (1.81, 5.24) and (1.8, 5.21) ..
			(1.78, 5.19) .. controls (1.77, 5.19) and (1.77, 5.19) ..
			(1.76, 5.19) --cycle
			(1.76, 5.92) .. controls (1.74, 5.92) and (1.73, 5.92) ..
			(1.72, 5.94) -- (1.15, 6.67) .. controls (1.13, 6.69) and (1.14, 6.72) ..
			(1.16, 6.73) .. controls (1.18, 6.75) and (1.2, 6.75) ..
			(1.22, 6.73) -- (1.79, 5.99) .. controls (1.81, 5.97) and (1.8, 5.94) ..
			(1.78, 5.93) .. controls (1.77, 5.92) and (1.77, 5.92) ..
			(1.76, 5.92) --cycle
			(1.76, 5.92);
			
			\path[fill=c565859,nonzero rule] (1.19, 4.45) .. controls (1.18, 4.45) and (1.17, 4.45) ..
			(1.16, 4.46) .. controls (1.14, 4.48) and (1.13, 4.5) ..
			(1.15, 4.52) -- (1.72, 5.26) .. controls (1.74, 5.28) and (1.77, 5.28) ..
			(1.78, 5.26) .. controls (1.8, 5.25) and (1.81, 5.22) ..
			(1.79, 5.2) -- (1.22, 4.47) .. controls (1.21, 4.46) and (1.2, 4.45) ..
			(1.19, 4.45) --cycle
			(0.62, 4.45) .. controls (0.61, 4.45) and (0.6, 4.45) ..
			(0.59, 4.46) .. controls (0.57, 4.48) and (0.56, 4.5) ..
			(0.58, 4.52) -- (1.72, 5.99) .. controls (1.74, 6.01) and (1.77, 6.01) ..
			(1.78, 6) .. controls (1.8, 5.98) and (1.81, 5.95) ..
			(1.79, 5.93) -- (0.65, 4.47) .. controls (0.64, 4.46) and (0.63, 4.45) ..
			(0.62, 4.45) --cycle
			(0.62, 5.19) .. controls (0.61, 5.19) and (0.6, 5.19) ..
			(0.59, 5.19) .. controls (0.57, 5.21) and (0.56, 5.24) ..
			(0.58, 5.26) -- (1.15, 5.99) -- (1.72, 6.73) .. controls (1.74, 6.75) and (1.77, 6.75) ..
			(1.78, 6.73) .. controls (1.8, 6.72) and (1.81, 6.69) ..
			(1.79, 6.67) -- (1.22, 5.94) -- (0.65, 5.2) .. controls (0.64, 5.19) and (0.63, 5.19) ..
			(0.62, 5.19) --cycle
			(0.62, 5.92) .. controls (0.61, 5.92) and (0.6, 5.92) ..
			(0.59, 5.93) .. controls (0.57, 5.94) and (0.56, 5.97) ..
			(0.58, 5.99) -- (1.15, 6.73) .. controls (1.17, 6.75) and (1.2, 6.75) ..
			(1.21, 6.73) .. controls (1.23, 6.72) and (1.24, 6.69) ..
			(1.22, 6.67) -- (0.65, 5.94) .. controls (0.64, 5.92) and (0.63, 5.92) ..
			(0.62, 5.92) --cycle
			(0.62, 5.92);
			
			\path[fill=c211c1d,nonzero rule] (2.11, 0.14) -- (2.37, 0.14) -- (2.37, 0) -- (2.11, 0) --cycle
			(2.11, 0.14);
			
			\path[fill=c211c1d,nonzero rule] (1.06, 0.14) -- (1.31, 0.14) -- (1.31, 0) -- (1.06, 0) --cycle
			(1.06, 0.14);
			
			\path[fill=c211c1d,nonzero rule] (0, 0.14) -- (0.26, 0.14) -- (0.26, 0) -- (0, 0) --cycle
			(0, 0.14);
			
			\path[fill=c565859,nonzero rule] (1.74, 6.85) .. controls (1.74, 6.85) and (1.73, 6.85) ..
			(1.72, 6.86) -- (1.19, 7.88) -- (0.64, 6.86) .. controls (0.64, 6.85) and (0.62, 6.85) ..
			(0.61, 6.85) .. controls (0.6, 6.86) and (0.6, 6.87) ..
			(0.61, 6.88) -- (1.17, 7.94) .. controls (1.17, 7.95) and (1.18, 7.95) ..
			(1.19, 7.95) .. controls (1.2, 7.95) and (1.2, 7.95) ..
			(1.21, 7.94) -- (1.76, 6.88) .. controls (1.77, 6.87) and (1.77, 6.86) ..
			(1.75, 6.85) .. controls (1.75, 6.85) and (1.75, 6.85) ..
			(1.74, 6.85) --cycle
			(1.74, 6.85);
			
			\path[fill=c7a7973,nonzero rule] (0.45, 6.87) -- (1.93, 6.87) -- (1.93, 6.65) -- (0.45, 6.65) --cycle
			(0.45, 6.87);
			
			\path[fill=c211c1d,nonzero rule] (0.43, 6.2) -- (0.78, 6.2) -- (0.78, 5.02) -- (0.43, 5.02) --cycle
			(0.43, 6.2);
			
			\path[fill=c7a7973,nonzero rule] (0.43, 6.2) -- (0.65, 6.2) -- (0.65, 5.02) -- (0.43, 5.02) --cycle
			(0.43, 6.2);
			
			\path[fill=c211c1d,nonzero rule] (1.51, 5.95) -- (1.86, 5.95) -- (1.86, 4.59) -- (1.51, 4.59) --cycle
			(1.51, 5.95);
			
			\path[fill=c7a7973,nonzero rule] (1.64, 5.95) -- (1.86, 5.95) -- (1.86, 4.59) -- (1.64, 4.59) --cycle
			(1.64, 5.95);
			
			\path[fill=c7a7973,nonzero rule] (1.89, 4.12) .. controls (1.76, 4.12) and (1.6, 4.15) ..
			(1.6, 3.74) .. controls (1.6, 3.33) and (1.76, 3.35) ..
			(1.89, 3.35) .. controls (2.02, 3.35) and (2.13, 3.52) ..
			(2.13, 3.74) .. controls (2.13, 3.95) and (2.02, 4.12) ..
			(1.89, 4.12) --cycle
			(1.89, 4.12);
			
			\path[fill=c211c1d,nonzero rule] (1.89, 4.12) -- (1.73, 4.12) .. controls (1.65, 4.12) and (1.57, 4.05) ..
			(1.53, 3.93) .. controls (1.48, 3.81) and (1.48, 3.67) ..
			(1.53, 3.55) .. controls (1.57, 3.42) and (1.65, 3.35) ..
			(1.73, 3.35) -- (1.89, 3.35) .. controls (1.81, 3.35) and (1.73, 3.43) ..
			(1.69, 3.55) .. controls (1.65, 3.67) and (1.65, 3.81) ..
			(1.69, 3.93) .. controls (1.73, 4.05) and (1.81, 4.12) ..
			(1.89, 4.12) --cycle
			(1.89, 4.12);
			
			\path[fill=c7a7973,nonzero rule] (1.51, 6.12) .. controls (1.51, 6.1) and (1.51, 6.07) ..
			(1.5, 6.05) .. controls (1.5, 6.02) and (1.49, 6) ..
			(1.49, 5.97) .. controls (1.48, 5.95) and (1.47, 5.93) ..
			(1.46, 5.91) .. controls (1.44, 5.89) and (1.43, 5.87) ..
			(1.42, 5.85) .. controls (1.4, 5.83) and (1.39, 5.82) ..
			(1.37, 5.8) .. controls (1.35, 5.79) and (1.33, 5.78) ..
			(1.31, 5.77) .. controls (1.29, 5.76) and (1.27, 5.75) ..
			(1.25, 5.74) .. controls (1.23, 5.74) and (1.21, 5.74) ..
			(1.19, 5.74) .. controls (1.17, 5.74) and (1.15, 5.74) ..
			(1.13, 5.74) .. controls (1.11, 5.75) and (1.09, 5.76) ..
			(1.07, 5.77) .. controls (1.05, 5.78) and (1.03, 5.79) ..
			(1.01, 5.8) .. controls (0.99, 5.82) and (0.98, 5.83) ..
			(0.96, 5.85) .. controls (0.95, 5.87) and (0.93, 5.89) ..
			(0.92, 5.91) .. controls (0.91, 5.93) and (0.9, 5.95) ..
			(0.89, 5.97) .. controls (0.88, 6) and (0.88, 6.02) ..
			(0.87, 6.05) .. controls (0.87, 6.07) and (0.87, 6.1) ..
			(0.87, 6.12) .. controls (0.87, 6.15) and (0.87, 6.17) ..
			(0.87, 6.2) .. controls (0.88, 6.22) and (0.88, 6.25) ..
			(0.89, 6.27) .. controls (0.9, 6.29) and (0.91, 6.32) ..
			(0.92, 6.34) .. controls (0.93, 6.36) and (0.95, 6.38) ..
			(0.96, 6.4) .. controls (0.98, 6.41) and (0.99, 6.43) ..
			(1.01, 6.44) .. controls (1.03, 6.46) and (1.05, 6.47) ..
			(1.07, 6.48) .. controls (1.09, 6.49) and (1.11, 6.5) ..
			(1.13, 6.5) .. controls (1.15, 6.51) and (1.17, 6.51) ..
			(1.19, 6.51) .. controls (1.21, 6.51) and (1.23, 6.51) ..
			(1.25, 6.5) .. controls (1.27, 6.5) and (1.29, 6.49) ..
			(1.31, 6.48) .. controls (1.33, 6.47) and (1.35, 6.46) ..
			(1.37, 6.44) .. controls (1.39, 6.43) and (1.4, 6.41) ..
			(1.42, 6.4) .. controls (1.43, 6.38) and (1.44, 6.36) ..
			(1.46, 6.34) .. controls (1.47, 6.32) and (1.48, 6.29) ..
			(1.49, 6.27) .. controls (1.49, 6.25) and (1.5, 6.22) ..
			(1.5, 6.2) .. controls (1.51, 6.17) and (1.51, 6.15) ..
			(1.51, 6.12) --cycle
			(1.51, 6.12);
			
			\path[fill=c565859,nonzero rule] (1.41, 6.85) .. controls (1.4, 6.85) and (1.4, 6.86) ..
			(1.39, 6.87) -- (1.19, 7.83) -- (0.98, 6.87) .. controls (0.97, 6.86) and (0.96, 6.85) ..
			(0.95, 6.85) .. controls (0.94, 6.85) and (0.93, 6.87) ..
			(0.93, 6.88) -- (1.17, 7.93) .. controls (1.17, 7.94) and (1.18, 7.95) ..
			(1.19, 7.95) .. controls (1.2, 7.95) and (1.2, 7.94) ..
			(1.21, 7.93) -- (1.44, 6.88) .. controls (1.44, 6.86) and (1.43, 6.85) ..
			(1.42, 6.85) .. controls (1.41, 6.85) and (1.41, 6.85) ..
			(1.41, 6.85) --cycle
			(1.41, 6.85);
			
			\path[fill=c211c1d,nonzero rule] (1.47, 8.09) .. controls (1.42, 8.09) and (1.38, 8.07) ..
			(1.36, 8.03) -- (1.01, 8.03) .. controls (0.99, 8.07) and (0.95, 8.09) ..
			(0.9, 8.09) .. controls (0.83, 8.09) and (0.78, 8.03) ..
			(0.78, 7.96) .. controls (0.78, 7.89) and (0.83, 7.83) ..
			(0.9, 7.83) .. controls (0.95, 7.83) and (0.99, 7.85) ..
			(1.01, 7.89) -- (1.36, 7.89) .. controls (1.38, 7.85) and (1.42, 7.83) ..
			(1.47, 7.83) .. controls (1.53, 7.83) and (1.59, 7.89) ..
			(1.59, 7.96) .. controls (1.59, 8.03) and (1.53, 8.09) ..
			(1.47, 8.09) --cycle
			(1.47, 8.09);
			
			\path[fill=c211c1d,nonzero rule] (1.47, 8.54) .. controls (1.42, 8.54) and (1.38, 8.52) ..
			(1.36, 8.47) -- (1.01, 8.47) .. controls (0.99, 8.51) and (0.95, 8.54) ..
			(0.9, 8.54) .. controls (0.83, 8.54) and (0.78, 8.48) ..
			(0.78, 8.41) .. controls (0.78, 8.33) and (0.83, 8.27) ..
			(0.9, 8.27) .. controls (0.95, 8.27) and (0.99, 8.3) ..
			(1.01, 8.34) -- (1.36, 8.34) .. controls (1.38, 8.3) and (1.42, 8.27) ..
			(1.47, 8.27) .. controls (1.53, 8.27) and (1.59, 8.33) ..
			(1.59, 8.41) .. controls (1.59, 8.48) and (1.53, 8.54) ..
			(1.47, 8.54) --cycle
			(1.47, 8.54);
			
			\path[fill=c211c1d,nonzero rule] (1.16, 8.78) -- (1.2, 8.78) -- (1.2, 7.96) -- (1.16, 7.96) --cycle
			(1.16, 8.78);
			
			\path[fill=c7a7973,nonzero rule] (0.98, 3.06) .. controls (0.85, 3.06) and (0.69, 3.08) ..
			(0.69, 2.67) .. controls (0.69, 2.26) and (0.85, 2.28) ..
			(0.98, 2.28) .. controls (1.11, 2.28) and (1.22, 2.46) ..
			(1.22, 2.67) .. controls (1.22, 2.89) and (1.11, 3.06) ..
			(0.98, 3.06) --cycle
			(0.98, 3.06);
			
			\path[fill=c211c1d,nonzero rule] (0.98, 3.06) -- (0.82, 3.06) .. controls (0.73, 3.06) and (0.66, 2.98) ..
			(0.61, 2.87) .. controls (0.57, 2.75) and (0.57, 2.6) ..
			(0.61, 2.48) .. controls (0.66, 2.36) and (0.73, 2.29) ..
			(0.82, 2.29) -- (0.98, 2.29) .. controls (0.9, 2.29) and (0.82, 2.36) ..
			(0.78, 2.48) .. controls (0.73, 2.6) and (0.73, 2.75) ..
			(0.78, 2.87) .. controls (0.82, 2.98) and (0.9, 3.06) ..
			(0.98, 3.06) --cycle
			(0.98, 3.06);
			
			\path[fill=c7a7973,nonzero rule] (1.88, 2.14) .. controls (2.01, 2.14) and (2.18, 2.16) ..
			(2.18, 1.75) .. controls (2.18, 1.34) and (2.01, 1.36) ..
			(1.88, 1.36) .. controls (1.75, 1.36) and (1.65, 1.54) ..
			(1.65, 1.75) .. controls (1.65, 1.96) and (1.75, 2.14) ..
			(1.88, 2.14) --cycle
			(1.88, 2.14);
			
			\path[fill=c211c1d,nonzero rule] (1.88, 2.14) -- (2.05, 2.14) .. controls (2.13, 2.14) and (2.21, 2.06) ..
			(2.25, 1.94) .. controls (2.29, 1.82) and (2.29, 1.68) ..
			(2.25, 1.56) .. controls (2.21, 1.44) and (2.13, 1.36) ..
			(2.05, 1.36) -- (1.88, 1.36) .. controls (1.97, 1.36) and (2.05, 1.44) ..
			(2.09, 1.56) .. controls (2.13, 1.68) and (2.13, 1.82) ..
			(2.09, 1.94) .. controls (2.05, 2.06) and (1.97, 2.14) ..
			(1.88, 2.14) --cycle
			(1.88, 2.14);
			
			\path[fill=c7a7973,nonzero rule] (0.53, 4.68) .. controls (0.66, 4.68) and (0.82, 4.7) ..
			(0.82, 4.29) .. controls (0.82, 3.88) and (0.66, 3.9) ..
			(0.53, 3.9) .. controls (0.4, 3.9) and (0.29, 4.07) ..
			(0.29, 4.29) .. controls (0.29, 4.5) and (0.4, 4.68) ..
			(0.53, 4.68) --cycle
			(0.53, 4.68);
			
			\path[fill=c211c1d,nonzero rule] (0.53, 4.68) -- (0.69, 4.68) .. controls (0.77, 4.68) and (0.85, 4.6) ..
			(0.89, 4.48) .. controls (0.94, 4.36) and (0.94, 4.22) ..
			(0.89, 4.1) .. controls (0.85, 3.98) and (0.77, 3.9) ..
			(0.69, 3.9) -- (0.53, 3.9) .. controls (0.61, 3.9) and (0.69, 3.98) ..
			(0.73, 4.1) .. controls (0.77, 4.22) and (0.77, 4.36) ..
			(0.73, 4.48) .. controls (0.69, 4.6) and (0.61, 4.68) ..
			(0.53, 4.68) --cycle
			(0.53, 4.68);
			
		\end{scope}
		}}
		
		\begin{tikzpicture}[scale=1, font=\footnotesize,line join=round, line cap=round, >=stealth]
				%\draw[gray,xstep = 1, ystep = 1] (0,0) grid (5,5);
				
				\path (-.35,0) pic[scale=.3]{radar}
				(0,0) coordinate (O)
				+(-30:2) coordinate (x) node[above right] {$x$ (Đ)}
				+(150:2) coordinate (x')
				+(40:5) coordinate (y) node[below right] {$y$ (B)}
				+(90:4) coordinate (z) node[right] {$z$}
				(3.3,3.6) coordinate (H)
				++(-90:2.5) coordinate (F)
				++(180:1.5) coordinate (G)
				
				;
				\draw[-stealth] (x')--(x);
				\draw[-stealth] (O)--(y);
				\draw[-stealth] (O)--(z);
				\draw (G)--(H)
				pic[draw, thin, angle radius = 12pt]{angle = F--G--H}
				pic[draw, thin, angle radius = 6pt]{right angle = H--F--G}
				
				;
				\draw[dashed] (G)--(F)--(H);
				
				\foreach \x/\g in {H/-45,O/-90,F/-90,G/-90}\fill (\x) circle (1pt)+(\g:3mm) node{$\x$};
				
		\end{tikzpicture}
	\end{center}
	\choiceTF
	{\True Tọa độ điểm $G \left(1; 0{,}5; 0 \right)$}
	{Phương trình đường thẳng $GH$ là $\dfrac{x-1}{1} = \dfrac{0{,}5 - y}{2} = \dfrac{x}{2}$}
	{\True Hình chiếu của đường thẳng $GH$ xuống mặt đất là đường thẳng đi qua điểm $I \left(2; 2{,}5; 0 \right)$}
	{\True Giả sử một đỉnh núi nằm ở điểm $M \left(5; 4{,}5; 3\right)$. Khi $HM$ vuông góc với đường bay $GH$ thì khoảng cách từ máy bay đến đỉnh núi tại thời điểm đó bằng $\sqrt{5}$ (km)}
	\loigiai{
	\begin{itemchoice}
			\itemch \textbf{Đúng}. Khi $t = 0$ ta có $G \left(1;0{,}5 ; 0 \right)$.
			\itemch \textbf{Sai}. Tại $t = 1$ ta có trực thăng ở vị trí có toạ độ $K\left(2; 2{,}5; 2 \right)$.\\
			Ta có $1$ vectơ chỉ phương của $AH$ là $\overrightarrow{u} = \left(1; 2; 2 \right)$.\\
			Suy ra phương trình đường $GH$ là $\dfrac{x-1}{1} = \dfrac{y - 0{,}5}{2} = \dfrac{z}{2}$.
			\itemch \textbf{Đúng}. Gọi $T$ là hình chiếu vuông góc của $K$ lên mặt đất khi đó $T\left(2; 2{,}5; 0 \right)$.\\
			Hình chiếu vuông góc của $GH$ xuống mặt đất là đường thẳng đi qua hai điểm $G$, $T$.\\
			Ta có $\overrightarrow{GT} = \left(1; 2; 0 \right)$ là một vectơ chỉ phương của $GT$.\\
			Phương trình đường thẳng $GT$ là $\heva{&x = 1 + t\\&y= 0{,}5 + 2t\\&z=0.}$\\
			Với $t = 1$ ta có $I \left(2; 2{,}5; 0\right)$ thuộc $GH$.\\
			Vậy hình chiếu vuông góc của $GH$ xuống mặt đất đi qua điểm $I \left(2; 2{,}5; 0\right)$.
			\itemch \textbf{Đúng}. Ta có $GH \colon \heva{&x = 1 + t\\&y = 0{,}5 + 2t\\&z = 2t}$ suy ra $H \left(1 + t; 0{,}5 + 2t; 2t \right)$.\\
			Khi đó $\overrightarrow{MH} = \left(t - 4; 2t - 4; 2t - 3 \right)$.\\
			Do $MH \perp GH$ nên ta có $\overrightarrow{MH} \cdot \overrightarrow{u} = 0 \Rightarrow t - 4 + 2\left(2t-4 \right) + 2 \left(2t-3 \right) = 0 \Rightarrow t = 2$.\\
			Suy ra $\overrightarrow{MH} = \left(-2; 0; 1 \right)$ suy ra $MH = \sqrt{5}$ km.
	\end{itemchoice}
	}
\end{ex}

\begin{ex}%[Nguồn: Bộ đề minh họa Moon 2024-2025]%[2H5V2-8]
	Trong một khu du lịch, người ta cho du khách trải nghiệm thiên nhiên bằng cách đu theo đường trượt zipline (là đường thẳng) từ vị trí $A$ cao 15 m của tháp một này sang vị trí $B$ cao $10$ m của tháp hai trong khung cảnh tuyệt đẹp xung quanh. Với hệ trục tọa độ $Oxyz$ cho trước (đơn vị: mét), tọa độ của $A$ và $B$ lần lượt là $(3; 2,5; 15)$ và $(21; 27,5; 10)$.
	\begin{center}
		\begin{tikzpicture}[line join=round, line cap=round,>=stealth,thick,scale=0.6]
			\path[fill=black,nonzero rule] (5.92, 8.84) .. controls (5.91, 8.84) and (5.89, 8.84) ..
			(5.89, 8.85) -- (5.84, 8.9) .. controls (5.8, 8.94) and (5.75, 8.95) ..
			(5.7, 8.94) -- (5.67, 8.93) .. controls (5.61, 8.91) and (5.55, 8.93) ..
			(5.52, 8.97) -- (5.41, 9.12) .. controls (5.4, 9.13) and (5.4, 9.14) ..
			(5.4, 9.15) .. controls (5.4, 9.16) and (5.41, 9.17) ..
			(5.42, 9.17) -- (5.45, 9.19) .. controls (5.46, 9.2) and (5.47, 9.2) ..
			(5.47, 9.2) -- (5.57, 9.19) .. controls (5.59, 9.19) and (5.6, 9.19) ..
			(5.61, 9.2) -- (5.72, 9.26) .. controls (5.74, 9.27) and (5.76, 9.26) ..
			(5.77, 9.24) .. controls (5.78, 9.23) and (5.77, 9.2) ..
			(5.76, 9.19) -- (5.66, 9.13) .. controls (5.66, 9.13) and (5.66, 9.13) ..
			(5.66, 9.13) -- (5.61, 9.11) -- (5.62, 9.1) .. controls (5.64, 9.07) and (5.68, 9.06) ..
			(5.71, 9.06) -- (5.74, 9.06) .. controls (5.79, 9.07) and (5.84, 9.05) ..
			(5.87, 9) -- (5.95, 8.9) .. controls (5.95, 8.89) and (5.96, 8.88) ..
			(5.95, 8.87) .. controls (5.95, 8.86) and (5.95, 8.85) ..
			(5.94, 8.84) .. controls (5.93, 8.84) and (5.92, 8.84) ..
			(5.92, 8.84) --cycle
			(5.74, 8.97) .. controls (5.78, 8.97) and (5.82, 8.95) ..
			(5.85, 8.92) -- (5.9, 8.86) .. controls (5.91, 8.86) and (5.92, 8.86) ..
			(5.93, 8.86) .. controls (5.93, 8.86) and (5.93, 8.87) ..
			(5.93, 8.87) .. controls (5.93, 8.88) and (5.93, 8.88) ..
			(5.93, 8.89) -- (5.86, 8.99) .. controls (5.83, 9.03) and (5.78, 9.05) ..
			(5.74, 9.04) -- (5.71, 9.04) .. controls (5.67, 9.04) and (5.63, 9.06) ..
			(5.6, 9.09) -- (5.59, 9.11) .. controls (5.58, 9.11) and (5.58, 9.12) ..
			(5.58, 9.12) .. controls (5.59, 9.12) and (5.59, 9.12) ..
			(5.59, 9.13) -- (5.65, 9.15) -- (5.75, 9.21) .. controls (5.75, 9.21) and (5.76, 9.22) ..
			(5.75, 9.23) .. controls (5.75, 9.24) and (5.74, 9.24) ..
			(5.73, 9.24) -- (5.62, 9.18) .. controls (5.6, 9.17) and (5.59, 9.17) ..
			(5.57, 9.17) -- (5.47, 9.18) .. controls (5.47, 9.18) and (5.47, 9.17) ..
			(5.47, 9.17) -- (5.43, 9.15) .. controls (5.42, 9.15) and (5.42, 9.15) ..
			(5.42, 9.14) .. controls (5.42, 9.14) and (5.42, 9.13) ..
			(5.42, 9.13) -- (5.54, 8.98) .. controls (5.57, 8.95) and (5.61, 8.93) ..
			(5.66, 8.95) -- (5.69, 8.96) .. controls (5.71, 8.96) and (5.72, 8.97) ..
			(5.74, 8.97) --cycle
			(5.74, 8.97);
			
			\path[fill=black,nonzero rule] (5.39, 9.19) .. controls (5.35, 9.19) and (5.31, 9.22) ..
			(5.31, 9.27) .. controls (5.31, 9.31) and (5.34, 9.34) ..
			(5.39, 9.35) .. controls (5.43, 9.35) and (5.47, 9.31) ..
			(5.47, 9.27) .. controls (5.47, 9.23) and (5.43, 9.19) ..
			(5.39, 9.19) --cycle
			(5.39, 9.32) .. controls (5.36, 9.32) and (5.33, 9.3) ..
			(5.33, 9.27) .. controls (5.33, 9.24) and (5.36, 9.21) ..
			(5.39, 9.21) .. controls (5.42, 9.21) and (5.45, 9.24) ..
			(5.45, 9.27) .. controls (5.44, 9.3) and (5.42, 9.32) ..
			(5.39, 9.32) --cycle
			(5.39, 9.32);
			
			\path[fill=black,nonzero rule] (5.73, 9.04) .. controls (5.72, 9.04) and (5.72, 9.05) ..
			(5.72, 9.05) -- (5.71, 9.18) .. controls (5.71, 9.19) and (5.72, 9.19) ..
			(5.72, 9.19) .. controls (5.73, 9.19) and (5.74, 9.19) ..
			(5.74, 9.18) -- (5.74, 9.05) .. controls (5.74, 9.05) and (5.73, 9.04) ..
			(5.73, 9.04) --cycle
			(5.73, 9.04);
			
			\path[fill=black,nonzero rule] (5.72, 9.24) .. controls (5.72, 9.24) and (5.71, 9.24) ..
			(5.71, 9.25) -- (5.71, 9.45) .. controls (5.71, 9.46) and (5.71, 9.46) ..
			(5.72, 9.46) .. controls (5.72, 9.46) and (5.73, 9.46) ..
			(5.73, 9.45) -- (5.73, 9.25) .. controls (5.73, 9.24) and (5.73, 9.24) ..
			(5.72, 9.24) --cycle
			(5.72, 9.24);
			
			\path[fill=black,nonzero rule] (5.72, 9.44) .. controls (5.68, 9.44) and (5.65, 9.47) ..
			(5.65, 9.5) .. controls (5.65, 9.54) and (5.68, 9.57) ..
			(5.71, 9.57) .. controls (5.75, 9.57) and (5.78, 9.54) ..
			(5.78, 9.51) .. controls (5.78, 9.47) and (5.75, 9.44) ..
			(5.72, 9.44) --cycle
			(5.71, 9.55) .. controls (5.69, 9.54) and (5.67, 9.53) ..
			(5.67, 9.5) .. controls (5.68, 9.48) and (5.69, 9.46) ..
			(5.72, 9.46) .. controls (5.74, 9.46) and (5.76, 9.48) ..
			(5.76, 9.51) .. controls (5.76, 9.53) and (5.74, 9.55) ..
			(5.71, 9.55) --cycle
			(5.71, 9.55);
			
			\path[fill=black,nonzero rule] (6.25, 9.19) .. controls (6.25, 9.19) and (6.25, 9.19) ..
			(6.25, 9.19) -- (5.76, 9.47) .. controls (5.75, 9.48) and (5.75, 9.48) ..
			(5.75, 9.49) .. controls (5.76, 9.49) and (5.76, 9.49) ..
			(5.77, 9.49) -- (6.26, 9.21) .. controls (6.26, 9.21) and (6.27, 9.2) ..
			(6.26, 9.2) .. controls (6.26, 9.19) and (6.26, 9.19) ..
			(6.25, 9.19) --cycle
			(6.25, 9.19);
			
			\path[fill=black,nonzero rule] (5.67, 9.52) .. controls (5.67, 9.52) and (5.67, 9.52) ..
			(5.67, 9.52) -- (5.31, 9.73) .. controls (5.31, 9.73) and (5.31, 9.73) ..
			(5.31, 9.74) .. controls (5.31, 9.75) and (5.32, 9.75) ..
			(5.32, 9.74) -- (5.68, 9.54) .. controls (5.68, 9.54) and (5.69, 9.53) ..
			(5.68, 9.53) .. controls (5.68, 9.52) and (5.68, 9.52) ..
			(5.67, 9.52) --cycle
			(5.67, 9.52);
			
			\path[fill=black,nonzero rule] (1.55, 11.85) -- (0.78, 11.5) -- (1.11, 11.5) -- (1.11, 11.17) -- (0.84, 11.17) -- (0.84, 10.71) -- (1.02, 10.71) -- (1.03, 10.7) -- (0.42, 6.72) -- (0.42, 6.7) -- (-0, 6.7) -- (-0, 6.56) -- (3.09, 6.56) -- (3.09, 6.7) -- (2.68, 6.7) -- (2.67, 6.72) -- (2.06, 10.7) -- (2.07, 10.71) -- (2.21, 10.71) -- (2.21, 11.17) -- (1.95, 11.17) -- (1.95, 11.5) -- (2.31, 11.5) --cycle
			(1.2, 11.37) -- (1.49, 11.37) -- (1.49, 11.17) -- (1.2, 11.17) --cycle
			(1.56, 11.37) -- (1.86, 11.37) -- (1.86, 11.17) -- (1.56, 11.17) --cycle
			(0.88, 11.13) -- (1.11, 11.13) -- (1.11, 11.07) -- (0.88, 11.07) --cycle
			(1.2, 11.13) -- (1.49, 11.13) -- (1.49, 11.07) -- (1.2, 11.07) --cycle
			(1.56, 11.13) -- (1.86, 11.13) -- (1.86, 11.07) -- (1.56, 11.07) --cycle
			(1.95, 11.13) -- (2.18, 11.13) -- (2.18, 11.07) -- (1.95, 11.07) --cycle
			(0.88, 11.04) -- (1.11, 11.04) -- (1.11, 10.98) -- (0.88, 10.98) --cycle
			(1.2, 11.04) -- (1.49, 11.04) -- (1.49, 10.98) -- (1.2, 10.98) --cycle
			(1.56, 11.04) -- (1.86, 11.04) -- (1.86, 10.98) -- (1.56, 10.98) --cycle
			(1.95, 11.04) -- (2.18, 11.04) -- (2.18, 10.98) -- (1.95, 10.98) --cycle
			(1.19, 10.71) -- (1.9, 10.71) -- (1.55, 10.2) --cycle
			(1.94, 10.53) -- (2, 10.17) -- (1.7, 10.17) --cycle
			(1.15, 10.53) -- (1.39, 10.17) -- (1.09, 10.17) --cycle
			(1.07, 10.04) -- (1.44, 10.04) -- (0.97, 9.36) --cycle
			(1.65, 10.04) -- (2.02, 10.04) -- (2.12, 9.36) --cycle
			(1.55, 9.95) -- (2.03, 9.25) -- (1.06, 9.25) --cycle
			(1.08, 9.12) -- (2.01, 9.12) -- (1.54, 8.56) --cycle
			(2.16, 9.09) -- (2.26, 8.51) -- (1.68, 8.51) --cycle
			(0.93, 9.09) -- (1.4, 8.51) -- (0.84, 8.51) --cycle
			(0.81, 8.37) -- (1.39, 8.37) -- (0.68, 7.52) --cycle
			(1.69, 8.37) -- (2.28, 8.37) -- (2.41, 7.49) --cycle
			(1.54, 8.34) -- (2.3, 7.41) -- (0.78, 7.41) --cycle
			(0.65, 7.26) -- (2.42, 7.26) -- (2.44, 7.24) -- (2.45, 7.25) -- (2.54, 6.7) -- (0.56, 6.7) -- (0.64, 7.25) --cycle
			(0.65, 7.26);
			
			\path[fill=black,nonzero rule] (10.77, 7.35) -- (10.01, 7) -- (10.33, 7) -- (10.33, 6.67) -- (10.07, 6.67) -- (10.07, 6.21) -- (10.25, 6.21) -- (10.26, 6.2) -- (9.65, 2.22) -- (9.64, 2.2) -- (9.23, 2.2) -- (9.23, 2.06) -- (12.32, 2.06) -- (12.32, 2.2) -- (11.9, 2.2) -- (11.9, 2.22) -- (11.28, 6.2) -- (11.29, 6.21) -- (11.44, 6.21) -- (11.44, 6.67) -- (11.17, 6.67) -- (11.17, 7) -- (11.54, 7) --cycle
			(10.42, 6.87) -- (10.72, 6.87) -- (10.72, 6.67) -- (10.42, 6.67) --cycle
			(10.79, 6.87) -- (11.09, 6.87) -- (11.09, 6.67) -- (10.79, 6.67) --cycle
			(10.1, 6.63) -- (10.33, 6.63) -- (10.33, 6.57) -- (10.1, 6.57) --cycle
			(10.42, 6.63) -- (10.72, 6.63) -- (10.72, 6.57) -- (10.42, 6.57) --cycle
			(10.79, 6.63) -- (11.09, 6.63) -- (11.09, 6.57) -- (10.79, 6.57) --cycle
			(11.17, 6.63) -- (11.4, 6.63) -- (11.4, 6.57) -- (11.17, 6.57) --cycle
			(10.1, 6.54) -- (10.33, 6.54) -- (10.33, 6.48) -- (10.1, 6.48) --cycle
			(10.42, 6.54) -- (10.72, 6.54) -- (10.72, 6.48) -- (10.42, 6.48) --cycle
			(10.79, 6.54) -- (11.09, 6.54) -- (11.09, 6.48) -- (10.79, 6.48) --cycle
			(11.17, 6.54) -- (11.4, 6.54) -- (11.4, 6.48) -- (11.17, 6.48) --cycle
			(10.42, 6.21) -- (11.13, 6.21) -- (10.77, 5.7) --cycle
			(11.17, 6.03) -- (11.22, 5.68) -- (10.92, 5.68) --cycle
			(10.38, 6.03) -- (10.62, 5.68) -- (10.32, 5.68) --cycle
			(10.3, 5.54) -- (10.67, 5.54) -- (10.19, 4.86) --cycle
			(10.88, 5.54) -- (11.24, 5.54) -- (11.35, 4.86) --cycle
			(10.77, 5.45) -- (11.25, 4.75) -- (10.29, 4.75) --cycle
			(10.31, 4.62) -- (11.24, 4.62) -- (10.77, 4.06) --cycle
			(11.39, 4.59) -- (11.48, 4.01) -- (10.91, 4.01) --cycle
			(10.15, 4.59) -- (10.63, 4.01) -- (10.06, 4.01) --cycle
			(10.04, 3.87) -- (10.62, 3.87) -- (9.91, 3.02) --cycle
			(10.92, 3.87) -- (11.5, 3.87) -- (11.64, 2.99) --cycle
			(10.77, 3.84) -- (11.52, 2.91) -- (10, 2.91) --cycle
			(9.88, 2.76) -- (11.65, 2.76) -- (11.66, 2.74) -- (11.68, 2.75) -- (11.76, 2.2) -- (9.78, 2.2) -- (9.87, 2.75) --cycle
			(9.88, 2.76);
			
			\path[draw=black,line cap=butt,line join=miter,line width=0.07cm,miter limit=4,cm={ 0.65,-0.38,0.38,0.65,(-2.78, 2.94)}] (-0, 13.19) --
			(12, 13.19);
			
			\path[fill=black,even odd rule] (10.03, 7.11) .. controls (10.08, 7.11) and (10.12, 7.07) ..
			(10.12, 7.01) .. controls (10.12, 6.96) and (10.08, 6.92) ..
			(10.03, 6.92) .. controls (9.98, 6.92) and (9.93, 6.96) ..
			(9.93, 7.01) .. controls (9.93, 7.07) and (9.98, 7.11) ..
			(10.03, 7.11) --cycle
			(10.03, 7.11);
			
			\path[fill=black,even odd rule] (2.23, 11.61) .. controls (2.28, 11.61) and (2.32, 11.57) ..
			(2.32, 11.52) .. controls (2.32, 11.47) and (2.28, 11.42) ..
			(2.23, 11.42) .. controls (2.18, 11.42) and (2.14, 11.47) ..
			(2.14, 11.52) .. controls (2.14, 11.57) and (2.18, 11.61) ..
			(2.23, 11.61) --cycle
			(2.23, 11.61);
		\end{tikzpicture}
	\end{center}
	\loigiai{
	\begin{center}
		\begin{tikzpicture}[>=stealth,line join=round,line cap=round,font=\footnotesize,scale=.9]
			\path
			(0,0)coordinate (A)
			(6,-1)coordinate (B)
			(2,-3)coordinate (T)
			($(A)!0.6!(B)$)coordinate (M)
			($(A)!(T)!(B)$)coordinate (M_0)
			;
			\draw (A)--(B) (T)--(M) (T)--(M_0);
			\foreach \i/\j in {A/180,B/00,T/-90,M/90,M_0/90}
			\fill[black] (\i) circle (1pt) (\i) node[shift={(\j:2.5mm)}]{$\i$};
			\draw pic[draw,angle radius=2mm]{right angle=T--M_0--B};
		\end{tikzpicture}
	\end{center}
	Giả sử bạn Tuấn đang đứng ở vị trí $T$ và $M$ là vị trí của du khách.\\
	Khi đó du khách gần bạn Tuấn nhất khi $TM\perp AB$.\\
	Ta có $\vv{AB}=(18;25;-5)$.\\
	Phương trình tham số của đường zipline là $d\colon \heva{&x=3+18t\\&y=2{,}5+25t\\&z=15-5t}(t\in\mathbb{R})$.\\
	Gọi $M(3+18t;2{,}5+25t;15-5t)$, ta có $\vv{TM}=(18t-9;25t-18{,5};15-5t)$.\\
	Khi đó
	$\vv{TM}\cdot \vv{u_d}=18(18t-9)+25(25t-18{,}5)-5(15-5t)=0\Leftrightarrow t=\dfrac{1399}{1948}\Rightarrow z_M\approx11{,}4$ (m).
	}
\end{ex}

%

\begin{ex}%[Nguồn: Bộ đề minh họa Moon 2024-2025]%[2H5V3-2]
	Trong không gian với hệ tọa độ $Oxyz$, cho mặt cầu $(S)$ nhận mặt phẳng $(Oxy)$ và mặt phẳng $(P)\colon x + 2y - z - 6 = 0$ là các mặt phẳng đối xứng. Biết khoảng cách từ gốc tọa độ $O$ đến một điểm $M$ trên mặt cầu $(S)$ có giá trị lớn nhất và nhỏ nhất lần lượt là $12$ và $2$, điểm $O$ nằm ngoài khối cầu $(S)$. Biết tung độ tâm mặt cầu $(S)$ có giá trị dương, vậy hoành độ tâm mặt cầu bằng bao nhiêu? (làm tròn kết quả đến hàng phần chục).
	
	\shortans{$-4{,}6$}
	
	\loigiai{
	Gọi $I(a;b;c)$ là tâm mặt cầu $(S)$ vì $(S)$ nhận $(Oxy)$ và $(P)$ là các mặt phẳng đối xứng nên $\heva{&I\in(Oxy)\\&I\in (P)}\Leftrightarrow\heva{&c=0\\&a+2b-c-6=0}\Leftrightarrow\heva{&a=6-2t\\&b=t\\&c=0}(\forall t\in\mathbb{R})\Rightarrow I\left(6-2t;t;0\right)$.\\
	Gọi điểm $M$ là điểm thuộc mặt cầu $(S)$.\\
	Ta có $\heva{&OM_{\max}=OI+R=12\\&OM_{\min}=OI-R=2}\Rightarrow OI=7$.\\
	Khi đó \begin{eqnarray*}
		&&OI^2=49\\
		&\Leftrightarrow& (6-2t)^2+t^2+0^2=49\\
		&\Leftrightarrow& 5t^2-24t-13=0\\
		&\Leftrightarrow&\hoac{&t=\dfrac{12+\sqrt{209}}{5}~(\text{thỏa})\\&t=\dfrac{12-\sqrt{209}}{5}~(\text{loại}).}
	\end{eqnarray*}
	Vậy hoành độ của tâm $I$ là $a=6-2t=\dfrac{6-2\sqrt{209}}{5}\approx -4,6$.
	}
\end{ex}

%

\begin{ex}%[Nguồn: Bộ đề minh họa Moon 2024-2025]%[2H5V3-2]
	Trong không gian $Oxyz$, cho mặt cầu $(S)\colon(x-1)^2+(y+1)^2+(z-2)^2=9$ và điểm $M(1 ; 3 ;-1)$. Từ điểm $M$ kẻ các tiếp tuyến $MA$, $MB$, $MC$ với mặt cầu $(S)$. Tâm của đường tròn ngoại tiếp tam giác $ABC$ là điểm $I(a ; b ; c)$. Giá trị của $a+b+3c$ bằng bao nhiêu?
	\loigiai{
	Mặt cầu $(S)$ có tâm $J(1;-1;2)$ và bán kính $R=3$.\\
	$\vec{MJ}=(0;-4;3)\Rightarrow MJ=5$.\\
	Xét $\triangle MAJ$ vuông tại $A$ ta có $MA=\sqrt{MJ^2-R^2}=\sqrt{5^2-3^2}=4$.\\
	$I$ là chân đường cao hạ từ $A$ xuống $MJ$. Suy ra $\dfrac{MI}{IJ}=\dfrac{AM}{AJ}=\dfrac{4}{3}$.
	\[\Rightarrow \vec{MI}=\dfrac{4}{3}\vec{IJ}\Rightarrow \heva{&a-1=\dfrac{4}{3}(1-a)\\&b-3=\dfrac{4}{3}(-1-b)\\&c+1=\dfrac{4}{3}(2-c)}\Leftrightarrow \heva{&a=1\\&b=\dfrac{5}{7}\\&c=\dfrac{5}{7}.}\]
	Vậy $a+b+3c=1+\dfrac{5}{7}+3\cdot \dfrac{5}{7}=\dfrac{27}{7}$.
	}
\end{ex}

\begin{ex}%[Nguồn: Bộ đề minh họa Moon 2024-2025]%[2H5V3-3]
	Trong không gian với hệ tọa độ $Oxyz$, cho hai mặt cầu $(S_1)$ và $(S_2)$ có phương trình lần lượt là $(S_1)\colon x^2+(y-3)^2+z^2=4$, $(S_2)\colon (x+4)^2+y^2+z^2=9$. Mặt cầu $(S)$ có bán kính bằng $1$, tâm $I$ và tiếp xúc ngoài với cả hai mặt cầu $(S_1)$, $(S_2)$. Khoảng cách từ gốc tọa độ $O$ đến điểm $I$ lớn nhất bằng bao nhiêu?
	\shortans[]{$4{,}8$}
	\loigiai{
	\begin{center}
		\begin{tikzpicture}[line join=round, line cap=round,>=stealth,font=\footnotesize,scale=1]
			\draw[step=1cm, gray, very thin,opacity=0.3] (-5,-1) grid (1,5);
			\draw[->] (0,-1.5)--(0,5.5) node[right]{$y$};
			\draw[->] (-5,0)--(1.5,0)node[above] {$x$};
			\path
			(0,0) coordinate (O)
			(0,3) coordinate (I_1)
			(-4,0) coordinate (I_2)
			($(I_1)!(O)!(I_2)$) coordinate (H)
			($(O)!2!(H)$) coordinate (I)
			;
			\draw (O)--(I) (I_1)--(I_2)--(I)--cycle;
			\pic[draw,angle radius=0.3cm]{right angle=I_1--O--I_2};
			\pic[draw,angle radius=0.3cm]{right angle=I_2--I--I_1};
			\foreach \i/\g in {O/-45,I/90,I_1/0,I_2/-90}{\draw[fill=black](\i) circle (1pt) ($(\i)+(\g:3mm)$) node[scale=1]{$\i$};}
		\end{tikzpicture}
	\end{center}
	Ta có $(S_1)\heva{&I_1(0;3;0)\\&R_1=2}$ và
	$(S_2)\heva{&I_2(-4;0;0)\\&R_2=3.}$\\
	Ta có $I_1I_2=5$ suy ra $(S_1)$ và $(S_2)$ tiếp xúc ngoài với nhau.\\
	Mặt cầu $(S)$ tiếp xúc ngoài với cả hai mặt cầu $(S_1)$ và $(S_2)$ suy ra
	$\heva{&II_1=3\\&II_2=4}\Rightarrow \triangle II_1I_2$ vuông tại $I$.\\
	Mà tam giác $\triangle OI_1I_2$ vuông tại $O$.\\
	Suy ra để khoảng cách từ gốc tọa độ $O$ đến điểm $I$ lớn nhất khi và chỉ khi $O$, $I_1$, $I_2$ và $I$ cùng thuộc một mặt phẳng.\\
	Ta có $\cos \widehat{OI_1I_2}= \dfrac{3}{5}$.\\
	Suy ra $\cos \widehat{OI_1I} =\cos 2\widehat{OI_1I_2}= 2 \cos^2 \widehat{OI_1I_2} -1 =2\cdot \left(\dfrac{3}{5} \right)^2 -1=-\dfrac{7}{25} $.\\
	Xét tam giác $\triangle OI_1I$ có
	\begin{eqnarray*}
		&&OI^2=OI_1^2+I_1I^2-2\cdot OI_1\cdot II_1\cdot \cos \widehat{OI_1I}\\
		&\Leftrightarrow& OI^2= 3^2+3^2-2\cdot 3\cdot 3\cdot \left(-\dfrac{7}{25} \right) \\
		&\Leftrightarrow& OI= \sqrt{9+9+2\cdot 3\cdot 3\cdot \left(\dfrac{7}{25}\right) }\\
		&\Leftrightarrow&OI=4{,}8.
	\end{eqnarray*}
	Vậy khoảng cách từ gốc tọa độ $O$ đến điểm $I$ lớn nhất bằng $4{,}8$.
	}
\end{ex}

%

\begin{ex}%[Nguồn: Bộ đề minh họa Moon 2024-2025]%[2H5V3-4]
	Các thiên thạch có đường kính lớn hơn $140$ m và có thể lại gần Trái Đất ở khoảng cách nhỏ hơn $7\,500\,000$ km được coi là những vật thể có khả năng va chạm gây nguy hiểm cho Trái Đất. Để theo dõi những thiên thạch này, người ta đã thiết lập các trạm quan sát các vật thể bay gần Trái Đất. Giả sử có một hệ thống quan sát có khả năng theo dõi các vật thể ở độ cao không vượt quá $6\,600$ km so với mực nước biển. Coi Trái Đất là khối cầu
	\begin{center}
		\begin{tikzpicture}[scale=0.9]
			% Draw the inner circle
			\path
			(0:0) coordinate (O)
			(135:4) coordinate (A)
			(25:4) coordinate (B)
			(A)--(B)--([turn]0:1.5) coordinate (N)
			(B)--(A)--([turn]0:1.5) coordinate (M)
			;
			\draw[thick] (0,0) circle(2)
			(0,0) circle(4)  (M)--(N) ;
			\draw[<->]
			(O)--(2,0)node[midway,above]{6400 km};
			\draw[<->](180:2)--(180:4)node[midway,above]{6600 km}
			;
			
			\foreach \x/\g in {M/90,A/110,B/70,N/90} \draw[fill=black] (\x) circle (.05)+(\g:0.3)node{$\x$};
		\end{tikzpicture}
	\end{center}
	\choiceTF
	{\True Đường thẳng $MN$ có phương trình tham số là $\heva{&x=6+3t \\&y=20+8t \\&z=-4t} \quad(t \in \mathbb{R})$}
	{Vị trí đầu tiên thiên thạch di chuyển vào phạm vi theo dõi của hệ thống quan sát là điểm $A(-3;-4; 12)$}
	{\True Khoảng cách giữa vị trí đầu tiên và vị trí cuối cùng mà thiên thạch di chuyển trong phạm vi theo dõi của hệ thống quan sát là $18\,900$ km (kết quả làm tròn đến hàng trăm theo đơn vị ki-lô-mét)}
	{\True Nếu thời gian di chuyền của thiên thạch trong phạm vi theo dõi của hệ thống quan sát là $3$ phút thì thời gian nó di chuyền từ $M$ đến $N$ là $6$ phút}
	\loigiai{
	\begin{enumerate}
		\item Đúng.
		
		Ta có $M(6; 20; 0)$, $N(-6;-12; 16)$, suy ra $\overrightarrow{MN}=(-12;-32; 16)$. Do đó đường thẳng $MN$ có một véc-tơ chỉ phương là $\overrightarrow{u}=(3; 8;-4)$.
		Đường thẳng $MN$ đi qua $M(6; 20; 0)$ nhận $\overrightarrow{u}=(3; 8;-4)$ làm véc-tơ chỉ phương có phương trình tham số là
		$$
		\heva{&x=6+3t \\
		&y=20+8t,(t \in \mathbb{R}) \\
		&z=-4t.	}
		$$
		\item Sai.
		
		Theo đề bài ta có phạm vi theo dõi của hệ thống quan sát là mặt cầu $(O)$ có phương trình
		$$
		x^2+y^2+z^2=13^2.
		$$
		
		Vì $A, B$ thuộc đường thẳng $MN$ nên $A(6+3a; 20+8a;-4a)$, $B(6+3b; 20+8b;-4b) a, b \in \mathbb{R}$. Hơn nữa $A$, $B$ thuộc mặt cầu $(O)$ nên $a$, $b$ là hai nghiệm của phương trình.
		$$
		\begin{aligned}
			& (6+3 t)^2+(20+8 t)^2+(-4 t)^2=13^2 \\
			\Leftrightarrow	& 89 t^2+356 t-267=0 \\
			\Leftrightarrow	& {\left[\begin{array}{l}
				t=-1 \\
				t=-3.
			\end{array}\right.}
		\end{aligned}
		$$
		
		Suy ra $A(3; 12; 4)$ và $B(-3;-4; 12)$ hoặc $A(-3;-4; 12)$ và $B(3; 12; 4)$.
		
		Do $MA< MB$ nên $A(3; 12; 4)$ (trong trường hợp này thì $MA=\sqrt{89} < \sqrt{801}=MB)$.
		Vậy điểm gặp đầu tiên là $A(3; 12; 4)$.
		\item Đúng.
		
		Ta có vị trí đầu tiên $A(3; 12; 4)$ và vị trí cuối cùng $B(-3;-4; 12)$ mà thiên thạch di chuyển trong phạm vi theo dõi của hệ thống quan sát.
		
		Suy ra $AB=\sqrt{(-3-3)^2+(-4-12)^2+(12-4)^2}=2\sqrt{89}$.
		Do đơn vị độ dài trên mỗi trục là 1000 km và kết quả làm tròn đến hàng trăm theo đơn vị ki-lô-mét nên $AB\approx 18900\mathrm{~km}$.
		
		\item Đúng.
		
		Ta có $AB=2\sqrt{89}, MN=\sqrt{(-6-6)^2+(-12-20)^2+(16-0)^2}=4\sqrt{89}$.
		
		Suy ra $MN=2AB$.
		
		Vậy thời gian nó di chuyển từ $M$ đến $N$ gấp $2$ lần thời gian nó di chuyển từ $A$ đến $B$ và bằng $2\cdot 3=6$ phút.
	\end{enumerate}
	}
\end{ex}

\begin{ex}%[Nguồn: Bộ đề minh họa Moon 2024-2025]%[2H5V3-4]
	\immini{Vệ tinh hoạt động dựa trên nguyên lý của vật lý Newton - một vật thể bị kéo bởi một lực hấp dẫn từ một vật thể khác sẽ chuyển động theo một quỹ đạo elip xung quanh vật thể đó. Để đưa vệ tinh lên quỹ đạo, người ta sử dụng các loại tên lửa đẩy khác nhau để cung cấp cho vệ tinh động lượng cần thiết để thoát khỏi trọng lực của Trái Đất và duy trì quỹ đạo ổn định. Để thuận tiện ta quy ước một quỹ đạo gần tròn thành một đường tròn. }{\begin{tikzpicture}[line join = round, line cap=round,>=stealth,font=\footnotesize,transform shape,scale=0.8]
		\draw
		(0,0) circle(3.5cm)
		;
		\draw[fill=black!30]
		(0,0) circle(1.6cm);
		\draw[<->] (0:3.5)--(0:1.6) node[pos=0.5,above]{$h$ km};
		\draw[<->] (0:0)--(0:1.6) node[pos=0.5,above]{$6\, 400$ km};
		\fill
		(3.5,0)circle(1.5pt)node[above right]{$B$};
	\end{tikzpicture}}
	Trong hệ tọa độ $Oxyz$, gốc tọa độ là tâm trái đất, một vệ tinh nhân tạo tạo quỹ đạo được coi như một đường tròn có bán kính $13440$ km có điểm xuất phát là điểm $B (4032;0;-5376)$ và đây cũng là điểm gần Trái Đất nhất của vệ tinh. Quỹ đạo của vệ tinh này nằm trên mặt phẳng vuông góc với trục tung và có tâm nằm trên đường thẳng $OB$. Coi Trái Đất là hình cầu hoàn hảo có bán kính bằng $6400$ km.
	\choiceTF
	{Phương trình mặt phẳng chứa quỹ đạo của vệ tinh là $x+z=0$}
	{\True Khi xuất phát tại điểm $B$ vệ tinh đang ở độ cao $320$ km so với mặt đất}
	{Quỹ đạo của tên lửa là đường tròn có tâm $I(-4032; 0;5120)$}
	{Khi Trái Đất quay, điểm cực Nam và cực Bắc của Trái Đất không thay đổi vị trí. Biết rằng điểm cực Nam của Trái Đất có tọa độ là $(0; 3840;5120)$. Khoảng cách gần nhất giữa vệ tinh và điểm cực Nam bằng $10112$ km (làm tròn kết quả đến hàng đơn vị)}
	\loigiai{
	\begin{itemchoice}
		\itemch Quỹ đạo của vệ tinh này nằm trên mặt phẳng vuông góc với trục tung nên một vectơ chỉ phương của
		mặt phẳng chứa quỹ đạo của vệ tinh là $(0; 1; 0)$.\\
		Khi đó, phương trình mặt phẳng chứa quỹ đạo của vệ tinh có dạng $y+a=0$.\\
		Quỹ đạo đi qua $B(4032;0;-5376)$ nên $0+a=0$ hay $a=0$.\\
		Vậy phương trình mặt phẳng chứa quỹ đạo của vệ tinh là $y=0$.
		\itemch Khoảng cách ngắn nhất từ Trái Đất đến vệ tinh bằng
		$$OB-R=\sqrt{4032^{2}+0^{2}+(-5376)^{2}}-6400=320(km).$$
		Vậy khi xuất phát tại điểm $B$ vệ tình đang ở độ cao $320$ km so với mặt đất.
		\itemch
		Quỹ đạo của vệ tinh có tâm nằm trên đường thẳng $OB$ nên $I$ nằm trên đường thẳng $OB$.\\
		Mặt khác $IB=R_{qd}=13440=2\cdot OB$ nên $O$ là trung điểm của $IB$.\\
		Khi đó
		$$\begin{cases}
			x_I=2x_O-x_B \\
			y_I=2y_O-y_B \\
			z_I=2z_O-z_B
		\end{cases}
		\Leftrightarrow
		\begin{cases}
			x_I=-4032\\
			y_I=0\\
		z_I=5376\end{cases}
		\Rightarrow I(-4032; 0; 5376).$$
		\itemch
		Gọi $H$ là hình chiếu của $K$ trên mặt phẳng chứa quỹ đạo $(\alpha)\colon y=0\Rightarrow H(0; 0; 5120)$.\\
		Ta có
		\begin{itemize}
			\item $KH=\mathrm{d}(K, (\alpha))=3840$.
			\item $IH=\sqrt{4032^2+(5376-5120)^2}=64\sqrt{3985}$.
			\item $NH=IN-IH=13440-64\sqrt{3985}$.
		\end{itemize}
		Nối $I$ và $H$ cắt vệ tinh tại $N$. Khi đó:
		$$KN_{\text{min}}=\sqrt{KH^2+NH^2}=\sqrt{3840^2+(13440-64\sqrt{3985})^2} \approx 10154 \text{ (km)}.$$
		
	\end{itemchoice}
	}
\end{ex}

% Câu này không biết gắn ID sao cho đúng.

\begin{bt}%[Nguồn: Bộ đề minh họa Moon 2024-2025]%[2H5V3-4]
	Trong hệ toạ độ $Oxyz$, có một mặt cầu $(S)\colon (x-1)^2+(y-2)^2+(z+1)^2=3$ và đường thẳng $\Delta\colon \dfrac{x+4}{6}=\dfrac{y-6}{-2}=\dfrac{z-2}{-1}$. Từ điểm $M\in \Delta$ kẻ các tiếp tuyến đến mặt cầu $(S)$ và gọi $(C)$ là tập hợp các tiếp điểm. Biết diện tích hình phẳng giới hạn bởi $(C)$ đạt giá trị nhỏ nhất thì $(C)$ nằm trên mặt phẳng $x+by+cz+d=0$. Tìm $b+c+d$.
	\par\shortans{$-2$}
	\loigiai{
	Ta có mặt cầu $(S)$ có tâm $I(1;2;-1)$, bán kính $R=\sqrt3$; đường thẳng $\Delta$ có $\overrightarrow{u}=(6;-2;-1)$ là một vectơ chỉ phương.\\
	Hình phẳng được giới hạn bởi $(C)$ là một hình tròn. Gọi $AB$ là đường kính đường tròn $(C)$.\\
	Với $M\in\Delta$, gọi $H=AB\cap IM$, khi đó $H$ là tâm đường tròn $(C)$.
	\begin{center}
		\begin{tikzpicture}[font=\footnotesize, line join=round, line cap=round, >=stealth, scale=1]
			\pgfmathsetmacro\bankinh{sqrt(3)}
			\pgfmathsetmacro\goc{acos(\bankinh/3)}
			\path (0,0) coordinate (I) (\goc:\bankinh) coordinate (A) (-\goc:\bankinh) coordinate (B) (3,0) coordinate (M)
			(intersection of A--B and I--M) coordinate (H)
			pic[draw, angle radius=2mm]{right angle=A--H--I}
			pic[draw, angle radius=2mm]{right angle=I--A--M}
			;
			\draw (I) circle (\bankinh) (I)--(A)--(M)--(B)--cycle (A)--(H) (I)--(M) (A)--(B);
			\foreach \x/\g in {I/180, M/0, A/60, B/-60, H/45}{
			\fill (\x) circle (1pt)+(\g:0.3)node{$\x$};
			}
		\end{tikzpicture}
	\end{center}
	Diện tích hình tròn $(C)$ là $S_{(C)}=\pi\cdot AH^2$. Do đó $S_{(C)}$ đạt giá trị nhỏ nhất khi và chỉ khi $AH$ đạt giá trị nhỏ nhất.\\
	Ta có $AH$ đạt giá trị nhỏ nhất $\Leftrightarrow IM$ đạt giá trị nhỏ nhất.\\
	Suy ra $M$ là hình chiếu của $I$ trên $\Delta$.\\
	Ta có $M\in\Delta$ nên $M(-4+6t;6-2t;2-t)$. Khi đó $\overrightarrow{IM}=(-5+6t;4-2t;3-t)$.\\
	Do $\overrightarrow{IM}\perp\overrightarrow{u} \Leftrightarrow 6(-5+6t)-2(4-2t)-1(3-t)=0 \Leftrightarrow t=1$.\\
	Do đó $M(2;4;1)$ và $\overrightarrow{IM}=(1;2;2)$, suy ra $IM=3$.\\
	Xét $\triangle IAM$ vuông tại $A$, ta có $IH=\dfrac{IA^2}{IM}=1$.\\
	Do $\overrightarrow{IH}$ và $\overrightarrow{IM}$ cùng phương, $IH=\dfrac{1}{3}IM$, suy ra $\overrightarrow{IH}=\dfrac{1}{3}\overrightarrow{IM}$.\\
	Suy ra $H\left(\dfrac{4}{3};\dfrac{8}{3};-\dfrac{1}{3}\right)$.\\
	Mặt phẳng chứa $(C)$ đi qua điểm $H$ và nhận $\overrightarrow{IM}$ là vectơ pháp tuyến có dạng
	\begin{eqnarray*}
		&& 1\left(x-\dfrac{4}{3}\right)+2\left(y-\dfrac{8}{3}\right)+2\left(z+\dfrac{1}{3}\right) = 0 \\
		&\Leftrightarrow& x+2y+2z-6=0.
	\end{eqnarray*}
	Suy ra $b=2$, $c=2$, $d=-6$. Vậy $T=b+c+d=-2$.
	}
\end{bt}

%

\begin{ex}%[Nguồn: Bộ đề minh họa Moon 2024-2025]%[2H5V3-4]
	Trong không gian với hệ tọa độ $Oxyz$, cho mặt cầu $(S)\colon (x-1)^2+y^2+(z+2)^2=25$ và điểm $A(2; 2; 0)$. Gọi $d$ là đường thẳng nằm trong mặt phẳng $(P)\colon x-2y+2z+2=0$ và đi qua điểm $A$ sao cho $d$ cắt $(S)$ theo một dây cung có độ dài nhỏ nhất. Biết $d$ có một vectơ chỉ phương là $(a; b;-1)$. Giá trị của $a+b$ bằng bao nhiêu?
	\shortans{2}
	\loigiai{
	\begin{center}
		\begin{tikzpicture}[line cap=butt,line join=miter,>=stealth,declare function={
			R=2.5;}]
			\path (0,0)coordinate (I) (-1.5,-1.5)coordinate (A)--+(5,0)coordinate (B)
			($(A)!(I)!(B)$)coordinate (H)
			;
			\draw (I)circle (R) (A)--(I)--(H)
			pic[draw, angle radius=2.5mm]{right angle=I--H--A};
			\draw[shorten <=-2cm,red] (A)--(B)node[above]{$d$};
			\foreach \x/\goc in {A/130,I/90,H/-90}{
			\draw[fill] (\x) circle (1pt) node[shift={(\goc:7pt)},font=\small]{$\x$};}
		\end{tikzpicture}
	\end{center}
	Mặt cầu $(S)$ có tâm $I(1;0;-2)$ và bán kính $R=5$.\\
	Mặt phẳng $(P)$ có một vectơ pháp tuyến là $\vec{n}=(1;-2;2)$.\\
	Ta có $\overrightarrow{IA}=(1;2;2)\Rightarrow IA=3<R$ nên $A$ nằm trong mặt cầu $(S)$.\\
	Gọi $H$ là hình chiếu vuông góc của $I$ trên $d$. Dễ thấy $IA\ge IH$.\\
	Do đó $d$ cắt $(S)$ theo một dây cung có độ dài nhỏ nhất khi $IH$ đạt giá trị lớn nhất hay $H\equiv A$.\\
	Gọi $\vec{u}$ là một vectơ chỉ phương của $d$, theo giả thiết ta có\\ $\vec{u}=-4\big[\vec{n},\overrightarrow{IA}\big]=(2;0;-1)$.\\
	Suy ra $a=2$, $b=0$. Vậy $a+b=2+0=2$.
	}
\end{ex}


