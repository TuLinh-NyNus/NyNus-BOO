%% Template khung cho các loại câu hỏi
%% Tài liệu này chứa các mẫu cho các loại câu hỏi khác nhau
%% Sử dụng các mẫu này để tạo câu hỏi mới

%% ===== 1. TRẮC NGHIỆM MỘT PHƯƠNG ÁN ĐÚNG (MC) =====

%% Mẫu cơ bản
\begin{ex}%[Nguồn: "Nguồn câu hỏi"] %[1P1V1-6] % QuestionID
Lời dẫn câu hỏi
\choice   % Có thể thay bằng \choice[1], \choice[2], \choice[4]
{ đáp án 1}               % Đáp án sai
{ đáp án 2}               % Đáp án sai
{\True đáp án 3}         % Đáp án đúng
{ đáp án 4}               % Đáp án sai
Lời dẫn bổ sung (nếu có)
\loigiai{
    Lời giải của câu hỏi
}
\end{ex}

%% Mẫu với hình ảnh (1 cột)
\begin{ex}%[Nguồn: "Nguồn câu hỏi"] %[1P1V1-6]
[XX.Y] %subcount
Lời dẫn câu hỏi
\begin{center}
    \begin{tikzpicture}
        % Mã vẽ hình ở đây
    \end{tikzpicture}
\end{center}
\choice
{ đáp án 1}
{ đáp án 2}
{\True đáp án 3}
{ đáp án 4}
\loigiai{
    Lời giải của câu hỏi
}
\end{ex}

%% Mẫu với hình ảnh (2 cột)
\begin{ex}%[Nguồn: "Nguồn câu hỏi"] %[1P1V1-6]
[XX.Y] %subcount
\immini[thm]  %[thm] có thể không có
{Lời dẫn câu hỏi
\choice
{ đáp án 1}
{ đáp án 2}
{\True đáp án 3}
{ đáp án 4}}
{\begin{tikzpicture}
    % Mã vẽ hình ở đây
\end{tikzpicture}}
\loigiai{
    Lời giải của câu hỏi
}
\end{ex}

%% ===== 2. TRẮC NGHIỆM NHIỀU PHƯƠNG ÁN ĐÚNG (TF) =====

%% Mẫu cơ bản
\begin{ex}%[Nguồn: "Nguồn câu hỏi"] %[1P1V1-6]
Lời dẫn câu hỏi
\choiceTF   % Có thể thay bằng \choiceTF[t], \choiceTFt, \choiceTF[1]
{\True đáp án 1}         % Đáp án đúng
{ đáp án 2}              % Đáp án sai
{\True đáp án 3}         % Đáp án đúng
{ đáp án 4}              % Đáp án sai
Lời dẫn bổ sung (nếu có)
\loigiai{
    Lời giải của câu hỏi
}
\end{ex}

%% Mẫu với hình ảnh (1 cột)
\begin{ex}%[Nguồn: "Nguồn câu hỏi"] %[1P1V1-6]
[XX.Y] %subcount
Lời dẫn câu hỏi
\begin{center}
    \begin{tikzpicture}
        % Mã vẽ hình ở đây
    \end{tikzpicture}
\end{center}
\choiceTF[t]  %[t], [1], [2], t có thể không có
{\True đáp án 1}
{ đáp án 2}
{\True đáp án 3}
{ đáp án 4}
\loigiai{
    Lời giải của câu hỏi
}
\end{ex}

%% Mẫu với hình ảnh (2 cột)
\begin{ex}%[Nguồn: "Nguồn câu hỏi"] %[1P1V1-6]
[XX.Y] %subcount
\immini[thm]  %[thm] có thể không có
{Lời dẫn câu hỏi
\choiceTF[t]  %[t], [1], [2], t có thể không có
{\True đáp án 1}
{ đáp án 2}
{\True đáp án 3}
{ đáp án 4}}
{\begin{tikzpicture}
    % Mã vẽ hình ở đây
\end{tikzpicture}}
\loigiai{
    Lời giải của câu hỏi
}
\end{ex}

%% ===== 3. TRẮC NGHIỆM TRẢ LỜI NGẮN (SA) =====

%% Mẫu cơ bản
\begin{ex}%[Nguồn: "Nguồn câu hỏi"] %[1P1V1-6]
Lời dẫn câu hỏi
\shortans{'đáp án'}      % hoặc \shortans[oly]{'đáp án'}
\loigiai{
    Lời giải của câu hỏi
}
\end{ex}

%% Mẫu với hình ảnh (1 cột)
\begin{ex}%[Nguồn: "Nguồn câu hỏi"] %[1P1V1-6]
[XX.Y] %subcount
Lời dẫn câu hỏi
\begin{center}
    \begin{tikzpicture}
        % Mã vẽ hình ở đây
    \end{tikzpicture}
\end{center}
\shortans[oly]{'Câu trả lời đúng'} %[oly] có thể không có
\loigiai{
    Lời giải của câu hỏi
}
\end{ex}

%% Mẫu với hình ảnh (2 cột)
\begin{ex}%[Nguồn: "Nguồn câu hỏi"] %[1P1V1-6]
[XX.Y] %subcount
\immini[thm]  %[thm] có thể không có
{Lời dẫn câu hỏi
\shortans[oly]{'Câu trả lời đúng'} %[oly] có thể không có}
{\begin{tikzpicture}
    % Mã vẽ hình ở đây
\end{tikzpicture}}
\loigiai{
    Lời giải của câu hỏi
}
\end{ex}

%% ===== 4. CÂU HỎI TỰ LUẬN (ES) =====

%% Mẫu cơ bản
\begin{ex}%[Nguồn: "Nguồn câu hỏi"] %[1P1V1-6]
Lời dẫn câu hỏi tự luận
\loigiai{
    Lời giải của câu hỏi
}
\end{ex}

%% Mẫu với hình ảnh (1 cột)
\begin{ex}%[Nguồn: "Nguồn câu hỏi"] %[1P1V1-6]
[XX.Y] %subcount
Lời dẫn câu hỏi tự luận
\begin{center}
    \begin{tikzpicture}
        % Mã vẽ hình ở đây
    \end{tikzpicture}
\end{center}
\loigiai{
    Lời giải của câu hỏi
}
\end{ex}

%% Mẫu với hình ảnh (2 cột)
\begin{ex}%[Nguồn: "Nguồn câu hỏi"] %[1P1V1-6]
[XX.Y] %subcount
\immini[thm]  %[thm] có thể không có
{Lời dẫn câu hỏi tự luận}
{\begin{tikzpicture}
    % Mã vẽ hình ở đây
\end{tikzpicture}}
\loigiai{
    Lời giải của câu hỏi
}
\end{ex}

%% ===== 5. CÂU HỎI GHÉP ĐÔI (MA) =====

%% Mẫu cơ bản
\begin{ex}%[Nguồn: "Nguồn câu hỏi"] %[1P1V1-6]
Lời dẫn câu hỏi ghép đôi
\matching
{Mục A1} {Mục B1}
{Mục A2} {Mục B2}
{Mục A3} {Mục B3}
{Mục A4} {Mục B4}
\loigiai{
    Lời giải của câu hỏi
}
\end{ex}

%% ===== CÁC CÔNG THỨC TOÁN HỌC THƯỜNG DÙNG =====

%% Phân số
% \frac{tử số}{mẫu số}

%% Căn thức
% \sqrt{biểu thức}
% \sqrt[n]{biểu thức} % Căn bậc n

%% Chỉ số và số mũ
% a^{số mũ}
% a_{chỉ số}

%% Tích phân
% \int_{cận dưới}^{cận trên} biểu thức \, dx

%% Đạo hàm
% \frac{d}{dx}f(x)
% \frac{d^2}{dx^2}f(x)

%% Giới hạn
% \lim_{x \to a} f(x)

%% Tổng và tích
% \sum_{i=1}^{n} a_i
% \prod_{i=1}^{n} a_i

%% Ma trận
% \begin{pmatrix} a & b \\ c & d \end{pmatrix}
% \begin{bmatrix} a & b \\ c & d \end{bmatrix}

%% Hệ phương trình
% \begin{cases}
%   x + y = 1 \\
%   x - y = 2
% \end{cases}

%% Vector
% \vec{AB}
% \overrightarrow{AB}

%% ===== CÁC LỆNH ĐẶC BIỆT =====

%% Lệnh \hoac và \heva
% \hoac{
%   x + y = 1 \\
%   x - y = 2
% }

% \heva{
%   x + y = 1 \\
%   x - y = 2
% }

%% Lệnh \immini để tạo bố cục 2 cột
% \immini[thm]{nội dung bên trái}{nội dung bên phải}

%% Lệnh \ck để thêm cách giải khác
% \ck{
%   Nội dung cách giải khác
% }
