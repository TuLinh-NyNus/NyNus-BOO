\section{HÀM SỐ}
%\subsection{TÓM TẮT LÝ THUYẾT}
\subsubsection{Khái niệm hàm số}
\begin{itemize}
	\item [\iconMT] \indam{Định nghĩa:} Giả sử $x$ và $y$ là hai đại lượng biến thiên và $x$ nhận giá trị thuộc tập số $\mathscr{D}$.	Nếu với \indamm{mỗi giá trị $x$ thuộc $\mathscr{D}$}, ta xác định được \indamm{một và chỉ một giá trị tương ứng $y$} thuộc tập hợp số thực $\mathbb{R}$ thì ta có một hàm số.
	\item [\iconMT] \indam{Lưu ý:}
	\begin{boxkn}
		\begin{itemize}
			\item Ta gọi $x$ là biến số và $y$ là hàm số của $x$.
			\item Tập hợp $\mathscr{D}$ được gọi là tập xác định của hàm số.
			\item Tập hợp $T$ gồm tất cả các giá trị $y$ (tương ứng với $x$ thuộc $\mathscr{D}$) gọi là tập giá trị của hàm số.
		\end{itemize}
	\end{boxkn}
	\item [\iconMT] \indam{Cách cho một hàm số:} Một hàm số có thể được cho bởi một công thức hoặc nhiều công thức; có thể cho bằng mô tả; cho bằng bảng hoặc cho bằng biểu đồ.
\end{itemize}

\subsubsection{Đồ thị của hàm số}
\begin{itemize}
	\item [\iconMT] \indam{Định nghĩa:} Cho hàm số $y=f(x)$ có tập xác định $\mathscr{D}$. Trên mặt phẳng toạ độ $Oxy$, đồ thị $(C)$ của hàm số là tập hợp tất cả các điểm $M(x;y)$ với $x \in \mathscr{D}$ và $y=f(x)$. Vậy $(C)=\{M(x;f(x)) \mid x \in \mathscr{D}\}$.
	\item [\iconMT] \indam{Lưu ý:} Điểm $M\left(x_{M};y_{M}\right)$ thuộc đồ thị hàm số $y=f(x)$ khi và chỉ khi $x_{M} \in \mathscr{D}$ và $y_{M}=f\left(x_{M}\right)$.
\end{itemize}
\subsubsection{Sự đồng biến, nghịch biến của hàm số}
\begin{itemize}
	\item [\iconMT] \indam{Khái niệm:} Với hàm số $y=f(x)$ xác định trên khoảng $(a;b)$, ta nói
	\begin{itemize}
		\item [\iconCH] Hàm số đồng biến trên khoảng $(a;b)$ nếu
		\boxmini{$\forall x_{1}, x_{2} \in(a ; b), x_{1}<x_{2} \Rightarrow f\left(x_{1}\right)<f\left(x_{2}\right)$}
		\item [\iconCH] Hàm số nghịch biến trên khoảng $(a;b)$ nếu
		\boxmini{$\forall x_{1}, x_{2} \in(a ; b), x_{1}<x_{2} \Rightarrow f\left(x_{1}\right)>f\left(x_{2}\right)$}
	\end{itemize}
	\item [\iconMT] \indam{Lưu ý:} Khi vẽ bảng biến thiên, xét từ trái sang phải, ta dùng mũi tên đi xuống để minh họa khoảng nghịch biến và mũi tên đi lên để minh họa khoảng đồng biến. 
\end{itemize}
