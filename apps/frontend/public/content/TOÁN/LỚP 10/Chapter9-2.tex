\section{XÁC SUẤT CỦA BIẾN CỐ}
\subsection{LÝ THUYẾT CẦN NHỚ}
\subsubsection{Định nghĩa cổ điển của xác suất}

\begin{enumerate}[\iconMT]
	\item 	\indam{Công thức tính:} Cho phép thử $T$ có không gian mẫu là $\Omega$. Giả thiết rằng các kết quả có thể của $T$ là đồng khả năng. Khi đó, nếu $E$ là một biến cố liên quan phép thử $T$ thì xác suất của $E$ được cho bởi công thức
	\boxmini{$\mathrm{P}(E) = \dfrac{n(E)}{n(\Omega)}$}
	Trong đó, $n(\omega)$ và $n(E)$ lần lượt là số phần tử của tập không gian mẫu $\omega$ và tập biến cố $E$.
	\item \indam{Nhận xét:}
	\begin{boxkn}
		\begin{itemize}
			
			\item $0\le \mathrm{P}(E)\le 1$, với mọi biến cố $E$.
			\item Với biến cố chắc chắn (là tập $\Omega$), ta có  $\mathrm{P}(\Omega)=1$.
			\item Với biến cố không thể (là tập $\varnothing$), ta có  $\mathrm{P}(\varnothing)=0$.
		\end{itemize}
	\end{boxkn}
\end{enumerate}
\subsubsection{Nguyên lý xác suất bé}

Nếu một biến cố có xác suất rất bé thì trong một phép thử biến cố đó sẽ không xảy ra.

\subsubsection{Ý nghĩa thực tế của xác suất}
Giả sử biến cố $A$ có xác suất là $\mathrm{P}$. Khi thực hiện phép thử $n$ lần ($n\ge 30$) thì số lần xuất hiện của biến cố $A$ sẽ xấp xỉ bằng $n\cdot \mathrm{P(A)}$ (khi $n$ càng lớn thì sai số tương đối càng bé).
