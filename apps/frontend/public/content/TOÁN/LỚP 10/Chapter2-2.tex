\section{HỆ BẤT PHƯƠNG TRÌNH BẬC NHẤT HAI ẨN}

\subsection{TÓM TẮT LÝ THUYẾT}
\subsubsection{Hệ bất phương trình bậc nhất hai ẩn}
\begin{itemize}
	\item [\iconMT] Là hệ bất phương trình gồm hai hay nhiều bất phương trình bậc nhất hai ẩn.
	\item [\iconMT] Cặp số $(x_0;y_0)$ là nghiệm của hệ bất phương trình bậc nhất hai ẩn khi $(x_0;y_0)$ đồng thời là nghiệm của tất cả các bất phương trình trong hệ đó.
\end{itemize}
\subsubsection{Biểu diễn miền nghiệm của hệ bất phương trình bậc nhất hai ẩn}
\begin{itemize}
	\item [\iconMT] Miền nghiệm của hệ là giao các miền nghiệm của các bất phương trình trong hệ.
	\item [\iconMT] Để biểu diễn miền nghiệm của hệ, ta làm như sau:
	\begin{boxdn}
	\begin{itemize}
		\item [$\bullet$] Trên cùng một mặt phẳng tọa độ, xác định miền nghiệm của mỗi bất phương trình bậc nhất hai ẩn trong hệ và gạch bỏ miền còn lại.
		\item [$\bullet$] Miền cuối cùng không bị gạch là miền nghiệm của hệ bất phương trình đã cho.
	\end{itemize}
	\end{boxdn}
\end{itemize}
\begin{vd}
Biểu diễn hình học tập nghiệm của hệ bất phương trình
$\heva{& 3x+y\leq 6 \quad (1)\\& x+y\leq 4 \quad(2)\\& x\geq 0 \quad(3)\\& y\geq 0 \quad(4).} \quad(\star)$ như sau: 
\immini{
\begin{itemize}
	\item [$\bullet$] Vẽ đường thẳng $d_1 \colon 3x+y=6$.\\
	Lấy điểm $M(1;1)$, thay tọa độ $M$ vào bất phương trình (1), ta được $3 \cdot 1 + 1 \le 6$ (thỏa mãn). Suy ra miền nghiệm của bất phương trình (1) là phần không bị gạch như hình vẽ 1, kể cả những điểm nằm trên $d_1$.
	\item [$\bullet$] Tương tự cho bất phương trình (2) (Hình 2) và bất phương trình (3) và (4) (Hình 3).
	\item [$\bullet$] Miền không bị tô đậm (hình tứ giác kể cả bốn cạnh của nó) trong hình vẽ bên là miền nghiệm của hệ bất phương trình đã cho.
\end{itemize}
}{
\begin{tikzpicture}[line join=round, line cap=round, >=stealth,font=\footnotesize, scale=0.6]
	\draw[->](-2,0)--(3,0) node[below right] {$x$};
	\draw[->](0,-1)--(0,5) node[above] {$y$};
	\node (0,0) [below left]{$ O $};
	\foreach \x in {-1,...,2}
	\draw[shift={(\x,0)},color=black] (0pt,2pt) -- (0pt,-2pt);
	\foreach \y in {1,...,5}
	\draw[shift={(0,\y)},color=black] (2pt,0pt) -- (-2pt,0pt);
	\draw[samples=100,smooth,domain=0.333:2.333] plot(\x,{-3*(\x)+6});
	\fill[pattern=north west lines,pattern color=orange] (0.333,5)--(3,5)--(3,-1)--(2.333,-1)--cycle;
	\draw[dashed] (1,0)node[below] {$1$}--(1,1)node[above] {$M$}--(0,1)node[left] {$1$};
	\node[below] at (0,-1.4) {Hình 1};
\end{tikzpicture}
\\
\begin{tikzpicture}[line join=round, line cap=round, >=stealth,font=\footnotesize, scale=0.6]
	\draw[->](-2,0)--(3,0) node[below right] {$x$};
	\draw[->](0,-1)--(0,5) node[above] {$y$};
	\node (0,0) [below left]{$ O $};
	\foreach \x in {-1,...,2}
	\draw[shift={(\x,0)},color=black] (0pt,2pt) -- (0pt,-2pt);
	\foreach \y in {1,...,5}
	\draw[shift={(0,\y)},color=black] (2pt,0pt) -- (-2pt,0pt);
	\draw[samples=100,smooth,domain=0.333:2.333] plot(\x,{-3*(\x)+6});
	\draw[samples=100,smooth,domain=-1:3] plot(\x,{-(\x)+4});
	\fill[pattern=north west lines,pattern color=orange] (0.333,5)--(3,5)--(3,-1)--(2.333,-1)--cycle;
	\fill[pattern=crosshatch dots,pattern color=blue] (-1,5)--(3,5)--(3,1)--cycle;
	\node[below] at (0,-1.4) {Hình 2};
\end{tikzpicture}
\hspace{0.2cm}
\begin{tikzpicture}[line join=round, line cap=round, >=stealth,font=\footnotesize, scale=0.6]
	\draw[->](-2,0)--(3,0) node[below right] {$x$};
	\draw[->](0,-1)--(0,5) node[above] {$y$};
	\node (0,0) [below left]{$ O $};
	\foreach \x in {-1,...,2}
	\draw[shift={(\x,0)},color=black] (0pt,2pt) -- (0pt,-2pt);
	\foreach \y in {1,...,3}
	\draw[shift={(0,\y)},color=black] (2pt,0pt) -- (-2pt,0pt);
	\draw[samples=100,smooth,domain=0.333:2.333] plot(\x,{-3*(\x)+6});
	\draw[samples=100,smooth,domain=-1:3] plot(\x,{-(\x)+4});
	\fill[pattern=north west lines,pattern color=orange] (0.333,5)--(3,5)--(3,-1)--(2.333,-1)--cycle;
	\fill[pattern=crosshatch dots,pattern color=blue] (-1,5)--(3,5)--(3,1)--cycle;
	\fill[pattern=north east lines,pattern color=cyan] (-2,-1)--(-2,5)--(0,5)--(0,-1)--cycle;
	\fill[pattern=crosshatch dots,pattern color=magenta] (-2,-1)--(-2,0)--(3,0)--(3,-1)--cycle;
	\node[below] at (0,-1.4) {Hình 3};
\end{tikzpicture}
}


	
\end{vd}