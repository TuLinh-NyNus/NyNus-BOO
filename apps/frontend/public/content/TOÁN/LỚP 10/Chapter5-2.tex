\section{CÁC SỐ ĐẶC TRƯNG ĐO XU THẾ TRUNG TÂM}
\subsection{TÓM TẮT LÝ THUYẾT}
\subsubsection{Số trung bình và trung vị}
\begin{itemize}
	\item [\iconMT] \indam{Số trung bình: } Kí hiệu là $\overline{x}$.
	\begin{tcolorbox}[colframe=orange,colback=white,boxrule=0.2mm,breakable]
		\begin{itemize}
			\item Với mẫu số liệu kích thước $n$ là $\{x_1; x_2;\ldots; x_{n}\}$ thì 
			\boxmini{$\overline{x}=\dfrac{x_1+x_2+\ldots + x_n}{n}$}
			\item Với mẫu số liệu được cho bởi bảng phân bố tần số thì \boxmini{$\overline{x}=\dfrac{n_1x_1+n_2x_2+\ldots+ n_kx_k}{n}$}
			Trong đó, $n_k$ là tần số của giá trị $x_k$ và $n_1+n_2+\cdots n_k=n$.
			\item Với mẫu số liệu được cho bởi bảng phân bố tần suất (tần số tương đối) thì
			\boxmini{$\overline{x}=f_1x_1+f_2x_2+\cdots + f_kx_k$}
			Trong đó, $f_k=\dfrac{n_k}{n}$ là tần suất (tần số tương đối) của giá trị $x_k$.
		\end{itemize}
	\end{tcolorbox}
	\textit{\indamm{Ý nghĩa của số trung bình:} Số trung bình cộng cho biết vị trí trung tâm của mẫu số liệu. Khi các số liệu trong mẫu ít sai lệch với số trung bình cộng ta có thể lấy số trung bình cộng làm đại diện cho mẫu số liệu.}
	\item [\iconMT] \indam{Số trung vị:} Trong trường hợp mẫu số liệu có giá trị bất thường (rất lớn hoặc rất bé so với đa số các giá trị khác), người ta không dùng số trung bình để đo xu thế trung tâm mà dùng \textbf{số trung vị}, được xác định như sau:
	\begin{tcolorbox}[colframe=orange,colback=white,boxrule=0.2mm]
			Giả sử có một mẫu gồm $n$ số liệu được sắp xếp theo thứ tự không giảm (hoặc không tăng). Khi đó \textbf{số trung vị $\mathrm{M_e}$} là
		\begin{itemize}
			\item Số liệu ở vị trí thứ $\dfrac{n+1}{2}$ nếu $n$ là lẻ.
			\item Trung bình cộng của hai số đứng giữa (số thứ $\dfrac{n}{2}$ và $\dfrac{n}{2}+1$) nếu $n$ là chẵn.
		\end{itemize}
	\end{tcolorbox}
\indamm{Ý nghĩa của số trung vị:}
\begin{itemize}
	\item \textit{Trung vị là giá trị chia đôi trong mẫu số liệu. Trung vị không bị ảnh hưởng bởi giá trị bất thường trong khi đó số trung bình cộng bị ảnh hưởng bởi giá trị bất thường.}
	\item \textit{Nếu những số liệu trong mẫu có sự chênh lệch lớn thì ta nên chọn thêm trung vị làm đại diện cho mẫu số liệu đó nhằm điều chỉnh một số hạn chế khi sử dụng số trung bình cộng. Những kết luận về đối tượng thống kê rút ra khi đó sẽ tin cậy hơn.}
\end{itemize}
\end{itemize}
\subsubsection{Tứ phân vị}
Trung vị chia mẫu ra làm hai phần. Trong thực tế người ta cũng quan tâm đến trung vị của mỗi phần đó. Ba trung vị này được gọi là  \indamm{tứ phân vị} của mẫu. Để tìm tứ phân vị của mẫu số liệu có $n$ giá trị, ta làm như sau:
\begin{tcolorbox}[colframe=orange,colback=white,boxrule=0.2mm,breakable]
	\begin{itemize}
		\item  Sắp thứ tự mẫu số liệu gồm $n$ số liệu thành một dãy không giảm.
		\item Tìm trung vị. Giá trị này là $Q_2$.
		\item Tìm trung vị của nửa số liệu bên trái $Q_2$ (không bao gồm $Q_2$ nếu $n$ lẻ). Giá trị này là $Q_1$.
		\item Tìm trung vị của nửa số liệu bên phải $Q_2$ (không bao gồm $Q_2$ nếu $n$ lẻ). Giá trị này là $Q_3$.
		\begin{center}
			\begin{tikzpicture}[scale=0.9, font=\footnotesize, line join=round, line cap=round, >=stealth]
				\path (0,0) coordinate(A) ++(0:2) coordinate(Q_1) ++(0:3.5) coordinate(Q_2) ++(0:2.5) coordinate(Q_3) ++(0:1.5) coordinate(B);
				\draw (A)node[text width=1.5cm,align=center,below=5pt] {Giá trị\\nhỏ nhất} -- (Q_1) node[below=5pt]{$Q_1$} -- (Q_2) node[below=5pt]{$Q_2$} -- (Q_3) node[below=5pt]{$Q_3$} -- (B)node[below=5pt, text width=1.5 cm, align=center]{Giá trị\\lớn nhất};
				\foreach \i in {A,Q_1,Q_2,Q_3,B} \draw (\i)++(-90:0.07) -- ++(90:0.14);
				\draw[decoration={brace}, decorate] ($(A)+(0,0.2)$)--($(Q_2)+(0,0.2)$) node[sloped,midway,above]{Nửa dãy phía dưới};
				\draw[decoration={brace}, decorate] ($(Q_2)+(0,0.2)$)--($(B)+(0,0.2)$) node[sloped,midway,above]{Nửa dãy phía trên};	
			\end{tikzpicture}
		\end{center}
		\item [] $Q_1$, $Q_2$, $Q_3$ được gọi là các \indamm{tứ phân vị} của mẫu số liệu.
	\end{itemize}
\end{tcolorbox}


\subsubsection{Mốt}
\begin{itemize}
	\item [] Mốt của mẫu số liệu là giá trị xuất hiện với tần số lớn nhất và được kí hiệu là $M_O$.
	\item [] \indamm{Lưu ý:}
	\begin{boxdn}
	\begin{itemize}
		\item [$\bullet$] Một mẫu số liệu có thể có nhiều mốt.
		\item [$\bullet$] Khi tất cả các giá trị trong mấu số liệu có tần số xuất hiện bằng nhau thì mẫu số liệu đó không có mốt.
	\end{itemize}
	\end{boxdn}
\end{itemize}