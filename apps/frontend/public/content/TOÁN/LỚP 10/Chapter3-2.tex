\section{HỆ THỨC LƯỢNG TRONG TAM GIÁC}

\subsection{TÓM TẮT LÝ THUYẾT}
\subsubsection{Định lý cô-sin}
	Cho tam giác $ ABC$ có $ BC=a$, $ AC=b$ và $ AB=c$.
	\begin{gachsoc}	\immini{Ta có
			\begin{listEX}
				\item[$\bullet$] $ a^2=b^2+c^2-2bc\cdot \cos A$. 
				\item[$\bullet$] $ b^2=c^2+a^2-2ca\cdot \cos B$. 
				\item[$\bullet$]  $ c^2=a^2+b^2-2ab\cdot \cos C$.
			\end{listEX}
	}
	{\begin{tikzpicture}[scale=0.7,font=\footnotesize,line join=round, line cap=round,>=stealth]
		\tkzDefPoints{0/0/B,1/2/A,4/0/C}
		\tkzDrawPoints[fill=black](A,B,C)
		\tkzDefMidPoint(A,B) \tkzGetPoint{c}
		\tkzDefMidPoint(C,B) \tkzGetPoint{a}
		\tkzDefMidPoint(A,C) \tkzGetPoint{b}
		\tkzDrawPolygon(A,B,C)
		\tkzLabelPoints[above](A) 
		\tkzLabelPoints[below](B,C,a)
		\tkzLabelPoints[left](c)
		\tkzLabelPoints[right](b)
		\end{tikzpicture}}
		\end{gachsoc}
	Áp dụng để tính góc
	\begin{gachsoc}
		\begin{listEX}[3]
			\item[\ding{172}]$ \cos A=\dfrac{b^2+c^2-a^2}{2bc}$.
			\item [\ding{173}]$ \cos B=\dfrac{c^2+a^2-b^2}{2ca}$.
			\item [\ding{174}]$ \cos C=\dfrac{a^2+b^2-c^2}{2ab}$.
		\end{listEX}
	\end{gachsoc}
\subsubsection{Định lý sin}
	\immini{
		Cho tam giác $ ABC$ có $ BC=a,AC=b$, $ AB=c$ và 
		$ R$ là bán kính đường tròn ngoại tiếp. 
		Ta có 
		\begin{gachsoc}
			$$ \dfrac{a}{\sin A}=\dfrac{b}{\sin B}=\dfrac{c}{\sin C}=2R$$
		\end{gachsoc}
	\begin{luuy}
		Ghi nhớ: Tỉ lệ "cạnh chia sin góc đối" thì bằng nhau.
	\end{luuy}
	}
	{
		\begin{tikzpicture}[scale=0.6,font=\footnotesize,line join=round, line cap=round,>=stealth]
		\tkzDefPoints{0/0/B,1/3/A,4/0/C}
		\tkzCircumCenter(A,B,C)  \tkzGetPoint{I}
		\tkzDrawPoints[fill=black](A,B,C,I)
		\tkzDrawCircle[circum](A,B,C)
		\tkzDefMidPoint(A,B) \tkzGetPoint{c}
		\tkzDefMidPoint(C,B) \tkzGetPoint{a}
		\tkzDefMidPoint(A,C) \tkzGetPoint{b}
		\tkzDefMidPoint(a,c) \tkzGetPoint{R}
		\tkzDrawPolygon(A,B,C)
		\tkzDrawSegments(I,A I,B I,C)
		\tkzLabelPoints[above](A) 
		\tkzLabelPoints[below](B,C,a,I,R)
		\tkzLabelPoints[left](c)
		\tkzLabelPoints[right](b)
		\end{tikzpicture}} 

\subsubsection{Công thức tính diện tích tam giác}
Gọi $S$ là diện tích tam giác $ABC$. Ta có
\begin{gachsoc}
	\begin{listEX}[2]
		\item [\ding{172}] $S=\dfrac{1}{2}a\cdot h_a=\dfrac{1}{2}b\cdot h_b=\dfrac{1}{2}c\cdot h_c.$
		\item [\ding{173}] $S=\dfrac{1}{2}bc\sin A=\dfrac{1}{2}ca\sin B=\dfrac{1}{2}ab\sin C$
		\item [\ding{174}] $S=\dfrac{abc}{4R}.$
		\item [\ding{175}] $S=p\cdot r$
		\item [\ding{176}] $S=\sqrt{p(p-a)(p-b)(p-c)}$.
	\end{listEX}
\end{gachsoc}
Trong đó:
\begin{listEX}
	\item [$\bullet$] $ h_a$, $h_b$, $h_c$ là độ dài đường cao lần lượt tương ứng với các cạnh $ BC$, $CA$, $AB$.
	\item [$\bullet$]  $ R$ là bán kính đường tròn ngoại tiếp tam giác.
	\item [$\bullet$]  $ r$ là bán kính đường tròn nội tiếp tam giác.
	\item [$\bullet$]  $ p=\dfrac{a+b+c}{2}$ là nửa chu vi tam giác.
\end{listEX} 

\begin{itemize}
	\item [] \indam{Công thức độ dài đường trung tuyến:} Cho tam giác $ ABC$ có $ m_a$, $ m_b$, $ m_c$ lần lượt là các trung tuyến kẻ từ $ A$, $ B$, $ C$. 
	Ta có
	\immini{
		\begin{listEX}
			\item [$\bullet$] $ m_a^2=\dfrac{b^2+c^2}{2}-\dfrac{a^2}{4}$.
			\item [$\bullet$]  $ m_b^2=\dfrac{a^2+c^2}{2}-\dfrac{b^2}{4}$.
			\item [$\bullet$]  $ m_{c}^2=\dfrac{a^2+b^2}{2}-\dfrac{c^2}{4}$.
		\end{listEX}
	}
	{
		\begin{tikzpicture}[scale=0.7,font=\footnotesize,line join=round, line cap=round,>=stealth]
			\tkzDefPoints{0/0/B,1/3/A,6/0/C}
			\tkzDefMidPoint(A,B) \tkzGetPoint{c}
			\tkzDefMidPoint(C,B) \tkzGetPoint{a}
			\tkzDefMidPoint(A,C) \tkzGetPoint{b}
			\coordinate (m_a) at ($ (A)!0.3!(a)$ );
			\coordinate (m_b) at ($ (B)!0.3!(b)$ );
			\coordinate (m_c) at ($ (C)!0.3!(c)$ );
			\tkzDrawPoints[fill=black](A,B,C,a,b,c)
			\tkzDrawPolygon(A,B,C)
			\tkzDrawSegments(a,A b,B c,C)
			\tkzLabelPoints[above](A) 
			\tkzLabelPoints[below](B,C,a,m_b,m_c)
			\tkzLabelPoints[left](c,m_a)
			\tkzLabelPoints[above right](b)
	\end{tikzpicture}}
\end{itemize}
