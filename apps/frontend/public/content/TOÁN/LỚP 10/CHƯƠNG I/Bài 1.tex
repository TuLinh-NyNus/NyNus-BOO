\section{MỆNH ĐỀ}

\subsection{TÓM TẮT LÝ THUYẾT}
\subsubsection{Mệnh đề, mệnh đề chứa biến}
\begin{itemize}
	\item [\iconMT] \indam{Mệnh đề:} Mệnh đề là một \indamm{ câu khẳng định} đúng hoặc sai. 
	\begin{boxkn}
		\begin{itemize}
			\item Câu khẳng định đúng gọi là mệnh đề đúng.
			\item Câu khẳng định sai gọi là mệnh đề \textbf{sai}.
			\item Một mệnh đề không thể vừa đúng hoặc vừa \textbf{sai}.
			\item Những mệnh đề liên quan đến toán học được gọi là mệnh đề toán học.
		\end{itemize} 
	\end{boxkn}
	\item [\iconMT] \indam{Mệnh đề chứa biến:} Mệnh đề chứa biến là một câu khẳng định chứa biến nhận giá trị trong một tập $X$ nào đó và với mỗi giá trị của biến thuộc $X$ ta được một mệnh đề.
	
\end{itemize}
\subsubsection{Mệnh đề phủ định}
\begin{itemize}
	\item [\iconMT] Cho mệnh đề $P$. Mệnh đề \lq\lq không phải $P$ \rq\rq gọi là mệnh đề phủ định của $P$.
	\item [\iconMT] Chú ý: 
		\begin{boxkn}
	\begin{itemize}
		\item  Mệnh đề phủ định của $P$, kí hiệu là $\overline{P}$.
		\item  Nếu $P$ đúng thì $\overline{P}$ sai, nếu $P$ sai thì $\overline{P}$ đúng.
	\end{itemize}
	\end{boxkn}
\end{itemize}
\subsubsection{Mệnh đề kéo theo và mệnh đề đảo}
	Cho hai mệnh đề $P$ và $Q$.
		\begin{itemize}
			\item[\iconMT] \indam{Mệnh đề kéo theo:} Mệnh đề "Nếu $P$ thì $Q$" gọi là mệnh đề kéo theo, kí hiệu  $P\Rightarrow Q$.
				\begin{boxkn}
			\begin{itemize}
				\item  Mệnh đề $P\Rightarrow Q$ còn được phát biểu là "$P$ kéo theo $Q$" hoặc "Từ $P$ suy ra $Q$"
				\item  Mệnh đề này chỉ sai khi $P$ đúng và $Q$ sai.
			\end{itemize}
				\end{boxkn}
				\begin{khung4}{Lưu ý}
					Xét định lý dạng $P\Rightarrow Q$. Khi đó, ta có thể phát biểu định lý này theo một trong 2 cách sau:
					\begin{itemize}
						\item[\ding{172}] $P$ là điều kiện đủ để có $Q$.
						\item[\ding{173}] $Q$ là điều kiện cần để có $P$.
					\end{itemize}
				\end{khung4}
			\item[\iconMT] \indam{Mệnh đề đảo:} Cho mệnh đề $P\Rightarrow Q$. Khi đó, $Q\Rightarrow P$ gọi là mệnh đề đảo của $P\Rightarrow Q$.
			\end{itemize}
\subsubsection{Mệnh đề tương đương}
\begin{itemize}
	\item [\iconMT] Cho hai mệnh đề $P$ và $Q$. Mệnh đề ``$P$ nếu và chỉ nếu $Q$''  gọi là hai mệnh đề tương đương.
	\item [\iconMT] Chú ý:
	\begin{boxkn}
	\begin{itemize}
		\item  Mệnh đề ``$P$ nếu và chỉ nếu $Q$'' được kí hiệu là $P\Leftrightarrow Q$.
		\item 	Mệnh đề $P\Leftrightarrow Q$ đúng khi cả $P\Rightarrow Q$ và $Q\Rightarrow P$ cùng đúng.
	\end{itemize}
	\end{boxkn}
\end{itemize}
			\begin{khung4}{Lưu ý}
				Xét định lý dạng $P \Leftrightarrow Q$. Khi đó, ta có thể phát biểu định lý này theo một trong 2 cách sau:
				\begin{itemize}
					\item[\ding{172}] $P$ là điều cần và đủ để có $Q$.
					\item[\ding{173}] $P$ khi và chỉ khi $Q$.
				\end{itemize}
			\end{khung4}
\subsubsection{Mệnh đề có chứa kí hiệu $\forall, \exists$}
	\begin{itemize}
		\item[\iconMT] Mệnh đề chứa kí hiệu với mọi: $\forall x \in X,\, P(x) $.
	\begin{boxkn}
		\begin{itemize}
			\item Mệnh đề này đúng khi tất cả các giá trị của $x \in X$ đều làm cho phát biểu $P(x)$ đúng.
			\item Nếu ta tìm được ít nhất một giá trị $x \in X$ làm cho $P(x)$ sai thì mệnh đề này \textbf{sai}.
		\end{itemize}
	\end{boxkn}
		\begin{vd}
			Mệnh đề "Bình phương mọi số thực đều không âm" được viết là
			$\forall x \in \mathbb{R},\, x^2 \ge 0.$
		\end{vd}
		\item[\iconMT] Mệnh đề chứa kí hiệu tồn tại: $\exists x \in X,\, P(x) $.
	\begin{boxkn}
		\begin{itemize}
			\item Mệnh đề này đúng khi ta tìm được ít nhất một giá trị của $x \in X$ làm cho phát biểu $P(x)$ đúng.
			\item Nếu tất cả giá trị của $x \in X$ đều làm cho $P(x)$ sai thì mệnh đề này \textbf{sai}.
		\end{itemize}
	\end{boxkn}
		\begin{vd}
			Mệnh đề "Có một số tự nhiên mà bình phương của nó bằng 3" được viết là
			$\exists x \in \mathbb{N},\, x^2=3.$
		\end{vd}
		\item[\iconMT] Phủ định của Mệnh đề chứa kí hiệu $\forall $, $\exists $.
	\begin{boxkn}
		\begin{itemize}
			\item Phủ định của mệnh đề ``$\forall x\in X,P\left( x \right)"$ là mệnh đề ``$\exists x\in X,\overline{P(x)}"$.
			\item Phủ định của mệnh đề ``$\exists x\in X,P\left( x \right)"$ là mệnh đề ``$\forall x\in X,\overline{P(x)}"$.
		\end{itemize}
	\end{boxkn}
	\end{itemize}