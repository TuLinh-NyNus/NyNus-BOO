\section{BẤT PHƯƠNG TRÌNH BẬC NHẤT HAI ẨN}

\subsection{TÓM TẮT LÝ THUYẾT}
\subsubsection{Bất phương trình bậc nhất hai ẩn}
\begin{itemize}
	\item [\iconMT] Bất phương trình bậc nhất hai ẩn $x$, $y$ có dạng tổng quát là 
		\boxmini{$ax+by\le c \quad (1)$}
	trong đó $a$, $b$, $c$ là những số thực đã cho, $a$ và $b$ không đồng thời bằng $0$, $x$ và $y$ là các ẩn số.
	\item [\iconMT] Nghiệm của bất phương trình là những cặp số $(x_0;y_0)$ thỏa mãn (1).
	\item [\iconMT] Các dạng khác $ax+by<c$; $ax+by\ge c$; $ax+by>c$,...
\end{itemize}

\subsubsection{Biểu diễn tập nghiệm của bất phương trình bậc nhất hai ẩn}
\begin{itemize}
	\item [\iconMT] Các bất phương trình bậc nhất hai ẩn thường có vô số nghiệm. Để mô tả tập nghiệm của chúng, ta sử dụng phương pháp biểu diễn hình học.
	\item [\iconMT] Giả sử muốn biểu diễn miền nghiệm của bất phương trình $ax+by \le c \quad(2)$, ta thực hiện các bước như sau:
	\begin{boxdn}
		\begin{itemize}
			\item [\ding{172}] Trên mặt phẳng tọa độ $Oxy$, vẽ đường thẳng $\Delta \colon ax+by=c$.
			\item [\ding{173}] Lấy một điểm $M_0\left(x_0;y_0\right)$ không thuộc $\Delta$.
			\item [\ding{174}] Thay $(x_0;y_0)$ vào (2), sẽ có một trong hai trường hợp xảy ra: 
			\begin{itemize}
				\item Nếu mệnh đề đúng thì miền nghiệm phải chứa $M_0$. Suy ra nửa mặt phẳng bờ $\Delta$ chứa $M_0$ là miền nghiệm của $(2)$.
				\item Nếu mệnh đề sai thì miền nghiệm không chứa $M_0$. Suy ra nửa mặt phẳng bờ $\Delta$ không chứa $M_0$ là miền nghiệm của $(2)$.
			\end{itemize}
		\end{itemize}
	\end{boxdn}
\end{itemize}
\begin{vd}
	Biểu diễn hình học tập nghiệm của bất phương trình $2x+y\le 3 \quad(\star)$, ta làm như sau:\\
	\immini{
		\begin{itemize}
			\item [$\bullet$] Vẽ đường thẳng $\Delta\colon 2x+y=3$.
			\item [$\bullet$] Lấy gốc tọa độ $O(0;0)$, ta thấy $O\notin \Delta $. 
			\item [$\bullet$] Thay tọa độ $O$ vào $(\star)$: $2 \cdot 0+0<3$ (thỏa). Suy ra nửa mặt phẳng bờ $\Delta$ chứa gốc tọa độ $O$ là miền nghiệm của bất phương trình đã cho (miền không bị tô đậm trong hình vẽ)
		\end{itemize}
	}
	{
		\begin{tikzpicture}[line join=round, line cap=round, >=stealth,font=\footnotesize, scale=0.8]
			\fill [pattern=north east lines,pattern color=gray] (-0.5,4)--(3,4)--(3,-1) -- (2,-1)--cycle;
			\draw[samples=100,smooth,domain=-0.5:2,red] plot(\x,{-2*(\x)+3})node[left] {$\Delta$};
			\draw[->](-2,0)--(3.2,0) node[below] {$x$};
			\draw[->](0,-1)--(0,4.2) node[right] {$y$};
			\node (0,0) [below left]{$ O $};
			\foreach \x in {-1,...,2}
			\draw[shift={(\x,0)},color=black] (0pt,2pt) -- (0pt,-2pt);
			\foreach \y in {1,...,3}
			\draw[shift={(0,\y)},color=black] (2pt,0pt) -- (-2pt,0pt);
			\draw[fill=black] (0,3) circle(1pt) node[left]{$3$};
			\draw[fill=black] (1.5,0)circle(1pt) node[below left]{\tiny $\dfrac{3}{2}$};
		\end{tikzpicture}
	}
\end{vd}
\begin{vd}
	Biểu diễn hình học tập nghiệm của bất phương trình $x-2y> 0 \quad(\star)$, ta làm như sau:\\
	\immini{
		\begin{itemize}
			\item [$\bullet$] Vẽ đường thẳng $\Delta\colon x-2y=0$.
			\item [$\bullet$] Lấy điểm $M(1;0)$, ta thấy $M\notin \Delta $. 
			\item [$\bullet$] Thay tọa độ $M$ vào $(\star)$: $1 \cdot 1-2\cdot0>0$ (thỏa). Suy ra miền nghiệm của bất phương trình là nửa mặt phẳng không bị gạch trong hình vẽ bên và không kể đường thẳng $\Delta$.
		\end{itemize}
	}
	{
		\begin{tikzpicture}[line join=round, line cap=round, >=stealth,font=\footnotesize, scale=0.8]
			\fill [pattern=north west lines,pattern color=gray] (-3,-1.5)--(-3,3)-- (4,3)-- (4,2)--cycle;
			\draw[samples=100,smooth,domain=-3:4,red] plot(\x,{0.5*(\x)})node[below] {$\Delta$};
			\draw[->](-3,0)--(4.5,0) node[below] {$x$};
			\draw[->](0,-2)--(0,3.3) node[right] {$y$};
			\node (0,0) [below right]{$ O $};
			\foreach \x in {-2,...,3}
			\draw[shift={(\x,0)},color=black] (0pt,2pt) -- (0pt,-2pt);
			\foreach \y in {-1,...,2}
			\draw[shift={(0,\y)},color=black] (2pt,0pt) -- (-2pt,0pt);
			\draw[fill=black] (1,0)circle(1pt) node[below]{$M$};
		\end{tikzpicture}
	}
\end{vd}
\begin{note}
	Miền nghiệm của bất phương trình $ax_0+by_0\leq c$ bỏ đi đường thẳng $ax+by=c$ là miền nghiệm của bất phương trình $ax_0+by_0<c$.
\end{note}
