\section{TỔNG VÀ HIỆU CỦA HAI VECTƠ}

\subsection{TÓM TẮT LÝ THUYẾT}
\subsubsection{Vectơ bằng nhau, vectơ đối nhau}
\begin{tcolorbox}[colframe=orange,colback=white,boxrule=0.2mm]
	\begin{itemize}
		\item  Hai vec tơ bằng nhau nếu chúng có cùng \textbf{độ lớn} và \textbf{cùng hướng}.
		\item  Hai vec tơ đối nhau nếu chúng có cùng \textbf{độ lớn} nhưng \textbf{ngược hướng}.
	\end{itemize}
\end{tcolorbox}
\vspace{-0.4cm}
	\immini{\indamm{Ví dụ:} Cho hình bình hành $ABCD$ tâm $O$, ta có vài kết quả sau
	\begin{itemize}
		\item [$\bullet$] Các vectơ bằng nhau $\overrightarrow{AB}=\overrightarrow{DC};\text{}\overrightarrow{AD}=\overrightarrow{BC};$ 
		$\overrightarrow{AO}=\overrightarrow{OC}$; $\overrightarrow{DO}=\overrightarrow{OB}$,...
		\item [$\bullet$] Các vectơ đối nhau: $\overrightarrow{AB}$ đối $\overrightarrow{CD}$; $\overrightarrow{BC}$ đối $\overrightarrow{DA}$; $\overrightarrow{OA}$ đối $\overrightarrow{OC}$; $\overrightarrow{OB}$ đối $\overrightarrow{OD}$;...
	\end{itemize}
}{\hspace{1cm}
\begin{tikzpicture}[scale=1, line join=round, line cap=round]
	\tkzDefPoints{0/0/A,4/0/B,5/2/C}
	\tkzDefPointBy[translation=from B to C](A)\tkzGetPoint{D}
	\tkzInterLL(A,C)(B,D)\tkzGetPoint{O}
	\tkzDrawSegments(A,B B,C C,D D,A A,C B,D)
	\tkzDrawPoints[size=2,fill=black](A,B,C,D,O)
	\tkzLabelPoints[below,font=\footnotesize](A, B, O)
	\tkzLabelPoints[above,font=\footnotesize](C,D)
	\node[below] at (2,-0.5) {Hình 1};
\end{tikzpicture}}
\subsubsection{Tổng của hai vectơ}
Phép cộng hai vectơ có tính chất giao hoán. Khi thực hiện phép toán cộng hai vec tơ, ta chú ý các quy tắc sau

\begin{itemize}
	\item [\iconMT] \indam{Quy tắc 3 điểm:}
	\begin{boxkn}
	\immini{Với ba điểm $A,B,C$ bất kì, ta luôn có
		\fbox{$\overrightarrow{AB}+\overrightarrow{BC}=\overrightarrow{AC}$}
		\begin{itemize}
			\item [$\bullet$] Dấu hiệu nhận biết là "điểm liên tiếp nhau".
			\item [$\bullet$] Các hệ thức tương tự
			\begin{listEX}[2]
				\item [] $\overrightarrow{BA}+\overrightarrow{AC}=\overrightarrow{BC}$
				\item [] $\overrightarrow{CB}+\overrightarrow{BA}=\overrightarrow{CA}$
			\end{listEX}
		\end{itemize}
		
	}{\hspace{1cm}
	\begin{tikzpicture}[scale=0.6, line join=round, line cap=round]
	\tkzDefPoints{0/0/A,5/0/C,2/2/B}
	\tkzDrawSegments(A,B B,C C,A)
	\tkzDrawPoints[size=2,fill=black](A,B,C)
	\tkzLabelPoints[above](B)
	\tkzLabelPoints[below](A,C)
	\end{tikzpicture}}
\end{boxkn}
	\item [\iconMT] \indam{Quy tắc hình bình hành:}
	\begin{boxkn}
	\immini{Xét hình bình hành $ABCD$, ta luôn có
		\fbox{$\overrightarrow{AB}+\overrightarrow{AD}=\overrightarrow{AC}$}
		\begin{itemize}
			\item [$\bullet$] Dấu hiệu nhận biết là "cùng gốc".
			\item [$\bullet$] Các hệ thức tương tự
			\begin{listEX}[2]
				\item [] $\overrightarrow{BA}+\overrightarrow{BC}=\overrightarrow{BD}$
				\item [] $\overrightarrow{CB}+\overrightarrow{CD}=\overrightarrow{CA}$
			\end{listEX}
		\end{itemize}
		
	}{\hspace{1cm}
		\begin{tikzpicture}[scale=0.6, line join=round, line cap=round]
		\tkzDefPoints{0/0/A,1/2/B,5/2/C,4/0/D}
		\tkzDrawSegments(A,B B,C C,D D,A)
		\tkzDrawPoints[size=2,fill=black](A,B,C,D)
		\tkzLabelPoints[below](A,D)
		\tkzLabelPoints[above](B,C)
		\end{tikzpicture}}
	\end{boxkn}
	\item [\iconMT] \indam{Quy tắc cộng vectơ đối:}
	\begin{boxkn}
	\begin{itemize}
		\item [$\bullet$] Nếu $\overrightarrow{a}$ và $\overrightarrow{b}$ đối nhau thì $\overrightarrow{a}+\overrightarrow{b}=\overrightarrow{0}$.
		\item [$\bullet$] Trong Hình 1 ở trên, ta có
		\begin{listEX}[3]
			\item [] $\overrightarrow{AD}+\overrightarrow{CB}=\overrightarrow{0}$
			\item [] $\overrightarrow{AB}+\overrightarrow{CD}=\overrightarrow{0}$
			\item [] $\overrightarrow{OA}+\overrightarrow{OC}=\overrightarrow{0}$
		\end{listEX}
	\end{itemize} 
\end{boxkn}
\indamm{Tính chất:} Với ba vectơ $\vec{a}$, $\vec{b}$, $\vec{c}$ tùy ý
\begin{tcolorbox}[colframe=orange,colback=white,boxrule=0.2mm]
	\begin{itemize}
		\item Tính chất giáo hoán: $\vec{a}+\vec{b}=\vec{b}+\vec{a}$.
		\item Tính chất kết hợp: $(\vec{a}+\vec{b})+\vec{c}=\vec{a}+(\vec{b}+\vec{c})$.
		\item Tính chất của vectơ-không: $\vec{a}+\vec{0}=\vec{a}$.
	\end{itemize}
\end{tcolorbox}
\end{itemize}
\subsubsection{Hiệu của hai vectơ}
	\begin{itemize}
		\item [\iconMT] \indam{Vectơ đối:} 
		\begin{itemize}
			\item [$\bullet$] Vectơ đối của $\vec{a}$ kí hiệu là $-\vec{a}$.
			\item [$\bullet$] Vectơ đối của $\overrightarrow{AB}$ là $\overrightarrow{BA}$, nghĩa là \fbox{$-\overrightarrow{AB}=\overrightarrow{BA}$} (\textit{dùng để làm mất dấu trừ trước vectơ}).
			\item [$\bullet$] Vectơ $\vec{0}$ được coi là vectơ đối của chính nó.
		\end{itemize} 
		\item [\iconMT] \indam{ Quy tắc trừ:} Với ba điểm $A,B,C$ bất kì, ta luôn có
		\fbox{$\overrightarrow{BC}=\overrightarrow{AC}-\overrightarrow{AB}$}
	\end{itemize}
\subsubsection{Công thức trung điểm, trọng tâm}
\begin{itemize}
	\item [\iconMT] \indam{Công thức trung điểm:}
\immini{\begin{itemize}
		\item [$\bullet$] Nếu $M$ là trung điểm của đoạn $AB$ thì $\overrightarrow{MA}+\overrightarrow{MB}=\overrightarrow{0}$.
		\item [$\bullet$] Tương tự $\overrightarrow{AM}+\overrightarrow{BM}=\overrightarrow{0}$.
	\end{itemize} }{
	\begin{tikzpicture}[scale=1, line join=round, line cap=round]
		\tikzset{label style/.style={font=\footnotesize}}
		\tkzDefPoints{0/0/A,4/0/B}
		\coordinate (M) at ($(A)!0.5!(B)$);
		\tkzDrawPoints[size=2,fill=black](A,B,M)
		\tkzDrawSegments(A,B)
		\tkzLabelPoints[above](A,B,M)	
\end{tikzpicture}}
	\item [\iconMT] \indam{Công thức trọng tâm:}
\immini{\begin{itemize}
		\item [$\bullet$] Nếu $G$ là trọng tâm của tam giác $ABC$ thì $$\overrightarrow{GA}+\overrightarrow{GB}+\overrightarrow{GC}=\overrightarrow{0}.$$
		\item [$\bullet$] Tương tự $\overrightarrow{AG}+\overrightarrow{BG}+\overrightarrow{CG}=\overrightarrow{0}$.
	\end{itemize} }{
		\begin{tikzpicture}[scale=1, line join=round, line cap=round]
		\tikzset{label style/.style={font=\footnotesize}}
		\tkzDefPoints{0/0/B,4/0/C,1/2/A}
		\coordinate (I) at ($(C)!0.5!(B)$);
		\coordinate (K) at ($(A)!0.5!(B)$);
		\coordinate (M) at ($(C)!0.5!(A)$);
		\tkzInterLL(A,I)(C,K)\tkzGetPoint{G}
		\tkzDrawSegments(A,B B,C C,A A,I C,K B,M)
		\tkzDrawPoints[size=2,fill=black](A,B,C,I,K,M,G)
		\tkzLabelPoints[below](C, B,I)
		\tkzLabelPoints[above](A)
		\tkzLabelPoints[above right](M)
		\tkzLabelPoints[above left](K)
		\tkzLabelPoints[above right](G)
\end{tikzpicture}}

[$\bullet$] Các công thức tính toán cần nhớ:
\begin{itemize}
	\item [*] Đường cao trong tam giác đều bằng \fbox{$\text{cạnh} \cdot \dfrac{\sqrt{3}}{2}$}.
	\item [*] Đường chéo của hình vuông bằng \fbox{$\text{cạnh} \cdot \sqrt{2}$}.
\end{itemize}
\end{itemize}