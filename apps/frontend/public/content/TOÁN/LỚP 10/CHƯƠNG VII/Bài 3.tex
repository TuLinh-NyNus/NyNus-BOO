\section{BA ĐƯỜNG CONIC}
\subsubsection{Elip}
	\begin{itemize}
		\item [\iconMT] \indam{Định nghĩa:} Cho hai điểm phân biệt $F_1$ và $F_2$. Đặt $F_1F_2=2c>0$. Tập hợp tất cả điểm $M$ thỏa $MF_1+MF_2=2a$, với $a>c>0$ là một elip.
		\item [\iconMT] \indam{Phương trình elip:} Trong mặt phẳng $Oxy$, cho hai điểm $F_1(-c;0)$,  $F_2(c;0)$ (hai tiêu điểm thuộc trục hoành). Với $b^2=a^2-c^2$, ta có phương trình elip có dạng \boxmini{$\dfrac{x^2}{a^2}+\dfrac{y^2}{b^2}=1$, với $a>b>0$.}
		\vspace{-0.8cm}
	\immini{
		\item [\iconMT] \indam{Hình dạng elip và các đại lượng liên quan:}\\
		\begin{tcolorbox}[colframe=orange,colback=orange!3,boxrule=0.2mm]
			\begin{itemize}
				\item Trục lớn $A_1A_2=2a$; Trục bé $B_1B_2=2b$;
				\item Tiêu cự $F_1F_2=2c$ và $a^2=b^2+c^2$.
				\item Tọa độ các đỉnh $A_1(-a;0)$, $A_2(a;0)$, $B_1(0;-b)$, $B_2(0;b)$.
				\item Tiêu điểm $F_1(-c;0)$,  $F_2(c;0)$.
			\end{itemize}
		\end{tcolorbox}
		}{\vspace{0.8cm}
			\begin{tikzpicture}[smooth,samples=300,scale=0.9,>=stealth,font=\footnotesize]
			\draw[->] (-3,0)--(3.2,0) node[below]{$x$};
			\draw[->] (0,-2)--(0,2.5) node[right]{$y$};
			\draw (0,0) node[below left]{$O$};
			\tkzDefPoints{0/0/O,-2.5/0/A_1,2.5/0/A_2,0/-1.5/B_1,0/1.5/B_2,-2/0/F_1,2/0/F_2}
			\draw[color=blue,thick] (O) ellipse (2.5cm and 1.5cm);
			\coordinate (M) at ($(O)+(60:2.5cm and 1.5cm)$);
			\tkzDrawPoints[size=3,fill=black](A_1,A_2,B_1,B_2,F_1,F_2,M)
			\tkzDrawSegments[dashed](M,F_1 M,F_2)
			\tkzLabelPoints[below right,font=\footnotesize](B_1)
			\tkzLabelPoints[above right,font=\footnotesize](B_2,A_2,M)
			\tkzLabelPoints[below ,font=\footnotesize](F_1,F_2)
			\tkzLabelPoints[above left,font=\footnotesize](A_1)
			\end{tikzpicture}}
	\end{itemize}
\subsubsection{Hypebol}
\begin{itemize}
	\item [\iconMT] \indam{Định nghĩa:} Cho hai điểm phân biệt $F_1$ và $F_2$. Đặt $F_1F_2=2c>0$.Tập hợp tất cả điểm $M$ thỏa $\big|MF_1-MF_2\big|=2a$, với $0<a<c$ là một hypebol.
	\item [\iconMT] \indam{Phương trình hypebol:} Trong mặt phẳng $Oxy$, cho hai điểm $F_1(-c;0)$,  $F_2(c;0)$ (hai tiêu điểm thuộc trục hoành). Với $b^2=c^2-a^2$, ta có phương trình hyperbol có dạng \boxmini{$\dfrac{x^2}{a^2}-\dfrac{y^2}{b^2}=1$, với $a,b>0$.}
	\vspace{-0.8cm}
	\immini{
		\item [\iconMT] \indam{Hình dạng hypebol và các đại lượng liên quan:}\\
		\begin{tcolorbox}[colframe=orange,colback=orange!3,boxrule=0.2mm]
			\begin{itemize}
				\item Đoạn thẳng $A_1A_2=2a$ gọi là \textbf{trục thực}, đoạn thẳng $B_1B_2=2b$ gọi là \textbf{trục ảo} của hypebol.
				\item Tiêu cự $F_1F_2=2c$ và $a^2+b^2=c^2$.
				\item Tọa độ các đỉnh $A_1(-a;0)$, $A_2(a;0)$.
				\item Tiêu điểm $F_1(-c;0)$,  $F_2(c;0)$.
				\item Giao điểm $O$ của hai trục là \textbf{tâm đối xứng} của hypebol.
			\end{itemize}
		\end{tcolorbox}
	}{\vspace{0.8cm}
		\begin{tikzpicture}[scale=1, font=\footnotesize, line join=round, line cap=round, >=stealth]
			\def\a{1.2};
			\def\b{1};
			\pgfmathsetmacro\c{sqrt((\a)^2+(\b)^2)};
			\draw[->] ($(-{\a},0)-(1.5,0)$)--($({\a},0)+(1.5,0)$) node[above]{$x$};
			\draw[->] ($(0,-{\b})-(0,1)$)--($(0,{\b})+(0,1)$) node[left]{$y$};
			\draw[thick,blue,name path=h1,smooth, samples=200, domain=\a:2.5] plot (\x,{((\b)/(\a))*sqrt((\x)^2-(\a)^2)});
			\draw[thick,blue,yscale=-1,name path=h2,smooth, samples=200, domain=\a:2.5] plot (\x,{((\b)/(\a))*sqrt((\x)^2-(\a)^2)});
			\draw[thick,blue,xscale=-1,name path=h2,smooth, samples=200, domain=\a:2.5] plot (\x,{((\b)/(\a))*sqrt((\x)^2-(\a)^2)});
			\draw[thick,blue,xscale=-1,yscale=-1,name path=h2,smooth, samples=200, domain=\a:2.5] plot (\x,{((\b)/(\a))*sqrt((\x)^2-(\a)^2)});
			%			\def\xm{2};
			%			\pgfmathsetmacro\ym{((\b)/(\a))*sqrt((\xm)^2-(\a)^2)};
			\path
			(0,0) coordinate (O)
			(-\c,0) coordinate (F_1)
			(\c,0) coordinate (F_2)
			(-\a,0) coordinate (A_1)
			(\a,0) coordinate (A_2)
			(0,-\b) coordinate (B_1)
			(0,\b) coordinate (B_2)
			%				(\xm,\ym) coordinate (M)
			(\a,\b) coordinate (P)
			;
			\draw[fill=black] (O) circle(1pt)++(-0.15,-0.3) node{$O$};
			\draw[fill=black] (F_1) circle(1pt) node[below]{$F_1$};
			\draw[fill=black] (F_2) circle(1pt) node[below]{$F_2$};
			%			\draw[fill=black] (M) circle(1pt) node[right]{$M(x;y)$};
			\draw[fill=black] (A_1) circle(1pt) node[below right]{$A_1$};
			\draw[fill=black] (A_2) circle(1pt) node[below left]{$A_2$};
			\draw[fill=black] (B_1) circle(1pt) node[below right]{$B_1$};
			\draw[fill=black] (B_2) circle(1pt) node[above right]{$B_2$};
			\draw[fill=black] (P) circle(1pt) node[above]{$P$};
			\draw[dashed] (A_1)|-(B_1)-|(A_2)|-(B_2)-|(A_1);
			\draw [dashed,smooth,samples=200, domain=-2.3:2.3] plot (\x,{((\b)/(\a)*(\x))});
			\draw [dashed,smooth,samples=200, domain=-2.3:2.3] plot (\x,{-((\b)/(\a)*(\x))});
	\end{tikzpicture}}
\end{itemize}

\subsubsection{Parabol}
\immini{
\begin{itemize}
	\item [\iconMT] \indam{Định nghĩa:} Cho một điểm $F$ và một đường thẳng $\Delta$ cố định không đi qua $F$. Tập hợp các điểm $M$ cách đều $F$ và $\Delta$ là một đường parabol. Điểm $F$ gọi là \textbf{tiêu điểm} và $\Delta$ gọi là \textbf{đường chuẩn} của parabol $(P)$.
		\boxmini{$MF=\mathrm{d}(M,\Delta)$}
	\item [\iconMT] \indam{Phương trình parabol:} Gọi $H$ là hình chiếu vuông góc của $F$ trên $\Delta$. Gọi khoảng cách từ tiêu điểm đến đường chuẩn là $p=HF$.
	Chọn hệ trục toạ độ $O x y$, với $O$ là trung điểm $HF$ (như hình vẽ) thì $F\left(\dfrac{p}{2} ; 0\right)$ và $\Delta: x=-\dfrac{p}{2}$. Khi đó, phương trình $(P)$ là \boxmini{$y^2=2px$, với $p>0$}
\end{itemize}
	}{
	\begin{tikzpicture}[scale=1.1, font=\footnotesize, line join=round, line cap=round, >=stealth]
		\def\xm{-1.3};
		\pgfmathsetmacro\ym{0.5*(\xm)^2};
		\path
		(0,0) coordinate (O)
		(0.5,0) coordinate (F)
		(\xm,\ym) coordinate (N)
		(\ym,-\xm) coordinate (M)
		;
		\draw[->] (-1.5,0)--(2.5,0) node[above]{$x$};
		\draw[->] (0,-2)--(0,2) node[right]{$y$};
		\draw[rotate=-90,smooth, samples=200, domain=-2:2,thick,blue] plot (\x,{0.5*(\x)^2});
		\draw (-0.5,2)--(-0.5,-2) node[left]{$\Delta$};
		\draw[fill=black] (O) circle(1pt)++(-0.2,-0.2) node{$O$};
		\draw[fill=black] (F) circle(1pt) +(0.5,-0.4) node{$F\left(\dfrac{p}{2};0\right)$};
		\draw[fill=black] (M) circle(1pt) node[right]{$M$};
		\draw (F)--(M)--(-0.5,-\xm);
		\draw ($(-0.5,-\xm)+(0.2,0)$)|-($(-0.5,-\xm)+(0,-0.2)$);
		\draw[fill=black] (-0.5,0) circle(1pt) node[below left]{$H$};
	\end{tikzpicture}
	}
\iconMT \indam{Hình dạng parabol và các đại lượng liên quan:}
	\begin{tcolorbox}[colframe=orange,colback=orange!3,boxrule=0.2mm]
		\begin{itemize}
			\item $O$ là đỉnh; $Ox$ gọi là \textbf{trục đối xứng} của parabol $(P)$.
			\item $p$ gọi là \textbf{tham số tiêu} của parabol $(P)$.
			\item Tiêu điểm $F\left(\dfrac{p}{2} ; 0\right)$ và đường chuẩn $\Delta: x=-\dfrac{p}{2}$.
		\end{itemize}
	\end{tcolorbox}

