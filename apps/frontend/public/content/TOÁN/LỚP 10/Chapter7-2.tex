\section{ĐƯỜNG TRÒN TRONG MẶT PHẲNG TỌA ĐỘ}
\subsection{LÝ THUYẾT CẦN NHỚ}
\subsubsection{Phương trình đường tròn có tâm và bán kính cho trước}
\immini{Trong mặt phẳng $Oxy$, đường tròn $\mathscr{(C)}\colon \heva{& \text{ tâm } I(a;b) \quad(1)\\& \text{ bán kính } R \quad(2)}$ có phương trình 
\boxmini{$(x-a)^2+(y-b)^2=R^2 \quad(\star)$}
}{\hspace{1cm}
\begin{tikzpicture}[scale=0.8]
	\tikzset{label style/.style={font=\footnotesize}}
	\tkzDefPoints{0/0/I,2/0/M}
	\tkzDrawCircle[radius](I,M)
	\tkzLabelPoints[below](I)
	\tkzLabelPoints[right](M)
	\tkzDrawPoints[size=3,fill=black](I,M)
	\tkzDrawSegments(I,M)
	\tkzLabelSegment[pos=0.5,below](I,M){$R$}
\end{tikzpicture}}
\subsubsection{Dạng khai triển}
\begin{itemize}
	\item [\iconMT] \indam{Phương trình đường tròn:}  Ta có thể khai triển $(\star)$, biến đổi phương trình về dạng 
	\boxmini{$x^2+y^2-2ax-2by+c=0 \quad (\star \star)$}
	trong đó $c=a^2+b^2-R^2$.
	\item [\iconMT] \indam{Chú ý:}
	\begin{gachsoc}
		\begin{itemize}
			\item [\ding{172}] Trong trường hợp tổng quát, điều kiện để $(\star \star)$ là phương trình đường tròn là $a^2+b^2-c>0$.
			\item [\ding{173}] Để xác định tâm và bán kính của đường tròn ở dạng $(\star \star)$, ta làm như sau:
			\begin{itemize}
				\item Đổi dấu hệ số trước $x$ và $y$ rồi chia 2, ta được tâm $I(a;b)$.
				\item Tính bán kính theo công thức $R=\sqrt{a^2+b^2-c}$. 
			\end{itemize}
		\end{itemize}
	\end{gachsoc}
\end{itemize}

\subsubsection{Phương trình tiếp tuyến của đường tròn}

\immini{\iconMT \indam{Công thức tiếp tuyến:} Cho đường tròn $\mathscr{(C)}$ có tâm $I(a;b)$ và bán kính $R$. Đường thẳng $\Delta $ là tiếp tuyến với $\mathscr{(C)}$ tại điểm $M_0\left(x_0;y_0\right)$. Khi đó
	\begin{itemize}
		\item [$\bullet$] $M_0\left(x_0;y_0\right)$ thuộc $\Delta $.
		\item [$\bullet$] $\overrightarrow{IM}_0=(x_0-a;y_0-b)$ là véc-tơ pháp tuyến của $\Delta $.
	\end{itemize}
	Do đó $\Delta $ có phương trình là
	\boxmini{$\left(x_0-a\right)\left(x-x_0\right)+\left(y_0-b\right)\left(y-y_0\right)=0.$}
}{\vspace{1cm}\hspace{1cm}
\begin{tikzpicture}[scale=0.7, font=\footnotesize, line join=round, line cap=round, >=stealth]
	\tkzDefPoints{0/0/I,-2/1/M_0}
	\clip (-3.5,-2.3) rectangle (2.5,2.6);
	\draw[->] (I)--(M_0);
	\tkzDrawCircle[radius](I,M_0)
	\tkzDefPointBy[rotation=center M_0 angle 90](I)\tkzGetPoint{M'}
	\tkzDefPointBy[rotation=center M_0 angle -90](I)\tkzGetPoint{M''}	
	\draw (M')--(M'');
	\node at (-3.2,-1) [above]{$\Delta$};
	\tkzDrawPoints[fill=black,size=3](I,M_0)
	\tkzLabelPoints[above left](M_0)
	\tkzLabelPoints[below](I)
\end{tikzpicture}
}

\iconMT \indam{Chú ý:} Đường thẳng $\Delta$ tiếp xúc $\mathscr{(C)}$ khi $\mathrm{d}(I,\Delta)=R$.


\begin{dang}{Tìm tâm và bán kính đường tròn.}
	\begin{itemize}
		\item [\iconMT] \indam{Xét dạng $(x-a)^2+(y-b)^2=m\quad(1)$.}
		\begin{itemize}
			\item [\ding{172}] Nếu $m > 0$ thì (1) là phương trình đường tròn có tâm $I\left(a;b\right)$ và bán kính $R=\sqrt{m}$.
			\item [\ding{173}] Nếu $m\le 0$ thì (1) không phải là phương trình đường tròn.
		\end{itemize}
		\item [\iconMT] \indam{Xét dạng $x^2+y^2-2ax-2by+c=0 \quad(2)$.} Xét dấu biểu thức $m=a^2+b^2-c$.
		\begin{itemize}
			\item [\ding{172}] Nếu $m > 0$ thì (2) là phương trình đường tròn có
			\begin{enumEX}{2}
				\item  tâm $I\left(a;b\right)$
				\item  bán kính $R=\sqrt{a^2+b^2-c}$.
			\end{enumEX}
			\item [\ding{173}] Nếu $m\le 0$ thì (2) không phải là phương trình đường tròn.
		\end{itemize}
	\end{itemize}
\end{dang}

\begin{dang}{Lập phương trình đường tròn}
	\begin{itemize}
		\item [\iconMT] \indam{Hướng tiếp cận 1.}  Từ giả thiết, giải tìm tọa độ tâm và tính độ dài bán kính. Sau đó, viết phương trình của $\mathscr{(C)}$ theo công thức $(x-a)^2+(y-b)^2=R^2$.
		\begin{boxkn}
			\begin{itemize}
				\item [\ding{172}] $\mathscr{(C)}$ có tâm $I$ và đi qua điểm $M$, suy ra $R=IM=\sqrt{(x_M-x_I)^2+(y_M-y_I)^2}$.
				\item [\ding{173}] $\mathscr{(C)}$ tiếp xúc với đường thẳng $\Delta $ tại $A$, suy ra $\mathrm{d}\left(I;\Delta\right)=R=IA$.\\
				\hspace*{2cm}
				\begin{tikzpicture}[scale=1]
					\tikzset{label style/.style={font=\footnotesize}}
					\tkzDefPoints{0/0/I,1/0/M}
					\tkzDrawCircle[radius](I,M)
					\tkzLabelPoints[below](I)
					\tkzLabelPoints[right](M)
					\tkzDrawPoints[size=3,fill=black](I,M)
					\tkzDrawSegments(I,M)
					\tkzLabelSegment[pos=0.5,below](I,M){\scriptsize $R$}
				\end{tikzpicture}
				\hspace{2cm}
				\begin{tikzpicture}[scale=0.9]
					\tikzset{label style/.style={font=\footnotesize}}
					\tkzDefPoints{0/0/I,0/-1/A,-2/-1/H,2/-1/K}
					\tkzDrawCircle[radius](I,A)
					\tkzLabelPoints[below](A)
					\tkzLabelPoints[above](I)
					\tkzDrawPoints[size=3,fill=black](I,A)
					\tkzDrawSegments(I,A H,K)
					\tkzLabelSegment[pos=0.5,right](I,A){\scriptsize$R$}
					\tkzLabelSegment[pos=1,below](H,K){\scriptsize$\Delta$}
				\end{tikzpicture}
			\end{itemize}
		\end{boxkn}
		\item [\iconMT] \indam{Hướng tiếp cận 2.} Thường dùng cho bài toán lập phương trình đường tròn ngoại tiếp tam giác (qua ba điểm) hoặc qua hai điểm và có tâm thuộc đường thẳng $d$.
		\begin{itemize}
			\item Giả sử phương trình đường tròn $\mathscr{(C)}$ là $x^2+y^2-2ax-2by+c=0 $
			\item Từ điều kiện của đề bài, ta thiết lập hệ phương trình với ba ẩn là $a,b,c$.
			\item Giải hệ tìm $a,b,c$. Từ đó suy ra phương trình đường tròn $\mathscr{(C)}$.
		\end{itemize}
	\end{itemize}
\end{dang}

\begin{dang}{Tiếp tuyến của đường tròn tại một điểm}
	\immini{
		Cho đường tròn $\mathscr{(C)}$ có tâm $I(a;b)$ và bán kính $R$. Đường thẳng $\Delta $ là tiếp tuyến với $\mathscr{(C)}$ tại điểm $M_0\left(x_0;y_0\right)$. Khi đó
		\begin{itemize}
			\item [$\bullet$] $M_0\left(x_0;y_0\right)$ thuộc $\Delta $.
			\item [$\bullet$] $\overrightarrow{IM}_0=(x_0-a;y_0-b)$ là véc-tơ pháp tuyến của $\Delta $.
		\end{itemize}
		Do đó $\Delta $ có phương trình là
		\begin{center}
			\fbox{$\left(x_0-a\right)\left(x-x_0\right)+\left(y_0-b\right)\left(y-y_0\right)=0.$}.
		\end{center}
	}{\hspace{1cm}
		\begin{tikzpicture}[scale=0.7, font=\footnotesize, line join=round, line cap=round, >=stealth]
			\tkzDefPoints{0/0/I,-2/1/M_0}
			\clip (-3.5,-2.3) rectangle (2.5,2.6);
			\draw[->] (I)--(M_0);
			\tkzDrawCircle[radius](I,M_0)
			\tkzDefPointBy[rotation=center M_0 angle 90](I)\tkzGetPoint{M'}
			\tkzDefPointBy[rotation=center M_0 angle -90](I)\tkzGetPoint{M''}	
			\draw (M')--(M'');
			\node at (-3.2,-1) [above]{$\Delta$};
			\tkzDrawPoints[fill=black,size=4](I,M_0)
			\tkzLabelPoints[above left](M_0)
			\tkzLabelPoints[below](I)
		\end{tikzpicture}
	}	
\end{dang}

\begin{dang}{Tiếp tuyến của đường tròn song song hoặc vuông góc với một đường cho trước}
	Cho đường tròn $\mathscr{(C)}$ có tâm $I(a;b)$ và bán kính $R$.
	\begin{itemize}
		\item [\iconMT] \indam{Loại 1:} Biết tiếp tuyến song song với $d \colon Ax + By +C=0$.
		\begin{itemize}
			\item [$\bullet$] Gọi $\Delta$ là tiếp tuyến và $\Delta \parallel d$ nên $\Delta$ có dạng $Ax + By +m=0$, với $m \ne C$.
			\item [$\bullet$] $\Delta$ tiếp xúc $\mathscr{(C)}$ khi và chỉ khi
			$$\mathrm{d}(I,\Delta)=R \Leftrightarrow \dfrac{\bigg|A\cdot a+ B \cdot b +m\bigg|}{\sqrt{A^2+B^2}}=R$$ 
		\end{itemize}
		Giải điều kiện này, tìm $m$.
		\begin{luuy}
			Chú ý rằng $m$ phải khác $C$. Nếu giải ra giá trị $m=C$ thì ta loại giá trị này. 
		\end{luuy}
		\item [\iconMT] \indam{Loại 2:} Biết tiếp tuyến vuông góc với $d \colon Ax + By +C=0$.
		\begin{itemize}
			\item [$\bullet$] Gọi $\Delta$ là tiếp tuyến và $\Delta \perp d$ nên $\Delta$ có dạng $Bx - Ay +n=0$.
			\item [$\bullet$] $\Delta$ tiếp xúc $\mathscr{(C)}$ khi và chỉ khi
			$$\mathrm{d}(I,\Delta)=R \Leftrightarrow \dfrac{\bigg|B\cdot a-A \cdot b +n\bigg|}{\sqrt{A^2+B^2}}=R$$ 
		\end{itemize}
		Giải điều kiện này, tìm $n$.
	\end{itemize}
\end{dang}	