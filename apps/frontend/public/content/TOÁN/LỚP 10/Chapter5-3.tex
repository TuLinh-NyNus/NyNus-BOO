\section{CÁC SỐ ĐẶC TRƯNG ĐO ĐỘ PHÂN TÁN}
\subsection{TÓM TẮT LÝ THUYẾT}
\subsubsection{Khoảng biến thiên và khoảng tứ phân vị}
Sắp xếp mẫu số liệu theo thứ tự không giảm. Ta có
\begin{itemize}
	\item [\iconMT] \indam{Khoảng biến thiên:} Kí hiệu là $R$, là hiệu số giữa giá trị lớn nhất và giá trị nhỏ nhất trong
	mẫu số liệu.
	\item [\iconMT] \indam{Khoảng tứ phân vị:} Kí hiệu là $ \triangle_Q $, là hiệu số giữa tứ phân vị thứ ba và tứ phân vị thứ nhất, tức là
	\boxmini{$ \triangle_ Q=Q_3-Q_1 $}
	\begin{tcolorbox}[colframe=orange,colback=white,boxrule=0.2mm]
		\indamm{Chú ý:}
		\begin{itemize}
			\item  Khoảng biến thiên dùng để đo độ phân tán của mẫu số liệu. Khoảng biến thiên càng lớn thì mẫu số liệu càng phân tán.
			\item Khoảng tứ phân vị cũng là một số đo độ phân tán của mẫu số liệu. Khoảng tứ phân vị càng lớn thì mẫu số liệu cảng phân tán.
			\item Về bản chất, khoảng tứ phân vị là khoảng biến thiên của $50 \%$ số liệu chính giữa của mẫu số liệu đã sắp xếp.
			\item Một số tài liệu gọi khoảng biến thiên là biên độ và khoảng tứ phân vị là độ trải giữa.
		\end{itemize}
	\end{tcolorbox}
\end{itemize}
\subsubsection{Phương sai và độ lệch chuẩn}
\begin{itemize}
	\item [\iconMT] \indam{Phương sai:} Để đo mức độ phân tán (so với số trung bình cộng) ta dùng phương sai $s^{2}$. Cách tính như sau:
	\begin{itemize}
		\item
		Với mẫu số liệu kích thước $\mathrm{n}$ là $\left\{x_{1}, x_{2}, \ldots, x_{n}\right\}$ thì
		\boxmini{$s^{2}=\dfrac{\left(x_{1}-\overline{x}\right)^{2}+\left(x_{2}-\overline{x}\right)^{2}+\ldots+\left(x_{n}-\overline{x}\right)^{2}}{n}=\dfrac{1}{n}\left(x_1^2+x_2^2+\cdots + x_n^2\right)-\overline{x}^2$}
		\item
		Với mẫu số liệu được cho bởi bảng phân bố tần số thì
		\boxmini{$s^{2}=\dfrac{n_1\left(x_{1}-\overline{x}\right)^{2}+n_2\left(x_{2}-\overline{x}\right)^{2}+\ldots+n_k\left(x_{k}-\overline{x}\right)^{2}}{n}=\dfrac{1}{n}\left(n_1x_1^2+n_2x_2^2+\cdots + n_kx_k^2\right)-\overline{x}^2$}
		Trong đó $n_k$ là tần số của giá trị $x_k$ và $n_1+n_2+ \cdots + n_k=n$.
	\end{itemize}
	\item [\iconMT] \indam{Độ lệch chuẩn:} Căn bậc hai của phương sai gọi là độ lệch chuẩn, kí hiệu là $s$. Ta có $s=\sqrt{s^2}$.
	\begin{tcolorbox}[colframe=orange,colback=white,boxrule=0.2mm]
		\indamm{Chú ý:}
		\begin{itemize}
			\item  Phương sai và độ lệch chuẩn càng lớn thì độ phân tán của các số liệu thống kê càng lớn.
			\item  Phương sai $s^2$ và độ lệch chuẩn $s$ đều được dùng để đánh giá mức độ phân tán của các số liệu thống kê (so với số trung bình cộng). Nhưng khi cần chú ý đến đơn vị đo thì ta dùng $s$ vì $s$ có cùng đơn vị đo với dấu hiệu được nghiên cứu.
		\end{itemize}
	\end{tcolorbox}
\end{itemize}
\subsubsection{Phát hiện số liệu bất thường hoặc không chính xác bằng biểu đồ hộp}
Trong mẫu số liệu thống kê, có khi gặp những giá trị quá lớn 
hoặc quá nhỏ so với đa số các giá trị khác. Những giá trị này được gọi 
là \indamm{giá trị bất thường}. Ta có thể dùng biểu đồ hộp để phát hiện các giá trị bất thường 
này.
\begin{center}
	\begin{tikzpicture}[>=stealth, font=\footnotesize, line join=round, line cap=round,transform shape,scale=1]
		\begin{scope}[scale=1]
			%% Vẽ hộp
			\foreach \t in {7, 8.5, 2, 11} \draw (\t,0.5)--(\t,2);
			\draw (5.2,0.5)--(8.5,0.5) (5.2,2)--(8.5,2);
			\draw (2,1.25)--(11,1.25);
			\fill[magenta!20!white] (5.2,0.5) rectangle (8.5,2);
			\draw[color=yellow, thick] (7,0.5)--(7,2);
			%% ghi các giá trị phía dưới
			\foreach \e/\f  in {2/\big(Q_1-1{,}5 \cdot  
				\Delta_{Q}\big),7/Q_2,5.2/Q_1,8.5/Q_3,11/\big(Q_3+1{,}5 \cdot 
				\Delta_{Q}\big)}
			{
				\node[shift={(-90:4mm)}] at (\e,0.5){$\f$};
			}
			%% Vẽ giá rị bất thường
			\foreach \btd  in {0.2,0.6}	
			\draw[fill=orange] (2-\btd,1.25) circle (0.08);
			\foreach \btt  in {0.2,0.4}	
			\draw[fill=orange]  (11+\btt,1.25) circle (0.08);
			\draw [thick,->](2-0.2,2.2)--(2-0.2,1.35);
			\draw [thick,->](11+0.4,2.2)--(11+0.4,1.35);
			\node [text = red,
			fill = cyan!30!white,rounded corners=8pt] at (2 -0.5,2.5){Các giá trị bất thường};
			\node [text = red,
			fill = cyan!30!white,rounded corners=8pt]  at (11 +0.5,2.5){Các giá trị bất thường};
			%%% Vẽ delta Q
			\draw (5.2,2 +0.2)--node[above]{$\Delta_{Q}$}(8.5,2+0.2);
			\draw (5.2,2 +0.1)--(5.2,2 +0.3) (8.5,2 +0.1)--(8.5,2 +0.3);
		\end{scope}
		\draw (-1,-1) rectangle (14,2+1);
	\end{tikzpicture}
\end{center}
Các số liệu \indamm{lớn hơn} $Q_3+1{,}5 \cdot 
\Delta_{Q}$ hoặc \indamm{bé hơn} $Q_1-1{,}5 \cdot 
\Delta_{Q}$ được xem là giá trị bất thường. Hay nói cách khác, các số liệu \indamm{không} thuộc đoạn $\left[Q_1-1{,}5 \cdot \Delta_{Q}; Q_3+1{,}5 \cdot \Delta_{Q} \right]$ là các số liệu bất thường.
