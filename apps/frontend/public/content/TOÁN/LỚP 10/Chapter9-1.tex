\section{KHÔNG GIAN MẪU VÀ BIẾN CỐ}
\subsection{LÝ THUYẾT CẦN NHỚ}
\subsubsection{Phép thử ngẫu nhiên, không gian mẫu}
	\begin{itemize}
		\item [$\bullet$]  \textbf{Phép thử ngẫu nhiên} (gọi tắt là phép thử) là một thí nghiệm hay một hành động mà kết quả của nó không thể biết được trước khi phép thử được thực hiện.
		\item [$\bullet$]  \textbf{Không gian mẫu} của phép thử (kí hiệu là $\Omega$) là tập hợp tất cả các kết quả có thể xảy ra khi thực hiện phép thử.
	\end{itemize}
\indamm{Chú ý}: Ta chỉ xét phép thử mà không gian mẫu gồm hữu hạn kết quả.
\subsubsection{Biến cố}
\immini{
		\begin{itemize}
			\item [$\bullet$] Biến cố là một tập con của không gian mẫu $\Omega $, kí hiệu $A$, $B$, $C$,...
			\item [$\bullet$] Một kết quả thuộc tập $A$ được gọi là kết quả thuận lợi cho biến cố $A$.
		\end{itemize}
}{
	\begin{tikzpicture}[scale=.6]
		\tkzDefPoints{0/0/A, 5/0/B, 0/3/D, 5/3/C}
		\fill[cyan!10] (A)--(B)--(C)--(D);
		\tkzDrawPolygon(A,B,C,D)
		\draw(4,0)--(4.5,-.5) node[below] {$\Omega$};
		\node (A) at (1.5,1.5) [circle, draw] {$A$};
\end{tikzpicture}}
\indamm{Chú ý}: 
\begin{itemize}
	\item [$\bullet$] Biến cố chắc chắn là biến cố luôn xảy ra, kí hiệu là $\Omega$.
	\item [$\bullet$] Biến cố không thể là biến cố không bao giờ xảy ra, ki hiệu là $\varnothing$.
\end{itemize}
