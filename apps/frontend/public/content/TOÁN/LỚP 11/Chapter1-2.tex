\section{GIÁ TRỊ LƯỢNG GIÁC CỦA MỘT GÓC LƯỢNG GIÁC}
\subsection{LÝ THUYẾT CẦN NHỚ}
\subsubsection{GIÁ TRỊ LƯỢNG GIÁC CỦA GÓC LƯỢNG GIÁC}
\begin{itemize}
	\item [] \indamm{ Ghi nhớ 1: } Giả sử $M(x_0;y_0)$ trên đường tròn lượng giác biểu diễn cho góc lượng giác có số đo $\alpha$.
	\begin{gachsoc}
		\immini{
			\begin{itemize}
				\item [\ding{172}] Tung độ $y_0$ của điểm $M$ gọi là sin của $\alpha $ và kí hiệu là $\sin \alpha$, hay $\sin \alpha = y_0$.
				\item [\ding{173}] Hoành độ $x_0$ của điểm $M$ gọi là côsin của $\alpha $ và kí hiệu là $\cos \alpha$, hay $\cos \alpha =x_0.$
				\item [\ding{174}] Nếu $x_0 \ne 0$ thì tỉ số $\dfrac{y_0}{x_0}\dfrac{\sin \alpha}{\cos \alpha}$ gọi là tang của góc $\alpha$, kí hiệu $\tan \alpha$. Nghĩa là
				$\tan \alpha =\dfrac{\sin \alpha}{\cos \alpha}, \text{ với } \cos \alpha \ne 0.$
				\item[\ding{175}] Nếu $y_0 \ne 0$ thì tỉ số $\dfrac{x_0}{y_0}\dfrac{\cos \alpha}{\sin \alpha}$ gọi là côtang của góc $\alpha$, kí hiệu $\cot \alpha$. Nghĩa là
				$\cot \alpha =\dfrac{\cos \alpha}{\sin \alpha}, \text{ với } \sin \alpha \ne 0.$
		\end{itemize}}{
			\begin{tikzpicture}[smooth,samples=300,scale=1.6,>=stealth]
				\path
				(0,0) coordinate (O)
				(0,1) coordinate (B)
				(1,0) coordinate (A)
				($(O)-(A)$) coordinate (A')
				($(O)-(B)$) coordinate (B')
				(O) + (140:1) coordinate (M)
				($(A)!(M)!(A')$) coordinate (x_0)
				($(B)!(M)!(B')$) coordinate (y_0)
				;
				\draw[->] (-1.2,0)--(1.5,0) node[below]{$x$};
				\draw[->] (0,-1.2)--(0,1.5) node[right]{$y$};
				\draw 
				(0,0) node[below right]{$O$}
				(O) circle (1 cm)
				(O)--(M)
				;
				\draw[dashed]
				(x_0) node[below]{$x_0$}--(M)--(y_0)node[right]{$y_0$}
				;
				\draw [shift={(0,0)},line width=0.5pt,red] (0:.3) arc (0:70:.3)node[right]{\scriptsize$\alpha$};
				\draw [shift={(0,0)},->,line width=0.5pt,red] (0:.3) arc (0:140:.3);
				\foreach \x/\g in {A/30, A'/150,B/30,B'/-30,M/150} \draw[fill=black](\x) circle (0.02) + (\g:0.2)node{$\x$};
l		\end{tikzpicture}}
	\end{gachsoc}
	\item [] \indamm{Ghi nhớ 2:} Ta có các kết quả sau được suy ra từ định nghĩa
	\begin{gachsoc}
		\begin{itemize}
			\item [\ding{172}] Vì $-1 \le x_0;\,y_0 \le 1$ nên
			\boxminit{
				$-1 \le \sin \alpha \le 1$; \quad
				$-1 \le \cos \alpha \le 1$.}
			\item [\ding{173}] $\sin \alpha $ và $\cos \alpha $ xác định với mọi $\alpha \in \mathbb{R}.$ Hơn nữa, $\forall k \in \mathbb{Z}$ ta có
			\boxminit{	$\sin \left( \alpha + k2 \pi \right) = \sin \alpha$ ; \quad
				$\cos \left( \alpha + k2 \pi \right) = \cos \alpha$.}
			\item [\ding{174}] $\tan \alpha $ xác định với mọi $\alpha \ne \dfrac{\pi}{2}+k\pi \left(k\in \mathbb{Z}\right)$; $\cot \alpha $ xác định với mọi $\alpha \ne k\pi \left(k\in \mathbb{Z}\right)$ và
			\boxminit{
				$\tan \left( \alpha + k \pi \right) = \tan \alpha$ ; \quad
				$\cot \left( \alpha + k \pi \right) = \cot \alpha$.}
		\end{itemize}
	\end{gachsoc}
\end{itemize}
\subsubsection{HỆ THỨC CƠ BẢN GIỮA CÁC GIÁ TRỊ LƯỢNG GIÁC CỦA GÓC LƯỢNG GIÁC}
	Đối với các giá trị lượng giác, ta có các hằng đẳng thức sau:
	\begin{gachsoc}
			\begin{listEX}[2]
			\item [\ding{172}] $\sin^2\alpha+\cos^2\alpha =1$.
			\item [\ding{173}] $1+\tan^2\alpha =\dfrac{1}{\cos^2\alpha}$, với $\alpha \ne \dfrac{\pi}{2}+k\pi$.
			\item [\ding{174}] $1+\cot^2\alpha =\dfrac{1}{\sin^2\alpha}$, với $\alpha \ne k\pi$.
			\item [\ding{175}] $\tan \alpha \cdot \cot \alpha =1$, với $\alpha \ne \dfrac{k\pi}{2}$.
		\end{listEX}
	\end{gachsoc}
\subsubsection{GIÁ TRỊ LƯỢNG GIÁC CỦA CÁC GÓC CÓ LIÊN QUAN ĐẶC BIỆT}
\begin{itemize}
	\item[] \indamm{ Góc đối nhau:} \fbox{$\alpha $ và $-\alpha $}  tương ứng với hai điểm "đại diện" là điểm $M$ và điểm $M'$. Muốn so sánh sin, ta so sánh tung độ; muốn so sánh cos, ta so sánh hoành độ. Hình vẽ bên, hai điểm $M$ và $M'$ đối xứng nhau qua trục hoành nên ta có kết quả sau:
	\begin{gachsoc}
		\immini{	\begin{itemize}
				\item [$\bullet$] $\cos \left( - \alpha \right) = \cos \alpha$
				\item [$\bullet$] $\sin \left( - \alpha \right) = -\sin \alpha$
				\item [$\bullet$] $\tan \left( - \alpha \right) = -\tan \alpha$
				\item [$\bullet$] $\cot \left( - \alpha \right) = -\cot \alpha$
			\end{itemize}
		}{
			\begin{tikzpicture}[>=stealth, scale= 1.15]
				\draw[->] (-1.6,0) -- (1.6,0) node[above] {$x$};
				\draw[->] (0,-1.4) -- (0,1.4) node[left] {$y$};
				\tkzDefPoints{0/0/O, 1/0/A, -1/0/A', 0/1/B, 0/-1/B'}
				\tkzDefPointBy[rotation = center O angle 50](A) \tkzGetPoint{M}
				\tkzDefPointBy[rotation = center O angle -50](A) \tkzGetPoint{M'}
				\tkzDrawPoints[size=5,fill=black](A,B,A',B',M,M',O)
				\tkzDrawCircle[radius](O,A)
				\draw [shift={(0,0)},->,line width=0.3pt] (0:.3) arc (0:50:.3);
				\draw [shift={(0,0)},->,line width=0.3pt] (0:.3) arc (0:-50:.3);
				\tkzDrawSegments(O,M O,M')
				\tkzDrawSegments[dashed](M,M')
				\node at (.4,.17) {\tiny $\alpha$};
				\node at (.4,-.17) {\tiny $-\alpha$};
				\tkzLabelPoints[below left,font=\footnotesize](A',O)
				\tkzLabelPoints[below right,font=\footnotesize](A,B',M')	
				\tkzLabelPoints[above right,font=\footnotesize](B,M)
		\end{tikzpicture}}
	\end{gachsoc}
	\item [] \indamm{ Góc bù nhau:} \fbox{$\alpha $ và $\pi-\alpha $} Hình vẽ bên, hai điểm $M$ và $M'$ đối xứng nhau qua trục tung nên ta có kết quả sau:
	\begin{gachsoc}
		\immini{	\begin{itemize}
				\item [$\bullet$] $\cos \left( \pi - \alpha \right) = -\cos \alpha$
				\item [$\bullet$] $\sin \left( \pi - \alpha \right) = \sin \alpha$
				\item [$\bullet$] $\tan \left( \pi - \alpha \right) = -\tan \alpha$
				\item [$\bullet$] $\cot \left( \pi - \alpha \right) = -\cot \alpha$
			\end{itemize}
		}{
			\begin{tikzpicture}[>=stealth, scale= 1.25]
				\draw[->] (-1.6,0) -- (1.6,0) node[above] {$x$};
				\draw[->] (0,-1.4) -- (0,1.4) node[left] {$y$};
				\tkzDefPoints{0/0/O, 1/0/A, -1/0/A', 0/1/B, 0/-1/B'}
				\tkzDefPointBy[rotation = center O angle 50](A) \tkzGetPoint{M}
				\tkzDefPointBy[rotation = center O angle 130](A) \tkzGetPoint{M'}
				\tkzDrawPoints[size=5,fill=black](A,B,A',B',M,M',O)
				
				\tkzDrawCircle[radius](O,A)
				\draw [shift={(0,0)},->,line width=0.3pt] (0:.3) arc (0:50:.3);
				\draw [shift={(0,0)},->,line width=0.3pt] (0:.2) arc (0:127:.2);
				\tkzDrawSegments(O,M O,M')
				\tkzDrawSegments[dashed](M,M')
				\node at (.4,.17) {\tiny $\alpha$};
				\node at (.1,.4) {\tiny $\pi-\alpha$};
				\tkzLabelPoints[below left,font=\footnotesize](A',O)
				\tkzLabelPoints[below right,font=\footnotesize](A,B')
				\tkzLabelPoints[above left,font=\footnotesize](M')	
				\tkzLabelPoints[above right,font=\footnotesize](B,M)
		\end{tikzpicture}}
	\end{gachsoc}
	
	\item [] \indamm{Góc hơn kém $\pi $:} \fbox{$\alpha $ và $\alpha+\pi$} Hình vẽ bên, hai điểm $M$ và $M'$ đối xứng nhau qua gốc $O$ nên ta có kết quả sau:
	\begin{gachsoc}
		\immini{	\begin{itemize}
				\item [$\bullet$] $\cos \left( \alpha+\pi  \right) = -\cos \alpha$
				\item [$\bullet$] $\sin \left( \alpha+\pi  \right) = -\sin \alpha$
				\item [$\bullet$] $\tan \left( \alpha+\pi  \right) = \tan \alpha$
				\item [$\bullet$] $\cot \left( \alpha+\pi  \right) = \cot \alpha$
			\end{itemize}
		}{
			\begin{tikzpicture}[>=stealth, scale= 1.2]
				\draw[->] (-1.6,0) -- (1.6,0) node[above] {$x$};
				\draw[->] (0,-1.4) -- (0,1.4) node[left] {$y$};
				\tkzDefPoints{0/0/O, 1/0/A, -1/0/A', 0/1/B, 0/-1/B'}
				\tkzDefPointBy[rotation = center O angle 50](A) \tkzGetPoint{M}
				\tkzDefPointBy[rotation = center O angle 230](A) \tkzGetPoint{M'}
				\tkzDrawPoints[size=5,fill=black](A,B,A',B',M,M',O)
				
				\tkzDrawCircle[radius](O,A)
				\draw [shift={(0,0)},->,line width=0.3pt] (0:.3) arc (0:50:.3);
				\draw [shift={(0,0)},->,line width=0.3pt] (0:.2) arc (0:230:.2);
				\tkzDrawSegments(O,M O,M')
				%	\tkzDrawSegments[dashed](M,M')
				\node at (.4,.17) {\tiny $\alpha$};
				\node at (-.3,.3) {\tiny $\pi +\alpha$};
				\tkzLabelPoints[below left,font=\footnotesize](A',M')
				\tkzLabelPoints[below right,font=\footnotesize](A,B',O)	
				\tkzLabelPoints[above right,font=\footnotesize](B,M)
		\end{tikzpicture}}
	\end{gachsoc}
	\item [] \indamm{Góc phụ nhau:} \fbox{$\alpha $ và $\dfrac{\pi}{2}-\alpha$} Hình vẽ bên, hai điểm $M$ và $M'$ có hoành độ và tung độ ngược nhau nên ta có kết quả sau:
	\begin{gachsoc}
		\immini{	\begin{itemize}
				\item [$\bullet$] $\cos \left( \dfrac{\pi}{2}-\alpha  \right) = \sin \alpha$
				\item [$\bullet$] $\sin \left( \dfrac{\pi}{2}-\alpha  \right) = \cos \alpha$
				\item [$\bullet$] $\tan \left( \dfrac{\pi}{2}-\alpha  \right) = \cot \alpha$
				\item [$\bullet$] $\cot \left( \dfrac{\pi}{2}-\alpha  \right) = \tan \alpha$
			\end{itemize}
		}{
			\begin{tikzpicture}[>=stealth, scale= 1.4]
				\draw[->] (-1.6,0) -- (1.6,0) node[above] {$x$};
				\draw[->] (0,-1.4) -- (0,1.4) node[left] {$y$};
				\tkzDefPoints{0/0/O, 1/0/A, -1/0/A', 0/1/B, 0/-1/B'}
				\tkzDefPointBy[rotation = center O angle 30](A) \tkzGetPoint{M}
				\tkzDefPointBy[rotation = center O angle 60](A) \tkzGetPoint{M'}
				\tkzDrawPoints[size=5,fill=black](A,B,A',B',M,M',O)
				\tkzDrawCircle[radius](O,A)
				\draw [shift={(0,0)},line width=0.3pt] (0:.2) arc (0:30:.2);
				\draw [shift={(0,0)},->,line width=0.3pt] (0:.4) arc (0:60:.4);
				\tkzDefPointBy[projection=onto O--A](M)\tkzGetPoint{H}
				\tkzDefPointBy[projection=onto O--B](M)\tkzGetPoint{K}
				\tkzDefPointBy[projection=onto O--A](M')\tkzGetPoint{H'}
				\tkzDefPointBy[projection=onto O--B](M')\tkzGetPoint{K'}
				\tkzDrawSegments(O,M O,M')
				\tkzDrawSegments[dashed](M,M' M,H M,K M',H' M',K')
				\node at (.26,.08) {\tiny $\alpha$};
				%\node at (.4,-.17) {\tiny $\tfrac{\pi}{2}-\alpha$};
				\tkzLabelPoints[below left,font=\footnotesize](A',O)
				\tkzLabelPoints[below right,font=\footnotesize](A,B')	
				\tkzLabelPoints[above right,font=\footnotesize](B,M,M')
		\end{tikzpicture}}
	\end{gachsoc}
\end{itemize}

\begin{dang}{Tính các giá trị lượng giác của một góc lượng giác}
	\begin{itemize}
		\item [\iconMT] \indam{Phương pháp:} Sử dụng nhóm công thức liên hệ giữa các giá trị lượng giác để tính toán.
		\item [\iconMT] \indam{Chú ý:} 
		\immini{ Nếu đề bài có giới hạn miền của góc $\alpha$, thì ta cần xem trên miền đó, các tỉ số lượng giác tương ứng sẽ mang dấu như thế nào. Cụ thể:
			
			\begin{tabular}{c|c|c|c|c|}
				\cline{2-5}
				& \multicolumn{4}{c|}{Góc phần tư} \\ \hline
				\multicolumn{1}{|c|}{Giá trị lượng giác} & I     & II     & III     & IV    \\ \hline
				\multicolumn{1}{|c|}{$\sin \alpha$}     &   $+$    &  $ +$      &    $-$    &   $-$  \\ \hline
				\multicolumn{1}{|c|}{$\cos \alpha$}     &   $+$    &  $ -$      &    $-$    &   $+$  \\ \hline
				\multicolumn{1}{|c|}{$\tan \alpha$}     &   $+$    &  $ -$      &    $+$    &   $-$  \\ \hline
				\multicolumn{1}{|c|}{$\cot \alpha$}     &   $+$    &  $ -$      &    $+$    &   $-$  \\ \hline
			\end{tabular}
		}{
			
			\begin{tikzpicture}[>=stealth, scale= 1.45]
				\draw[->] (-1.6,0) -- (1.6,0) node[above] {$x$};
				\draw[->] (0,-1.4) -- (0,1.4) node[right] {$y$};
				\tkzDefPoints{0/0/O, 1/0/A}
				\tkzDefPointBy[symmetry = center O](A)\tkzGetPoint{A'}
				\tkzDefPointBy[rotation = center O angle 90](A) \tkzGetPoint{B}
				\tkzDefPointBy[rotation = center O angle -90](A) \tkzGetPoint{B'}
				\tkzDefPointBy[rotation = center O angle 130](A) \tkzGetPoint{M}
				\tkzDrawCircle[radius](O,A)
				\draw [shift={(0,0)},->,line width=0.3pt] (0:.3) arc (0:130:.3);
				\tkzDefPointBy[projection = onto O--B](M)\tkzGetPoint{K}
				\tkzDefPointBy[projection = onto O--A'](M)\tkzGetPoint{H}
				\tkzDrawPoints[color=black](O,M,A,A',B,H,K,B')
				\tkzDrawSegments(O,M)
				\tkzDrawSegments[dashed](M,K M,H)
				\node at (.3,.3) {\footnotesize$\alpha$};
				\node[above right] at (0.9,0.8) {$\mathrm{I}$};
				\node[above left] at (-0.9,0.8) {$\mathrm{II}$};
				\node[below right] at (0.9,-0.8) {$\mathrm{IV}$};
				\node[below left] at (-0.9,-0.8) {$\mathrm{III}$};
				\tkzLabelPoints[below left,font=\footnotesize](A')
				\tkzLabelPoints[below right,font=\footnotesize](A,O,B')	
				\tkzLabelPoints[above left,font=\footnotesize](M,B)
		\end{tikzpicture}}
	\end{itemize}
\end{dang}
