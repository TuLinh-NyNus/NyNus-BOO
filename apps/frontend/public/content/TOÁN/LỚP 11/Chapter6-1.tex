%\setcounter{section}{17}
\section{PHÉP TÍNH LŨY THỪA}
\subsection{LÝ THUYẾT CẦN NHỚ}
\subsubsection{Lũy thừa với số mũ nguyên}
	\begin{itemize}
		\item [$\bullet$] Lũy thừa với số mũ nguyên dương: Cho $a\in \mathbb{R}, n\in {\mathbb{N}}^{*}$, khi đó: $a^n=\underbrace{a.a.a...a}_{n \text{ thừa số}}$.
		\item [$\bullet$] Lũy thừa với số mũ nguyên âm: Cho $a \ne 0, n\in {\mathbb{N}}^{*}$, khi đó: $a^{-n}=\dfrac{1}{a^n}$.
		\item [$\bullet$] Với $a \ne 0$, ta quy ước $a^0=1$; \quad $0^0$ và $0^{-n}\, (n \in \mathbb{N}^{*})$ không có nghĩa.
	\end{itemize}

\subsubsection{Lũy thừa với số mũ hữu tỉ}
	Cho $a>0$ và số hữu tỉ $r=\dfrac{m}{n}$; trong đó $m\in \mathbb{Z},\,n\in \mathbb{N},\,n\ge 2$. Khi đó: $a^r={a}^{\tfrac{m}{n}}=\sqrt[n]{a^m}.$
\subsubsection{Công thức biến đổi lũy thừa cần nhớ}
	Cho cơ số $a,b>0$ và hai số thực $m,n$. Khi đó, ta có:
	\begin{tcolorbox}[colframe=cyan,colback=cyan!4,boxrule=0.2mm]
		\begin{listEX}[3]
			\item [\ding{172}] $a^0=1$; $a^1=a$.
			\item [\ding{173}] $a^{-1}=\dfrac{1}{a}$; $a^{-n}=\dfrac{1}{a^n}$.
			\item [\ding{174}] $\sqrt{a}=a^{\tfrac{1}{2}}$; $\sqrt[n]{a^m}={a}^{\tfrac{m}{n}}$.
			\item [\ding{175}] $a^{m+n}=a^m\cdot a^n$.
			\item [\ding{176}] $a^{m-n}=\dfrac{a^m}{a^n}$.
			\item [\ding{177}] $a^{m\cdot n}=\left(a^m\right)^n=\left(a^n\right)^m$.
			\item [\ding{178}] $(ab)^n=a^n \cdot b^n$.
			\item [\ding{179}] $\left(\dfrac{a}{b}\right)^n=\dfrac{a^n}{b^n}$.
			\item [\ding{180}]  $\left(\dfrac{a}{b}\right)^n=\left(\dfrac{b}{a}\right)^{-n}$.
		\end{listEX}
	\end{tcolorbox}
\subsubsection{So sánh hai lũy thừa cùng cơ số}
Cho cơ số $a>0$ và hai số thực $x,y$. Khi đó, ta có:
\begin{listEX}[2]
	\item [\ding{172}] Nếu $a>1$ thì $a^x>a^y\Leftrightarrow x>y$.
	\item [\ding{173}] Nếu $0<a<1$ thì $a^x>a^y\Leftrightarrow x<y$.
\end{listEX}
