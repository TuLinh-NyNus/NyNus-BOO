\section{PHÉP CHIẾU VUÔNG GÓC. GÓC GIỮA ĐƯỜNG THẲNG VÀ MẶT PHẲNG}
\subsection{KIẾN THỨC CẦN NHỚ}
\subsubsection{Phép chiếu vuông góc}
Phép chiếu song song lên mặt phẳng $(P)$ theo phương $\Delta$ vuông góc với $(P)$ được gọi là \emph{phép chiếu vuông góc} lên mặt phẳng $(P)$.
\begin{boxkn}
\begin{itemize}
	\item Vì phép chiếu vuông góc lên một mặt phẳng là một trường hợp đặc biệt của phép chiếu song song nên nó có mọi tính chất của phép chiếu song song.
	\item Phép chiếu vuông góc lên mặt phẳng $(P)$ còn được gọi đơn giản là phép chiếu lên mặt phẳng $(P)$. \emph{Hình chiếu vuông góc} $\mathscr{H}'$ của hình $\mathscr{H}$ trên mặt phẳng $(P)$ còn được gọi là \emph{hình chiếu} của $\mathscr{H}$ trên mặt phẳng $(P)$.
\end{itemize}
\end{boxkn}
\subsubsection{Định lý ba đường vuông góc}
\immini{
	Cho đường thẳng $a$ nằm trong mặt phẳng $(\alpha)$ và $b$ là đường thẳng không thuộc $(\alpha)$ đồng thời không vuông góc với $(\alpha)$. Gọi $b'$ là hình chiếu vuông góc của $b$ trên $(\alpha)$. Khi đó $a$ vuông góc với $b$ khi và chỉ khi a vuông góc với $b'$.
}{
	\begin{tikzpicture}[line cap=round,line join=round,scale=0.6,font=\footnotesize]
		\path 
		(-4,-4) coordinate (M)
		(-2,-1) coordinate (Q)
		(2,-4) coordinate (N)
		(-1,-2.5) coordinate (A')
		(2.2,-2.5) coordinate (B')
		($(Q)+(N)-(M)$) coordinate (P)
		($(Q)!(A')!(P)$) coordinate (m)
		(intersection of A'--m and Q--P) coordinate (m_1)
		($(Q)!(B')!(P)$) coordinate (n)
		(intersection of B'--n and Q--P) coordinate (n_1)
		;
		\coordinate[label={above}:$A$] (A) at ($(m_1)+(0,1)$);
		\coordinate[label={above}:$B$] (B) at ($(n_1)+(0,2)$);
		\coordinate (H) at ($(A')!0.3!(B')$);
		\coordinate (x) at ($(H)+(Q)-(M)$);
		\coordinate (H_1) at ($(H)!0.4!(x)$);
		\coordinate[label={right}:$a$] (H_2) at ($(H)!-0.4!(x)$);
		\foreach \p/\r in {A'/-90,B'/-90}
		\fill (\p) circle (1.5pt) node[shift={(\r:3mm)}]{$\p$};
		\draw[dashed] (m_1)--(n_1) (H)--(H_1);
		\draw ($(A)!1.2!(B)$)--($(B)!1.2!(A)$)node[pos=0.4,above]{$b$} 	
		($(A')!1.2!(B')$)--($(B')!1.2!(A')$)node[pos=0.7,above]{$b'$}  
		(M)--(N)--(P)--(n_1) (m_1)--(Q)--(M) (H_2)--(H) (A)--(A') (B)--(B');
		\draw 
		pic[draw, "\scriptsize$(P)$",angle radius=1cm]{angle = N--M--Q}
		pic [draw,angle radius=2mm]{right angle=H_2--H--B'};	
	\end{tikzpicture}
}
\subsubsection{Góc giữa đường thẳng và mặt phẳng}
	\immini{
	Cho đường thẳng $d$ và mặt phẳng $(\alpha)$.
	\begin{itemize}
		\item [$\bullet$] Trường hợp $d \perp (\alpha)$ thì góc giữa đường thẳng $d$ và $(\alpha)$ bằng $90^\circ$.
		\item [$\bullet$] Trường hợp $d$ không vuông góc với $(\alpha)$ thì góc giữa $d$ và và $(\alpha)$ bằng góc giữa $d$ hình chiếu $d'$ của nó trên $(\alpha)$.
	\end{itemize}
\begin{luuy}
	Nếu $\varphi$ là góc giữa đường thẳng $d$ và mặt phẳng $(\alpha)$ thì ta luôn có $0^\circ \leq \varphi \leq 90^\circ$.
\end{luuy}
}{\vspace{1cm}
	\begin{tikzpicture}[font=\footnotesize, line join=round, line cap=round,>=stealth,scale=0.6]
		\tikzset{label style/.style={font=\footnotesize}}
		\tkzInit[ymin=-5,ymax=0.5,xmin=-4.1,xmax=4.1]
		\tkzClip %cắt bớt phần ko gian dư
		\tkzDefPoints{-4/-4/M, -2/-1.5/Q, 2/-4/N, -0.5/-3/H}
		\tkzDefPointBy[translation = from M to N](Q) \tkzGetPoint{P}%phép tịnh tiến biến Q thành P
		\tkzDefLine[perpendicular=through H,K=0.4](Q,P) \tkzGetPoint{h}%đường qua H vuông góc QP
		\tkzInterLL(H,h)(Q,P) \tkzGetPoint{h_1}
		\coordinate[label={above right}:$A$] (A) at ($(h_1)+(0,1)$);
		\tkzDefLine[parallel=through H](Q,P) \tkzGetPoint{x}%đường qua H song song MQ
		\coordinate[label={above right}:$O$] (O) at ($(H)!0.4!(x)$);
		\tkzInterLL(A,O)(N,P) \tkzGetPoint{O_1}
		\tkzInterLL(A,O)(P,Q) \tkzGetPoint{O_3}
		\coordinate (O_2) at ($(O)!3.3!(O_1)$);
		\tkzLabelPoints[below](H)
		\tkzDrawLines[add=0 and 0.3](O,H O,A)
		\tkzDrawSegments[dashed](h_1,O_3 O,O_1)
		\tkzDrawSegments(M,N N,P P,O_3 h_1,Q Q,M A,H O_1,O_2)
		\tkzLabelLine[below left,pos=1.3](O,A){$d$}
		\tkzLabelLine[above,pos=1.3](O,H){$d'$}
		\tkzMarkRightAngle(A,H,O)
		\tkzMarkAngles[arc=l,size=0.7cm](A,O,H)
		\tkzLabelAngle[pos=0.9](A,O,H){$\varphi$}
		\tkzMarkAngles[arc=l,size=1.1cm](N,M,Q)
		\tkzLabelAngle[pos=0.7](N,M,Q){$\alpha$}
		\tkzDrawPoints[fill=black](A,H,O)
	\end{tikzpicture}
}

\begin{dang}{Góc giữa đường thẳng và mặt phẳng}
	Cho đường thẳng $d$ và mặt phẳng $(P)$ cắt nhau.
	\begin{itemize}
	\item	\immini{ [\iconMT] Nếu $d\perp (P)$ thì $(d,(P))=90^{\circ}$.
			\item [\iconMT] Nếu $d$ không vuông $(P)$ thì  để xác định góc giữa $d$ và $(P)$, ta thường làm như sau
			\begin{itemize}
				\item Xác định giao điểm $O$ của $d$ và $(P)$.
				\item Lấy một điểm $A$ trên $d$ ($A$ khác $O$). Xác định hình chiếu vuông góc (vuông góc) $H$ của $A$ lên $(P)$. Lúc đó $(d,(P))=(d,d')=\widehat{AOH}$.
			\end{itemize}
			\begin{luuy}
				$0^{\circ}\leq (d,(P))\leq 90^{\circ}$.
			\end{luuy}
		}{
			\begin{tikzpicture}[scale=0.7,scale=0.9]
				\tikzset{label style/.style={font=\scriptsize}}
				\tkzInit[ymin=-5,ymax=0.5,xmin=-4.1,xmax=4.1]
				\tkzClip %cắt bớt phần ko gian dư
				\tkzDefPoints{-4/-4/M, -2/-1.5/Q, 2/-4/N, -0.5/-3/H}
				\tkzDefPointBy[translation = from M to N](Q) \tkzGetPoint{P}%phép tịnh tiến biến Q thành P
				\tkzDefLine[perpendicular=through H,K=0.4](Q,P) \tkzGetPoint{h}%đường qua H vuông góc QP
				\tkzInterLL(H,h)(Q,P) \tkzGetPoint{h_1}
				\coordinate[label={above right}:$A$] (A) at ($(h_1)+(0,1)$);
				\tkzDefLine[parallel=through H](Q,P) \tkzGetPoint{x}%đường qua H song song MQ
				\coordinate[label={above right}:$O$] (O) at ($(H)!0.4!(x)$);
				\tkzInterLL(A,O)(N,P) \tkzGetPoint{O_1}
				\tkzInterLL(A,O)(P,Q) \tkzGetPoint{O_3}
				\coordinate (O_2) at ($(O)!3.3!(O_1)$);
				\tkzLabelPoints[below](H)
				\tkzDrawLines[add=0 and 0.3](O,H O,A)
				\tkzDrawSegments[dashed](h_1,O_3 O,O_1)
				\tkzDrawSegments(M,N N,P P,O_3 h_1,Q Q,M A,H O_1,O_2)
				\tkzLabelLine[below left,pos=1.3](O,A){$d$}
				\tkzLabelLine[above,pos=1.3](O,H){$d'$}
				\tkzMarkRightAngle(A,H,O)
				\tkzMarkAngles[arc=l,size=0.7cm](A,O,H)
				\tkzLabelAngle[pos=0.9](A,O,H){$\varphi$}
				\tkzMarkAngles[arc=l,size=1.1cm](N,M,Q)
				\tkzLabelAngle[pos=0.7](N,M,Q){$\alpha$}
		\end{tikzpicture}}		
	\end{itemize}
\end{dang}
