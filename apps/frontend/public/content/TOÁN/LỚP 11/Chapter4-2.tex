\section{HAI ĐƯỜNG THẲNG SONG SONG}
\subsection{KIẾN THỨC CẦN NHỚ}
\subsubsection{VỊ TRÍ TƯƠNG ĐỐI CỦA HAI ĐƯỜNG THẲNG}
Trong không gian, cho hai đường thẳng $a$ và $b$.
\begin{enumerate}[\iconMT]
	\item \indam{Các trường hợp có thể xảy ra:}
	\begin{itemize}
		\item [$\bullet$] Nếu $a$ và $b$ đồng phẳng (cùng thuộc một mặt phẳng) thì chúng có các khả năng: cắt nhau; song song nhau hoặc trùng nhau.
		\item [$\bullet$] Nếu $a$ và $b$ không đồng phẳng (không tồn tại mặt phẳng chưa được cả $a$ và $b$) thì ta nói $a$ và $b$ chéo nhau.
	\end{itemize}
\vspace{-0.5cm}
	\begin{tabular}{llll}
		\begin{tikzpicture}[scale=0.6,font=\scriptsize]
			\tkzDefPoints{0/0/A, 5/0/B, 6/3/C}
			\coordinate (D) at ($(A)+(C)-(B)$);
			\tkzDrawPolygon(A,B,C,D)
			\tkzMarkAngle[size=.85](B,A,D)
			\draw (A) node[above right]{$\alpha$};
			\tkzDefPoints{1/2/E, 4.5/0.5/F, 0.8/1.5/G, 5/2.5/H}
			\draw (E)--(F) (G)--(H);
			\tkzInterLL(E,F)(G,H) \tkzGetPoint{M}
			\tkzDrawPoints[size=5,fill=black](M)
			\tkzLabelPoints[above](M)
			\draw (F) node[above]{$a$};
			\draw (H) node[below]{$b$};
		\end{tikzpicture}
		&\begin{tikzpicture}[scale=0.6,font=\scriptsize]
			\tkzDefPoints{0/0/A, 5/0/B, 6/3/C}
			\coordinate (D) at ($(A)+(C)-(B)$);
			\tkzDrawPolygon(A,B,C,D)
			\tkzMarkAngle[size=.85](B,A,D)
			\draw (A) node[above right]{$\alpha$};
			\tkzDefPoints{0.8/1.5/G, 5/2.5/H, 1/0.5/I}
			\coordinate (K) at ($(H)+(I)-(G)$);
			\draw (G)--(H) (I)--(K);
			\draw ($(G)!0.8!(H)$) node[above]{$a$};
			\draw ($(I)!0.8!(K)$) node[above]{$b$};
		\end{tikzpicture}
		&\begin{tikzpicture}[scale=0.6,font=\scriptsize]
			\tkzDefPoints{0/0/A, 5/0/B, 6/3/C}
			\coordinate (D) at ($(A)+(C)-(B)$);
			\tkzDrawPolygon(A,B,C,D)
			\tkzMarkAngle[size=.85](B,A,D)
			\draw (A) node[above right]{$\alpha$};
			\tkzDefPoints{0.5/0.8/G, 5/2.5/H}
			\draw (G)--(H);
			\draw ($(G)!0.2!(H)$) node[above]{$a$};
			\draw ($(G)!0.2!(H)$) node[below]{$b$};
		\end{tikzpicture}
		&\begin{tikzpicture}[scale=0.6,font=\scriptsize]
			\tkzDefPoints{0/0/A, 5/0/B, 6/3/C}
			\coordinate (D) at ($(A)+(C)-(B)$);
			\tkzDrawPolygon(A,B,C,D)
			\tkzMarkAngle[size=.85](B,A,D)
			\draw (A) node[above right]{$\alpha$};
			\tkzDefPoints{0.8/0.5/G, 4.5/1.5/H, 2/3.5/E, 2.5/-0.5/F}
			\tkzInterLL(A,B)(E,F) \tkzGetPoint{K}
			\coordinate (I) at ($(E)!0.4!(F)$);
			\draw (G)--(H) (E)--(I) (K)--(F);
			\draw[dashed] (I)--(K);
			\draw ($(E)!0.1!(F)$) node[above right]{$a$};
			\draw ($(G)!0.9!(H)$) node[above]{$b$};
			\draw [fill=black] (I) circle(1pt);
			\tkzLabelPoints[left](I)
		\end{tikzpicture}\\
		\small * $a$ cắt $b$ & \small * $a$ song song $b$ & \small * $a$ trùng $b$ & \small * $a$ chéo $b$\\
		\small * Kí hiệu $a \cap b = M$ & \small * Kí hiệu $a \parallel b $  & \small * Kí hiệu 	$a \equiv b$ & \small * $a$, $b$ không điểm chung
	\end{tabular}
	\item \indam{Chú ý:}
	\begin{gachsoc}
		Cho hai đường thẳng $a$ và $b$ phân biệt.
		\begin{itemize}
			\item [$\bullet$] Khi kiểm tra hai đường thẳng $a$ và $b$ \textbf{song song} hay \textbf{cắt nhau} thì trước tiên chúng phải đồng phẳng (cùng thuộc một mặt phẳng nào đó);
			\item [$\bullet$] Khi $a$ và $b$ không có điểm chung thì chúng có thể song song hoặc chéo nhau. Vấn đề này các bạn hay bị nhầm lẫn, cần chú ý. 
		\end{itemize}
	\end{gachsoc}
\end{enumerate}

\subsubsection{CÁC ĐỊNH LÝ VÀ HỆ QUẢ CẦN NHỚ}
\begin{enumerate}[\iconMT]
	\item \indam{Định lý 1:} Trong không gian, qua một điểm không nằm trên đường thẳng cho trước, có một và chỉ một đường thẳng song song với đường thẳng đã cho.
	\item \indam{Định lý 2:} Hai đường thẳng phân biệt cùng song song với đường thẳng thứ ba thì song song với nhau.
\immini{	\item \indam{Định lý 3:} Nếu ba mặt phẳng phân biệt đôi một cắt nhau theo ba giao tuyến phân biệt thì ba giao tuyến đó hoặc đồng quy hoặc đôi một song song với nhau.}{
	\vspace{-1.5cm}
\begin{tikzpicture}[line cap=round,line join=round,x=1.0cm,y=1.0cm, font=\scriptsize]
	\begin{scope}[>=stealth,scale=0.5]
		\tikzset{label style/.style={font=\footnotesize}}
		\tkzDefPoints{0/0/A,0/5/B, 3/4/C,3/-1/D, -3/3/E, -3/-2/F, 0/4/I}
		\coordinate (J) at ($(A)!0.77!(F)$);
		\coordinate (M) at ($(A)!0.5!(D)$);
		\coordinate (c) at ($(I)!0.5!(B)$);
		\coordinate (a) at ($(I)!0.5!(J)$);
		\coordinate (b) at ($(I)!0.5!(M)$);
		\tkzDrawSegments[dashed](A,M A,J A,I)
		\tkzDrawSegments(F,J E,F E,B B,I B,C C,D D,M M,I M,J I,J)
		\tkzMarkAngles[size=0.8cm](F,E,B)
		\tkzLabelAngles[pos=0.5,rotate=30](B,E,F){$\alpha$}
		\tkzMarkAngles[size=0.8cm](B,C,D)
		\tkzLabelAngles[pos=0.5,rotate=320](D,C,B){$\beta$}
		\tkzMarkAngles[size=0.8cm](M,J,I)
		\tkzLabelAngles[pos=0.5,rotate=30](M,J,I){$\gamma$}
		%\tkzMarkAngles[size=0.5cm]
		\tkzLabelPoints[right](b,c)
		\tkzLabelPoints[left](a)
		
	\end{scope}
	\begin{scope}[xshift=5cm,>=stealth,scale=0.5]
		\tikzset{label style/.style={font=\footnotesize}}
		\tkzDefPoints{0/0/A,0/5/B, 3/4/C,3/-1/D, -3/3/E, -3/-2/F, 0/4/I}
		\coordinate (J) at ($(A)!0.6!(F)$);
		\coordinate (P) at ($(B)!0.6!(E)$);
		\coordinate (Q) at ($(B)!0.5!(C)$);
		\tkzInterLL(A,B)(Q,P)    \tkzGetPoint{I}
		\coordinate (M) at ($(A)!0.5!(D)$);
		\coordinate (c) at ($(I)!0.5!(B)$);
		\coordinate (a) at ($(I)!0.5!(J)$);
		\coordinate (b) at ($(I)!0.5!(M)$);
		\tkzDrawSegments[dashed](A,M A,J A,I)
		\tkzDrawSegments(F,J E,F E,B B,I B,C C,D D,M M,Q M,J J,P P,Q)
		\tkzMarkAngles[size=0.8cm](F,E,B)
		\tkzLabelAngles[pos=0.5,rotate=30](B,E,F){$\alpha$}
		\tkzMarkAngles[size=0.8cm](B,C,D)
		\tkzLabelAngles[pos=0.5,rotate=320](D,C,B){$\beta$}
		\tkzMarkAngles[size=0.8cm](M,J,I)
		\tkzLabelAngles[pos=0.5,rotate=30](M,J,P){$\gamma$}
		%\tkzMarkAngles[size=0.5cm]
		\tkzLabelPoints[right](b,c)
		\tkzLabelPoints[left](a)
	\end{scope} 
\end{tikzpicture}}
	\begin{luuy}
		\textbf{Hệ quả:} Nếu hai mặt phẳng phân biệt lần lượt chứa hai đường thẳng song song thì giao tuyến của chúng (nếu có) cũng song song với hai đường thẳng đó hoặc trùng với một trong hai đường thẳng đó.

	\end{luuy}
\end{enumerate}

\begin{dang} {Xét vị trí tương đối của hai đường thẳng}
	Cho hai đường thẳng $a$ và $b$ phân biệt. Xét vị trí tương đối của $a$ với $b$, ta thực hiện các bước sau:
	\begin{enumerate}[\iconMT]
		\item \indamm{Bước 1:}  Kiểm tra xem hai đường thẳng $a$ và $b$ có đồng phẳng không?
		\begin{itemize}
			\item Nếu $a$ và $b$ không đồng phẳng thì $a$ và $b$ chéo nhau.
			\item Nếu $a$ và $b$ đồng phẳng chuyển sang bước 2.
		\end{itemize}
		\item \indamm{Bước 2:} Kiểm tra xem $a$ và $b$ có điểm chung hay không?
		\begin{itemize}
			\item Nếu $a$ và $b$ không có điểm chung thì $a\parallel b$.
			\item Nếu $a$ và $b$ có một điểm chung thì $a$ và $b$ cắt nhau.
		\end{itemize}
	\end{enumerate}
\end{dang}

\begin{dang}{Chứng minh hai đường thẳng song song}
	\indamm{Phương pháp thường dùng:}
	\immini{\begin{itemize}
			\item [\ding{172}] Sử dụng các kết quả của hình học phẳng như:
			\begin{itemize}
				\item  Cặp cạnh đối hình bình hành thì song song nhau;...
				\item  Đường trung bình của tam giác thì song song và bằng nửa cạnh đáy.
			\end{itemize}
			\item [\ding{173}] Sử dụng tỉ lệ (Định lý Talet) (hình vẽ bên)
			\begin{itemize}
				\item Nếu $\dfrac{AE}{AB}=\dfrac{AF}{AC}$ thì $EF \parallel BC$.
				\item Chú ý tỉ lệ trọng tâm:  $AG=\dfrac{2}{3}AM$.
			\end{itemize}
	\end{itemize}}{\vspace{2cm}
		\begin{tikzpicture}[scale=0.8, line join=round, line cap=round]
			\tkzDefPoints{2/3/A,0/0/B,5/0/C}
			\coordinate (M) at ($(B)!0.5!(C)$);
			\tkzCentroid(A,B,C)\tkzGetPoint{G}
			\coordinate (E) at ($(A)!0.3!(B)$);
			\coordinate (F) at ($(A)!0.3!(C)$);
			\tkzDrawPolygon(A,B,C)
			\tkzDrawPoints[size=5,fill=black](A,B,C,E,F,M,G)
			\tkzDrawSegments(E,F A,M)
			\tkzLabelPoints[below](B,M,C)
			\tkzLabelPoints[above](A)
			\tkzLabelPoints[below left](G)
			\tkzLabelPoints[left](E)
			\tkzLabelPoints[right](F)
	\end{tikzpicture}}
	
\end{dang}

\begin{dang}{Xác định giao tuyến $d$ của hai mặt phẳng cắt nhau}
	\indamm{Ta thực hiện một trong hai cách sau đây:}
	\begin{itemize}
		\item [\iconCH] \indamm{Cách 1:}Tìm hai điểm chung phân biệt (đã xét ở bài học trước)
		\item [\iconCH] \indamm{Cách 2:} Tìm 1 điểm chung. Sau đó nếu hai mặt phẳng có cặp đường thẳng song song nhau thì giao tuyến $d$ sẽ đi qua điểm chung và song song (hoặc trùng) với một trong hai đường thẳng đó.
	\end{itemize}
\end{dang}	