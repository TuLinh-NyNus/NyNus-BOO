\section{GIỚI HẠN CỦA HÀM SỐ}
\subsection{TÓM TẮT LÝ THUYẾT}
\subsubsection{GIỚI HẠN HỮU HẠN CỦA HÀM SỐ TẠI MỘT ĐIỂM}
\begin{itemize}
	\item [\iconMT] \indam{Định nghĩa:} Cho khoảng $K$ chứa điểm $x_0$ và hàm số $y=f(x)$ xác định trên $K$ hoặc trên $K\setminus \left\{{x_0}\right\} $. Ta nói hàm số $y=f(x)$ có giới hạn là số $L$ khi $x$ dần tới $x_0$ nếu với dãy số $\left({x_n}\right)$ bất kì, $x_n\in K\setminus \left\{{x_0}\right\}$ và $x_n\to x_0$, ta có $f\left({x_n}\right)\to L $. Kí hiệu:
	$$\lim\limits_{x\to x_0}f(x)=L \text{ hay } f(x)\to L \text{ khi } x\to x_0 $$
	\item [\iconMT] \indam{Nhận xét:} $\lim\limits_{x\to x_0}x=x_0;$ $\lim\limits_{x\to x_0}c=c$ với $c$ là hằng số.
	\item [\iconMT] \indam{Định lí về giới hạn hữu hạn:}
	\begin{gachsoc}
		\begin{enumerate}
			\item Giả sử $\lim\limits_{x\to x_0}f(x)=L$ và $\lim\limits_{x\to x_0}g(x)=M$. Khi đó:
			\begin{listEX}[2]
				\item [\ding{172}] $ \lim\limits_{x\to x_0}\left[{f(x)+g(x)}\right]=L+M;$
				\item [\ding{173}] $ \lim\limits_{x\to x_0}\left[{f(x)-g(x)}\right]=L-M;$
				\item [\ding{174}] $ \lim\limits_{x\to x_0}\left[{f(x)\cdot g(x)}\right]=L\cdot M;$
				\item [\ding{175}] $ \lim\limits_{x\to x_0}\dfrac{f(x)}{g(x)}=\dfrac{L}{M}$ (nếu $M\ne 0$).
			\end{listEX}
			\item Nếu $f(x)\geqslant 0$ và $\lim\limits_{x\to x_0}f(x)=L$, thì $L\geqslant 0$ và $\lim\limits_{x\to x_0}\sqrt{{f(x)}}=\sqrt{L} $.
		\end{enumerate}
	\end{gachsoc}
\item [\iconMT] \indam{Giới hạn một bên:}
\begin{itemize}
	\item Cho hàm số $y=f(x)$ xác định trên $\left({x_0;b}\right) $. Số $L$ được gọi là giới hạn bên phải của hàm số $y=f(x)$ khi $x\to x_0$ nếu với dãy số $\left({x_n}\right)$ bất kì, $x_0<x_n<b$ và $x_n\to x_0$, ta có $f\left({x_n}\right)\to L $.\\
	Kí hiệu: $$\lim\limits_{x\to x_0^+}f(x)=L. $$
	\item Cho hàm số $y=f(x)$ xác định trên $\left({a;x_0}\right) $.
	Số $L$ được gọi là giới hạn bên trái của hàm số $y=f(x)$ khi $x\to x_0$ nếu với dãy số $\left({x_n}\right)$ bất kì, $a<x_n<x_0$ và $x_n\to x_0$, ta có $f\left({x_n}\right)\to L $. \\
	Kí hiệu: $$\lim\limits_{x\to x_0^-}f(x)=L. $$
\end{itemize}
\begin{luuy}
	Điều kiện để tồn tại giới hạn:
	\[\lim\limits_{x\to x_0}f\left(x\right)=L\Leftrightarrow \lim\limits_{x\to x_0^+}f\left(x\right)=\lim\limits_{x\to x_0^-}f\left(x\right)=L\]
\end{luuy}
\end{itemize}
\subsubsection{GIỚI HẠN HỮU HẠN CỦA HÀM SỐ TẠI VÔ CỰC}
\begin{itemize}
	\item [\iconMT] \indam{Định nghĩa:}
	\begin{itemize}
		\item Cho hàm số $y=f(x)$ xác định trên $\left({a;+\infty}\right) $. Ta nói hàm số $y=f(x)$ có giới hạn là số $L$ khi $x\to +\infty $ nếu với dãy số $\left({x_n}\right)$ bất kì, $x_n>a$ và $x_n\to +\infty $, ta có $f\left({x_n}\right)\to L $. Kí hiệu: $$\lim\limits_{x\to +\infty}f(x)=L. $$
		\item Cho hàm số $y=f(x)$ xác định trên $\left({-\infty;a}\right) $. Ta nói hàm số $y=f(x)$ có giới hạn là số $L$ khi $x\to -\infty $ nếu với dãy số $\left({x_n}\right)$ bất kì, $x_n<a$ và $x_n\to -\infty $, ta có $f\left({x_n}\right)\to L $. 
		Kí hiệu: $$\lim\limits_{x\to -\infty}f(x)=L. $$
	\end{itemize}
	\item [\iconMT] \indam{Chú ý:} 
	\begin{gachsoc}
	\begin{itemize}
		\item  Với $c$, $k$ là hằng số và $k$ nguyên dương, ta luôn có:
				\begin{listEX}[2]
				\item [\ding{172}] $\lim\limits_{x\to +\infty}c=c$
				\item [\ding{173}] $\lim\limits_{x\to -\infty}c=c$
				\item [\ding{174}] $\lim\limits_{x\to +\infty}\dfrac{c}{x^k}=0$
				\item [\ding{175}] $\lim\limits_{x\to -\infty}\dfrac{c}{x^k}=0$.
			\end{listEX}
	\item  Định lí về giới hạn hữu hạn của hàm số khi $x\to x_0$ vẫn còn đúng khi $x\to +\infty $ hoặc $x\to -\infty $.
\end{itemize}
\end{gachsoc}
\end{itemize}

\subsubsection{GIỚI HẠN VÔ CỰC CỦA HÀM SỐ}
\begin{itemize}
	\item [\iconMT] \indam{Định nghĩa :} Cho hàm số $y=f(x)$ xác định trên $\left({a;+\infty}\right) $. Ta nói hàm số $y=f(x)$ có giới hạn là $-\infty $ khi $x\to +\infty $ nếu với dãy số $\left({x_n}\right)$ bất kì, $x_n>a$ và $x_n\to +\infty $, ta có $f\left({x_n}\right)\to -\infty $.	Kí hiệu: $\lim\limits_{x\to +\infty}f(x)=-\infty $.
	\item [\iconMT] \indam{Nhận xét:}   $\lim\limits_{x\to +\infty}f(x)=+\infty \Leftrightarrow \lim\limits_{x\to +\infty}\left({-f(x)}\right)=-\infty $.
	\item [\iconMT] \indam{Một vài giới hạn đặc biệt:}
	\begin{gachsoc}
		\begin{itemize}
			\item $\lim\limits_{x\to +\infty}x^k=+\infty $, với $k$ nguyên dương.
			\item $\lim\limits_{x\to -\infty}x^k=\left\{\begin{aligned}& +\infty \text{ nếu } k \text{ chẵn} \\
				& -\infty \text{ nếu } k \text{ lẻ}. 
			\end{aligned}\right. $
		\end{itemize}
	\end{gachsoc}
\item [\iconMT] \indam{Một vài quy tắc về giới hạn vô cực:}
\begin{itemize}
	\item [$\bullet$] Quy tắc tìm giới hạn của tích $f(x)\cdot g(x)$
	\begin{center}
		\begin{tabular}{|c|c|c|}
			\hline
			$\lim\limits_{x\to x_0}f(x)=L$ & $\lim\limits_{x\to x_0}g(x)$   & $\lim\limits_{x\to x_0}\left[{f(x)\cdot g(x)}\right]$ \\
			\hline
			$L>0$   & $+\infty $   & $+\infty $ \\
			\hline
			$L>0$    & $-\infty $    & $-\infty $ \\
			\hline
			$L<0$   & $+\infty $   & $-\infty $ \\
			\hline
			$L<0$    & $-\infty $    & $+\infty $ \\
			\hline
		\end{tabular}
	\end{center}
	\item [$\bullet$] Quy tắc tìm giới hạn của thương $\dfrac{f(x)}{g(x)}$
	\begin{center}
		\begin{tabular}{|c|c|c|c|}
			\hline
			$\lim\limits_{x\to x_0}f(x)=L$ & $\lim\limits_{x\to x_0}g(x)$   & Dấu của $g(x)$ & $\lim\limits_{x\to x_0}\dfrac{f(x)}{g(x)}$\\
			\hline
			$L$   & $\pm \infty $   & Tùy ý & $0$\\
			\hline
			$L>0$   & $0$   & $+$ & $+\infty $\\
			\hline
			$L>0$    & $0$   & $-$ & $-\infty $\\
			\hline
			$L<0$   & $0$   & $+$ & $-\infty $\\
			\hline
			$L<0$    & $0$    & $-$ & $+\infty $\\
			\hline
		\end{tabular}
	\end{center}  
Các quy tắc trên vẫn đúng khi thay $x \to x_0$ bởi  $x \to x_0^+$, $x \to x_0^-$ hoặc $x \to +\infty$, $x \to -\infty$.
\end{itemize}
\end{itemize}

\begin{dang}{Giới hạn của hàm số khi $x \to x_0$. Khử dạng vô định $\displaystyle\frac{0}{0}$ }
	Xét giới hạn $\lim\limits_{x\to x_0}\dfrac{f(x)}{g(x)}$.
	\begin{enumerate}[\iconMT]
		\item \indamm{Phương pháp giải:} Thay $x_0$ vào $\dfrac{f(x)}{g(x)}$ để kiểm tra, sẽ có một trong các trường hợp:
		\begin{itemize}
			\item [\ding{172}] Tử số $f(x_0)=a$ và mẫu số $g(x_0)=b \ne 0$,  ta suy ra luôn kết quả $$\lim\limits_{x\to x_0}\dfrac{f(x)}{g(x)}=\dfrac{f(x_0)}{g(x_0)}=\dfrac{a}{b}.$$
			\item [\ding{173}]  Cả tử số và mẫu số đều bằng 0 hay $f(x_0)=g(x_0)=0$, ta xem đây là dạng vô định $\displaystyle\frac{0}{0}$. Khử dạng vô định này bằng cách phân tích nhân tử $x-x_0$.\\
			Phân tích $f(x)=(x-x_0) \cdot f_1(x)$ và $g(x)=(x-x_0) \cdot g_1(x)$. Khi đó
			$$\lim\limits_{x\to x_0}\frac{f(x)}{g(x)}=\lim\limits_{x\to x_0}\frac{(x-x_0)f_1(x)}{(x-x_0)g_1(x)}=\lim\limits_{x\to x_0}\frac{f_1(x)}{g_1(x)} \quad (1)$$
			Ta tiếp tục tính giới hạn (1).
			\item [\ding{174}] Tử số $f(x_0) \ne 0$ và mẫu số $g(x_0)=0$. Ta áp dụng các định lý liên quan đến giới hạn vô cực để tìm kết quả.
		\end{itemize}
	\end{enumerate}
	\begin{luuy}
		Một số cách phân tích nhân tử thường dùng:
		\begin{itemize}
			\item [$\bullet$] Nếu $f(x)=ax^2+bx+c$ có hai nghiệm $x_1$, $x_2$ thì $f(x)=a(x-x_1)(x-x_2)$.
			\item [$\bullet$] Nếu $f(x)$ là một đa thức bậc ba, bậc bốn,...ta có thể dùng phương pháp chia đa thức.
			\item [$\bullet$] Nếu $f(x)$ là biểu thức chứa căn, ta dùng cách nhân lượng liên hợp.	
		\end{itemize}
	\end{luuy}
\end{dang}

\begin{dang}{Giới hạn của hàm số khi $x \to \pm \infty$. Khử dạng vô định $\dfrac{\infty }{\infty };\infty -\infty ;0\cdot\infty$}
\end{dang}


\begin{dang}{Giới hạn một bên. Sự tồn tại giới hạn}
	Phương pháp tính $\lim\limits_{x\to x_0^-}f(x)$ và $\lim\limits_{x\to x_0^+}f(x)$ hoàn toàn tương tự như bài toán tính $\lim\limits_{x\to x_0} f(x)$.
	\begin{luuy}
		Lưu ý: $\lim\limits_{x\to x_0} f(x)=L$ khi và chỉ khi $\lim\limits_{x\to x_0^-}f(x)=\lim\limits_{x\to x_0^+}f(x)=L$.
	\end{luuy}
\end{dang}

