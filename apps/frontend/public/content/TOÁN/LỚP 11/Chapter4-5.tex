\section{PHÉP CHIẾU PHẲNG SONG SONG}
\subsection{KIẾN THỨC CẦN NHỚ}
\subsubsection{ĐỊNH NGHĨA}
Cho mặt phẳng $(\alpha)$ và đường thẳng $\Delta$ cắt $(\alpha)$. Với mỗi điểm $M$ trong không gian ta xác định điểm $M'$ như sau:
\immini{\begin{itemize}
		\item Nếu $M$ thuộc $\Delta$ thì $M'$ là giao điểm của $\Delta$ và $(\alpha)$.
		\item Nếu $M$ không thuộc $\Delta$ thì $M'$ là giao điểm của $(\alpha)$ và đường thẳng qua $M$ song song $\Delta$.
		\item Điểm $M'$ gọi là hình chiếu song song của $M$ trên $(\alpha)$ theo phương $\Delta$.
		\item Phép đặt tương ứng mối điểm $M$ với hình chiếu $M'$ của nó được gọi là \indamm{phép chiếu song song} lên $(\alpha)$ theo phương $\Delta$.
		\item Mặt phẳng $(\alpha)$ gọi là mặt phẳng chiếu; phương $\Delta$ gọi là \indamm{phương chiếu.}
	\end{itemize}}{
\begin{tikzpicture}[scale=0.9]
	\tkzInit[xmin=-3.5,xmax=5,ymin=-3, ymax=4] \tkzClip[space=.1]
	\tkzDefPoints{-3/0/A, 3/0/B, 4/2/C, -2/2/D, 0/1/Q, -1/2/P, 0/4/M, 3/1/x}
	%\tkzDrawCircle(O,A)
	\tkzDefLine[parallel=through M,K=1](P,Q)\tkzGetPoint{N}
	\tkzInterLL(M,N)(Q,x) \tkzGetPoint{M'}
	\tkzInterLL(A,D)(P,Q) \tkzGetPoint{y}
	\tkzInterLL(A,B)(P,Q) \tkzGetPoint{y'}
	\tkzInterLL(M,M')(B,C) \tkzGetPoint{t}
	\tkzDefLine[parallel=through y',K=1](P,Q)\tkzGetPoint{z}
	\tkzDefLine[parallel=through t,K=1](P,Q)\tkzGetPoint{t'}
	%\tkzInterLC(C,M)(O,A) \tkzGetPoints{E}{C}
	%\tkzInterLC(A,B)(A,C) \tkzGetPoints{y}{P}
	%\tkzInterLC(C,P)(O,A) \tkzGetPoints{C}{Q}
	%\tkzTangent[at=Q](O) \tkzGetPoint{x}
	\tkzDrawSegments(A,B B,C C,D D,A P,Q M,N P,y N,M' y',z t,t')
	\tkzDrawSegments[dashed](Q,y' M',t)
	%\tkzDrawLine[add = 1 and 1](Q,x)
	\tkzDrawPoints[fill=white](M,M')
	\tkzLabelSegment[above right](P,y){$\Delta$}
	%\tkzLabelPoints[above](C)
	%\tkzLabelPoint[left](M)
	\tkzLabelPoints[below left](M,M')
	\tkzMarkAngle[size=0.6cm,opacity=.5](B,A,D)
	\tkzLabelAngle[pos =0.4](D,A,B){$\alpha$}
	%\tkzLabelPoints[below left](O,P)
\end{tikzpicture}}
\subsubsection{TÍNH CHẤT}
Phép chiếu song song có các tính chất sau:
\begin{itemize}
	\item[\ding{172}] Biến ba điểm thẳng hàng thành ba điểm thẳng hàng.
	\item[\ding{173}] Biến đường thẳng  thành đường thẳng , biến tia thành tia, đoạn thẳng thành đoạn thẳng.
	\item[\ding{174}] Biến hai đường thẳng song song thành hai đường thẳng song song  hoặc trùng nhau.
	\item[\ding{175}] Giữ nguyên tỉ số độ dài của hai đoạn thẳng cùng nằm trên một đường thẳng hoặc nằm trên hai đường thẳng song song.
\end{itemize}

\subsubsection{HÌNH BIỂU DIỄN CỦA MỘT HÌNH KHÔNG GIAN}
\begin{itemize}
	\item[\ding{172}] Hình biểu diễn của hình trong không gian là hình chiếu song song của hình đó trên một mặt phẳng theo một phương chiếu nào đó hoặc hình đồng dạng với hình chiếu đó.
	\item[\ding{173}] Hình biểu diễn của một hình không gian ( trong trường hợp hình phẳng nằm trong mặt phẳng không song song với phương chiếu) có các tính chất sau:
		\begin{itemize}
			\item Hình biểu diễn của một tam giác là một tam giác.
			\item Hình biểu diễn của hình chữ nhật, hình vuông, hình thoi, hình bình hành là hình bình hành.
			\item Hình biểu diễn của hình thang $ABCD$ với $AB\parallel CD$ là một hình thang $A'B'C'D'$ với $A'B'\parallel C'D'$ thoả mãn $\dfrac{AB}{CD}=\dfrac{A'B'}{C'D'}$.
			\item Hình biểu diễn của hình tròn là hình elip.
		\end{itemize}
\end{itemize}
