\section{ĐẠO HÀM CẤP HAI}
\subsection{KIẾN THỨC CẦN NHỚ}
\subsubsection{Định nghĩa}
	Giả sử hàm số $y = f(x)$ có đạo hàm tại mỗi điểm $x \in (a; b)$. Khi đó hệ thức $y'= f'(x)$ xác định một hàm số mới trên khoảng $(a; b)$. Nếu hàm số $y' = f'(x)$ lại có đạo hàm tại $x$ thì ta gọi đạo hàm của $y'$ là đạo hàm cấp hai của hàm số $y = f(x)$.
	\begin{gachsoc}
	\begin{itemize}
		\item [$\bullet$] Kí hiệu $y''$ hoặc $f''(x)$
		\item [$\bullet$] Công thức tính $f''(x)=[f'(x)]'$.
	\end{itemize}
\end{gachsoc}
\subsubsection{Ý nghĩa cơ học của đạo hàm cấp hai}
	Một chuyển động có phương trình $s=f(t)$ thì đạo hàm cấp hai (nếu có) của hàm số $f(t)$ là gia tốc tức thời của chuyển động. Ta có 
		\boxmini{$a(t)=f''(t)$}