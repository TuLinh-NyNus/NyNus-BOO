\section{GIỚI HẠN CỦA DÃY SỐ}
\subsection{TÓM TẮT LÝ THUYẾT}
\subsubsection{GIỚI HẠN HỮU HẠN CỦA DÃY SỐ}
\begin{itemize}
	\item [\iconMT] \indam{Định nghĩa 1:} Dãy số $(u_n)$ có giới hạn là $0$ khi $n$ dần tới dương vô cực nếu $|u_n|$ có thể nhỏ hơn một số dương bé tuỳ ý, kể từ một số hạng nào đó trở đi.
	Kí hiệu
	\boxmini{$\lim\limits_{n \to +\infty}u_n=0$}
	\item [\iconMT]  \indam{Định nghĩa 2:} Dãy số $(u_n)$ có giới hạn là $a$ nếu $\lim\limits_{n \to +\infty}(u_n-a)=0$. Kí hiệu
	\boxmini{$\lim\limits_{n \to +\infty}u_n=a$}	
	\begin{luuy}
		Ta có thể viết $\lim u_n=a$ thay cho cách viết $\lim\limits_{n \to +\infty}u_n=a$ (không cần viết chỉ số $n \to +\infty$)
	\end{luuy}
	\item [\iconMT] \indam{Một vài giới hạn đặc biệt:} (\textit{có thể xem như công thức})
	\begin{gachsoc}
		\begin{itemize}
			\begin{multicols}{3}
				\item $\lim \dfrac{1}{n}=0$;
				\item $\lim \dfrac{1}{n^k}=0$, với $k\in \mathbb{N}^*$;
				\item $\lim \dfrac{1}{\sqrt{n}}=0$;
				\item $\lim C=C,\, \forall C\in \mathbb{R}$;
				\item $\lim \dfrac{1}{q^n}=0$, với $|q|>1$;
				\item $\lim q^n=0$, nếu $|q|<1$.
			\end{multicols}
		\end{itemize}
	\end{gachsoc}
\end{itemize}
\subsubsection{ĐỊNH LÝ VỀ GIỚI HẠN HỮU HẠN CỦA DÃY SỐ}
\begin{itemize}
	\item [\iconMT] Nếu $\lim u_n=a$ và $\lim v_n=b$ thì ta có:
	\begin{gachsoc}
		\begin{itemize}
			\begin{multicols}{3}
				\item $\lim \left(u_n \pm v_n\right)=a+b$;
				\item $\lim \left( u_n.v_n\right)=a.b$;
				\item $\lim \left(\dfrac{u_n}{v_n}\right)=\dfrac{a}{b}$, với $b\ne 0$;
				\item $\lim \sqrt{u_n}=\sqrt{a}$, với $a\ge 0$;
				\item $\lim |u_n|=|a|$;
				\item $\lim \left(k. u_n\right)=k. a$ $(k\in \mathbb{R})$.
			\end{multicols}
		\end{itemize}
	\end{gachsoc}
	\item [\iconMT] Định lý "kẹp giữa":
	\begin{gachsoc}
		\begin{itemize}
			\item Nếu $0\le \left|u_n\right|\le v_n$, $\forall n\in \mathbb{N}^*$ và $\lim v_n=0$ thì $\lim u_n=0$.
			\item Nếu $w_n\le u_n\le v_n$, $\forall n\in \mathbb{N}^*$ và $\lim w_n=\lim v_n=a$ thì $\lim u_n=a$.
		\end{itemize}
	\end{gachsoc}
\end{itemize}

\subsubsection{TỔNG CỦA CẤP SỐ NHÂN LÙI VÔ HẠN}
\begin{itemize}
	\item [\iconMT] Cấp số nhân vô hạn $(u_n)$ có công bội $q$ thoả mãn $|q|<1$ được gọi là \textit{cấp số nhân lùi vô hạn}.
	\item [\iconMT] Cho cấp số nhân lùi vô hạn $(u_n)$, ta có tổng của cấp số nhân lùi vô hạn đó là $$S=u_1+u_2+u_3+...+u_n+...=\dfrac{u_1}{1-q}, (|q|<1)$$
\end{itemize}

\subsubsection{GIỚI HẠN VÔ CỰC CỦA DÃY SỐ}
\begin{itemize}
	\item [\iconMT] \indam{Định nghĩa 1:} Ta nói dãy số $(u_n)$ có giới hạn $+\infty$ khi $n\to +\infty$, nếu $u_n$ có thể lớn hơn một số dương bất kì, kể từ một số hạng nào đó trở đi.	Kí hiệu
	\boxmini{$\lim u_n=+\infty$}
	\item [\iconMT] \indam{Định nghĩa 2:} Ta nói dãy số $(u_n)$ có giới hạn $-\infty$ khi $n\to +\infty$, nếu $\lim (-u_n)=+\infty$. Kí hiệu
	\boxmini{$\lim u_n=-\infty$}
	\item [\iconMT] \indam{Một số giới hạn đặc biệt:}
		\begin{listEX}[2]
			\item [\ding{172}] $\lim n^k=+\infty$, với $k \in \mathbb{N^*}$.
			\item [\ding{173}] $\lim q^n=+\infty$, với $q>1$.
		\end{listEX} 
	\item [\iconMT] \indam{Một số quy tắc tính giới hạn vô cực:}
	\begin{itemize}
		\item [\ding{172}] Quy tắc tìm giới hạn của tích $u_n\cdot v_n$
		\begin{center}
			\begin{tabular}{|c|c|c|}
				\hline
				$\lim u_n=L$ & $\lim v_n = \infty$   & $\lim \left[{u_n\cdot v_n}\right]$ \\
				\hline
				$L>0$   & $+\infty $   & $+\infty $ \\
				\hline
				$L>0$    & $-\infty $    & $-\infty $ \\
				\hline
				$L<0$   & $+\infty $   & $-\infty $ \\
				\hline
				$L<0$    & $-\infty $    & $+\infty $ \\
				\hline
			\end{tabular}
		\end{center}
		\item [\ding{173}]Quy tắc tìm giới hạn của thương $\dfrac{u_n}{v_n}$
		\begin{center}
			\begin{tabular}{|c|c|c|c|}
				\hline
				$\lim u_n=L$ & $\lim v_n$   & Dấu của $v_n$ & $\lim \dfrac{u_n}{v_n}$\\
				\hline
				$L$   & $\pm \infty $   & Tùy ý & $0$\\
				\hline
				$L>0$   & $0$   & $+$ & $+\infty $\\
				\hline
				$L>0$    & $0$   & $-$ & $-\infty $\\
				\hline
				$L<0$   & $0$   & $+$ & $-\infty $\\
				\hline
				$L<0$    & $0$    & $-$ & $+\infty $\\
				\hline
			\end{tabular}
		\end{center}  
	\end{itemize}
\end{itemize}

\begin{dang}{Khử vô định dạng $\dfrac{\infty}{\infty}$}
	Xét giưới hạn: $\lim\dfrac{u_n}{v_n}$.
	\begin{enumerate}[\iconMT]
		\item \indamm{Phương pháp giải:}
		\begin{itemize}
			\item Đặt nhân tử $n^k$ có tính "quyết định $\infty$" ở tử và mẫu.
			\item Khử bỏ $n^k$, đưa giới hạn về dạng xác định được.
			\item Áp dụng định lý về giới hạn hữu hạn để tính kết quả.
		\end{itemize}
		\item \indamm{Chú ý:} Trong trường hợp hàm mũ, ta đặt đại lượng "quyết định $\infty$" có dạng $a^n$.
	\end{enumerate}
\end{dang}

\begin{dang}{Khử vô định dạng $\infty -\infty $}
	Xét các giưới hạn dạng: \fbox{$\lim\left( \sqrt{u_n}-v_n\right)$} hoặc \fbox{$\lim\left( \sqrt{u_n}-\sqrt{v_n}\right)$}.
	\begin{enumerate}[\iconMT]
		\item \indamm{Phương pháp giải:}
		\begin{itemize}
			\item Nhân thêm lượng liên hợp:
			\vskip 0.3 cm
			\begin{itemize}
				\item [\ding{172}]$\lim\left( \sqrt{u_n}-v_n\right)=\lim \dfrac{\left(\sqrt{u_n}-v_n\right)\left( \sqrt{u_n}+v_n\right)}{\sqrt{u_n}+v_n}=\lim \dfrac{u_n-v_n^2}{\sqrt{u_n}+v_n}$
				\vskip 0.3 cm
				\item [\ding{173}] $\lim\left( \sqrt{u_n}-\sqrt{v_n}\right)=\lim \dfrac{\left(\sqrt{u_n}-\sqrt{v_n}\right)\left( \sqrt{u_n}+\sqrt{v_n}\right)}{\sqrt{u_n}+\sqrt{v_n}}=\lim \dfrac{u_n-v_n}{\sqrt{u_n}+\sqrt{v_n}}$
				\vskip 0.3 cm
			\end{itemize} 
			\item Biến đổi biểu thức cần tính giới hạn về Dạng 1 (phân thức, đặt $n^k$) 
		\end{itemize}
		\item \indamm{Chú ý:} Đôi khi, ta còn sử dụng liên hợp bậc ba để giải các bài toán tính giới hạn của những dãy số mà công thức tổng quát của nó có chứa ẩn trong dấu căn bậc ba. 
		$$\sqrt[3]{A}-B=\dfrac{\left( \sqrt[3]{A}-B\right) \left(\sqrt[3]{A}^2+\sqrt[3]{A}\cdot B + B^2 \right) }{\sqrt[3]{A}^2+\sqrt[3]{A}\cdot B + B^2}=\dfrac{A-B^3}{\sqrt[3]{A}^2+\sqrt[3]{A}\cdot B + B^2}$$
	\end{enumerate}
\end{dang}

\begin{dang}{Một số quy tắc tính giới hạn vô cực}
	\begin{itemize}
		\item [\ding{172}] Quy tắc tìm giới hạn của tích $u_n\cdot v_n$
		\begin{center}
			\begin{tabular}{|c|c|c|}
				\hline
				$\lim u_n=L$ & $\lim v_n = \infty$   & $\lim \left[{u_n\cdot v_n}\right]$ \\
				\hline
				$L>0$   & $+\infty $   & $+\infty $ \\
				\hline
				$L>0$    & $-\infty $    & $-\infty $ \\
				\hline
				$L<0$   & $+\infty $   & $-\infty $ \\
				\hline
				$L<0$    & $-\infty $    & $+\infty $ \\
				\hline
			\end{tabular}
		\end{center}
		\item [\ding{173}]Quy tắc tìm giới hạn của thương $\dfrac{u_n}{v_n}$
		\begin{center}
			\begin{tabular}{|c|c|c|c|}
				\hline
				$\lim u_n=L$ & $\lim v_n$   & Dấu của $v_n$ & $\lim \dfrac{u_n}{v_n}$\\
				\hline
				$L$   & $\pm \infty $   & Tùy ý & $0$\\
				\hline
				$L>0$   & $0$   & $+$ & $+\infty $\\
				\hline
				$L>0$    & $0$   & $-$ & $-\infty $\\
				\hline
				$L<0$   & $0$   & $+$ & $-\infty $\\
				\hline
				$L<0$    & $0$    & $-$ & $+\infty $\\
				\hline
			\end{tabular}
		\end{center}  
	\end{itemize}
	
\end{dang}

\begin{dang}{Tổng của cấp số nhân lùi vô hạn}
	\begin{itemize}
		\item [\iconMT] Cấp số nhân vô hạn $(u_n)$ có công bội $q$ thoả mãn $|q|<1$ được gọi là \textit{cấp số nhân lùi vô hạn}.
		\item [\iconMT] Cho cấp số nhân lùi vô hạn $(u_n)$, Xét $S=u_1+u_2+u_3+...+u_n+...$. Khi đó, ta có công thức tính  \boxmini{$S=\dfrac{u_1}{1-q}$}
	\end{itemize}
\end{dang}