\section{PHÉP TÍNH LÔGARIT}
\subsection{LÝ THUYẾT CẦN NHỚ}
\subsubsection{Định nghĩa và tính chất}
	\begin{itemize}
			\item [\iconMT]\indam{Định nghĩa:} Cho hai số dương $a, b$ với $a\neq 1$. Số $\alpha$ thỏa mãn đẳng thức $a^{\alpha}=b$ được gọi là lôgarit cơ số $a$ của $b$ và kí hiệu là $\log_ab$.
			$$\alpha=\log_ab \Leftrightarrow a^{\alpha}=b.$$
			\item [\iconMT]\indam{Tính chất:} Cho hai số dương $a, b$ với $a\neq 1$, ta có tính chất sau:
			\begin{boxdn}
			\begin{listEX}[2]
				\item [\ding{172}] \,\,$\log_a1=0$.
				\item [\ding{173}] \,\,$\log_aa=1$.
				\item [\ding{174}] \,\,$a^{\log_ab}=b$.
				\item [\ding{175}] \,\,$\log_aa^{\alpha}=\alpha.$
			\end{listEX}
		\end{boxdn}	
	\end{itemize}

\subsubsection{Các công thức lôgarit cần nhớ}
		Cho các số dương $a$, $b$, $b_1$, $b_2$,...$b_n$ với $a\neq 1$, ta có các quy tắc sau:
		\begin{itemize}
			\item [\iconMT] \indam{Công thức biến đổi tích thương:}
			\begin{boxdn}
			\begin{listEX}[2]
				\item [\ding{172}] \,\,$\log_a\big(b_1b_2\big)=\log_ab_1+\log_ab_2$;
				\item [\ding{173}] \,\,$\log_a\left( \dfrac{b_1}{b_2}\right)=\log_ab_1-\log_ab_2$.
			\end{listEX}
		\end{boxdn}
		\begin{luuy}
			\underline{Ghi nhớ}: Lô ga của một tích thành một tổng; Lô ga của một thương thành một hiệu.
		\end{luuy}
			\item [\iconMT] \indam{Công thức biến đổi số mũ:}
			\begin{boxdn}
			\begin{listEX}[2]
			\item [\ding{172}] \,\,$\log_ab^{m}=m \cdot\log_ab$.
			\item [\ding{173}] \,\, $\log_{a}{\left( \dfrac{1}{b}\right) }=-\log_ab$.
			\end{listEX}
			\end{boxdn}
		\begin{luuy}
			\underline{Ghi nhớ}: Với điều kiện $b \ne 0$ thì	$\log_ab^{2n}=2n\cdot \log_a|b|$.
		\end{luuy}
		\item [\iconMT] \indam{Công thức đổi cơ số:}
		\begin{boxdn}
		\begin{listEX}
			\item [\ding{172}] \,\, $\log_ab=\dfrac{1}{\log_ba}$, với $b\neq 1$
			\item [\ding{173}] \,\, $\log_ab=\dfrac{\log_cb}{\log_ca}$, với $a, b, c>0$ và $a\neq 1, c\neq 1$
			\item [\ding{174}] \,\,  $\log_ab \cdot  \log_bc = \log_ac$, với $a, b, c>0$ và $a\neq 1, b\neq 1$
		\end{listEX}
	\end{boxdn}
	\end{itemize}

\subsubsection{Lôgarít thập phân và lôgarit tự nhiên}
	\begin{itemize}
			\item [\iconMT]  \indam{Lôgarit thập phân:}Lôgarit cơ số $10$ gọi là lôgarit thập phân. [\faCheck] $\log_{10}N$, ($N>0$) được viết là $\log N$ hay $\lg N$.
			\item [\iconMT] \indam{Lôgarít tự nhiên:} Lôgarit cơ số $e$ gọi là lôgarit tự nhiên.  [\faCheck] $\log_eN$, ($N>0$) được viết là $\ln N$.
	\end{itemize}