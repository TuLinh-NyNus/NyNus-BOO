\section{GÓC LƯỢNG GIÁC}
\subsection{LÝ THUYẾT CẦN NHỚ}

\subsubsection{GÓC LƯỢNG GIÁC}
\begin{itemize}
	\item[\iconMT] \indam{Góc lượng giác và số đo của góc lượng giác:} Trong mặt phẳng, cho hai tia $Oa$, $Ob$. Xét tia $Om$ cùng nằm trong mặt phẳng này. 
	\item [] \indamm{Ghi nhớ 1:}
		\begin{itemize}
			\item 	Nếu tia $Om$ quay quanh điểm $O$, theo một chiều nhất định từ $Oa$ đến $Ob$, thì ta nói nó quét một góc lượng giác với tia đầu $Oa$, tia cuối $Ob$ và kí hiệu là $(Oa, Ob)$.
		\end{itemize}
	\hspace*{1.5cm}
	\begin{tikzpicture}[line width=1pt, >=Stealth]
		\def\bk{0.5}
		\def\dr{2.3}
		\draw (0,0)coordinate(O)node[below left=-1pt]{$O$} ;
		\draw (O) -- (0:\dr)node[pos=0.8, below]{$a$};
		\draw (O) --++(60:\dr)coordinate(M)node[left]{$b$};
		\draw [dashed,cyan] (O) --++(20:\dr)coordinate(M)node[above]{$m$};
		\foreach \a in {0.2}{
			\draw [->, red] ($(O)+(0:\bk+\a)$) arc (0:60:{\bk+\a} and {\bk+ \a})
			;
		}
		\draw [->,cyan] ($(O)!0.6!(M)$) arc (20:50:{0.5*\dr} and  {0.5*\dr})node[pos=0.5, above right=-1pt]{$+$} ;
	\end{tikzpicture} \hspace{1.7cm}
	\begin{tikzpicture}[line width=1pt, >=Stealth]
		\def\bk{0.5}
		\def\dr{2.3}
		\draw (0,0)coordinate(O)node[below left=-2pt]{$O$} ;
		\foreach \a in {0.2}{
			\draw [->, red] ($(O)+(0:\bk+\a)$) arc (0:270:{\bk+\a} and {\bk+ \a})
			($(O)+(270:\bk+\a)$) arc (270:360:{\bk+\a+0.2} and {\bk +\a}) arc (360:415:{\bk+\a} and {\bk +\a})
			;
		}
		\draw (O) -- (0:\dr)node[pos=0.8, below]{$a$};
		\draw (O) -- (45:\dr)node[pos=0.8, below]{$b$};
		\draw [dashed,cyan] (O) --++(80:\dr)coordinate(M)node[above]{$m$};
		\draw [->,cyan] ($(O)!0.5!(M)$) arc (30:70:{0.7*\dr} and  {0.2*\dr})node[pos=0.5, above]{$+$} ;
	\end{tikzpicture}
	\hspace{1.7cm}
	\begin{tikzpicture}[line width=1pt, >=Stealth]
		\def\bk{0.5}
		\def\dr{2.3}
		\draw (0,0)coordinate(O)node[below left=-2pt]{$O$} ;
		\foreach \a in {0.2}{
			\draw [->, red] ($(O)+(0:\bk+\a)$) arc (0:-315:{\bk+\a} and {\bk+ \a})
			;
		}
		\draw (O) -- (0:\dr)node[pos=0.8, below]{$a$};
		\draw (O) -- (45:\dr)node[pos=0.8, below]{$b$};
		\draw [dashed,cyan] (O) --++(120:\dr)coordinate(M)node[above]{$m$};
		\draw [->,cyan] ($(O)!0.5!(M)$) arc (120:70:{0.7*\dr} and  {1*\dr})node[pos=0.5, above]{$-$} ;
	\end{tikzpicture}
	\item [] \indamm{Ghi nhớ 2:}
		\begin{itemize}
			\item Khi tia $Om$ quay một góc $\alpha^\circ$, ta nói số đo của góc lượng giác $(Oa, Ob)$ bằng $\alpha^\circ$, kí hiệu $\text{sđ}(Oa, Ob)=\alpha^\circ$ hoặc $(Oa, Ob)=\alpha^\circ$.
			\item Mỗi góc lượng giác gốc $O$ được xác định bởi tia đầu $Oa$, tia cuối $Ob$ và số đo $\alpha^\circ$ của nó.\\
			\begin{tikzpicture}[line width=1pt, >=Stealth]
				\def\bk{0.5}
				\def\dr{2.3}
				\draw (0,0)coordinate(O)node[below left]{$O$} ;
				\foreach \a in {0.2}{
					\draw [->, red] ($(O)+(0:\bk+\a)$) arc (0:270:{\bk+\a} and {\bk+ \a})
					($(O)+(270:\bk+\a)$) arc (270:360:{\bk+\a +0.15} and {\bk +\a})
					;
				}
				\draw (O) -- (0:\dr)node[pos=0.8, below]{$a, b$};
				\draw [dashed,cyan] (O) --++(30:\dr)coordinate(M)node[above]{$m$};
				\draw [->,cyan] ($(O)!0.5!(M)$) arc (30:70:{0.5*\dr} and  {0.5*\dr})node[pos=0.5, above right=-1pt]{$+$} ;
				\node[below right] at (-1,-1) {\footnotesize$\text{sđ}(Oa, Ob)=360^\circ$};
			\end{tikzpicture} \hspace{1cm}
			\begin{tikzpicture}[line width=1pt, >=Stealth]
				\def\bk{0.5}
				\def\dr{2.1}
				\path
				(0:0) coordinate (O)
				(0:\dr) coordinate (A)
				(45:\dr) coordinate (B)
				;
				\draw pic["\tiny$45^\circ$",draw,angle eccentricity=1.8,angle radius=0.3cm]{angle=A--O--B};
				\draw (0,0)coordinate(O)node[below left=-2pt]{$O$} ;
				\foreach \a in {0.3}{
					\draw [->, red] ($(O)+(0:\bk+\a)$) arc (0:270:{\bk+\a} and {\bk+ \a})
					($(O)+(270:\bk+\a)$) arc (270:360:{\bk+\a+0.2} and {\bk +\a}) arc (360:415:{\bk+\a} and {\bk +\a})
					;
				}
				\draw (O) -- (0:\dr)node[pos=0.8, below]{$a$};
				\draw (O) -- (45:\dr)node[pos=0.8, below]{$b$};
				\draw [dashed,cyan] (O) --++(80:\dr)coordinate(M)node[above]{$m$};
				\draw [->,cyan] ($(O)!0.5!(M)$) arc (30:70:{0.7*\dr} and  {0.2*\dr})node[pos=0.5, above]{$+$} ;
				\node[below right] at (-1,-1) {\footnotesize$\text{sđ}(Oa, Ob)=405^\circ$};
			\end{tikzpicture}
			\hspace{1cm}
			\begin{tikzpicture}[line width=1pt, >=Stealth]
				\def\bk{0.5}
				\def\dr{2.3}
				\draw (0,0)coordinate(O)node[above]{$O$} ;
				\foreach \a in {0.2}{
					\draw [ red] ($(O)+(0:\bk+\a)$) arc (0:-270:{\bk+\a} and {\bk+ \a})
					($(O)+(-270:\bk+\a)$) arc (-270:-360:{\bk+\a +0.15} and {\bk +\a})
					;
				}
				\foreach \a in {0.35}{
					\draw [->, red] ($(O)+(0:\bk+\a)$) arc (0:-90:{\bk+\a} and {\bk+ \a})
					($(O)+(-90:\bk+\a)$) arc (-90:-180:{\bk+\a} and {\bk +\a})
					;
				}
				\draw (O) -- (0:\dr)node[pos=0.8, below]{$a$};
				\draw [dashed,cyan] (O) --++(-30:\dr)coordinate(M)node[above]{$m$};
				\draw [->,cyan] ($(O)!0.5!(M)$) arc (-30:-70:{0.5*\dr} and  {0.5*\dr})node[pos=0.5, below right=-1pt]{$-$} ;
				\draw [red] (O)--++(180:\dr)node[below]{$b$};
				\node[below right] at (-1,-1.6) {\footnotesize$\text{sđ}(Oa, Ob)=-540^\circ$};
			\end{tikzpicture}
			\item Số đo của các góc lượng giác có cùng tia đầu $Oa$ và tia cuối $Ob$ sai khác nhau một bội nguyên của $360^{\circ}$ nên có công thức tổng quát là 
			\boxminit{$\text{sđ}(Oa, Ob)=\alpha^{\circ}+k 360^{\circ}, \text{ với } k \in \mathbb{Z}$}
		\end{itemize}
	\item[\iconMT] \indam{Hệ thức Chasles:} Với ba tia $Oa$, $Ob$, $Oc$ bất kì, ta có
	$$\text{sđ}(Oa,Ob)+\text{sđ}(Ob,Oc)=\text{sđ}(Oa, Oc)+k360^{\circ} \quad \text{với } k \in \mathbb{Z}.$$
\end{itemize}

\subsubsection{ĐƠN VỊ ĐO GÓC VÀ ĐỘ DÀI CUNG TRÒN}
\begin{itemize}
	\item[\iconMT] \indam{Đơn vị đo góc và cung tròn}
	\begin{itemize}
		\item Đơn vị độ ($^\circ$): Chia đường tròn thành 360 cung tròn bằng nhau thì góc ở tâm chắn bởi cung đó sẽ có số đo là $1^\circ$.
		\item Đơn vị rađian (rad): Trên đường tròn, nếu một cung tròn có độ dài bằng bán kính thì ta nói cung đó có số đo là 1 rad. Khi đó, góc ở tâm chắn cung đó cũng có số đo 1 rad.
		\begin{luuy}
			Khi viết số đo một góc theo đơn vị rad, ta thường không viết chữ rad sau số đo. Chẳng hạn góc $\dfrac{\pi}{2}$ ta hiểu là góc $\dfrac{\pi}{2}$ rad.
		\end{luuy}
		\item Mối liên hệ giữa độ và rađian: Độ dài đường tròn là $2\pi R$ nên có số đo là $2 \pi$ rad tương ứng với $360^\circ$. Suy ra
		\boxminit{$1^\circ=\dfrac{\pi}{180}\, \mathrm{rad} \quad \text{ và } \quad 1\, \mathrm{rad}=\left(\dfrac{180}{\pi}\right)^\circ$}
	\end{itemize}
	\item[\iconMT] \indam{Độ dài cung tròn:} Một cung của đường tròn bán kính $R$ có số đo $\alpha$ rad thì sẽ có độ dài là $l=R\alpha$.
\end{itemize}

\subsubsection{ĐƯỜNG TRÒN LƯỢNG GIÁC}
\immini{
	\begin{itemize}
		\item Trong mặt phẳng tọa độ, đường tròn tâm $O$ bán kính 1, cùng với gốc $A(1;0)$ và chiều quay dương (như quy ước) gọi là đường tròn lượng giác.
		\item Cho góc lượng giác số đo $\alpha$. Trên đường tròn lượng giác, tồn tại duy nhất điểm $M$ sao cho góc lượng giác $(OA,O M)$ bằng $\alpha$ (hình bên). Khi đó, $M$ gọi là điểm biểu diễn của góc có số đo $\alpha$ trên đường tròn lượng giác.
	\end{itemize}
	
}{
	\begin{tikzpicture}[smooth,samples=300,scale=1.6,>=stealth]
		\path
		(0,0) coordinate (O)
		(0,1) coordinate (B)
		(1,0) coordinate (A)
		($(O)-(A)$) coordinate (A')
		($(O)-(B)$) coordinate (B')
		(O) + (140:1) coordinate (M)
		;
		\draw[->] (-1.2,0)--(1.5,0) node[below]{$x$};
		\draw[->] (0,-1.4)--(0,1.5) node[right]{$y$};
		\draw 
		(0,0) node[below right]{$O$}
		(O) circle (1 cm)
		(O)--(M)
		;
		\draw [shift={(0,0)},->,line width=0.3pt] (10:1.2) arc (10:30:1.2);
		\node at (1.3,.5) {$+$};
		\draw [shift={(0,0)},line width=0.5pt,red] (0:.3) arc (0:70:.3)node[right]{\scriptsize$\alpha$};
		\draw [shift={(0,0)},->,line width=0.5pt,red] (0:.3) arc (0:140:.3);
		\foreach \x/\g in {A/-50, A'/230,B/30,B'/-30,M/150} \draw[fill=black](\x) circle (0.02) + (\g:0.3)node{$\x$};
	\end{tikzpicture}}

\begin{dang}{Đổi đơn vị giữa độ và rađian. Độ dài cung tròn}
	\begin{enumerate}[\faCheckSquareO]
		\item Sử dụng cộng thức chuyển đổi giữa số đo độ và số đo rađian:
		\begin{listEX}[2]
			\item[$\bullet$]  $1^\circ=\dfrac{\pi}{180}\, \mathrm{rad}$
			\item[$\bullet$]   $1\, \mathrm{rad}=\left(\dfrac{180}{\pi}\right)^\circ$.
		\end{listEX}
		\item Xét đường tròn có bán kính $R$.
		\begin{itemize}
			\item[$\bullet$]   Cung tròn có số đo $\alpha \,\left(0\le \alpha \le 2\pi \right)$ thì có độ dài là $l=R\alpha$.
			\item[$\bullet$]   Cung tròn có số đo $a^\circ \, \left(0\le a\le 360\right)$ thì có độ dài là $l=\dfrac{\pi a}{180}.R$.
		\end{itemize}
	\end{enumerate}
\end{dang}

\begin{dang}{Số đo của góc lượng giác. Hệ thức Chasles}
	\begin{itemize}
		\item [$\bullet$] Khi xác định số đo của góc lượng giác, ta cần chú ý đến chiều quay (chiều dương ngược kim đồng hồ, chiều âm cùng kim đồng hồ). Từ đó xác định chính xác số đo của góc lượng giác $(Oa,Ob)$.
		\item [$\bullet$] Giả sử $\alpha^\circ$ là một số đo của góc lượng giác $(Oa,Ob)$. Suy ra số đo các góc lượng giác có cùng tia đầu $Oa$, tia cuối $Ob$ có dạng $\alpha^\circ+k\cdot 360^\circ$, với $k \in 
		\mathbb{Z}$.
		\item [$\bullet$] Hệ thức Chasles: $\text{sđ}(Ob,Oc)=\text{sđ}(Oa,Oc)-\text{sđ}(Oa, Ob)+k360^{\circ} \quad \text{với } k \in \mathbb{Z}.$
	\end{itemize}
	
\end{dang}

\begin{dang}{Biểu diễn góc lượng giác trên đường tròn lượng giác}
	Chọn gốc $A\left(1;0\right)$ làm điểm đầu. Để biểu diễn góc lượng giác có số đo $\alpha $ trên đường tròn lượng giác ta cần chọn điểm cuối $M$ trên đường tròn lượng giác sao cho $(OA, OM)=\alpha$. 
	\begin{luuy}
		Nếu $\big|\alpha\big|>2\pi$ ta phân tích $\alpha =\beta + k 2\pi$, với $-\pi<\beta <\pi$. Khi đó, ta chỉ cần xác định điểm cuối $M$ trên đường tròn lượng giác sao cho $(OA, OM)=\beta.$
	\end{luuy}
\end{dang}

