\section{QUY TẮC TÍNH ĐẠO HÀM}

\subsubsection{Quy tắc tính đạo hàm của tổng hiệu, tích thương}
Giả sử $u=u(x)$, $v=v(x)$ là các hàm số có đạo hàm tại điểm $x$ thuộc khoảng xác định. Ta có các quy tắc sau:
\begin{tcolorbox}[colframe=cyan,colback=red!1!white,boxrule=0.3mm]
	\begin{itemize}
		\begin{multicols}{2}
			\item [\ding{172}] $(u+v)'=u'+v'$
			\item [\ding{173}] $(u-v)'=u'-v'$
			\item [\ding{174}] $(u\cdot v)'=u'\cdot v+v' \cdot u$
			\item [\ding{175}] $\left( \dfrac{u}{v}\right)'=\dfrac{u' \cdot v-v' \cdot u}{v^2}$,  với $v \ne 0$.
			\item [\ding{176}] $(k \cdot u)'=k \cdot u'$, với $k$ là hằng số.
			\item [\ding{177}] $\left( \dfrac{1}{v}\right)'=-\dfrac{v'}{v^2}$, với $v \ne 0$.
		\end{multicols}
	\end{itemize}
\end{tcolorbox}

\subsubsection{Đạo hàm của hàm hợp}
Cho hàm số $y=f(u)$, với $u=u(x)$. Ta có công thức sau:
\boxmini{$[f(u)]'=f'(u) \cdot u'_x$}
\subsubsection{Bảng đạo hàm của một số hàm số sơ cấp cơ bản và hàm hợp}
\begin{center}
\begin{tikzpicture}[xscale=8,yscale=1,font=\footnotesize]
	\begin{scope}[shift={(-.5,.5)}]
		\fill[pink!5] (0,-1) rectangle (2,-12);
		\fill[cyan!25] (0,0) rectangle (2,-1);
		\draw [line width=0.5pt](0,0) grid (2,-12)
		(1,-1)--(2,-1) (1,-2)--(2,-2)
		(1,-3)--(2,-3) (0,-4)--(2,-4)
		(1,-5)--(2,-5) (0,-6)--(2,-6)
		(1,-7)--(2,-7) (0,-8)--(2,-8)
		(1,-9)--(2,-9) (1,-10)--(2,-10)
		(1,-11)--(2,-11) (0.15,0)--(0.15,-12)
		;
	\end{scope}
	\path
	(-0.42,0) node{\text{\textbf{STT}}}
	(-0.42,-1) node{\circEX{1}} 
	(-0.42,-2) node{\circEX{2}}
	(-0.42,-3) node{\circEX{3}}
	(-0.42,-4) node{\circEX{4}}
	(-0.42,-5) node{\circEX{5}}
	(-0.42,-6) node{\circEX{6}}
	(-0.42,-7) node{\circEX{7}}
	(-0.42,-8) node{\circEX{8}}
	(-0.42,-9) node{\circEX{9}}
	(-0.42,-10) node{\circEX{10}}
	(-0.42,-11) node{\circEX{11}} 
	;
	\path
	(0.08,0) node{\text{\textbf{Đạo hàm của một số hàm sơ cấp cơ bản}}}  
	(-0.2,-1) node[right]{$\left(x^n\right)'=n \cdot x^{n-1}$}
	(-0.2,-2) node[right]{$\left({\dfrac{1}{x}}\right)'=-\dfrac{1}{x^2}$}    
	(-0.2,-3) node[right]{$\left(\sqrt{x}\right)'=\dfrac{1}{2\sqrt{x}}$}
	(-0.2,-4) node[right]{$\left({\sin x}\right)'=\cos x$}    
	(-0.2,-5) node[right]{$\left({\cos x}\right)'=-\sin x$}
	(-0.2,-6) node[right]{$\left({\tan x}\right)'=\dfrac{1}{{\cos}^2x}$}    
	(-0.2,-7) node[right]{$\left({\cot x}\right)'=-\dfrac{1}{{\sin}^2x}$}
	(-0.2,-8) node[right]{$\left({e^x}\right)'=e^x$}    
	(-0.2,-9) node[right]{$\left({a^x}\right)'=a^x.\ln a$}
	(-0.2,-10) node[right]{$\left({\ln x}\right)'=\dfrac{1}{x}$}
	(-0.2,-11) node[right]{$\left({\log_a x}\right)'=\dfrac{1}{x \cdot \ln a}$}
	(1,0) node{\text{\textbf{Đạo hàm của hàm hợp tương ứng, với u = u(x)}}}    
	(0.8,-1) node[right]{$\left(u^n\right)'=n \cdot u^{n-1} \cdot u'$}
	(0.8,-2) node[right]{$\left({\dfrac{1}{u}}\right)'=-\dfrac{u'}{u^2}$}    							
	(0.8,-3) node[right]{$\left(\sqrt{u}\right)'=\dfrac{u'}{2\sqrt{u}}$}
	(0.8,-4) node[right]{$\left({\sin u}\right)'=u' \cdot \cos u$}    
	(0.8,-5) node[right]{$\left({\cos u}\right)'=-u' \cdot \sin u$}
	(0.8,-6) node[right]{$\left({\tan u}\right)'=\dfrac{u'}{{\cos}^2u}$}    
	(0.8,-7) node[right]{$\left({\cot u}\right)'=-\dfrac{u'}{{\sin}^2u}$}
	(0.8,-8) node[right]{$\left({e^u}\right)'=u' \cdot e^u$}    
	(0.8,-9) node[right]{$\left({a^u}\right)'=u' \cdot a^u.\ln a$}
	(0.8,-10) node[right]{$\left({\ln u}\right)'=\dfrac{u'}{u}$}
	(0.8,-11) node[right]{$\left({\log_a u}\right)'=\dfrac{u'}{u \cdot \ln a}$}
	;
\end{tikzpicture}
\end{center}

\begin{dang}{Tính đạo hàm dạng tích hoặc thương}
	\begin{itemize}
		\item [\iconCH] \indamm{ Các quy tắc cần nhớ:}  
		\begin{listEX}[1]
			\item [\ding{172}] $(u\cdot v)'=u'\cdot v+v' \cdot u$
			\item [\ding{173}] $\left( \dfrac{u}{v}\right)'=\dfrac{u' \cdot v-v' \cdot u}{v^2}$,  với $v \ne 0$.
			\item [\ding{174}] $\left( \dfrac{1}{v}\right)'=-\dfrac{v'}{v^2}$, với $v \ne 0$.
		\end{listEX}
		\item [\iconCH] \indamm{Công thức giải nhanh:}
		\begin{boxdl}
			\begin{itemize}
				\item [\ding{172}]  $\left(\dfrac{ax+ b}{cx + d}\right)' = \dfrac{ad - bc}{\left(cx + d\right)^2}$.
				\item [\ding{173}]  $\left(\dfrac{a_1 x^2 + b_1 x + c_1}{a_2 x^2 + b_2 x + c_2}\right)' = \dfrac{\begin{vmatrix}
						a_1&b_1\\
						a_2&b_2
					\end{vmatrix}x^2 + 2 \cdot \begin{vmatrix}
						a_1&c_1\\
						a_2&c_2
					\end{vmatrix}x + \begin{vmatrix}
						b_1&c_1\\
						b_2&c_2
				\end{vmatrix}}{\left(a_2 x^2 + b_2 x + c_2\right)^2}$
				\begin{itemize}
					\item [$\bullet$] Định thức trên tử số, ta nhân theo chiều "dấu huyền" trừ cho "dấu sắc".
					\item [$\bullet$] Nghĩa là: $\begin{vmatrix}
						a_1&b_1\\
						a_2&b_2
					\end{vmatrix}=a_1b_2-a_2b_1.$
				\end{itemize}
			\end{itemize}
		\end{boxdl}
	\end{itemize}
\end{dang}

\begin{dang}{Viết phương trình tiếp tuyến}
	\subsubsection{Đề bài cho trước $(x_0;y_0)$:}
	\begin{itemize}
		\item [\ding{172}] Tính $f'(x_0)$.
		\item [\ding{173}] Thay vào công thức \fbox{$y=f'(x_0)(x-x_0)+y_0$}, thu gọn kết quả về dạng $y=Ax+B$.
	\end{itemize}
	\subsubsection{Đề bài chưa cho đầy đủ $(x_0;y_0)$, ta thường gặp các loại sau:}
	\begin{itemize}
		\item [\ding{172}] Cho biết trước $x_0$ hoặc $y_0$. Ta chỉ việc thay giá trị đó vào hàm số $y=f(x)$, sẽ tính được đại lượng còn lại.
		\item [\ding{173}] Cho trước 1 điều kiện giải. Ta chỉ việc giải điều kiện đó, tìm $x_0$.
		\begin{itemize}
			\item [$\bullet$] Nếu đề bài cho biết hệ số góc của tiếp tuyến $k=a$, ta giải $f'(x)=a$ tìm nghiệm $x_0$.
			Đề bài thường cho hệ số góc tiếp tuyến dưới các dạng sau:
			\begin{tcolorbox}[colframe=cyan,colback=red!1!white,boxrule=0.1mm]
				\begin{itemize}
					\item Tiếp tuyến $d\parallel\Delta\colon y=ax+b\Rightarrow k=a$
					\item Tiếp tuyến $d\perp\Delta\colon y=ax+b\Rightarrow k=-\dfrac{1}{a}$
				\end{itemize}
			\end{tcolorbox}
			
			\item [$\bullet$] Nếu đề bài cho tiếp điểm là giao điểm của hai đồ thị $y=f(x)$ và $y=g(x)$, ta giải $f(x)=g(x)$ để tìm nghiệm $x_0$.
		\end{itemize}
	\end{itemize}
\end{dang}