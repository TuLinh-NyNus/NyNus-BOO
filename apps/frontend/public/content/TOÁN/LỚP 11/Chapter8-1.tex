%\setcounter{section}{21}
\section{HAI ĐƯỜNG THẲNG VUÔNG GÓC}
\subsection{KIẾN THỨC CẦN NHỚ}

\subsubsection{Góc giữa hai đường thẳng}
\begin{itemize}
	\item [\iconMT] \indam{Định nghĩa:} 
	\immini{
		Góc giữa hai đường thẳng $m$ và $n$ trong không gian, kí hiệu $(m, n)$, là góc giữa hai đường thẳng $a$ và $b$ cùng đi qua một điểm và tương ứng song song với $m$ và $n$. Ta tóm tắt cách dựng như sau:
		\vspace{0.2cm}
		\begin{gachsoc}
			\begin{listEX}[1]
				\item [\ding{172}] Chọn điểm $O$. Qua $O$, kẻ $a \parallel m$ và $b \parallel n$;
				\item [\ding{173}] Góc giữa $m$ và $n$ bằng góc giữa $a$ và $b$.
			\end{listEX}
		\end{gachsoc}
		
	}{	\begin{tikzpicture}[scale=0.8,font=\footnotesize]
			\def \c{0.5}
			\def \d{-0.7}
			\coordinate[label=above:$O$] (O) at (0,0);
			\fill (O)circle(1.5pt);
			\path
			(-2,-2*\c) coordinate (A1)
			(3,3*\c) coordinate (B1)
			(-0.2,-0.2*\c+1) coordinate (A2)
			(2.5,2.5*\c+1) coordinate (B2)
			
			(-2,-2*\d) coordinate (C1)
			(2,2*\d) coordinate (D1)
			(-0.4,-0.4*\d-0.8) coordinate (C2)
			(1.8,1.8*\d-0.8) coordinate (D2)		
			
			($(O)!0.8!(B1)$)node[above]{$a$}
			($(O)!0.5!(D1)$)node[above]{$b$}
			($(A2)!0.3!(B2)$)node[above]{$m$}
			($(C2)!0.7!(D2)$)node[above]{$n$}
			;
			\draw (A1)--(B1)
			(A2)--(B2)
			(C1)--(D1)
			(C2)--(D2)
			;
	\end{tikzpicture}}

	\item [\iconMT] \indam{Chú ý:} 
		\begin{gachsoc}
		\begin{itemize}
			\item [\ding{172}] Gọi $\varphi$ là góc giữa hai đường thẳng $a$, $b$ thì $0^\circ \le \varphi \le 90^\circ$.
			\item [\ding{173}] Nếu hai đường thẳng $a$, $b$ song song hoặc trùng nhau thì $ \varphi =0^\circ$.
		\end{itemize}
	\end{gachsoc}	
\end{itemize}
\subsubsection{Hai đường thẳng vuông góc}
\begin{itemize}
	\item [\iconMT] \indam{Định nghĩa:} 
	\immini{Hai đường thẳng được gọi là vuông góc nếu góc tạo bởi giữa chúng bằng $90^\circ$. 
		$$a\perp b\Leftrightarrow \left(a,b\right)=90^\circ.$$}
	{\begin{tikzpicture}[scale=0.6]
			\tkzDefPoints{0/0/O,-2/0/a,0.5/0/A,0/-0.5/B,0/2/b}
			\tkzDrawSegments(a,A b,B)
			\tkzDrawPoints(O)
			\tkzLabelPoints[above right](a,b)
			\tkzMarkRightAngle(b,O,a)
	\end{tikzpicture}}
	\item [\iconMT] \indam{Định lý:} 
		\immini{Nếu một đường thẳng vuông góc với một trong hai đường thẳng song song thì nó sẽ vuông góc với đường còn lại. 
		$$\heva{&a\parallel b\\&d\perp a}\Rightarrow d\perp b.$$}
	{\begin{tikzpicture}[scale=0.6]
			\tkzDefPoints{0/0/O,-3/0/a,1.5/0/A,-4/1.2/b,0/1.2/M,1/1.2/B,0/-1/D,0/3/d}
			\tkzDrawSegments(d,D a,A b,B)
			\tkzDrawPoints(O,M)
			\tkzLabelPoints[above left](d)
			\tkzLabelPoints[above right](a,b)
			\tkzMarkRightAngle(d,O,a)
			\tkzMarkRightAngle(d,M,b)
	\end{tikzpicture}}
	\begin{luuy}
		Hai đường thẳng vuông góc với nhau có thể chéo nhau hoặc cắt nhau.
	\end{luuy}
\end{itemize}

\begin{dang}{Xác định góc giữa hai đường thẳng}
	Trong không gian, giả sử cần xác định góc giữa hai đường thẳng $AB$ và $CD$. Ta có thể thực hiện các bước như sau:
	\immini{\begin{listEX}[1]
			\item [\ding{172}] Chọn gốc $A$, dựng $AE \parallel CD$;
			\item [\ding{173}] Kết luận góc giữa $AB$ và $CD$ bằng góc giữa $AB$ và $AE$.
			\item [\ding{174}] Xác định góc giữa $AB$ và $AE$. Có thể dùng hệ quả định lý cô-sin:
			$$\cos A = \dfrac{AB^2+AE^2-BE^2}{2AB \cdot AE}$$
	\end{listEX}}{\hspace{0.5cm}
		\begin{tikzpicture}[scale=0.7, font=\footnotesize,>=stealth]
			\path
			%	Vẽ mp
			(0,0) coordinate (A)
			(-2,-1) coordinate (C)
			(1,-2) coordinate (D)
			(3,-1) coordinate (E)
			(4,1) coordinate (B)
			;
			\draw (A)--(B) (C)--(D) (A)--(E);
			\draw[dashed] (E)--(B);
			\foreach \x/\g in {A/180,B/90,C/180,D/-90,E/-90}\draw[fill=black] (\x) circle (.05) +(\g:.5)node{\footnotesize$\x$};
	\end{tikzpicture}}
	\begin{luuy}
		Ta có thể chọn điểm gốc khác điểm $A$, ưu tiên cho điểm dễ dựng hình.
	\end{luuy}
\end{dang}

\begin{dang}{Chứng minh hai đường thẳng vuông góc}
	Để chứng minh hai đường thẳng $\triangle$ và $\triangle'$ vuông góc với nhau ta có thể sử dụng tính chất vuông góc trong mặt phẳng, cụ thể:
	\begin{itemize}
		\item Tam giác $ABC$ vuông tại $A$ khi và chỉ khi $\widehat{BAC}=90^{\circ} \Leftrightarrow \widehat{ABC}+\widehat{ACB}=90^{\circ}.$
		\item Tam giác $ABC$ vuông tại $A$ khi và chỉ khi $AB^2+AC^2=BC^2.$
		\item Tam giác $ABC$ vuông tại $A$ khi và chỉ khi trung tuyến xuất phát từ $A$ có độ dài bằng nửa cạnh $BC$.
		\item Nếu tam giác $ABC$ cân tại $A$ thì đường trung tuyến xuất phát từ $A$ cũng là đường cao của tam giác.
	\end{itemize}
	Ngoài ra, chúng ta cũng sử dụng tính chất: Nếu $d\perp \triangle$ và $\triangle'\parallel d$ thì $\triangle'$ cũng vuông góc với đường thẳng $\triangle.$
\end{dang}