\section{ĐƯỜNG THẲNG VUÔNG GÓC VỚI MẶT PHẲNG}
\subsection{KIẾN THỨC CẦN NHỚ}
\subsubsection{Định nghĩa}
\vspace{-0.65cm}
\immini{
	\begin{gachsoc}
		Đường thẳng $d$ được gọi là vuông góc với mặt phẳng $(\alpha)$ nếu $d$ vuông góc với mọi đường thẳng $a$ nằm trong mặt phẳng $(\alpha)$. 	Khi đó ta còn nói $(\alpha)$ vuông góc $d$ và kí hiệu $d \perp (\alpha)$ hoặc $(\alpha) \perp d$. 
	\end{gachsoc}
}{\hspace{0.5cm}
	\begin{tikzpicture}[scale=1,font=\footnotesize, line join=round, line cap=round,>=stealth]
		\tkzDefPoints{0/0/A, 1/1/B, 3/0/D, 2/2/m1}
		\coordinate (C) at ($(D)+(B)-(A)$);
		\draw (A)--(B)--(C)--(D)--cycle;
		\tkzMarkAngle[size=.5](D,A,B)
		\draw (0.06,-0.07) node[above right]{$\alpha$};
		\coordinate (m2) at ($(m1)-(0,2.2)$);
		\draw (m1) node[right]{$d$};
		\tkzInterLL(B,C)(m1,m2) \tkzGetPoint{m3}
		\coordinate (m4) at ($(m1)!0.6!(m2)$); 
		\tkzInterLL(A,D)(m1,m2) \tkzGetPoint{m5}
		\tkzDrawSegments(m1,m4 m2,m5)
		\tkzDrawSegments[dashed](m4,m5)
		\coordinate (e1) at ($(A)!0.8!(B)$);
		\coordinate (e2) at ($(A)!0.6!(D)$);
		\tkzDrawLines[add = -0.15 and -0.15](e1,e2)
		\coordinate (e3) at ($(e1)!0.25!(e2)$);
		\draw (e3) node[above,rotate=-30]{$a$};
	\end{tikzpicture}
}
\subsubsection{Điều kiện để đường thẳng vuông góc với mặt phẳng}\vspace{-0.8cm}
\immini{
	\begin{gachsoc}
		Nếu một đường thẳng vuông góc với hai đường thẳng cắt nhau cùng thuộc một mặt phẳng thì nó vuông góc với mặt phẳng ấy.
	\end{gachsoc}
	\begin{luuy}
		Tóm tắt định lí: \quad
		$\heva{&a,b \subset (\alpha)\\&a \cap b= \varnothing\\&d \perp a\\&d \perp b}\Rightarrow d \perp (\alpha).$
	\end{luuy}
}
{\begin{tikzpicture}[scale=1.2,font=\footnotesize, line join=round, line cap=round,>=stealth]
		\tkzDefPoints{0/0/A, 1/1.5/B, 3/0/D, 2.5/2/m1}
		\coordinate (C) at ($(D)+(B)-(A)$);
		\draw (A)--(B)--(C)--(D)--cycle;
		\tkzMarkAngle[size=.5](D,A,B)
		\draw (0.06,-0.06) node[above right]{$\alpha$};
		\coordinate (m2) at ($(m1)-(0,2.2)$);
		\draw (m1) node[left]{$d$};
		\tkzInterLL(B,C)(m1,m2) \tkzGetPoint{m3}
		\coordinate (m4) at ($(m1)!0.5!(m2)$); 
		\tkzInterLL(D,A)(m1,m2) \tkzGetPoint{m5}
		\draw (m1)--(m4);
		\draw[dashed] (m4)--(m5);
		\tkzDrawLines[add = 2 and -0.1](m2,m5)
		\coordinate (m6) at ($(m1)!0.15!(m2)$); 
		\coordinate (e1) at ($(A)!0.8!(B)$);
		\coordinate (e2) at ($(A)!0.75!(D)$);
		\coordinate (e3) at ($(e1)!0.2!(e2)$);
		\coordinate (e4) at ($(e1)!0.8!(e2)$);
		\draw (e3) node[above, rotate=-30]{$a$};
		\tkzDrawLines[add = -0.15 and -0.1](e1,e2)
		%vecto n
		\coordinate (e5) at ($(B)!0.6!(C)$); 
		\tkzDrawLines[add = -0.25 and -0.14](A,e5)
		\coordinate (e6) at ($(A)!0.3!(e5)$);
		\coordinate (e7) at ($(A)!0.74!(e5)$);
		\draw (e6) node[above, rotate=30]{$b$};
		\tkzInterLL(e1,e2)(A,e5) \tkzGetPoint{O}
		\tkzDrawPoints[fill=black](O)
		\tkzLabelPoints[below](O)
	\end{tikzpicture}
}

\subsubsection{Tính chất}
\begin{itemize}
	\item [\iconMT] \indam{Tính chất 1:} Có duy nhất một mặt phẳng đi qua một điểm cho trước và vuông góc với một đường thẳng cho trước. 
	\begin{center}
		\begin{tikzpicture}[scale=1.2,font=\footnotesize, line join=round, line cap=round,>=stealth]
			\tkzDefPoints{0/0/A, 1/1/B, 3/0/D, 2.5/2/m1}
			\coordinate (C) at ($(D)+(B)-(A)$);
			\draw (A)--(B)--(C)--(D)--cycle;
			\tkzMarkAngle[size=.5](D,A,B)
			\draw (0.06,-0.07) node[above right]{$\alpha$};
			\coordinate (m2) at ($(m1)-(0,2.2)$);
			\draw (m1) node[right]{$d$};
			\tkzInterLL(B,C)(m1,m2) \tkzGetPoint{m3}
			\coordinate (m4) at ($(m1)!0.6!(m2)$); 
			\tkzInterLL(A,D)(m1,m2) \tkzGetPoint{m5}
			\tkzDrawSegments[thick](m1,m4 m2,m5)
			\tkzDrawSegments[dashed](m4,m5)
			\coordinate (e1) at ($(A)!0.8!(B)$);
			\coordinate (e2) at ($(A)!0.6!(D)$);
			\coordinate (O) at ($(e1)!0.5!(e2)$);
			\tkzDrawPoints[size=1.5,fill=black](O)
			\tkzLabelPoints[above](O)
			\tkzDrawPoints[fill=black](O)
		\end{tikzpicture}\qquad\qquad\qquad
		\begin{tikzpicture}[font=\footnotesize, line join=round, line cap=round,>=stealth,xscale=0.8,yscale=0.6]
			\tkzDefPoints{0/0/E, 3/1/F, 0/5/H, 2/4.5/M, 1/2/I, -2/2/A}%điểm A ngay góc alpha, G ngay góc beta
			\tkzDefPointBy[translation = from E to H](F)
			\tkzGetPoint{G}
			\tkzDefPointBy[symmetry = center I](A)
			\tkzGetPoint{B}
			\tkzInterLL(F,G)(M,B)
			\tkzGetPoint{C}
			\tkzInterLL(F,G)(I,B)
			\tkzGetPoint{D}
			\tkzInterLL(E,H)(M,A)
			\tkzGetPoint{J}
			\tkzInterLL(E,H)(I,A)
			\tkzGetPoint{K}
			\tkzDrawPoints[fill=black](A,B,I,M)
			\tkzLabelPoints[below](A,B,I)
			\tkzLabelPoints[above](M)
			\tkzDrawSegments(E,F F,D C,G G,H H,E M,I M,B B,I A,K A,J)
			\tkzDrawSegments[dashed](M,J I,K C,D)
			\tkzMarkRightAngle[size=.2](M,I,B)
		\end{tikzpicture}
	\end{center}
	
	\begin{luuy}
		Chú ý:
		Mặt phẳng trung trực của đoạn thẳng $AB$ là mặt phẳng đi qua trung điểm $I$ của đoạn thẳng $AB$ và vuông góc với đường thẳng $AB$.
	\end{luuy}
	\item [\iconMT] \indam{Tính chất 2:} Có duy nhất một đường thẳng đi qua một điểm cho trước và vuông góc với một mặt phẳng cho trước. \\
	\centerline{\begin{tikzpicture}[scale=1.2,font=\footnotesize, line join=round, line cap=round,>=stealth]
			\tkzDefPoints{0/0/A, 1/1/B, 3/0/D, 2/2/m1}
			\coordinate (C) at ($(D)+(B)-(A)$);
			\draw (A)--(B)--(C)--(D)--cycle;
			\tkzMarkAngle[size=.5](D,A,B)
			\draw (0.06,-0.07) node[above right]{$\alpha$};
			\coordinate (m2) at ($(m1)-(0,2.2)$);
			\draw (m1) node[right]{$d$};
			\tkzInterLL(B,C)(m1,m2) \tkzGetPoint{m3}
			\coordinate (m4) at ($(m1)!0.6!(m2)$); 
			\tkzInterLL(A,D)(m1,m2) \tkzGetPoint{m5}
			\tkzDrawSegments[thick](m1,m4 m2,m5)
			\tkzDrawSegments[dashed](m4,m5)
			\coordinate (e1) at ($(A)!0.8!(B)$);
			\coordinate (e2) at ($(A)!0.6!(D)$);
			\coordinate (O) at ($(m1)!0.3!(m2)$);
			\tkzDrawPoints[size=1.5,fill=black](O)
			\tkzLabelPoints[left](O)
			\tkzDrawPoints[fill=black](O)
	\end{tikzpicture}}
\end{itemize}
\subsubsection{Liên hệ giữa quan hệ song song và quan hệ vuông góc của đường thẳng và mặt phẳng}
\begin{itemize}
	\item [\iconMT] \indam{Tính chất 3:} 
	\immini{
		\begin{enumerate}
			\item Cho hai đường thẳng song song. Mặt phẳng nào vuông góc với đường thẳng này thì cũng vuông góc với đường thẳng kia.
			\begin{luuy}
				Tóm tắt:\quad
				\fbox{$\heva{&a \parallel b\\&(\alpha) \perp a} \Rightarrow (\alpha) \perp b.$}
			\end{luuy}
			\item Hai đường thẳng phân biệt cùng vuông góc với một mặt phẳng thì song song với nhau.
			\begin{luuy}
				Tóm tắt:\quad
				\fbox{$\heva{&a \perp (\alpha)\\&b \perp (\alpha)\\&a \not \equiv b} \Rightarrow a \parallel b.$}
			\end{luuy}
		\end{enumerate}
	}{\vspace{1cm}
		\begin{tikzpicture}[font=\footnotesize, line join=round, line cap=round,>=stealth,scale=0.7]
			\tkzDefPoints{0/0/A, 5/0/B, 6/3/C, 2/1.5/I, 4/1.5/J, 2/4/K}
			\tkzDefPointBy[translation = from B to C](A)
			\tkzGetPoint{D}
			\tkzDefPointBy[symmetry = center I](K)
			\tkzGetPoint{P}
			\tkzDefPointBy[translation = from I to J](K)
			\tkzGetPoint{K'}
			\tkzDefPointBy[translation = from I to J](P)
			\tkzGetPoint{P'}
			\tkzInterLL(A,B)(K,P)
			\tkzGetPoint{Q}
			\tkzInterLL(A,B)(K',P')
			\tkzGetPoint{Q'}
			\tkzDrawSegments[dashed](I,Q J,Q')
			\tkzDrawSegments(A,B B,C C,D D,A K,I Q,P K',J Q',P')
			\tkzMarkAngle[size=.85](B,A,D)
			\draw (A) node [above right] {$\alpha$};
			\draw ($(K)$) node [below left] {$a$};
			\draw ($(K')$) node [below left] {$b$};
	\end{tikzpicture}}
	\item [\iconMT] \indam{Tính chất 4:}
	\immini{\begin{enumerate}
			\item Cho hai mặt phẳng song song. Đường thẳng nào vuông góc với mặt phẳng này thì cũng vuông góc với mặt phẳng kia.
			\begin{luuy}
				Tóm tắt: \quad
				\fbox{$\heva{&(\alpha) \parallel (\beta)\\&a \perp (\alpha)} \Rightarrow a \perp (\beta).$}
			\end{luuy}
			\item Hai mặt phẳng phân biệt cùng vuông góc với một đường thẳng thì song song với nhau.
			\begin{luuy}
				Tóm tắt: \quad
				\fbox{$\heva{& (\alpha) \perp a\\&(\beta) \perp a\\&(\alpha) \not \equiv (\beta)} \Rightarrow (\alpha) \parallel (\beta).$}
			\end{luuy}
		\end{enumerate}
	}{
		\vspace*{1cm}
		\begin{tikzpicture}[font=\footnotesize, line join=round, line cap=round,>=stealth,scale=0.9]
			\tkzInit[ymin=-3.6,ymax=1.6,xmin=-4.1,xmax=2.6]
			\tkzClip %cắt bớt phần ko gian dư
			\tkzDefPoints{-4/-3/A, 1/-3/B, -2.5/-1.5/C, 0/-2/O}
			\tkzDefPointBy[translation = from A to B](C) \tkzGetPoint{D}%phép tịnh tiến
			\coordinate (A') at ($(A)+(0,2)$);
			\tkzDefPointBy[translation = from A to B](A') \tkzGetPoint{B'}
			\tkzDefPointBy[translation = from A to C](A') \tkzGetPoint{C'}
			\tkzDefPointBy[translation = from A to B](C') \tkzGetPoint{D'}
			\tkzDefLine[perpendicular=through O,K=0.8](A,B) \tkzGetPoint{d}%đường qua O vuông góc AB
			\tkzInterLL(O,d)(A,B) \tkzGetPoint{d_1}
			\tkzInterLL(O,d)(A',B') \tkzGetPoint{d_2}
			\coordinate (O') at ($(d_2)+(0,1)$);
			\coordinate[label={below left}:$a$] (O'_1) at ($(O')+(0,1.5)$);
			\coordinate (O_1) at ($(O)+(0,-1.5)$);
			\tkzDrawSegments[dashed](d_1,O d_2,O')
			\tkzDrawSegments(A,B B,D D,C C,A A',B' B',D' D',C' C',A' O_1,d_1 O,d_2 O',O'_1)
			\tkzMarkAngles[arc=l,size=1.1cm](B,A,C B',A',C')
			\tkzLabelAngle[pos=0.7](B,A,C){$\beta$}
			\tkzLabelAngle[pos=0.7](B',A',C'){$\alpha$}
	\end{tikzpicture}}
	\item [\iconMT] \indam{Tính chất 5:}
	\immini{
		\begin{enumerate}
			\item Cho đường thẳng $a$ và mặt phẳng $(\alpha)$ song song với nhau. Đường thẳng nào vuông góc với mặt phẳng $(\alpha)$ thì cũng vuông góc với $a$.
			\begin{luuy}
				Tóm tắt:\quad
				\fbox{$\heva{&a \parallel (\alpha)\\&b \perp (\alpha)} \Rightarrow b \perp a.$}
			\end{luuy}
			\item Nếu một đường thẳng và một mặt phẳng  (không chứa đường thẳng đó) cùng vuông góc với một đường thẳng khác thì chúng song song với nhau.
			\begin{luuy}
				Tóm tắt:\quad
				\fbox{$\heva{&a \not \subset (\alpha)\\&a \perp b\\&(\alpha) \perp b} \Rightarrow a \parallel (\alpha).$}
			\end{luuy}
	\end{enumerate}}{\vspace{1.7cm}
		\begin{tikzpicture}[scale=0.9, line join=round, line cap=round]
			\tkzDefPoints{0/0/A,5/0/B,6.5/2/C, 3/1/O}
			\draw (1,3)--(6,3)node[above]{$a$};
			\coordinate (D) at ($(A)+(C)-(B)$);
			\coordinate (b) at ($(O)+(0,2.5)$);
			\coordinate (b') at ($(O)+(0,-2.5)$);
			\tkzInterLL(b,b')(A,B)\tkzGetPoint{I}
			\tkzDrawSegments(b,O I,b')
			\tkzDrawSegments[dashed](O,I)
			\tkzDrawPolygon(A,B,C,D)
			\tkzLabelPoints[left](b)
			\tkzMarkAngles[size=0.7cm,arc=l](B,A,D)
			\tkzLabelAngles[pos=0.4,rotate=10](B,A,D){\scriptsize $\alpha$}			
	\end{tikzpicture}}
\end{itemize}

\begin{dang}{Chứng minh đường thẳng vuông góc với mặt phẳng}
	\indamm{Một trong hai cách thường dùng:}\\
	\begin{minipage}[b]{8cm}
		\begin{itemize}
			\item [\circEX{1}] Chứng minh $\Delta$ vuông góc với hai đường thẳng $a,b$ cắt nhau thuộc $(\alpha)$.
			\begin{luuy}
				Tóm tắt:\quad \fbox{$\heva{& a \text{ cắt } b\\& \Delta \perp a\\&\Delta \perp b} \Rightarrow \Delta \perp (\alpha)$}
			\end{luuy}
			\item [\circEX{2}] Chứng minh $\Delta$ song song với đường thẳng $d$, trong đó $d$ vuông góc với $(\alpha)$.
			\begin{luuy}
				Tóm tắt:\quad \fbox{$\heva{&\Delta \parallel d\\&d  \perp (\alpha)} \Rightarrow \Delta \perp (\alpha)$}
			\end{luuy}
		\end{itemize}
	\end{minipage}\hspace{1cm}
	\begin{minipage}[b]{10cm}
		\begin{tikzpicture}[scale=0.9]
			\tkzDefPoints{0/0/A, 5/0/B, 6/2/C, 1/2/D, 2.5/1/E, 2.5/3/F, 2.5/0/G, 3/0.5/M, 4.5/1.5/N, 3/1.5/P, 4.5/0.5/Q}
			\draw (4.5,1.5) node[right] {\footnotesize a};
			\draw (4.5,0.5) node[right] {\footnotesize b};
			\draw (2.5,3) node[right] {\footnotesize $\Delta$};
			\tkzDrawSegments(A,B B,C C,D D,A E,F M,N P,Q)
			\tkzDrawSegments[dashed](E,G)
			\tkzMarkAngles[size=0.8cm,arc=l](B,A,D)
			\tkzLabelAngles[pos=0.45,rotate=10](B,A,D){\scriptsize$\alpha$}
		\end{tikzpicture}
		
		\begin{tikzpicture}[scale=0.9]
			\tkzDefPoints{0/0/A, 5/0/B, 6/2/C, 1/2/D, 2.5/1/E, 2.5/3/F, 2.5/0/G, 3/0.5/M, 4.5/1.5/N, 3/1.5/P, 4.5/0.5/Q, 4.5/3/R, 4.5/-0.2/S}
			\draw (2.5,3) node[right] {\footnotesize $\Delta$};
			\draw (4.5,3) node[right] {\footnotesize $d$};
			\tkzDrawSegments(A,B B,C C,D D,A E,F Q,R)
			\tkzDrawSegments[dashed](E,G Q,S)
			\tkzMarkAngles[size=0.8cm,arc=l](B,A,D)
			\tkzLabelAngles[pos=0.45,rotate=10](B,A,D){\scriptsize$\alpha$}
		\end{tikzpicture}
	\end{minipage}
	\indamm{Khi giải toán, ta chú ý đến các "quan hệ" vuông góc thường gặp sau:}
	\begin{boxkn}
		\begin{itemize}
			\item [\ding{172}] Đường trung tuyến trong tam giác cân (hạ từ đỉnh cân), trong tam giác đều thì vuông góc với cạnh đáy.
			\item [\ding{173}] Đường chéo hình thoi, đường chéo hình vuông thì vuông góc nhau.
			\item [\ding{174}] Xét trong tam giác, ta có thể kiểm tra quan hệ vuông góc của hai cạnh bằng cách thử định lý Pytago.
			\item [\ding{175}] Trong không gian, khi đề bài cho giải thiết "$SA$ vuông với đáy", ta có thể suy ra $SA$ vuông với các đường nằm trong mặt đáy.
		\end{itemize}
	\end{boxkn}
\end{dang}

\begin{dang}{Chứng minh hai đường thẳng vuông góc}
	\immini{ Chứng minh đường thẳng $\Delta$ vuông góc với $d$, ta có thể chứng minh $\Delta$ vuông góc với mặt phẳng $(\alpha)$ chứa $d$, nghĩa là
		\begin{center}
			\fbox{$\heva{&\Delta \perp (\alpha)\\& d \subset (\alpha) }\Rightarrow \Delta \perp d.$}
		\end{center}
	}{
		\begin{tikzpicture}[scale=0.9]
			\tkzDefPoints{0/0/A, 5/0/B, 6/2/C, 1/2/D, 2.5/1/E, 2.5/3/F, 2.5/0/G, 3/0.5/M, 4.5/1.5/N, 3/1.5/P, 4.5/0.5/Q}
			\draw (4.5,1.5) node[right] {\footnotesize d};
			\draw (2.5,3) node[right] {\footnotesize $\Delta$};
			\tkzDrawSegments(A,B B,C C,D D,A E,F M,N)
			\tkzDrawSegments[dashed](E,G)
			\tkzMarkAngles[size=0.7cm,arc=l](B,A,D)
			\tkzLabelAngles[pos=0.4,rotate=10](B,A,D){\scriptsize $\alpha$}
		\end{tikzpicture}
	}
\end{dang}