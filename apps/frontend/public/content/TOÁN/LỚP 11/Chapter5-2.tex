\section{TRUNG VỊ VÀ TỨ PHÂN VỊ CỦA MẪU SỐ LIỆU GHÉP NHÓM}

\subsection{LÝ THUYẾT CẦN NHỚ}
Cho mẫu số liệu ghép nhóm
\begin{center}
	\begin{tabular}{|c|c|c|c|c|c|}
		\hline
		Nhóm	&$\left[u_1;u_2 \right)$ & $\ldots$ & $\left[u_m;u_{m+1} \right)$ & $\ldots$ &$\left[u_k;u_{k+1} \right)$ \\
		\hline
		Tần số 	& $n_1$ & $\ldots$ & $n_m$ & $\ldots$ & $n_k$ \\
		\hline
	\end{tabular}	
\end{center}

\subsubsection{Trung vị $M_e$ của mẫu số liệu ghép nhóm}
\begin{enumerate}[\iconMT]
	\item \indam{Công thức tính:} 
	\begin{listEX}[2]
		\item [\ding{172}] Gọi $ n $ là cỡ mẫu.
		\item [\ding{173}] Giả sử nhóm $ \left[ u_m ; u_{m+1}\right) $ chứa trung vị;
		\item [\ding{174}] $ n_m $ là tần số của nhóm chứa trung vị;
		\item [\ding{175}] $ C=n_1+n_2+ \cdots +n_{m-1} $.
	\end{listEX}
		Khi đó \boxmini{$ M_e = u_m+\dfrac{\dfrac{n}{2}-C}{n_m}\cdot \left( u_{m+1}-u_m\right)$}

		\item \indam{Ý nghĩa:} Trung vị của mẫu số liệu ghép nhóm xấp xỉ cho trung vị của mẫu số liệu gốc, nó chia mẫu số liệu thành hai phần, mỗi phần chứa $50\%$ giá trị.
\end{enumerate}

\subsubsection{Tứ phân vị của mẫu số liệu ghép nhóm}
\begin{enumerate}[\iconMT]
	\item \indam{Tứ phân vị thứ hai $Q_2$:} Cũng hính là \indamm{trung vị} của mẫu số liệu ghép nhóm.
	\item \indam{Tứ phân vị thứ nhất $Q_1$:} Các bước tìm $Q_1$ như sau
	\begin{itemize}
		\item Giả sử nhóm $ \left[ u_m ; u_{m+1}\right)$ chứa tứ phân vị thứ nhất và $ n_m $ là tần số của nhóm tứ phân vị thứ nhất;
		\item $ C=n_1+n_2+\cdots +n_{m-1} $.
	\end{itemize}
	Khi đó \boxmini{$ Q_1=u_m+\dfrac{\dfrac{n}{4}-C}{n_m}\cdot \left( u_{m+1}-u_m\right)$}
	\item \indam{Tứ phân vị thứ ba $Q_3$:} Các bước tìm $Q_3$ như sau
	\begin{itemize}
		\item Giả sử nhóm $ \left[u_j ; u_{j+1} \right) $ chứa tứ phân vị thứ ba và $ n_j $ là tần số của nhóm chứa tứ phân vị thứ ba;
		\item $ C=n_1+n_2+\ldots+n_{j-1}$.
	\end{itemize}
	Khi đó \boxmini{$Q_3=u_j+\dfrac{\dfrac{3n}{4}-C}{n_j}\cdot \left(u_{j+1}-u_j \right)$}
\end{enumerate}
\begin{note}
	 Các tứ phân vị của mẫu số liệu ghép nhóm xấp xỉ cho các tứ phân vị của mẫu số liệu gốc, chúng chia mẫu số liệu gồm $4$ phần, mỗi phần chứa $25\%$ giá trị.
\end{note}




