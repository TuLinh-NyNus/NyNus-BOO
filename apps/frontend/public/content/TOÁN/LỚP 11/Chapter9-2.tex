\section{CÔNG THỨC NHÂN XÁC SUẤT}
\subsection{LÝ THUYẾT CẦN NHỚ}
\subsubsection{BIẾN CỐ ĐỘC LẬP}
\begin{enumerate}[\iconMT]
	\item \indam{Định nghĩa:} Cho hai biến cố $A$ và $B$. Hai biến cố $A$ và $B$ được gọi là độc lập nếu việc xảy ra hay không xảy ra của biến cố này không làm ảnh hưởng đến xác suất xảy ra của biến cố kia.
	\item \indam{Chú ý:} Nếu $A, B$ là hai biến cố độc lập thì mỗi cặp biến cố sau cũng độc lập: $A$ và $\overline{B} ; \overline{A}$ và $B ; \overline{A}$ và $\overline{B}$.
\end{enumerate}
\subsubsection{CÔNG THỨC NHÂN XÁC SUẤT}

Nếu hai biến cố $A$ và $B$ là độc lập thì \boxmini{$\mathrm{P}(A B)=\mathrm{P}(A) . \mathrm{P}(B)$}
\begin{luuy}
	Nếu $\mathrm{P}(A B)\ne \mathrm{P}(A) . \mathrm{P}(B)$ thì $A$ và $B$ không độc lập.
\end{luuy}

