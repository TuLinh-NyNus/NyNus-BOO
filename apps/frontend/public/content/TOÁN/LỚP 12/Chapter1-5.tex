\section{ỨNG DỤNG ĐẠO HÀM VÀ KHẢO SÁT HÀM SỐ ĐỂ GIẢI QUYẾT MỘT SỐ BÀI TOÁN THỰC TIỄN}
\subsection{LÝ THUYẾT CẦN NHỚ}
\subsubsection{Tốc độ thay đổi của một đại lượng}
Ta có đạo hàm $f'(a)$ là tốc độ thay đổi tức thời của đại lượng $y=f(x)$ đối với $x$ tại điểm $x=a$. Dưới đây, chúng ta xem xét một số ứng dụng của ý tưởng này đối với vật lí, hoá học, sinh học và kinh tế: 
\begin{itemize}
	\item Nếu $s=s(t)$ là hàm vị trí của một vật chuyển động trên một đường thẳng thì $v=s'(t)$ biểu thị vận tốc tức thời của vật (tốc độ thay đổi của độ dịch chuyển theo thời gian). Tốc độ thay đổi tức thời của vận tốc theo thời gian là gia tốc tức thời của vật:
	$$
	a(t)=v'(t)=s''(t).
	$$
	\item Nếu $C=C(t)$ là nồng độ của một chất tham gia phản ứng hoá học tại thời điểm $t$, thì $C'(t)$ là tốc độ phản ứng tức thời (tức là độ thay đổi nồng độ) của chất đó tại thời điểm $t$.
	\item Nếu $P=P(t)$ là số lượng cá thể trong một quần thể động vật hoặc thực vật tại thời điểm $t$, thì $P'(t)$ biểu thị tốc độ tăng trưởng tức thời của quần thể tại thời điểm $t$.
	\item  Nếu $C=C(x)$ là hàm chi phí, tức là tổng chi phí khi sản xuất $x$ đơn vị hàng hoá, thì tốc độ thay đổi tức thời $C'(x)$ của chi phí đối với số lượng đơn vị hàng được sản xuất được gọi là chi phí biên.
	\item Về ý nghĩa kinh tế, chi phí biên $C'(x)$ xấp xỉ với chi phí để sản xuất thêm một đơn vị hàng hoá tiếp theo, tức là đơn vị hàng hoá thứ $x+1$ (xem SGK Toán 11 tập hai, trang 87, bộ sách Kết nối tri thức với cuộc sống). 
\end{itemize}

