\section{PHƯƠNG TRÌNH ĐƯỜNG THẲNG}
\subsection{LÝ THUYẾT CẦN NHỚ}
\subsubsection{Vec tơ chỉ phương của đường thẳng}
\vspace{-0.3cm}
\begin{itemize}
	\item [\iconMT] \immini{ \indam{Định nghĩa:} Véc tơ chỉ phương  $\vec{u}$ của đường thẳng $d$ là những véc tơ khác $\vec{0}$ và có giá song song hoặc trùng với $d$. 
		\item [\iconMT] \indam{Chú ý:} 
		\begin{boxdn}
			\begin{itemize}
				\item [$\bullet$] $\vec{u} \ne \vec{0}$ và có giá song song hoặc trùng với $d$. 
				\item [$\bullet$] Nếu $\vec{u}$ và $\vec{u'}$ cùng là véc tơ chỉ phương của $d$ thì $\vec{u'} = k \cdot \vec{u}$ (\textit{tọa độ tỉ lệ nhau}).
			\end{itemize}
		\end{boxdn}
	}{\vspace{0.5cm}\hspace{1cm}
		\begin{tikzpicture}[scale=0.8, line join=round, line cap=round,>=stealth]
			\draw[thick] (0,0)--(4,0)node[below right]{$d$};
			\draw[->,blue] (1,1)--(3,1)node[below]{\scriptsize$\vec{u}$};
			\draw[->,violet] (2.3,1.5)--(3.7,1.5)node[above]{\scriptsize$\vec{u'}$};
	\end{tikzpicture}}
\end{itemize}
\subsubsection{Phương trình tham số của đường thẳng}
\begin{itemize}
	\item [\iconMT] \indam{Công thức:} Đường thẳng $d$ đi qua điểm $M(x_0;y_0;z_0)$ và nhận $\vec{u}=(u_1;u_2;u_3)$ làm véc tơ chỉ phương có phương trình là 
	\boxminit{$\heva{&x=x_0+u_1t\\&y=y_0+u_2t\\&z=z_0+u_3t} \quad \left( t \in \mathbb{R}\right) \quad (1) $}
	\item [\iconMT] \indam{Chú ý:}
	\begin{boxdn}
		\begin{itemize}
			\item [\ding{172}] Phương trình các trục tọa độ: 
			\begin{listEX}[3]
				\item [$\bullet$] $Ox \colon \heva{&x=t\\&y=0\\&z=0}$ .
				\item [$\bullet$] $Oy \colon \heva{&x=0\\&y=t\\&z=0}$ .
				\item [$\bullet$] $Oz \colon \heva{&x=0\\&y=0\\&z=t}$ .
			\end{listEX}
			\item [\ding{173}] Nếu $u_1$, $u_2$ và $u_3$ đều khác $0$ thì $(1)$ có thể được viết dưới dạng
			\boxminit{$\dfrac{x-x_0}{u_1}=\dfrac{y-y_0}{u_2}=\dfrac{z-z_0}{u_3} \quad (2)$}
			$(2)$ được gọi là phương trình chính tắc của đường thẳng $d$.
		\end{itemize}
	\end{boxdn}
\end{itemize}
\subsubsection{Vị trị tương đối giữa hai đường thẳng}
Cho hai đường thẳng 
\begin{itemize}
	\item [$\bullet$] $\Delta_1$ qua điểm $M(x_0;y_0;z_0)$, vec tơ chỉ phương $\vec{u}=(u_1;u_2;u_3)$;
	\item [$\bullet$] $\Delta_2$ qua điểm $N(x_0';y_0';z_0')$, vec tơ chỉ phương $\vec{v}=(v_1;v_2;v_3)$.
\end{itemize}
\begin{tcolorbox}[colframe=cyan,colback=red!3!white,boxrule=0.5mm]
\begin{minipage}[b]{9cm}
	\begin{listEX}[1]
		\item [] \indamm{Trường hợp 1:} Nếu $\bigg[\vec{u},\vec{v}\bigg] = \vec{0}$ và 
		\begin{itemize}
			\item [$\bullet$] $\bigg[\vec{u},\vec{MN}\bigg]\ne \vec{0}$  thì $\Delta_1$ song song $\Delta_2$; 
			\item [$\bullet$] $\bigg[\vec{u},\vec{MN}\bigg]  =\vec{0}$  thì $\Delta_1$ trùng $\Delta_2$.
		\end{itemize}
		\item [] \indamm{Trường hợp 2:} Nếu $\bigg[\vec{u},\vec{v}\bigg] \ne \vec{0}$ và 
		\begin{itemize}
			\item [$\bullet$] $\bigg[\vec{u},\vec{v}\bigg] \cdot \vec{MN} \ne 0$  thì $\Delta_1$ chéo $\Delta_2$; 
			\item [$\bullet$] $\bigg[\vec{u},\vec{v}\bigg] \cdot \vec{MN} =0$  thì $\Delta_1$ cắt $\Delta_2$.
		\end{itemize}
	\end{listEX}
\end{minipage}\hspace{1cm}
\begin{minipage}[b]{6cm}
	\begin{khung4}{Đặc biệt}
		\vskip 0.2cm
		Nếu $\vec{u} \perp \vec{v}$ hay
		$$u_1 \cdot v_1 +u_2 \cdot v_2+  u_3 \cdot v_3  +  =0 $$
		thì $\Delta_1$ vuông góc với $\Delta_2$		
		\vskip 0.2cm
	\end{khung4}
	\vspace{1.5cm}
\end{minipage}
\end{tcolorbox}
