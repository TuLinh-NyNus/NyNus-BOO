\section{XÁC SUẤT CÓ ĐIỀU KIỆN}
\subsection{LÝ THUYẾT CẦN NHỚ}
\subsubsection{Định nghĩa xác suất có điều kiện}
	\begin{enumerate}[\iconCH]
		\item \indamm{Định nghĩa:} Cho hai biến cố $A$ và $B$. Xác suất của biến cố $A$, tính trong điều kiện biết rằng biến cố $B$ đã xảy ra, được gọi là xác suất của $A$ với điều kiện $B$ và kí hiệu là $\mathrm{P}(A\mid B)$.
		\item \indamm{Công thức tính:} Cho hai biến cố $A$ và $B$ bất kì, với $\mathrm{P}(B)>0$. Khi đó 
		\boxminit{$\mathrm{P}(A \mid B)=\dfrac{\mathrm{P}(A B)}{\mathrm{P}(B)}$}
	\end{enumerate}
		
\subsubsection{Công thức nhân xác suất}
\begin{enumerate}[\iconCH]
	\item \indamm{Định nghĩa:} 	Với hai biến cố $A$ và $B$ bất kì, ta có
	\boxminit{$\mathrm{P}(A B)=\mathrm{P}(B) \cdot \mathrm{P}(A \mid B)$}
	Công thức trên được gọi là \textbf{\textit{công thức nhân xác suất}}.
	\item \indamm{Chú ý:}
	\begin{boxkn}
	\begin{listEX}[1]
		\item [\ding{172}] Vì $AB=BA$ nên với hai biến cố $A$ và $B$ bất kì, ta cũng có
		\boxminit{$\mathrm{P}(A B)=\mathrm{P}(A) \cdot \mathrm{P}(B \mid A)$}
		\item [\ding{173}] Nếu $A$ và $B$ là hai biến cố độc lập thì 
		\boxminit{$\mathrm{P}(A B)=\mathrm{P}(A) \cdot \mathrm{P}(B)$}
	\end{listEX}
\end{boxkn}
\end{enumerate}
