\section{CÔNG THỨC XÁC SUẤT TOÀN PHẦN -- CÔNG THỨC BAYES}
\subsection{LÝ THUYẾT CẦN NHỚ}
\subsubsection{Công thức xác suất toàn phần}
	Cho hai biến cố $A$ và $B$ là hai biến cố tùy ý. Khi đó
	\boxminit{$\mathrm{P}(A)=\mathrm{P}(B)\cdot \mathrm{P}(A|B)+\mathrm{P}(\overline{B})\cdot \mathrm{P}(A|\overline{B})$}
\subsubsection{Công thức Bayes}
	Cho hai biến cố $A$ và $B$ với $\mathrm{P}(A)>0$. Khi đó
	\boxminit{$\mathrm{P}\left(B|A\right)=\dfrac{\mathrm{P}(B)\cdot\mathrm{P}\left(A|B\right)}{\mathrm{P}(A)}$}
	Công thức Bayes còn được viết dưới dạng $\mathrm{P}(B | A)=\dfrac{\mathrm{P}(B) \cdot \mathrm{P}(A \mid B)}{\mathrm{P}(B)\cdot \mathrm{P}(A | B)+\mathrm{P}\left(\overline{B}\right)\cdot  \mathrm{P}\left(A | \overline{B}\right)}$.
\begin{note}
	\begin{enumerate}[\iconCH]
		\item \indamm{Chú ý 1:} Các công thúc cần nhớ
		\begin{listEX}[2]
			\item [\ding{172}] $P(A)+P(\bar{A})=1$.
			\item [\ding{173}] $P(A \mid B)+P(\bar{A} \mid B)=1$.
			\item [\ding{174}] $P(A \cap B)+P(A \cap \bar{B})=P(A)$.
			\item [\ding{175}] $P(A \cap B)+P(\bar{A} \cap B)=P(B)$.
		\end{listEX}
		\item \indamm{Chú ý 2:} Công thức xác suất toàn phần và Công thức Bayes được áp dụng trong các trường hợp sự việc bài toán đề cập đến gồm nhiều giai đoạn có sự liên đới nhau trong quá trình xảy ra.
	\end{enumerate}
\end{note}