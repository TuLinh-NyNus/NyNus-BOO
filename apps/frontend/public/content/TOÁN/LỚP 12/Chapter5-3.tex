\section{CÔNG THỨC TÍNH GÓC TRONG KHÔNG GIAN}
\subsection{LÝ THUYẾT CẦN NHỚ}
\subsubsection{Góc giữa hai mặt phẳng}
\begin{itemize}
	\item [\iconMT] \indam{Công thức:} Gọi $\vec{n_1}=(a_1;b_1;c_1)$, $\vec{n_2}=(a_2;b_2;c_2)$ lần lượt là véc tơ pháp tuyến của $(P)$ và $(Q)$; $\varphi$ là góc giữa hai mặt phẳng $(P)$ và $(Q)$, với $0^\circ \leq \varphi \leq 90^\circ$.
	Khi đó
	\boxminit{$\cos \varphi =\bigg|\cos\left(\vec{n_1}, \vec{n_2}\right) \bigg| =\dfrac{\bigg|a_1a_2+b_1b_2+c_1c_2\bigg|}{\sqrt{a_1^2+b_1^2+c_1^2} \cdot \sqrt{a_2^2+b_2^2+c_2^2}}$}
	\item [\iconMT] \indam{Chú ý:}
	\begin{itemize}
		\item [$\bullet$] Nếu $(P)$ song song hoặc trùng $(Q)$ thì $\varphi =0^\circ$.
		\item [$\bullet$] Nếu $(P)\perp (Q)$ thì $\varphi =90^\circ$. Khi đó $\vec{n_1}\cdot \vec{n_2}=0 \Leftrightarrow a_1a_2+b_1b_2+c_1c_2=0$.
	\end{itemize}
\end{itemize}
\subsubsection{Góc giữa hai đường thẳng}
\begin{itemize}
	\item [\iconMT] \indam{Công thức:}  Gọi $\vec{u}=(u_1;u_2;u_3)$, $\vec{v}=(v_1;v_2;v_3)$ lần lượt là véc tơ chỉ phương của  $d_1$ và $d_2$; $\varphi$ là góc giữa hai đường thẳng $d_1$ và $d_2$, với $0^\circ \leq \varphi \leq 90^\circ$.
	Khi đó
	\boxminit{$\cos \varphi =\bigg|\cos\left(\vec{u}, \vec{v}\right) \bigg| =\dfrac{\bigg|u_1v_1+u_2v_2+u_3v_3\bigg|}{\sqrt{u_1^2+u_2^2+u_3^2} \cdot \sqrt{v_1^2+v_2^2+v_3^2}}$}
	\item [\iconMT] \indam{Chú ý:}
	\begin{itemize}
		\item [$\bullet$] Nếu $d_1$ song song hoặc trùng $d_2$ thì $\varphi =0^\circ$.
		\item [$\bullet$] Nếu $d_1\perp d_2$ thì $\varphi =90^\circ$. Khi đó $\vec{u} \cdot\vec{u} =0 \Leftrightarrow u_1v_1+u_2v_2+u_3v_3=0$.
	\end{itemize}
\end{itemize}

\subsubsection{Góc giữa đường thẳng và mặt phẳng}
\begin{itemize}
	\item [\iconMT] \indam{Công thức:}  Gọi $\vec{u}=(u_1;u_2;u_3)$, $\vec{n}=(A;B;C)$ lần lượt là véc tơ chỉ phương của  $d$ và véc tơ pháp tuyến của $(P)$; $\varphi$ là góc giữa đường thẳng $d$ và mặt phẳng $(P)$, với $0^\circ \leq \varphi \leq 90^\circ$.
	Khi đó
	\boxminit{$\sin \varphi =\bigg|\cos\left(\vec{u}, \vec{n}\right) \bigg| =\dfrac{\bigg|u_1A+u_2B+u_3C\bigg|}{\sqrt{u_1^2+u_2^2+u_3^2} \cdot \sqrt{A^2+B^2+C^2}}$}
	\item [\iconMT] \indam{Chú ý:}
	\begin{itemize}
		\item [$\bullet$] Nếu $d$ song song hoặc trùng $(P)$ thì $\varphi =0^\circ$, khi đó $\vec{u} \perp \vec{n}$
		\item [$\bullet$] Nếu $d$ vuông góc với $(P)$ thì $\varphi =90^\circ$, khi đó $\vec{u} =k \cdot \vec{n}$.
	\end{itemize}
\end{itemize}