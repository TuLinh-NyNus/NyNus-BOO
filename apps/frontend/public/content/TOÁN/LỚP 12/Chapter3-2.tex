\section{PHƯƠNG SAI VÀ ĐỘ LỆCH CHUẨN CỦA MẪU SỐ LIỆU GHÉP NHÓM}
\subsection{LÝ THUYẾT CẦN NHỚ}
	Xét mẫu số liệu ghép nhóm cho bởi bảng sau:
\begin{center}
	\begin{tabular}{|c|c|c|c|c|}
		\hline Nhóm &{$\left[u_1; u_2\right)$} &{$\left[u_2; u_3\right)$} & $\ldots$ &{$\left[u_k; u_{k+1}\right)$} \\
		\hline Giá trị đại diện & $c_1$ & $c_2$ & $\ldots$ & $c_k$ \\
		\hline Tần số & $n_1$ & $n_2$ & $\ldots$ & $n_k$ \\
		\hline
	\end{tabular}
\end{center}
\begin{enumerate}[\iconMT] 
	\item \indam{Phương sai:} Phuơng sai của mẫu số liệu ghép nhóm, kí hiệu $S^2$, được tính bởi công thức
	$$S^2=\dfrac{1}{n}\left[n_1\left(c_1-\bar{x}\right)^2+n_2\left(c_2-\bar{x}\right)^2+\cdots+n_k\left(c_k-\bar{x}\right)^2\right],$$
	trong đó: $n=n_1+n_2+\cdots+n_k$ là cỡ mẫu; $\bar{x}=\dfrac{1}{n}\left(n_1 c_1+n_2 c_2+\cdots+n_k c_k\right)$ là số trung bình.
	\item \indam{Độ lệch chuẩn:} Độ lệch chuẩn của mẫu số liệu ghép nhóm, kí hiệu $S$, là căn bậc hai số học của phương sai, nghĩa là $S=\sqrt{S^2}$.
	\item \indam{Ý nghĩa:}
	\begin{itemize}
		\item [\iconCH] Phương sai (độ lệch chuẩn) của mẫu số liệu ghép nhóm là giá trị xấp xỉ cho phương sai (độ lệch chuẩn) của mẫu số liệu gốc. Chúng được dùng để đo mức độ phân tán của mẫu số liệu ghép nhóm xung quanh số trung bình của mẫu số liệu. Phương sai và độ lệch chuẩn càng lớn thì dữ liệu càng phân tán.
		\item [\iconCH] Độ lệch chuẩn có cùng đơn vị với đơn vị của mẫu số liệu.
	\end{itemize}
\begin{note}
	\begin{enumerate}
		\item Phương sai của mẫu số liệu ghép nhóm có thể được tính theo công thức sau:
		$$S^2=\dfrac{1}{n}\left(n_1 c_1^2+n_2 c_2^2+\cdots+n_k c_k^2\right)-\overline{x}^2.$$
		\item Trong thống kê, người ta còn dùng đại lượng sau để đo mức độ phân tán của mẫu số liệu ghép nhóm:
		$$\hat{s}^2=\dfrac{1}{n-1}\left[n_1\left(c_1-\overline{x}\right)^2+n_2\left(c_2-\overline{x}\right)^2+\cdots+n_k\left(c_k-\overline{x}\right)^2\right].$$
	\end{enumerate}
\end{note}
\end{enumerate}
