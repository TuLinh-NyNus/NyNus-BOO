\section{BIỂU THỨC TỌA ĐỘ CỦA CÁC PHÉP TOÁN VECTƠ}
\subsection{LÝ THUYẾT CẦN NHỚ}
\subsubsection{Biểu thức tọa độ của phép toán cộng, trừ, nhân một số thực với một véctơ}
Trong không gian $Oxyz$, cho hai véc-tơ $\vec{a} = (a_1;a_2;a_3)$, $\vec{b} = (b_1; b_2; b_3)$ và số $k$. Khi đó
\begin{gachsoc}
	\begin{itemize}
		\item [\ding{172}] $\vec{a}+\vec{b}=(a_1+b_1;a_2+b_2;a_3+b_3)$;
		\item [\ding{173}] $\vec{a}-\vec{b}=(a_1-b_1;a_2-b_2;a_3-b_3)$;
		\item [\ding{174}] $k\vec{a} = (ka_1; ka_2; ka_3)$.
	\end{itemize}
\end{gachsoc}
\begin{luuy}
	Cho hai véc-tơ $\vec{a}=(a_1;a_2;a_3)$, $\vec{b}=(b_1;b_2;b_3)$, $\vec{b}\ne \vec{0}$. Hai véc-tơ $\vec{a}$, $\vec{b}$ cùng phương khi và chỉ khi tồn tại một số thực $k$ sao cho $\heva{&a_1=k b_1\\& a_2= k b_2\\& a_3= k b_3.}$
\end{luuy}
\subsubsection{Biểu thức tọa độ của tích vô hướng hai véctơ}
Trong không gian $Oxyz$, tích vô hướng của hai véc-tơ $\vec{a} = (a_1;a_2;a_3)$ và $\vec{b} = (b_1; b_2; b_3)$ được xác định bởi công thức
\[\vec{a} \cdot \vec{b} = a_1b_1 + a_2b_2 + a_3b_3. \]
\begin{note}
	\begin{itemize}
		\item[\ding{172}] $\vec{a} \perp \vec{b} \Leftrightarrow a_1b_1 + a_2b_2 + a_3b_3 = 0$;
		\item[\ding{173}] $\left| \vec{a} \right| = \sqrt{a_1^2 + a_2^2 +a_3^2}$; \quad $AB=\sqrt{(x_B-x_A)^2+(y_B-y_A)^2+(z_B-z_A)^2}$.
		\item[\ding{174}] $\cos \left(\vec{a}; \vec{b}\right) = \dfrac{\vec{a}\cdot \vec{b}}{\left|\vec{a}\right| \cdot \left|\vec{b}\right|} = \dfrac{a_1b_1 + a_2b_2 + a_3b_3}{\sqrt{a_1^2 + a_2^2 +a_3^2} \cdot \sqrt{b_1^2 + b_2^2 +b_3^2}}$ (với $\vec{a} \ne \vec{0}$ và $\vec{b} \ne \vec{0}$).
	\end{itemize}
\end{note}
\subsubsection{Biểu thức tọa độ của tích có hướng hai véctơ}
Cho hai véc-tơ $\vec{a}=(a_1;a_2;a_3)$ và $\vec{b}=(b_1;b_2;b_3)$ không cùng phương. Khi đó vec tơ $$\vec{w}=\bigg(a_2b_3-b_2a_3\,;\,a_3b_1-b_3a_1\,;\,a_1b_2-b_1a_2 \bigg)$$ vuông góc với cả hai véc tơ $\vec{a}$ và $\vec{b}$.
\begin{luuy}
	\begin{itemize}
		\item [\ding{172}] Véc tơ $\vec{w}$ xác định như trên còn gọi là \indamm{tích có hướng} của hai véc tơ $\vec{a}$, $\vec{b}$, kí hiệu  $\vec{w}=\big[\vec{a},\vec{a}\big]$.
		\item [\ding{173}] Quy ước $\left|\begin{array}{l}
			{a_1}\quad{a_2}\\
			{b_1}\quad{b_2}
		\end{array}\right|=a_1b_2-a_2b_1$ thì
		$$\left[\vec a ,\vec b\right]=\left(\left|\begin{array}{l}
			{a_2}\quad{a_3}\\
			{b_2}\quad{b_3}
		\end{array}\right|;\left|\begin{array}{l}
			{a_3}\quad {a_1}\\
			{b_3}\quad{b_1}
		\end{array}\right|;\left|\begin{array}{l}
			{a_1}\quad{a_2}\\
			{b_1}\quad{b_2}
		\end{array}\right|\right)=\bigg(a_2b_3-b_2a_3\,;\,a_3b_1-b_3a_1\,;\,a_1b_2-b_1a_2 \bigg) $$
		\item [\ding{174}] $\vec{a}$ không cùng phương với $\vec{b}$ $\Leftrightarrow \left[\vec a ,\vec b\right] \ne \vec{0}$.
	\end{itemize}
\end{luuy}
\subsubsection{Biểu thức tọa độ trung điểm đoạn thẳng, trọng tâm tam giác}
\vspace{-0.7cm}
\immini{Trong không gian $Oxyz$, tọa độ trung điểm và trong tâm được xác định như sau:
	\begin{itemize}
		\item [\ding{172}] Tọa độ trung điểm $M$ của đoạn thẳng $AB$ là
		\[ M\left(\dfrac{x_A + x_B}{2}; \dfrac{y_A + y_B}{2}; \dfrac{z_A + z_B}{2} \right).\]
		\item [\ding{173}] Tọa độ trọng tâm $G$ của tam giác $ABC$ là
		\[ G\left(\dfrac{x_A + x_B +x_C}{3}; \dfrac{y_A + y_B +y_C}{3}; \dfrac{z_A + z_B + z_C}{3} \right).\]
\end{itemize}}{\vspace{1cm}
	\begin{tikzpicture}[scale=0.8, font=\footnotesize, line join=round, line cap=round]
		\begin{scope}
			\foreach \x\y\t in {-2/-2/A, 0/0/B}
			\coordinate (\t) at (\x,\y);
			\coordinate (M) at ($(A)!0.5!(B)$);
			\foreach \a\b in {A/B}
			\draw[] (\a)--(\b);
			\foreach \t\g in {A/-90, B/40,M/1200}
			\draw[fill=black] (\t)circle(0.6pt) +(\g:8pt)node{$\t$};
		\end{scope}
		\begin{scope}[xshift=3cm]
			\foreach \x\y\t in {0/0/A, -2/-2/B, 2.5/-2/C}
			\coordinate (\t) at (\x,\y);
			\coordinate (M) at ($(A)!0.5!(B)$);
			\coordinate (N) at ($(A)!0.5!(C)$);
			\coordinate (K) at ($(C)!0.5!(B)$);
			\coordinate (G) at ($(A)!2/3!(K)$);
			\foreach \a\b in {A/B, B/C, A/C, A/K, M/C, B/N}
			\draw[] (\a)--(\b);
			\foreach \t\g in {A/90, B/-100, C/-80, M/120, N/40, K/-90,G/60}
			\draw[fill=black] (\t)circle(0.8pt) +(\g:10pt)node{$\t$};
		\end{scope}
\end{tikzpicture}}
