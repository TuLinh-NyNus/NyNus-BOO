\section{ỨNG DỤNG HÌNH HỌC CỦA TÍCH PHÂN}
\subsection{LÝ THUYẾT CẦN NHỚ}
\subsubsection{Diện tích hình phẳng giới hạn bởi đồ thị $y=f(x)$, trục hoành và hai đường thẳng $x=a$, $x=b$\quad $(a<b)$.}
\begin{enumerate}[\iconMT]
	\item \indam{Công thức tính:}\\
	\begin{minipage}[b]{6cm}
		\boxmini{$S_{(H)}=\displaystyle \int\limits_{a}^{b} \big|f(x)\big|\mathrm{\,d}x$}
		\vspace{1.5cm}
	\end{minipage}\hspace{1cm}
	\begin{minipage}[b]{9cm}
		\begin{khung4}{Lưu ý}
			\begin{itemize}
				\item [\iconCH] Nếu $f(x)$ không đổi dấu trên $[a;b]$ thì $$\displaystyle \int\limits_{a}^{b} \big|f(x)\big|\mathrm{\,d}x=\bigg|\displaystyle \int\limits_{a}^{b} f(x)\mathrm{\,d}x\bigg|.$$
				\item [\iconCH] Nếu đề chưa cho (\textit{hoặc thiếu}) cận, thì ta giải phương trình $f(x)=0$ để tìm cận.
			\end{itemize}
		\end{khung4}
	\end{minipage}
	\item \indam{Minh họa hình ảnh:}\\
	\hspace*{1cm}\begin{tikzpicture}[smooth,samples=300,scale=0.8,>=stealth,font=\footnotesize]
		\draw[->] (-0.8,0)--(4.5,0) node[below]{$x$};
		\draw[->] (0,-0.5)--(0,3) node[right]{$y$};
		\draw (0,0) node[below left]{$O$};
		\draw[blue,line width=0.7pt,domain=0.3:3.2] plot(\x,{0.3*(\x-1)^2+1})node[above right]{\scriptsize $y=f(x)$};
		\draw[fill=black] (1,0) circle(1pt) (3,0) circle(1pt);
		\draw[dashed] (1,0)node[below]{$a$}--(1,1) (3,0)node[below]{$b$}--(3,2.2);
		\fill[pattern=north west lines] (1,0)--plot[domain=1:3](\x,{0.3*(\x-1)^2+1})--(3,0)--cycle;
		\draw[->] (2.5,0.5)--(3.7,1) node[right]{$S$};
		\node[right] at (-1,-2.2) {\fbox{$S=\displaystyle \int\limits_{a}^{b} \big|f(x)\big|\mathrm{\,d}x=\displaystyle \int\limits_{a}^{b} f(x)\mathrm{\,d}x$}};
	\end{tikzpicture}
	\hspace{1cm}
	\begin{tikzpicture}[smooth,samples=300,scale=0.8,x=1.1cm,>=stealth,font=\footnotesize]
		\draw[->] (-0.8,0)--(3.5,0) node[below]{$x$};
		\draw[->] (0,-1)--(0,2) node[right]{$y$};
		\draw (0,0) node[below left]{$O$};
		\draw[blue,line width=0.7pt,domain=0.3:2.8] plot(\x,{(\x)*(\x-1.5)*(\x-2.5)})node[above right]{\scriptsize $y=f(x)$};
		\draw[fill=black] (0.5,0) circle(1pt) (1.5,0) circle(1pt) (2,0) circle(1pt);
		\draw[dashed] (0.5,0)node[below]{$a$}--(0.5,1) (2,0)node[above]{$b$}--(2,-0.5);
		\fill[pattern=north east lines] (0.5,0)--plot[domain=0.5:2](\x,{(\x)*(\x-1.5)*(\x-2.5)})--(2,0)--cycle;
		\node[right] at (-1,-2.2) {\fbox{$S=\displaystyle \int\limits_{a}^{b} \big|f(x)\big|\mathrm{\,d}x=\displaystyle \int\limits_{a}^{c} f(x)\mathrm{\,d}x-\displaystyle \int\limits_{c}^{b} f(x)\mathrm{\,d}x$}};
		\node[below] at (1.5,0) {$c$};
	\end{tikzpicture}
\end{enumerate}

\subsubsection{Diện tích hình phẳng giới hạn bởi đồ thị $y=f(x)$, $y=g(x)$ và hai đường thẳng $x=a$, $x=b$\quad $(a<b)$.}
\begin{enumerate}[\iconMT]
	\item \indam{Công thức tính:}\\
	\begin{minipage}[b]{6cm}
		\boxmini{$S_{(H)}=\displaystyle \int\limits_{a}^{b} \bigg|f(x)-g(x)\bigg|\mathrm{\,d}x.$}
		\vspace{1.5cm}
	\end{minipage}\hspace{1cm}
	\begin{minipage}[b]{9cm}
		\begin{khung4}{Lưu ý}
			\begin{itemize}
				\item [\iconCH] Nếu $f(x)-g(x)$ không đổi dấu trên $[a;b]$ thì $$\displaystyle \int\limits_{a}^{b} \big|f(x)-g(x)\big|\mathrm{\,d}x=\bigg|\displaystyle \int\limits_{a}^{b} (f(x)-g(x))\mathrm{\,d}x\bigg|.$$
				\item [\iconCH] Nếu đề chưa cho (\textit{hoặc thiếu}) cận, thì ta giải phương trình $f(x)=g(x)$ để tìm cận.
			\end{itemize}
		\end{khung4}
	\end{minipage}
\newpage
	\item \indam{Minh họa hình ảnh:}\\
	\hspace*{1cm}\begin{tikzpicture}[smooth,samples=300,scale=0.8,>=stealth,font=\footnotesize]
		\draw[->] (-0.8,0)--(5.5,0) node[below]{$x$};
		\draw[->] (0,-0.5)--(0,3) node[right]{$y$};
		\draw (0,0) node[below left]{$O$};
		\draw[blue,line width=0.7pt,domain=0.3:3.2] plot(\x,{0.3*(\x-1)^2+1.5})node[right]{\scriptsize $y=f(x)$};
		\draw[magenta,line width=0.7pt,domain=0.3:3.4] plot(\x,{0.3*(\x-1.5)^2+0.5})node[below right]{\scriptsize $y=g(x)$};
		\draw[fill=black] (1,0) circle(1pt) (3,0) circle(1pt);
		\draw[dashed] (1,0)node[below]{$a$}--(1,1.5) (3,0)node[below]{$b$}--(3,2.7);
		\fill[pattern=north west lines,smooth] plot[domain=1:3](\x,{0.3*(\x-1)^2+1.5})--(3,1.175)--plot[domain=3:1](\x,{0.3*(\x-1.5)^2+0.5});
		\draw[->] (2.2,1.5)--(1.6,2.5) node[above]{$S$};
		\node[right] at (-0.8,-2) {\fbox{$S=\displaystyle \int\limits_{a}^{b} \bigg[f(x)-g(x)\bigg]\mathrm{\,d}x$}};
	\end{tikzpicture}
	\hspace{2cm}
	\begin{tikzpicture}[smooth,samples=300,scale=0.8,y=0.8cm,>=stealth,font=\footnotesize]
		\draw[->] (-0.8,0)--(7,0) node[below]{$x$};
		\draw[->] (0,-0.5)--(0,5) node[right]{$y$};
		\draw (0,0) node[below left]{$O$};
		\draw[blue,line width=0.7pt,domain=0.7:5.3] plot(\x,{0.5*(\x-1)*(\x-3)*(\x-5)+3})node[above right]{\scriptsize $y=f(x)$};
		\draw[magenta,line width=0.7pt,domain=0.5:5.3] plot(\x,{0.5*(\x-3)^2+1})node[below right]{\scriptsize $y=g(x)$};
		\draw[fill=blue] (1,0) circle(1pt) (4,0) circle(1pt) (5,0) circle(1pt)
		(1,3) circle(1.5pt) (4,1.5) circle(1.5pt) (5,3) circle(1.5pt);
		\draw[dashed] (1,0)node[below]{$a$}--(1,3)
		(4,0)node[below]{$c$}--(4,1.5)
		(5,0)node[below]{$b$}--(5,3);
		\fill[pattern=north west lines] (1,5)--plot[domain=1:5](\x,{0.5*(\x-3)^2+1})--(5,3)--plot[domain=5:1](\x,{0.5*(\x-1)*(\x-3)*(\x-5)+3})--cycle;
		\node[right] at (-0.8,-2.4) {\fbox{$S=\displaystyle \int\limits_{a}^{c} \bigg[f(x)-g(x)\bigg]\mathrm{\,d}x+\displaystyle \int\limits_{c}^{b} \bigg[g(x)-f(x)\bigg]\mathrm{\,d}x$}};
	\end{tikzpicture}
\end{enumerate}

\subsubsection{Tính thể tích của vật thể giới hạn bởi hai mặt phẳng $x=a$, $x=b$.}
\immini{Cho $T$ là một vật thể nằm giới hạn bởi hai mặt phẳng $x=a$ và $x=b$ (hình vẽ). Gọi $S(x)$ là diện tích thiết diện của vật thể cắt bởi mặt phẳng vuông góc với trục hoành tại điểm có hoành độ $x$, $(a \le x \le b)$.	Giả sử $S(x)$ là một hàm liên tục. Khi đó thể tích của vật thể $T$ được tính theo công thức
	\begin{center}
		\boxmini{$V=\displaystyle\int\limits_a^bS(x) \mathrm{\,d}x$}
	\end{center}
}{\hspace{0.5cm}
	\begin{tikzpicture}[>=stealth, scale=0.8]
		\draw plot[smooth,tension=.65] coordinates{(1,2) (2.5,2.3) (3.5,2.2)};
		\draw[dashed] plot[smooth,tension=.65] coordinates{(3.5,2.2) (4,2)};
		\draw plot[smooth,tension=.65] coordinates{(4,2) (5,2.2) (5.5,2.1)};
		\draw[dashed] plot[smooth,tension=.65] coordinates{(5.5,2.1) (6,2)};
		\draw plot[smooth,tension=.65] coordinates{(1,1) (2.3,0.5) (3.5,0.8)};
		\draw[dashed] plot[smooth,tension=.65] coordinates{(3.5,0.8) (4,1)};
		\draw plot[smooth,tension=.65] coordinates{(4,1) (5,0.7) (5.5,0.8)};
		\draw[dashed] plot[smooth,tension=.65] coordinates{(5.5,0.8) (6,1)};
		\draw[dashed] (1,1) arc (-90:90:.2 and 0.5);
		\draw (1,2) arc (90:270:.2 and 0.5);
		\draw[dashed] (4,1) arc (-90:90:.2 and 0.5);
		\draw (4,2) arc (90:270:.2 and 0.5);
		\draw (6,1) arc (-90:270:.2 and 0.5);
		\fill[pattern=north east lines] (4,1) arc (-90:90:.2 and 0.5)--(4,2) arc (90:270:.2 and 0.5)--cycle;
		\draw (-.5,0)--(0.5,0) (1,0)--(3.5,0) (4,0)--(5.5,0);
		\draw[dashed] (0.5,0)--(1,0) (3.5,0)--(4,0) (5.5,0)--(6,0);
		\draw[->] (6,0)--(7,0)node[below]{$x$};
		\draw (0.5,-1)--(0.5,3)--(1.5,3.5)--(1.5,2.2) (1.5,.8)--(1.5,-0.5)--(0.5,-1);
		\draw[dashed](1.5,2.2)--(1.5,.8);
		\draw[dashed] (1,1)--(1,0)node[below]{$a$};
		\coordinate (A) at (0.5,3);
		\coordinate (B) at (1.5,3.5);
		\coordinate (C) at (1.5,2.2);
		\tkzMarkAngle[size=.6](A,B,C);
		\draw (1.3,3.2) node {\footnotesize $P$};
		\draw (3.5,-1)--(3.5,3)--(4.5,3.5)--(4.5,2) (4.5,1)--(4.5,-0.5)--(3.5,-1);
		\draw[dashed](4.5,2)--(4.5,1);
		\draw[dashed] (4,1)--(4,0)node[below]{$x$};
		\coordinate (D) at (3.5,3);
		\coordinate (E) at (4.5,3.5);
		\coordinate (F) at (4.5,2);
		\tkzMarkAngle[size=.6](D,E,F);
		\draw (4.3,3.2) node {\footnotesize $R$};
		\draw (5.5,-1)--(5.5,3)--(6.5,3.5)--(6.5,-0.5)--(5.5,-1);
		\draw[dashed] (6,1)--(6,0)node[below]{$b$};
		\coordinate (G) at (5.5,3);
		\coordinate (H) at (6.5,3.5);
		\coordinate (K) at (6.5,-0.5);
		\tkzMarkAngle[size=.6](G,H,K);
		\draw (6.3,3.2) node {\footnotesize $Q$};
		\draw (0,.3) node {$O$};
		\fill (0,0) circle(1pt);
		\draw[->] (4,1.5)--(4.7,1.7) node[right] {\scriptsize $S(x)$};
\end{tikzpicture}}
\subsubsection{Tính thể tích khối tròn xoay khi cho hình phẳng giới hạn bởi $y=f(x)$, trục hoành và hai đường thẳng $x=a$, $x=b$.}
\immini{Cho hình phẳng $(H)$ giới hạn bởi $y=f(x)$, trục hoành và hai đường thẳng $x=a$, $x=b$ (\textit{phần gạch sọc}). Khi cho $(H)$ quay quanh trục $Ox$, ta được một khối tròn xoay. Thể tích khối này được tính theo công thức sau
	\begin{center}
		\boxmini{$V=\pi \displaystyle\int\limits_a^b \bigg[f(x)\bigg]^2 \mathrm{\,d}x$}
\end{center}}{\hspace{0.5cm}
	\begin{tikzpicture}[line join=round, line cap=round,>=stealth,thick,scale=.4]
		\tikzset{label style/.style={font=\normalsize}}
		%%Nhập giới hạn đồ thị và hàm số cần vẽ
		\def \xmin{-.5}
		\def \xmax{11}
		\def \ymin{-4}
		\def \ymax{4.5}
		%\draw[xstep=1 cm, ystep=1 cm,gray,thin] (\xmin,\ymin) grid (\xmax,\ymax);
		\def \hamso{0-0.01864083398323228*((\x)-1.0)^(3.0)+0.35126127715584465*((\x)-1.0)^(2.0)-1.5117402856674746*((\x)-1.0)+3.106314636847822}
		%%Tự động
		\draw[->] (\xmin,0)--(11,0) node[below left] {$x$};
		\draw[->] (0,\ymin)--(0,\ymax) node[below left] {$y$};
		\draw (0,0) node [below left] {$O$};
		\draw[dashed] (2.04,-1.89) arc(-90:90:.5 cm and 1.89 cm);
		\draw (2.04,1.89) arc(90:270:.5 cm and 1.89 cm);
		
		%\draw[dashed] (5.59,-1.77) arc(-90:90:.5 cm and 1.77 cm);
		%	\draw (5.59,1.77) arc(90:270:.5 cm and 1.77 cm);
		
		\draw (9,0) ellipse (1 cm and 3.96 cm);
		
		\draw[fill=black] (2.04,0) node [below] {$a$} circle (1.2pt);
		\draw[fill=black] (9,0) node [below] {$b$} circle (1.2pt);
		\draw[fill=black] (5.5,1.8) node [above,rotate=30] {\scriptsize $y=f(x)$};
		%%Tự động
		\begin{scope}
			\clip (\xmin+0.01,\ymin+0.01) rectangle (\xmax-0.01,\ymax-0.01);
			\draw[samples=350,domain=1:9.,smooth,variable=\x] plot (\x,{\hamso});
			\draw[samples=350,domain=1:9,smooth,variable=\x] plot (\x,{-0+0.01864083398323228*((\x)-1.0)^(3.0)-0.35126127715584465*((\x)-1.0)^(2.0)+1.5117402856674746*((\x)-1.0)-3.106314636847822});		
			\draw[pattern=north west lines,opacity=0.5] (2.04,0)--(2.04,1.89)plot[domain=2.04:9] (\x,{0-0.01864083398323228*((\x)-1.0)^(3.0)+0.35126127715584465*((\x)-1.0)^(2.0)-1.5117402856674746*((\x)-1.0)+3.106314636847822})--(9.0,0)--(2.04,0);
		\end{scope}
\end{tikzpicture}}

